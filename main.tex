\documentclass[paper=b5, fontsize=12pt, pagesize=auto, chapterprefix, headsepline=true, parskip=half, DIV=12]{scrbook}
\usepackage{scrlayer-scrpage}
\setkomafont{pagehead}{\textcolor{Mahogany}}

\usepackage[utf8]{inputenc}
\usepackage{microtype}

\usepackage[svgnames, dvipsnames]{xcolor}
\usepackage[many]{tcolorbox}

\usepackage{amsmath,amssymb,amsthm}
\allowdisplaybreaks
\usepackage[T1]{fontenc}
\usepackage{newtxtext,newtxmath}
\addtokomafont{disposition}{\rmfamily}

\theoremstyle{definition}
\newtheorem{defn}{Definition}[section]
\newtheorem{thm}[defn]{Theorem}
\newtheorem{proper}[defn]{Properties}
\tcbset{common/.style={
  colback=blue!8,
  boxrule=1pt,
  boxsep=1pt,
  left=3pt,right=3pt,top=6pt,bottom=6pt,
  oversize=0pt,
  sharp corners,
  breakable,
  before skip=\topsep,
  after skip=\topsep}
}
\tcolorboxenvironment{defn}{common}
\tcolorboxenvironment{thm}{common}
\tcolorboxenvironment{proper}{common}
\newtheorem{exmp}{Example}[section]
\tcolorboxenvironment{exmp}{
  colback=green!8,
  boxrule=1pt,
  boxsep=1pt,
  left=3pt,right=3pt,top=6pt,bottom=6pt,
  oversize=0pt,
  sharp corners,
  breakable,
  before skip=\topsep,
  after skip=\topsep,
}
\newenvironment{solution}{\begin{proof}[Solution]}{\end{proof}}

\usepackage{listings}
\lstdefinestyle{mystyle}{
    language=Python, 
    basicstyle=\footnotesize\ttfamily,
    backgroundcolor=\color{gray!20},
    keywordstyle=\color{blue!80}\bfseries,
    commentstyle=\color{Green},
    breaklines=true,
    showstringspaces=false,
    belowskip=0pt
}
\lstset{style=mystyle}

\renewcommand{\chapterformat}{\raggedleft \colorbox{Mahogany}{%
\centering\textit{\textcolor{white}{{\Large Chapter} {\Huge \thechapter}}}}}
\renewcommand{\chapterheadendvskip}{\vspace{3mm}\hrule\par\vspace{6mm}}
\renewcommand{\footnoterule}{\hrule\vspace{2mm}}

\usepackage[lastexercise,answerdelayed]{exercise}
\renewcounter{Exercise}[chapter]
\renewcommand{\theExercise}{\thechapter.\arabic{Exercise}}
\renewcommand{\ExerciseHeader}{\noindent \textbf{Exercise \theExercise} }
\renewcommand{\AnswerHeader}{\noindent \textbf{Exercise \theExercise} }
\usepackage{imakeidx}
\makeindex[intoc]
\newcommand{\keywordhl}[1]{\textbf{\textit{#1}}}
\usepackage{biblatex}
\addbibresource{reference.bib}
\nocite{*}

\usepackage{enumitem}
\usepackage{mathtools}
\usepackage{bm,esint}
\usepackage{siunitx}
\DeclareSIPrePower\quartic{4}
\usepackage{physics}
\usepackage[open,openlevel=2,atend,numbered]{bookmark}

%% (from code by Claudio Beccari in TeX and TUG News, Vol. 2, 1993)
\newcommand\Tstrut{\rule{0pt}{2.3ex}}         % "top" strut
\newcommand\Bstrut{\rule[-1.2ex]{0pt}{0pt}}   % "bottom" strut
\newcommand\TBstrut{\Tstrut\Bstrut}           % "top and bottom" strut

\usepackage{hyperref} % hyper reference
\hypersetup{
    colorlinks,
    allcolors = blue!70!green,
    pdfauthor = Benjamin Loi,
    pdftitle = Intro. to Linear Algebra for Earth Sci. Students,
    pdfsubject = Draft Version,
    pdfkeywords = {Mathematics, Earth Science}
}

\usepackage{pgfplots} % plots
\pgfplotsset{ % plots option
  compat=1.18,
  samples=100
}
\usepackage{lscape}

\usetikzlibrary{calc, matrix, angles,quotes, arrows.meta}
\usepackage{tikz-3dplot}
\tikzset{>=latex}

\includeonly{chapter0, chapter1_Intro_to_Matrix, chapter2_Inv_Det, chapter3_Sol_LinSys, chapter4_Intro_to_Vector, chapter5_Vec_Geometry, chapter6_Vec_Space, chapter6X_More_Vec_Space, chapter7_Complex_Block, chapter8_Eigen, chapter9_Ortho_Normal, chapter10_Quad, chapter11_Inner_Prod, chapter12_Least_sq, chapter13_DFT, chapter14_Markov, answers, reference}

\begin{document}

\KOMAoptions{twoside=false}
\begin{titlepage}
    {\Huge\raggedright Introduction to Linear Algebra \par}
    {\Large\raggedright \textit{with Earth Science Applications} \hfill\textcolor{Mahogany}{\rule{3mm}{3mm}} \par}
    \vspace{3mm}\hrule\par
    {\Large\raggedleft Draft Version \hfill C.L. Loi \par}
    \vfill
    {\Large\raggedleft CUHK-EESC/NTU-AS \par}
\end{titlepage}
\begin{titlepage}
\begin{center}
Introduction to Linear Algebra with Earth Science Applications

Copyright ©, C.L. Loi, 2024 

All rights reserved. Any reproduction of this work, in part or in whole, is prohibited without the written permission of the author.
\end{center}
\end{titlepage}
\KOMAoptions{twoside=semi}
\newpage


\chapter*{Preface}
This is a Linear Algebra textbook specifically designed for students that are majoring in any Earth Science related subjects like Geophysics and Atmospheric Sciences, who are also interested in Mathematics. With these target readers in mind, we set out to provide an adequate treatment about Linear Algebra concepts that enables them to tackle relevant Earth Science problems. In each chapter, we focus on a selected Linear Algebra topic, motivated by Earth Science examples and supplemented with Python programming tutorials. At the end of each chapter, a number of exercises are given for elucidating the concepts and working on Earth Science projects. \par
{\raggedleft Benjamin Loi \par}

%\setcounter{tocdepth}{1}
\tableofcontents
\chapter{Introduction to Matrices and Linear Systems}

Although the Earth System is well-known to be filled with non-linear processes, we still benefit from learning how to work with linear systems, by which many Earth Science problems can be approximated. This actually works well in a number of cases. For instance, in Atmosphere Sciences, we often consider what is called a \textit{perturbation equation}, which assumes that deviations from the mean state are small enough to neglect quadratic terms. The most fundamental usage of Linear Algebra in Applied Sciences is to formulate, analyze and solve \textit{linear systems of equations}. Some examples in Earth Sciences are mapping the depth of overlying soil layers underground, as well as chemical balances in various subsystems of the Earth. \textit{Matrices} are one of the most central ingredients in Linear Algebra that can be used to described such systems, and we are going to address the basic aspects related to them in the first chapter.

\section{Definition and Operations of Matrices}
\label{section:matrixdefn}

\subsection{Basic Structure of Matrices}
\index{Matrix}\keywordhl{Matrices} are rectangular arrays of numbers, the entries of which can be real or complex. For now we will work with the simpler case of real matrices first. A matrix having $m$ rows and $n$ columns is called an $m \times n$ matrix. The class of matrices with the same number of rows and columns, i.e.\ $m = n$, are known as \index{Matrix!Square Matrix}\keywordhl{square matrices}. Below shows some examples of matrices.

\begin{minipage}{0.5\textwidth}
$\begin{bmatrix}
1.17 & 2.01 & -2.15 & 5 \\
1.44 & 3.61 & 2.88 & -3
\end{bmatrix}$
\end{minipage}%
\begin{minipage}{0.5\textwidth}
$\begin{bmatrix}
\sqrt{2} - \frac{4}{\sqrt{5}}\imath \\
0 \\
1.27 \\
\sqrt{3} \imath
\end{bmatrix}$
\end{minipage}\par
\begin{minipage}[b]{0.5\textwidth}
A $2 \times 4$ real matrix.
\end{minipage}%
\begin{minipage}[b]{0.5\textwidth}
A $4 \times 1$ complex matrix.
\end{minipage}\par
\begin{minipage}[b]{0.5\textwidth}
$\begin{bmatrix}
3 & \sqrt{2} & 9 \\
0 & -4\pi & \frac{1}{6} \\
5.11 & 2 & -1
\end{bmatrix}$\par\vspace{4pt}
A $3 \times 3$ real, square matrix.
\end{minipage}

Given any matrix $A$, its entry at row $i$ and column $j$ will be denoted as $A_{ij}$. For example,
\begin{align*}
& A =
\begin{tikzpicture}[baseline=-\the\dimexpr\fontdimen22\textfont2\relax]
\matrix(mymatrix)[matrix of math nodes, left delimiter={[}, 
right delimiter={]}, anchor=center, row sep=2pt,
column sep=2pt, nodes={text width=19pt, align=center}, ampersand replacement=\&]
{2 \& 1 \& 7 \& \frac{8}{9} \\
\textcolor{red}{5} \& -\frac{1}{3} \& 5 \& 0 \\
-3 \& 4.38 \& 6 \& -1.66 \\};
\node at (mymatrix-1-1)[above, font=\footnotesize, yshift=10, Green] {Col 1};
\node at (mymatrix-2-4)[right, font=\footnotesize, xshift=24, Green] {Row 2};
\draw [Green, dashed, fill=blue!50, fill opacity=0.2] (mymatrix-1-1.north west) rectangle (mymatrix-3-1.south east);
\draw [Green, dashed, fill=blue!50, fill opacity=0.2] ([shift={(0pt,3pt)}]mymatrix-2-1.north west) rectangle ([shift={(0pt,-3pt)}]mymatrix-2-4.south east);
\end{tikzpicture}
& \textcolor{red}{A_{21} = 5}
\end{align*}
Short Exercise: Find $A_{13}$, $A_{22}$, $A_{34}$ and $A_{42}$ .\footnote{$A_{13} = 7$, $A_{22} = -\frac{1}{3}$, $A_{34} = -1.66$, $A_{42}$ does not exist.}

\subsection{Matrix Operations} 
\subsubsection{Addition and Subtraction}
Addition and subtraction between two matrices $A$ and $B$ are carried out \textit{entry-wise}, which means that if $C = A \pm B$, then $C_{ij} = A_{ij} \pm B_{ij}$. This implies that the two matrix operands must be of the same shape, and addition/subtraction is not defined for two matrices with different shapes. For instance, if we have
\begin{align*}
A &=
\begin{bmatrix}
1 & 2 \\
3 & 4 \\
5 & 6
\end{bmatrix} &
B &= 
\begin{bmatrix}
1 & 1 \\
0 & 8.5 \\
1 & -7
\end{bmatrix}
\end{align*}
Then
\begin{align*}
A+B &= 
\begin{bmatrix}
1 & 2 \\
3 & 4 \\
5 & 6
\end{bmatrix}
+
\begin{bmatrix}
1 & 1 \\
0 & 8.5 \\
1 & -7
\end{bmatrix} \\
&= 
\begin{bmatrix}
1+1 & 2+1 \\
3+0 & 4+8.5 \\
5+1 & 6+(-7)
\end{bmatrix} \\
&= 
\begin{bmatrix}
2 & 3 \\
3 & 12.5 \\
6 & -1
\end{bmatrix}
\end{align*}
Short Exercise: Find $A-B$.\footnote{$A-B = \begin{bmatrix}
0 & 1 \\
3 & -4.5 \\
4 & 13
\end{bmatrix}$\\}

\subsubsection{Scalar Multiplication} Multiplying a matrix by a number (\textit{scalar}) constitutes a \index{Scalar Multiplication}\keywordhl{scalar multiplication}, in which all entries are multiplied by that scalar. It is illustrated by the example below.
\begin{align*}
A &= 
\begin{bmatrix}
2 & -5.3 & 6 \\
-1 & 4.1 & -3
\end{bmatrix} \\
3A &= 3
\begin{bmatrix}
2 & -5.3 & 6 \\
-1 & 4.1 & -3
\end{bmatrix} \\
&=
\begin{bmatrix}
3(2) & 3(-5.3) & 3(6) \\
3(-1) & 3(4.1) & 3(-3)
\end{bmatrix} \\
&=
\begin{bmatrix}
6 & -15.9 & 18 \\
-3 & 12.3 & -9
\end{bmatrix}
\end{align*}
Short Exercise: Find $\frac{1}{4}A$.\footnote{$\frac{1}{4}A = \begin{bmatrix}
\frac{1}{2} & -1.325 & \frac{3}{2} \\
-\frac{1}{4} & 1.025 & -\frac{3}{4}
\end{bmatrix}$}

\subsubsection{Matrix Multiplication/Matrix Product} Meanwhile, multiplication between two matrices, commonly referred to as \index{Matrix Product (Matrix Multiplication)}\keywordhl{matrix multiplication/matrix product}, is not entry-wise. It can be only carried out if the number of columns of the first matrix $A$ equals to the number of rows of the second matrix $B$, let's say $r$. In other words, they need to be of the shapes $m \times r$ and $r \times n$ respectively. The resulting matrix $AB$ will then have the shape $m \times n$, which means that the number of rows/columns of the output matrix follows the first/second input matrix respectively. The following two examples explain this requirement.
\begin{align*}
& A = 
\begin{bmatrix}
1 & 2.1 & 2 \\
1 & 3 & 5
\end{bmatrix} &
& B = 
\begin{bmatrix}
1 \\
5 \\
\sqrt{7} 
\end{bmatrix}
\end{align*}
Since the shapes of $A$ and $B$ are $2 \times 3$ and $3 \times 1$ so that the number of columns in $A$ and the number of rows in $B$ are both $3$, the matrix product $AB$ is possible. The resulting matrix will be of size $2 \times 1$. On the other hand, $BA$ is not defined if we reverse the order of the matrix product. Meanwhile, for
\begin{align*}
& C = 
\begin{bmatrix}
1 & 2 & 3 & 4 \\
2 & 0 & 6 & 1
\end{bmatrix} &
& D = 
\begin{bmatrix}
3.44 & 1.07\\
0 & 5.96\\
-4.3 & 2.75
\end{bmatrix}
\end{align*}
as the number of columns in $C$ is $4$, which is not equal to the number of rows in $D$ ($3$), the matrix product $CD$ is undefined in this case. (However, $DC$ is just valid, and what will be its shape?\footnote{$DC$ will be a $3 \times 4$ matrix.}) Now we are ready to see how the entries in matrix product is exactly computed.

\begin{defn}[Matrix Product/Matrix Multiplcation]
\label{defn:matprod}
Given an $m \times r$ matrix $A$ and another $r \times n$ matrix $B$, we denote the matrix product between $A$ and $B$ as $AB$ that will have the shape of $m \times n$. To calculate any entry in $AB$ at row $i$ and column $j$, we select row $i$ from the first matrix $A$ and column $j$ from the second matrix $B$. Subsequently, take the products within each of the $r$ pairs of numbers from that row and column. Their sum will then be the required value of the element, i.e.\
\begin{align*}
(AB)_{ij} &= A_{i1}B_{1j} + A_{i2}B_{2j} + A_{i3}B_{3j} + ... + A_{ir}B_{rj} \\
&= \sum_{k=1}^{r} A_{ik}B_{kj}
\end{align*}
again, $r$ is the number of columns/rows in the first/second matrix.
\end{defn}
\begin{exmp}
Calculate the matrix product $C = AB$, where
\begin{align*}
& A = 
\begin{bmatrix}
\textcolor{red}{1} & \textcolor{red}{3} & \textcolor{red}{5} \\
2 & 4 & 6 
\end{bmatrix} &
& B = 
\begin{bmatrix}
1 & \textcolor{blue}{4} \\
2 & \textcolor{blue}{5} \\
3 & \textcolor{blue}{6}
\end{bmatrix}
\end{align*}
\end{exmp}
\begin{solution}
The output will be a $2 \times 2$ matrix. Using the definition above, we have
\begin{align*}
C_{11} = (AB)_{11} &= A_{11}B_{11} + A_{12}B_{21} + A_{13}B_{31} \\
&= (1)(1) + (3)(2) + (5)(3) = 22 \\
C_{12} = (AB)_{12} &= \textcolor{red}{A_{11}}\textcolor{blue}{B_{12}} + \textcolor{red}{A_{12}}\textcolor{blue}{B_{22}} + \textcolor{red}{A_{13}}\textcolor{blue}{B_{32}} \\
&= \textcolor{red}{(1)}\textcolor{blue}{(4)} + \textcolor{red}{(3)}\textcolor{blue}{(5)} + \textcolor{red}{(5)}\textcolor{blue}{(6)} = 49
\end{align*}
Hence the entries along the first row of $C$ will be 22 and 49. The remaining entries at the second row can be found in a similar way, and the readers are encouraged to do this themselves. You should be able to get
\begin{align*}
C = 
\begin{bmatrix}
22 & 49 \\
28 & 64
\end{bmatrix}   
\end{align*}
\end{solution}

Matrix product has some important properties, listed as follows.
\begin{proper}
\label{proper:matmul}
If $A$, $B$, $C$ are some matrices having compatible shapes (\textit{conformable}) so that the matrix multiplication operations below are valid, then
\begin{align*}
\underbrace{A\cdots A}_{k \text{ times}} &= A^k &\text{$k$-th power of a (square) matrix} \\
(AB)C &= A(BC) = ABC &\text{Associative Property} \\
(A \pm B)C &= AC \pm BC &\text{Distributive Property} \\
A(B \pm C) &= AB \pm AC &\text{Distributive Property}
\end{align*}
\end{proper}
Another important observation is that, in general $AB \neq BA$ even if the matrix products $AB$ and $BA$ are both well-defined, so they are not \textit{commutative}. However, there are some exceptions to this.\footnote{A trivial exception is that $A=B$.}
\begin{exmp}
Calculate $-2A + 3B$, where
\begin{align*}
& A = 
\begin{bmatrix}
1 & 6 & 9 \\
4 & 4 & 6 
\end{bmatrix} &
& B = 
\begin{bmatrix}
4 & 8 & 6 \\
-5 & 0 & 3
\end{bmatrix}
\end{align*}
\end{exmp}
\begin{solution}
\begin{align*}
-2A + 3B &= 
-2\begin{bmatrix}
1 & 6 & 9 \\
4 & 4 & 6 
\end{bmatrix}
+3\begin{bmatrix}
4 & 8 & 6 \\
-5 & 0 & 3
\end{bmatrix} \\
&= \begin{bmatrix}
-2 & -12 & -18 \\
-8 & -8 & -12 
\end{bmatrix}
+ \begin{bmatrix}
12 & 24 & 18 \\
-15 & 0 & 9
\end{bmatrix} \\
&= \begin{bmatrix}
10 & 12 & 0 \\
-23 & -8 & -3
\end{bmatrix}
\end{align*}
\end{solution}

\begin{exmp}
Compute $(A+3B)(2A-B)$, where
\begin{align*}
& A = 
\begin{bmatrix}
1 & 2 \\
3 & 5 
\end{bmatrix} &
& B = 
\begin{bmatrix}
-2 & 0 \\
4 & -1
\end{bmatrix}
\end{align*}
\end{exmp}
\begin{solution}
Using the distributive property in Properties \ref{proper:matmul}, the expression can be expanded to
\begin{align*}
(A+3B)(2A-B) &= A(2A-B) + (3B)(2A-B)\\
&= A(2A) + A(-B) + (3B)(2A) + (3B)(-B) \\
&= 2A^2 - AB + 6BA - 3B^2
\end{align*}
Bear in mind that $AB \neq BA$. We calculate each of the terms, which gives
\begin{align*}
A^2 &=
\begin{bmatrix}
1 & 2 \\
3 & 5 
\end{bmatrix}
\begin{bmatrix}
1 & 2 \\
3 & 5 
\end{bmatrix} \\
&=
\begin{bmatrix}
(1)(1)+(2)(3) & (1)(2)+(2)(5) \\
(3)(1)+(5)(3) & (3)(2)+(5)(5) 
\end{bmatrix} \\
&=
\begin{bmatrix}
7 & 12 \\
18 & 31 
\end{bmatrix}
\end{align*}
\begin{align*}
AB &= 
\begin{bmatrix}
1 & 2 \\
3 & 5 
\end{bmatrix}
\begin{bmatrix}
-2 & 0 \\
4 & -1
\end{bmatrix} \\
&=
\begin{bmatrix}
(1)(-2)+(2)(4) & (1)(0)+(2)(-1) \\
(3)(-2)+(5)(4) & (3)(0)+(5)(-1) 
\end{bmatrix} \\
&= 
\begin{bmatrix}
6 & -2 \\
14 & -5 
\end{bmatrix}
\end{align*}
Similarly, it is not difficult to obtain
\begin{align*}
BA &= 
\begin{bmatrix}
-2 & -4 \\
1 & 3 
\end{bmatrix} &
B^2 &= 
\begin{bmatrix}
4 & 0 \\
-12 & 1 
\end{bmatrix} 
\end{align*}
Hence the final answer will be
\begin{align*}
&\quad 2A^2 - AB + 6BA - 3B^2 \\
&= 
2\begin{bmatrix}
7 & 12 \\
18 & 31 
\end{bmatrix}
-\begin{bmatrix}
6 & -2 \\
14 & -5 
\end{bmatrix}
+6\begin{bmatrix}
-2 & -4 \\
1 & 3 
\end{bmatrix}
-3\begin{bmatrix}
4 & 0 \\
-12 & 1 
\end{bmatrix} \\
&=
\begin{bmatrix}
14 & 24 \\
36 & 62 
\end{bmatrix}
-\begin{bmatrix}
6 & -2 \\
14 & -5 
\end{bmatrix}
+\begin{bmatrix}
-12 & -24 \\
6 & 18 
\end{bmatrix}
-\begin{bmatrix}
12 & 0 \\
-36 & 3 
\end{bmatrix} \\
&= 
\begin{bmatrix}
-16 & 2 \\
64 & 82 
\end{bmatrix}
\end{align*}
\end{solution}
Alternatively, one can evaluate $C = A+3B$ and $D = 2A-B$ first, and subsequently calculate the matrix dot product $CD$. (This is actually easier and more efficient.) The readers should try this as an exercise.

\subsubsection{Matrix Equation Manipulation}
For any matrix equation, one can do addition, subtraction and multiplication on both sides of the equation. However, one important note is that multiplying a matrix to an equation requires that the same matrix to be inserted to the left (or right) on both sides, respecting the order. So, for a matrix equation like (assuming the shapes of matrices are compatible),
\begin{align}
AB-C = DE+F \label{eqn:matrixmatdemo}
\end{align}
if we want to multiply the equation by some matrix $G$, then two possibilities are
\begin{align*}
G(AB-C) &= G(DE+F) \\
(AB-C)G &= (DE+F)G
\end{align*}
but we have, in general
\begin{align*}
G(AB-C) &\neq (DE+F)G \\
(AB-C)G &\neq G(DE+F)
\end{align*}
Doing successive matrix multiplications follows the same principle, step by step. Using the same example of Equation (\ref{eqn:matrixmatdemo}), given another matrix $H$, we note some valid outcomes.
\begin{align*}
HG(AB-C) &= HG(DE+F) \\
(AB-C)GH &= (DE+F)GH \\
GH(AB-C) &= GH(DE+F) \\
H(AB-C)G &= H(DE+F)G \\
G(AB-C)H &= G(DE+F)H 
\end{align*}
However, be careful that cancellation at both sides may not be correct. If $AB = AC$, then we cannot conclude that $B = C$ for certain. Nevertheless, in the next chapter we will see one of the scenarios where cancellation actually works.

\section{Definition of Linear Systems of Equations}
\label{section:deflinsys}
The prime application of matrices is to deal with \index{Linear System of Equations}\keywordhl{linear systems (of equations)} as mentioned in the introduction. To understand what a linear system is, we first have to know the definition of a \index{Linear Equation}\keywordhl{linear equation} (in multiple variables, let's say $x_1, x_2, \ldots$, or $x, y, \ldots$). In a linear equation, for any additive term, there is at most one variable (unknown), with a power of one, times some constant coefficient, like $x$, $-\sqrt{5}x$, $-y$, $2.33y$. This means that there are no cross-product terms such as $1.68xy$, variables with a power that is not one, like $x^3$, or non-linear functions, including $\sin{x}, e^{y}$. For $n$ variables, a linear equation has the following form.
\begin{defn}[Linear Equation]
A linear equation is an equation in the form of
\begin{align*}
\sum_{j=1}^n a_jx_j = a_1x_1 + a_2x_2 + a_3x_3 + \cdots + a_nx_n = h
\end{align*}
where $x_1, x_2, \ldots, x_n$ are the unknowns, while $a_1, a_2, \ldots, a_n$ and $h$ are some constants. If $h = 0$, then it is known as a \index{Linear Equation!Homogeneous Linear Equation}\keywordhl{homogeneous linear equation}.
\end{defn}
Short Exercise: Determine whether the equations below are (a) linear, and if they are linear, then (b) homogeneous or not. The unknowns are $x, y, z$. \footnote{Linear/Inhomogeneous, Non-linear, Linear/Inhomogeneous, Non-linear, Linear/Homogeneous, Non-linear.}
\begin{enumerate}
    \item $3x + 4.7y = 2\sqrt{2}$ 
    \item $\cos x + \ln y = 0$
    \item $7\pi x - z = 2$ 
    \item $x^2 + 3.8y^{-3/2} = 1$
    \item $1.05x + 3.17y + 6.44z = 0$
    \item $xyz = 8$ 
\end{enumerate}
A system of linear equations are then simply a family of $m$ linear equations in a set of some unknowns, $m \geq 1$.
\begin{defn}[Linear System of Equations]
\label{defn:linsys}
A linear system of size $m \times n$, i.e. $m$ linear equations in $n$ unknowns ($x_1, x_2, \ldots, x_n$), has the form of
\begin{equation*}
\begin{cases}
\sum_{j=1}^n a_j^{(1)}x_j = a_1^{(1)}x_1 + a_2^{(1)}x_2 + a_3^{(1)}x_3 + \cdots + a_n^{(1)}x_n &= h^{(1)} \\
\sum_{j=1}^n a_j^{(2)}x_j = a_1^{(2)}x_1 + a_2^{(2)}x_2 + a_3^{(2)}x_3 + \cdots + a_n^{(2)}x_n &= h^{(2)} \\
\vdots \\
\sum_{j=1}^n a_j^{(m)}x_j = a_1^{(m)}x_1 + a_2^{(m)}x_2 + a_3^{(m)}x_3 + \cdots + a_n^{(m)}x_n &= h^{(m)} \\
\end{cases}   
\end{equation*}
If $h^{(1)}, h^{(2)}, \ldots, h^{(m)}$ on R.H.S. are all zeros, i.e. all the equations are homogeneous, then the system is called a \index{Linear System of Equations!Homogeneous Linear System of Equations}\keywordhl{homogeneous linear system (of equations)}.
\end{defn}
It is not hard to see that for any homogeneous linear system, it always has the trivial solution of $x_j = 0$ for $j = 1, 2 \ldots, n$, or expressed as $\vec{x} = \textbf{0}$. However, such trivial solution may not be the only solution to the system, as we shall see in Chapter \ref{chap:SolLinSys}. Below shows some examples of linear systems.
\begin{flalign*}
&\begin{cases}
3.3x + 4y &= 5 \\
7x + 9.7y &= 13.1
\end{cases}&   
\end{flalign*}
A $2 \times 2$ linear system with two equations, two unknowns.
\begin{flalign}
\label{eqn:linsys1}
&\begin{cases}
x + 2y - 4z &= 3 \\
x - y + 3z &= -4
\end{cases}&   
\end{flalign}
A $2 \times 3$ linear system with two equations, three unknowns.
\begin{flalign}
\label{eqn:linsys2}
&\begin{cases}
x + 2.2y + 3z &= 0 \\
2x + 3z &= 0 \\
4x - 5.6y &= 0
\end{cases}&  
\end{flalign}
A $3 \times 3$ homogeneous linear system (homogeneous as the constants on R.H.S. are all zeros), notice that the coefficients of $y$ and $z$ in the second/third equations are zeros as well and do not appear explicitly.\par

The above formulation of a linear system closely resembles a tabular structure. Therefore, we are motivated to represent such systems with the language of matrices, which have an appearance of tabular arrays. Indeed, it is possible to rewrite an $m \times n$ linear system as $A\vec{x} = \vec{h}$, where $A$ is an $m \times n$ matrix with entries copied from the coefficients in front of the variables, arranged like in Definition \ref{defn:linsys}. In this book sometimes we will call it a \index{Matrix!Coefficient Matrix}\textit{coefficient matrix}. Meanwhile, $\vec{x}$ is a \textit{column vector} (an $n \times 1$ matrix) holding the $n$ unknowns, and $\vec{h}$ is another column vector (an $m \times 1$ matrix) that contains the $m$ constants on R.H.S. of the linear system.
\begin{proper}
\label{proper:linsysmat}
For a linear system defined as in Definition \ref{defn:linsys}, it can be rewritten as $A\vec{x} = \vec{h}$, where $A_{ij} = a_{j}^{(i)}$, $\vec{x} = x_j$, and $\vec{h} = h^{(i)}$.
\end{proper}
Using the second example (Equation (\ref{eqn:linsys1})) above as an illustration, we can easily verify that
\begin{equation*}
\begin{cases}
x + 2y - 4z &= 3 \\
x - y + 3z &= -4
\end{cases}   
\end{equation*}
can be expressed as (you should check it by expanding the matrix product)
\begin{align*}
\begin{bmatrix}
1 & 2 & -4 \\
1 & -1 & 3 
\end{bmatrix}
\begin{bmatrix}
x \\
y \\
z
\end{bmatrix}
=
\begin{bmatrix}
3 \\
-4
\end{bmatrix}
\end{align*}
An even simpler representation is the \index{Matrix!Augmented Matrix}\keywordhl{augmented matrix} which omits the unknowns and concatenates $A$ and $\vec{h}$.
\begin{equation*}
\left[\begin{array}{@{\,}ccc|c@{\,}}
1 & 2 & -4 & 3\\
1 & -1 & 3 & -4
\end{array}\right]
\end{equation*}
Short Exercise: Write down the augmented matrix for the linear system in Equation (\ref{eqn:linsys2}).\footnote{$
\left[\begin{array}{@{\,}ccc|c@{\,}}
1 & 2.2 & 3 & 0 \\
2 & 0 & 3 & 0 \\
4 & -5.6 & 0 & 0 \\
\end{array}\right]$}

\section{Elementary Row Operations}
When we construct a matrix, it is natural to think about how to manipulate its structure. \index{Elementary Row Operations}\keywordhl{Elementary row operations} provide such possibility in three ways, outlined in the following definition.
\begin{defn}[Elementary Row Operations]
\label{defn:elerowop}
Denote the $p$-th row of a matrix as $R_{p}$. The three types of elementary row operations are
\begin{enumerate}
\item Multiplying a row $R_{p}$ by any non-zero constant $c \neq 0$;
\item Adding another row $R_{q}$ times any non-zero constant $c \neq 0$, to a row $R_{p}$, such that the new $p$-th row becomes $R_{p} + cR_{q}$;
\item Swapping a row $R_{p}$ with another row $R_{q}$.
\end{enumerate}
To facilitate computation, we denote these three kinds of operations using the following notations.
\begin{enumerate}
\item $cR_{p} \rightarrow R_{p}$,
\item $R_{p} + cR_{q} \rightarrow R_{p}$,
\item $R_{p} \leftrightarrow R_{q}$
\end{enumerate}
\end{defn}
For example, the matrix $A$
\begin{equation*}
\begin{bmatrix}
1 & 2 & 3 \\
5 & 7 & 11
\end{bmatrix}
\end{equation*}
can be transformed to a new matrix $A'$
\begin{equation*}
\begin{bmatrix}
1 & 2 & 3 \\
3 & 3 & 5
\end{bmatrix}
\end{equation*}
if we apply the elementary row operation of subtracting $2R_1$ from $R_2$  (i.e.\ $R_2 - 2R_1 \to R_2$). \\
\\
Short Exercise: Find out the resulting matrix $A''$ if we multiply the first row of $A'$ by $3$ and then subtract the second row from the first row.\footnote{
$\begin{bmatrix}
0 & 3 & 4 \\
3 & 3 & 5
\end{bmatrix}$}\par
Attentive readers may have noticed that these three types of elementary row operations resemble we have been always doing to the equations when solving a linear system as taught in high school. We re-introduce them as elementary row operations here first as they allow a systematic treatment of linear systems and matrices in later chapters.

\section{Earth Science Applications}
\label{section:ch1earth}
\begin{exmp}
\label{exmp:seismic1}
Seismic wave follows \textit{Snell's Law} like a light ray when it comes to refraction. Assuming the ground can be modelled as a two-layer system (see Figure \ref{fig:seismic1}), and we know a particular train of seismic wave generated from an underground source that reaches the ground receiver travels at an angle of $\theta_1 = \SI{45}{\degree}$/$\theta_2 = \SI{60}{\degree}$ to the vertical at the top/bottom layer. ($\theta_1$ can be found by analyzing the seismic waveform, and then $\theta_2$ can be estimated by Snell's Law given we know about the densities of the respective layers.) Given that the horizontal and vertical distance between the seismic source and the surface receiver are $d = \SI{120}{\m}$ and $h = \SI{80}{\m}$, construct a linear system for this situation in two unknowns: the depth of the top layer $y$ and the horizontal displacement $x$ (in meters) where the wave reaches at the interface relative to the source.
\end{exmp}
\begin{figure}
\centering
\begin{tikzpicture}
    \draw [black, bottom color=brown!50!orange, top color=orange!50] (0,-0.4) rectangle (8,2.4);
    \draw [black, bottom color=brown!50, top color=brown!10] (0,2.4) rectangle (8,4);
    \coordinate (source) at (1,0);
    \draw [red] (source) circle (0.1);
    \draw [red] (source) circle (0.2);
    \draw [red] (source) circle (0.3);
    \coordinate (interface) at (5.4,2.4);
    \coordinate (receive) at (7,4);
    \draw [black, fill] (6.9,4) rectangle (7.1,4.4);
    \draw [thick, dotted] (interface) -- ++ (0,1) node (interup){};
    \draw [thick, dotted] (interface) -- ++ (0,-1) node (interdown){};
    \draw [red, thick] (source) -- (interface);
    \draw [red, thick, ->] (interface) -- (receive);
    \pic [draw, "$\theta_2 = \SI{60}{\degree}$", angle eccentricity=2, font=\scriptsize, pic text options={xshift=-2}] {angle = source--interface--interdown};
    \pic [draw, "$\theta_1 = \SI{45}{\degree}$", angle eccentricity=2.5, font=\scriptsize, pic text options={xshift=2}] {angle = receive--interface--interup};
    \draw [blue, thick, <->] (7,2.4) -- (7,4) node[below, sloped, midway]{$y$};
    \draw [thick, <->] (8.2,0) -- (8.2,4) node[below, sloped, midway]{$h = \SI{80}{\m}$};
    \draw [blue, thick, <->] (1,2.3) -- (5.4,2.3) node[below, sloped, midway]{$x$};
    \draw [thick, <->] (1,-0.6) -- (7,-0.6) node[below, sloped, midway]{$d = \SI{120}{\m}$};
\end{tikzpicture}
\caption{The underground schematic for the seismic ray in Example \ref{exmp:seismic1}.}
\label{fig:seismic1}
\end{figure}
\begin{solution}
We can deduce two equations from the given information. Consider the upper portion of the seismic ray, from basic trigonometry, we know that 
\begin{align*}
\frac{d-x}{y} &= \tan \theta_1 \\
d-x &= (\tan\theta_1) y \\
x + (\tan\theta_1) y &= d
\end{align*}
Similarly, for the lower portion of the seismic ray, we have
\begin{align*}
\frac{x}{h-y} &= \tan \theta_2 \\
x &= (\tan \theta_2) h - (\tan\theta_2) y \\
x + (\tan\theta_2) y &= (\tan \theta_2) h
\end{align*}
The corresponding linear system is
\begin{align*}
\begin{cases}
x + (\tan\theta_1) y &= d \\
x + (\tan\theta_2) y &= (\tan \theta_2) h
\end{cases}
\end{align*}
where $x$ and $y$ are the unknowns to be solved. $d$, $h$, $\theta_1$ and $\theta_2$ (and hence $\tan\theta_1$ and $\tan\theta_2$) are constants. Expressing the system in matrix form, we have
\begin{align*}
\begin{bmatrix}
1 & \tan\theta_1 \\
1 & \tan\theta_2
\end{bmatrix}
\begin{bmatrix}
x \\
y
\end{bmatrix}
=
\begin{bmatrix}
d \\
(\tan \theta_2) h
\end{bmatrix}
\end{align*}
Substituting the provided values for the constants ($\tan\theta_1 = \tan(\SI{45}{\degree}) = 1$, $\tan\theta_2 = \tan(\SI{60}{\degree}) = \sqrt{3}$), we have
\begin{align*}
\begin{bmatrix}
1 & 1 \\
1 & \sqrt{3}
\end{bmatrix}
\begin{bmatrix}
x \\
y
\end{bmatrix}
=
\begin{bmatrix}
120 \\
80\sqrt{3}
\end{bmatrix}
\end{align*}
\end{solution}

\begin{exmp}
\label{exmp:multilayer1}
The radiation transfer across the atmosphere of any planet (including the Earth) in the Solar system can be compared to a \textit{multi-layer model} with fully absorbing layers (note that it is a very simplistic approach). Assume there are $N$ such layers and the total rate of incident Solar radiation reaching the surface is $E_{in}$. Each of the layers also emits radiation to the other two layers directly above/below it. The rate of radiative emission for the $j$-th layer that has a temperature $T_j$ is $E_j = \sigma T_j^4$ according to the \textit{Stefan–Boltzmann Law}, with $\sigma = \SI{5.67e-8}{\W \per \square\m \per \quartic\K}$. The overall scenario can be seen in Figure \ref{fig:multilayer1}. Formulate a linear system that represents the energy equilibrium (incoming radiation = outgoing radiation) of all layers and the surface, with $E_j$ being the unknowns, over $j = 1, 2, \ldots, N, N+1$.
\end{exmp}
\begin{figure}
\centering
\begin{tikzpicture}
    \fill [SkyBlue!70] (0,0) rectangle (10, 1.5) node[below left, black]{Layer N, $T_N$};
    \fill [SkyBlue!60] (0,1.5) rectangle (10, 3) node[below left, black]{Layer N-1, $T_{N-1}$};
    \fill [SkyBlue!50] (0,3) rectangle (10, 6.5) node[below left, black, scale=3]{$\vdots$};
    \fill [SkyBlue!40] (0,6.5) rectangle (10, 8) node[below left, black]{Layer 3, $T_3$};
    \fill [SkyBlue!30] (0,8) rectangle (10, 9.5) node[below left, black]{Layer 2, $T_2$};
    \fill [SkyBlue!20] (0,9.5) rectangle (10, 11) node[below left, black]{Layer 1, $T_1$};
    \fill [Brown!50!Orange] (-1,-1.2) rectangle (11, 0) node[below left, black]{Surface, $T_s = T_{N+1}$};
    \draw [black] (-1,0) -- (11,0) ;
    \draw [black] (0,1.5) -- (10,1.5);
    \draw [black] (0,3) -- (10,3);
    \draw [black] (0,6.5) -- (10,6.5);
    \draw [black] (0,8) -- (10,8);
    \draw [black] (0,9.5) -- (10,9.5);
    \draw [black] (0,11) -- (10,11);
    \draw [Mahogany, ->, line width=5pt] (0.5,12) -- (0.5,0) node[align=left, pos=-0.05]{Incoming Solar\\ Radiation $E_{in}$};
    \draw [Green, thick, ->] (2.5,10.5) -- (2.5,11.5) node[above]{$E_1 = \sigma T_1^4$};
    \draw [Green, thick, ->] (2.5,10) -- (2.5,9) node[below]{$E_1 = \sigma T_1^4$};
    \draw [Green, thick, ->] (4.5,9) -- (4.5,10) node[above]{$E_2 = \sigma T_2^4$};
    \draw [Green, thick, ->] (4.5,8.5) -- (4.5,7.5) node[below]{$E_2 = \sigma T_2^4$};
    \draw [Green, thick, ->] (6.5,7.5) -- (6.5,8.5) node[above]{$E_3 = \sigma T_3^4$};
    \draw [Green, thick, ->] (6.5,7) -- (6.5,6) node[below]{$E_3 = \sigma T_3^4$};
    \draw [Green, thick, ->] (8.5,6) -- (8.5,7);
    \draw [Green, thick, ->] (3.5,2.5) -- (3.5,3.5) node[above]{$E_{N-1} = \sigma T_{N-1}^4$};
    \draw [Green, thick, ->] (3.5,2) -- (3.5,1) node[below]{$E_{N-1} = \sigma T_{N-1}^4$};
    \draw [Green, thick, ->] (5.5,1) -- (5.5,2) node[above]{$E_N = \sigma T_N^4$};
    \draw [Green, thick, ->] (5.5,0.5) -- (5.5,-0.5) node[below]{$E_N = \sigma T_N^4$};
    \draw [Green, thick, ->] (7.5,-0.5) -- (7.5,0.5) node[above right, pos=0.6]{$E_{N+1} = \sigma T_{N+1}^4$};
    \draw [Green, thick, ->] (1.5,3.5) -- (1.5,2.5);
\end{tikzpicture}
\caption{The atmospheric profile with multiple ($N$) absorbing layers in Example \ref{exmp:multilayer1}. The surface is treated as an extra $N$+1-th layer.}
\label{fig:multilayer1}
\end{figure}
\begin{solution}
Considering the energy equilibrium for the first (topmost) layer, we have
\begin{align*}
-2E_1 + E_2 = 0 
\end{align*}
Going down to the second layer, it is
\begin{align*}
E_1 - 2E_2 + E_3 = 0
\end{align*}
In general, for the $j$-th layer in the middle, where $j$ runs from $2$ to $N$, we can similarly obtain
\begin{align*}
E_{j-1} - 2E_j + E_{j+1} = 0
\end{align*}
Finally, for the surface (the $N+1$-th layer), we have
\begin{align*}
E_N - E_{N+1} + E_{in} &= 0 \\
E_N - E_{N+1} &= -E_{in} 
\end{align*}
Summarizing all the $N+1$ equations, they can be expressed in matrix form as
\begin{align*}
\begin{bmatrix}
-2 & 1 & 0 & \cdots & 0 & 0 & 0 \\
1 & -2 & 1 & & 0 & 0 & 0 \\
0 & 1 & -2 & & 0 & 0 & 0 \\
\vdots & & & \ddots & & & \vdots \\
0 & 0 & 0 & & -2 & 1 & 0 \\
0 & 0 & 0 & & 1 & -2 & 1 \\
0 & 0 & 0 & \cdots & 0 & 1 & -1
\end{bmatrix}
\begin{bmatrix}
E_1 \\
E_2 \\
E_3 \\
\vdots \\
E_{N-1} \\
E_N \\
E_{N+1}
\end{bmatrix}
=
\begin{bmatrix}
0 \\
0 \\
0 \\
\vdots \\
0 \\
0 \\
-E_{in}
\end{bmatrix}
\end{align*}
Particularly, for $N=4$, it is
\begin{align*}
\begin{bmatrix}
-2 & 1 & 0 & 0 & 0 \\
1 & -2 & 1 & 0 & 0 \\
0 & 1 & -2 & 1 & 0 \\
0 & 0 & 1 & -2 & 1 \\
0 & 0 & 0 & 1 & -1
\end{bmatrix}
\begin{bmatrix}
E_1 \\
E_2 \\
E_3 \\
E_4 \\
E_5
\end{bmatrix}
=
\begin{bmatrix}
0 \\
0 \\
0 \\
0 \\
-E_{in}
\end{bmatrix}
\end{align*}
\end{solution}

\begin{exmp}
\label{exmp:seaion}
The seawater in oceans contains a variety of dissolved salts in the form of ions. Most of them are sodium (Na$^+$), magnesium (Mg$^{2+}$), chlorine (Cl$^-$) and sulphate (SO$_4^{2-}$). Consider a sample of seawater and assume the concentrations of other ions are negligible. There are two major constraints over the individual concentrations of each type of ions ($n=$ [Na$^+$], $m=$ [Mg$^{2+}$], $c=$ [Cl$^-$], $s=$ [SO$_4^{2-}$]). First, the overall charge of the seawater has to be neutral. Second, their concentrations should add up to the measured salinity (the total mass concentration of salts, inferred by electrical conductivity). It is given that the salinity of the seawater sample is $\SI{34}{psu}$ (\SI{1}{psu} = \SI{1}{\g\per\kg} which is the unit preferred in oceanography). Write down the corresponding linear system that consists of two equations for this situation.
\end{exmp}
\begin{solution}
The first constraint, charge neutrality, can be simply translated to
\begin{align*}
n + 2m - c - 2s = 0
\end{align*}
While for the second constraint about salinity, we need to convert the molar concentration (\textit{molarity}) of each ion species to mass concentration through multiplying it by their respective molar mass (Na$^+$: \SI{23.0}{\g \per \mol}, Mg$^{2+}$: \SI{24.3}{\g \per \mol}, Cl$^-$: \SI{35.5}{\g \per \mol}, SO$_4^{2-}$: \SI{96.1}{\g \per \mol}). This will lead to
\begin{align*}
23.0 n + 24.3 m + 35.5 c + 96.1s = 34.0 \\
\end{align*}
Thus, the linear system will be just
\begin{align*}
\begin{cases}
n + 2m - c - 2s &= 0 \\
23.0 n + 24.3 m + 35.5 c + 96.1s &= 34.0
\end{cases}
\end{align*}
and its matrix form is
\begin{align*}
\begin{bmatrix}
1 & 2 & -1 & -2 \\
23.0 & 24.3 & 35.5 & 96.1    
\end{bmatrix}
\begin{bmatrix}
n \\
m \\
c \\
s
\end{bmatrix}
=
\begin{bmatrix}
0 \\
34.0
\end{bmatrix}
\end{align*}
\end{solution}

\begin{exmp}
\label{exmp:weatherstats}
There are four weather stations in proximity. Each of them measures the local air temperature $T_i$, where $i = 0,1,2,3$. Assume that the spatial pattern of temperature over the region approximately follows a linear gradient such that both $\partial T/\partial x$ and $\partial T/\partial y$ can be treated as constants. Assign the location of the first station to be the origin $(0,0)$, and the relative locations of the second/third/fourth station are $(10,20)$, $(25,15)$, $(-10,5)$ (in \si{\km}). The measured temperature of the four stations at some time are $\SI{27.1}{\degree C}, \SI{27.3}{\degree C}, \SI{27.4}{\degree C}, \SI{26.9}{\degree C}$. Set up a linear system for finding $\partial T/\partial x$ and $\partial T/\partial y$.
\end{exmp}
\begin{solution}
Since the temperature gradients $\partial T/\partial x$ and $\partial T/\partial y$ are assumed to be constants, we have, by Taylor expansion in both $x$ and $y$,
\begin{align*}
T_i = T_0 + \frac{\partial T}{\partial x}(\Delta x) + \frac{\partial T}{\partial y}(\Delta y)
\end{align*}
for $i = 1,2,3$ where $\Delta x$ and $\Delta y$ are the $x$/$y$-distances relative to the station at the origin. Substituting the provided data, we get
\begin{align*}
\begin{cases}
27.3 &= 27.1 + \frac{\partial T}{\partial x}(10) + \frac{\partial T}{\partial y}(20) \\ 
27.4 &= 27.1 + \frac{\partial T}{\partial x}(25) + \frac{\partial T}{\partial y}(15) \\
26.9 &= 27.1 + \frac{\partial T}{\partial x}(-10) + \frac{\partial T}{\partial y}(5) 
\end{cases}
\end{align*}
Reorganizing them gives
\begin{align*}
\begin{cases}
10\frac{\partial T}{\partial x} + 20\frac{\partial T}{\partial y} &= 0.2 \\ 
25\frac{\partial T}{\partial x} + 15\frac{\partial T}{\partial y} &= 0.3 \\
-10\frac{\partial T}{\partial x} + 5\frac{\partial T}{\partial y} &= -0.2
\end{cases}
\end{align*}
The matrix form of this linear system will then be
\begin{align*}
\begin{bmatrix}
10 & 20 \\
25 & 15 \\
-10 & 5
\end{bmatrix}
\begin{bmatrix}
\frac{\partial T}{\partial x} \\
\frac{\partial T}{\partial y} 
\end{bmatrix}
=
\begin{bmatrix}
0.2 \\
0.3 \\
-0.2
\end{bmatrix}
\end{align*}
\end{solution}

We will talk about how to solve the linear systems in these four examples in Section \ref{section:ch3earth}.

\section{Python Programming}
We will use the package \texttt{numpy} and \texttt{scipy} throughout the book to solve linear algebra problems via \textit{Python} programming. First, we can define a 2D \texttt{numpy} array that works as a matrix.
\begin{lstlisting}
import numpy as np
myMatrix1 = np.array([[1, 4], [5, 3]])
print(myMatrix1)
\end{lstlisting}
which gives
\begin{lstlisting}
[[1 4]
 [5 3]]
\end{lstlisting}
representing the matrix
\begin{align*}
\begin{bmatrix}
1 & 4 \\
5 & 3
\end{bmatrix}
\end{align*}
We can similarly define another matrix:
\begin{lstlisting}
myMatrix2 = np.array([[1, 3], [5, 6]])    
\end{lstlisting}
Addition, subtraction, and scalar multiplication are straight-forward.
\begin{lstlisting}
myMatrix3 = 3*myMatrix1 - 4*myMatrix2
print(myMatrix3)
\end{lstlisting}
The above code produces
\begin{lstlisting}
[[ -1   0]
 [ -5 -15]]
\end{lstlisting}
and you can verify the answer by hand. Meanwhile, matrix product is done by the function \texttt{np.matmul()}.
\begin{lstlisting}
myMatrix4 = np.matmul(myMatrix1, myMatrix2) # or equivalently myMatrix1 @ myMatrix2
print(myMatrix4)
\end{lstlisting}
gives
\begin{lstlisting}
[[21 27]
 [20 33]]
\end{lstlisting}
To select a specfic entry, use indexing by square brackets. The first index/second index represents row/column. Beware that each index starts at zero in \textit{Python}. So putting the number $1$ in the first/second index actually means the second row/column. So
\begin{lstlisting}
print(myMatrix4[1,0])
\end{lstlisting}
refers to the entry at row $2$, column $1$ of \verb|myMatrix4| which is $20$. Also, we can select the $i$-th row (or the $j$-th column) by \verb|<Matrix>[i-1, :]| (\verb|<Matrix>[:, j-1]|), where the colon \verb|:| implies selecting along the entire row (column). For example,
\begin{lstlisting}
print(myMatrix3[0,:])
print(myMatrix4[:,1])
\end{lstlisting}
gives
\verb|[-1  0]| and \verb|[27 33]| respectively. Now let's see how to perform elementary row operations. It will be easier and less error-prone if we copy the array before performing the operations.
\begin{lstlisting}
myMatrix5 = np.copy(myMatrix4)
myMatrix5[0,:] = myMatrix5[0,:]/3
print(myMatrix5)
\end{lstlisting}
The lines above, when executed, divide the second row of \verb|myMatrix5| (which is a copy of \verb|myMatrix4|) by $3$, and give
\begin{lstlisting}
[[ 7  9]
 [20 33]]    
\end{lstlisting}
Meanwhile, the subsequent lines below
\begin{lstlisting}
myMatrix5[1,:] = myMatrix5[1,:] - 2*myMatrix5[0,:]
print(myMatrix5)
\end{lstlisting}
proceed to subtract $2$ times the first row from the second row, and produce
\begin{lstlisting}
[[ 7  9]
 [ 6 15]]
\end{lstlisting}
Row interchange is a bit more tricky.
\begin{lstlisting}
myMatrix6 = np.copy(myMatrix4)
myMatrix6[[0, 1],:] = myMatrix6[[1, 0],:]    
\end{lstlisting}
This swaps the first and second row. (You can swap columns in a similar way.) Printing out the new matrix by \verb|print(myMatrix6)| shows
\begin{lstlisting}
[[20 33]
 [21 27]]
\end{lstlisting}
An important pitfall is that, since our inputs to \texttt{np.array} are all integers, the previous arrays will automatically have a data type of \texttt{int} (integer). This may produce unexpected errors when the calculation leads to decimals/fractions. If it is the case, then we can avoid such bugs by declaring the array with the keyword \verb|dtype=float| to use \textit{floating point numbers}, like
\begin{lstlisting}
myMatrix1 = np.array([[1, 4], [5, 3]], dtype=float) 
\end{lstlisting}
when printed out via \verb|print(myMatrix1)| it gives
\begin{lstlisting}
[[1. 4.]
 [5. 3.]]    
\end{lstlisting}
Notice the newly appeared decimal points after the original integers. Alternatively, we can add decimal points to the integer entries during the array initialization, as
\begin{lstlisting}
myMatrix1 = np.array([[1., 4.], [5., 3.]]) 
\end{lstlisting}

\section{Exercises}
\begin{Exercise}
Let 
\begin{align*}
& A =
\begin{bmatrix}
1 & 2 \\
5 & -1 \\
\end{bmatrix}
& B =
\begin{bmatrix}
-4 & 3 \\
-2 & 7 \\
\end{bmatrix}
\end{align*}
Find:
\begin{enumerate}[label=(\alph*)]
\item $A+B$,
\item $2A-\frac{3}{2}B$,
\item $AB$,
\item $BA$.
\end{enumerate}
\end{Exercise}
\begin{Answer}
\begin{enumerate}[label=(\alph*)]
\item $\begin{bmatrix}
-3 & 5 \\
3 & 6 \\
\end{bmatrix}$
\item $\begin{bmatrix}
8 & -\frac{1}{2} \\
13 & -\frac{25}{2} \\
\end{bmatrix}$
\item $\begin{bmatrix}
-8 & 17 \\
-18 & 8 \\
\end{bmatrix}$
\item $\begin{bmatrix}
11 & -11 \\
33 & -11 \\
\end{bmatrix}$
\end{enumerate}   
\end{Answer}

\begin{Exercise}
Let 
\begin{align*}
& A =
\begin{bmatrix}
0 & 1 \\
3 & -1 \\
4 & 2 \\
\end{bmatrix}
& B =
\begin{bmatrix}
-1 & 0 & -2 \\
-2 & 1 & 3 \\
\end{bmatrix}
\end{align*}
Find:
\begin{enumerate}[label=(\alph*)]
\item $AB$,
\item $BA$.
\end{enumerate}
\end{Exercise}
\begin{Answer}
\begin{enumerate}[label=(\alph*)]
\item $\begin{bmatrix}
-2 & 1 & 3 \\
-1 & -1 & -9 \\
-8 & 2 & -2 \end{bmatrix}$
\item $\begin{bmatrix}
-8 & -5 \\
15 & 3
\end{bmatrix}$
\end{enumerate}
\end{Answer}

\begin{Exercise}
Let
\begin{align*}
A &=
\begin{bmatrix}
4 & 6\\
3 & 3
\end{bmatrix} \\
B &= 
\begin{bmatrix}
2 & 0\\
1 & 2
\end{bmatrix} \\
C &= 
\begin{bmatrix}
3 & 9 & 1\\
4 & 3 & -1
\end{bmatrix}
\end{align*}
Find:
\begin{enumerate}[label=(\alph*)]
\item $(A+B)C$,
\item $AC+BC$,
\item $(AB)C$,
\item $A(BC)$.
\end{enumerate}
\end{Exercise}
\begin{Answer}
\begin{enumerate}[label=(\alph*)]
\item $\begin{bmatrix}
42 & 72 & 0 \\ 
32 & 51 & -1 \\
\end{bmatrix}$
\item Same as above
\item $\begin{bmatrix}
90 & 162 & 2 \\
51 & 99 & 3
\end{bmatrix}$
\item Same as above
\end{enumerate}
\end{Answer}

\begin{Exercise}
Let 
\begin{align*}
& A =
\begin{bmatrix}
1 & 2 & 4\\
1 & 3 & 9\\
7 & 2 & -1
\end{bmatrix}
& B =
\begin{bmatrix}
1 & 5 & -2\\
4 & 3 & 1\\
0 & 2 & 3
\end{bmatrix}
\end{align*}
Find:
\begin{enumerate}[label=(\alph*)]
\item $(A+B)(2A-B)$,
\item $(\frac{3}{2}A - B)(-A + \frac{1}{2}B)$.
\end{enumerate}
\end{Exercise}
\begin{Answer}
\begin{enumerate}[label=(\alph*)]
\item $\begin{bmatrix}
16 & 23 & 129 \\
133 & 33 & 102 \\
27 & 9 & 128
\end{bmatrix}$
\item $\begin{bmatrix}
-\frac{233}{4} & -\frac{19}{4} & \frac{69}{2} \\
-\frac{339}{4} & -16 & 31 \\
\frac{109}{4} & \frac{33}{4} & -\frac{289}{4} \\
\end{bmatrix}$
\end{enumerate}
\end{Answer}

\begin{Exercise}
Let 
\begin{align*}
& A =
\begin{bmatrix}
1 & 0 & 3\\
2 & 1 & 6\\
5 & 2 & 0
\end{bmatrix}
& B =
\begin{bmatrix}
2 & 3 & 5\\
1 & 3 & 8\\
4 & 0 & 7
\end{bmatrix}
\end{align*}
Find:
\begin{enumerate}[label=(\alph*)]
\item $A^2$,
\item $B^2$,
\item $AB$, 
\item $BA$.
\end{enumerate}
\end{Exercise}
\begin{Answer}
\begin{enumerate}[label=(\alph*)]
\item $\begin{bmatrix}
16 & 6 & 3 \\
34 & 13 & 12 \\
9 & 2 & 27
\end{bmatrix}$
\item $\begin{bmatrix}
27 & 15 & 69 \\
37 & 12 & 85 \\
36 & 12 & 69 \\
\end{bmatrix}$
\item $\begin{bmatrix}
14 & 3 & 26 \\
29 & 9 & 60 \\
12 & 21 & 41
\end{bmatrix}$
\item $\begin{bmatrix}
33 & 13 & 24 \\
47 & 19 & 21 \\
39 & 14 & 12
\end{bmatrix}$
\end{enumerate}    
\end{Answer}

\begin{Exercise}
Rewrite the following system of linear equations in matrix form.
\begin{equation*}
\begin{cases}
3y - 4z &= 6\\
5x - y + 2z &= 13\\
6x + z &= 8
\end{cases}
\end{equation*}
\end{Exercise}
\begin{Answer}
\begin{align*}
\begin{bmatrix}
0 & 3 & -4 \\
5 & -1 & 2 \\
6 & 0 & 1
\end{bmatrix}
\begin{bmatrix}
x \\
y \\
z
\end{bmatrix}
=
\begin{bmatrix}
6 \\ 
13 \\
8
\end{bmatrix}    
\end{align*}
or
\begin{align*}
\left[\begin{array}{@{}ccc|c@{}}
0 & 3 & -4 & 6 \\
5 & -1 & 2 & 13 \\
6 & 0 & 1 & 8
\end{array}\right]    
\end{align*}
\end{Answer}

\begin{Exercise}
For the following matrix,
\[
\begin{bmatrix}
2 & 3 & 5 & 7\\
1 & 2 & 4 & 8\\
1 & 3 & 6 & 10
\end{bmatrix}
\]
Find the results if the following elementary row operations are applied on it: 
\begin{enumerate}[label=(\alph*)]
\item Multiplying the third row by a factor of 2, and then subtracting the third row by the second row,
\item Adding the first row by 3 times the third row, and then interchanging the first and second row, and finally subtract the third row by 2 times the first row.\\
\end{enumerate}
\end{Exercise}
\begin{Answer}
\begin{enumerate}[label=(\alph*)]
\item $\begin{bmatrix}
2 & 3 & 5 & 7 \\
1 & 2 & 4 & 8 \\
1 & 4 & 8 & 12
\end{bmatrix}$
\item (WIP)
\end{enumerate}
\end{Answer}

\begin{Exercise}
\label{ex:lapse}
The \textit{dry adiabatic lapse rate}, which is the rate of decrease in air temperature when an unsaturated air parcel rises, is about $\Gamma_{dry} = \SI{9.8}{\celsius \per \km}$. When the temperature of the air parcel falls below the \textit{dew point}, the air saturates and condensation occurs. Typically, dew point temperature of an air parcel will decrease at a rate of roughly $\Gamma_{dew} = \SI{2}{\celsius \per \km}$. Now, an air parcel with an initial air temperature/dew point temperature of $T_{a,ini} = \SI{25.4}{\celsius}$ / $T_{dew, ini} = \SI{17.8}{\celsius}$ at the ground starts to rise. Let $z_{cd}$ and $T_{cd}$ be the height above the ground (in \si{\km}) and temperature (in \si{\celsius}) of the air parcel when condensation occurs. Construct a linear system with $z_{cd}$ and $T_{cd}$ as the unknowns to represent this situation.
\end{Exercise}
\begin{Answer}
The air temperature/dew point at any height $z$ before saturation is $T_a = T_{a,ini} - (\Gamma_{dry})z$ and $T_{dew} = T_{dew, ini} - (\Gamma_{dew})z$ respectively. At the condensation level $z = z_{cd}$, the air temperature equals to the dew point temperature $T_a = T_{dew} = T_{cd}$, and hence we have
\begin{align*}
T_{a,ini} - \Gamma_{dry}(z_{cd}) = T_{dew, ini} - \Gamma_{dew}(z_{cd}) = T_{cd}
\end{align*}
which can be separated into two equations
\begin{align*}
\begin{cases}
T_{a,ini} - \Gamma_{dry}(z_{cd}) &= T_{cd} \\
T_{dew, ini} - \Gamma_{dew}(z_{cd}) &= T_{cd}
\end{cases}
\end{align*}
Rearranging to put the unknowns $z_{cd}$ and $T_{cd}$ to L.H.S., we obtain
\begin{align*}
\begin{cases}
T_{cd} + \Gamma_{dry}(z_{cd}) &= T_{a,ini}\\
T_{cd} + \Gamma_{dew}(z_{cd}) &= T_{dew, ini} 
\end{cases}
\end{align*}
or, in matrix form
\begin{align*}
\begin{bmatrix}
1 & \Gamma_{dry} \\
1 & \Gamma_{dew} \\
\end{bmatrix}
\begin{bmatrix}
T_{cd} \\
z_{cd}
\end{bmatrix}
=
\begin{bmatrix}
T_{a,ini} \\   
T_{dew, ini}
\end{bmatrix}
\end{align*}
Plugging in the lapse rates, we have
\begin{align*}
\begin{bmatrix}
1 & 9.8 \\
1 & 2 \\
\end{bmatrix}
\begin{bmatrix}
T_{cd} \\
z_{cd}
\end{bmatrix}
=
\begin{bmatrix}
25.4 \\   
17.8
\end{bmatrix}
\end{align*}
\end{Answer}

\begin{Exercise}
\label{ex:animals}
In some ancient Chinese Mathematics texts, the problem of \textit{Chickens and Rabbits in the Same Cage} was posed. \textit{"Now there are some chickens and rabbits placed in the same cage, with a total number of $35$ heads and $94$ legs. How many chickens and rabbits are there respectively?"} Given the fact that a chicken (rabbit) has two (four) legs (and obviously only one head), write down the corresponding linear system in terms of the numbers of chickens $x$ and rabbits $y$.
\end{Exercise}
\begin{Answer}
Obviously, there are $35$ chickens and rabbits in total, and $x + y = 35$. Considering the total amount of legs, we also have $2x + 4y = 94$. Hence the required linear system is
\begin{align*}
\begin{cases}
x + y &= 35\\
2x + 4y &= 94
\end{cases}    
\end{align*}
In matrix form, it is
\begin{align*}
\begin{bmatrix}
1 & 1 \\
2 & 4 
\end{bmatrix}
\begin{bmatrix}
x \\
y
\end{bmatrix}
=
\begin{bmatrix}
35 \\
94
\end{bmatrix}
\end{align*}
\end{Answer}
\chapter{Inverses and Determinants}
\label{chapter:invdet}

In this chapter, we are going to discuss two important concepts about matrices: their \textit{inverses} and \textit{determinants}. They will appear from time to time in the remaining parts of this book. To derive them, we first need to introduce some prerequisite ideas, including the \textit{identity matrix}, \textit{transpose}, and the methods of \textit{Gaussian Elimination} and \textit{cofactor expansion}.

\section{Identity Matrices and Transpose}

\subsection{Identity Matrices}
One important class of matrices is the \index{Identity Matrix}\keywordhl{identity matrices}. They are $n \times n$ square matrices, where $n$ can be any positive integer, with entries along the \textit{main diagonal} (where the row index equals column index) being $1$ and other off-diagonal elements being $0$. Usually, they are denoted by $I_n$, or simply by $I$.
\begin{align*}
I_2 &= 
\begin{bmatrix}
\mathcolor{red}{1} & 0 \\
0 & \mathcolor{red}{1}
\end{bmatrix}
& I_3 &= 
\begin{bmatrix}
\mathcolor{red}{1} & 0 & 0 \\
0 & \mathcolor{red}{1} & 0 \\
0 & 0 & \mathcolor{red}{1}
\end{bmatrix}
\end{align*}
\textit{Identity matrices of size $2 \times 2$ and $3 \times 3$ with the main diagonal $1$s highlighted.}
\begin{defn}[Identity Matrix]
\label{defn:identity}
An identity matrix $I_n$ of the square shape $n \times n$ is defined as $[I_{n}]_{ij} = 1$, for $i = j$, and $[I_{n}]_{ij} = 0$, for $i \neq j$, where $1 \leq i,j \leq n$.
\end{defn}
Short Exercise: Explicitly write down $I_5$.\footnote{$I_5=
\begin{bmatrix}
1 & 0 & 0 & 0 & 0 \\
0 & 1 & 0 & 0 & 0 \\
0 & 0 & 1 & 0 & 0 \\
0 & 0 & 0 & 1 & 0 \\
0 & 0 & 0 & 0 & 1
\end{bmatrix}$\\}\\
\\
One important property of identity matrices is
\begin{proper}
\label{proper:identity}
The matrix product between any matrix $A$ with an identity matrix $I$ always returns $A$ whenever the matrix product is well-defined. If $A$ is of the shape $m \times n$, then 
\begin{align}
AI_n = I_mA = A    
\end{align}
If A is now a square matrix such that $m=n$ (and $I_m = I_n = I$), then we simply have 
\begin{align}
AI = IA = A    
\end{align}
\end{proper}
In other words, the identity $I$ can be regarded as "$1$" in the world of matrices. This is one of the cases where $AB = BA$ commutes (if both of them are square and either one is the identity matrix). Using the matrix
\begin{align*}
A =
\begin{bmatrix}
a & b & c \\
d & e & f
\end{bmatrix}
\end{align*}
as an example, the readers can try to computed $AI_3$ and $I_2A$ to see if the results are $A$ itself.\footnote{
\begin{align*}
AI_3 &=
\begin{bmatrix}
a & b & c \\
d & e & f
\end{bmatrix}
\begin{bmatrix}
1 & 0 & 0\\
0 & 1 & 0 \\
0 & 0 & 1
\end{bmatrix} \\
&=
\begin{bmatrix}
(a)(1) + (b)(0) + (c)(0) & (a)(0) + (b)(1) + (c)(0) &  (a)(0) + (b)(0) + (c)(1) \\
(d)(1) + (e)(0) + (f)(0) & (d)(0) + (e)(1) + (f)(0) &  (d)(0) + (e)(0) + (f)(1)
\end{bmatrix} \\
&=
\begin{bmatrix}
a & b & c \\
d & e & f
\end{bmatrix}
= A    
\end{align*} The calculation of $I_2A = A$ is similar. }

\subsection{Transpose}
\index{Transpose}\keywordhl{Transpose} of a matrix, denoted by adding the superscript $^T$, is formed by interchanging its rows and columns, that is, flipping the elements about the main diagonal.
\begin{defn}[Transpose]
The transpose of an $m \times n$ matrix $A$, denoted as $A^T$, is constructed according to the relation 
\begin{align}
[A^T]_{pq} = A_{qp}
\end{align}
where $1 \leq p \leq n$, $1 \leq q \leq m$, i.e.\ swapping the row and column indices. Now, $A^T$ is an $n \times m$ matrix.
\end{defn}
Two examples are given below to show the effect of applying transpose on matrices.
\begin{align*}
A &= 
\begin{bmatrix}
1 & 2 & 3 \\
4 & 5 & 6
\end{bmatrix}
& A^T &= 
\begin{bmatrix}
1 & 4 \\
2 & 5 \\
3 & 6
\end{bmatrix} \\
B &= 
\begin{tikzpicture}[baseline=-\the\dimexpr\fontdimen22\textfont2\relax]
\matrix(mymatrix)[matrix of math nodes, left delimiter={[}, 
right delimiter={]}, row sep=1pt, column sep=-1pt, outer sep=-7pt, nodes={text width=14pt, align=center}, ampersand replacement=\&]
{1 \& -4 \& 3 \\
2 \& -2 \& 0 \\
-3 \& 1 \& 4 \\};
\draw (mymatrix-1-1.center) edge[dashed, thick, Red] (mymatrix-3-3.center);
\draw (mymatrix-1-2.center) edge[dashed, Green, <->] (mymatrix-2-1.center);
\draw (mymatrix-1-3.center) edge[dashed, Green, <->] (mymatrix-3-1.center);
\draw (mymatrix-2-3.center) edge[dashed, Green, <->] (mymatrix-3-2.center);
\end{tikzpicture}
& B^T &= 
\begin{tikzpicture}[baseline=-\the\dimexpr\fontdimen22\textfont2\relax]
\matrix(mymatrix)[matrix of math nodes, left delimiter={[}, 
right delimiter={]}, row sep=1pt, column sep=-1pt, outer sep=-7pt, nodes={text width=14pt, align=center}, ampersand replacement=\&]
{1 \& 2 \& -3 \\
-4 \& -2 \& 1 \\
3 \& 0 \& 4 \\};
\draw (mymatrix-1-1.center) edge[dashed, thick, Red] (mymatrix-3-3.center);
\draw (mymatrix-1-2.center) edge[dashed, Green, <->] (mymatrix-2-1.center);
\draw (mymatrix-1-3.center) edge[dashed, Green, <->] (mymatrix-3-1.center);
\draw (mymatrix-2-3.center) edge[dashed, Green, <->] (mymatrix-3-2.center);
\end{tikzpicture}
\end{align*}
Particularly, in the second example, we have outlined the main diagonal (red dashed line) of $B$ (as well as $B^T$) and how the elements flip about it (green dashed arrows) when transpose is carried out. Some useful properties of transpose are listed as follows.
\begin{proper}
\label{proper:transp}
For two matrices $A$ and $B$, we have
\begin{enumerate}
\item $(cA)^T = cA^T$, where $c$ is any constant;
\item $(A^T)^T = A$, i.e.\ transposing twice returns the original matrix (which is obvious);
\item $(A \pm B)^T = A^T \pm B^T$, if $A$ and $B$ have the same shape;
\item $(AB)^T = B^TA^T$, if $A$ and $B$ are conformable;
\item $A_{kk} = A^T_{kk}$ for any $k$ that $A_{kk}$ is defined, i.e.\ the main diagonal is unaffected by transpose.
\end{enumerate}
\end{proper}
Short Exercise: Show that $(ABC)^T = C^TB^TA^T$ if the matrices have compatible shapes for the matrix multiplication.\footnote{By (4), $(ABC)^T = ((AB)(C))^T = C^T(AB)^T = C^TB^TA^T$.}

\subsection{Symmetric Matrices}
A \index{Symmetric Matrix}\keywordhl{symmetric matrix} has its elements mirrored across the main diagonal. Taking a transpose of such a matrix will leave it unchanged. Implicitly, the matrix is required to be square.
\begin{defn}[Symmetric Matrix]
If an $n \times n$ square matrix $A$ and its transpose $A^T$ are equal, i.e.\ \begin{align}
A_{pq} = [A^T]_{pq} = A_{qp}    
\end{align} for all $1 \leq p, q \leq n$, or simply 
\begin{align}
A = A^T    
\end{align} then $A$, and also $A^T$, are symmetric.
\end{defn}
As an example,
\begin{flalign*}
&\begin{bmatrix}
1 & 2 & -2 \\
2 & 0 & 4 \\
-2 & 4 & 3
\end{bmatrix}&
\end{flalign*}
is a $3 \times 3$ symmetric matrix.\par
Short Exercise: Show that $Y = XX^T$ and $Z = X^TX$ are symmetric for any matrix $X$.\footnote{By (4) of Properties \ref{proper:transp}, $Y^T = (XX^T)^T = (X^T)^T(X)^T = XX^T = Y$, similar goes for $Z = X^TX$.}\par
In contrast, we also have \index{Skew-symmetric Matrix}\keywordhl{skew-symmetric matrices} such that 
\begin{align}
A^T = -A    
\end{align} This automatically forces the elements along the main diagonal in the matrix to be all zeros.
\begin{flalign*}
&\begin{bmatrix}
0 & 2 & 1 \\
-2 & 0 & -3 \\
-1 & 3 & 0
\end{bmatrix}&
\end{flalign*}
\textit{A $3 \times 3$ skew-symmetric matrix.}

\section{Inverses}
\label{section:inv}
\subsection{Definition and Properties of Inverses}
\label{subsection:invsub}
\index{Inverse}\keywordhl{Inverse} of a square matrix, denoted by appending the superscript $^{-1}$, is another square matrix such that the matrix product between these two matrices (in either order) yields an identity matrix.
\begin{defn}[Inverse]
\label{defn:inverse}
An $n \times n$ square matrix $B$ is said to be the inverse of another $n \times n$ square matrix $A$ if 
\begin{align}
AB = BA = I_n    
\end{align}
This inverse matrix is denoted as $B = A^{-1}$, and the relation becomes 
\begin{align}
AA^{-1} = A^{-1}A = I    
\end{align}
The opposite direction also holds, i.e.\ $A$ is the inverse of $A^{-1}$. Hence, we say that $A$ and $A^{-1}$ are the inverse of each other.
\end{defn}
If there exists an inverse $A^{-1}$ for the square matrix $A$, then both $A$ and $A^{-1}$ are called \index{Invertible}\keywordhl{invertible}. Otherwise, $A$ is said to be \index{Singular}\keywordhl{singular}. This is another situation in which a matrix product $AB = BA$ (if $B=A^{-1}$) can commute.\footnote{$AA^{-1} = I$ implies $A^{-1}A = I$ and vice versa. However, while appearing innocent, showing this is actually not trivial and prone to circular logic. A heuristic way to "prove" it is to note that
\begin{align*}
AA^{-1} &= I \\
AA^{-1}A &= IA \\
A(A^{-1}A) &= A
\end{align*}
which implies that multiplying $A$ by $A^{-1}A$ returns $A$ itself, so it should be reasonable to assume $A^{-1}A = I$. In fact, this is guaranteed if $A$ is indeed invertible.}\par %We can show why $AA^{-1} = I$ implies $A^{-1}A = I$, or vice versa.
%\begin{proof}
%Assume only $AA^{-1} = I$ is true, then multiplying $A$ to the right on both sides of the equation leads to
%\begin{align*}
%AA^{-1}A &= IA \\
%A(A^{-1}A) &= A & & \text{(Properties \ref{proper:matmul} and Properties \ref{proper:identity})}\\
%AP &= A
%\end{align*}
%where we write $P = A^{-1}A$. The above implies that multiplying $A$ by $P$ returns $A$ itself. We are tempted to claim that $P = I$ by observing Properties \ref{proper:identity}. However, it is not trivial to show that $P$ cannot be any matrix other than $I$, and to do so we have to wait until later chapters. Nevertheless, it is indeed true as long as $A$ is invertible. Therefore, $P = A^{-1}A = I$. The opposite direction is proved similarly.    
%\end{proof}
In the last chapter, we only define addition, subtraction, and multiplication for matrices, omitting division like an elephant in the room. The inverse serves as a remedy for this by acting as the reciprocal in the world of matrices. This allows us to "divide" on both sides of a matrix equation provided the relevant inverses exist. Remember, in the last chapter, we mentioned that cancellation on both sides may not work for some equations like $AB = AC$. But if the inverse $A^{-1}$ exists, then by multiplying it to the left on both sides of the equation, we can effectively "divide by $A$" to get
\begin{align*}
AB &= AC \\
A^{-1}AB &= A^{-1}AC \\
(A^{-1}A)B &= (A^{-1}A)C & \text{(Associative, from Properties \ref{proper:matmul})} \\
IB &= IC & \text{(Definition \ref{defn:inverse})} \\
B &= C & \text{(Properties \ref{proper:identity})}
\end{align*}
so that cancellation holds in this situation. Take the matrix equation $AG = H$ as another example, if $A$ has an inverse $A^{-1}$, then we may do a matrix "division" as follows:
\begin{align*}
AG &= H \\
A^{-1}AG &= A^{-1}H \\
\textcolor{gray}{((A^{-1}A)G = IG =)} \, G &= A^{-1}H & 
\begin{aligned}
& \text{(Properties \ref{proper:matmul} and \ref{proper:identity},} \\    
& \text{Definition \ref{defn:inverse})}
\end{aligned} 
\end{align*}
In addition, the inverse of a matrix, if exists, must be unique.
\begin{proper}[Uniqueness of Inverse]
\label{proper:uniqueinverse}
If $A$ has an inverse $A^{-1}$, it is unique.
\end{proper}
\begin{proof}
This property can be proved easily by first assuming that the invertible matrix $A$ has two different inverses, $B$ and $C$. Subsequently, by Definition \ref{defn:inverse}, we have $BA = I$ (and also $AC = I$). Multiplying by $C$ to the right on both sides gives
\begin{align*}
BAC &= IC \\
B(AC) &= C & \text{(Properties \ref{proper:matmul} and \ref{proper:identity})}\\
B(I) &= C & \text{($AC = I$ from assumption)} \\
B &= C & \text{(Properties \ref{proper:identity})}
\end{align*}
So, $B$ and $C$ are actually the same matrix, implying that the inverse of $A$ is unique.    
\end{proof}
\begin{exmp}
Let 
\begin{align*}
& A =
\begin{bmatrix}
4 & 6 \\
3 & 5
\end{bmatrix}
& B =
\begin{bmatrix}
\frac{5}{2} & -3 \\
-\frac{3}{2} & 2
\end{bmatrix}
\end{align*}
Show that $A$ and $B$ are the inverse of each other.
\end{exmp}
\begin{solution}
\begin{align*}
AB &= 
\begin{bmatrix}
4 & 6 \\
3 & 5
\end{bmatrix}
\begin{bmatrix}
\frac{5}{2} & -3 \\
-\frac{3}{2} & 2
\end{bmatrix} \\
&= 
\begin{bmatrix}
(4)(\frac{5}{2})+(6)(-\frac{3}{2}) & (4)(-3)+(6)(2) \\
(3)(\frac{5}{2})+(5)(-\frac{3}{2}) & (3)(-3)+(5)(2)
\end{bmatrix} \\
&= 
\begin{bmatrix}
1 & 0 \\
0 & 1
\end{bmatrix} = I_2
\end{align*}
We leave it to the readers to show $BA = I$ too as an exercise. Hence, $AB = BA = I$, $A$ and $B$ are indeed the inverse of each other.
\end{solution}

The following are some properties of inverses.
\begin{proper}
\label{proper:inverse}
If a square matrix $A$ is invertible and has an inverse $A^{-1}$, then
\begin{enumerate}
\item $(cA)^{-1} = \frac{1}{c}A^{-1}$, for any constant $c \neq 0$;
\item $(A^{-1})^{-1} = A$, i.e.\ the inverse of an inverse returns the original matrix;
\item $(A^n)^{-1} = (A^{-1})^n$, for any positive integer $n$;
\item $(AB)^{-1} = B^{-1}A^{-1}$, provided that $B$ is invertible too (and they are square matrices of the same size);
\item $(A^T)^{-1} = (A^{-1})^T$.
\end{enumerate}
\end{proper}
However, $(A\pm B)^{-1}$ may not be equal to $A^{-1} \pm B^{-1}$, or even may be singular. We shall briefly prove (4) here.
\begin{proof}
It is given that $A$ and $B$ is invertible, and by Definition \ref{defn:inverse}, we have $AA^{-1} = I$, as well as 
\begin{align*}
BB^{-1} = I    
\end{align*}
Multiplying by $A$ and $A^{-1}$ to the left and right on both sides of the above equation respectively yields
\begin{align*}
ABB^{-1}A^{-1} &= AIA^{-1} \\
AB(B^{-1}A^{-1}) &= (AI)A^{-1} = AA^{-1} & \text{(Properties \ref{proper:matmul} and \ref{proper:identity})} \\
&= I & \text{(Definition \ref{defn:inverse})}
\end{align*}
This shows that multiplying $AB$ by $B^{-1}A^{-1}$ produces an identity matrix, and therefore $(AB)^{-1} = B^{-1}A^{-1}$ is the unique inverse of $AB$ by Definition \ref{defn:inverse} and Properties \ref{proper:uniqueinverse}.
\end{proof}
Short Exercise: Show that $(ABC)^{-1} = C^{-1}B^{-1}A^{-1}$ if $A$, $B$ and $C$ are invertible and conformable.\footnote{By (4), $(ABC)^{-1} = ((AB)(C))^{-1} = C^{-1}(AB)^{-1} = C^{-1}B^{-1}A^{-1}$.}\par
(4) of Properties \ref{proper:inverse} explicitly shows that the product $AB$ is invertible if $A$ and $B$ are themselves invertible. The converse is actually true as well.\footnote{Let's assume $AB$ is invertible and has an inverse $C = (AB)^{-1}$, hence we have $(AB)C = I$ by Definition \ref{defn:inverse}, (notice that $A$, $B$, and $C$ are all square matrices of the same extent) and by Properties \ref{proper:matmul}, $A(BC) = I$. Using Definition \ref{defn:inverse} (as well as Properties \ref{proper:uniqueinverse}) again, we immediately identify $BC$ as the inverse of $A$ and $A$ is invertible. The case for $B$ is similarly proved.} Hence
\begin{proper}
\label{proper:ABinv}
For two square matrices $A$ and $B$, $AB$ is invertible if and only if $A$ and $B$ are invertible.
\end{proper}

\subsection{(Reduced) Row Echelon Form}
\label{section:echelon}
Naturally, the next question is how to compute the inverse of any square matrix. For this, we have to understand a specific form of matrices called the \index{Row Echelon Form}\index{Row Echelon Form}\index{Reduced Row Echelon Form}\keywordhl{(reduced) row echelon form} first. A matrix is in reduced row echelon form (\textit{RREF}) when it satisfies the following requirements.
\begin{defn}[(Reduced) Row Echelon Form]
\label{defn:rref}
A matrix is in row echelon form if
\begin{enumerate}
\item The first non-zero number in every row is $1$, which is known as the \textit{"Leading 1"} (sometimes referred to as a \textit{pivot});
\item \textit{"Leading 1"} of a lower row must appear farther to the right than that of any higher row;
\item Any row consisting of all zeros is placed at the bottom;
\item If additionally, any column containing a leading $1$ (sometimes called a \textit{pivotal column}) has zeros elsewhere in that column, then it is in \textit{reduced} row echelon form.
\end{enumerate}
\end{defn}
It is apparent that all identity matrices are in (reduced) row echelon form. Examples of row echelon form (but not \textit{reduced}), with the leading $1$s highlighted are
\begin{align*}
A &=
\begin{bmatrix}
\textcolor{red}{1} & 2 & 0 \\
0 & \textcolor{red}{1} & 1 \\
0 & 0 & \textcolor{red}{1}
\end{bmatrix}
& B &=
\begin{bmatrix}
\textcolor{red}{1} & 3 & 1 & 2 \\
0 & 0 & \textcolor{red}{1} & 5 \\
0 & 0 & 0 & \textcolor{red}{1}
\end{bmatrix} \\
C &=
\begin{bmatrix}
\textcolor{red}{1} & 4 \\
0 & \textcolor{red}{1} \\
0 & 0 
\end{bmatrix}
& D &=
\begin{bmatrix}
0 & \textcolor{red}{1} & 0 & 2 \\
0 & 0 & 0 & \textcolor{red}{1} \\
0 & 0 & 0 & 0
\end{bmatrix}
\end{align*}
Meanwhile, examples of \textit{reduced} row echelon form are
\begin{align*}
& G =
\begin{bmatrix}
\textcolor{red}{1} & 0 & 0 \\
0 & 0 & \textcolor{red}{1}\\
0 & 0 & 0 
\end{bmatrix}
& H =
\begin{bmatrix}
\textcolor{red}{1} & 0 & 2 & 0 \\
0 & \textcolor{red}{1} & 1 & 0 \\
0 & 0 & 0 & \textcolor{red}{1}
\end{bmatrix}
\end{align*}
The following matrices are \textit{not} in row echelon form. (why?)\footnote{$P$ violates (2) and $Q$ does not satisfy (1) and (3) of Definition \ref{defn:rref}.}
\begin{align*}
& P =
\begin{bmatrix}
1 & 0 & 0 \\
0 & 0 & 1 \\
0 & 1 & 0 \\
\end{bmatrix}
& Q =
\begin{bmatrix}
1 & 0 & 0 \\
0 & 0 & 0 \\
0 & 3 & 1
\end{bmatrix}
\end{align*}
Short Exercise: Decide if the following matrices are in (reduced) row echelon form or not.\footnote{Yes, Yes (reduced), No.}
\begin{align*}
X &= 
\begin{bmatrix}
1 & 0 & 0 & 1 \\
0 & 1 & 2 & 0 \\
0 & 0 & 0 & 1
\end{bmatrix}
& Y&=
\begin{bmatrix}
1 & 0 & 0 & 0 & 0 \\
0 & 0 & 0 & 1 & 0 \\
0 & 0 & 0 & 0 & 1 \\
0 & 0 & 0 & 0 & 0
\end{bmatrix}
& Z&=
\begin{bmatrix}
1 & 0 & 0 \\
0 & 0 & 1 \\
0 & 0 & 1 
\end{bmatrix}
\end{align*}
We have studied elementary row operations in the last chapter, which now can be used to transform matrices into their reduced row echelon form. The procedure is comprised of two major parts, the \textit{forward phase}, converting the matrix to row echelon form first, and the \textit{backward phase}, eventually transforming it into reduced row echelon form. The first phase is also named \index{Gaussian Elimination}\keywordhl{Gaussian Elimination}, and together they are called \index{Gauss-Jordan Elimination}\keywordhl{Gauss-Jordan Elimination}\footnote{Often we just write Gaussian Elimination in place of Gauss-Jordan Elimination.}. We demonstrate the entire procedure using an example.
\begin{exmp}
Carry out Gauss-Jordan Elimination on the following matrix to make it become reduced row echelon form.
\begin{align*}
A =
\begin{bmatrix}
2 & 1 & 4 & 6 \\
3 & 3 & 1 & 0 \\
1 & 2 & 3 & 4
\end{bmatrix}    
\end{align*}
\end{exmp}
\begin{solution}
At each step of the forward phase, the strategy is to look at the leftmost column that has at least one non-zero entry (any column consisting of full zeros is ignored). Along that column, we either find an existing leading $1$, or create a leading $1$ via multiplying some row having a starting entry $a$ that is as large as possible in magnitude, by the constant $1/a$. (The leading entry selected by this algorithm is commonly called the \index{Pivot}\keywordhl{pivot}, and the process is called \index{Pivoting}\keywordhl{pivoting}.) The row holding the leading $1$ is subsequently put at the top, by an interchanging of rows if needed. In this example, such rows will be highlighted in red.
\begin{align*}
\left[\begin{array}{@{\,}wc{10pt}wc{10pt}wc{10pt}wc{10pt}@{\,}}
2 & 1 & 4 & 6 \\[3pt]
3 & 3 & 1 & 0 \\[3pt]
1 & 2 & 3 & 4
\end{array}\right]
&\to
\left[\begin{array}{@{\,}wc{10pt}wc{10pt}wc{10pt}wc{10pt}@{\,}}
\mathcolor{red}{1} & \mathcolor{red}{\frac{1}{2}} & \mathcolor{red}{2} & \mathcolor{red}{3} \\[3pt]
3 & 3 & 1 & 0 \\[3pt]
1 & 2 & 3 & 4
\end{array}\right]
& \frac{1}{2}R_1 \to R_1
\end{align*}
We have picked the first row $R_1$ for the leading $1$ through multiplying it by a factor of $\frac{1}{2}$ here, but a leading $1$ can be obtained from the other two rows as well. Subsequently, we make all the entries below the leading $1$ along that \textit{pivotal column} become zero, by adding the top row (which now holds the leading $1$), times $-a_i$ (where $a_i$ is the corresponding leading entry in the $i$-th row) to the other rows. Those zeros produced in this way will be highlighted in blue.
\begin{align*}
\left[\begin{array}{@{\,}wc{10pt}wc{10pt}wc{10pt}wc{10pt}@{\,}}
\mathcolor{red}{1} & \mathcolor{red}{\frac{1}{2}} & \mathcolor{red}{2} & \mathcolor{red}{3} \\[3pt]
3 & 3 & 1 & 0 \\[3pt]
1 & 2 & 3 & 4
\end{array}\right]
&\to
\left[\begin{array}{@{\,}wc{10pt}wc{10pt}wc{10pt}wc{10pt}@{\,}}
\mathcolor{red}{1} & \mathcolor{red}{\frac{1}{2}} & \mathcolor{red}{2} & \mathcolor{red}{3} \\[3pt]
\mathcolor{blue}{0} & \frac{3}{2} & -5 & -9 \\[3pt]
1 & 2 & 3 & 4
\end{array}\right]
& R_2-3R_1 \to R_2 \\
&\to
\left[\begin{array}{@{\,}wc{10pt}wc{10pt}wc{10pt}wc{10pt}@{\,}}
\mathcolor{red}{1} & \mathcolor{red}{\frac{1}{2}} & \mathcolor{red}{2} & \mathcolor{red}{3} \\[3pt]
\mathcolor{blue}{0} & \frac{3}{2} & -5 & -9 \\[3pt]
\mathcolor{blue}{0} & \frac{3}{2} & 1 & 1
\end{array}\right]
& R_3-R_1 \to R_3
\end{align*}
The first iteration is finished. We now repeat the same process over the remaining submatrix made up of elements that are not yet highlighted in colour, from left to right recursively.
\begin{align*}
\left[\begin{array}{@{\,}wc{10pt}wc{10pt}wc{10pt}wc{10pt}@{\,}}
\mathcolor{red}{1} & \mathcolor{red}{\frac{1}{2}} & \mathcolor{red}{2} & \mathcolor{red}{3} \\[3pt]
\mathcolor{blue}{0} & \frac{3}{2} & -5 & -9 \\[3pt]
\mathcolor{blue}{0} & \frac{3}{2} & 1 & 1
\end{array}\right]
&\to
\left[\begin{array}{@{\,}wc{10pt}wc{10pt}wc{10pt}wc{10pt}@{\,}}
\mathcolor{red}{1} & \mathcolor{red}{\frac{1}{2}} & \mathcolor{red}{2} & \mathcolor{red}{3} \\[3pt]
\mathcolor{blue}{0} & \mathcolor{red}{1} & \mathcolor{red}{-\frac{10}{3}} & \mathcolor{red}{-6} \\[3pt]
\mathcolor{blue}{0} & \frac{3}{2} & 1 & 1
\end{array}\right]
& \frac{2}{3}R_2 \to R_2 \\
&\to
\left[\begin{array}{@{\,}wc{10pt}wc{10pt}wc{10pt}wc{10pt}@{\,}}
\mathcolor{red}{1} & \mathcolor{red}{\frac{1}{2}} & \mathcolor{red}{2} & \mathcolor{red}{3} \\[3pt]
\mathcolor{blue}{0} & \mathcolor{red}{1} & \mathcolor{red}{-\frac{10}{3}} & \mathcolor{red}{-6} \\[3pt]
\mathcolor{blue}{0} & \mathcolor{blue}{0} & 6 & 10
\end{array}\right]
& R_3-\frac{3}{2}R_2 \to R_3 \\
&\to
\left[\begin{array}{@{\,}wc{10pt}wc{10pt}wc{10pt}wc{10pt}@{\,}}
\mathcolor{red}{1} & \mathcolor{red}{\frac{1}{2}} & \mathcolor{red}{2} & \mathcolor{red}{3} \\[3pt]
\mathcolor{blue}{0} & \mathcolor{red}{1} & \mathcolor{red}{-\frac{10}{3}} & \mathcolor{red}{-6} \\[3pt]
\mathcolor{blue}{0} & \mathcolor{blue}{0} & \mathcolor{red}{1} & \mathcolor{red}{\frac{5}{3}}
\end{array}\right]
& \frac{1}{6}R_3 \to R_3 
\end{align*}
Now, all entries below every leading $1$ are zeros, and the forward phase is completed. We have obtained the row echelon form as an intermediate product. The backward phase is done similarly but in a bottom-up fashion, from right to left. By adding appropriate multiples of lower rows to higher rows, we turn all the non-zero elements above the leading $1$ along every pivotal column into zeros. Non-pivotal columns (the last column here) are ignored.
\begin{align*}
\left[\begin{array}{@{\,}wc{10pt}wc{10pt}wc{10pt}wc{10pt}@{\,}}
1 & \frac{1}{2} & 2 & 3 \\[3pt]
0 & 1 & -\frac{10}{3} & -6 \\[3pt]
0 & 0 & 1 & \frac{5}{3}
\end{array}\right]
&\to
\left[\begin{array}{@{\,}wc{10pt}wc{10pt}wc{10pt}wc{10pt}@{\,}}
1 & \frac{1}{2} & 2 & 3 \\[3pt]
0 & 1 & 0 & -\frac{4}{9} \\[3pt]
0 & 0 & 1 & \frac{5}{3}
\end{array}\right]
& R_2 + \frac{10}{3}R_3 \to R_2 \\
&\to
\left[\begin{array}{@{\,}wc{10pt}wc{10pt}wc{10pt}wc{10pt}@{\,}}
1 & \frac{1}{2} & 0 & -\frac{1}{3} \\[3pt]
0 & 1 & 0 & -\frac{4}{9} \\[3pt]
0 & 0 & 1 & \frac{5}{3}
\end{array}\right]
& R_2 - 2R_3 \to R_1 \\
&\to
\left[\begin{array}{@{\,}wc{10pt}wc{10pt}wc{10pt}wc{10pt}@{\,}}
1 & 0 & 0 & -\frac{1}{9} \\[3pt]
0 & 1 & 0 & -\frac{4}{9} \\[3pt]
0 & 0 & 1 & \frac{5}{3}
\end{array}\right]
& R_1 - \frac{1}{2}R_2 \to R_1
\end{align*}
The matrix is now in reduced row echelon form as required. The amount of leading $1$s in the RREF of the matrix is known as its \textit{rank}, which equals $3$ here.
\end{solution}
Short Exercise: Repeat the example above but start by interchanging $R_1$ and $R_3$.\footnote{For checking, after the first iteration, it will be
\begin{align*}
\begin{bmatrix}
1 & 2 & 3 & 4 \\
0 & -3 & -8 & -12 \\
0 & -3 & -2 & -2 
\end{bmatrix}    
\end{align*}
and the result will be the same.}\par
From the short exercise above, we can see that even if we apply different elementary row operations (particularly for the creation of leading $1$s) during Gauss-Jordan Elimination, we will acquire the same reduced echelon form in the end. In fact,
\begin{thm}[Uniqueness of Reduced Row Echelon Form]
\label{thm:uniquerref}
The reduced row echelon form (RREF) of a matrix is unique.
\end{thm}
We shall omit the proof here. The following property further reveals how elementary row operations are relevant to reduced row echelon form.
\begin{proper}
\label{proper:rowequiv}
If a matrix can be transformed into another matrix by elementary row operations, they are said to be \index{Row Equivalent}\keywordhl{row equivalent}.
\end{proper}
Since for any pair of row equivalent matrices, either of them can be transformed into the other one by elementary row operations and hence can be further transformed into the reduced row echelon form of the other matrix, by Theorem \ref{thm:uniquerref}, the uniqueness of RREF implies that
\begin{proper}
\label{proper:rowequivreduce}
Row equivalent matrices have the same reduced row echelon form. Particularly, they are row equivalent to this RREF. If two matrices have different reduced row echelon forms, then they are not row equivalent, and vice versa.
\end{proper}
Let's go through one more simple example of Gauss-Jordan Elimination.
\begin{exmp}
\label{exmp:rref2}
Transform the following matrix into reduced row echelon form.
\begin{align*}
A =
\begin{bmatrix}
2 & 2 & 1 \\
6 & 4 & 1 \\
2 & 3 & 2 \\
2 & 1 & 0
\end{bmatrix}    
\end{align*}
\end{exmp}
\begin{solution}
One possible way to do the forward elimination is
\begin{align*}
\left[\begin{array}{@{\,}wc{10pt}wc{10pt}wc{10pt}@{\,}}
2 & 2 & 1 \\[3pt]
6 & 4 & 1 \\[3pt]
2 & 3 & 2 \\[3pt]
2 & 1 & 0
\end{array}\right]
&\to
\left[\begin{array}{@{\,}wc{10pt}wc{10pt}wc{10pt}@{\,}}
6 & 4 & 1 \\[3pt]
2 & 2 & 1 \\[3pt]
2 & 3 & 2 \\[3pt]
2 & 1 & 0
\end{array}\right]
& R_1 \leftrightarrow R_2 \\
&\to
\left[\begin{array}{@{\,}wc{10pt}wc{10pt}wc{10pt}@{\,}}
1 & \frac{2}{3} & \frac{1}{6} \\[3pt]
2 & 2 & 1 \\[3pt]
2 & 3 & 2 \\[3pt]
2 & 1 & 0
\end{array}\right]
& \frac{1}{6}R_1 \to R_1 \\
&\to
\left[\begin{array}{@{\,}wc{10pt}wc{10pt}wc{10pt}@{\,}}
1 & \frac{2}{3} & \frac{1}{6} \\[3pt]
0 & \frac{2}{3} & \frac{2}{3} \\[3pt]
0 & \frac{5}{3} & \frac{5}{3} \\[3pt]
0 & -\frac{1}{3} & -\frac{1}{3}
\end{array}\right]
& 
\begin{aligned}
R_2 - 2R_1 &\to R_2 \\
R_3 - 2R_1 &\to R_3 \\
R_4 - 2R_1 &\to R_4 
\end{aligned}\\
&\to
\left[\begin{array}{@{\,}wc{10pt}wc{10pt}wc{10pt}@{\,}}
1 & \frac{2}{3} & \frac{1}{6} \\[3pt]
0 & 1 & 1 \\[3pt]
0 & \frac{5}{3} & \frac{5}{3} \\[3pt]
0 & -\frac{1}{3} & -\frac{1}{3}
\end{array}\right]
& \frac{3}{2}R_2 \to R_2 \\
&\to
\left[\begin{array}{@{\,}wc{10pt}wc{10pt}wc{10pt}@{\,}}
1 & \frac{2}{3} & \frac{1}{6} \\[3pt]
0 & 1 & 1 \\[3pt]
0 & 0 & 0 \\[3pt]
0 & 0 & 0
\end{array}\right]
&
\begin{aligned}
R_3 - \frac{5}{3}R_2 &\to R_3\\
R_4 + \frac{1}{3}R_2 &\to R_4     
\end{aligned}
\end{align*}
The backward elimination is straightforward.
\begin{align*}
\left[\begin{array}{@{\,}wc{10pt}wc{10pt}wc{10pt}@{\,}}
1 & \frac{2}{3} & \frac{1}{6} \\[3pt]
0 & 1 & 1 \\[3pt]
0 & 0 & 0 \\[3pt]
0 & 0 & 0
\end{array}\right] 
&\to
\left[\begin{array}{@{\,}wc{10pt}wc{10pt}wc{10pt}@{\,}}
1 & 0 & -\frac{1}{2} \\[3pt]
0 & 1 & 1 \\[3pt]
0 & 0 & 0 \\[3pt]
0 & 0 & 0
\end{array}\right]
& R_1 - \frac{2}{3}R_2 \to R_1
\end{align*}
The rank of the matrix can be readily seen to be $2$.
\end{solution}

\subsection{Finding Inverses by Gaussian Elimination}
\label{subsection:invGauss}
With Gaussian Elimination, obtaining the inverse $A^{-1}$ of any invertible matrix $A$ is now possible. We start by writing an identity matrix $I$ of the same shape and concatenate this identity matrix to the right of $A$, leading to an augmented form of $[A|I]$. Then we carry out elementary row operations simultaneously on both sides of $[A|I]$ such that the matrix on the left, originally as $A$, is reduced to the identity matrix $I$ by Gaussian Elimination. The identity matrix on the right will then be transformed into the desired inverse by the same set of elementary row operations and the concatenated matrix will now appear as $[I|A^{-1}]$. 
\begin{exmp}
Find the inverse of
\begin{align*}
A =
\begin{bmatrix}
1 & 4 & 5 \\
0 & 2 & 3 \\
0 & 1 & 1
\end{bmatrix}
\end{align*}
by Gaussian Elimination.
\end{exmp}
\begin{solution}
Appending a $3 \times 3$ identity matrix to the right, we have
\begin{align*} 
\left[\begin{array}{@{}wc{10pt}wc{10pt}wc{10pt}|wc{10pt}wc{10pt}wc{10pt}@{\,}}
1 & 4 & 5 & 1 & 0 & 0 \\
0 & 2 & 3 & 0 & 1 & 0 \\
0 & 1 & 1 & 0 & 0 & 1
\end{array}\right] 
& \to
\left[\begin{array}{@{}wc{10pt}wc{10pt}wc{10pt}|wc{10pt}wc{10pt}wc{10pt}@{\,}}
1 & 4 & 5 & 1 & 0 & 0 \\
0 & 0 & 1 & 0 & 1 & -2 \\
0 & 1 & 1 & 0 & 0 & 1
\end{array}\right] & R_2 - 2R_3 \to R_2 \\
& \to
\left[\begin{array}{@{}wc{10pt}wc{10pt}wc{10pt}|wc{10pt}wc{10pt}wc{10pt}@{\,}}
1 & 4 & 5 & 1 & 0 & 0 \\
0 & 1 & 1 & 0 & 0 & 1 \\
0 & 0 & 1 & 0 & 1 & -2 
\end{array}\right] & R_2 \leftrightarrow R_3 \\
& \to
\left[\begin{array}{@{}wc{10pt}wc{10pt}wc{10pt}|wc{10pt}wc{10pt}wc{10pt}@{\,}}
1 & 4 & 5 & 1 & 0 & 0 \\
0 & 1 & 0 & 0 & -1 & 3 \\
0 & 0 & 1 & 0 & 1 & -2 
\end{array}\right] & R_2 - R_3 \to R_2 \\
& \to
\left[\begin{array}{@{}wc{10pt}wc{10pt}wc{10pt}|wc{10pt}wc{10pt}wc{10pt}@{\,}}
1 & 4 & 0 & 1 & -5 & 10 \\
0 & 1 & 0 & 0 & -1 & 3 \\
0 & 0 & 1 & 0 & 1 & -2 
\end{array}\right] & R_1 - 5R_3 \to R_1 \\
& \to
\left[\begin{array}{@{}wc{10pt}wc{10pt}wc{10pt}|wc{10pt}wc{10pt}wc{10pt}@{\,}}
1 & 0 & 0 & 1 & -1 & -2 \\
0 & 1 & 0 & 0 & -1 & 3 \\
0 & 0 & 1 & 0 & 1 & -2 
\end{array}\right] & R_1 - 4R_2 \to R_1 
\end{align*}
Hence the required inverse is
\begin{align*}
A^{-1} =
\begin{bmatrix}
1 & -1 & -2 \\
0 & -1 & 3 \\
0 & 1 & -2 
\end{bmatrix}    
\end{align*}
\end{solution}
Short Exercise: Verify the inverse of $A^{-1}$ above is just $A$ by the same method. \footnote{You should be able to retrieve the matrix $A$ back. The first column of $A^{-1}$ already contains a leading 1 and elements below which are zeros. A possible next step is to multiply $R_2$ by $-1$ and then subtract $R_3$ by $R_2$.}\par
The underlying reason why the above procedure can produce the inverse matrix is the equivalence between elementary row operations and multiplication by appropriate \index{Elementary Matrix}\keywordhl{elementary matrices}.
\begin{proper}[Elementary Matrices]
\label{proper:elementarymat}
Any elementary row operation on an $m \times n$ matrix can be represented by multiplying it to the left with a suitable \textit{elementary matrix}. Such a matrix is essentially the one that appears after applying that particular elementary row operation on an identity matrix. For the three types of elementary row operations described in Definition \ref{defn:elerowop}:
\begin{enumerate}
\item $cR_{p} \to R_{p}$, $c \neq 0$,
\item $R_{p} + cR_{q} \to R_{p}$,
\item $R_{p} \leftrightarrow R_{q}$
\end{enumerate}
their corresponding elementary matrices $E$ are square ($m \times m$), and \textit{invertible} (see the following remark) in which
\begin{enumerate}
\item $E_{kk} = 1$ for any $k$, except $E_{pp} = c$;
\item $E_{kk} = 1$ for all $k$, with $E_{pq} = c$;
\item $E_{kk} = 1$ for any $k$, except $E_{pp} = 0$ and $E_{qq} = 0$, with $E_{pq} = E_{qp} = 1$. 
\end{enumerate}
Entries not mentioned are all zeros.
\end{proper}
Since it is quite abstract, it is useful to have some actual examples.
\begin{align*}
&
\begin{bmatrix}
1 & 0 & 0 \\
0 & 2 & 0 \\
0 & 0 & 1
\end{bmatrix} & \text{Multiplying $R_2$ by a factor of $2$: } 2R_2 \to R_2 \\
&
\begin{bmatrix}
1 & 3 & 0 \\
0 & 1 & 0 \\
0 & 0 & 1
\end{bmatrix} & \text{Adding 3 times $R_2$ to $R_1$: } R_1 + 3R_2 \to R_1 \\
&
\begin{bmatrix}
0 & 0 & 1 \\
0 & 1 & 0 \\
1 & 0 & 0
\end{bmatrix} & \text{Swapping $R_1$ and $R_3$: } R_1 \leftrightarrow R_3 
\end{align*}
The third type of elementary matrices listed above is known as \textit{permutation matrices}. Any elementary row operation can be apparently undone by an inverse elementary row operation (addition vs subtraction, multiplication vs division ($c \neq 0$), swapping twice). Accordingly, any elementary matrix has another corresponding elementary matrix as its inverse, and the readers are invited to think about their forms in the exercise below. \par
Short Exercise: Write down the inverses of the three example elementary matrices above.\footnote{$
\begin{bmatrix}
1 & 0 & 0 \\
0 & \frac{1}{2} & 0 \\
0 & 0 & 1
\end{bmatrix},
\begin{bmatrix}
1 & -3 & 0 \\
0 & 1 & 0 \\
0 & 0 & 1
\end{bmatrix},
\begin{bmatrix}
0 & 0 & 1 \\
0 & 1 & 0 \\
1 & 0 & 0
\end{bmatrix}
$\\}\par
For instance, consider a matrix
\begin{align*}
\begin{bmatrix}
1 & 4 & 3 \\
2 & 5 & 1 \\
-1 & 0 & 2
\end{bmatrix}     
\end{align*}
then the action of subtracting $R_2$ from $R_3$, $R_3 - R_2 \to R_3$. can be expressed as
\begin{align*}
\begin{bmatrix}
1 & 0 & 0 \\
0 & 1 & 0 \\
0 & -1 & 1
\end{bmatrix} 
\begin{bmatrix}
1 & 4 & 3 \\
2 & 5 & 1 \\
-1 & 0 & 2
\end{bmatrix} 
&= 
\begin{bmatrix}
1 & 4 & 3 \\
2 & 5 & 1 \\
-3 & -5 & 1
\end{bmatrix} 
\end{align*}
Short Exercise: Find out the $3 \times 3$ elementary matrix for subtracting $2$ times the third row from the first row. What happens when we apply this elementary matrix to the left of the matrix above? \footnote{$
\begin{bmatrix}
1 & 0 & -2 \\
0 & 1 & 0 \\
0 & 0 & 1
\end{bmatrix}$:  
$\begin{bmatrix}
1 & 0 & -2 \\
0 & 1 & 0 \\
0 & 0 & 1
\end{bmatrix}
\begin{bmatrix}
1 & 4 & 3 \\
2 & 5 & 1 \\
-3 & -5 & 1
\end{bmatrix}
=
\begin{bmatrix}
7 & 14 & 1 \\
2 & 5 & 1 \\
-3 & -5 & 1
\end{bmatrix}
$}\par
Now we are ready to see why finding inverses by Gaussian Elimination works.
\begin{thm}
\label{thm:Gausselimprincip}
If a matrix $A$ can be converted to an identity matrix $I$ as its reduced row echelon form by Gaussian Elimination, then it is invertible since the same steps can in turn be applied on $I$, producing its inverse $A^{-1}$. 
\end{thm}
Using the language of Properties \ref{proper:rowequivreduce}, the matrix $A$ has to be row equivalent to $I$ for $A^{-1}$ to exist. This also means if Gaussian Elimination fails to reduce $A$ to $I$ (i.e.\ the RREF of $A$ is some matrix other than the identity), then $A^{-1}$ does not exist.
\begin{proof}
Assume $A$ is invertible and hence $AA^{-1} = I$ (Definition \ref{defn:inverse}). From Properties \ref{proper:elementarymat}, when doing Gaussian Elimination over $A$, the $i$-th elementary row operation executed can be represented by an elementary matrix, denoted as $E_{i}$, for $i = 1,2,\ldots,n$ where $n$ is the total number of steps. If we multiply these $E_{i}$ successively to the left on both sides of the equation $AA^{-1} = I$, we have
\begin{align*}
E_n \cdots E_{3}E_{2}E_{1} AA^{-1} &= E_n \cdots E_{3}E_{2}E_{1}I \\
(E_n \cdots E_{3}E_{2}E_{1}A) A^{-1} &= E_n \cdots E_{3}E_{2}E_{1}I & \text{(Properties \ref{proper:matmul})} \\
(I)A^{-1} &= E_n \cdots E_{3}E_{2}E_{1}I \\
A^{-1} &= E_n \cdots E_{3}E_{2}E_{1}I & \text{(Properties \ref{proper:identity})}
\end{align*}
from the second line to the third line, we have $E_n \cdots E_{3}E_{2}E_{1}A = I$, because the elementary row operations during Gaussian Elimination, represented by $E_i$, $i = 1,2,\ldots,n$, reduce $A$ to $I$ as we demand in the assumption. With $A^{-1} = E_n \cdots E_{3}E_{2}E_{1}I$, we immediately see that the same set of elementary matrices and hence elementary row operations can also transform $I$ into $A^{-1}$, explicitly showing that $A$ is invertible.    
\end{proof}
As a corollary, because we have $E_n \cdots E_{3}E_{2}E_{1}A = I$ from above, and all $E_i$ are invertible by Properties \ref{proper:elementarymat}, we can multiply their inverses $E'_i = E_i^{-1}$ (which are also elementary matrices) to the left on both sides successively, where $i$ runs backward from $n$ to $1$. This leads to
\begin{align}
E_{1}^{-1}E_{2}^{-1}E_{3}^{-1}\cdots E_n^{-1}E_n \cdots E_{3}E_{2}E_{1}A &= E_{1}^{-1}E_{2}^{-1}E_{3}^{-1}\cdots E_n^{-1}I \nonumber \\
A &= E'_{1}E'_{2}E'_{3}\cdots E'_n
\end{align}
as each of the pairs $E_n^{-1}E_n$, $E_{n-1}^{-1}E_{n-1}$, $\ldots$, $E_2^{-1}E_2$, $E_1^{-1}E_1$ on L.H.S. cancels out to produce $I$, and hence
\begin{proper}
\label{proper:invseqelement}
All invertible matrices can be written as a product of some sequence of elementary matrices. 
\end{proper}

\section{Determinants}
\label{section:det}
\subsection{Computing Determinants}
The \index{Determinant}\keywordhl{determinant} of a \textit{square} matrix $A$, denoted by $\det(A)$ or $\abs{A}$, is a number associated with intrinsic geometrical properties of the matrix which can help us to find its inverse (Determinant of non-square matrices is undefined). The determinant of a $1 \times 1$ matrix is equal to the matrix's only entry. Meanwhile, determinants of $2 \times 2$ and $3 \times 3$ matrices can be calculated by a trick called \index{Sarrus' Rule}\keywordhl{Sarrus' Rule}.
\subsubsection{Sarrus' Rule}
\begin{proper}[Sarrus' Rule]
\label{proper:sarrus}
Determinants of size $2 \times 2$ and $3 \times 3$ matrices can be found by the Sarrus' Rule. For a $2 \times 2$ matrix
\begin{align*}
A =
\begin{bmatrix}
a_1 & b_1 \\
a_2 & b_2
\end{bmatrix} 
\end{align*}
Its determinant is computed by
\begin{center}
\begin{tikzpicture}
% https://tex.stackexchange.com/questions/32978/typesetting-a-matrix-with-crossing-arrows-on-it
\matrix[matrix of math nodes, inner sep=5pt, outer sep=-5pt] (sarrus) 
{a_{11} & a_{12} \\
a_{21} & a_{22} \\ };
\path
($(sarrus-1-1.north west)-(0.5em,0)$) edge[thick] ($(sarrus-2-1.south west)-(0.5em,0)$)
($(sarrus-1-2.north east)+(0.5em,0)$) edge[thick] ($(sarrus-2-2.south east)+(0.5em,0)$)
(sarrus-1-1.center)                          edge[red, line width=1, ->]           (sarrus-2-2.south east)
(sarrus-2-1.center)                          edge[blue, line width=1, ->, dashed]  (sarrus-1-2.north east);
\end{tikzpicture}
\end{center}
\vspace{-25pt}
\begin{align}
\abs{A} &= \textcolor{red}{a_{11}a_{22}} - \textcolor{blue}{a_{21}a_{12}}
\end{align}
which is the product of entries crossed by the red arrow, minus the blue one. Similarly, for a $3 \times 3$ matrix
\begin{align*}
A =
\begin{bmatrix}
a_{11} & a_{12} & a_{13} \\
a_{21} & a_{22} & a_{23} \\
a_{31} & a_{32} & a_{33}
\end{bmatrix} 
\end{align*}
Its determinant can be found by
\begin{center}
\begin{tikzpicture}
\matrix[matrix of math nodes, inner sep=5pt, outer sep=-5pt] (sarrus) 
{a_{11} & a_{12} & a_{13} & \textcolor{gray}{a_{11}} & \textcolor{gray}{a_{12}} \\
a_{21} & a_{22} & a_{23} & \textcolor{gray}{a_{21}} & \textcolor{gray}{a_{22}} \\
a_{31} & a_{32} & a_{33} & \textcolor{gray}{a_{31}} & \textcolor{gray}{a_{32}} \\ };
\path
($(sarrus-1-1.north west)-(0.5em,0)$) edge[thick] ($(sarrus-3-1.south west)-(0.5em,0)$)
($(sarrus-1-3.north east)+(0.5em,0)$) edge[thick] ($(sarrus-3-3.south east)+(0.5em,0)$)
(sarrus-1-1.center)                          edge[red, line width=1, ->]           (sarrus-3-3.south east)
(sarrus-1-2.center)                          edge[red, line width=1, ->]           (sarrus-3-4.south east)
(sarrus-1-3.center)                          edge[red, line width=1, ->]           (sarrus-3-5.south east)
(sarrus-3-1.center)                          edge[blue, line width=1, ->, dashed]  (sarrus-1-3.north east)
(sarrus-3-2.center)                          edge[blue, line width=1, ->, dashed]  (sarrus-1-4.north east)
(sarrus-3-3.center)                          edge[blue, line width=1, ->, dashed]  (sarrus-1-5.north east);
\end{tikzpicture}
\end{center}
\vspace{-25pt}
\begin{align}
\abs{A} ={}& \begin{aligned}
&(\textcolor{red}{a_{11}a_{22}a_{33}} + \textcolor{red}{a_{12}a_{23}a_{31}} + \textcolor{red}{a_{13}a_{21}a_{32}}) \\
&- (\textcolor{blue}{a_{31}a_{22}a_{13}} + \textcolor{blue}{a_{32}a_{23}a_{11}} + \textcolor{blue}{a_{33}a_{21}a_{12}})    
\end{aligned}
\end{align}
\end{proper}

\begin{exmp}
Find the determinant of the following matrix.
\begin{align*}
A &=
\begin{bmatrix}
1 & 2 & 4 \\
-5 & 0 & -3 \\
4 & 3 & 1
\end{bmatrix} 
\end{align*}
\end{exmp}
\begin{solution}
By Sarrus's Rule (Properties \ref{proper:sarrus}), we have
\begin{align*}
|A| &=
\begin{vmatrix}
1 & 2 & 4 \\
-5 & 0 & -3 \\
4 & 3 & 1
\end{vmatrix} \\
&= ((1)(0)(1) + (2)(-3)(4) + (4)(-5)(3)) \\
&\quad- ((4)(0)(4) + (3)(-3)(1) + (1)(-5)(2)) \\
&= (0 - 24 - 60) - (0 - 9 - 10) \\
&= -65
\end{align*}
\end{solution}

\subsubsection{Cofactor Expansion}
Another commonly used method to calculate determinants is \textit{Cofactor Expansion}, also known as \textit{Laplace Expansion}. Before discussing cofactor expansion, it is necessary to know what \textit{cofactors} are.
\begin{defn}[Cofactor and Minor]
\label{defn:cofactor}
The \index{Cofactor}\keywordhl{cofactor} $C_{ij}$ at the $(i, j)$ position of a matrix $A$ is simply the determinant of the submatrix formed by deleting the $i$-th row and $j$-th column of $A$, $M_{ij}$ (called the \index{Minor}\keywordhl{minor} at $(i, j)$), times the factor of $(-1)^{i+j}$, that is
\begin{align}
C_{ij} = (-1)^{i+j} \det(M_{ij})    
\end{align}
\end{defn}
The $(-1)^{i+j}$ factor can be visualized as a checkerboard pattern like
\begin{align*}
\begin{bmatrix}
+ & - & + & \cdots \\
- & + & - &  \\
+ & - & + &  \\
\vdots & &  & \ddots
\end{bmatrix}
\end{align*}
So, for a matrix like
\begin{center}
\begin{tikzpicture}
\matrix[matrix of math nodes, inner sep=4pt, outer sep=-4pt, left delimiter={[}, right delimiter={]}] (cofactor) 
{1 & \mathcolor{blue}{3} & \mathcolor{blue}{5} \\
2 & 4 & 6 \\
3 & \mathcolor{blue}{5} & \mathcolor{blue}{7} \\ };
\path
(cofactor-1-1.north) edge[red, thick] (cofactor-3-1.south)
(cofactor-2-1.west) edge[red, thick] (cofactor-2-3.east);
\end{tikzpicture}
\end{center}
Its cofactor at $(2, 1)$ is
\begin{align*}
C_{21} &= (-1)^{(2+1)}
\begin{vmatrix}
\mathcolor{blue}{3} & \mathcolor{blue}{5} \\
\mathcolor{blue}{5} & \mathcolor{blue}{7}
\end{vmatrix} & \text{(Definition \ref{defn:cofactor})} \\
&= (-1)((3)(7) - (5)(5)) & \text{(Properties \ref{proper:sarrus})} \\
&= 4
\end{align*}
Short Exercise: Find $C_{13}$ and $C_{32}$ for the matrix above.\footnote{$C_{13} = (-1)^{1+3}\begin{vmatrix}
2 & 4 \\
3 & 5
\end{vmatrix} = (1)((2)(5)-(3)(4)) = -2$, similarly $C_{32} = 4$.}\\
\\
With \index{Cofactor Expansion}\index{Laplace Expansion}\keywordhl{Cofactor (Laplace) Expansion}, the determinant of a matrix is computed as the sum of products between each entry and the corresponding cofactor along a picked row/column of the matrix.
\begin{proper}[Cofactor/Laplace Expansion]
\label{proper:cofactorex}
The determinant of an $n \times n$ square matrix $A$, $\abs{A}$, can be found by selecting either a fixed row $i$, or column $j$, and adding up the products of every entry-cofactor pair along that row/column. For the former case (selected the $i$-th row), the determinant is computed as
\begin{subequations}
\label{eqn:cofactorexrow}
\begin{align}
\abs{A} &= A_{i1}C_{i1} + A_{i2}C_{i2} + \cdots + A_{in}C_{in} \\
&= \sum_{k=1}^{n} A_{ik}C_{ik}
\end{align}    
\end{subequations}
For the latter case (fixed the $j$-th column), the determinant is similarly found by
\begin{subequations}
\begin{align}
\abs{A} &= A_{1j}C_{1j} + A_{2j}C_{2j} + \cdots + A_{nj}C_{nj} \\
&= \sum_{k=1}^{n} A_{kj}C_{kj}
\end{align}
\end{subequations}
where each of the cofactors $C_{ij}$ is defined as in Definition \ref{defn:cofactor}. \textcolor{red}{Important: regardless of which row or column is chosen, the result is always the same.\footnotemark}
\end{proper}
\footnotetext{This is due to the uniqueness of the determinant as an "alternating $k$-linear form".}
\begin{exmp}
Again, for the matrix
\begin{align*}
A =
\begin{bmatrix}
\mathcolor{red}{1} & \mathcolor{red}{3} & \mathcolor{red}{5} \\
2 & 4 & 6 \\
3 & 5 & 7 
\end{bmatrix}   
\end{align*}
Find its determinant via cofactor expansion.
\end{exmp}
\begin{solution}
According to (\ref{eqn:cofactorexrow}) in Properties \ref{proper:cofactorex}, if we choose the first row to be expanded, its determinant will be
\begin{align*}
|A| &= A_{11}C_{11} + A_{12}C_{12} + A_{13}C_{13} \\
&= (\textcolor{red}{1})\left((-1)^{1+1}
\begin{vmatrix}
4 & 6 \\
5 & 7
\end{vmatrix}\right)
+
(\textcolor{red}{3})\left((-1)^{1+2}
\begin{vmatrix}
2 & 6 \\
3 & 7
\end{vmatrix}\right) \\
& \quad + 
(\textcolor{red}{5})\left((-1)^{1+3}
\begin{vmatrix}
2 & 4 \\
3 & 5
\end{vmatrix}\right) & \text{(Definition \ref{defn:cofactor})}\\
&= (1)(-2) + (3)(4) + (5)(-2) = 0 & \text{(Properties \ref{proper:sarrus})} 
\end{align*}
\end{solution}
Short Exercise: Confirm the answer by carrying out cofactor expansion on another row or column.\footnote{You should able to get $\abs{A} = 0$, no matter which row/column is selected.}
\begin{exmp}
\label{exmp:4x4det}
Find the determinant of
\begin{align*}
A = 
\begin{bmatrix}
1 & 4 & 4 & 4 \\
2 & 0 & 4 & 6 \\
2 & 1 & 1 & 0 \\
6 & 2 & 3 & 1
\end{bmatrix}
\end{align*}\
\end{exmp}
\begin{solution}
It is a $4 \times 4$ matrix and we have to apply cofactor expansion. We can choose any row or column that contains some zero(s) to reduce the computation. Here we pick the second column and by Properties \ref{proper:cofactorex}, we have
\begin{align*}
\abs{A} &= 
(4)(-1)^{1+2}
\begin{vmatrix}
2 & 4 & 6 \\
2 & 1 & 0 \\
6 & 3 & 1 \\
\end{vmatrix}
+ (0)(-1)^{2+2}
\begin{vmatrix}
1 & 4 & 4 \\
2 & 1 & 0 \\
6 & 3 & 1 \\
\end{vmatrix} \\
& \quad + (1)(-1)^{3+2}
\begin{vmatrix}
1 & 4 & 4 \\
2 & 4 & 6 \\
6 & 3 & 1 \\
\end{vmatrix} 
+ (2)(-1)^{4+2}
\begin{vmatrix}
1 & 4 & 4 \\
2 & 4 & 6 \\
2 & 1 & 0
\end{vmatrix}     
\end{align*}
By Sarrus' Rule (Properties \ref{proper:sarrus}), we can calculate each of the four $3 \times 3$ determinants (the detailed calculations are omitted, notice that we don't need to actually compute the second determinant) and obtain
\begin{align*}
\abs{A} = (-4)(-6) + 0 + (-1)(50) + (2)(18) = 10
\end{align*}
\end{solution}
Finally, we can derive two simple results about determinants from the perspective of cofactor expansion.
\begin{proper}
\label{proper:zerodet}
If a matrix has a row/column with full zeros, or two identical/proportional rows/columns, then it has a determinant of zero.
\end{proper}
The first case is trivial (just do the expansion along the row/column with full zeros) and we will show the second case alongside the introduction of the properties of determinants in the upcoming subsection.

\subsection{Properties of Determinants} There are some notable properties of determinants. First of all, it is very easy to see that the determinant for any $n \times n$ identity matrix $I_n$ is just $1$. Second, there is a close relation between elementary row operations/elementary matrices and (their effects on) determinants, noted as follows.
\begin{proper}
\label{proper:elementaryopdet}
The three types of elementary row operations in Definition \ref{defn:elerowop}, when applied on some square matrix $A$,
\begin{enumerate}
\item $cR_{p} \to R_{p}$, $c \neq 0$,
\item $R_{p} + cR_{q} \to R_{p}$,
\item $R_{p} \leftrightarrow R_{q}$,
\end{enumerate}
change the determinant of $A$ by a factor of $c$, $1$ (unchanged), and $-1$ (switching the sign), respectively.
\end{proper}
\begin{proper}
\label{proper:elementarymatdet}
The three types of elementary matrices $E$ in Properties \ref{proper:elementarymat} that correspond to the elementary row operations in Definition \ref{defn:elerowop},
\begin{enumerate}
\item $E_{kk} = 1$ for any $k$, except $E_{pp} = c$ ($cR_{p} \to R_{p}$, $c \neq 0$),
\item $E_{kk} = 1$ for all $k$, with $E_{pq} = c$ ($R_{p} + cR_{q} \to R_{p}$),
\item $E_{kk} = 1$ for any $k$, except $E_{pp} = 0$ and $E_{qq} = 0$, with $E_{pq} = E_{qp} = 1$ ($R_{p} \leftrightarrow R_{q}$),
\end{enumerate}
have a determinant of $c$, $1$, and $-1$, respectively.
\end{proper}
We will prove the above properties for the second kind of elementary row
operations/elementary matrices (corresponding to addition/subtraction) in Appendix \ref{section:invdetappend}. The properties for the two other types of elementary matrices are easy to show and we will take them for granted, such that we can establish the second case in Properties \ref{proper:zerodet}, which is in turn used for demonstrating the later results.\par
Since the determinants of elementary matrices, by Properties \ref{proper:elementarymatdet}, coincide exactly with the factors by how the determinant of a square matrix $A$ changes when the corresponding elementary row operations are applied on $A$ (represented by multiplication to the left of $A$ by these elementary matrices) as shown in Properties \ref{proper:elementaryopdet}, we conclude that
\begin{proper}
\label{proper:elementarytimesdet}
For any elementary matrix $E$ and a square matrix $A$, we have
\begin{align}
\det(EA) = \det(E)\det(A)
\end{align}
\end{proper}
This property will be of use when we later prove other properties of determinant. However, before doing so, we will demonstrate how to utilize Properties \ref{proper:elementaryopdet} (or equivalently \ref{proper:elementarymatdet}) to ease the calculation of determinants.
\begin{exmp}
Re-do Example \ref{exmp:4x4det} utilizing Properties \ref{proper:elementaryopdet}.
\end{exmp}
\begin{solution}
We can factor out the $2$ in the second row and subtract $3$ times the third row from the fourth row. By Properties \ref{proper:elementaryopdet}, we have
\begin{align*}
|A| = 
\begin{vmatrix}
1 & 4 & 4 & 4 \\
2 & 0 & 4 & 6 \\
2 & 1 & 1 & 0 \\
6 & 2 & 3 & 1
\end{vmatrix}
&=
2
\begin{vmatrix}
1 & 4 & 4 & 4 \\
1 & 0 & 2 & 3 \\
2 & 1 & 1 & 0 \\
6 & 2 & 3 & 1
\end{vmatrix} = 
2
\begin{vmatrix}
1 & 4 & 4 & 4 \\
1 & 0 & 2 & 3 \\
2 & 1 & 1 & 0 \\
0 & -1 & 0 & 1
\end{vmatrix} 
\end{align*}
The new determinant can be computed by doing cofactor expansion along the fourth row which now contains two zeros. With Properties \ref{proper:cofactorex} and \ref{proper:sarrus}, it is
\begin{align*}
\begin{vmatrix}
1 & 4 & 4 & 4 \\
1 & 0 & 2 & 3 \\
2 & 1 & 1 & 0 \\
0 & -1 & 0 & 1
\end{vmatrix}
&= 0 + (-1)^{4+2}(-1)
\begin{vmatrix}
1 & 4 & 4 \\
1 & 2 & 3 \\
2 & 1 & 0 
\end{vmatrix} 
+ 0 + (-1)^{4+4}(1)
\begin{vmatrix}
1 & 4 & 4 \\
1 & 0 & 2 \\
2 & 1 & 1 
\end{vmatrix} \\
&= 0 + (-1)(9) + 0 + (1)(14) = 5
\end{align*}
and hence $\abs{A} = 2(5) = 10$.
\end{solution}

With Properties \ref{proper:elementarytimesdet}, we can unearth the relation between the invertibility of a square matrix and its determinant.
\begin{proper}
\label{proper:invnonzerodet}
An invertible matrix has a non-zero determinant. Otherwise, a singular matrix has a determinant of zero.
\end{proper}
\begin{proof}
Let's denote the matrix in question as $A$. For the case in which $A$ is invertible, by Properties \ref{proper:invseqelement} it can be written as the product of some elementary matrices $E_1$, $E_2$, $\ldots$, $E_{n-1}$, $E_n$, i.e.\
\begin{align*}
A = E_{1}E_{2} \cdots E_{n-1}E_n
\end{align*}
Taking the determinant of both sides, we have
\begin{align*}
\det(A) &= \det(E_{1}E_{2} \cdots E_{n-1}E_n)
\end{align*}
By repetitively using Properties \ref{proper:elementarytimesdet}, we have
\begin{align*}
\det(A) &= \det(E_{1}(E_{2} \cdots E_{n-1}E_n)) \\
&= \det(E_1) \det(E_{2} \cdots E_{n-1}E_n) \\
&= \det(E_1) \det(E_{2}) \det(\cdots E_{n-1}E_n) \\
&= \det(E_1) \det(E_{2}) \cdots \det(E_{n-1})\det(E_n)
\end{align*}
Since by Properties \ref{proper:elementarymatdet}, all elementary matrices have a non-zero determinant (particularly we have required $c \neq 0$ when multiplying a row), i.e. $\det(E_i) \neq 0$ for all $i$, we have $\det(A) \neq 0$. We will not go through the details for singular matrices, which are put in the footnote below for reference.\footnote{By Theorem \ref{thm:Gausselimprincip}, singular matrices have reduced row echelon forms that are not the identity. Observe that all other square RREFs that are not the identity must have at least one row of full zeros, and by Properties \ref{proper:zerodet} they will have a determinant of zero.} 
\end{proof}
Other properties of determinants include:
\begin{proper}
\label{proper:properdet}
For any $n \times n$ square matrices $A$ and $B$, we have
\begin{enumerate}
\item $\det(A^T) = \det(A)$;
\item $\det(kA) = k^n \det(A)$, for any constant $k$;
\item $\det(AB) = \det(A)\det(B)$; and
\item $\det(A^{-1}) = \dfrac{1}{\det(A)}$, if $A$ is invertible.
\end{enumerate}
By extension of (3), $\det(A_1A_2\cdots A_n) = \det(A_1)\det(A_2)\cdots\det(A_n)$.
\end{proper}
For instance, if
\begin{align*}
&A = 
\begin{bmatrix}
2 & 3 \\
5 & 9 \\
\end{bmatrix}
&B = 
\begin{bmatrix}
4 & 5 \\
1 & 0 \\
\end{bmatrix}
\end{align*}
then
\begin{align*}
&\abs{A} = (2)(9) - (3)(5) = 3 
&\abs{B} = (4)(0) - (5)(1) = -5
\end{align*}
\begin{align*}
AB &= 
\begin{bmatrix}
2 & 3 \\
5 & 9 \\
\end{bmatrix}
\begin{bmatrix}
4 & 5 \\
1 & 0 \\
\end{bmatrix} \\
&= 
\begin{bmatrix}
(2)(4)+(3)(1) & (2)(5)+(3)(0) \\
(5)(4)+(9)(1) & (5)(5)+(9)(0)
\end{bmatrix} \\
&= 
\begin{bmatrix}
11 & 10 \\
29 & 25
\end{bmatrix} \\
\abs{AB} &= (11)(25) - (10)(29) \\
&= -15 = (3)(-5) = \abs{A}\abs{B}
\end{align*}
So we can see in this case, $\det(AB) = \det(A)\det(B)$ indeed. We put the formal proof for (3) of Properties \ref{proper:properdet} in the footnote for reference.\footnote{There are two cases to consider, $A$ being invertible or singular. If $A$ is singular, then by Properties \ref{proper:ABinv}, $AB$ is also singular, and by Properties \ref{proper:invnonzerodet}, both $\det(A)$ and $\det(AB)$ will be zero, and the equality holds trivially. Otherwise, if $A$ is invertible, then we can follow the idea in the proof of Properties \ref{proper:invnonzerodet}, and let $A = E_{1}E_{2} \cdots E_{n-1}E_n$ as a sequence of elementary matrices. By using Properties \ref{proper:elementarytimesdet} back and forth, we have
\begin{align*}
\det(AB) &= \det(E_{1}E_{2} \cdots E_{n-1}E_nB) \\
&= \det(E_1) \det(E_{2}) \cdots \det(E_{n-1})\det(E_n)\det(B) & \text{(Properties \ref{proper:elementarytimesdet})} \\
&= (\det(E_1) \det(E_{2}) \cdots \det(E_{n-1})\det(E_n))\det(B) \\
&= \det(E_{1}E_{2} \cdots E_{n-1}E_n)\det(B) & \text{(Properties \ref{proper:elementarytimesdet})} \\
&= \det(A)\det(B)
\end{align*}
So the equality is true in both cases.}

Short Exercise: Prove (4) of Properties \ref{proper:properdet}.\footnote{Consider $A^{-1}A=I$, and take determinant on both sides. By (3), we have
\begin{align*}
\det(A^{-1}A) &= \det(I) \\
\det(A^{-1})\det(A) &= 1 & \text{(The identity always has a determinant of $1$)} \\
\det(A^{-1}) &= \frac{1}{\det(A)}
\end{align*}}

\subsection{Finding Inverses by Adjugate}
An alternative method to compute the inverse of a matrix is by using its \index{Adjugate}\keywordhl{adjugate}, which is the transpose of its cofactor matrix associated with it.
\begin{defn}[Adjugate]
For a matrix $A$, its adjugate is defined as
\begin{align}
[\text{adj}(A)]_{pq} = (C_{pq})^T = C_{qp}
\end{align}
where $C_{pq}$ is the cofactor of $A$ at $(p, q)$, formulated as in Definition \ref{defn:cofactor}.
\end{defn}
\begin{proper}
\label{proper:invadj}
The inverse of a matrix $A$ can be computed from its adjugate by
\begin{align}
A^{-1} = \frac{1}{\det(A)}\text{adj}(A)
\end{align}
\end{proper}
From this formula, it is obvious that singular matrices, having a determinant of zero, do not have an inverse.
\begin{exmp}
\label{exmp:2x2}
For a $2 \times 2$ matrix
\begin{align*}
A = \begin{bmatrix}
a & b \\
c & d
\end{bmatrix}    
\end{align*}
It is not difficult to see that its determinant is $ad - bc$, and the adjugate matrix is
\begin{align*}
\text{adj}(A) = 
\begin{bmatrix}
d & -c \\
-b & a 
\end{bmatrix}^T = 
\begin{bmatrix}
d & -b \\
-c & a 
\end{bmatrix}    
\end{align*}
So the inverse, if $ad - bc \neq 0$, is
\begin{align}
A^{-1} = 
\frac{1}{ad-bc}
\begin{bmatrix}
d & -b \\
-c & a 
\end{bmatrix}
\end{align}
\end{exmp}
\begin{exmp}
Find the inverse of the following matrix by evaluating its adjugate.
\begin{align*}
A &= 
\begin{bmatrix}
1 & 2 & 3 \\
1 & 3 & 5 \\
1 & 4 & 11
\end{bmatrix}
\end{align*}
\end{exmp}
\begin{solution}
First of all, by Sarrus' Rule (Properties \ref{proper:sarrus})
\begin{align*}
|A| &= ((1)(3)(11) + (2)(5)(1) + (3)(1)(4))\\
&\quad- ((3)(3)(1) + (1)(5)(4) + (2)(1)(11)) \\
&= (33 + 10 + 12) - (9 + 20 + 22) \\
&= 4
\end{align*}
The adjugate matrix is
\begin{align*}
\text{adj}(A) &=
\begin{bmatrix*}[r]
\begin{vmatrix}
3 & 5 \\
4 & 11
\end{vmatrix}
&
-\begin{vmatrix}
1 & 5 \\
1 & 11
\end{vmatrix}
&
\begin{vmatrix}
1 & 3 \\
1 & 4
\end{vmatrix}\\[10pt]
-\begin{vmatrix}
2 & 3 \\
4 & 11
\end{vmatrix}
&
\begin{vmatrix}
1 & 3 \\
1 & 11
\end{vmatrix}
&
-\begin{vmatrix}
1 & 2 \\
1 & 4
\end{vmatrix}\\[10pt]
\begin{vmatrix}
2 & 3 \\
3 & 5
\end{vmatrix}
&
-\begin{vmatrix}
1 & 3 \\
1 & 5
\end{vmatrix}
&
\begin{vmatrix}
1 & 2 \\
1 & 3 
\end{vmatrix} 
\end{bmatrix*}^{\color{red}{T}} \\
&= 
\begin{bmatrix}
13 & \textcolor{red}{-6} & 1 \\
\textcolor{red}{-10} & 8 & -2 \\
1 & -2 & 1
\end{bmatrix}^{\color{red}{T}} = 
\begin{bmatrix}
13 & \textcolor{red}{-10} & 1 \\
\textcolor{red}{-6} & 8 & -2 \\
1 & -2 & 1
\end{bmatrix}
\end{align*}
(be careful not to forget the transpose!) Putting the pieces together according to the formula in Properties \ref{proper:invadj}, we have
\begin{align*}
A^{-1} &= \frac{1}{\det(A)}\text{adj}(A) \\
&= \frac{1}{4}
\begin{bmatrix}
13 & -10 & 1 \\
-6 & 8 & -2 \\
1 & -2 & 1
\end{bmatrix} \\
&= 
\begin{bmatrix}
\frac{13}{4} & -\frac{5}{2} & \frac{1}{4} \\[3pt]
-\frac{3}{2} & 2 & -\frac{1}{2} \\[3pt]
\frac{1}{4} & -\frac{1}{2} & \frac{1}{4}
\end{bmatrix}
\end{align*}
\end{solution}
A summarizing point to be emphasized is that
\begin{thm}[Equivalence Statements]
\label{thm:equiv1}
For a square matrix $A$, the followings are equivalent:
\begin{enumerate}[label=(\alph*)]
\item $A$ is invertible, i.e.\ $A^{-1}$ exists,
\item $\det(A) \neq 0$,
\item The reduced row echelon form of $A$ is the identity $I$.
\end{enumerate}
\end{thm}
which is just a rephrasing of Properties \ref{proper:invnonzerodet} and Theorem \ref{thm:Gausselimprincip}. 
Particularly, invertibility is equivalent to a non-zero determinant. We will see the expansion of these equivalence statements in later chapters.

\section{Python Programming}
\label{section:ch2python}
To create an identity matrix of size $n$, we use \verb|np.identity(n)|. For example,
\begin{lstlisting}
import numpy as np
I4 = np.identity(4)
print(I4)
\end{lstlisting}
returns
\begin{lstlisting}
[[1. 0. 0. 0.]
 [0. 1. 0. 0.]
 [0. 0. 1. 0.]
 [0. 0. 0. 1.]]
\end{lstlisting}
Applying transpose on a matrix is simple where we just add \verb|.T| after the array variable, like
\begin{lstlisting}
myMatrix1 = np.array([[1.,  0., 3.],
                      [1.,  4., 1.],
                      [-1., 2., 4.]])
print(myMatrix1)
print(myMatrix1.T)
\end{lstlisting}
yields
\begin{lstlisting}
[[ 1.  0.  3.]
 [ 1.  4.  1.]
 [-1.  2.  4.]]
[[ 1.  1. -1.]
 [ 0.  4.  2.]
 [ 3.  1.  4.]]
\end{lstlisting}
Finding the inverse of a matrix requires the \verb|scipy.linalg| library and call the \verb|inv| function.
\begin{lstlisting}
from scipy import linalg
myMatrix2 = linalg.inv(myMatrix1)
print(myMatrix2)
print(myMatrix1 @ myMatrix2) # Check: should give the identity
\end{lstlisting}
gives the expected results of
\begin{lstlisting}
[[ 0.4375   0.1875  -0.375  ]
 [-0.15625  0.21875  0.0625 ]
 [ 0.1875  -0.0625   0.125  ]]
[[1. 0. 0.]
 [0. 1. 0.]
 [0. 0. 1.]]
\end{lstlisting}
Meanwhile, we can use the \verb|det| function to calculate the determinant of a matrix as follows. First,
\begin{lstlisting}
print(linalg.det(myMatrix1))
\end{lstlisting}
gives the expected output of \verb|32.0|. As another example,
\begin{lstlisting}
myMatrix3 = np.array([[3.,  1.,  3., 2.],
                      [0., -1., -3., 1.],
                      [1., -1., -2., 0.],
                      [2.,  0.,  1., 0.]])
print(linalg.det(myMatrix3))  
\end{lstlisting}
produces an extremely small value of \verb|1.11022302e-16|. In fact, the matrix
\begin{align*}
\begin{bmatrix}
3 & 1 & 3 & 2 \\
0 & -1 & -3 & 1 \\
1 & -1 & -2 & 0 \\
2 & 0 & 1 & 0    
\end{bmatrix}
\end{align*}
has a determinant of exactly zero. It is an artifact of numerical error when using floating point numbers. If we keep going ahead and compute its inverse by \verb|linalg.inv(myMatrix3)|, we will obtain an absurd output of
\begin{lstlisting}
[[ 1.200959e+15 -2.401919e+15  3.602879e+15 -3.602879e+15]
 [ 6.004799e+15 -1.200959e+16  1.801439e+16 -1.801439e+16]
 [-2.401919e+15  4.803839e+15 -7.205759e+15  7.205759e+15]
 [-1.200959e+15  2.401919e+15 -3.602879e+15  3.602879e+15]]
\end{lstlisting}
that have entries of extremely large magnitude. This phenomenon is due to the "extremely small determinant", through Properties \ref{proper:invadj}, magnifying the adjugate by being in the denominator. (The actual computation does not use Properties \ref{proper:invadj} directly but this is a heuristic perspective to view the problem.) To prevent this, we can add a \verb|if| condition to look for singularity, defining a function like
\begin{lstlisting}
def safe_inv(matrix):
    if np.abs(linalg.det(matrix)) < np.finfo(float).eps:
        print("Warning: The matrix is highly singular!")
        return(np.nan)
    else:
        return(linalg.inv(matrix))
\end{lstlisting}
where \verb|np.finfo(float).eps| gives the so-called \textit{machine epsilon} $\epsilon$ (the order of relative round-off error) of \verb|float| and we want the absolute value of the determinant to be larger than that. Subsequently, calling \verb|safe_inv(myMatrix3)| will print a warning. Finally, we note that we can use \verb|sympy| to acquire the reduced row echelon form of a matrix. Let's use the matrix in Example \ref{exmp:rref2} for demonstration.
\begin{lstlisting}
import sympy

myMatrix4 = np.array([[2., 2., 1.],
                      [6., 4., 1.],
                      [2., 3., 2.],
                      [2., 1., 0.]])
myMatrix4_sympy = sympy.Matrix(myMatrix4) # Convert the numpy array to a sympy matrix
print(myMatrix4_sympy.rref())
\end{lstlisting}
then returns two objects
\begin{lstlisting}
(Matrix([
[1, 0, -0.5],
[0, 1,  1.0],
[0, 0,    0],
[0, 0,    0]]), (0, 1))    
\end{lstlisting}
The first one is the reduced row echelon form we want, and the second is a tuple that keeps the column indices of the pivots. \verb|sympy| also does \textit{zero testing} such that
\begin{lstlisting}
myMatrix3_sympy = sympy.Matrix(myMatrix3)
print(myMatrix3_sympy**(-1))    
\end{lstlisting}
raises properly the error of
\begin{lstlisting}
NonInvertibleMatrixError("Matrix det == 0; not invertible.") sympy.matrices.common.NonInvertibleMatrixError: Matrix det == 0; not invertible. 
\end{lstlisting}

\section{Exercises}

\begin{Exercise}
Find the determinant of the matrix below by inspection.
\begin{align*}
\begin{bmatrix}
1 & 2 & 3 & 4 & 5 \\
0 & 6 & 7 & 8 & 9 \\
0 & 0 & 10 & 11 & 12 \\
0 & 0 & 0 & 13 & 14 \\
0 & 0 & 0 & 0 & 15
\end{bmatrix}    
\end{align*}
By the same logic, derive a general formula for the determinant of any upper(lower)-triangular matrix.\footnote{An upper(lower)-triangular matrix is a matrix whose elements below (above) the main diagonal are all zeros.}
\end{Exercise}
\begin{Answer}
(Applying cofactor expansion along the leftmost column recursively) The determinant is just the product of the diagonal elements $= (1)(6)(10)(13)(15) = 11700$.    
\end{Answer}

\begin{Exercise}
Let
\begin{align*}
&A =
\begin{bmatrix}
2 & 3\\
5 & 7
\end{bmatrix}
&B =
\begin{bmatrix}
4 & 6\\
0 & 1
\end{bmatrix}  
\end{align*}
Verify:
\begin{enumerate}[label=(\alph*)]
\item $(AB)^T = B^TA^T$,
\item $(AB)^{-1} = B^{-1}A^{-1}$, and
\item $\det(AB) = \det(A)\det(B)$.
\end{enumerate} for this particular case.
\end{Exercise}
\begin{Answer}
\begin{enumerate}[label=(\alph*)]
\item 
$\begin{bmatrix}
8 & 20\\
15 & 37
\end{bmatrix}
=
\begin{bmatrix}
4 & 0 \\
6 & 1
\end{bmatrix}
\begin{bmatrix}
2 & 5 \\
3 & 7
\end{bmatrix}
$
\item $\begin{bmatrix}
-\frac{37}{4} & \frac{15}{4}\\
5 & -2    
\end{bmatrix} = 
\begin{bmatrix}
\frac{1}{4} & -\frac{3}{2}\\
0 & 1 
\end{bmatrix}
\begin{bmatrix}
-7 & 3\\
5 & -2  
\end{bmatrix}$
\item $\begin{vmatrix}
8 & 15\\
20 & 37
\end{vmatrix}
= -4 = (-1)(4) =
\begin{vmatrix}
2 & 3\\
5 & 7
\end{vmatrix}
\begin{vmatrix}
4 & 6\\
0 & 1
\end{vmatrix}$
\end{enumerate}
\end{Answer}

\begin{Exercise}
If
\begin{align*}
A =
\begin{bmatrix}
3 & 2 & 9\\
1 & 2 & 3\\
4 & 0 & 4
\end{bmatrix}  
\end{align*}
Find its inverse by 
\begin{enumerate}[label=(\alph*)]
\item Gaussian Elimination, and
\item Determinant and adjugate.
\end{enumerate}
\end{Exercise}
\begin{Answer}
\begin{enumerate}[label=(\alph*)]
\item \begin{align*}
& \left[\begin{array}{@{}ccc|ccc@{}}
3 & 2 & 9 & 1 & 0 & 0 \\
1 & 2 & 3 & 0 & 1 & 0 \\
4 & 0 & 4 & 0 & 0 & 1
\end{array}\right] \\
\to & 
\left[\begin{array}{@{}ccc|ccc@{}}
1 & 2 & 3 & 0 & 1 & 0 \\
3 & 2 & 9 & 1 & 0 & 0 \\
4 & 0 & 4 & 0 & 0 & 1
\end{array}\right] & R_1 \leftrightarrow R_2 \\    
\to & 
\left[\begin{array}{@{}ccc|ccc@{}}
1 & 2 & 3 & 0 & 1 & 0 \\
0 & -4 & 0 & 1 & -3 & 0 \\
0 & -8 & -8 & 0 & -4 & 1
\end{array}\right] & 
R_2 - 3R_1 \to R_2, R_3 - 4R_1 \to R_3 \\
\to & 
\left[\begin{array}{@{}ccc|ccc@{}}
1 & 2 & 3 & 0 & 1 & 0 \\
0 & 1 & 0 & -\frac{1}{4} & \frac{3}{4} & 0 \\
0 & 1 & 1 & 0 & \frac{1}{2} & -\frac{1}{8}
\end{array}\right] & 
-\frac{1}{4}R_2 \to R_2, -\frac{1}{8}R_3 \to R_3 \\
\to & 
\left[\begin{array}{@{}ccc|ccc@{}}
1 & 2 & 3 & 0 & 1 & 0 \\
0 & 1 & 0 & -\frac{1}{4} & \frac{3}{4} & 0 \\
0 & 0 & 1 & \frac{1}{4} & -\frac{1}{4} & -\frac{1}{8}
\end{array}\right] & 
R_3 - R_2 \to R_3 \\
\to &
\left[\begin{array}{@{}ccc|ccc@{}}
1 & 0 & 0 & -\frac{1}{4} & \frac{1}{4} & \frac{3}{8} \\
0 & 1 & 0 & -\frac{1}{4} & \frac{3}{4} & 0 \\
0 & 0 & 1 & \frac{1}{4} & -\frac{1}{4} & -\frac{1}{8}
\end{array}\right] & 
R_1 - 3R_3 - 2R_2 \to R_1 
\end{align*}
\item $\det(A) = -32$ and
\begin{align*}
\text{adj}(A) &=
\begin{bmatrix}
\begin{vmatrix}
2 & 3 \\
0 & 4
\end{vmatrix} &
-\begin{vmatrix}
1 & 3 \\
4 & 4
\end{vmatrix} &
\begin{vmatrix}
1 & 2 \\
4 & 0
\end{vmatrix} 
\\
-\begin{vmatrix}
2 & 9 \\
0 & 4
\end{vmatrix} &
\begin{vmatrix}
3 & 9 \\
4 & 4
\end{vmatrix} &
-\begin{vmatrix}
3 & 2 \\
4 & 0
\end{vmatrix}
\\
\begin{vmatrix}
2 & 9 \\
2 & 3
\end{vmatrix} &
-\begin{vmatrix}
3 & 9 \\
1 & 3
\end{vmatrix} &
\begin{vmatrix}
3 & 2 \\
1 & 2
\end{vmatrix} 
\end{bmatrix}^T \\
&=
\begin{bmatrix}
8 & 8 & -8 \\
-8 & -24 & 8 \\
-12 & 0 & 4
\end{bmatrix}^T \\
&=
\begin{bmatrix}
8 & -8 & -12 \\
8 & -24 & 0 \\
-8 & 8 & 4
\end{bmatrix}   
\end{align*}
Hence 
\begin{align*}
A^{-1} &= \frac{1}{\det(A)} \text{adj}(A) \\
&= -\frac{1}{32}
\begin{bmatrix}
8 & -8 & -12 \\
8 & -24 & 0 \\
-8 & 8 & 4
\end{bmatrix} \\
&= 
\begin{bmatrix}
-\frac{1}{4} & \frac{1}{4} & \frac{3}{8} \\
-\frac{1}{4} & \frac{3}{4} & 0 \\
\frac{1}{4} & -\frac{1}{4} & -\frac{1}{8}
\end{bmatrix}
\end{align*}
\end{enumerate}
\end{Answer}

\begin{Exercise}
Let
\begin{align*}
&A =
\begin{bmatrix}
0 & 2 & 5\\
0 & 4 & 9\\
1 & 2 & 1
\end{bmatrix}
&B =
\begin{bmatrix}
2 & 3 & 4\\
2 & 4 & 6\\
3 & 5 & 8
\end{bmatrix}  
\end{align*}
Verify:
\begin{enumerate}[label=(\alph*)]
\item $(AB)^T = B^TA^T$,
\item $(AB)^{-1} = B^{-1}A^{-1}$, and
\item $\det(AB) = \det(A)\det(B)$.
\end{enumerate} for this particular case. 
\end{Exercise}
\begin{Answer}
\begin{enumerate}[label=(\alph*)]
\item $\begin{bmatrix}
19 & 35 & 9 \\
33 & 61 & 16 \\
52 & 96 & 24
\end{bmatrix}
=
\begin{bmatrix}
2 & 2 & 3\\
3 & 4 & 5\\
4 & 6 & 8
\end{bmatrix} 
\begin{bmatrix}
0 & 0 & 1\\
2 & 4 & 2\\
5 & 9 & 1
\end{bmatrix}$
\item $
\begin{bmatrix}
18 & -10 & 1 \\
-6 & 3 & 1 \\
-\frac{11}{4} & \frac{7}{4} & - 1
\end{bmatrix}
=
\begin{bmatrix}
1 & -2 & 1 \\
1 & 2 & -2 \\
-1 & -\frac{1}{2} & 1 \\
\end{bmatrix}
\begin{bmatrix}
7 & -4 & 1 \\
-\frac{9}{2} & \frac{5}{2} & 0 \\
2 & -1 & 0 \\
\end{bmatrix}$
\item
$ \begin{vmatrix}
19 & 33 & 52 \\
35 & 61 & 96 \\
9 & 16 & 24
\end{vmatrix} = -4 = (-2)(2) =
\begin{vmatrix}
0 & 2 & 5\\
0 & 4 & 9\\
1 & 2 & 1
\end{vmatrix}
\begin{vmatrix}
2 & 3 & 4\\
2 & 4 & 6\\
3 & 5 & 8
\end{vmatrix}$
\end{enumerate}
\end{Answer}

\begin{Exercise}
Show that
\begin{align*}
A = 
\begin{bmatrix}
1 & 2 & 3 \\
3 & 0 & -1 \\
2 & 1 & 1 
\end{bmatrix}
\end{align*}
is singular.
\end{Exercise}
\begin{Answer}
Either by evaluating the determinant to show that $|A| = 0$, or find its reduced row echelon form which is
\begin{align*}
\begin{bmatrix}
1 & 0 & -\frac{1}{3} \\
0 & 1 & \frac{5}{3} \\
0 & 0 & 0
\end{bmatrix}
\end{align*}
and not equal to the identity.
\end{Answer}


\begin{Exercise}
Given
\begin{align*}
A =
\begin{bmatrix}
1 & 9 & 1 & 4\\
0 & 6 & 2 & 8\\
1 & 9 & 3 & 9\\
0 & 9 & 0 & 1
\end{bmatrix}  
\end{align*}
Find its determinant, inverse, and determinant of the inverse. 
\end{Exercise}
\begin{Answer}
\begin{align*}
\det(A) &= -42\\
\det(A^{-1}) &= -\frac{1}{42}\\
A^{-1} &= 
\def\arraystretch{1.25}
\begin{bmatrix}
\frac{9}{7} & -\frac{3}{14} & -\frac{2}{7} & -\frac{6}{7} \\
-\frac{1}{21} & -\frac{1}{21} & \frac{1}{21} & \frac{1}{7} \\
-\frac{11}{7} & -\frac{15}{14} & \frac{11}{7} & \frac{5}{7} \\
\frac{3}{7} & \frac{3}{7} & -\frac{3}{7} & -\frac{2}{7}
\end{bmatrix}
\end{align*}
\end{Answer}

\begin{Exercise}
For the following matrix,
\begin{align*}
A = 
\begin{bmatrix}
p & 1 & 2\\
0 & 2 & p\\
4 & -2 & 0
\end{bmatrix} 
\end{align*}
Find the values of $p$ such that $A$ is invertible.
\end{Exercise}
\begin{Answer}
By cofactor expansion along the first column, we can obtain the determinant of $A$ as
\begin{align*}
|A| = 2p^2 + 4p - 16
\end{align*}
which has two roots, $p = -4$ and $p = 2$ such that $|A| = 0$ and $A$ is not invertible. All values of $p$ other than $p = -4$ and $p = 2$ make $A$ invertible.
\end{Answer}

\begin{Exercise}
\label{ex:symskew}
Show that for any square matrix $A$, $A + A^T$ is symmetric, and $A - A^T$ is skew-symmetric. Hence show that any square matrix $A$ can be written as the sum of a symmetric matrix and a skew-symmetric matrix with an explicit formula.
\end{Exercise}
\begin{Answer}
$(A + A^T)^T = A^T + (A^T)^T = A^T+A = A+A^T$,\\
and $(A - A^T)^T = A^T - (A^T)^T = A^T-A = -(A - A^T)$. We can split $A$ into
\begin{align*}
A &= A + \frac{1}{2}(A^T - A^T) \\
&= \frac{1}{2}A + \frac{1}{2}A + \frac{1}{2}A^T - \frac{1}{2}A^T \\
&= \frac{1}{2}A + \frac{1}{2}A^T + \frac{1}{2}A - \frac{1}{2}A^T \\
&= \frac{1}{2}(A + A^T) + \frac{1}{2}(A - A^T)
\end{align*}
where the first term is symmetric and the second term is skew-symmetric.
\end{Answer}

\begin{Exercise}
Prove that if $A$ is an invertible $n \times n$ matrix, $\abs{A} \neq 0$, then we have
\begin{align*}
\det(\text{adj}(A))=(\det(A))^{n-1}    
\end{align*}
using Properties \ref{proper:properdet} and \ref{proper:invadj}.
\end{Exercise}
\begin{Answer}
\begin{align*}
A^{-1} &= \frac{1}{\det(A)}\text{adj}(A) \\
\det(A^{-1}) &= \det(\frac{1}{\det(A)}\text{adj}(A)) & \text{(Notice that $\frac{1}{\det(A)}$ is now a scalar)}\\
\frac{1}{\det(A)} &= (\frac{1}{\det(A)})^n\det(\text{adj}(A)) \\
\det(\text{adj}(A)) &= (\det(A))^{n-1}  
\end{align*}
\end{Answer}
\chapter{Solutions for Linear Systems}
\label{chap:SolLinSys}

The last chapter has introduced the necessary machinery for solving linear systems and now we are going to see how to apply them under suitable circumstances. Remember, in the first chapter, we have formulated some problems about linear systems of equations appearing in the Earth System, and they will be solved accordingly.

\section{Number of Solutions for Linear Systems}

Before tackling any linear system, we may like to know there are how many solutions. In fact, there are only three possibilities.
\begin{thm}[Number of Solutions for a Linear System]
For a system of linear equations $A\vec{x} = \vec{h}$ (recall Definition \ref{defn:linsys} and Properties \ref{proper:linsysmat}), it has either:
\begin{enumerate}
\item No solution,
\item An unique solution, or
\item Infinitely many solutions.
\end{enumerate}
for the unknowns $\vec{x}$.
\end{thm}
This can be illustrated by considering a linear system with two equations and two unknowns, with each equation representing a line. There are three types of scenarios.
\begin{equation*}
\begin{cases}
a_1x + b_1y &= h_1 \\
a_2x + b_2y &= h_2
\end{cases}   
\end{equation*}
\begin{center}
\begin{tikzpicture}
\begin{axis}
[axis y line=middle,
axis x line=middle,
xlabel=$x$,ylabel=$y$,
enlargelimits=0.2,
xmin=-5,xmax=5,
ymin=-5,ymax=5,ticks=none]
\addplot[mark=none, color=blue] {x+1} node[below, yshift=-20]{$x-y=-1$};
\addplot[mark=none, color=red] {-2*x+3} node[above left, xshift=-10]{$2x+y=3$};
\addplot[only marks, color=Green] coordinates {(2/3, 5/3)} node[left, xshift=-10]{($\frac{2}{3}, \frac{5}{3}$)};
\end{axis}
\end{tikzpicture} \\
One Solution: Two non-parallel lines (red/blue) intersecting at one point (green).
\end{center}
\begin{center}
\begin{tikzpicture}
\begin{axis}
[axis y line=middle,
axis x line=middle,
xlabel=$x$,ylabel=$y$,
enlargelimits=0.2,
xmin=-5,xmax=5,
ymin=-5,ymax=5,ticks=none]
\addplot[mark=none, color=blue] {1/2*x+1} node[above, xshift=-10]{$x - 2y = -2$};
\addplot[mark=none, color=red] {1/2*x-2} node[below, yshift=-10]{$x - 2y = 4$};
\end{axis}
\end{tikzpicture} \\
No Solution: Two parallel lines never touch each other.
\end{center}
\begin{center}
\begin{tikzpicture}
\begin{axis}
[axis y line=middle,
axis x line=middle,
xlabel=$x$,ylabel=$y$,
enlargelimits=0.2,
xmin=-5,xmax=5,
ymin=-5,ymax=5,ticks=none]
\addplot[mark=none, color=Green] {-4*x-3};
\node[blue] at (-4,2) {$4x+y=-3$};
\node[red] at (3,-3) {$8x+2y=-6$};
\end{axis}
\end{tikzpicture} \\
Infinitely Many Solutions: Two parallel lines overlap each other. 
\end{center}
It goes similarly for any linear system of three unknowns in which equations represent planes instead. The readers can try to imagine and visualize the possibilities. (The intersection of two non-parallel planes will be a line.) In fact, this theorem about the existence of solutions is true for any number of variables and equations. Some readers may think if there can be finitely many solutions only. Unfortunately, it is impossible. Assume there are at least two distinct solutions $\vec{x}_1$, $\vec{x}_2$ to the system $A\vec{x} = \vec{h}$, then it is easy to show by construction all $\vec{x}_t = t\vec{x}_1 + (1-t)\vec{x}_2$ for any $t$ will be valid solutions which are infinitely many. \\ 
\\
Naturally, the next question is about how to find out which case the linear system belongs to. The following theorem reveals the relation between the number of solutions for a \textit{square} linear system and the determinant of its coefficient matrix.
\begin{thm}
\label{thm:sqlinsysunique}
For a square linear system $A\vec{x} = \vec{h}$, if the coefficient matrix $A$ is invertible, i.e. $\det(A) \neq 0$, there is always only one unique solution. However, if $A$ is singular, $\det(A) = 0$, then it has either no solution, or infinitely many solutions.
\end{thm}
As a consequence, if the homogeneous linear system $A\vec{x} = \textbf{0}$ is singular with $\det(A) = 0$, since it always has a trivial solution of $\vec{x} = \textbf{0}$, the above theorem implies that the homogeneous system must have infinitely many solutions (since it does not have no solution). We defer the proof of Theorem \ref{thm:sqlinsysunique}, as well as the discussion about non-square systems, until we start actually solving linear systems in the next subsection. \\
Short Exercise: By inspection, determine the number of solutions for the following linear systems.\footnote{These two homogeneous linear system has a determinant of $-1$ and $0$, and hence by Theorem \ref{thm:sqlinsysunique} the first system has a unique solution and the second one has infinitely many solutions.}
\begin{align*}
&
\begin{bmatrix}
2 & 1 & 6 \\
3 & 0 & 4 \\
1 & 1 & 5 \\
\end{bmatrix}
\begin{bmatrix}
x \\
y \\
z
\end{bmatrix}
=
\begin{bmatrix}
0 \\
0 \\
0
\end{bmatrix}
&
\begin{bmatrix}
1 & 4 & 3 \\
1 & 5 & 2 \\
1 & 3 & 4 \\
\end{bmatrix}
\begin{bmatrix}
x \\
y \\
z
\end{bmatrix}
=
\begin{bmatrix}
0 \\
0 \\
0
\end{bmatrix}
\end{align*}

\section{Solving Linear Systems}
Finally it is the time to get down to solving linear systems (preferably written in form of matrices), and we have two methods to choose.
\begin{enumerate}
\item By Gaussian Elimination, for linear system in any shape, or
\item By Inverse, which is apparently only applicable for square, invertible coefficient matrices.
\end{enumerate}

\subsection{Solving Linear Systems by Gaussian Elimination}
\label{subsection:SolLinSysGauss}

Like in Section \ref{subsection:invGauss}, applying Gaussian Elimination on the augmented matrix (introduced at the end of Section \ref{section:deflinsys}) of a linear system can yield the solution at right hand side. The principles involving elementary row operations are the same as stated in Theorems \ref{thm:elementarymat} and \ref{thm:Gausselimprincip}, but with $A\vec{x} = \vec{h}$ instead of $AA^{-1} = I$. In addition, the coefficient matrix $A$ can be non-square, but we will look at the easier case of a coefficient matrix $A$ first.

\subsubsection{Square Systems}
\begin{exmp}
Solve the following linear system by Gaussian Elimination.
\begin{align*}
\begin{bmatrix}
1 & 0 & 1 \\
1 & 1 & 4 \\
2 & 0 & 3
\end{bmatrix}
\begin{bmatrix}
x \\
y \\
z
\end{bmatrix}
=
\begin{bmatrix}
3 \\
10 \\
8
\end{bmatrix}
\end{align*}
\end{exmp}
\begin{solution}
We re-write the system in augmented form and apply Gaussian Elimination, aiming to reduce the matrix to the left into the identity.
\begin{align*}
\left[\begin{array}{@{}ccc|c@{}}
1 & 0 & 1 & 3 \\
1 & 1 & 4 & 10 \\
2 & 0 & 3 & 8
\end{array}\right] 
& \to 
\left[\begin{array}{@{}ccc|c@{}}
1 & 0 & 1 & 3 \\
0 & 1 & 3 & 7 \\
0 & 0 & 1 & 2
\end{array}\right] 
& R_2-R_1 \to R_2, R_3-2R_1 \to R_3 \\
& \to 
\left[\begin{array}{@{}ccc|c@{}}
1 & 0 & 0 & 1 \\
0 & 1 & 0 & 1 \\
0 & 0 & 1 & 2
\end{array}\right] 
& R_2-2R_3 \to R_2, R_1-R_3 \to R_1
\end{align*}
which translates to
\begin{align*}
\begin{cases}
x = 1 \\
y = 1 \\
z = 2
\end{cases}
& \text{or} 
& \vec{x} = 
\begin{bmatrix}
x \\
y \\
z
\end{bmatrix}
=
\begin{bmatrix}
1 \\
1 \\
2
\end{bmatrix}
\end{align*}
Note that we have successfully converted the coefficient matrix to the identity along the way, which by Theorem \ref{thm:equiv1} the coefficient matrix is invertible. This explains the first part of Theorem \ref{thm:sqlinsysunique} as in this case every unknown is associated only to a leading 1 in the corresponding column  and a unique solution can always be derived.
\end{solution}

\begin{exmp}
\label{ex:nosol}
Solve the linear system of
\begin{align*}
\begin{bmatrix}
3 & 7 & 2 \\
1 & 1 & 0 \\
0 & 2 & 1 
\end{bmatrix}
\begin{bmatrix}
x \\
y \\
z
\end{bmatrix}
=
\begin{bmatrix}
8 \\
2 \\
2
\end{bmatrix}   
\end{align*}
\end{exmp}
\begin{solution} Again, we apply Gaussian Elimination on the augmented matrix to obtain
\begin{align*}
\left[\begin{array}{@{}ccc|c@{}}
3 & 7 & 2 & 8 \\
1 & 1 & 0 & 2 \\
0 & 2 & 1 & 2
\end{array}\right] 
& \to 
\left[\begin{array}{@{}ccc|c@{}}
1 & 1 & 0 & 2 \\
3 & 7 & 2 & 8 \\
0 & 2 & 1 & 2
\end{array}\right] 
& R_1 \leftrightarrow R_2 \\
& \to 
\left[\begin{array}{@{}ccc|c@{}}
1 & 1 & 0 & 2 \\
0 & 4 & 2 & 2 \\
0 & 2 & 1 & 2
\end{array}\right] 
& R_2-3R_1 \to R_2 \\
& \to 
\left[\begin{array}{@{}ccc|c@{}}
1 & 1 & 0 & 2 \\
0 & 1 & \frac{1}{2} & \frac{1}{2} \\
0 & 2 & 1 & 2
\end{array}\right] 
& \frac{1}{4}R_2 \to R_2 \\
& \to 
\left[\begin{array}{@{}ccc|c@{}}
1 & 1 & 0 & 2 \\
0 & 1 & \frac{1}{2} & \frac{1}{2} \\
0 & 0 & 0 & 1
\end{array}\right] 
& R_3 - 2R_2 \to R_3
\end{align*}
The last row corresponds to $0 = 1$ which is contradictory. As a consequence, the system is inconsistent, no solution exists.
\end{solution}


\begin{exmp}
\label{ex:mulsol}
Find the solution for the following linear system.
\begin{align*}
\begin{bmatrix}
1 & 2 & 1 \\
2 & 5 & 3 \\
0 & 1 & 1 
\end{bmatrix}
\begin{bmatrix}
x \\
y \\
z
\end{bmatrix}
=
\begin{bmatrix}
1 \\
2 \\
0
\end{bmatrix}   
\end{align*}
\end{exmp}
\begin{solution} 
Gaussian Elimination leads to
\begin{align*}
\left[\begin{array}{@{}ccc|c@{}}
1 & 2 & 1 & 1 \\
2 & 5 & 3 & 2 \\
0 & 1 & 1 & 0
\end{array}\right] 
& \to 
\left[\begin{array}{@{}ccc|c@{}}
1 & 2 & 1 & 1 \\
0 & 1 & 1 & 0 \\
0 & 1 & 1 & 0
\end{array}\right] 
& R_2 - 2R_1 \to R_2 \\
& \to 
\left[\begin{array}{@{}ccc|c@{}}
1 & 0 & -1 & 1 \\
0 & 1 & 1 & 0 \\
0 & 0 & 0 & 0
\end{array}\right] 
& R_3-R_2 \to R_3, R_1-2R_2 \to R_1
\end{align*}
Now, the last row corresponds to $0 = 0$, implying one equation is spurious. This also means that it has a \index{Free Variable}\keywordhl{Free Variable}, which means that we can assign one unknown as a parameter for expressing other variables. We choose such unknowns according to the rule that they should not be fixed to a pivot in the reduced coefficient matrix. As the variables $x$ and $y$ already correspond to the two pivots in the first/second columns, we can only let $z = t$. From the first row and second row, we obtain $x = 1+t$, $y = -t$ respectively. Therefore,
\begin{align*}
\vec{x} = 
\begin{bmatrix}
x \\
y \\
z
\end{bmatrix}
=
\begin{bmatrix}
1+t \\
-t \\
t
\end{bmatrix}
=
\begin{bmatrix}
1 \\
0 \\
0
\end{bmatrix}
+ t
\begin{bmatrix}
1 \\
-1 \\
1
\end{bmatrix}
\end{align*}
where $-\infty < t < \infty$ is any scalar. The first column vector 
\begin{align*}
\begin{bmatrix}
1 \\
0 \\
0
\end{bmatrix}    
\end{align*}
is the so-called \index{Particular Solution}\keywordhl{Particular Solution}. When it is complemented by the second column vector which is multiplied by the free parameter $t$
\begin{align*}
t
\begin{bmatrix}
1 \\
-1 \\
1
\end{bmatrix}    
\end{align*}
they constitute the entire set of \index{General Solution}\keywordhl{General Solution}. 
\end{solution}
Short Exercise: Try plugging in any number $t$ to the general solution and verify the consistency.\footnote{Let's say $t=1$ and $\tilde{x} = 
\begin{bmatrix}
1 \\
0 \\
0
\end{bmatrix}
+ (1)
\begin{bmatrix}
1 \\
-1 \\
1
\end{bmatrix}
=
\begin{bmatrix}
2 \\
-1 \\
1
\end{bmatrix}$, then clearly $A\tilde{x} = 
\begin{bmatrix}
1 & 2 & 1 \\
2 & 5 & 3 \\
0 & 1 & 1 
\end{bmatrix}
\begin{bmatrix}
2 \\
-1 \\
1
\end{bmatrix}
= 
\begin{bmatrix}
1 \\
2 \\
0
\end{bmatrix}$. It can become a new particular solution by noting that the original solution can be rewritten as
\begin{align*}
\vec{x} =
\begin{bmatrix}
1 \\
0 \\
0
\end{bmatrix}
+ t
\begin{bmatrix}
1 \\
-1 \\
1
\end{bmatrix}
= 
\begin{bmatrix}
1 \\
0 \\
0
\end{bmatrix}
+
\begin{bmatrix}
1 \\
-1 \\
1
\end{bmatrix}
+
(t-1)
\begin{bmatrix}
1 \\
-1 \\
1
\end{bmatrix}
=
\begin{bmatrix}
2 \\
-1 \\
1
\end{bmatrix}
+
t'
\begin{bmatrix}
1 \\
-1 \\
1
\end{bmatrix}
= \tilde{x} + t'
\begin{bmatrix}
1 \\
-1 \\
1
\end{bmatrix}
\end{align*}
where we "extract" $\tilde{x}$ from generating a shifted free parameter $t' = t-1$ and according to this relation, it represents the same set of general solution as the original expression.}\par
The general solution encompasses all possible solutions to the linear system. For broader situations, it can contain more than one pairs of free parameter and column vector (or none, for the rather trivial cases of zero or a unique solution). The amount of free variables can be seen to be the number of columns in the coefficient matrix, minus the number of pivots in the reduced row echelon form. In case of multiple free variables, we assign the corresponding amount of free parameters to the non-pivots and apply the same procedure to get a set of general solution. \\
\\
Meanwhile, the particular solution can be set to any valid solution to the system (the choice does not affect the structure of any column vector that comes along with a free parameter, see the footnote to the short exercise above). If the linear system is homogeneous, then the zero vector will always be a possible particular solution. \\
\\
We have seen in the previous two examples that if the reduced row echelon form of the square coefficient matrix has some row of full zeros, then it either leads to no solution (if inconsistent) or infinitely many solutions (if consistent). Since such a matrix at the same time has a determinant of zero (by Properties \ref{proper:zerodet}) and is singular. This establishes the second part of Theorem \ref{thm:sqlinsysunique}. \\
\\
For non-square coefficient matrices, two cases occur.
\begin{enumerate}
\item There are more equations (rows) than unknowns (columns). The system is \index{Overdetermined}\keywordhl{Overdetermined}. The reduced row echelon form then must have at least one row of full zeros. If any one of them is inconsistent, then contradiction will arise just like in Example \ref{ex:nosol} and there will be no solution. However, if all zero rows are consistent (i.e. $0=0$), then there still can be a unique solution or infintely many of them.
\item There are fewer equations (rows) than unknowns (columns). The system is said to be \index{Underdetermined}\keywordhl{Underdetermined}. There must be unknowns that are non-pivots in the reduced row echelon form of the coefficient matrix. Hence free variables, and infinitely many solutions ensue if there is no \textit{inconsistent} row of full zeros (then there is no solution). The calculation is similar to that in Example \ref{ex:mulsol}.
\end{enumerate}
Let's see some examples for non-square linear systems.\
\subsubsection{Overdetermined Systems}
\begin{exmp}
Find the solution to the following overdetermined system, if any.
\begin{align*}
\begin{bmatrix}
1 & 4 & 0 \\
2 & 2 & 3 \\
1 & 1 & 2 \\
0 & 3 & 1 
\end{bmatrix}
\begin{bmatrix}
x \\
y \\
z 
\end{bmatrix}
=
\begin{bmatrix}
4 \\
8 \\
3 \\
5
\end{bmatrix}   
\end{align*}
\end{exmp}
\begin{solution}
\begin{align*}
\left[\begin{array}{@{}ccc|c@{}}
1 & 4 & 0 & 4\\
2 & 2 & 3 & 8\\
1 & 1 & 2 & 3\\
0 & 3 & 1 & 5\\
\end{array}\right] 
& \to 
\left[\begin{array}{@{}ccc|c@{}}
1 & 4 & 0 & 4\\
0 & -6 & 3 & 0\\
0 & -3 & 2 & -1\\
0 & 3 & 1 & 5\\
\end{array}\right] 
& R_2 - 2R_1 \to R_2, R_3 - R_1 \to R_3 \\ 
& \to
\left[\begin{array}{@{}ccc|c@{}}
1 & 4 & 0 & 4\\
0 & 1 & -\frac{1}{2} & 0\\
0 & -3 & 2 & -1\\
0 & 3 & 1 & 5\\
\end{array}\right] 
& -\frac{1}{6}R_2 \to R_2  \\    
& \to
\left[\begin{array}{@{}ccc|c@{}}
1 & 4 & 0 & 4\\
0 & 1 & -\frac{1}{2} & 0\\
0 & 0 & \frac{1}{2} & -1\\
0 & 0 & \frac{5}{2} & 5\\
\end{array}\right] 
& R_3 + 3R_2 \to R_3, R_4 - 3R_2 \to R_4  \\   
& \to
\left[\begin{array}{@{}ccc|c@{}}
1 & 4 & 0 & 4\\
0 & 1 & -\frac{1}{2} & 0\\
0 & 0 & 1 & -2\\
0 & 0 & \frac{5}{2} & 5\\
\end{array}\right] 
& 2R_3 \to R_3 \\ 
& \to
\left[\begin{array}{@{}ccc|c@{}}
1 & 4 & 0 & 4\\
0 & 1 & -\frac{1}{2} & 0\\
0 & 0 & 1 & -2\\
0 & 0 & 0 & 10\\
\end{array}\right] 
& R_4 - \frac{5}{2}R_3 \to R_4 \\ 
\end{align*}
The last row is inconsistent and hence the overdetermined system has no solution.
\end{solution}

\begin{exmp}
Show that there are infinitely many solution to the following overdetermined system.
\begin{align*}
\begin{bmatrix}
1 & 1 & 2 \\
1 & 2 & 5 \\
2 & 1 & 1 \\
1 & 0 & -1
\end{bmatrix}
\begin{bmatrix}
x \\
y \\ 
z
\end{bmatrix}
=
\begin{bmatrix}
2 \\
3 \\
3 \\
1
\end{bmatrix}   
\end{align*}
\end{exmp}
\begin{solution}
\begin{align*}
\left[\begin{array}{@{}ccc|c@{}}
1 & 1 & 2 & 2 \\
1 & 2 & 5 & 3 \\
2 & 1 & 1 & 3 \\
1 & 0 & -1 & 1
\end{array}\right] 
& \to 
\left[\begin{array}{@{}ccc|c@{}}
1 & 1 & 2 & 2 \\
0 & 1 & 3 & 1 \\
0 & -1 & -3 & -1\\
0 & -1 & -3 & -1
\end{array}\right] 
& \begin{aligned}
R_2 - R_1 \to R_2, R_3 - 2R_1 \to R_3 \\
R_4 - R_1 \to R_4
\end{aligned}\\ 
& \to 
\left[\begin{array}{@{}ccc|c@{}}
1 & 1 & 2 & 2 \\
0 & 1 & 3 & 1 \\
0 & 0 & 0 & 0 \\
0 & 0 & 0 & 0
\end{array}\right] 
& R_3 + R_2 \to R_3, R_4 + R_2 \to R_4 \\
& \to 
\left[\begin{array}{@{}ccc|c@{}}
1 & 0 & -1 & 1 \\
0 & 1 & 3 & 1 \\
0 & 0 & 0 & 0 \\
0 & 0 & 0 & 0
\end{array}\right] 
& R_1- R_2 \to R_1 
\end{align*}
Two out of the four equations are redundant and there are effectively two constraints only, over the three variables. We can let the non-pivot unknown $z = t$ be a free variable like in Example \ref{ex:mulsol}, and derive $x = 1+t$, $y = 1-3t$ from the first two rows. Thus the general solution is
\begin{align*}
\vec{x} = 
\begin{bmatrix}
x \\
y \\
z
\end{bmatrix}
=
\begin{bmatrix}
1+t \\
1-3t \\
t
\end{bmatrix}
=
\begin{bmatrix}
1 \\
1 \\
0
\end{bmatrix}
+ t
\begin{bmatrix}
1 \\
-3 \\
1
\end{bmatrix}
\end{align*}
where $\begin{bmatrix}
1 \\
1 \\
0    
\end{bmatrix}$
is a particular solution.
\end{solution}

\subsubsection{Underdetermined Systems}
\begin{exmp}
Solve the following underdetermined system.
\begin{align*}
\begin{bmatrix}
1 & 1 & 2 & 1 \\
2 & 1 & 3 & 2 \\
0 & 1 & 1 & 2 
\end{bmatrix}
\begin{bmatrix}
x \\
y \\
u \\
v
\end{bmatrix}
=
\begin{bmatrix}
0 \\
1 \\
1
\end{bmatrix}
\end{align*}
\end{exmp}
\begin{solution}
\begin{align*}
\left[\begin{array}{@{}cccc|c@{}}
1 & 1 & 2 & 1 & 0\\
2 & 1 & 3 & 2 & 1\\
0 & 1 & 1 & 2 & 1
\end{array}\right] 
& \to 
\left[\begin{array}{@{}cccc|c@{}}
1 & 1 & 2 & 1 & 0\\
0 & -1 & -1 & 0 & 1\\
0 & 1 & 1 & 2 & 1
\end{array}\right] 
& R_2 - 2R_1 \to R_2 \\
& \to 
\left[\begin{array}{@{}cccc|c@{}}
1 & 1 & 2 & 1 & 0\\
0 & 1 & 1 & 2 & 1\\
0 & -1 & -1 & 0 & 1
\end{array}\right] 
& R_2 \leftrightarrow R_3 \\
& \to 
\left[\begin{array}{@{}cccc|c@{}}
1 & 1 & 2 & 1 & 0\\
0 & 1 & 1 & 2 & 1\\
0 & 0 & 0 & 2 & 2
\end{array}\right] 
& R_3+R_2 \to R_3 \\
& \to 
\left[\begin{array}{@{}cccc|c@{}}
1 & 1 & 2 & 1 & 0\\
0 & 1 & 1 & 2 & 1\\
0 & 0 & 0 & 1 & 1
\end{array}\right] 
& \frac{1}{2}R_3 \to R_3 \\
& \to 
\left[\begin{array}{@{}cccc|c@{}}
1 & 1 & 2 & 0 & -1\\
0 & 1 & 1 & 0 & -1\\
0 & 0 & 0 & 1 & 1
\end{array}\right] 
& R_2-2R_3 \to R_2, R_1-R_3 \to R_1 \\
& \to 
\left[\begin{array}{@{}cccc|c@{}}
1 & 0 & 1 & 0 & 0\\
0 & 1 & 1 & 0 & -1\\
0 & 0 & 0 & 1 & 1
\end{array}\right] 
& R_1-R_2 \to R_1
\end{align*} 
From the third row, we have $v = 1$ immediately. The only unknown that is not associated to a pivot is $u$ and we can let $u = t$ be a free variable. From the first two equations, we retrieve $y = -1-t$ and $x = -t$, and therefore the general solution is
\begin{align*}
\vec{x} = 
\begin{bmatrix}
x \\
y \\
u \\
v
\end{bmatrix} 
=
\begin{bmatrix}
-t \\
-1-t \\
t \\
1
\end{bmatrix}
=
\begin{bmatrix}
0 \\
-1 \\
0 \\
1
\end{bmatrix}
+ t
\begin{bmatrix}
-1 \\
-1 \\
1 \\
0
\end{bmatrix}
\end{align*}
with 
$\begin{bmatrix}
0 \\
-1 \\
0 \\
1    
\end{bmatrix}$ as a particular solution.
\end{solution}

\subsection{Solving Linear Systems by Inverse}
\label{subsection:SolLinSysInv}
For a square linear system $A\vec{x} = \vec{h}$, if $A$ has a non-zero determinant and is invertible, then we can utilize its inverse to recover the solution. Remember that multiplying the inverse to a matrix returns an identity matrix, it is possible to multiply the inverse $A^{-1}$ to the left on both sides of the equation $A\vec{x} = \vec{h}$ to cancel out the $A$ at the L.H.S., which leads to
\begin{align*}
A^{-1}A\vec{x}= (A^{-1}A)\vec{x} &= A^{-1}\vec{h} \\
\vec{x} = I\vec{x} &= A^{-1}\vec{h} &\text{(Definition \ref{defn:inverse} and Properties \ref{proper:identity})}
\end{align*}
This solution is unique, guaranteed by Theorem \ref{thm:sqlinsysunique}.
\begin{exmp}
Given a linear system $A\vec{x} = \vec{h}$
\begin{align*}
\begin{bmatrix}
1 & -1 & -2 \\
0 & 3 & 1 \\
1 & 0 & -1
\end{bmatrix}
\begin{bmatrix}
x \\
y \\
z
\end{bmatrix}
=
\begin{bmatrix}
3 \\
2 \\
3
\end{bmatrix}
\end{align*}
It can be checked that the inverse of the coefficient matrix is
\begin{align*}
\begin{bmatrix}
1 & -1 & -2 \\
0 & 3 & 1 \\
1 & 0 & -1
\end{bmatrix}^{-1}   
=
\begin{bmatrix}
-\frac{3}{2} & -\frac{1}{2} & \frac{5}{2} \\
\frac{1}{2} & \frac{1}{2} & -\frac{1}{2} \\
-\frac{3}{2} & -\frac{1}{2} & \frac{3}{2}
\end{bmatrix}
\end{align*}
The readers are encouraged to verify the inverse. Subsequently, we have the solution to the linear system as $\vec{x} = A^{-1}\vec{h}$
\begin{align*}
\vec{x} = 
\begin{bmatrix}
x \\
y \\
z
\end{bmatrix}
=    
\begin{bmatrix}
-\frac{3}{2} & -\frac{1}{2} & \frac{5}{2} \\
\frac{1}{2} & \frac{1}{2} & -\frac{1}{2} \\
-\frac{3}{2} & -\frac{1}{2} & \frac{3}{2}
\end{bmatrix}
\begin{bmatrix}
3 \\
2 \\
3
\end{bmatrix}
=
\begin{bmatrix}
2 \\
1 \\
-1
\end{bmatrix}
\end{align*}
\end{exmp}
Doing Gaussian Elimination to find the inverse and then compute the solution by $\vec{x} = A^{-1}\vec{h}$ in Section \ref{subsection:SolLinSysInv} is somehow the same as using Gaussian Elimination directly to solve the linear system suggested by Section \ref{subsection:SolLinSysGauss}. Hypothetically, if there are a large amount of linear systems which all share the same coefficient matrix $A$, but different $\vec{h}_k$ to be solved, then the former approach maybe more efficient. However, in computer, calculation of inverse can be unstable (see Section \ref{section:ch2python}). Besides, Theorem \ref{thm:equiv1} can be extended as below by incorporating Theorem \ref{thm:sqlinsysunique}:
\begin{thm}
\label{thm:equiv2}[Equivalence Statement, ver.\ 2]
For a square matrix $A$, the followings are equivalent:
\begin{enumerate}[label=(\alph*)]
\item $A$ is invertible, i.e. $A^{-1}$ exists,
\item $\det(A) \neq 0$,
\item The reduced row echelon form of $A$ is $I$,
\item The linear system $A\vec{x} = \vec{h}$ has a unique solution, particularly $A\vec{x} = \textbf{0}$ has only the trivial solution $\vec{x} = \textbf{0}$.
\end{enumerate}
\end{thm}

\section{Earth Science Applications}
\label{sec:ch3earth}
Now we are going to revisit and find the solutions to the two linear system problems in Section \ref{sec:ch1earth}.
\begin{exmp}
Solve for the horizontal displacement $x$ and depth of top layer $y$ in the seismic ray problem of Example \ref{exmp:seismic1}.
\end{exmp}
\begin{solution}
The linear system is
\begin{align*}
\begin{bmatrix}
1 & 1 \\
1 & \sqrt{3}
\end{bmatrix}
\begin{bmatrix}
x \\
y
\end{bmatrix}
=
\begin{bmatrix}
1200 \\
800\sqrt{3}
\end{bmatrix}
\end{align*}
Since it is just a $2 \times 2$ coefficient matrix, we can directly use the expression in Example \ref{ex:2x2} to find its inverse, which is
\begin{align*}
\frac{1}{\sqrt{3}-1}
\begin{bmatrix}
\sqrt{3} & -1 \\
-1 & 1
\end{bmatrix}
=
\frac{1+\sqrt{3}}{2}
\begin{bmatrix}
\sqrt{3} & -1 \\
-1 & 1
\end{bmatrix}
\end{align*}
and solve the system by multiplying the inverse following the method demonstrated in Section \ref{subsection:SolLinSysInv}, leading to
\begin{align*}
\begin{bmatrix}
x \\
y
\end{bmatrix}
=
\frac{1+\sqrt{3}}{2}
\begin{bmatrix}
\sqrt{3} & -1 \\
-1 & 1
\end{bmatrix}
\begin{bmatrix}
1200 \\
800\sqrt{3}
\end{bmatrix}
=
\begin{bmatrix}
600+200\sqrt{3}\\
600-200\sqrt{3}
\end{bmatrix}
\end{align*}
Therefore the required horizontal displacement and depth of top layer are about $\SI{946.4}{\m}$ and $\SI{253.6}{\m}$ respectively.
\end{solution}

\begin{exmp}
Find the radiative loss $R_j$ and hence temperature $T_j$ in each layer of the multi-layer model in Example \ref{exmp:multilayer1}. In particular, what is the temperature at the surface ($j = N+1$)?
\end{exmp}
\begin{solution}
The linear system is
\begin{align*}
\begin{bmatrix}
-2 & 1 & 0 & \cdots & 0 & 0 & 0 \\
1 & -2 & 1 & & 0 & 0 & 0 \\
0 & 1 & -2 & & 0 & 0 & 0 \\
\vdots & & & \ddots & & & \vdots \\
0 & 0 & 0 & & -2 & 1 & 0 \\
0 & 0 & 0 & & 1 & -2 & 1 \\
0 & 0 & 0 & \cdots & 0 & 1 & -1
\end{bmatrix}
\begin{bmatrix}
R_1 \\
R_2 \\
R_3 \\
\vdots \\
R_{N-1} \\
R_N \\
R_{N+1}
\end{bmatrix}
=
\begin{bmatrix}
0 \\
0 \\
0 \\
\vdots \\
0 \\
0 \\
-R_{in}
\end{bmatrix}
\end{align*}
where $N$ is any positive integer. Since $N$ can be arbitrarily large, we may wish to avoid the direct computation of a massive inverse. Instead, we resort to a tactful way of row reduction to reveal the pattern of $R_j$. Rather than starting the reduction at the top as usual, we build up at the bottom, subtracting the lower row from the row directly above it and then moving up a row, repeated until we reach the top. 
\begin{align*}
& \left[\begin{array}{@{}ccccccc|c@{}}
-2 & 1 & 0 & \cdots & 0 & 0 & 0 & 0\\
1 & -2 & 1 & & 0 & 0 & 0 & 0\\
0 & 1 & -2 & & 0 & 0 & 0 & 0\\
\vdots & & & \ddots & & & \vdots & \vdots\\
0 & 0 & 0 & & -2 & 1 & 0 & 0\\
0 & 0 & 0 & & 1 & -2 & 1 & 0\\
0 & 0 & 0 & \cdots & 0 & 1 & -1 & -R_{in}
\end{array}\right] \\
\to &
\left[\begin{array}{@{}ccccccc|c@{}}
-2 & 1 & 0 & \cdots & 0 & 0 & 0 & 0\\
1 & -2 & 1 & & 0 & 0 & 0 & 0\\
0 & 1 & -2 & & 0 & 0 & 0 & 0\\
\vdots & & & \ddots & & & \vdots & \vdots\\
0 & 0 & 0 & & -2 & 1 & 0 & 0\\
0 & 0 & 0 & & 1 & -1 & 0 & -R_{in}\\
0 & 0 & 0 & \cdots & 0 & 1 & -1 & -R_{in}
\end{array}\right] & R_N + R_{N+1} \to R_N \\
\to &
\left[\begin{array}{@{}ccccccc|c@{}}
-2 & 1 & 0 & \cdots & 0 & 0 & 0 & 0\\
1 & -2 & 1 & & 0 & 0 & 0 & 0\\
0 & 1 & -2 & & 0 & 0 & 0 & 0\\
\vdots & & & \ddots & & & \vdots & \vdots\\
0 & 0 & 0 & & -1 & 0 & 0 & -R_{in}\\
0 & 0 & 0 & & 1 & -1 & 0 & -R_{in}\\
0 & 0 & 0 & \cdots & 0 & 1 & -1 & -R_{in}
\end{array}\right] & R_{N-1} + R_{N} \to R_{N-1} \\
\to & \quad\vdots & \text{(Keep moving up)} \\
\to &
\left[\begin{array}{@{}ccccccc|c@{}}
-2 & 1 & 0 & \cdots & 0 & 0 & 0 & 0\\
1 & -2 & 1 & & 0 & 0 & 0 & 0\\
0 & 1 & -1 & & 0 & 0 & 0 & -R_{in}\\
\vdots & & & \ddots & & & \vdots & \vdots\\
0 & 0 & 0 & & -1 & 0 & 0 & -R_{in}\\
0 & 0 & 0 & & 1 & -1 & 0 & -R_{in}\\
0 & 0 & 0 & \cdots & 0 & 1 & -1 & -R_{in}
\end{array}\right] \\
\to &
\left[\begin{array}{@{}ccccccc|c@{}}
-2 & 1 & 0 & \cdots & 0 & 0 & 0 & 0\\
1 & -1 & 0 & & 0 & 0 & 0 & -R_{in}\\
0 & 1 & -1 & & 0 & 0 & 0 & -R_{in}\\
\vdots & & & \ddots & & & \vdots & \vdots\\
0 & 0 & 0 & & -1 & 0 & 0 & -R_{in}\\
0 & 0 & 0 & & 1 & -1 & 0 & -R_{in}\\
0 & 0 & 0 & \cdots & 0 & 1 & -1 & -R_{in}
\end{array}\right] & R_2 + R_3 \to R_2 \\
\to &
\left[\begin{array}{@{}ccccccc|c@{}}
-1 & 0 & 0 & \cdots & 0 & 0 & 0 & -R_{in}\\
1 & -1 & 0 & & 0 & 0 & 0 & -R_{in}\\
0 & 1 & -1 & & 0 & 0 & 0 & -R_{in}\\
\vdots & & & \ddots & & & \vdots & \vdots\\
0 & 0 & 0 & & -1 & 0 & 0 & -R_{in}\\
0 & 0 & 0 & & 1 & -1 & 0 & -R_{in}\\
0 & 0 & 0 & \cdots & 0 & 1 & -1 & -R_{in}
\end{array}\right] & R_1 + R_2 \to R_1
\end{align*}
\end{solution}

\section{Python Programming}

\section{Exercises}

\begin{Exercise}
Solve the following linear system.
\begin{align*}
\begin{cases}
5x + y + 3z &= 6\\
2x - y + z &= \frac{7}{2}\\
3x + 2y - 4z &= -\frac{13}{2}
\end{cases}
\end{align*}
\end{Exercise}

\begin{Exercise}
Solve $A\vec{x} = \vec{h}_k$, where
\begin{align*}
&A =
\begin{bmatrix}
6 & 7 & 7\\
1 & 0 & 2\\
2 & 1 & 1
\end{bmatrix}
&\vec{x} =
\begin{bmatrix}
x\\
y\\
z
\end{bmatrix} \\
& \vec{h}_1 =
\begin{bmatrix}
-1 \\
5 \\
1
\end{bmatrix}
& \vec{h}_2 =
\begin{bmatrix}
19/4 \\
1 \\
5/4
\end{bmatrix}
\end{align*}
\end{Exercise}

\begin{Exercise}
Derive the solution to the following linear system.
\begin{align*}
\begin{cases}
3x + 4z &= 2\\
x + y + 2z &= -1\\
x - 2y &= 0
\end{cases}
\end{align*}
\end{Exercise}

\begin{Exercise}
Solve the following linear system.
\begin{align*}
\begin{cases}
m + n - p - 3q &= 2\\
m - q &= 5\\
3m + 2n - 2p - 7q &= 9
\end{cases}
\end{align*}
\end{Exercise}

\begin{Exercise}
For the following linear system,
\begin{align*}
\begin{bmatrix}
1 & 0 & \alpha \\
0 & \alpha & 0 \\
\alpha & 0 & 1
\end{bmatrix}
\begin{bmatrix}
x \\
y \\
z
\end{bmatrix}
=
\begin{bmatrix}
\alpha \\
0 \\
\alpha
\end{bmatrix}   
\end{align*}
Find the values of $\alpha$ so that the system has no solution, or infinitely many solutions.
\end{Exercise}

\begin{Exercise}
In a geology field trip, an outcrop is examined. It is observed that the rock mainly consists of crystals of three distinct colors (gray/pink/black). Assume that crystal of each color corresponds to exactly one type of mineral. Three samples are gathered, have their densities measured and percentage volumes of the three types of crystal analyzed. The data are as follows:
\begin{center}
\begin{tabular}{|c|c|c|c|c|}
\hline
 & gray & pink & black & density (g/cm$^3$) \\
\hline
Sample A & 40\% & 50\% & 10\% & 2.645\\
\hline
Sample B & 55\% & 40\% & 5\% & 2.6325\\
\hline
Sample C & 45\% & 45\% & 10\% & 2.65\\
\hline
\end{tabular}
\end{center}
From the data, infer the densities of the constituent minerals.
\end{Exercise}

\begin{Exercise}
Ohm's law relates voltage change to current and resistance by $V=IR$. In addition, Kirchhoff’s Second Law states that: The voltage gain balances the voltage drop around any closed loop. The clockwise convention is adopted, i.e. around a loop, a battery with its positive terminal pointing in clockwise direction is considered a voltage gain, and clockwise current passing through a resistor is deemed as a voltage drop. Together with the knowledge that current at a junction must conserve, as stated by Kirchhoff's First Law, find $I_1$, $I_2$, $I_3$ (assumed flowing in the direction as indicated in the figure) for the following circuit.
\begin{center}
\fbox{\includegraphics[scale = 0.4]{circuit.jpg}}
\end{center}
\end{Exercise}
You will obtain two equations by considering any two loops with Kirchhoff’s Second Law, and one from Kirchhoff's First Law. So, there are three equations, for the three unknown currents.

\begin{Exercise}
Dispersion Relation. (In Construction)
\end{Exercise}
\chapter{Introduction to Vectors}

After three chapters of discussion about matrices, it is the time to talk about another closely related concept in linear algebra, that is, vectors. While \textit{vectors} and \textit{vector spaces} have strictly mathematical definitions which make them abstract, we will take a more physical point of view with the special case of (finite-dimensional) geometric vectors first.

\section{Definition and Operations of Geometric Vectors}

\subsection{Basic Structure of Vectors in the Real $n$-space $\mathbb{R}^n$}
A \index{Vector}\index{Vector!Geometric Vector}\keywordhl{(Geometric) Vector} is a physical quantity represented by an ordered tuple of components (numbers), e.g. $(1, 8, 7, 4)$, $(1-\imath, 1+3\imath, 2)$. It has a \textit{magnitude (length)} and \textit{direction}, resembling an arrow. Some real-life examples are: two-dimensional flow velocity $(u, v)$, relative position of an airplane to a ground radar $(x, y, z)$.
\begin{defn}[$n$-dimensional Vector]
\label{defn:geometvec}
A $n$-dimensional vector consists of $n$ ordered elements called \index{Components}\keywordhl{Components} and are denoted by either an arrow or boldface, like $\vec{v}$ or $\textbf{v}$. It is expressed as
\begin{align*}
\vec{v} &=
\begin{bmatrix}
v_1 \\
v_2 \\
v_3 \\
\vdots \\
v_n
\end{bmatrix}
=
(v_1, v_2, v_3, \ldots, v_n)^T
\end{align*}
\end{defn}
A $n$-dimensional vector can be regarded to be an $n \times 1$ (\index{Vector!Column Vector}\keywordhl{Column Vector}) or $1 \times n$ matrix (\index{Vector!Row Vector}\keywordhl{Row Vector}) and vice versa, depending on the situation. Usually the form of a column vector is more commonly taken than row vector and the column form is assumed throughout the book if it is not further specified.\par
\begin{tikzpicture}
\draw[->] (-2.5,0)--(2.5,0) node[right]{$x$};
\draw[->] (0,-2.5)--(0,2.5) node[above]{$y$};
\draw[blue,->] (0,0)--(2,1) node[anchor=south]{$\vec{v} = (2,1)^T$};
\draw[Gray,dashed] (2,1)--(2,0) node[below]{$x = 2$};
\draw[Gray,dashed] (2,1)--(0,1) node[left]{$y = 1$};
\node[below left]{$O$}; 
\end{tikzpicture}\\    
A 2D vector drawn in an x-y plane.\par
{\fbox{\includegraphics[scale = 0.5]{higos.jpg}}\\
Forecast for \textit{Typhoon Higos}. (taken from \href{http://www.hkww.org/weather/tcarchive.html}{Hong Kong Weather Watch}) Its horiztonal movement is a two-dimensional vector, even though the speed and direction are given instead of the velocities in $x$ and $y$-direction (they can be converted to each other).\par}

Implicit in the definition of $n$-dimensional vectors is the $n$-dimensional space they are residing in. Assume the components of those vectors are all real, then the set of all such vectors constitutes the \index{Real $n$-space}\keywordhl{Real $n$-space $\mathbb{R}^n$}.
\begin{defn}[The Real $n$-space $\mathbb{R}^n$]
\label{defn:real_nspace}
The Real $n$-space $\mathbb{R}^n$ is defined as the set of all possible vectors (or "points") $\vec{v} = (v_1, v_2, v_3, \ldots, v_n)^T$ as defined in Definition \ref{defn:geometvec}, where $v_i$ can take any \textit{real} value, for $i = 1,2,3,\ldots,n$. The elements in $\mathbb{R}^n$ are known as $n$-dimensional \textit{real} vectors.
\end{defn}
While we have not clearly defined what a vector space is, we note that $\mathbb{R}^n$ fulfills the requirements of being a vector space. The detailed discussion of this aspect will be deferred to Chapter \ref{chap:vec_space}. Meanwhile, the complex counterpart will be explored in Chapter ??.\\
\\
An $n$-dimensional real geometric vectors as in Definition \ref{defn:geometvec} and \ref{defn:real_nspace} can be written as the sum of $n$ \index{Standard Unit Vector}\keywordhl{Standard Unit Vectors} that have a magnitude of $1$, denoted by $\hat{e}_p$, oriented in the positive direction along the $p$-th coordinates axes, $p = 1,2,\ldots,n$. The coordinate axes are perpendicular (or more accurately orthogonal, introduced later in this chapter) to each other and such coordinate system is known as the \index{Cartesian Coordinate System}\keywordhl{Cartesian Coordinate System}. Particularly in the three-dimensional (real) space $\mathbb{R}^3$, $\hat{e}_1 = \hat{i} = (1,0,0)^T$, $\hat{e}_2 = \hat{j} = (0,1,0)^T$, $\hat{e}_3 = \hat{k} = (0,0,1)^T$ correspond to $x$, $y$, $z$ axes respectively. 
\begin{defn}[Standard Unit Vector]
\label{defn:standardunitvec}
A standard unit vector $\hat{e}_p$ consists of $1$ at the $p$-th entry and $0$ elsewhere. Mathematically, $[\hat{e}_p]_q = 1$ when $q=p$ and $[\hat{e}_p]_q = 0$ when $q\neq p$.
\end{defn}
Below is an example of a vector in 3D $x$-$y$-$z$ space ($\mathbb{R}^3$).
\begin{center}
\begin{tikzpicture}[x={(1.8cm, -0.4cm)}, y={(1.4cm, 1.2cm)}, z={(0cm, 2cm)}]
\draw [->] (0,0,0) -- (2,0,0) node [below right] {$x$};
\draw [->] (0,0,0) -- (0,2,0) node [above] {$y$};
\draw [->] (0,0,0) -- (0,0,2) node [left] {$z$};
\draw [->, thick, red] (0,0,0) -- (1,0,0) node [below left] {$\hat{i} = (1,0,0)^T$};
\draw [->, thick, red] (0,0,0) -- (0,1,0) node [above right, midway, sloped] {$\hat{j} = (0,1,0)^T$} ; 
\draw [->, thick, red] (0,0,0) -- (0,0,1) node [left] {$\hat{k} = (0,0,1)^T$};
\draw [Gray, dashed] (1.8,0.4,0) -- (1.8,0,0) node[right, midway]{$y=0.4$}; 
\draw [Gray, dashed] (1.8,0.4,0) -- (0,0.4,0) node[above, midway, sloped]{$x=1.8$}; 
\draw [Gray, dashed] (1.8,0.4,0) -- (1.8,0.4,1.1) node[midway, right]{$z=1.1$};
\draw [->, blue] (0,0,0) -- (1.8,0.4,1.1) node [right] {$\vec{v} = (1.8,0.4,1.1)^T$};
/\end{tikzpicture}
\begin{align*}
\vec{v} &= 
\begin{bmatrix}
1.8 \\
0.4 \\
1.1
\end{bmatrix}
= 1.8
\begin{bmatrix}
1 \\
0 \\
0
\end{bmatrix}
+ 0.4
\begin{bmatrix}
0 \\
1 \\
0
\end{bmatrix}
+ 1.1
\begin{bmatrix}
0 \\
0 \\
1
\end{bmatrix} 
= 1.8\hat{i} + 0.4\hat{j} + 1.1\hat{k} \\
&= (1.8,0.4,1.1)^T
\end{align*}
\end{center}
where we have written $\vec{v}$ in two forms, as a tuple and sum of the standard unit vectors $\hat{i}, \hat{j}, \hat{k}$.

\subsection{Fundamental Vector Operations}
\label{section:vectoraddmul}
\subsubsection{Addition and Subtraction}
Same as their matrix counterpart, addition and subtraction between vectors is element-wise. Again, they are only valid for vectors of the same dimension. For $\vec{w} = \vec{u} \pm \vec{v}$, we have $w_i = u_i \pm v_i$. If
\begin{align*}
&\vec{u} =
\begin{bmatrix}
1 \\
2
\end{bmatrix}
&
\vec{v} =
\begin{bmatrix}
2 \\
-1
\end{bmatrix}
\end{align*}
then
\begin{align*}
\vec{u} + \vec{v} =
\begin{bmatrix}
\textcolor{red}{1} \\
\textcolor{red}{2}
\end{bmatrix}
+
\begin{bmatrix}
\textcolor{blue}{2} \\
\textcolor{blue}{-1}
\end{bmatrix}
&= 
\begin{bmatrix}
\textcolor{Green}{3} \\
\textcolor{Green}{1}
\end{bmatrix}
\\
\vec{u} - \vec{v} =
\begin{bmatrix}
\textcolor{red}{1} \\
\textcolor{red}{2}
\end{bmatrix}
-
\begin{bmatrix}
\textcolor{blue}{2} \\
\textcolor{blue}{-1}
\end{bmatrix}
&= 
\begin{bmatrix}
\textcolor{Green}{-1} \\
\textcolor{Green}{3}
\end{bmatrix}
\end{align*}
\begin{center}
\begin{tikzpicture}[scale=0.8]
\draw[->] (-3,0)--(3,0) node[right]{$x$};
\draw[->] (0,-3)--(0,3) node[above]{$y$};
\draw[red,-stealth] (0,0)--(1,2) node[anchor=south west]{$\vec{u} = (1,2)^T$};
\draw[blue,-stealth] (1,2)--(3,1) node[anchor=south west]{$\vec{v} = (2,-1)^T$};
\draw[Green,-stealth] (0,0)--(3,1) node[anchor=north west]{$\vec{u} + \vec{v} = (3,1)^T$};
\node[below left]{$O$}; 
\end{tikzpicture}\\
Addition: The tail of the blue vector is placed to the head of the red vector, and the resultant green vector is from the origin to the head of blue vector.
\end{center}
\begin{center}
\begin{tikzpicture}[scale=0.8]
\draw[->] (-3,0)--(3,0) node[right]{$x$};
\draw[->] (0,-3)--(0,3) node[above]{$y$};
\draw[red,-stealth] (0,0)--(1,2) node[anchor=south west]{$\vec{u} = (1,2)^T$};
\draw[blue,-stealth] (1,2)--(-1,3) node[anchor=south east]{$-\vec{v} = (-2,1)^T$};
\draw[Green,-stealth] (0,0)--(-1,3) node[anchor=north east]{$\vec{u} - \vec{v} = (-1,3)^T$};
\node[below left]{$O$}; 
\end{tikzpicture}\\
Subtraction: Similar to addition but with the blue vector oriented in the opposite direction.
\end{center}

\subsubsection{Scalar Multiplication} 
Multiplying a scalar (number) to a vector means that all components are multiplied by that scalar.
\begin{align*}
2
\begin{bmatrix}
2 \\
0 \\
1 \\
9
\end{bmatrix}
=
\begin{bmatrix}
4 \\
0 \\
2 \\
18
\end{bmatrix}
\end{align*}
Looking back at vector subtraction, it can be viewed as addition with a factor of $-1$.
\begin{align*}
\begin{bmatrix}
7 \\
5 \\
9
\end{bmatrix}
-
\begin{bmatrix}
3 \\
6 \\
9
\end{bmatrix}
=
\begin{bmatrix}
7 \\
5 \\
9
\end{bmatrix}
+ (-1)
\begin{bmatrix}
3 \\
6 \\
9
\end{bmatrix}
=
\begin{bmatrix}
7 \\
5 \\
9
\end{bmatrix}
+
\begin{bmatrix}
-3 \\
-6 \\
-9
\end{bmatrix}
=
\begin{bmatrix}
4 \\
-1 \\
0
\end{bmatrix}
\end{align*}

\subsubsection{Length and Unit Vector} \index{Length}\index{Magnitude}\keywordhl{Length (Magnitude)}, or more formally \index{Euclidean Norm}\keywordhl{Euclidean Norm}, of a vector $\vec{v}$ is based on a generalized version of \index{Pythagoras’ Theorem}\keywordhl{Pythagoras’ Theorem}, and is evaluated to be the square root of the sum of squares of components.
\begin{defn}[Vector Length]
\label{defn:vectorlength}
Length, or magnitude of a $n$-dimensional \textit{real} vector $\vec{v}$, denoted by $\norm{\vec{v}}$, is given by
\begin{align*}
\norm{\vec{v}} &= \sqrt{v_1^2 + v_2^2 + v_3^2 + \cdots + v_n^2} \\
&= \sqrt{\sum_{k=1}^{n} v_k^2}
\end{align*}
\end{defn}
For instance, the length of a two-dimensional vector follows the usual Pythagoras' Theorem as below. \\
\begin{tikzpicture}
\draw[->] (-2.5,0)--(2.5,0) node[right]{$x$};
\draw[->] (0,-2.5)--(0,2.5) node[above]{$y$};
\draw[blue,-stealth] (0,0)--(2,1) node[anchor=south west]{$\norm{\vec{v}} = \sqrt{\Delta x^2 + \Delta y^2} = \sqrt{2^2+1^2} = \sqrt{5}$};
\draw[Gray,dashed] (2,1)--(2,0) node[below]{$x = 2$};
\draw[Gray,dashed] (2,1)--(0,1) node[left]{$y = 1$};
\node[below left]{$O$}; 
\end{tikzpicture}\\
Here is another example which is three-dimensional.
\begin{center}
\begin{tikzpicture}[x={(1.8cm, -0.6cm)}, y={(1.6cm, 1.0cm)}, z={(0cm, 2cm)}]
\draw [->] (-1,0,0) -- (2,0,0) node [right] {$x$};
\draw [->] (0,-1,0) -- (0,2,0) node [above] {$y$};
\draw [->] (0,0,0) -- (0,0,2) node [left] {$z$};
\node[below] (0,0,0){$O$};
\draw [Gray, dashed] (1.6,-0.2,0) -- (1.6,0,0) node[below, pos=-0.5, sloped]{$y=-1$}; 
\draw [Gray, dashed] (1.6,-0.2,0) -- (0,-0.2,0) node[below, midway, sloped]{$x=8$}; 
\draw [Gray, dashed] (1.6,-0.2,0) -- (1.6,-0.2,0.8) node[midway, right]{$z=4$};
\draw [->, blue] (0,0,0) -- (1.6,-0.2,0.8) node [right] {$\vec{w} = (8,-1,4)^T$};
\end{tikzpicture}
\begin{align*}
\vec{w} &= 
\begin{bmatrix}
8 \\
-1 \\
4
\end{bmatrix}
& \norm{\vec{w}}&=
\sqrt{8^2 + (-1)^2 + 4^2} = 9 
\end{align*}
\end{center}
We can create a \index{Vector!Unit Vector}\keywordhl{Unit Vector} from some vector $\vec{v}$ that has a length of $1$ and orients in the same direction as $\vec{v}$ is simply produced by dividing (normalizing) $\vec{v}$ by its distance $\norm{\vec{v}}$.
\begin{defn}[Unit Vector]
\label{defn:unitvec}
The unit vector corresponding to a non-zero vector $\vec{v}$ is denoted as $\hat{v}$ and is given by
\begin{align*}
\hat{v} &= \frac{1}{\norm{\vec{v}}}\vec{v}
\end{align*}
where $\norm{\vec{v}}$ is defined as in Definition \ref{defn:vectorlength}. 
\end{defn}
Note that despite vectors can carry physical units, unit vectors  are all \textit{dimensionless} when formulated in this way. \\
Short Exercise: Find a unit vector for $\vec{w} = (8, -1, 4)^T$ in the previous example, and verify that it has a length of $1$.\footnote{$\norm{\vec{w}} = 9$, $\hat{w} = \vec{w}/\norm{\vec{w}} = \frac{1}{9}(8,-1,4)^T = (\frac{8}{9}, -\frac{1}{9}, \frac{4}{9})^T$, $\norm{\hat{w}} = \sqrt{(\frac{8}{9})^2 + (-\frac{1}{9})^2 + (\frac{4}{9})^2} = 1$.}

\section{Special Vector Operations}
\label{vectorops}
Now we are going to introduce two special types of vector operations: \textit{dot product}, and \textit{cross product}. 

\subsection{Dot Product}
\index{Dot Product}\keywordhl{(Real) Dot Product} (or \index{Scalar Product}\keywordhl{Scalar Product}) is defined for two (real) vectors that have the same number of dimension. Its value is the sum of products of paired components between the two vectors. In other words, it can be regarded as the matrix product between a row vector ($1 \times m$ matrix) and a column vector ($m \times 1$ matrix). 
\begin{defn}[Dot Product (Real)]
\label{defn:dotreal}
The dot product between two $n$-dimensional \textit{real} vectors $\vec{u}$ and $\vec{v}$ in $\mathbb{R}^n$ are denoted as either $\vec{u} \cdot \vec{v}$, or by matrix notation $\textbf{u}^T\textbf{v}$. They are defined as
\begin{align*}
\vec{u} \cdot \vec{v} = \textbf{u}^T\textbf{v} &= u_1v_1 + u_2v_2 + u_3v_3 + \cdots + u_nv_n \\
&= \sum_{k=1}^{n} u_kv_k
\end{align*}
which is a scalar quantity.
\end{defn}
Conversely, it can be said that entries of a matrix product are vector dot products between the corresponding row and column. It is emphasized that we are restricting ourselves to real entries since complex vectors introduce extra complications. Then, for two \textit{real} matrices expressed in the form of combined row/column vectors,
\begin{align*}
A &= [\vec{u}^{(1)}|\vec{u}^{(2)}|\cdots|\vec{u}^{(m)}]^T 
& B &= [\vec{v}^{(1)}|\vec{v}^{(2)}|\cdots|\vec{v}^{(m)}]\\
&= 
\begin{bmatrix}
\vec{u}^{(1)}_1  & \vec{u}^{(1)}_2 & \cdots & \vec{u}^{(1)}_n \\
\vec{u}^{(2)}_1  & \vec{u}^{(2)}_2 & \cdots & \vec{u}^{(2)}_n \\
\vdots & \vdots & & \vdots \\
\vec{u}^{(m)}_1  & \vec{u}^{(m)}_2 & \cdots & \vec{u}^{(m)}_n
\end{bmatrix} 
& &= 
\begin{bmatrix}
\vec{v}^{(1)}_1  & \vec{v}^{(2)}_1 & \cdots & \vec{v}^{(m)}_1 \\
\vec{v}^{(1)}_2  & \vec{v}^{(2)}_2 & \cdots & \vec{v}^{(m)}_2 \\
\vdots & \vdots & & \vdots \\
\vec{v}^{(1)}_n  & \vec{v}^{(2)}_n & \cdots & \vec{v}^{(m)}_n \\
\end{bmatrix} 
\end{align*}
(notice that the expression of $A$ has a transpose) their matrix product $AB$ can be written as
\begin{align*}
AB =
\begin{bmatrix}
\vec{u}^{(1)} \cdot \vec{v}^{(1)} & \vec{u}^{(1)} \cdot \vec{v}^{(2)} & \cdots & \vec{u}^{(1)} \cdot \vec{v}^{(m)} \\
\vec{u}^{(2)} \cdot \vec{v}^{(1)} & \vec{u}^{(2)} \cdot \vec{v}^{(2)} & \cdots & \vec{u}^{(2)} \cdot \vec{v}^{(m)} \\
\vdots & \vdots & & \vdots \\
\vec{u}^{(m)} \cdot \vec{v}^{(1)} & \vec{u}^{(m)} \cdot \vec{v}^{(2)} & \cdots & \vec{u}^{(m)} \cdot \vec{v}^{(m)} \\
\end{bmatrix}
\end{align*}
In addition, it is easy to see that
\begin{proper}
\label{proper:lengthdot}
The length of a vector, as defined in Definition \ref{defn:vectorlength}, can be written using its dot product between itself as
\begin{align*}
\norm{\vec{v}} &= \sqrt{\vec{v} \cdot \vec{v}} & &\text{or} &
\norm{\vec{v}}^2 &= \vec{v} \cdot \vec{v}
\end{align*}
\end{proper}
Notice that $\vec{v} \cdot \vec{v} = v_1^2 + v_2^2 + v_3^2 + \cdots + v_n^2 \geq 0$. This quantity is always strictly greater than zero ($\vec{v} \cdot \vec{v} > 0$) unless $\vec{v} = \textbf{0}$ is the zero vector (then $\vec{v} \cdot \vec{v} = 0$), which makes sense physically given that it represents length.

\begin{exmp}
\label{exmp:dotproduct5d}
If $\vec{u} = (1, 2, 3, 4, 5)^T$ and $\vec{v} = (-1, 0, 1, 0, -1)^T$, find the dot product $\vec{u} \cdot \vec{v} = \textbf{u}^T\textbf{v}$.
\end{exmp}
\begin{solution}
\begin{align*}
\vec{u} \cdot \vec{v} &= (1)(-1) + (2)(0) + (3)(1) + (4)(0) + (5)(-1) = -3
\end{align*}
Alternatively,
\begin{align*}
\textbf{u}^T\textbf{v} &=
\begin{bmatrix}
1 & 2 & 3 & 4 & 5
\end{bmatrix}
\begin{bmatrix}
-1 \\
0 \\
1 \\
0 \\
-1
\end{bmatrix}
= -3
\end{align*}
\end{solution}

Here are some properties of dot product.
\begin{proper}
\label{proper:dotproper}
For three $n$-dimensional vectors $\vec{u}$, $\vec{v}$ and $\vec{w}$, the following establishes.
\begin{align*}
\vec{u} \cdot \vec{v} &= \vec{v} \cdot \vec{u} &\text{Commutative Property} \\
\vec{u} \cdot (\vec{v} \pm \vec{w}) &= \vec{u} \cdot \vec{v} \pm \vec{u} \cdot \vec{w} &\text{Distributive Property} \\
(\vec{u} \pm \vec{v}) \cdot \vec{w} &= \vec{u} \cdot \vec{w} \pm \vec{v} \cdot \vec{w} &\text{Distributive Property} \\
(a\vec{u}) \cdot (b\vec{v}) &= ab(\vec{u} \cdot \vec{v}) &\text{where $a$, $b$ are some constants}
\end{align*}
Additionally, if $A$ is an $n \times n$ square matrix, then
\begin{align*}
\vec{u} \cdot (A\vec{v}) &= \textbf{u}^T(A\textbf{v}) = (A^T\textbf{u})^T\textbf{v} = (A^T\vec{u}) \cdot \vec{v} \\
(A\vec{u}) \cdot \vec{v} &= (A\textbf{u})^T\textbf{v} = \textbf{u}^T(A^T\textbf{v}) = \vec{u} \cdot (A^T\vec{v})
\end{align*}
where we have used Definition \ref{defn:dotreal} and Properties \ref{proper:transp}.
\end{proper}
\begin{exmp}
For $\vec{u} = (1,3,1)^T$ and $\vec{v} = (2,-1,1)^T$, find $\norm{(\vec{u} + \vec{v})}^2 = (\vec{u} + \vec{v}) \cdot (\vec{u} + \vec{v})$.
\end{exmp}
\begin{solution}
By Properties \ref{proper:dotproper}, we can rewrite the expression as
\begin{align*}
(\vec{u} + \vec{v}) \cdot (\vec{u} + \vec{v}) &= \vec{u} \cdot (\vec{u} + \vec{v}) + \vec{v} \cdot (\vec{u} + \vec{v}) \\
&= \vec{u} \cdot \vec{u} + \vec{u} \cdot \vec{v} + \vec{v} \cdot \vec{u} + \vec{v} \cdot \vec{v} \\
&= \vec{u} \cdot \vec{u} + 2 \vec{u} \cdot \vec{v} + \vec{v} \cdot \vec{v}
\end{align*}
Subsequently,
\begin{align*}
&\quad \vec{u} \cdot \vec{u} + 2 \vec{u} \cdot \vec{v} + \vec{v} \cdot \vec{v} \\
&= (1,3,1)^T \cdot (1,3,1)^T + 2((1,3,1)^T \cdot (2,-1,1)^T) + (2,-1,1)^T \cdot (2,-1,1)^T \\
&= (1^2 + 3^2 + 1^2) + 2((1)(2)+(3)(-1)+(1)(1)) + (2^2 + (-1)^2 + 1^2) \\
&= 11 + 2(0) + 6 \\
&= 17
\end{align*}
Alternatively, one can calculate $\vec{w} = \vec{u} + \vec{v} = (1,3,1)^T + (2,-1,1)^T = (3,2,2)^T$ and find $\vec{w} \cdot \vec{w} = \norm{\vec{w}}^2$ instead. (which is easier)
\end{solution}
\begin{exmp}
Given $\vec{u}$ and $\vec{v}$ as defined in the example above, if
\begin{align*}
A =
\begin{bmatrix}
1 & 2 & 1 \\
2 & 0 & 3 \\
1 & 1 & -1
\end{bmatrix}
\end{align*}
verify that $\vec{u} \cdot (A\vec{v}) = (A^T\vec{u}) \cdot \vec{v}$.
\end{exmp}
\begin{solution}
\begin{align*}
A\vec{v} &= 
\begin{bmatrix}
1 & 2 & 1 \\
2 & 0 & 3 \\
1 & 1 & -1
\end{bmatrix}
\begin{bmatrix}
2 \\
-1 \\
1
\end{bmatrix} \\
&=
\begin{bmatrix}
(1)(2) + (2)(-1) + (1)(1) \\
(2)(2) + (0)(-1) + (3)(1) \\
(1)(2) + (1)(-1) + (-1)(1) 
\end{bmatrix} \\
&=
\begin{bmatrix}
1 \\
7 \\
0
\end{bmatrix} \\
\vec{u} \cdot (A\vec{v}) &= (1,3,1)^T \cdot (1,7,0)^T \\
&= (1)(1) + (3)(7) + (1)(0) \\
&= 22
\end{align*}
On the other hand,
\begin{align*}
A^T\vec{u} &= 
\begin{bmatrix}
1 & 2 & 1 \\
2 & 0 & 1 \\
1 & 3 & -1
\end{bmatrix}
\begin{bmatrix}
1 \\
3 \\
1
\end{bmatrix} \\
&=
\begin{bmatrix}
(1)(1) + (2)(3) + (1)(1) \\
(2)(1) + (0)(3) + (1)(1) \\
(1)(1) + (3)(3) + (-1)(1) 
\end{bmatrix} \\
&=
\begin{bmatrix}
8 \\
3 \\
9
\end{bmatrix} \\
(A^T\vec{u}) \cdot \vec{v} &= (8,3,9)^T \cdot (2,-1,1)^T \\
&= (8)(2) + (3)(-1) + (9)(1) \\
&= 22
\end{align*}
\end{solution}

\subsubsection{Geometric Meaning of Dot Product}
The geometric meaning of dot product is embedded in the relation below.
\begin{proper}
\label{proper:dotgeo}
For two vectors $\vec{u}$ and $\vec{v}$ that are of the same dimension, we have
\begin{align*}
\vec{u} \cdot \vec{v} = \norm{\vec{u}}\norm{\vec{v}}\cos\theta
\end{align*}
where $\theta$ is the angle between $\vec{u}$ and $\vec{v}$. Furthermore, if $\hat{u}$ and $\hat{v}$ are unit vectors (Definition \ref{defn:unitvec}), it reduces to
\begin{align*}
\hat{u} \cdot \hat{v} = \cos\theta    
\end{align*}
\end{proper}
This means that the dot product between two vectors $\vec{u}$ and $\vec{v}$ is geometrically the signed product between $\vec{u}$ and the parallel component (projection) of $\vec{v}$ onto $\vec{u}$ (or vice versa), which is illustrated in the figure below. While an angle has a clear physical meaning only in a two/three-dimensional space, such relation generalizes the idea of an angle to higher dimensions.
\begin{center}
\begin{tikzpicture}[scale=1.3]
\coordinate (0) at (0,0);
\coordinate (vecu) at (4,1);
\coordinate (vecv) at (1,2);
\draw[->](0)--(vecu) node[right]{$\vec{u}$};
\draw[->](0)--(vecv) node[above]{$\vec{v}$};
\draw[dashed] (1,2)--(24/17, 6/17);
\draw[red] (24/17+0.2, 6/17+0.05)--(24/17+0.15, 6/17+0.25)--(24/17-0.05, 6/17+0.2);
\pic[draw, "$\theta$", angle eccentricity=1.5] {angle = vecu--0--vecv};
\draw[blue, very thick] (0,0)--(24/17, 6/17) node[below, shift={(0mm, -2mm)}]{$\norm{\vec{v}}\cos\theta$};
\end{tikzpicture}
\end{center}
\begin{exmp}
Find the angle between $\vec{u}$ and $\vec{v}$ in Example \ref{exmp:dotproduct5d}.
\end{exmp}
\begin{solution}
From Example \ref{exmp:dotproduct5d}, we have $\vec{u} \cdot \vec{v} = -3$, and
\begin{align*}
\norm{\vec{u}} &= \sqrt{1^2 + 2^2 + 3^2 + 4^2 + 5^2} = \sqrt{55} \\
\norm{\vec{v}} &= \sqrt{(-1)^2 + 0^2 + 1^2 + 0^2 + (-1)^2} = \sqrt{3} 
\end{align*}
By Properties \ref{proper:dotgeo}, we have
\begin{align*}
\cos\theta &= \frac{\vec{u} \cdot \vec{v}}{\norm{\vec{u}}\norm{\vec{v}}} \\
&= \frac{-3}{(\sqrt{55})(\sqrt{3})} \\
&\approx -0.2335 \\
\theta &\approx \SI{1.806}{\radian}
\end{align*}
\end{solution}
By Properties \ref{proper:dotgeo}, if the absolute value of the dot product $|\vec{u} \cdot \vec{v}|$ is equal to $\norm{\vec{u}}\norm{\vec{v}}$, where $\vec{u}$ and $\vec{v}$ are non-zero vectors, then it implies that $\cos\theta = \pm 1$, $\theta$ is either $0$ or $\pi$, and hence the two vectors are parallel. On the other hand, we have the following observation.
\begin{proper}
\label{proper:dotorth}
If the dot product between two vectors $\vec{u}$ and $\vec{v}$ is zero ($\vec{u} \cdot \vec{v} = \vec{v} \cdot \vec{u} = 0$), then by Properties \ref{proper:dotgeo}, $\cos\theta = 0$ and the angle $\theta$ between $\vec{u}$ and $\vec{v}$ is $\frac{\pi}{2}$. In this case, $\vec{u}$ and $\vec{v}$ are said to be perpendicular, or \textit{orthogonal}. The converse is also true. 
\end{proper}
From this, the concept of "\index{Orthogonal}\keywordhl{Orthogonal}" becomes an extension of "perpendicular" in higher dimensions. It is easy to see that the standard unit vectors of $\mathbb{R}^n$ are orthogonal. Note that \textit{the zero vector is regarded to be orthogonal to any vector}, so even if $\vec{u}$ or $\vec{v}$ is a zero vector, this properties still hold. \par
Some may notice that as $-1 \leq \cos\theta \leq 1$, if $|\vec{u} \cdot \vec{v}| > \norm{\vec{u}}\norm{\vec{v}}$, then $\theta$ will be undefined in Properties \ref{proper:dotgeo}. However, the \index{Cauchy–Schwarz Inequality}\keywordhl{Cauchy–Schwarz Inequality} ensures this will not happen.
\begin{thm}[Cauchy–Schwarz Inequality]
\label{thm:CauchySch}
Given two \textit{real} vectors $\vec{u}$ and $\vec{v}$ that are $n$-dimensional ($\mathbb{R}^n$), the following inequality holds.
\begin{align*}
|\vec{u} \cdot \vec{v}| &\leq \norm{\vec{u}}\norm{\vec{v}} \\
|u_1v_1+u_2v_2+\cdots+u_nv_n| &\leq \sqrt{u_1^2+u_2^2+\cdots+u_n^2}\sqrt{v_1^2+v_2^2+\cdots+v_n^2}
\end{align*}
\end{thm}
\begin{proof}
Consider $\vec{w} = \vec{u} + t\vec{v}$, where $t$ is any scalar, then $\norm{\vec{w}}^2 = \vec{w}\cdot\vec{w}$ can be written as a quadratic polynomial as
\begin{align*}
(\vec{u} + t\vec{v}) \cdot (\vec{u} + t\vec{v}) &= \norm{\vec{u}}^2 + 2t(\vec{u} \cdot \vec{v}) + t^2\norm{\vec{v}}^2
\end{align*}
using Properties \ref{proper:dotproper}. Now notice that from Properties \ref{proper:lengthdot}, we have
\begin{align*}
\norm{\vec{u}}^2 + 2t(\vec{u} \cdot \vec{v}) + t^2\norm{\vec{v}}^2 =(\vec{u} + t\vec{v}) \cdot (\vec{u} + t\vec{v}) = \norm{\vec{u} + t\vec{v}}^2 \geq 0    
\end{align*}
Since it is a quadratic polynomial, and we have shown that it is always greater than or equal to zero, i.e.\ has no root or a repeated root, it means that the discriminant must be negative or zero. So,
\begin{align*}
\Delta = b^2 - 4ac &\leq 0 \\
(2(\vec{u} \cdot \vec{v}))^2 - 4\norm{\vec{u}}^2\norm{\vec{v}}^2 &\leq 0 \\
(\vec{u} \cdot \vec{v})^2 - \norm{\vec{u}}^2\norm{\vec{v}}^2 &\leq 0 \\
(\vec{u} \cdot \vec{v})^2 &\leq \norm{\vec{u}}^2\norm{\vec{v}}^2 \\
|\vec{u} \cdot \vec{v}| &\leq \norm{\vec{u}}\norm{\vec{v}}
\end{align*}
\end{proof}
Short Exercise: Think about under what circumstances the Cauchy–Schwarz Inequality turns into an equality (i.e.\ $|\vec{u} \cdot \vec{v}| = \norm{\vec{u}}\norm{\vec{v}}$).\footnote{When $\vec{u}$ and $\vec{v}$ are parallel, i.e. $\vec{u} = k\vec{v}$ for some scalar $k$, or $\vec{v} = \textbf{0}$.}

\begin{exmp}
Prove the \index{Cosine Law}\keywordhl{Cosine Law} by considering the triangle below
\begin{center}
\begin{tikzpicture}
\coordinate (0) at (0,0);
\draw[red,-{Latex[length=5mm, width=2mm]}] (0)--(5,1) node[right](vecu){$\vec{u}$};
\draw[blue,-{Latex[length=5mm, width=2mm]}] (0)--(-1,3) node[above](vecv){$\vec{v}$};
\pic[draw, "$\theta$", angle eccentricity=1.5] {angle = vecu--0--vecv};
\draw[Green,-{Latex[length=5mm, width=2mm]}] (-1,3)--(5,1) node[midway, above, shift={(0mm, 3mm)}]{$\vec{u} - \vec{v}$};
\end{tikzpicture}
\end{center}
and expanding the dot product $\norm{(\vec{u}-\vec{v})}^2 = (\vec{u}-\vec{v}) \cdot (\vec{u}-\vec{v})$.    
\end{exmp}
\begin{solution}
Let denote the lengths $\norm{\vec{u}}$, $\norm{\vec{v}}$, $\norm{(\vec{u}-\vec{v})}$ be $a$, $b$, $c$, then
\begin{align*}
c^2 = \norm{(\vec{u}-\vec{v})}^2 &= (\vec{u}-\vec{v}) \cdot (\vec{u}-\vec{v}) & \text{(Properties \ref{proper:lengthdot})} \\
&= \vec{u} \cdot \vec{u} - \vec{u} \cdot \vec{v} - \vec{v} \cdot \vec{u} + \vec{v} \cdot \vec{v} & \text{(Properties \ref{proper:dotproper})} \\
&= \norm{\vec{u}}^2 - 2\vec{u} \cdot \vec{v} + \norm{\vec{v}}^2 & \text{(Properties \ref{proper:lengthdot} and \ref{proper:dotproper})} \\
&= \norm{\vec{u}}^2 - 2\norm{\vec{u}}\norm{\vec{v}}\cos\theta + \norm{\vec{v}}^2 & \text{(Properties \ref{proper:dotgeo})} \\
&= a^2 - 2ab\cos\theta + b^2
\end{align*}
\end{solution}

\subsection{Cross Product}
\label{section:crossprod}
Another important type of vector product is the \index{Cross Product}\keywordhl{Cross Product} (or sometimes just \index{Vector Product}\keywordhl{Vector Product}), which returns a three-dimensional vector from two other three-dimensional vectors as inputs. \textit{The output vector has to be orthogonal to the two input vectors}, and the direction is determined by the \index{Right Hand Rule}\keywordhl{Right Hand Rule}. Motivated by these requirements, we have the following basic definitions of cross product between the three standard unit vectors in $\mathbb{R}^3$.
\begin{defn}
\label{defn:crossijk}
The computation of cross products (denoted by $\times$) involving the standard unit vectors $\hat{i}$, $\hat{j}$, $\hat{k}$ in $\mathbb{R}^3$ obeys the following rules.
\begin{enumerate}
\item $\hat{i} \times \hat{j} = \hat{k}$, $\hat{j} \times \hat{i} = -\hat{k}$,
\item $\hat{j} \times \hat{k} = \hat{i}$, $\hat{k} \times \hat{j} = -\hat{i}$,
\item $\hat{k} \times \hat{i} = \hat{j}$, $\hat{i} \times \hat{k} = -\hat{j}$, and
\item $\hat{i} \times \hat{i} = \hat{j} \times \hat{j} = \hat{k} \times \hat{k} = \textbf{0}$
\end{enumerate}
\end{defn}
\begin{minipage}{0.45\textwidth}
\begin{center}
% https://tex.stackexchange.com/questions/287284/drawing-a-diagram-of-a-three-cycle
\begin{tikzpicture}[->,scale=2]
   \node (i) at (90:1cm)  {\huge$\hat{i}$};
   \node (j) at (-30:1cm) {\huge$\hat{j}$};
   \node (k) at (210:1cm) {\huge$\hat{k}$};

   \draw (70:1cm)  arc (70:-10:1cm);
   \draw (-50:1cm) arc (-50:-130:1cm);
   \draw (190:1cm) arc (190:110:1cm);
\end{tikzpicture} \\
A cyclic diagram for memorizing Definition \ref{defn:crossijk}. A clockwise / anti-clockwise permutation produces a positive / negative unit vector of the third.
\end{center}
\end{minipage}
\hfill
\begin{minipage}{0.5\textwidth}
\begin{center}
% https://tikz.net/righthand_rule/
\begin{tikzpicture}[scale=0.8]
  \coordinate (O) at (1.0,0.7);
  \coordinate (WT) at ( 2.9,-1.1);
  \coordinate (T1) at ( 2.3, 0.7);
  \coordinate (T2) at ( 1.75, 2.3);
  \coordinate (T3) at ( 2.0, 3.1);
  \coordinate (T4) at (1.38, 3.15);
  \coordinate (T5) at ( 0.9, 2.3);
  \coordinate (T6) at ( 0.85, 1.2);
  \coordinate (T7) at ( 0.85, 0.2);
  \coordinate (I1) at (-1.1, 2.45);
  \coordinate (I2) at (-2.9, 3.45);
  \coordinate (I3) at (-3.3, 2.9);
  \coordinate (I4) at (-1.5, 1.8);
  \coordinate (I5) at (-0.9, 1.1);
  \coordinate (I6) at (-0.9, 0.3);
  \coordinate (M1) at (-2.1, 0.9);
  \coordinate (M2) at (-3.95,0.55);
  \coordinate (M3) at (-4.0,-0.15);
  \coordinate (M4) at (-2.3, 0.05);
  \coordinate (M5) at (-1.1, 0.20);
  \coordinate (R1) at (-1.9,-0.1);
  \coordinate (R2) at (-1.8,-0.7);
  \coordinate (R3) at (-0.3,-1.5);
  \coordinate (R4) at ( 0.1,-1.7);
  \coordinate (R5) at ( 0.1,-1.0);
  \coordinate (R6) at (-0.5,-0.7);
  \coordinate (R7) at (-1.2,-0.3);
  \coordinate (P1) at (-1.9,-1.3);
  \coordinate (P2) at (-0.8,-1.9);
  \coordinate (P3) at (-0.2,-2.1);
  \coordinate (P4) at (-0.05,-1.65);
  \coordinate (W1) at ( 0.4,-2.9);
  \coordinate (W2) at ( 1.6,-3.5);
  
  % HAND
  \fill[pink!25]
    (WT) -- (T6) -- (I5) -- (M5) -- (R2) -- (P2) -- (W2) to[out=25,in=-90] cycle;
  \draw[fill=pink!25]
    (WT) to[out=120,in=-60]
    (T1) to[out=120,in=-90]
    (T2) to[out=80,in=-110]
    (T3) to[out=80,in=50,looseness=1.5]
    (T4) to[out=-130,in=80]
    (T5) to[out=-100,in=70]
    (T6) to[out=-100,in=100]
    (T7)
    (T6) to[out=150,in=-30]
    (I1) to[out=150,in=-30]
    (I2) to[out=150,in=145,looseness=1.7]
    (I3) to[out=-30,in=150]
    (I4) to[out=-30,in=105]
    (I5) to[out=-75,in=90]
    (I6)
    (I5) to[out=-170,in=10]
    (M1) to[out=-170,in=10]
    (M2) to[out=-170,in=-175,looseness=1.8]
    (M3) to[out=5,in=-170]
    (M4) to[out=10,in=-170]
    (M5)
    (M5) to[out=-160,in=50]
    (R1) to[out=-130,in=140,looseness=1.2]
    (R2) to[out=-30,in=160]
    (R3) --
    (R4) to[out=-20,in=-20,looseness=1.5]
    (R5) --
    (R6) to[out=140,in=8,looseness=0.9]
    (R7)
    (R2) to[out=-160,in=155]
    (P1) to[out=-35,in=150]
    (P2) to[out=-30,in=160]
    (P3) to[out=-20,in=-30,looseness=1.5]
    (R4)
    (P2) to[out=-50,in=140]
    (W1) to[out=-40,in=160]
    (W2);
    
  \draw[->, blue, line width=1] (O) -- (128:3.2) coordinate(X) node[above=6,left=-6,scale=1.5] {$\vec{u}$};
  \draw[->, red, line width=1] (O) -- (-182:3.2) coordinate(Y) node[above=5,left=-6,scale=1.5] {$\vec{v}$};
  \draw[->, Purple, line width=2] (O) -- (62:3.2) node[above=-1,scale=1.5] {$\textcolor{blue}{\vec{u}} \textcolor{Purple}{\;\times\;} \textcolor{red}{\vec{v}}$};
  \draw pic[->, "$\theta$", draw=black, thick, angle radius=30, angle eccentricity=1.2] {angle = X--O--Y};
\end{tikzpicture}\\
Demonstration of the right hand rule.
\end{center}
\end{minipage}
\par
Cross product shares pretty much the same properties with dot product (compared to Properties \ref{proper:dotproper}), except the (anti-)commutative part as the right hand rule specifically requires chirality that is compatible with Definition \ref{defn:crossijk}. Also, be aware of the difference that cross product produces another three-dimensional vector, but dot product yields a scalar.
\begin{proper}
\label{proper:crossproper}
For two three-dimensional vectors $\vec{u}$ and $\vec{v}$, we have
\begin{align*}
\vec{u} \times \vec{v} &= -\vec{v} \times \vec{u} &\text{Anti-commutative Property} \\
\vec{u} \times (\vec{v} \pm \vec{w}) &= \vec{u} \times \vec{v} \pm \vec{u} \times \vec{w} &\text{Distributive Property} \\
(\vec{u} \pm \vec{v}) \times \vec{w} &= \vec{u} \times \vec{w} \pm \vec{v} \times \vec{w} &\text{Distributive Property} \\
(a\vec{u}) \times (b\vec{v}) &= ab(\vec{u} \times \vec{v}) &\text{where $a$, $b$ are some constants}
\end{align*}
\end{proper}
The calculation of cross product then follows from the rules above, leading to the determinant shorthand below. 
\begin{proper}
\label{proper:crossdet}
For $\vec{u} = (u_1, u_2, u_3)^T$ and $\vec{v} = (v_1, v_2, v_3)^T$ in $\mathbb{R}^3$, their cross product $\vec{u} \times \vec{v}$ can be written in the form of a determinant as
\begin{align*}
\vec{u} \times \vec{v} =
\begin{vmatrix}
\hat{i} & \hat{j} & \hat{k} \\
u_1 & u_2 & u_3 \\
v_1 & v_2 & v_3
\end{vmatrix}
\end{align*}
\end{proper}
\begin{proof}
Starting from Definition \ref{defn:crossijk} and Properties \ref{proper:crossproper}, we have
\begin{align*}
\vec{u} \times \vec{v} &= (u_1\hat{i} + u_2\hat{j} + u_3\hat{k}) \times (v_1\hat{i} + v_2\hat{j} + v_3\hat{k}) \\
&= u_1v_1(\hat{i}\times\hat{i}) + u_1v_2(\hat{i}\times\hat{j}) + u_1v_3(\hat{i}\times\hat{k}) \\
&\quad +u_2v_1(\hat{j}\times\hat{i}) + u_2v_2(\hat{j}\times\hat{j}) + u_2v_3(\hat{j}\times\hat{k}) \\
&\quad +u_3v_1(\hat{k}\times\hat{i}) + u_3v_2(\hat{k}\times\hat{j}) + u_3v_3(\hat{k}\times\hat{k}) & \text{(Properties \ref{proper:crossproper})}\\
&= u_1v_1(\textbf{0}) + u_1v_2(\hat{k}) - u_1v_3(\hat{j}) \\
&\quad -u_2v_1(\hat{k}) + u_2v_2(\textbf{0}) + u_2v_3(\hat{i}) \\
&\quad +u_3v_1(\hat{j}) - u_3v_2(\hat{i}) + u_3v_3(\textbf{0}) & \text{(Definition \ref{defn:crossijk})}\\
&= (u_2v_3 - u_3v_2)\hat{i} + (u_3v_1 - u_1v_3)\hat{j} + (u_1v_2 - u_2v_1)\hat{k} 
\end{align*}
Meanwhile, expanding along the first row of the determinant form
\begin{align*}
\begin{vmatrix}
\hat{i} & \hat{j} & \hat{k} \\
u_1 & u_2 & u_3 \\
v_1 & v_2 & v_3
\end{vmatrix} 
&= 
\hat{i}
\begin{vmatrix}
u_2 & u_3 \\
v_2 & v_3
\end{vmatrix} 
- \hat{j}
\begin{vmatrix}
u_1 & u_3 \\
v_1 & v_3
\end{vmatrix} 
+ \hat{k}
\begin{vmatrix}
u_1 & u_2 \\
v_1 & v_2
\end{vmatrix} \\
&= (u_2v_3 - u_3v_2)\hat{i} + (u_3v_1 - u_1v_3)\hat{j} + (u_1v_2 - u_2v_1)\hat{k}
\end{align*}
yields the identical results.
\end{proof}

\begin{exmp}
Given two vectors
\begin{align*}
&\vec{u} =
\begin{bmatrix}
1 \\
0 \\
2
\end{bmatrix}
&\vec{v} =
\begin{bmatrix}
3 \\
-1 \\
1
\end{bmatrix}
\end{align*}
Find $\vec{u} \times \vec{v}$.
\end{exmp}
\begin{solution}
\begin{align*}
\vec{u} \times \vec{v} &=
\begin{vmatrix}
\hat{i} & \hat{j} & \hat{k} \\
1 & 0 & 2 \\
3 & -1 & 1
\end{vmatrix} \\
&= 
\hat{i}
\begin{vmatrix}
0 & 2 \\
-1 & 1 
\end{vmatrix}
- \hat{j}
\begin{vmatrix}
1 & 2 \\
3 & 1 
\end{vmatrix}
+ \hat{k}
\begin{vmatrix}
1 & 0 \\
3 & -1 
\end{vmatrix}
& \begin{aligned}
\text{(Cofactor Expansion} \\ 
\text{along the first row)}
\end{aligned}\\
&= 2\hat{i} + 5\hat{j} - \hat{k} = (2,5,-1)^T
\end{align*} 
\end{solution}
Short Exercise: Check if $\vec{u} \times \vec{v}$ is orthogonal to $\vec{u}$ and $\vec{v}$ by finding the corresponding dot products.\footnote{$\vec{u} \cdot (\vec{u} \times \vec{v}) = (1,0,2)^T\cdot(2,5,-1)^T = (1)(2) + (0)(5) + (2)(-1) = 0$, $\vec{v} \cdot (\vec{u} \times \vec{v}) = (3,-1,1)^T\cdot(2,5,-1)^T = (3)(2) + (-1)(5) + (1)(-1) = 0$. In both cases the zero dot product shows they are orthogonal.}\\
Short Exercise: Following the short exercise above, show in general, $\vec{u} \cdot (\vec{u} \times \vec{v}) = \vec{v} \cdot (\vec{u} \times \vec{v}) = 0$.\footnote{From the derivation of Properties \ref{proper:crossdet}, $\vec{u} \times \vec{v} = (u_2v_3 - u_3v_2)\hat{i} + (u_3v_1 - u_1v_3)\hat{j} + (u_1v_2 - u_2v_1)\hat{k}$, and $\vec{u} \cdot (\vec{u} \times \vec{v}) = u_1(u_2v_3 - u_3v_2) + u_2(u_3v_1 - u_1v_3) + u_3(u_1v_2 - u_2v_1) = 0$ where all terms cancel out, similar for $\vec{v}$.}


\subsubsection{Geometric Meaning of Cross Product} Similar to vector dot product, vector cross product has a geometric interpretation.
\begin{proper}
\label{proper:crossgeo}
Given two vectors $\vec{u}$ and $\vec{v}$ which are both three-dimensional, the magnitude (length) of $\vec{u} \times \vec{v}$ is related to the angle between $\vec{u}$ and $\vec{v}$ as
\begin{align*}
\norm{\vec{u} \times \vec{v}} = \norm{\vec{u}}\norm{\vec{v}}\sin\theta
\end{align*}
\end{proper}
Immediately, we know that if $\vec{u}$ and $\vec{v} = k\vec{u}$, where $k$ is some constant, are parallel, their cross product will be a zero vector as $\theta = 0$ (or $\pi$) and $\sin \theta = 0$. This is equivalent to the statement of $\vec{u} \times \vec{u} = \textbf{0}$. (You can also arrive at this conclusion with Properties \ref{proper:crossproper}.\footnote{The anti-commutative property requires $\vec{u}\times\vec{u} = -\vec{u}\times\vec{u}$ and hence $2(\vec{u}\times\vec{u}) = \textbf{0}$.})

\begin{exmp}
If $\vec{u} = (1,2,3)^T$, and $\vec{v} = (-1,1,0)^T$, find $(\vec{u} + 2\vec{v}) \times (\vec{u} - \vec{v}) $.
\end{exmp}
\begin{solution}
Observe that
\begin{align*}
(\vec{u} + 2\vec{v}) \times (\vec{u} - \vec{v}) &= \vec{u} \times (\vec{u} - \vec{v}) + 2\vec{v} \times (\vec{u} - \vec{v}) \\
&= \vec{u} \times \vec{u} - \vec{u} \times \vec{v} + 2\vec{v} \times \vec{u} - 2\vec{v} \times \vec{v} \\
&= \textbf{0} - \vec{u} \times \vec{v} - 2\vec{u} \times \vec{v} - 2(\textbf{0}) \\
&= -3\vec{u} \times \vec{v}
\end{align*}
where the fact that $\vec{u} \times \vec{u} = 0$, $\vec{v} \times \vec{v} = 0$ and Properties \ref{proper:crossproper} are used. Now, with Properties \ref{proper:crossdet}, we have
\begin{align*}
-3\vec{u} \times \vec{v} &=  
-3
\begin{vmatrix}
\hat{i} & \hat{j} & \hat{k} \\
1 & 2 & 3 \\
-1 & 1 & 0
\end{vmatrix} \\
&= -3\left(\hat{i}
\begin{vmatrix}
2 & 3 \\
1 & 0 
\end{vmatrix}
- \hat{j}
\begin{vmatrix}
1 & 3 \\
-1 & 0 
\end{vmatrix}
+ \hat{k}
\begin{vmatrix}
1 & 2 \\
-1 & 1 
\end{vmatrix}\right) \\
&= -3(-3\hat{i}-3\hat{j}+3\hat{k}) \\
&= 9\hat{i}+9\hat{j}-9\hat{k} = (9,9,-9)^T
\end{align*}
The readers can try the alternative of computing $\vec{u}+2\vec{v}$ and $\vec{u} - \vec{v}$ first and then finally their cross product.
\end{solution}

Finally, cancellation of dot product or cross product at both sides of an equation is generally not correct, and here is a table summarizing the inputs and outputs of dot/cross product for clarification.
\begin{center}
\begin{tabular}{|p{35mm}|p{55mm}|p{25mm}|}
\hline
 & Input & Output \\
\hline
Dot Product, or Scalar Product ($\cdot$) & Two vectors of the same dimension, the order does not matter (commutative) & A scalar\\
\hline
Cross Product, or Vector Product ($\times$) & Two three-dimensional vectors ($\mathbb{R}^3$), the order is important (anti-commutative) & Another three-dimensional vector
\\
\hline
\end{tabular}
\end{center}

\section{Earth Science Applications}
\begin{exmp}
\label{exmp:Coriolis}
The \textit{Coriolis Effect} is a phenomenon describing the deflection of motion due to rotation of the Earth. It introduces an apparent force known as \textit{Coriolis Force} which is given by $\overrightarrow{F_\text{cor}} = -2\vec{\Omega} \times \vec{v}$ where $\Omega = \norm{\vec{\Omega}} = \SI{7.292e-5}{\radian \per \s}$ represents the angular speed of Earth's rotation, and $\vec{\Omega}$ is oriented in the direction of the North Pole. Define the local frame of reference (see Figure \ref{fig:Coriolis}) with the $x$-direction being the zonal direction, $y$-direction being the meridional direction, and $z$-direction being the zenith direction (normal to the Earth's surface), then we have $\vec{v} = (u,v,w) = u\hat{i} + v\hat{j} + w\hat{k}$ as the flow velocity in this local Cartesian coordinate system with unit vectors $\hat{i}, \hat{j}, \hat{k}$ along the $x$, $y$, $z$ axes. It can be seen that $\vec{\Omega} = (\Omega \cos\varphi) \hat{j} + (\Omega \sin \varphi) \hat{k}$ where $\varphi$ is the latitude. Now by expanding $\overrightarrow{F_\text{cor}} = -2\vec{\Omega} \times \vec{v}$ show that the components of Coriolis Force along the local $x,y,z$ directions are
\begin{align*}
F_{\text{cor},x} &= 2\Omega (v\sin\varphi - w\cos\varphi) \\
F_{\text{cor},y} &= -2\Omega u \sin\varphi \\
F_{\text{cor},z} &= 2\Omega u \cos\varphi
\end{align*}
The \textit{Coriolis Parameter} $f$ is usually used to denote the factor $2\Omega\sin\varphi$.
\end{exmp}
\begin{figure}[h!]
\centering
\begin{tikzpicture}
\coordinate (0) at (0,0);
\draw[black, bottom color=blue!40, top color=green!40, shading angle=-23.5] (0,0) circle (2);
\node[Mahogany] at (0,2.3) {Earth};
\draw[dashed,->] (66.5:-3) -- (66.5:3) node[above]{$\vec{\Omega}$};
\draw[dashed] (0) -- (-23.5:2) node(vecE){};
\path (0) -- (-23.5:2) node[midway, sloped, below]{Equator};
\draw[dashed] (0) -- (10:2) node(vecL){};
\draw pic["$\varphi$", draw=black, thick, angle eccentricity=1.5] {angle = vecE--0--vecL};
\draw[red, ->] (10:2) --++ (10:1.5) node[right]{$\hat{k}$};
\draw[red, ->] (10:2) --++ (100:1.5) node[above]{$\hat{j}$};
\draw[red, fill=gray!20] (10:2) circle (0.25) node[below right, yshift=-6]{$\hat{i}$};
\draw[red] (10:2) --++ (45:0.25);
\draw[red] (10:2) --++ (45:-0.25);
\draw[red] (10:2) --++ (-45:0.25);
\draw[red] (10:2) --++ (-45:-0.25);
\coordinate (P) at (5,-0.5);
\draw[red, ->] (P) --++ (10:1.5) node[right]{$\hat{k}$};
\draw[red, ->] (P) --++ (100:1.5) node[above](vecJ){$\hat{j}$};
\draw[dashed,->] (P) --++ (66.5:2) node[above](vecOM){};
\node at (8,1.7) {$\vec{\Omega} = (\Omega \cos \varphi)\hat{j} + (\Omega \sin \varphi)\hat{k}$};
\draw pic["$\varphi$", draw=black, thick, angle eccentricity=1.5] {angle = vecOM--P--vecJ};
\end{tikzpicture}
\caption{An illustration of the coordinate frame in Example \ref{exmp:Coriolis}.}
\label{fig:Coriolis}
\end{figure}
\begin{solution}
Using Properties \ref{proper:crossdet} to expand $\overrightarrow{F_\text{cor}}$ gives
\begin{align*}
-2\overrightarrow{\Omega} \times \vec{v} &= -2((\Omega \cos\varphi) \hat{j} + (\Omega \sin \varphi) \hat{k}) \times (u\hat{i} + v\hat{j} + w\hat{k}) \\
&= -2
\begin{vmatrix}
\hat{i} & \hat{j} & \hat{k} \\
0 & \Omega\cos\varphi & \Omega\sin\varphi \\
u & v & w 
\end{vmatrix} \\
&= -2[(w\Omega\cos\varphi - v\Omega\sin\varphi)\hat{i} + (u\Omega\sin\varphi)\hat{j} - (u\Omega\cos\varphi)\hat{k}] \\
&= [2\Omega(v\sin\varphi - w\cos\varphi)]\hat{i} + (-2\Omega u\sin\varphi)\hat{j} + (2\Omega u\cos\varphi)\hat{k}
\end{align*}
The $\hat{i}$, $\hat{j}$, $\hat{k}$ components correspond to $F_{\text{cor},x}$, $F_{\text{cor},y}$, $F_{\text{cor},z}$ respectively. Assume $w$ is negligible, then $F_{\text{cor},x} = fv$ and $F_{\text{cor},y} = -fu$.
\end{solution}

\section{Python Programming}
\label{section:ch4python}
We can use one-dimensional \verb|numpy| arrays as vectors. 
\begin{lstlisting}
import numpy as np

myVec1 = np.array([-1., 2., 4.])
myVec2 = np.array([2., 1., 3.])
\end{lstlisting}
Addition, subtraction, and scalar multiplication works just like for matrices.
\begin{lstlisting}
myVec3 = -myVec1 + 2*myVec2
print(myVec3)
\end{lstlisting}
gives the expected output of \verb|[5. 0. 2.]|. We can select a component of any vector by indexing. Again, remember that indices in \textit{Python} start from zero. \verb|print(myVec3[1])| then returns \verb|0.0|. The magnitude of a vector can be checked with \verb|np.linalg.norm|. For example,
\begin{lstlisting}
print(np.linalg.norm(myVec1))    
\end{lstlisting}
produces \verb|4.58257569495584| ($\sqrt{(-1)^2 + 2^2 + 4^2} = \sqrt{21}$). Dot product is computed via \verb|np.dot| as follows.
\begin{lstlisting}
myDot = np.dot(myVec1, myVec2)
print(myDot)
\end{lstlisting}
which outputs \verb|12.0| (as $(-1)(2) + (2)(1) + (4)(3) = 12$). Similarly, cross product is found by \verb|np.cross|.
\begin{lstlisting}
myCross = np.cross(myVec1, myVec2)
print(myCross)
\end{lstlisting}
then gives
\begin{lstlisting}
[ 2. 11. -5.]   
\end{lstlisting}
and we can check if the cross product is orthogonal to the two input vectors.
\begin{lstlisting}
# All lines below return zero.
print(np.dot(myVec1, myCross))
print(np.dot(myVec2, myCross))
print(np.dot(myVec3, myCross))
\end{lstlisting}
Dot product is defined for any two vectors with the same dimension, but cross product is only defined for three-dimensional vectors (or in some other sense two-dimensional), so
\begin{lstlisting}
myVec4 = np.array([1., 3., 2., 0.])
myVec5 = np.array([2., 1., 0., -1.])
print(np.dot(myVec4, myVec5))
\end{lstlisting}
yields a valid output of $5.0$, but
\begin{lstlisting}
print(np.cross(myVec4, myVec5))
\end{lstlisting}
raises the error of
\begin{lstlisting}
ValueError: incompatible dimensions for cross product
(dimension must be 2 or 3)    
\end{lstlisting}
Finally, we note that following \href{https://stackoverflow.com/questions/2827393/angles-between-two-n-dimensional-vectors-in-python}{this Stack Overflow post} (2827393), we can compute the unit vector of any given vector and angle between two vectors (based from the second observation in Properties \ref{proper:dotgeo}, $\theta = \cos^{-1}(\hat{u} \cdot \hat{v})$).
\begin{lstlisting}
def unit_vector(vector):
    """ Returns the unit vector of the vector.  """
    return vector / np.linalg.norm(vector)

def angle_between(v1, v2):
    """ Returns the angle in radians between vectors 'v1' and 'v2'. """
    v1_u = unit_vector(v1)
    v2_u = unit_vector(v2)
    return np.arccos(np.clip(np.dot(v1_u, v2_u), -1.0, 1.0))    
\end{lstlisting}
The \verb|np.clip| is to avoid numerical round-off error that causes the dot product of the two normalized input vectors to just fall outside (e.g. \verb|1.0000000000000002|) the valid range $[-1, 1]$ of $\cos^{-1}$. The naive way of (here the lists will be cast to one-dimensional arrays automatically during calculation.) 
\begin{lstlisting}
np.arccos(np.dot([1., 0, 0], [2., 0, 0]))
\end{lstlisting}
leads to the warning of
\begin{lstlisting}
RuntimeWarning: invalid value encountered in arccos  
nan
\end{lstlisting}
but
\begin{lstlisting}
angle_between([1., 0, 0], [2., 0, 0])
\end{lstlisting}
gives \verb|0.0| properly. Trying this on \verb|myVec4| and \verb|myVec5| with
\begin{lstlisting}
print(unit_vector(myVec4))
print(angle_between(myVec4, myVec5))
\end{lstlisting}
produces a unit vector of \verb|[0.267 0.802 0.535 0.   ]|, and an angle of \verb|0.993757| (in radians).

\section{Exercises}

\begin{Exercise}
For $\vec{u} = (1, 3, 3, 7)^T$ and $\vec{v} = (1, 2, 2, 5)^T$, find
\begin{enumerate}[label=(\alph*)]
\item $\vec{u} + \vec{v}$,
\item $\frac{3}{2}\vec{u} - \frac{1}{2}\vec{v}$,
\item $\vec{u} \cdot \vec{v}$,
\item $\vec{v} \cdot \vec{u}$,
\item $(\vec{u} - 2\vec{v}) \cdot (2\vec{u} + \vec{v})$.
\end{enumerate}
\end{Exercise}
\begin{Answer}
\begin{enumerate}[label=(\alph*)]
\item $(2, 5, 5, 12)^T$ 
\item $(1, \frac{7}{2}, \frac{7}{2}, 8)^T$
\item $(1)(1) + (3)(2) + (3)(2) + (7)(5) = 48$
\item $(1)(1) + (2)(3) + (2)(3) + (5)(7) = 48$
\item $\vec{u}-2\vec{v} = (-1, -1, -1, -3)^T, 2\vec{u} + \vec{v} = (3, 8, 8, 19)^T, (\vec{u} - 2\vec{v}) \cdot (2\vec{u} + \vec{v}) = (-1)(3) + (-1)(8) + (-1)(8) + (-3)(19) = -76$ 
\end{enumerate}
\end{Answer}

\begin{Exercise}
\label{ex:ch4prob_coplanar}
For $\vec{u} = (7, 4, 1)^T$, $\vec{v} = (8, 1, 1)^T$, and
\begin{align*}
A = 
\begin{bmatrix}
1 & 1 & 1\\
0 & 1 & 1\\
0 & 0 & 1
\end{bmatrix}
\end{align*}
Verify that
\begin{enumerate}[label=(\alph*)]
\item $\vec{u} \times \vec{v} = -\vec{v} \times \vec{u}$, 
\item $\vec{u} \cdot (\vec{Av}) = (A^T\vec{u}) \cdot \vec{v}$, 
\item Compute $(3\vec{u} - 4\vec{v}) \cdot (\vec{u} \times \vec{v})$, is the answer what you expect?
\end{enumerate}
\end{Exercise}
\begin{Answer}
\begin{enumerate}[label=(\alph*)]
\item 
\begin{align*}
\vec{u} \times \vec{v} &= 
\begin{vmatrix}
\hat{i} & \hat{j} & \hat{k}\\
7 & 4 & 1\\
8 & 1 & 1    
\end{vmatrix}
= 3\hat{i} + \hat{j} - 25\hat{k} = (3, 1, -25)^T \\
\vec{v} \times \vec{u} &=
\begin{vmatrix}
\hat{i} & \hat{j} & \hat{k}\\
8 & 1 & 1 \\  
7 & 4 & 1
\end{vmatrix}
= -3\hat{i} - \hat{j} + 25\hat{k} = (-3, -1, 25)^T
\end{align*} 
\item 
\begin{align*}
A\vec{v} &= 
\begin{bmatrix}
1 & 1 & 1\\
0 & 1 & 1\\
0 & 0 & 1
\end{bmatrix}
\begin{bmatrix}
8 \\
1 \\
1
\end{bmatrix}
=
\begin{bmatrix}
10 \\
2 \\
1
\end{bmatrix} \\
\vec{u} \cdot (\vec{Av}) 
&= (7, 4, 1)^T \cdot (10, 2, 1)^T \\
&= (7)(10) + (4)(2) + (1)(1) \\
&= 79 \\
A^T\vec{u} &= 
\begin{bmatrix}
1 & 0 & 0\\
1 & 1 & 0\\
1 & 1 & 1
\end{bmatrix}
\begin{bmatrix}
7 \\ 
4 \\
1
\end{bmatrix}
=
\begin{bmatrix}
7 \\
11 \\
12
\end{bmatrix} \\
(A^T\vec{u}) \cdot \vec{v}
&= (7, 11, 12)^T \cdot (8, 1, 1)^T \\
&= (7)(8) + (11)(1) + (12)(1) \\
&= 79 \\
\end{align*}
\item By (a), $\vec{u} \times \vec{v} = (3, 1, -25)^T$ and $(3\vec{u} - 4\vec{v}) = (-11,8,-1)^T$, then
\begin{align*}
(3\vec{u} - 4\vec{v}) \cdot (\vec{u} \times \vec{v}) &= (-11,8,-1)^T \cdot (3, 1, -25)^T \\
&= (-11)(3) + (8)(1) + (-1)(-25) = 0
\end{align*}
This makes sense as we have shown that $\vec{u} \cdot (\vec{u} \times \vec{v}) = \vec{v} \cdot (\vec{u} \times \vec{v}) = 0$ in a previous short exercise, and therefore by distributive property $(\alpha\vec{u} + \beta\vec{v}) \cdot (\vec{u} \times \vec{v}) = 0$ for any $\alpha$ and $\beta$.
\end{enumerate}
\end{Answer}

\begin{Exercise}
For $\vec{u} = (1, -3, 9)^T$ and $\vec{v} = (1, -2, 4)^T$, find
\begin{enumerate}[label=(\alph*)]
\item Their unit vectors $\hat{u}$ and $\hat{v}$,
\item The angle between them, by calculating their dot product,
\item The cross product $\vec{u} \times \vec{v}$, and 
\item Show that the vector obtained from the cross product above is orthogonal (perpendicular) to $\vec{u}$ and $\vec{v}$, by calculating the corresponding dot products.
\end{enumerate}
\end{Exercise}
\begin{Answer}
\begin{enumerate}[label=(\alph*)]
\item 
\begin{align*}
\norm{\vec{u}} &= \sqrt{1^2 + (-3)^2 + 9^2} = \sqrt{91} \\
\hat{u} &= (\frac{1}{\sqrt{91}}, -\frac{3}{\sqrt{91}}, \frac{9}{\sqrt{91}})^T \\
\norm{\vec{v}} &= \sqrt{1^2 + (-2)^2 + 4^2} = \sqrt{21} \\
\hat{v} &= (\frac{1}{\sqrt{21}}, -\frac{2}{\sqrt{21}}, \frac{4}{\sqrt{21}})^T  
\end{align*}
\item 
\begin{align*}
\vec{u} \cdot \vec{v} = (1)(1) + (-3)(-2) + (9)(4) = 43 \\
\cos\theta = \frac{43}{\sqrt{21}\sqrt{91}} \approx 0.9836 \\
\theta \approx \SI{0.181}{\radian}    
\end{align*}
\item $\vec{u} \times \vec{v} = \begin{vmatrix}
\hat{i} & \hat{j} & \hat{k}\\
1 & -3 & 9\\
1 & -2 & 4
\end{vmatrix}
= 6\hat{i} + 5\hat{j} + \hat{k} = (6, 5, 1)^T$
\item $\vec{u} \cdot (\vec{u} \times \vec{v}) = (1, -3, 9)^T \cdot (6, 5, 1)^T = (1)(6) + (-3)(5) + (9)(1) = 0$, $\vec{v} \cdot (\vec{u} \times \vec{v}) = (1)(6) + (-2)(5) + (4)(1) = 0$
\end{enumerate}
\end{Answer}

\begin{Exercise}
The following table contains incomplete data about the movement of several typhoons at some moments. Complete the table by filling in the blanks. The first one has been done as an example.
\begin{center}
\footnotesize
\begin{tabular}{|c|c|c|c|c|}
\hline
Typhoon Name & Time & Speed & Direction & Vector Form\\
\hline
Nuri & 2008/08/22, 08:00 & \SI{13}{\km \per \hour} & \SI{315}{\degree} & $(-9.192, 9.192)$\\
\hline
Vicente & 2012/07/24, 02:00 & \SI{18}{\km \per \hour} & \SI{299}{\degree} & \\
\hline
Linfa & 2015/07/09, 23:00 & & & $(-13.595, -6.339)$\\
\hline
Mangkhut & 2018/09/16, 22:00 & & \SI{288}{\degree} & $(\quad, 7.725)$\\
\hline
\end{tabular}
\end{center}
\end{Exercise}
\begin{Answer}
\begin{center}
\footnotesize
\begin{tabular}{|c|c|c|c|c|}
\hline
Typhoon Name & Time & Speed & Direction & Vector Form\\
\hline
Nuri & 2008/08/22, 08:00 & \SI{13}{\km \per \hour} & \SI{315}{\degree} & $(-9.192, 9.192)$\\
\hline
Vicente & 2012/07/24, 02:00 & \SI{18}{\km \per \hour} & \SI{299}{\degree} & $\textcolor{red}{(-15.743, 8.727)}$\\
\hline
Linfa & 2015/07/09, 23:00 & \textcolor{red}{\SI{15}{\km \per \hour}} & \textcolor{red}{\SI{245}{\degree}} & $(-13.595, -6.339)$\\
\hline
Mangkhut & 2018/09/16, 22:00 & \textcolor{red}{\SI{25}{\km \per \hour}} & \SI{288}{\degree} & $(\textcolor{red}{-23.776}, 7.725)$\\
\hline
\end{tabular}
\end{center}    
\end{Answer}

\begin{Exercise}
Prove the Triangular Inequality.
\begin{align*}
\norm{\vec{u} + \vec{v}} \leq \norm{\vec{u}} + \norm{\vec{v}}
\end{align*}
\end{Exercise}
\begin{Answer}
\begin{align*}
\norm{\vec{u} + \vec{v}}^2 &= (\vec{u} + \vec{v}) \cdot (\vec{u} + \vec{v}) \\
&= \norm{\vec{u}}^2 + 2(\vec{u}\cdot\vec{v}) + \norm{\vec{v}}^2 \\
&\leq \norm{\vec{u}}^2 + 2\norm{\vec{u}}\norm{\vec{v}} + \norm{\vec{v}}^2 \\
&= (\norm{\vec{u}} + \norm{\vec{v}})^2
\end{align*}
\end{Answer}

\begin{Exercise}
\label{ex:parallellaw}
Prove the Parallelogram Law. (See Figure \ref{fig:parallellaw})
\begin{align*}
2\norm{\vec{u}}^2 + 2\norm{\vec{v}}^2 = \norm{\vec{u}+\vec{v}}^2 + \norm{\vec{u}-\vec{v}}^2
\end{align*}
\end{Exercise}
\begin{figure}
\centering
\begin{tikzpicture}
\draw[blue, ->] (0,0) -- (5,1) node[right]{$\vec{u}$};
\draw[red, ->] (0,0) -- (1,3) node[left]{$\vec{v}$};
\draw[Purple, ->] (1,3) -- (5,1) node[pos=0.75, sloped, above]{$\vec{u} - \vec{v}$};
\draw[Green, ->] (0,0) -- (6,4) node[pos=0.75, sloped, above]{$\vec{u} + \vec{v}$};
\draw[blue, dashed, ->] (1,3) -- (6,4);
\draw[red, dashed, ->] (5,1) -- (6,4);
\end{tikzpicture}
\caption{The parallelogram constructed by vectors for Exercise \ref{ex:parallellaw}.}
\label{fig:parallellaw}
\end{figure}
\begin{Answer}
\begin{align*}
\norm{\vec{u}+\vec{v}}^2 + \norm{\vec{u}-\vec{v}}^2 &= (\vec{u}+\vec{v}) \cdot (\vec{u}+\vec{v}) + (\vec{u}-\vec{v}) \cdot (\vec{u}-\vec{v}) \\
&= (\norm{\vec{u}}^2 + 2(\vec{u} \cdot \vec{v}) + \norm{\vec{v}}^2) + (\norm{\vec{u}}^2 - 2(\vec{u} \cdot \vec{v}) + \norm{\vec{v}}^2) \\
&= 2\norm{\vec{u}}^2 + 2\norm{\vec{v}}^2
\end{align*}
\end{Answer}

\begin{Exercise}
Show that Coriolis Force derived in Example \ref{exmp:Coriolis} does zero work and hence is consistent with the fact that it is an apparent force and never produces/consumes energy by itself.
\end{Exercise}
\begin{Answer}
In Example \ref{exmp:Coriolis}, we have
\begin{align*}
\overrightarrow{F_\text{cor}} &= (2\Omega(v\sin\varphi - w\cos\varphi))\hat{i} + (-2\Omega u\sin\varphi)\hat{j} + (2\Omega u\cos\varphi)\hat{k}    
\end{align*}
and hence the rate of work done is
\begin{align*}
&\quad \overrightarrow{F_\text{cor}} \cdot \vec{v} \\
&= [(2\Omega(v\sin\varphi - w\cos\varphi))\hat{i} + (-2\Omega u\sin\varphi)\hat{j} + (2\Omega u\cos\varphi)\hat{k}] \cdot (u\hat{i} + v\hat{j} + w\hat{k}) \\
&= (2\Omega(v\sin\varphi - w\cos\varphi))u + (-2\Omega u\sin\varphi)v + (2\Omega u\cos\varphi)w \\
&= 2\Omega uv\sin\varphi - 2\Omega uw\sin\varphi - 2\Omega uv\sin\varphi + 2\Omega uw\sin\varphi = 0
\end{align*}
Alternatively, note that $\overrightarrow{F_\text{cor}} = -2\overrightarrow{\Omega} \times \vec{v}$ and $({\Omega} \times \vec{v}) \cdot \vec{v} = 0$ always holds.
\end{Answer}
\chapter{Vector Geometry}

Vectors provides valuable assistance when it comes to describing geometric objects. In this chapter we are going to exploit the knowledge learnt in the previous chapters to solve geometry problems and inspect more deeply the intimate relationship between vectors, dot/cross products, and geometry.

\section{Lines and Planes}
\textit{(Straight) lines} and \textit{planes} are geometric shapes of importance in two/three-dimensional real spaces ($\mathbb{R}^2$ and $\mathbb{R}^3$) and due to their simplicity they will be frequently seen. They can be expressed either in terms of (a) an equation, and (b) vectors. We will start from the easier case of a line.\\
\\
Since a straight line is a one-dimensional object, the vector form of such a line can be expressed by a fixed vector that points to its initial position, plus another vector oriented along the line's direction, times an arbitrary parameter which controls its extension or contraction, so that it traces out the line when the parameter changed continuously.
\begin{center}
\begin{tikzpicture}
\draw[->] (-1,0)--(4,0) node[right]{$x$};
\draw[->] (0,-1)--(0,4) node[above]{$y$};
\draw[line width=1.5, orange, ->] (0,0)--(-1,0.5) node[above left, xshift=4]{$\vec{r} = (-1,\frac{1}{2})^T$};
\draw[blue] (-1.6, 0.2)--(4.4, 3.2) node[left, yshift=6]{$x - 2y = -2$};
\draw[Green!40, thick, ->] (0,0)--(0.5,1.25);
\draw[Green!50, thick, ->] (0,0)--(1,1.5);
\draw[Green!60, thick, ->] (0,0)--(1.5,1.75);
\draw[Green!70, thick, ->] (0,0)--(2,2);
\draw[Green!85, thick, ->] (0,0)--(2.5,2.25);
\draw[Green, thick, ->] (0,0)--(3,2.5);
\draw[line width=1.5, red, ->] (-1,0.5)--(1,1.5) node[above, pos=1.1, sloped]{$\vec{u} = (2,1)^T$};
\node[below left]{$O$}; 
\end{tikzpicture}\\
The graph of $\color{blue}x - 2y = -2$ can take the vector form of ${\color{Green}\overrightarrow{OP}} = {\color{orange}\vec{r}} + {\color{Green}t} {\color{red}\vec{u}} = {\color{orange}(-1,\frac{1}{2})^T} + {\color{Green}t}{\color{red}(2,1)^T}$. The orange/red arrow represents the initial position/direction, and the locus of green arrow is controlled by $\color{Green}t$ like a slider. The cases for $\color{Green}t = 0.75, 1, 1.25, 1.5, 1.75, 2$ are shown.
\end{center}
Short Exercise: Choose any value of $t$ and substitute that value into the expression of $\overrightarrow{OP}$ above to see if the $x$ and $y$-components satisfy the starting equation. Also, try to increase/decrease the value of $t$ to observe how the vector traces out the desired straight line.\footnote{Let's say $t = -0.25$, $\overrightarrow{OP}=(-1,0.5)^T + (-0.25)(2,1)^T = (-1.5, 0.25)^T$, $x - 2y = (-1.5) - 2(0.25) = -2$.}

\subsection{Translating Equation Form to Vector Form}
The general equation form of a line on an $x$-$y$ plane is $ax + by = h$, resembling a linear system of one equation with two unknowns. From Section \ref{subsection:SolLinSysGauss}, it can be observed that it has infinitely many solutions and possesses a free variable. Let $y = t$, then rearranging the equation we have $x = (h - bt)/a$ where $t$ is any scalar. Denote the origin as $O$ and any point on the line as $P$, then 
\begin{align*}
\overrightarrow{OP} =
\begin{bmatrix}
x \\
y
\end{bmatrix}
=
\begin{bmatrix}
\frac{h}{a} - \frac{b}{a}t\\
t
\end{bmatrix}
= 
\begin{bmatrix}
\frac{h}{a}\\
0
\end{bmatrix}
+ t
\begin{bmatrix}
-\frac{b}{a}\\
1
\end{bmatrix}
\end{align*}
This is one possible vector form (\textit{parameterization}) of the line. Its idea can be borrowed from Example \ref{exmp:mulsol}, with $(\frac{h}{a}, 0)^T$ being the initial position/particular solution, and $(-\frac{b}{a}, 1)^T$ as the direction of that line, multiplied by a free parameter (complementary solution) to complete the general solution. For example, if we have $3x - 2y = 5$, then by the same method, we get
\begin{align*}
\begin{bmatrix}
x \\
y
\end{bmatrix}
=
\begin{bmatrix}
\frac{5}{3} + \frac{2}{3}t\\
t
\end{bmatrix}
= 
\begin{bmatrix}
\frac{5}{3}\\
0
\end{bmatrix}
+ t
\begin{bmatrix}
\frac{2}{3}\\
1
\end{bmatrix}    
\end{align*}
Bear in mind that the direction vector (representing complementary solution) can be scaled freely. In addition, any initial position vector (particular solution) can be chosen as long as it links to a point on the line and satisfies the equation. (You may refer to the discussion about particular/complementary solution in Section \ref{subsection:SolLinSysGauss}.) Hence there is no unique vector form for a line. For instance,
\begin{align*}
\begin{bmatrix}
1\\
3
\end{bmatrix}
+ t_1
\begin{bmatrix}
2 \\
4
\end{bmatrix}     
\end{align*}
is equivalent to
\begin{align*}
\begin{bmatrix}
-1\\
-1
\end{bmatrix}
+ t_2
\begin{bmatrix}
1 \\
2
\end{bmatrix}     
\end{align*}
for the line equation $2x - y = -1$.\\
\\
Short Exercise: Check the equivalence of the two vector forms above by choosing a value for $t_1$ and finding the corresponding $t_2$ so that the vector points to the same position.\footnote{For example, if $t_1 = 1$, we have $(1, 3)^T + (1)(2, 4)^T = (3 ,7)^T$ as a point on the line, and for the another vector form $(-1,-1)^T + t_2(1, 2)^T = (3 ,7)^T$ to coincide we will have $t_2 = 4$. In this case, it can be shown that the general relation between the two forms is determined by $t_2 = 2t_1 + 2$, as
\begin{align*}
\begin{bmatrix}
1\\
3
\end{bmatrix}
+ t_1
\begin{bmatrix}
2 \\
4
\end{bmatrix}
=
\left(
\begin{bmatrix}
-1\\
-1
\end{bmatrix}
+
\begin{bmatrix}
2 \\
4
\end{bmatrix}
\right)
+ 2t_1
\begin{bmatrix}
1 \\
2
\end{bmatrix}
=
\begin{bmatrix}
-1\\
-1
\end{bmatrix}
+ (2t_1+2)
\begin{bmatrix}
1 \\
2
\end{bmatrix}
\end{align*}}\\
Short Exercise: What is the vector form of the equation $ax + by = h$ for the degenerate case $a=0$?\footnote{The equation is reduced to $y = \frac{h}{b}$ and we select $x = t$ as the free variable instead.\\ 
$\begin{bmatrix}
x \\
y
\end{bmatrix}
=
\begin{bmatrix}
t \\
\frac{h}{b}
\end{bmatrix}
=
\begin{bmatrix}
0 \\
\frac{h}{b}
\end{bmatrix}
+
t
\begin{bmatrix}
1 \\
0
\end{bmatrix}$}

\subsection{Recovering Equation Form from Vector Form} On the other hand, inferring line equation from the vector form is not straight-forward at first sight. Since the vector form of a line always contains an arbitrary parameter, which is absent in the equation form, the motivation is to remove the parameter through some manipulation.\\
\\
Remember that from Properties \ref{proper:dotorth} the dot product between orthogonal (perpendicular) vectors returns zero. This means that by carrying out dot product with the \index{Normal Vector}\keywordhl{normal vector} of the line which is orthogonal to the direction vector, on both sides of the vector form will eliminate the parameter and recover the line equation. For example, given that
\begin{align*}
\begin{bmatrix}
x\\
y
\end{bmatrix}
=
\begin{bmatrix}
1\\
3
\end{bmatrix}
+ t
\begin{bmatrix}
1 \\
4
\end{bmatrix} 
\end{align*}
We know that $(4, -1)^T$ is a normal vector orthogonal to the direction vector (see the next short exercise). So, by taking dot product with $(4, -1)^T$ on both sides, we have
\begin{align*}
\begin{bmatrix}
x\\
y
\end{bmatrix}
\cdot
\begin{bmatrix}
4\\
-1
\end{bmatrix}
&=
\left(\begin{bmatrix}
1\\
3
\end{bmatrix}
+ t
\begin{bmatrix}
1 \\
4
\end{bmatrix}\right)
\cdot
\begin{bmatrix}
4\\
-1
\end{bmatrix}
=
\begin{bmatrix}
1\\
3
\end{bmatrix}
\cdot
\begin{bmatrix}
4\\
-1
\end{bmatrix}
+ t
\begin{bmatrix}
1 \\
4
\end{bmatrix} 
\cdot
\begin{bmatrix}
4\\
-1
\end{bmatrix}\\
4x - y &= ((1)(4) + (3)(-1)) + ((1)(4)+(4)(-1))t = 1 + 0t = 1 \\
\Rightarrow 4x - y &= 1
\end{align*}
Notice that the coefficients of the equation are the same as the components of the normal vector.\\
\\
Short Exercise: Verify that $(a, b)$ is always orthogonal to $(b, -a)$, and vice versa.\footnote{$(a, b)^T \cdot (b, -a)^T = (a)(b) + (b)(-a) = 0$}

\subsection{Generalizing to Higher Dimensions}
\label{section:vecgeohighdim}
Similar concepts can be applied on the equation and vector form for planes. General form of equation of a plane in three-dimensional space is $ax + by + cz = h$, which is a linear system of one equation with three unknowns. From the analysis in Section \ref{subsection:SolLinSysGauss} we know there are two free variables and two (non-parallel) direction vectors for such a plane. By assigning any two (non-pivotal) unknowns as the free variables, we then obtain the vector form of the plane. 
\par
Recall from Section \ref{section:crossprod}, the cross product of any two non-parallel vectors on the plane will give a third vector normal to the plane. Subsequently, we can take the dot product with this newly obtained normal vector to convert the vector form back to a plane equation just like what we have done for lines in the last subsection. Again, the coefficients of the plane equation match the components of the normal vector, differed at most by a multiplicative factor.
\begin{center}
\begin{tikzpicture}[x={(-0.4cm, -0.7cm)}, y={(0.7cm, -0.2cm)}, z={(0cm, 0.8cm)}]
\node[below right]{$O$}; 
\draw[thick,->] (0,0,0) -- (4,0,0) node[anchor=north east]{$x$};
\draw[thick,->] (0,0,0) -- (0,4,0) node[anchor=north west]{$y$};
\draw[thick,->] (0,0,0) -- (0,0,4) node[anchor=south east]{$z$};
\filldraw[draw=Green, fill=Green!20, opacity=0.6]
(3,0,0) -- (0,3/2,0) -- (0,0,1) -- cycle;
\node[Green] at (2,3) {$x + 2y + 3z = 3$};
\draw[very thick,->,blue] (0,0,1) -- (1,0,2/3) node[below left]{$(1, 0, -\frac{1}{3})^T$};
\draw[very thick,->,blue] (0,0,1) -- (0,1,1/3) node[above right]{$(0, 1, -\frac{2}{3})^T$};
\draw[very thick,->,red] (0,0,1) -- (1,2,4) node[above right]{Normal Vector $= (\frac{1}{3}, \frac{2}{3}, 1)^T$};
\end{tikzpicture} 
\end{center}
The plane represented by the equation $x + 2y + 3z = 3$. Notice that the normal vector can be found via computing $(1, 0, -\frac{1}{3})^T \times (0, 1, -\frac{2}{3})^T = (\frac{1}{3}, \frac{2}{3}, 1)^T$. The normal vector is magnified for the purpose of illustration.

\begin{exmp}
Transform the plane equation $2x+3y+z = 4$ to vector form and convert the acquired vector form back to the starting equation to check consistency.
\end{exmp}
\begin{solution}
For the first part, we can let $y=s$, $z=t$, then from the plane equation we have $x = \frac{1}{2}(4-3s-t)$ and hence
\begin{align*}
\begin{bmatrix}
x \\
y \\
z
\end{bmatrix}
=
\begin{bmatrix}
\frac{1}{2}(4-3s-t) \\
s \\
t
\end{bmatrix}
=
\begin{bmatrix}
2 \\
0 \\
0
\end{bmatrix}
+ s
\begin{bmatrix}
-\frac{3}{2} \\
1 \\
0
\end{bmatrix}
+ t
\begin{bmatrix}
-\frac{1}{2} \\
0 \\
1
\end{bmatrix}
\end{align*}
where $-\infty < s < \infty$, $-\infty < t < \infty$ are the two free parameters. To recover the original equation, we can find the normal vector by doing cross product on the two direction vectors obtained above. By Properties \ref{proper:crossdet}, we can acquire a normal vector of
\begin{align*}
\begin{bmatrix}
-\frac{3}{2} \\
1 \\
0
\end{bmatrix}
\times
\begin{bmatrix}
-\frac{1}{2} \\
0 \\
1
\end{bmatrix}
=
\begin{vmatrix}
\hat{i} & \hat{j} & \hat{k} \\
-\frac{3}{2} & 1 & 0 \\
-\frac{1}{2} & 0 & 1
\end{vmatrix}
= \hat{i} + \frac{3}{2}\hat{j} + \frac{1}{2}\hat{k}
\end{align*}
The next step is to take the dot product on both sides of the vector equation with the normal vector just retrieved.
\begin{align*}
\begin{bmatrix}
x \\
y \\
z
\end{bmatrix}
&=
\begin{bmatrix}
2 \\
0 \\
0
\end{bmatrix}
+ s
\begin{bmatrix}
-\frac{3}{2} \\
1 \\
0
\end{bmatrix}
+ t
\begin{bmatrix}
-\frac{1}{2} \\
0 \\
1
\end{bmatrix} \\
\begin{bmatrix}
x \\
y \\
z
\end{bmatrix}
\cdot
\begin{bmatrix}
1 \\
\frac{3}{2} \\
\frac{1}{2}
\end{bmatrix}
&=
\begin{bmatrix}
2 \\
0 \\
0
\end{bmatrix}
\cdot
\begin{bmatrix}
1 \\
\frac{3}{2} \\
\frac{1}{2}
\end{bmatrix}
+ s
\begin{bmatrix}
-\frac{3}{2} \\
1 \\
0
\end{bmatrix}
\cdot
\begin{bmatrix}
1 \\
\frac{3}{2} \\
\frac{1}{2}
\end{bmatrix}
+ t
\begin{bmatrix}
-\frac{1}{2} \\
0 \\
1
\end{bmatrix}
\cdot
\begin{bmatrix}
1 \\
\frac{3}{2} \\
\frac{1}{2}
\end{bmatrix} \\
x + \frac{3}{2}y + \frac{1}{2}z &= 2 + s(0) + t(0) = 2 \\
\rightarrow 2x+3y+z &= 4
\end{align*}
\end{solution}
The correspondence between the coefficients of a linear equation and components of its normal vector is not a coincidence. In fact, even for higher dimensional cases, where there is no intuitive geometric interpretation, it is still true.
\begin{proper}
\label{proper:normalhyperplane}
An equation of the form $a_1x_1 + a_2x_2 + a_3x_3 + \cdots + a_nx_n = h$ in $\mathbb{R}^n$ has a normal vector of $(a_1, a_2, a_3, \ldots, a_n)^T$.
\end{proper}
The procedures carried in the last example can be similarly applied to higher dimensional situations where the equation now represents a \index{Hyperplane}\keywordhl{hyperplane}\footnote{A hyperplane in the $n$-dimensional real space $\mathbb{R}^n$ can be think of as an "$(n-1)$-dimensional flat surface".}.

\section{Further Geometric Applications of Dot Product}
\subsection{Projection}
We have mentioned in Properties \ref{proper:dotgeo} that dot product between two vectors is related to the projection of one vector onto another. By rearranging the formula of Properties \ref{proper:dotgeo}, we can derive the length of projection as follows.
\begin{center}
\begin{tikzpicture}[scale=1.3]
\coordinate (0) at (0,0);
\coordinate (vecu) at (4,1);
\coordinate (vecv) at (1,2);
\draw[->](0)--(vecu) node[right]{$\vec{u}$};
\draw[->](0)--(vecv) node[above]{$\vec{v}$};
\draw[dashed] (1,2)--(24/17, 6/17);
\draw[red] (24/17+0.2, 6/17+0.05)--(24/17+0.15, 6/17+0.25)--(24/17-0.05, 6/17+0.2);
\pic[draw, "$\theta$", angle eccentricity=1.5] {angle = vecu--0--vecv};
\draw[blue, very thick] (0,0)--(24/17, 6/17) node[below, shift={(0mm, -2mm)}]{$\norm{\vec{v}}\cos\theta$};
\end{tikzpicture}
\end{center}
\begin{proper}
\label{proper:proj}
For two real vectors $\vec{u}$ and $\vec{v}$ having the same dimension, the \index{Scalar Projection}\keywordhl{(signed) scalar projection} of $\vec{v}$ onto $\vec{u}$ is computed according to
\begin{align*}
\text{proj}_u v = \norm{\vec{v}} \cos\theta = \frac{\vec{v} \cdot \vec{u}}{\norm{\vec{u}}}    
\end{align*}
If we want to give directionality to the projection, then we can supply its unit vector $\hat{u}$ to make it a \index{Vector Projection}\keywordhl{vector projection}:
\begin{align*}
\overrightarrow{\text{proj}}_u v &= (\text{proj}_u v) \hat{u} = \frac{\vec{v} \cdot \vec{u}} {\norm{\vec{u}}} \hat{u} = \frac{\vec{v} \cdot \vec{u}}{\norm{\vec{u}}^2} \vec{u} \\
&= (\text{proj}_u v) \frac{\vec{u}}{\norm{\vec{u}}}
\end{align*}
where we have used Definition \ref{defn:unitvec} to write out the unit vector $\hat{u}$ as $\frac{\vec{u}}{\norm{\vec{u}}}$.
\end{proper}

\begin{exmp}
\label{exmp:projectuv}
Find the projection of $\vec{v} = -2\hat{i} + 3\hat{j} - \hat{k}$ onto $\vec{u} = 4\hat{i} + \hat{j} - 3\hat{k}$.
\end{exmp}
\begin{solution}
According to Properties \ref{proper:proj}, The signed scalar projection of $\vec{v}$ into $\vec{u}$ is
\begin{align*}
\text{proj}_u v &= \frac{\vec{v} \cdot \vec{u}}{\norm{\vec{u}}}  \\
&= \frac{(-2)(4)+(3)(1)+(-1)(-3)}{\sqrt{(4)^2+(1)^2+(-3)^2}} \\
&= -\frac{2}{\sqrt{26}} = -\frac{\sqrt{26}}{13}
\end{align*}
and the vector projection is 
\begin{align*}
\overrightarrow{\text{proj}}_u v &= (\text{proj}_u v) \frac{\vec{u}}{\norm{\vec{u}}} \\
&= (-\frac{\sqrt{26}}{13}) \left(\frac{4\hat{i} + \hat{j} - 3\hat{k}}{\sqrt{26}}\right) \\
&= -\frac{1}{13} (4\hat{i} + \hat{j} - 3\hat{k}) = (-\frac{4}{13}, -\frac{1}{13}, \frac{3}{13})^T
\end{align*}   
\end{solution}

\subsection{Distance} Distance of a point to a line/plane (in $\mathbb{R}^2$/$\mathbb{R}^3$ respectively) can be found by projecting any vector starting somewhere from the line/plane to the point, onto the normal vector of that line/plane, as illustrated in the figures below.
\begin{center}
\begin{tikzpicture}
\draw (0,0) -- (5,-1);
\draw[red,->] (2.5,-0.5) -- (2.9, 1.5) node[right]{Normal Vector $\vec{N}$};
\coordinate[label = {[Green]left:$P$}] (P) at (1,1);
\node at (P) [circle,fill,inner sep=1pt,Green]{};
\draw[thick,->,Green] (2.5,-0.5) -- (P);
\draw[dashed] (10/13, -2/13) -- (P);
\draw[dashed] (71/26, 17/26) -- (P);
\draw[line width=2pt, blue] (2.5,-0.5) -- (71/26, 17/26) node[below right]{Distance = Projection of $\vec{P}$ onto $\vec{N}$};
\end{tikzpicture}\\
\begin{tikzpicture}[x={(-0.2cm, -0.7cm)}, y={(0.7cm, -0.2cm)}, z={(-0.3cm, 0.6cm)}]
\filldraw[draw=SkyBlue, fill=SkyBlue!20]
(0,0,0) -- (2.5,-1.5,0) -- (3,3,0) -- (-1.5,2.5,0) -- cycle;
\draw[thick,->,red] (1.5,1,0) -- (1.5,1,4) node[left]{Normal Vector $\vec{N}$};
\coordinate[label = {[Green]right:$P$}] (P) at (0,2,2.5);
\node at (P) [circle,fill,inner sep=1pt,Green]{};
\draw[thick,->,Green] (1.5,1,0) -- (P);
\draw[dashed] (1.5,1,2.5) -- (P);
\draw[dashed] (0,2,0) -- (P);
\draw (0,2.2,0) -- (0,1.8,0);
\draw (0.2,2,0) -- (-0.2,2,0);
\draw[line width=2pt, blue] (1.5,1,0) -- (1.5,1,2.5) node[below left]{Distance = Projection of $\vec{P}$ onto $\vec{N}$};
\end{tikzpicture}
\end{center}

\begin{exmp}
Find the distance from the plane $x-2y+3z = 6$ to the point $(3,3,6)^T$.
\end{exmp}
\begin{solution}
From the equation of the plane, and by Properties \ref{proper:normalhyperplane}, it can be inferred that the normal vector of the plane is $\hat{i} - 2\hat{j} + 3\hat{k}$. We can select any point on the plane as we wish, let's say $(4,2,2)^T$, and the vector from such a point to the point $(3,3,6)^T$ in question is simply their difference $(3,3,6)^T - (4,2,2)^T = -\hat{i} + \hat{j} + 4\hat{k}$. Subsequently, the distance is found from the length of the projection of this vector $-\hat{i} + \hat{j} + 4\hat{k}$ onto the normal vector of the plane $\hat{i} - 2\hat{j} + 3\hat{k}$. By Properties \ref{proper:proj}, it is
\begin{align*}
\frac{(-\hat{i} + \hat{j} + 4\hat{k}) \cdot (\hat{i} - 2\hat{j} + 3\hat{k})}{\norm{\hat{i} - 2\hat{j} + 3\hat{k}}} = \frac{(-1)(1)+(1)(-2)+(4)(3)}{\sqrt{(1)^2 + (-2)^2 + (3)^2}} = \frac{9}{\sqrt{14}}
\end{align*}
\end{solution}
Sometimes the calculation may lead to a negative value for the projection and we may want to take the absolute value. The case of finding the distance of a point to a line of $\mathbb{R}^3$ is considered in Exercise \ref{ex:dist_pt_line_R3}.

\section{Further Geometric Applications of Cross Product}
Unless specified, all vectors in this section is assumed to be of $\mathbb{R}^3$.
\subsection{Area}
The area of the parallelogram formed by two vectors $\vec{u}$, $\vec{v}$ are simply the absolute value of their cross product.
\begin{center}
\begin{tikzpicture}
\coordinate (0) at (0,0);
\coordinate (vecu) at (4,0);
\coordinate (vecv) at (1,3);
\draw[thick, ->] (0)--(vecu) node[right]{$\vec{u}$};
\draw[thick, ->] (0)--(vecv) node[above left]{$\vec{v}$};
\draw[thick, ->, dashed, gray] (4,0)--(5,3);
\draw[thick, ->, dashed, gray] (1,3)--(5,3);
\draw[dashed, Green] (1,3)--(4,0);
\draw[dashed, blue] (1,3)--(1,0) node[pos=0.75, right]{$\norm{\vec{v}}\sin\theta$};
\draw[red] (1+0.2, 0)--(1+0.2, 0+0.2)--(1, 0+0.2);
\pic[draw, "$\theta$",angle eccentricity=1.5] {angle = vecu--0--vecv};
\end{tikzpicture}
\end{center}
\begin{proper}
\label{proper:areaparallelogram}
Directly from Properties \ref{proper:crossgeo}, the area of the parallelogram formed by two vectors $\vec{u}$ and $\vec{v}$ is
\begin{align*}
\norm{\vec{u} \times \vec{v}} = \norm{\vec{u}}\norm{\vec{v}}\sin\theta
\end{align*}
Similarly, the area of triangle made by $\vec{u}$ and $\vec{v}$ is half of the above:
\begin{align*}
\frac{1}{2}\norm{\vec{u} \times \vec{v}} = \frac{1}{2}\norm{\vec{u}}\norm{\vec{v}}\sin\theta   
\end{align*}
\end{proper}

\begin{exmp}
Find the area of the parallelogram formed by $\vec{u} = (-1, -2, 4)^T$ and $\vec{v} = (3, 0, 1)^T$.
\end{exmp}
\begin{solution}
By Properties \ref{proper:crossdet}, the cross product between the two given vectors is
\begin{align*}
\vec{u} \times \vec{v} &=
\begin{vmatrix}
\hat{i} & \hat{j} & \hat{k} \\
-1 & -2 & 4 \\
3 & 0 & 1
\end{vmatrix} \\
&= -2\hat{i} + 13\hat{j} + 6\hat{k}
\end{align*}
Therefore, as suggested by Properties \ref{proper:areaparallelogram}, the required area is
\begin{align*}
\norm{\vec{u} \times \vec{v}} &= \sqrt{(-2)^2 + (13)^2 + (6)^2} \\
&= \sqrt{209}
\end{align*}
\end{solution}

\subsection{Volume}
Meanwhile, volume of parallelepiped (see the figure below) formed by three vectors $\vec{u}$, $\vec{v}$, $\vec{w}$ is given by the absolute value of the so-called \index{Scalar Triple Product}\keywordhl{scalar triple product} as follows.
\begin{center}
\begin{tikzpicture}
\coordinate (0) at (0,0,0);
\fill[fill=orange!20, opacity=0.5] (0) -- (4,0,0) -- (5,0,-2) -- (1,0,-2) -- cycle;
\draw[thick, ->] (0)--(4,0,0) node[right](vecu){$\vec{u}$};
\draw[thick, ->] (0)--(1,0,-2) node[below right](vecv){$\vec{v}$};
\draw[thick, ->] (0)--(1,3,-1) node[above](vecw){$\vec{w}$};
\draw[thick, ->, gray, dashed] (4,0,0)--(5,0,-2);
\draw[thick, ->, gray, dashed] (1,0,-2)--(2,3,-3);
\draw[thick, ->, gray, dashed] (1,3,-1)--(5,3,-1);
\draw[thick, ->, gray, dashed] (4,0,0)--(5,3,-1);
\draw[thick, ->, gray, dashed] (1,0,-2)--(5,0,-2);
\draw[thick, ->, gray, dashed] (1,3,-1)--(2,3,-3);
\draw[thick, ->, gray, dashed] (5,0,-2)--(6,3,-3);
\draw[thick, ->, gray, dashed] (5,3,-1)--(6,3,-3);
\draw[thick, ->, gray, dashed] (2,3,-3)--(6,3,-3);
\draw[blue, dashed] (1,3,-1)--(1,0,-1) node[midway, right]{$\norm{\vec{w}}\cos\theta$};
\draw[blue] (0.9,-0.1,-1) -- (1.1,0.1,-1);
\draw[blue] (0.9,0.1,-1) -- (1.1,-0.1,-1);
\draw[thick, ->, red] (0,0,0) -- (0,3,0) node[above](vecn){$\vec{u}\times\vec{v}$};
\pic[draw, "$\theta$",angle eccentricity=1.75] {angle = vecw--0--vecn};
\draw[orange, <-] (3,0,-1) to [in=190, out=70] (5,2,-1) node[right]{Base Area $= \norm{\vec{u}\times\vec{v}}$};
\end{tikzpicture}
\end{center}
\begin{proper}[Scalar Triple Product]
\label{proper:parallelpiped}
The volume of parallelepiped formed by three vectors $\vec{u}$, $\vec{v}$, and $\vec{w}$ is calculated as
\begin{align*}
\norm{\vec{u} \times \vec{v}} \norm{\vec{w}} \cos\theta = |(\vec{u} \times \vec{v}) \cdot \vec{w}| =
\text{abs}\left(
\begin{vmatrix}
u_1 & u_2 & u_3 \\
v_1 & v_2 & v_3 \\
w_1 & w_2 & w_3
\end{vmatrix}\right)
\end{align*}
where
\begin{align*}
(\vec{u} \times \vec{v}) \cdot \vec{w} =
\begin{vmatrix}
u_1 & u_2 & u_3 \\
v_1 & v_2 & v_3 \\
w_1 & w_2 & w_3
\end{vmatrix}    
\end{align*}
is the scalar triple product of $\vec{u}$, $\vec{v}$, and $\vec{w}$. Also, by applying Properties \ref{proper:elementaryopdet}, the determinant form of scalar triple product indicates that
\begin{align*}
&\quad (\vec{u} \times \vec{v}) \cdot \vec{w} = (\vec{v} \times \vec{w}) \cdot \vec{u} = (\vec{w} \times \vec{u}) \cdot \vec{v} \\
&= -(\vec{v} \times \vec{u}) \cdot \vec{w} = -(\vec{w} \times \vec{v}) \cdot \vec{u} = -(\vec{u} \times \vec{w}) \cdot \vec{v}
\end{align*}
\end{proper}
\begin{proof}
We will prove the determinant formula shown above for $(\vec{u} \times \vec{v}) \cdot \vec{w}$ briefly. By Properties \ref{proper:crossdet}, we have
\begin{align*}
\vec{u} \times \vec{v} &= (u_2v_3 - u_3v_2)\hat{i} + (u_3v_1 - u_1v_3)\hat{j} + (u_1v_2 - u_2v_1)\hat{k}     
\end{align*}
and then according to Definition \ref{defn:dotreal}
\begin{align*}
(\vec{u} \times \vec{v}) \cdot \vec{w} &= (u_2v_3 - u_3v_2, u_3v_1 - u_1v_3, u_1v_2 - u_2v_1)^T \cdot (w_1, w_2, w_3)^T \\ 
&= (u_2v_3 - u_3v_2)(w_1) + (u_3v_1 - u_1v_3)(w_2) + (u_1v_2 - u_2v_1)(w_3) 
\end{align*}
which is equal to
\begin{align*}
\begin{vmatrix}
u_1 & u_2 & u_3 \\
v_1 & v_2 & v_3 \\
w_1 & w_2 & w_3
\end{vmatrix}
= w_1(u_2v_3 - u_3v_2) - w_2(u_1v_3 - u_3v_1) + w_3(u_1v_2 - u_2v_1) 
\end{align*}
where we do cofactor expansion (Properties \ref{proper:cofactorex}) along the third row of the determinant.
\end{proof}
If the volume of parallelepiped evaluated from the scalar triple product is zero, it implies that the three vectors involved are \index{Co-planar}\keywordhl{co-planar}, i.e.\ lying on the same plane.
\begin{proper}
Given three vectors $\vec{u}$, $\vec{v}$, and $\vec{w}$, if their scalar triple product $(\vec{u} \times \vec{v}) \cdot \vec{w} = 0$ equals to zero, then $\vec{u}$, $\vec{v}$, and $\vec{w}$ are co-planar and lie on the same plane, and vice versa.
\end{proper}
Note that if $\vec{w} = \alpha \vec{u} + \beta \vec{v}$, where $\alpha$ and $\beta$ are some scalars, then $\vec{u}$, $\vec{v}$, $\vec{w}$ are co-planar, and $(\vec{u} \times \vec{v}) \cdot \vec{w} = (\vec{u} \times \vec{v}) \cdot (\alpha \vec{u} + \beta \vec{v}) = \alpha ((\vec{u} \times \vec{v}) \cdot \vec{u}) + \beta ((\vec{u} \times \vec{v}) \cdot \vec{v}) = \alpha(0) + \beta(0) = 0$ as both $\vec{u} \cdot (\vec{u} \times \vec{v})$ and $\vec{v} \cdot (\vec{u} \times \vec{v})$ equal to zero.

\begin{exmp}
Find the volume of the parallelepiped formed by $\vec{u} = (1,-2,2)^T$, $\vec{v}=(-1,-1,1)^T$ and $\vec{w}=(2,1,0)^T$.
\end{exmp}
\begin{solution}
By Properties \ref{proper:parallelpiped}, the triple scalar product of the three given vectors is
\begin{align*}
(\vec{u} \times \vec{v}) \cdot \vec{w} &=
\begin{vmatrix}
1 & -2 & 2 \\
-1 & -1 & 1 \\
2 & 1 & 0
\end{vmatrix}
= -3
\end{align*}
and the volume is $\abs{-3} = 3$.
\end{solution}

\subsubsection{Generalization to other dimensions}
Given that the volume of parallelepiped formed by three vectors is equal to the absolute value of the corresponding matrix determinant as derived above, it is natural to ask if similar results hold for other numbers of dimension. In fact, Properties \ref{proper:parallelpiped} can be generalized to include length, area and the so-called \textit{$n$-volume} (Volume equivalent of $n$ vectors in the $n$-dimensional space).
\begin{proper}
\label{proper:nvolume}
For $n$ vectors of $\mathbb{R}^n$, their $n$-volume is the absolute value of the determinant of matrix formed by these column (or row) vectors. When $n=1,2,3$, the $n$-volume corresponds to the usual notions of length, area and volume.
\end{proper}
We can check the legitimacy of the last sentence in Properties \ref{proper:nvolume} by noticing it is consistent with Properties \ref{proper:areaparallelogram} about area of two vectors on the $x$-$y$ plane. Given $\vec{u} = (u_1, u_2)^T$ and $\vec{v} = (v_1, v_2)^T$, by Properties \ref{proper:nvolume} the area of the parallelogram formed by them is
\begin{align*}
\begin{vmatrix}
u_1 & u_2 \\
v_1 & v_2
\end{vmatrix} = u_1v_2 - v_1u_2
\end{align*}
Alternatively, we can treat $\vec{u}$ and $\vec{v}$ as two three-dimensional vectors $(u_1, u_2, 0)^T$ and $(v_1, v_2, 0)^T$ such that they have a zero $z$-component and remain lying on the $x$-$y$ plane. Then according to the previous Properties \ref{proper:areaparallelogram}, the area is computed by $\norm{\vec{u} \times \vec{v}}$, where by Properties \ref{proper:crossdet},
\begin{align*}
\vec{u} \times \vec{v} &= 
\begin{vmatrix}
\hat{i} & \hat{j} & \hat{k} \\
u_1 & u_2 & 0 \\
v_1 & v_2 & 0
\end{vmatrix} \\
&= \hat{k}\begin{vmatrix}
u_1 & u_2 \\
v_1 & v_2
\end{vmatrix} \text{ (Cofactor Expansion along the third row)}\\
&= (u_1v_2 - v_1u_2)\hat{k} = (0,0,u_1v_2 - v_1u_2)^T
\end{align*}
Hence $\norm{\vec{u} \times \vec{v}} = \sqrt{(0)^2 + (0)^2 + (u_1v_2 - v_1u_2)^2} = u_1v_2 - v_1u_2$, which coincides with the expression we just derived from Properties \ref{proper:nvolume}.

\subsubsection{Remarks}
The solution of a linear system can be considered as a point/line/plane/hyperplane too, depending on the number of free variables and thus direction vectors in the complementary part ($0$/$1$/$2$ or more). We may also like to call it a \textit{solution space}. However, while such shapes surely occupy space geometrically, we have been shying away from defining what really means by a \textit{vector space} mathematically, which will be the main point of discussion in the next chapter.

\section{Useful Vector Identities}
In this section, we will prove some key vector identities that may be of utilities to some readers.
\begin{proper}[Vector Triple Product]
\label{proper:triplecross}
The \index{Vector Triple Product}\keywordhl{vector triple product} of three vectors $\vec{u}$, $\vec{v}$, $\vec{w}$ is defined as
\begin{align*}
\vec{u} \times (\vec{v} \times \vec{w}) = (\vec{u} \cdot \vec{w})\vec{v} - (\vec{u} \cdot \vec{v})\vec{w}
\end{align*}
\end{proper}
\begin{proof}
By Properties \ref{proper:crossdet}, the L.H.S. can be expanded into
\begin{align*}
&\quad\vec{u} \times (\vec{v} \times \vec{w}) \\
&= (u_1\hat{i} + u_2\hat{j} + u_3\hat{k}) \\
&\quad \times [(v_2w_3 - v_3w_2)\hat{i} + (v_3w_1 - v_1w_3)\hat{j} + (v_1w_2 - v_2w_1)\hat{k}] \\
&= 
\begin{vmatrix}
\hat{i} & \hat{j} & \hat{k} \\
u_1 & u_2 & u_3 \\
v_2w_3 - v_3w_2 & v_3w_1 - v_1w_3 & v_1w_2 - v_2w_1 
\end{vmatrix}
\end{align*}
The $\hat{i}$ component along the $x$-direction is
\begin{align*}
&\quad u_2(v_1w_2 - v_2w_1) - u_3(v_3w_1 - v_1w_3) \\
&= u_2w_2v_1 + u_3w_3v_1 - u_2v_2w_1 - u_3v_3w_1 \\
&= u_1w_1v_1 + u_2w_2v_1 + u_3w_3v_1 - u_1v_1w_1 - u_2v_2w_1 - u_3v_3w_1 \\
&= (u_1w_1 + u_2w_2 + u_3w_3)v_1 - (u_1v_1 + u_2v_2 + u_3v_3)w_1 \\
&= (\vec{u} \cdot \vec{w})v_1 - (\vec{u} \cdot \vec{v})w_1
\end{align*}
which is equal to the $\hat{i}$ component on the R.H.S. and similar results can be shown for the $\hat{j}$, $\hat{k}$ components and the equality establishes.
\end{proof}

\begin{proper}[Jacobi Identity]
\begin{align*}
\vec{u} \times (\vec{v} \times \vec{w}) + \vec{v} \times (\vec{w} \times \vec{u}) + \vec{w} \times (\vec{u} \times \vec{v}) = \textbf{0}    
\end{align*}
\end{proper}
\begin{proof}
By Properties \ref{proper:triplecross}, we have
\begin{align*}
&\quad \vec{u} \times (\vec{v} \times \vec{w}) + \vec{v} \times (\vec{w} \times \vec{u}) + \vec{w} \times (\vec{u} \times \vec{v}) \\
&= [(\vec{u} \cdot \vec{w})\vec{v} - (\vec{u} \cdot \vec{v})\vec{w}] \\
&\quad + [(\vec{v} \cdot \vec{u})\vec{w} - (\vec{v} \cdot \vec{w})\vec{u}] \\
&\quad + [(\vec{w} \cdot \vec{v})\vec{u} - (\vec{w} \cdot \vec{u})\vec{v}] \\
&= [(\vec{u} \cdot \vec{w})\vec{v} - (\vec{w} \cdot \vec{u})\vec{v}] \\
&\quad + [(\vec{v} \cdot \vec{u})\vec{w} - (\vec{u} \cdot \vec{v})\vec{w}] \\
&\quad + [(\vec{w} \cdot \vec{v})\vec{u} - (\vec{v} \cdot \vec{w})\vec{u}] \\
&= 0\vec{v} + 0\vec{w} + 0\vec{u} = \textbf{0}
\end{align*}
\end{proof}

\begin{proper}[Lagrange's Identity]
\begin{align*}
\norm{\vec{u} \times \vec{v}}^2 = \norm{\vec{u}}^2 \norm{\vec{v}}^2 - (\vec{u} \cdot \vec{v})^2
\end{align*}
\end{proper}
\begin{proof}
Manipulating the geometric formulae of dot/cross product, we have
\begin{align*}
\norm{\vec{u} \times \vec{v}}^2 &= \norm{\vec{u}}^2 \norm{\vec{v}}^2 \sin^2 \theta  &\text{(Properties \ref{proper:crossgeo})} \\
&= \norm{\vec{u}}^2 \norm{\vec{v}}^2 (1 - \cos^2 \theta) \\
&= \norm{\vec{u}}^2 \norm{\vec{v}}^2 - \norm{\vec{u}}^2 \norm{\vec{v}}^2 \cos^2 \theta \\
&= \norm{\vec{u}}^2 \norm{\vec{v}}^2 - (\vec{u} \cdot \vec{v})^2 &\text{(Properties \ref{proper:dotgeo})}
\end{align*}
\end{proof}

The last identity is the \index{Cosine Law!Cosine Law for Spherical Trigonometry}\keywordhl{Cosine Law for Spherical Trigonometry}.
\begin{proper}[Cosine Law for Spherical Trigonometry]
\label{proper:cosinesphere}
\begin{align*}
\cos c = \cos a \cos b + \sin a \sin b \cos C    
\end{align*}
where $a$, $b$, $c$ are the (subtended angle of) three arcs (in radians) of a spherical triangle on a unit sphere and $C$ is the angle between the two arcs $a$ and $b$, as shown in Figure \ref{fig:cosinesphere}.
\end{proper}
\begin{figure}
\centering\tdplotsetmaincoords{35}{125}
% https://tex.stackexchange.com/questions/380316/gray-shaded-sphere-with-tikz-3dplot
% https://latexdraw.com/draw-a-sphere-in-latex-using-tikz/#t-1610955504379
\begin{tikzpicture}[scale=2.5,tdplot_main_coords]
    % spherical background
    \draw [ball color=white,very thin] (0,0,0) circle (1cm) ;
    % equator
    \coordinate (0) at (0,0,0);
    % spherical triangle
    \begin{scope}[rotate=15]
    \tdplotdefinepoints(0,0,1)(0,0.1,0.99)(0.1,0,0.99)
    \tdplotdrawpolytopearc[thick, purple]{0.15}{below, purple}{$C$}
    \tdplotdefinepoints(0,0,0)(0,0,1)(3^0.5/2,0,0.5)
    \tdplotdrawpolytopearc[thick, red]{1}{left, red}{$a$}
    \tdplotdefinepoints(0,0,0)(0,0,1)(0,0.8,0.6)
    \tdplotdrawpolytopearc[thick, blue]{1}{above, blue}{$b$}
    \tdplotdefinepoints(0,0,0)(0,0.8,0.6)(3^0.5/2,0,0.5)
    \tdplotdrawpolytopearc[thick, Green]{1}{below, Green}{$c$}
    \draw[dashed, color=black!60] (0,0,0) -- (0,0,1) node(C){};
    \draw[dashed, color=black!60] (0,0,0) -- (3^0.5/2,0,0.5) node(B){};
    \draw[dashed, color=black!60] (0,0,0) -- (0,0.8,0.6) node(A){};
    \end{scope}
\end{tikzpicture}
\begin{tikzpicture}
    \draw[line width=1.5, ->] (0,2) -- (2,2) node[midway,above]{Rotation};
    \node at (0,0){};
\end{tikzpicture}
\tdplotsetmaincoords{80}{115}
% https://tex.stackexchange.com/questions/380316/gray-shaded-sphere-with-tikz-3dplot
\begin{tikzpicture}[scale=2.5,tdplot_main_coords]
    % spherical background
    \draw [ball color=white,very thin] (0,0,0) circle (1cm) ;
    % equator
    \coordinate (0) at (0,0,0);
    % spherical triangle
    \tdplotsetrotatedcoords{90}{90}{90};
    \draw[orange, dashed,
        tdplot_rotated_coords] (0,1,0) arc (90:252:1) node[orange, pos=0.84, right, align=left, font=\scriptsize]{Prime \\ Meridian \\};
    \tdplotdefinepoints(0,0,0)(0,0,1)(0,0.6,-0.8)
    \tdplotdrawpolytopearc[dashed, color=white]{1}{}{}
    \draw[dashed, ->, color=black!60] (0,0,0) -- (0,0,1) node(C){};
    \draw[dashed, ->, color=black!60] (0,0,0) -- (3^0.5/2,0,0.5) node(B)[black, font=\tiny, pos=0.75, sloped, below]{$\hat{v} = (\sin a, 0, \cos a)$};
    \draw[dashed, ->, color=black!60] (0,0,0) -- (0,0.8,0.6) node(A)[black, font=\tiny, pos=0.35, sloped, below]{$\hat{w} = (\sin b \cos C, \sin b \sin C, \cos b)$};
    \tdplotdefinepoints(0,0,0)(0,0,1)(3^0.5/2,0,0.5)
    \tdplotdrawpolytopearc[thick, red]{1}{left, red}{$a$}
    \tdplotdefinepoints(0,0,0)(0,0,1)(0,0.8,0.6)
    \tdplotdrawpolytopearc[thick, blue]{1}{below, blue}{$b$}
    \tdplotdefinepoints(0,0,0)(0,0.8,0.6)(3^0.5/2,0,0.5)
    \tdplotdrawpolytopearc[thick, Green]{1}{below, Green}{$c$}
    \node at (0,0.04,0.88)[purple]{$C$};
    \node at (0,0,1.15)[align=left, font=\scriptsize]{North Pole \\ $\hat{u} = (0,0,1)$};
\end{tikzpicture}
\caption{The spherical triangle on a unit sphere as described in Properties \ref{proper:cosinesphere}.}
\label{fig:cosinesphere}
\end{figure}
\begin{proof}
For the given spherical triangle, we can always rotate the coordinate system (see Figure \ref{fig:cosinesphere}) while keeping its shape intact, such that the corner $C$ is positioned exactly at the north pole ($\hat{u} = (0,0,1)^T$) and one of the two arcs starting from corner $C$ (let's say $a$) lies along the Prime Meridian (angle from the $x$-axis is $\SI{0}{\degree}$, i.e. $y = 0$). The vector $\hat{v}$ at the end of arc $a$ will then have a direction of $(\sin a, 0, \cos a)^T$. The vector $\hat{w}$ to the remaining corner at the intersection of arcs $b$ and $c$ will similarly have a $z$-component of $\cos b$, and its projection on $x$-$y$ plane will be $\sin b$ and the $x$/$y$-component will then be $\sin b \cos C$ and $\sin b \sin C$, i.e. $\hat{w} = (\sin b \cos C, \sin b \sin C, \cos b)^T$. Now consider the dot product $\hat{v} \cdot \hat{w}$. The geometric meaning of dot product (Properties \ref{proper:dotgeo}) implies that it is the angle between $\hat{v}$ and $\hat{w}$, that is, $\hat{v} \cdot \hat{w} = \cos c$. On the other hand,
\begin{align*}
\hat{v} \cdot \hat{w} &= (\sin a, 0, \cos a)^T \cdot (\sin b \cos C, \sin b \sin C, \cos b)^T \\
&= (\sin a) (\sin b \cos C) + (0) (\sin b \sin C) + (\cos a) (\cos b) \\
&= \cos a \cos b + \sin a \sin b \cos C 
\end{align*}
Therefore, equaling the two expressions of $\hat{v} \cdot \hat{w}$ gives the desired formula of $\cos c = \cos a \cos b + \sin a \sin b \cos C$.
\end{proof}

\section{Earth Science Applications}
\begin{exmp}
\label{exmp:Haversine}
Derive the \textit{Haversine Formula} for finding the great-circle distance between any two points on a sphere with their latitudes/longitudes provided. Hence find the distance between New York (\SI{40.73}{\degree N}, \SI{73.94}{\degree W}) and Warsaw (\SI{52.24}{\degree N}, \SI{21.02}{\degree E}).
\end{exmp}
\begin{solution}
Denote the latitudes/longitudes of the two locations by $\varphi_{1,2}$ and $\lambda_{1,2}$. Staring from the Cosine Law for Spherical Trigonometry (Properties \ref{proper:cosinesphere}) with corner $C$ still fixed at north pole but arc $a$ not necessarily along the Prime Meridian, we have $C = \lambda_2 - \lambda_1$, $a = \frac{\pi}{2} - \varphi_1$, $b = \frac{\pi}{2} - \varphi_2$, and
\begin{align*}
\cos c &= \cos a \cos b + \sin a \sin b \cos C \\
\cos c &= \cos (\frac{\pi}{2} - \varphi_1) \cos (\frac{\pi}{2} - \varphi_2) + \sin (\frac{\pi}{2} - \varphi_1) \sin (\frac{\pi}{2} - \varphi_2) \cos (\lambda_2 - \lambda_1) \\
\cos c &= \sin \varphi_1 \sin \varphi_2 + \cos \varphi_1 \cos \varphi_2 \cos (\lambda_2 - \lambda_1)
\end{align*}
The \textit{haversine} of an angle $\theta$ is $\text{hav}(\theta) = \sin^2 (\frac{\theta}{2}) = \frac{1}{2}(1-\cos\theta)$ and therefore $\cos\theta = 1 - 2\ \text{hav}(\theta)$. Subsequently,
\begin{align*}
\cos c &= \sin \varphi_1 \sin \varphi_2 + \cos \varphi_1 \cos \varphi_2 (1 - 2\ \text{hav}(\lambda_2 - \lambda_1)) \\ 
\cos c &= \sin \varphi_1 \sin \varphi_2 + \cos \varphi_1 \cos \varphi_2 - 2 \cos \varphi_1 \cos \varphi_2\ \text{hav}(\lambda_2 - \lambda_1) \\
\cos c &= \cos(\varphi_2 - \varphi_1) - 2 \cos \varphi_1 \cos \varphi_2\ \text{hav}(\lambda_2 - \lambda_1) \\
(1 - 2\ \text{hav}(c)) &= (1 - 2\ \text{hav}(\varphi_2 - \varphi_1)) - 2 \cos \varphi_1 \cos \varphi_2\ \text{hav}(\lambda_2 - \lambda_1) \\
\text{hav}(c) &= \text{hav}(\varphi_2 - \varphi_1) + \cos \varphi_1 \cos \varphi_2\ \text{hav}(\lambda_2 - \lambda_1)
\end{align*}
where we have used the trigonometric identity $\cos(\theta-\phi) = \cos \theta \cos \phi + \sin \theta \sin \phi$ in the middle. The Haversine Formula is now established and we can use it to calculate the angle $c$ subtended by the arc between two locations and hence their distance by $d = rc$ where $r$ is the radius (of the Earth, \SI{6370}{\km}). For New York (\SI{40.73}{\degree N}, \SI{73.94}{\degree W}) and Warsaw (\SI{52.24}{\degree N}, \SI{21.02}{\degree E}), $\lambda_1 = \SI{-73.94}{\degree}$, $\lambda_2 = \SI{21.02}{\degree}$, $\varphi_1 = \SI{40.73}{\degree}$, $\varphi_2 = \SI{52.24}{\degree}$, and
\begin{align*}
\text{hav}(c) &= \text{hav}(\SI{52.24}{\degree} - \SI{40.73}{\degree}) \\
&\quad+ \cos (\SI{40.73}{\degree}) \cos (\SI{52.24}{\degree})\ \text{hav}(\SI{21.02}{\degree} - (\SI{-73.94}{\degree})) \\
&= \text{hav}(\SI{11.51}{\degree}) + \cos (\SI{40.73}{\degree}) \cos (\SI{52.24}{\degree})\ \text{hav}(\SI{94.96}{\degree}) \\
&= \sin^2 (\frac{\SI{11.51}{\degree}}{2}) + \cos (\SI{40.73}{\degree}) \cos (\SI{52.24}{\degree}) \sin^2 (\frac{\SI{94.96}{\degree}}{2}) \\
\sin^2 (\frac{c}{2}) &\approx 0.26214 \\
c &\approx \SI{61.6}{\degree} = \SI{1.075}{\radian}
\end{align*}
and therefore the required distance is $d = rc = (\SI{6370}{\km})(\SI{1.075}{\radian}) \approx \SI{6848}{\km}$. The value computed by the Haversine Formula will be slightly off from the true value since the Earth is not a perfect sphere but rather an oblate one.
\end{solution}

\begin{exmp}
The Earth's magnetic field can be approximated by a magnetic dipole, so that the magnetic field lines on the Earth's surface are oriented from the geomagnetic North Pole to geomagnetic South Pole (like longitudinal lines but for the geomagnetic dipole). In 2020, the geomagnetic North Pole is at \SI{80.7}{\degree N}, \SI{72.7}{\degree W}. Find the magnetic declination (angle from the geographic North to geomagnetic North) at Tokyo (\SI{35.65}{\degree N}, \SI{139.84}{\degree E}) according to this \textit{geomagnetic dipole model}.
\end{exmp}
\begin{solution}
To find the magnetic declination we need to calculate the three arcs of the spherical triangle with its three corners at the geographic/geomagnetic North Pole and Tokyo. The arc distance between geographic/geomagnetic North Pole $d$ is simply $\SI{90}{\degree} - \SI{80.7}{\degree} = \SI{9.3}{\degree}$. Similarly, the arc from the geographic North Pole to Tokyo is $a = \SI{90}{\degree} - \SI{35.65}{\degree} = \SI{54.35}{\degree}$. We can use the Haversine Formula derived in the last example to obtain the arc from the geomagnetic North Pole to Tokyo, which yields
\begin{align*}
\text{hav}(t) &= \text{hav}(\SI{80.7}{\degree} - \SI{35.65}{\degree}) \\ 
&\quad + \cos(\SI{35.65}{\degree}) \cos(\SI{80.7}{\degree})\ \text{hav}((\SI{-72.7}{\degree}) - \SI{139.84}{\degree}) \\
&= \text{hav}(\SI{45.05}{\degree}) + \cos(\SI{35.65}{\degree}) \cos(\SI{80.7}{\degree})\ \text{hav}(\SI{-212.54}{\degree}) \\
&\approx 0.26777 \\
c &\approx \SI{62.3}{\degree}
\end{align*}
Denote the declination angle by $D$. By Properties \ref{proper:cosinesphere}, we have
\begin{align*}
\cos d &= \cos a \cos t + \sin a \sin t \cos D \\
\cos (\SI{9.3}{\degree}) &= \cos (\SI{54.35}{\degree}) \cos (\SI{62.3}{\degree}) + \sin (\SI{54.35}{\degree}) \sin (\SI{62.3}{\degree}) \cos D \\
\cos D &\approx 0.9951 \\
D &\approx \pm \SI{5.7}{\degree}
\end{align*}
To determine the sign, we note that concluded from the longitudes of Tokyo and geomagnetic North, the geomagnetic North is located to the east of Tokyo, and hence $D = \SI{5.7}{\degree E}$. However, we note that the actual declination is \SI{7.8}{\degree W} which has an opposite sign and is far from our answer (you can extract the value from \href{https://www.ngdc.noaa.gov/geomag/calculators/magcalc.shtml}{https://www.ngdc.noaa.gov/geomag/calculators/magcalc.shtml}). The reason is that the geomagnetic dipole is only a rough first-order approximation, while in reality the Earth's magnetic field has a much more complex structure.
\end{solution}

\section{Python Programming}
Projection as in Properties \ref{proper:proj} can be calculated by \verb|numpy| functions and let's wrap them up in our self-defined function as below.
\begin{lstlisting}
def scalar_projection(u, v):
    """ Calculates the scalar projection of v onto u. """
    return np.dot(u,v) / np.linalg.norm(u)
\end{lstlisting}
This computes the scalar projection of $\vec{v}$ onto $\vec{u}$. Testing with Example \ref{exmp:projectuv} shows 
\begin{lstlisting}
u = np.array([4., 1., -3.])
v = np.array([-2., 3., -1.])
print(scalar_projection(u, v))    
\end{lstlisting}
a consistent output of \texttt{-0.39223}. Incorporating the unit vector function (\verb|unit_vector()|) defined in the last chapter's programming section, we obtain the vector projection.
\begin{lstlisting}
def vector_projection(u, v):
    """ Calculates the vector projection of v onto u. """
    return scalar_projection(u, v) * unit_vector(u)

print(vector_projection(u, v))    
\end{lstlisting}
This results in \texttt{[-0.3077 -0.0769  0.2308]} which matches the example's answer. Area of parallelogram formed by two vectors is the magnitude of their cross product and the corresponding function is typed below.
\begin{lstlisting}
def area_parallelogram(u, v):
    """ Calculate the area of parallelogram formed by two vectors u and v. """
    return np.linalg.norm(np.cross(u,v))
\end{lstlisting}
\verb|print(area_parallelogram(u, v))| then gives \texttt{18.974}. Meanwhile, the function to compute volume of parallelepiped made up of three vectors can be defined such that it uses the determinant formula in Properties \ref{proper:parallelpiped}.
\begin{lstlisting}
def volume_parallelepiped(u, v, w):
    """ Calculate the volume of parallelepiped formed by two vectors u, v, w. """
    return np.abs(np.linalg.det(np.c_[u,v,w]))

w = np.array([1., 2., -3.])
print(volume_parallelepiped(u, v, w))
\end{lstlisting}
(\verb|np.c_[]| is a short hand of combining arrays column by column) produces \texttt{14.00000...04} due to numerical round-off error (the true answer would be just $14$). Finally, let's conclude this section by defining the Haversine Formula in Example \ref{exmp:Haversine}.
\begin{lstlisting}
def Haversine_dist(latlon1, latlon2):
    """ Haversine Formula for computing the great-circle 
        distance between two places on the Earth. 
        Input: (lat1, lon1), (lat2, lon2) in degrees.
        Output: Great-circle distance in km.
    """
    R_Earth = 6370 # Earth's Radius
    lat1, lon1 = latlon1[0], latlon1[1]
    lat2, lon2 = latlon2[0], latlon2[1]
    # Converting degree to radian
    lat1_rad, lon1_rad, lat2_rad, lon2_rad = np.deg2rad(lat1), np.deg2rad(lon1), np.deg2rad(lat2), np.deg2rad(lon2) 
    # Haversine's Formula
    hav_c = np.sin((lat2_rad-lat1_rad)/2)**2 + np.cos(lat1_rad)*np.cos(lat2_rad)*np.sin((lon2_rad-lon1_rad)/2)**2 
    arc_c = 2*np.arcsin(np.sqrt(hav_c)) # Inverting to get the great-circle arc angle
    return(R_Earth*arc_c) # Arc angle to arc length
\end{lstlisting}
Using the latitudes and longitudes of New York and Warsaw in Example \ref{exmp:Haversine} for testing, \verb|Haversine_dist((40.73, -73.94), (52.24, 21.02))| outputs \texttt{6847.76}.

\section{Exercises}

\begin{Exercise}
Parameterize the following equations into vector form.
\begin{enumerate}[label=(\alph*)]
\item $6x + 8y = 9$,
\item $x + 9y + 9z = 7$,
\item $y = 3, -\infty < x < \infty$, and
\item $2x + z = 9, -\infty < y < \infty$.
\end{enumerate}
\end{Exercise}
\begin{Answer}
\begin{enumerate}[label=(\alph*)]
\item
$\begin{bmatrix}
x\\
y    
\end{bmatrix}
=
\begin{bmatrix}
\frac{3}{2}\\
0
\end{bmatrix}
+
t
\begin{bmatrix}
-\frac{4}{3}\\
1
\end{bmatrix}
$
\item
$\begin{bmatrix}
x\\
y\\
z
\end{bmatrix}
=
\begin{bmatrix}
7\\
0\\
0
\end{bmatrix}
+
s
\begin{bmatrix}
-9\\
1\\
0
\end{bmatrix}
+
t
\begin{bmatrix}
-9\\
0\\
1
\end{bmatrix}
$
\item 
$\begin{bmatrix}
x\\
y
\end{bmatrix}
=
\begin{bmatrix}
0\\
3
\end{bmatrix}
+
t
\begin{bmatrix}
1\\
0
\end{bmatrix}
$
\item
$\begin{bmatrix}
x\\
y\\
z
\end{bmatrix}
=
\begin{bmatrix}
\frac{9}{2}\\
0\\
0
\end{bmatrix}
+
s
\begin{bmatrix}
0\\
1\\
0
\end{bmatrix}
+
t
\begin{bmatrix}
-\frac{1}{2}\\
0\\
1
\end{bmatrix}$
\end{enumerate}
\end{Answer}

\begin{Exercise}
Eliminate the parameters and find the direct equation.
\begin{enumerate}[label=(\alph*)]
\item \begin{align*}
\begin{bmatrix}
x\\
y
\end{bmatrix}
=
\begin{bmatrix}
2\\
9
\end{bmatrix}
+
t
\begin{bmatrix}
1\\
1
\end{bmatrix}
\end{align*}
\item \begin{align*}
\begin{bmatrix}
x\\
y\\
z
\end{bmatrix}
=
\begin{bmatrix}
6\\
3\\
2
\end{bmatrix}
+
s
\begin{bmatrix}
7\\
4\\
1
\end{bmatrix}
+
t
\begin{bmatrix}
8\\
0\\
5
\end{bmatrix}
\end{align*}
\end{enumerate}
where $-\infty < s,t < \infty$.
\end{Exercise}
\begin{Answer}
\begin{enumerate}[label=(\alph*)]
\item
Normal vector to the line is 
$\begin{bmatrix}
1\\
-1
\end{bmatrix}
$.\\
Equation: $\begin{bmatrix}
1\\
-1
\end{bmatrix}
\cdot
\begin{bmatrix}
x\\
y
\end{bmatrix}
=
\begin{bmatrix}
1\\
-1
\end{bmatrix}
\cdot
\begin{bmatrix}
2\\
9
\end{bmatrix}
\rightarrow
x - y = -7$
\item
Normal vector to the plane is 
$\begin{bmatrix}
7\\
4\\
1
\end{bmatrix}
\times
\begin{bmatrix}
8\\
0\\
5
\end{bmatrix}
=
\begin{bmatrix}
20\\
-27\\
-32
\end{bmatrix}
$.\\
Equation:
$\begin{bmatrix}
20\\
-27\\
-32\\
\end{bmatrix}
\cdot
\begin{bmatrix}
x\\
y\\
z
\end{bmatrix}
=
\begin{bmatrix}
20\\
-27\\
-32\\
\end{bmatrix}
\cdot
\begin{bmatrix}
6\\
3\\
2
\end{bmatrix}
\rightarrow
20x - 27y - 32z = -25$
\end{enumerate}
\end{Answer}

\begin{Exercise}
\label{ex:dist_pt_line_R3}
Find the distance of the point $(3,2,9)^T$ to the plane $x + 2y + 5z = 10$, as well as the distance of the point $(3,2,9)^T$ to the line 
\begin{align*}
\begin{bmatrix}
x\\
y\\
z
\end{bmatrix}
=
t
\begin{bmatrix}
0\\
1\\
2
\end{bmatrix}
\end{align*}
where $-\infty < t < \infty$.
\end{Exercise}
\begin{Answer}
Part 1: Choose $(0,0,2)^T$ as a reference point on the plane.\\
Projection of the vector from $(0,0,2)^T$ to $(3,2,9)^T$: $(3-0)\hat{i} + (2-0)\hat{j} + (9-2)\hat{k} = 3\hat{i} + 2\hat{j} + 7\hat{k}$ onto the normal vector $\hat{i} + 2\hat{j} + 5\hat{k}$ of the plane is
\[\frac{(3)(1)+(2)(2)+(7)(5)}{\sqrt{1^2 + 2^2 + 5^2}} = \frac{42}{\sqrt{30}}\] which is the required distance.\\
Part 2: Choose $(0,1,2)^T$ as a reference point along the line. Find the projection of $(3,2,9)^T - (0,1,2)^T = 3\hat{i} + 1\hat{j} + 7\hat{k}$ onto the direction vector $\hat{j} + 2\hat{k}$, which is
\begin{align*}
\frac{(3)(0)+(1)(1)+(7)(2)}{0^2 + 1^2 + 2^2} (\hat{j} + 2\hat{k}) &= 3(\hat{j} + 2\hat{k}) = 3\hat{j} + 6\hat{k}
\end{align*}
The displacement vector between the point and line (which is orthogonal to the line) is then $(3\hat{i} + 1\hat{j} + 7\hat{k}) - (3\hat{j} + 6\hat{k}) = 3\hat{i} - 2\hat{j} + \hat{k}$ and the required distance equals to $\sqrt{3^2 + (-2)^2 + 1^2} = \sqrt{14}$.
\end{Answer}

\begin{Exercise}
Prove that the shortest distance between two lines, $\vec{u} = \vec{a} + s\hat{l}$ and $\vec{v} = \vec{b} + t\hat{m}$, where $-\infty < s,t < \infty$, $\vec{a}, \vec{b}$ are some arbitrary vectors and $\hat{l}, \hat{m}$ are some fixed, non-parellel unit vectors representing direction of the two lines, is
\begin{align*}
\text{Dist}(u, v) = \frac{(\hat{a}-\hat{b}) \cdot (\hat{l} \times \hat{m})}{\norm{\hat{l} \times \hat{m}}}
\end{align*}
Hints: Geometrically, the distance between these two lines is the projection of any vector from one line to another onto the vector normal to the plane made by $\hat{l}$ and $\hat{m}$.
\begin{align*}
\frac{(\vec{v} - \vec{u}) \cdot (\hat{l} \times \hat{m})}{\norm{\hat{l} \times \hat{m}}}
\end{align*}
Draw a diagram to convince yourself it is true. What does it imply if $\vec{a} \cdot (\hat{l} \times \hat{m}) = \vec{b} \cdot (\hat{l} \times \hat{m})$?
\end{Exercise}
\begin{Answer}
Using the hints, we have the distance as
\begin{align*}
\frac{(\vec{v} - \vec{u}) \cdot (\hat{l} \times \hat{m})}{\norm{\hat{l} \times \hat{m}}} &= \frac{[(\vec{b} + \hat{m}t) - (\vec{a} + \hat{l}s)] \cdot (\hat{l} \times \hat{m})}{\norm{\hat{l} \times \hat{m}}} \\
&= \frac{(\vec{b} - \vec{a}) \cdot (\hat{l} \times \hat{m}) + [\hat{m} \cdot (\hat{l} \times \hat{m})] t - [\hat{l} \cdot (\hat{l} \times \hat{m})] s}{\norm{\hat{l} \times \hat{m}}}
\end{align*}
Notice that $\hat{l} \times \hat{m}$ is orthogonal to both $\hat{l}$ and $\hat{m}$, and thus $\hat{l} \cdot (\hat{l} \times \hat{m}) = \hat{m} \cdot (\hat{l} \times \hat{m}) = 0$ both vanish. Therefore we are left with
\begin{align*}
\frac{(\vec{b} - \vec{a}) \cdot (\hat{l} \times \hat{m})}{\norm{\hat{l} \times \hat{m}}} 
\end{align*} 
If $\vec{a} \cdot (\hat{l} \times \hat{m}) = \vec{b} \cdot (\hat{l} \times \hat{m})$, then the numerator $(\vec{b} - \vec{a}) \cdot (\hat{l} \times \hat{m}) = 0$ equals to zero such that the two lines intersect. In this case, the values of $s$ or $t$ at the point of intersection ($\vec{u} = \vec{v}$) can be found by applying a cross product with $\hat{m}$ on $\vec{u} = \vec{a} + \hat{l}s = \vec{b} + \hat{m}s = \vec{v}$ and note that $\hat{m} \times \hat{m} = \vec{0}$, and hence
\begin{align*}
(\vec{a} + \hat{l}s) \times \hat{m} &= (\vec{b} + \hat{m}s) \times \hat{m} \\
\vec{a} \times \hat{m} + s(\hat{l} \times \hat{m}) &= \vec{b} \times \hat{m} + s (\hat{m} \times \hat{m}) =  \vec{b} \times \hat{m} + s\vec{0}\\
s (\hat{l} \times \hat{m}) &= (\vec{b} - \vec{a}) \times \hat{m}
\end{align*}
$s$ is then inferred from the scaling ratio of $(\vec{b} - \vec{a}) \times \hat{m}$ to $(\hat{l} \times \hat{m})$. $t$ is found similarly.
\end{Answer}

\begin{Exercise}
Prove Sine Law with vector notation by considering the triangle below
\begin{center}
\begin{tikzpicture}
\draw[red,-{Latex[length=5mm, width=2mm]}] (0,0)--(2,3) node[midway, above left]{$\vec{a}$};
\draw[blue,-{Latex[length=5mm, width=2mm]}] (2,3)--(4,1) node[midway, above right]{$\vec{b}$};
\draw[Green,-{Latex[length=5mm, width=2mm]}] (4,1)--(0,0) node[midway, below]{$\vec{c}$};
\coordinate (B) at (0,0);
\coordinate (C) at (2,3);
\coordinate (A) at (4,1);
\pic[draw, "A", red, angle eccentricity=1.5] {angle = C--A--B};
\pic[draw, "B", blue, angle eccentricity=1.5] {angle = A--B--C};
\pic[draw, "C", Green, angle eccentricity=1.5] {angle = B--C--A};
\end{tikzpicture}
\end{center}
and equating three expressions of its area $\frac{1}{2}\norm{\vec{a}\times\vec{b}} = \frac{1}{2}\norm{\vec{b}\times\vec{c}} = \frac{1}{2}\norm{\vec{c}\times\vec{a}}$. Properties \ref{proper:areaparallelogram} will be useful.
\end{Exercise}
\begin{Answer}
\begin{align*}
&\quad \frac{1}{2}\norm{\vec{a}\times\vec{b}} = \frac{1}{2}\norm{\vec{b}\times\vec{c}} = \frac{1}{2}\norm{\vec{c}\times\vec{a}} \\
&\rightarrow \frac{1}{2}\norm{\vec{a}}\norm{\vec{b}}\sin{C} = \frac{1}{2}\norm{\vec{b}}\norm{\vec{c}}\sin{A} = \frac{1}{2}\norm{\vec{c}}\norm{\vec{a}}\sin{B} \\
&\rightarrow \frac{\sin{A}}{a} = \frac{\sin{B}}{b} = \frac{\sin{C}}{c}
\end{align*}
where we divide the entire equality by $abc = \norm{\vec{a}}\norm{\vec{b}}\norm{\vec{c}}$).
\end{Answer}

\begin{Exercise}
By extending Properties \ref{proper:parallelpiped}, derive a vector formula for the volume of a tetrahedron (pyramid).
\begin{center}
\begin{tikzpicture}
\coordinate (0) at (0,0,0);
\draw[thick, ->] (0)--(4,0,0) node[right](vecu){$\vec{u}$};
\draw[thick, ->] (0)--(1,0,-2) node[above right](vecv){$\vec{v}$};
\draw[thick, ->] (0)--(1,3,-1) node[above left](vecw){$\vec{w}$};
\draw[thick, gray, dashed] (4,0,0)--(1,0,-2);
\draw[thick, gray, dashed] (1,3,-1)--(1,0,-2);
\draw[thick, gray, dashed] (1,3,-1)--(4,0,0);
\end{tikzpicture}
\end{center}
\end{Exercise}
\begin{Answer}
It is just $\frac{1}{6}\abs{(\vec{u} \times \vec{v}) \cdot \vec{w}}$.
\end{Answer}

\begin{Exercise}
For $\vec{u} = (1,2,3)^T$, $\vec{v} = (2,1,5)^T$, $\vec{w} = (1,4,0)^T$, find
\begin{enumerate}[label=(\alph*)]
\item Area of the parallelogram formed by $\vec{u}$ and $\vec{v}$,
\item Volume of the parallelepiped formed by $\vec{u}$, $\vec{v}$ and $\vec{w}$,
\item Redo the above for $\vec{w} = (1,5,4)^T$, what does the result tell you?
\end{enumerate}
\end{Exercise}
\begin{Answer}
\begin{enumerate}[label=(\alph*)]
\item $\vec{u} \times \vec{v} =
\left|
\begin{array}{ccc}
\hat{i} & \hat{j} & \hat{k}\\
1 & 2 & 3\\
2 & 1 & 5
\end{array}
\right|
=
7\hat{i} + \hat{j} - 3\hat{k}$\\
Area = $\sqrt{7^2 + 1^2 + (-3)^2} = \sqrt{59}$
\item Volume is the absolute value of $|\vec{u} \times \vec{v}| \cdot \vec{w} 
= |(7\hat{i} + \hat{j} - 3\hat{k}) \cdot (\hat{i} + 4\hat{j})| = |(7)(1) + (1)(4) + (-3)(0)| = 11$
\item Volume = $\text{abs}\left|
\begin{array}{ccc}
1 & 2 & 3\\
2 & 1 & 5\\
1 & 5 & 4
\end{array}
\right|
= 0$. \\
So the three vectors are co-planar.
\end{enumerate}
\end{Answer}

\begin{Exercise}
Find the geometric interpretation of solutions of the following systems of linear equations.
\begin{enumerate}[label=(\alph*)]
\item \begin{align*}
\begin{cases}
x + 2y + 2z &= 3\\
3x - y + 3z &= 2\\
x - 2y - z &= -1
\end{cases}
\end{align*}
\item
\begin{align*}
\begin{cases}
2x - y - z &= 3\\
x + y + 2z &= -1\\
x + 4y + 7z &= -6
\end{cases}
\end{align*}
\end{enumerate}
\end{Exercise}
\begin{Answer}
\begin{enumerate}[label=(\alph*)]
\item The solution refers to the point $(1,1,0)$.
\item By Gaussian Elimination, one possible form of general solution is 
\begin{align*}
\begin{bmatrix}
x\\
y\\
z
\end{bmatrix}
=
\begin{bmatrix}
\frac{2}{3}\\
-\frac{5}{3}\\
0
\end{bmatrix}
+
t
\begin{bmatrix}
-\frac{1}{3}\\
-\frac{5}{3}\\
1
\end{bmatrix}    
\end{align*} 
Therefore, the solution space is a line parallel to 
$-\frac{1}{3}\hat{i} - \frac{5}{3}\hat{j} + \hat{k}$ and passing through the point $(\frac{2}{3}, -\frac{5}{3}, 0)^T$.
\end{enumerate}
\end{Answer}
\chapter{Vector Spaces and Coordinate Bases}
\label{chap:vec_space}

The previous chapters have provided a basic understanding of matrices and vectors separately. What bridge these two quantities together are the concepts of \textit{vector (sub)spaces}, \textit{linear combination}, \textit{span}, \textit{linear independence}. With all of these, we can revisit the process of Gaussian Elimination from the view of \textit{Column-row factorization}. Then, we will learn how to find \textit{coordinate bases} for vector spaces in order to represent vectors in different coordinate systems. Finally, we are going to investigate the so-called \textit{four fundamental subspaces} induced by a matrix and see how they are interconnected via the \textit{Rank-Nullity Theorem}.

\section{Making of the Real $n$-space $\mathbb{R}^n$}

\subsection{$\mathbb{R}^n$ as a Vector Space}

We have briefly mentioned under Definition \ref{defn:real_nspace} that the real $n$-space $\mathbb{R}^n$ is mathematically a vector space, but without stating the actual requirements. In fact, to be qualified as a \index{Vector Space}\keywordhl{vector space}, a set has to satisfy the ten axioms below. We will limit ourselves to \index{Real Vector Space}\keywordhl{real vector spaces} for now.
\begin{defn}[Axioms of a (Real) Vector Space]
\label{defn:realvecspaceaxiom}
A \textit{real} vector space is a non-empty set $\mathcal{V}$ with the zero vector \textbf{0}, such that for all elements (vectors) $\vec{u}, \vec{v}, \vec{w} \in \mathcal{V}$ in the set, and \textit{real} numbers (as the \textit{scalars}) $a, b \in \mathbb{R}$ (for a complex vector space replace $\mathbb{R}$ by $\mathbb{C}$ here), we have
\begin{enumerate}
\item $\vec{u} + \vec{v} \in \mathcal{V}$ (Closure under Vector Addition: Addition between two vectors is defined and the resulting vector is still in the vector space.)
\item $\vec{u} + \vec{v} = \vec{v} + \vec{u}$ (Commutative Property of Addition)
\item $(\vec{u} + \vec{v}) + \vec{w} = \vec{u} + (\vec{v} + \vec{w})$ (Associative Property of Addition)
\item $\vec{u} + \textbf{0} = \textbf{0} + \vec{u} = \vec{u}$ (Zero Vector as the Additive Identity)
\item For any $\vec{u}$, there exists $\vec{w}$ such that $\vec{u} + \vec{w} = \textbf{0}$. This $\vec{w}$ is denoted as $-\vec{u}$. (Existence of Additive Inverse)
\item $a\vec{u} \in \mathcal{V}$ (Closure under Scalar Multiplication: Multiplying a vector by any scalar (a real [complex] number for a real [complex] vector space) is defined and the resulting vector is still in the vector space.)
\item $a(\vec{u} + \vec{v}) = a\vec{u} + a\vec{v}$ (Distributive Property of Scalar Multiplication)
\item $(a+b)\vec{u} = a\vec{u} + b\vec{u}$ (Distributive Property of Scalar Multiplication)
\item $a(b\vec{u}) = (ab)\vec{u}$ (Associative Property of Scalar Multiplication)
\item $1\vec{u} = \vec{u}$ (The real number $1$ as the Multiplicative Identity)
\end{enumerate}
\end{defn}
The real $n$-space $\mathbb{R}^n$ satisfies all the axioms above and is finite-dimensional, particularly $n$-dimensional (the notion of dimension here should be intuitive, but we will go through it more precisely later), with addition and scalar multiplication being the usual ones as defined in Section \ref{section:vectoraddmul}, and the zero vector is simply $\textbf{0} = (0,0,0,\ldots,0)^T$ with $n$ zeros. We will not do it here but interested readers can try to justify all of them. \par
Building the definition of a vector space from these axioms allows the generalization and application of its utilities to other sets that share the same abstract structure. \textcolor{red}{However, for most usages, we will focus on $\mathbb{R}^n$\footnote{We actually have a very good reason to do so, as we will see in the next chapter: any $n$-dimensional real vector space is \textit{isomorphic} to and can be treated like $\mathbb{R}^n$.}, and the vector space axioms are provided above mainly for reference.} We defer the treatment of complex vector spaces to Chapter \ref{chap:complex}.

\subsection{Subspaces of $\mathbb{R}^n$}
\label{section:Rnsubspace}

It will be very boring if we consider only the whole $\mathbb{R}^n$ as a vector space. In the last chapter, we show that geometrically there can be lower-dimensional shapes like lines/planes/hyperplanes residing in $\mathbb{R}^n$. This raises the question if we can similarly find \index{Subspace}\keywordhl{subspaces} embedded in $\mathbb{R}^n$ that is a subset of $\mathbb{R}^n$ which still fulfills the aforementioned vector space axioms such that it is a vector space in its own right. Nevertheless, to determine if a subset of vector space is a subspace, we don't need to check all the ten axioms but rather just two of them.
\begin{thm}[Criteria for a Subspace]
\label{thm:subspacecriteria}
If $\mathcal{W}$ is a non-empty subset of a (real) vector space $\mathcal{V}$ (i.e. $\mathcal{W} \subseteq \mathcal{V}$), then $\mathcal{W}$ is called a (real) subspace of $\mathcal{V}$ if the following criteria are satisfied:
\begin{enumerate}
\item For any $\vec{u}, \vec{v} \in \mathcal{W}$, $\vec{u} + \vec{v} \in \mathcal{W}$ (Closed under Addition)
\item For any scalar $a$ ($\in \mathbb{R}$) and $\vec{u} \in \mathcal{W}$, $a\vec{u} \in \mathcal{W}$ (Closed under Scalar Multiplication). Particularly, when $a = 0$, $0\vec{u} = \textbf{0} \in \mathcal{W}$ so that a subspace always contains the zero vector of $\mathcal{V}$.
\end{enumerate}
These are the same requirements of (1) and (6) in Definition \ref{defn:realvecspaceaxiom}. An equivalent condition is that, for any $\vec{u}, \vec{v} \in \mathcal{W}$ and two scalars $a$ and $b$, $a\vec{u} + b\vec{v} \in \mathcal{W}$.
\end{thm}
\begin{exmp}
Consider the following subsets of $\mathbb{R}^2$ and decide if they are subspaces of $\mathbb{R}^2$ by verifying the two criteria listed in Theorem \ref{thm:subspacecriteria}.
\begin{enumerate}[label=(\alph*)]
\item The line $x - 2y = 0$,
\item The $y$-axis,
\item The positive $y$-axis,
\item The line $2x + y = 1$,
\item The parabola $y = x^2$,
\item The point $(-1,1)^T$,
\item The first quadrant $x > 0$, $y > 0$,
\item The origin $\textbf{0} = (0,0)^T$,
\item $\mathbb{R}^2$ itself.
\end{enumerate}
\end{exmp}
\begin{solution}
\begin{enumerate}[label=(\alph*)]
\item The vector form of the line is $\mathcal{W} = \{(x,y)^T = t(2,1)^T \mid -\infty < t < \infty\}$. To check the first condition, let's say $\vec{u} = t_1(2,1)^T \in \mathcal{W}$ and $\vec{v} = t_2(2,1)^T \in \mathcal{W}$ are vectors in $\mathcal{W}$ for some $t_1$ and $t_2$, then $\vec{u} + \vec{v} = t_1(2,1)^T + t_2(2,1)^T = (t_1 + t_2)(2,1)^T = s(2,1)^T \in \mathcal{W}$ where $s = t_1 + t_2$, also lies on the same straight line of $x-2y = 0$ and is another vector in $\mathcal{W}$, so $\mathcal{W}$ is closed under addition. To check the second condition, this time we simply let $\vec{u} = t(2,1)^T \in \mathcal{W}$. Subsequently, $a\vec{u} = at(2,1)^T = r(2,1)^T \in \mathcal{W}$, for any scalar $a$ and $r = at$, so it is also closed under scalar multiplication. Hence the line $x-2y = 0$ is a subspace of $\mathbb{R}^2$.
\item Same arguments as above but with $\mathcal{W} = \{(x,y)^T = t(0,1)^T \mid -\infty < t < \infty\}$, so the $y$-axis is also a subspace of $\mathbb{R}^2$.
\item For any point on the positive $y$-axis, multiplying it by a negative number places it on the negative $y$-axis instead, so it is not closed under scalar multiplication and thus not a subspace of $\mathbb{R}^2$. 
\item Denote the collection of points on the line as $\mathcal{W}$. Pick $\vec{u} = (1, -1)^T \in \mathcal{W}$ and $\vec{v} = (0, 1)^T \in \mathcal{W}$, then $\vec{u} + \vec{v} = (1, 0)^T \notin \mathcal{W}$ as $2(1) + (0) = 2 \neq 1$, so it is not closed under addition and fails to be a subspace of $\mathbb{R}^2$.
\item Denote the collection of points on the parabola as $\mathcal{W}$. Pick $\vec{u} = (1,1)^T \in \mathcal{W}$ and $\vec{v} = (2,4)^T \in \mathcal{W}$, then $\vec{u} + \vec{v} = (3,5)^T \notin \mathcal{W}$ is apparently not on the parabola, so it is not closed under addition and can't be a subspace of $\mathbb{R}^2$.
\item It is easy to see that it fails to be closed under either addition or scalar multiplication (for example, take $a(-1,1)^T$ with $a\neq 1$) and is not a subspace of $\mathbb{R}^2$.
\item Denote the collection of points on the first quadrant as $\mathcal{W}$. Pick $\vec{u} = (1,1)^T \in \mathcal{W}$ (or any other point), then multiplying it by $-1$ will produce $(-1)\vec{u} = -(1,1)^T = (-1,-1)^T \notin \mathcal{W}$ which is outside the first quadrant. Therefore, it is not closed under scalar multiplication and hence not a subspace of $\mathbb{R}^2$.
\item It trivially satisfies the two criteria ($\textbf{0}$ is the only element in the set, $\textbf{0} + \textbf{0} = \textbf{0}$ and $a\textbf{0} = \textbf{0}$ for any scalar $a$) and is a subspace of $\mathbb{R}^2$.
\item $\mathbb{R}^2$ is a vector space to begin with and technically a subset of itself (it trivially contains itself) so by definition it is a subspace of $\mathbb{R}^2$.
\end{enumerate}
\end{solution}
Generalizing the above discussion, we can easily infer that for $\mathbb{R}^2$, only the origin (the zero subspace), an infinitely long straight line that passes through the origin, or $\mathbb{R}^2$ itself can be its subspaces (see the schematic in Figure \ref{fig:R2subspace}). We often use the phrase \index{Proper Subspace}\keywordhl{proper subspaces} to exclude the accommodating vector space itself ($\mathbb{R}^2$ in this case). For any $\mathbb{R}^n$, the \index{Zero Subspace}\keywordhl{zero subspace} $\{\textbf{0}\}$ and \index{Improper Subspace}\keywordhl{improper subspace} $\mathbb{R}^n$ are always two subspaces of it. 
\begin{figure}
    \centering
    \begin{tikzpicture}
    \filldraw[draw=none, fill=blue, opacity=0.1]
    (-3.2,3.2) -- (3.2,3.2) -- (3.2,-3.2) -- (-3.2,-3.2) -- cycle;
    \draw[blue, line width = 0.9, ->] (-3,0)--(3,0) node[right, align=left, xshift=5, yshift=-5]{$x$: The entire $x$-axis \\
    as a one-dimensional subspace};
    \draw[->] (0,-3)--(0,3) node[above]{$y$}; 
    \draw[red, line width = 0.9] circle (2);
    \node[blue, align=left, anchor=center] at (2.5, 3.5) {The $xy$ plane itself ($\mathbb{R}^2$) \\ is an improper subspace.};
    \draw[red, line width = 0.9] plot[smooth, domain=-2.5:2.5] (\x, 0.4 * \x * \x) node[right, align=left, yshift=5]{A parabola is not a subspace.};
    \draw[red, line width = 0.9] plot[smooth, domain=-2.75:2.75] (\x, 0.5* \x -1) node[right, align=left, yshift=22]{An infinitely long straight line \\
    not passing through the origin \\
    is not a subspace. (but \textit{affine})};
    \draw[blue, line width = 0.9] plot[smooth, domain=-2.4:2.4] (\x, -1.2 * \x) node[right, align=left]{An infinitely long straight line \\passing through the origin \\as a one-dimensional subspace};
    \node[red,rotate=-15] at (-0.75, -2.25) {A circle is not a subspace.};
    \node[align=left, anchor=north west]{\large $O$: \\ The origin is\\ the zero subspace};
    \end{tikzpicture}
    \caption{\textit{Some examples (blue) and non-examples (red) of subspaces in $\mathbb{R}^2$.}}
    \label{fig:R2subspace}
\end{figure}\par
Short Exercise: Determine if the following subsets of $\mathbb{R}^3$ is a subspace of $\mathbb{R}^3$.\footnote{Yes, No, Yes, No, Yes, No, Yes, No, No. In fact, all possible subspaces of $\mathbb{R}^3$ are $\{\textbf{0}\}$, any infinitely long line/extending plane through the origin and $\mathbb{R}^3$ itself.}
\begin{enumerate}[label=(\alph*)]
\item The origin $\textbf{0} = (0,0,0)^T$,
\item The point $(1,2,3)^T$,
\item The line $(x,y,z)^T = t(-1, 1, 2)^T$ for any scalar $t$,
\item The line $(x,y,z)^T = (1, -1, 3) + t(1, 2, -1)^T$ for any scalar $t$,
\item The plane $x + 2y - 3z = 0$,
\item The plane $x + y + 4z = 5$,
\item $\mathbb{R}^3$ itself,
\item The sphere $x^2 + y^2 + z^2 = 1$,
\item The cone $x^2 + y^2 = z^2$.
\end{enumerate}
Further generalization motivated by the short exercise above leads to an intuitive result that, for $\mathbb{R}^n$, all its possible subspaces are geometrically "flat shapes" that pass through the origin and extend infinitely. On the other hand, any "curved shape" will not qualify as a subspace. From now on, we assume all vector (sub)spaces mentioned are finite-dimensional (again, we will clarify this notion later) unless otherwise specified.

\subsection{Span by Linear Combinations of Vectors}
\label{section:span}
The last section sees subspaces from a top-down perspective as some subsets of a larger vector space. Here, we are going to take another look at them with a bottom-up perspective, about how to generate a subspace of $\mathbb{R}^n$ from some vectors it contains. To do so, we need to first understand what is a \index{Linear Combination}\keywordhl{linear combination} of vectors.
\begin{defn}[Linear Combination of Vectors]
\label{defn:linearcomb}
A linear combination of vectors $\vec{v}^{(1)}, \vec{v}^{(2)}, \vec{v}^{(3)}, \ldots, \vec{v}^{(q)} \in \mathcal{V}$ where $\mathcal{V}$ is some vector space has the form of
\begin{align}
\sum_{j=1}^q c_j\vec{v}^{(j)} = c_1\vec{v}^{(1)} + c_2\vec{v}^{(2)} + c_3\vec{v}^{(3)} + \cdots + c_q\vec{v}^{(q)} \label{eqn:linearcomb}
\end{align}
where the coefficients $c_j$ are some scalars (real [complex] numbers for a real [complex] vector space) and the number of vectors $q$ has to be \textit{finite}.
\end{defn}
As a small example, if there are two vectors $\vec{u} = (1,2)^T$ and $\vec{v} = (3,4)^T \in \mathbb{R}^2$, then $\vec{h} = (5,6)^T \in \mathbb{R}^2$ can be written as a linear combination of $\vec{u}$ and $\vec{v}$ because $\vec{h} = (5,6)^T = -(1,2)^T + 2(3,4)^T = -\vec{u} + 2\vec{v}$.\par
Short Exercise: If $\vec{h} = (1,4)^T$ instead, express it as a linear combination of $\vec{u}$ and $\vec{v}$.\footnote{$(1,4)^T = 4(1,2)^T - (3,4)^T$.}\par

Attentive readers may realize that the short exercise above can be thought of as a task to find out the solution (if any) for the system
\begin{align*}
\begin{bmatrix}
1 & 3 \\
2 & 4 \\
\end{bmatrix}
\begin{bmatrix}
c_1 \\
c_2
\end{bmatrix} =
\begin{bmatrix}
1 \\
4
\end{bmatrix}
\end{align*}
Extending this, to decide whether a vector $\vec{h} \in \mathbb{R}^n$ can be written as the linear combination of other vectors $\vec{v}^{(j)} \in \mathbb{R}^n$, $j = 1, 2, \ldots, q$, is equivalent to determining whether the linear system $A\vec{x} = \vec{h}$ has a solution, where $A$ is equal to (writing out $\vec{v}^{(j)}$ in a matrix column by column)
\begin{align*}
A = \left[\begin{array}{@{}c|c|c|c@{}}
| & | & & | \\
\vec{v}^{(1)} & \vec{v}^{(2)} & \cdots & \vec{v}^{(q)} \\[-4pt]
| & | & & |
\end{array}\right]
\end{align*}
Here, the matrix product $A\vec{x}$ is a compact way to represent a linear combination of the column vectors that have been condensed into $A$.
\begin{proper}
\label{proper:linearcombmatrix}
A linear combination $c_1\vec{v}^{(1)} + c_2\vec{v}^{(2)} + c_3\vec{v}^{(3)} + \cdots + c_q\vec{v}^{(q)}$ made up of some vectors $\vec{v}^{(1)}, \vec{v}^{(2)}, \vec{v}^{(3)}, \ldots, \vec{v}^{(q)} \in \mathbb{R}^n$ as in (\ref{eqn:linearcomb}) of Definition \ref{defn:linearcomb}, can be expressed by the matrix product $A\vec{x}$, where
\begin{align}
&A = \left[\begin{array}{@{}c|c|c|c@{}}
| & | & & | \\
\vec{v}^{(1)} & \vec{v}^{(2)} & \cdots & \vec{v}^{(q)} \\[-4pt]
| & | & & |
\end{array}\right]
&\vec{x} =
\begin{bmatrix}
c_1 \\
c_2 \\
c_3 \\
\vdots \\
c_q
\end{bmatrix}
\end{align}
From now on, we will just simply write $A = [\vec{v}^{(1)} | \vec{v}^{(2)} | \cdots | \vec{v}^{(q)}]$ and similarly for other matrices formed by column vectors when applicable to save space.
\end{proper}
From this perspective, the first/second/last column of a matrix $A$ can be extracted by
\begin{align*}
A&
\begin{bmatrix}
1 \\
0 \\
0 \\
\vdots \\
0
\end{bmatrix}
&
A&
\begin{bmatrix}
0 \\
1 \\
0 \\
\vdots \\
0
\end{bmatrix}
&
A&
\begin{bmatrix}
0 \\
0 \\
0 \\
\vdots \\
1
\end{bmatrix}
\end{align*}
and it goes similarly for any other column. Below is a small example to demonstrate the equivalence between matrix-vector products and linear combinations.
\begin{align*}
\begin{tikzpicture}[baseline=-\the\dimexpr\fontdimen22\textfont2\relax]
\matrix(mymatrix)[matrix of math nodes, left delimiter={[}, 
right delimiter={]}, anchor=center, row sep=1pt, column sep=1pt, outer sep=-2pt, nodes={text width=12pt, align=center}, ampersand replacement=\&]
{5 \& 1 \& -1 \& 2 \\
2 \& 3 \& 0 \& 7 \\
4 \& -2 \& 3 \& 1 \\};
\draw [draw=none, fill=red!50, fill opacity=0.4] (mymatrix-1-2.north west) rectangle (mymatrix-3-2.south east);
\end{tikzpicture}
\begin{bmatrix}
0 \\
\color{red}{1} \\
0 \\
0
\end{bmatrix} 
&=
\begin{tikzpicture}[baseline=-\the\dimexpr\fontdimen22\textfont2\relax]
\matrix(mymatrix)[matrix of math nodes, left delimiter={[}, 
right delimiter={]}, anchor=center, row sep=1pt, column sep=1pt, outer sep=-2pt, nodes={text width=12pt, align=center}, ampersand replacement=\&]
{1 \\
3 \\
-2 \\};
\draw [draw=none, fill=red!50, fill opacity=0.4] (mymatrix-1-1.north west) rectangle (mymatrix-3-1.south east);
\end{tikzpicture} \\
\begin{tikzpicture}[baseline=-\the\dimexpr\fontdimen22\textfont2\relax]
\matrix(mymatrix)[matrix of math nodes, left delimiter={[}, 
right delimiter={]}, anchor=center, row sep=1pt, column sep=1pt, outer sep=-2pt, nodes={text width=12pt, align=center}, ampersand replacement=\&]
{5 \& 1 \& -1 \& 2 \\
2 \& 3 \& 0 \& 7 \\
4 \& -2 \& 3 \& 1 \\};
\draw [draw=none, fill=red!50, fill opacity=0.4] (mymatrix-1-2.north west) rectangle (mymatrix-3-2.south east);
\draw [draw=none, fill=Green!50, fill opacity=0.4] (mymatrix-1-1.north west) rectangle (mymatrix-3-1.south east);
\draw [draw=none, fill=blue!50, fill opacity=0.4] (mymatrix-1-3.north west) rectangle (mymatrix-3-3.south east);
\draw [draw=none, fill=gray, fill opacity=0.4] (mymatrix-1-4.north west) rectangle (mymatrix-3-4.south east);
\end{tikzpicture}
\begin{bmatrix}
\color{Green}{-1} \\
\color{red}{2} \\
\color{blue}{3} \\
0
\end{bmatrix} 
&= 
\begin{tikzpicture}[baseline=-\the\dimexpr\fontdimen22\textfont2\relax]
\matrix(mymatrix)[matrix of math nodes, left delimiter={[}, 
right delimiter={]}, anchor=center, row sep=1pt, column sep=1pt, outer sep=-2pt, nodes={text width=12pt, align=center}, ampersand replacement=\&]
{5 \& 1 \& -1 \& 2 \\
2 \& 3 \& 0 \& 7 \\
4 \& -2 \& 3 \& 1 \\};
\draw [draw=none, fill=red!50, fill opacity=0.4] (mymatrix-1-2.north west) rectangle (mymatrix-3-2.south east);
\draw [draw=none, fill=Green!50, fill opacity=0.4] (mymatrix-1-1.north west) rectangle (mymatrix-3-1.south east);
\draw [draw=none, fill=blue!50, fill opacity=0.4] (mymatrix-1-3.north west) rectangle (mymatrix-3-3.south east);
\draw [draw=none, fill=gray, fill opacity=0.4] (mymatrix-1-4.north west) rectangle (mymatrix-3-4.south east);
\end{tikzpicture}
\left(
\begin{bmatrix}
\color{Green}{-1} \\
0 \\
0 \\
0
\end{bmatrix} 
+
\begin{bmatrix}
0 \\
\color{red}{2} \\
0 \\
0
\end{bmatrix} 
+
\begin{bmatrix}
0 \\
0 \\
\color{blue}{3} \\
0
\end{bmatrix} 
+
\begin{bmatrix}
0 \\
0 \\
0 \\
0
\end{bmatrix} 
\right) \\
&=
(\textcolor{Green}{-1})
\begin{tikzpicture}[baseline=-\the\dimexpr\fontdimen22\textfont2\relax]
\matrix(mymatrix)[matrix of math nodes, left delimiter={[}, 
right delimiter={]}, anchor=center, inner sep=3pt, outer sep=-2pt, nodes={text width=12pt, align=center}, ampersand replacement=\&]
{5 \\
2 \\
4 \\};
\draw [draw=none, fill=Green!50, fill opacity=0.4] (mymatrix-1-1.north west) rectangle (mymatrix-3-1.south east);
\end{tikzpicture}
+
(\textcolor{red}{2})
\begin{tikzpicture}[baseline=-\the\dimexpr\fontdimen22\textfont2\relax]
\matrix(mymatrix)[matrix of math nodes, left delimiter={[}, 
right delimiter={]}, anchor=center, inner sep=3pt, outer sep=-2pt, nodes={text width=12pt, align=center}, ampersand replacement=\&]
{1 \\
3 \\
-2 \\};
\draw [draw=none, fill=red!50, fill opacity=0.4] (mymatrix-1-1.north west) rectangle (mymatrix-3-1.south east);
\end{tikzpicture}
+
(\textcolor{blue}{3})
\begin{tikzpicture}[baseline=-\the\dimexpr\fontdimen22\textfont2\relax]
\matrix(mymatrix)[matrix of math nodes, left delimiter={[}, 
right delimiter={]}, anchor=center, inner sep=3pt, outer sep=-2pt, nodes={text width=12pt, align=center}, ampersand replacement=\&]
{-1 \\
0 \\
3 \\};
\draw [draw=none, fill=blue!50, fill opacity=0.4] (mymatrix-1-1.north west) rectangle (mymatrix-3-1.south east);
\end{tikzpicture}
+
0
\begin{tikzpicture}[baseline=-\the\dimexpr\fontdimen22\textfont2\relax]
\matrix(mymatrix)[matrix of math nodes, left delimiter={[}, 
right delimiter={]}, anchor=center, inner sep=3pt, outer sep=-2pt, nodes={text width=12pt, align=center}, ampersand replacement=\&]
{2 \\
7 \\
1 \\};
\draw [draw=none, fill=gray, fill opacity=0.4] (mymatrix-1-1.north west) rectangle (mymatrix-3-1.south east);
\end{tikzpicture} \\
&=
\begin{bmatrix}
-6 \\
4 \\
1
\end{bmatrix}
\end{align*}
\begin{exmp}
Show that $\vec{h} = (2,4,3)^T$ cannot be written as a linear combination of $\vec{v}^{(1)} = (-1, 0, 1)^T$ and $\vec{v}^{(2)} = (1, 1, 0)^T$.
\end{exmp}
\begin{solution}
Following the above discussion, the objective is equivalent to showing that the linear system
\begin{align*}
\begin{bmatrix}
-1 & 1 \\
0 & 1 \\
1 & 0
\end{bmatrix}
\begin{bmatrix}
c_1 \\
c_2
\end{bmatrix}
=
\begin{bmatrix}
2 \\
4 \\
3
\end{bmatrix}
\end{align*}
has no solution. We can apply the method of Gaussian Elimination as demonstrated in Section \ref{subsection:SolLinSysGauss}, which leads to
\begin{align*}
\left[\begin{array}{@{\,}wc{10pt}wc{10pt}|wc{10pt}@{}}
-1 & 1 & 2\\
0 & 1 & 4\\
1 & 0 & 3
\end{array}\right] 
&\to
\left[\begin{array}{@{\,}wc{10pt}wc{10pt}|wc{10pt}@{}}
1 & 0 & 3 \\
0 & 1 & 4 \\
-1 & 1 & 2
\end{array}\right] & R_1 \leftrightarrow R_3 \\
&\to
\left[\begin{array}{@{\,}wc{10pt}wc{10pt}|wc{10pt}@{}}
1 & 0 & 3 \\
0 & 1 & 4 \\
0 & 0 & 1
\end{array}\right] & R_3 + R_1 - R_2 \to R_3
\end{align*}
The last row is inconsistent and hence there is no solution to the linear system and $\vec{h}$ cannot be expressed by a linear combination of $\vec{v}^{(1)}$ and $\vec{v}^{(2)}$.
\end{solution}
With the idea of linear combination, we can define the \keywordhl{span} generated by a \textit{finite} set of vectors.
\begin{defn}[Span]
\label{defn:span}
The span of $q$ vectors in a set $\mathcal{\beta} = \{\vec{v}^{(1)}, \vec{v}^{(2)}, \vec{v}^{(3)}, \ldots, \allowbreak \vec{v}^{(q)}\}$ where all of them are from the same vector space $\mathcal{V}$, i.e. $\vec{v}^{(j)} \in \mathcal{V}$ for $j = 1, 2, \ldots, q$, is another set that contains all their possible linear combinations as given by (\ref{eqn:linearcomb}) in Definition \ref{defn:linearcomb}, and is denoted by
\begin{align}
\mathcal{B} = \text{span}(\mathcal{\beta}) = \begin{aligned}
&\left\{\sum_{j=1}^{q} c_j\vec{v}^{(j)} \mid \text{for all possible values}\right. \\    
&\left.\text{of the scalars $c_j$ with $\vec{v}^{(j)} \in \mathcal{\beta}$}\right\}
\end{aligned}
\end{align}
Again we will limit ourselves to the cases where the scalar coefficients $c_j$ are real and $q$ has to be finite. If the $\vec{v}^{(j)}$ are from the real $n$-space, i.e. $\vec{v}^{(j)} \in \mathbb{R}^n$, then as suggested by Properties \ref{proper:linearcombmatrix}, the span can be thought in the form of 
\begin{align}
\mathcal{B} = \text{span}(\mathcal{\beta}) = \{A\vec{x} \mid \text{for any } \vec{x} \in \mathbb{R}^q\}
\end{align}
with $A = [\vec{v}^{(1)}|\vec{v}^{(2)}|\cdots|\vec{v}^{(q)}]$ is an $n \times q$ matrix and
$\vec{x} = (c_1, c_2, \ldots, c_q)^T$ being the coefficient vector.
\end{defn}
For example, the span of $\mathcal{\beta}_1 = \{(-1,1)^T\}$ is simply $t(-1,1)^T$ where $-\infty < t < \infty$, or the line $y = -x$. The span of $\mathcal{\beta}_2 = \{(1,0,2)^T, (0,1,-1)^T\}$ (notice that the two vectors are not a constant multiple of each other and thus non-parallel) is $s(1,0,2)^T + t(0,1,-1)^T$ where $-\infty < s,t < \infty$, or represented by the plane $2x - y - z = 0$ (see Section \ref{section:vecgeohighdim}). Adding more vectors to the \textit{spanning set} does not always imply that the corresponding span will be larger. For example, the span of $\mathcal{\beta}_3 = \{(1,0)^T, (0,1)^T\}$ and $\mathcal{\beta}_4 = \{(1,0)^T, (0,1)^T, (1,1)^T, (1,-1)^T\}$ are both apparently $\mathbb{R}^2$. This issue will be addressed in the next subsection.
\begin{exmp}
\label{exmp:S3S4}
Show that any vector in $\mathbb{R}^2$ can be written as infinitely many different linear combinations of the four vectors in the set $\mathcal{\beta}_4$ mentioned above.
\end{exmp}
\begin{solution}
This is to decide if the linear system
\begin{align*}
\begin{bmatrix}
1 & 0 & 1 & 1 \\
0 & 1 & 1 & -1
\end{bmatrix}
\begin{bmatrix}
c_1 \\
c_2 \\
c_3 \\
c_4
\end{bmatrix}
=
\begin{bmatrix}
x \\
y
\end{bmatrix}    
\end{align*}
has infinitely many solutions for any pair of $(x,y)$. The augmented form
\begin{align*}
\left[\begin{array}{@{}cccc|c@{}}
1 & 0 & 1 & 1 & x \\
0 & 1 & 1 & -1 & y
\end{array}\right]
\end{align*}
is already in reduced row echelon form. There is a corresponding pivot for both $x$ and $y$ in the first two columns, and no zero row is present, which means that there would not be any inconsistency and we can always construct a family of solutions by setting the non-pivotal unknowns to be free variables, let's say $c_3 = s$ and $c_4 = t$. Then we have $c_1 = x - s - t$, $c_2 = y - s + t$ from the rows. As a result, any linear combination in the form of
\begin{align*}
(x-s-t)(1,0)^T + (y-s+t)(0,1)^T + s(1,1)^T + t(1,-1)^T
\end{align*}
will produce the vector $(x,y)^T$ with any value of $s$ and $t$ as desired, and there are infinitely many of them. This example shows that a vector (in this case any arbitrary vector of $\mathbb{R}^2$) can possibly be written as more than one linear combination of the constituent vectors in the spanning set (here $\mathcal{\beta}_4$).
\end{solution}
An essential property of spans is that they are subspaces and vice versa. This fact integrates the top-down (it is a subset of a larger vector space) and bottom-up (it is formed by linear combinations of vectors) views of subspaces.
\begin{proper}
\label{proper:subspace_n_span}
The span of a subset of some vectors in $\mathcal{V}$ is a subspace of $\mathcal{V}$. A subspace $\mathcal{W}$ of $\mathcal{V}$ is always some span (not necessarily unique) of some vectors in $\mathcal{V}$.
\end{proper}
We leave the proof for showing the span $\rightarrow$ subspace direction in the footnote\footnote{We check if the two criteria in Theorem \ref{thm:subspacecriteria} hold for a span. Let the span be the one defined in Definition \ref{defn:span}, then any vector in the span can be written as $\smash{\sum_{j=1}^{q} c_j\vec{v}^{(j)}}$ for some constants $c_j$. Let $\vec{u} = \smash{\sum_{j=1}^{q} a_j\vec{v}^{(j)}} \in \text{span}(\mathcal{\beta})$ and $\vec{v} = \smash{\sum_{j=1}^{q} b_j\vec{v}^{(j)}} \in \text{span}(\mathcal{\beta})$ are both in the span for some sets of constants $a_j$ and $b_j$, then their sum $\vec{u} + \vec{v} = \smash{\sum_{j=1}^{q} a_j\vec{v}^{(j)} + \sum_{j=1}^{q} b_j\vec{v}^{(j)}} = \smash{\sum_{j=1}^{q} (a_j + b_j)\vec{v}^{(j)}} = \smash{\sum_{j=1}^{q} \gamma_j\vec{v}^{(j)}} \in \text{span}(\mathcal{\beta})$ where $\gamma_j = a_j + b_j$ is also in the span, and hence it closed under addition. Similarly, writing $\alpha\vec{v} = \alpha(\smash{\sum_{j=1}^{q} b_j\vec{v}^{(j)}}) = \smash{\sum_{j=1}^{q} (\alpha b_j)\vec{v}^{(j)}}$ shows that $\alpha\vec{v} \in \text{span}(\mathcal{\beta})$ and the span is closed under scalar multiplication and we are done.} 
and that for the subspace $\rightarrow$ span direction in Appendix \ref{section:vecspaceappend}. Subsequently, we say $\mathcal{W} = \text{span}(\mathcal{\beta})$ is a subspace of $\mathcal{V}$ ($\mathcal{W} \subseteq \mathcal{V}$) \textit{generated} by the set $\mathcal{\beta}$ and $\mathcal{\beta}$ is known as a \index{Spanning Set}\index{Generating Set}\keywordhl{spanning/generating set} for $\mathcal{W}$. This duality between subspace and span is consistent when we look at them from a geometric point of view: As mentioned at the end of Section \ref{section:Rnsubspace} before, subspaces can be thought of as "flat shapes", or put differently, "linear objects" of infinite extent; Meanwhile, a span precisely consists of all possible linear combinations of vectors. These spanning vectors represent straight directions that extend infinitely long and also produce a "linear shape" (also see Figure \ref{fig:directsumeachsubspace}). \par
Applying Properties \ref{proper:subspace_n_span} on Definition \ref{defn:span}, we can say that the span generated by the column vectors $\vec{v}^{(j)}$ in $A = [\vec{v}^{(1)}|\vec{v}^{(2)}|\cdots|\vec{v}^{(q)}]$ forms a subspace $\mathcal{W} \subseteq \mathcal{V}$ better known as the \index{Column Space}\keywordhl{column space} of $A$.
\begin{defn}[Column Space]
\label{defn:colspace}
The column space of an $n \times q$ matrix $A$ is the span generated by the $q$ column vectors $\vec{v}^{(j)} \in \mathbb{R}^n$ that comprise $A = [\vec{v}^{(1)}|\vec{v}^{(2)}|\cdots|\vec{v}^{(q)}]$ as suggested in Definition \ref{defn:span}.
\end{defn}
Finally, a result related to Properties \ref{proper:subspace_n_span} is noted below.
\begin{proper}
\label{proper:WcontainsspanS}
Any subspace of $\mathcal{V}$ that contains a subset $\mathcal{B}'$ of some vectors in $\mathcal{V}$ also contains $\text{span}(\mathcal{B}')$.
\end{proper}
%\begin{proof}
%Let $\mathcal{S} = \{\vec{u}_1, \vec{u}_2, \ldots, \vec{u}_q\}$. For any vector $\vec{v} \in \text{span}(\mathcal{S})$, by Definition \ref{defn:span}, it can be written as some linear combination $\vec{v} = c_1\vec{u}_1 + c_2\vec{u}_2 + \cdots + c_q\vec{u}_q$ where $c_j$ are some constants and $\vec{u}_j \in \mathcal{S}$. Denote the subspace that contains $\mathcal{S}$ by $\mathcal{W}$. Since $\mathcal{S} \subseteq \mathcal{W}$, $\vec{u}_1, \vec{u}_2 \ldots, \vec{u}_q \in \mathcal{W}$ as well. By recursively applying the alternative version of Theorem \ref{thm:subspacecriteria} to add up the $\vec{u}_j$\footnote{By the theorem, $c_1\vec{u}_1 + c_2\vec{u}_2$ is in the subspace. Using the theorem again, $(c_1\vec{u}_1 + c_2\vec{u}_2) + c_3\vec{u}_3$ is also in the subspace, and so on.}, $\vec{v} = c_1\vec{u}_1 + c_2\vec{u}_2 + \cdots + c_q\vec{u}_q$ is shown to be included in $\mathcal{W}$. Since this can be done for any $\vec{v} \in \text{span}(\mathcal{S})$, $\text{span}(\mathcal{S}) \subseteq \mathcal{W}$.
%\end{proof}

%

\subsection{Linear Independence, CR Factorization}
\label{section:linearind}

Another key concept in this chapter is \textit{linear independence}, which has profound implications in Linear Algebra. Given a set of vectors, if every one of them cannot be expressed as a linear combination of other members, or speaking loosely, each of them is not "dependent" on other vectors, then such a set of vectors is said to be \index{Linearly Independent}\keywordhl{linearly independent}. Otherwise, if at least one of them can be expressed as some linear combination of other vectors, then the set is known as \index{Linearly Dependent}\keywordhl{linearly dependent}.\par
To check the linear independence of $q$ vectors, one may indeed directly show that for every vector $\vec{v}^{(j)}$ in the set, $j = 1,2,3,\ldots,q$, it cannot be written as the linear combination of other vectors $\vec{v}^{(k)}$ in the set, $k \neq j$. A slightly easier way is to look at the linear combination of just the first $j-1$ vectors (from $\vec{v}^{(1)}$ up to $\vec{v}^{(j-1)}$) for $\vec{v}^{(j)}$. However, it is very tedious if the amount of vectors is large. Fortunately, we have a theorem that significantly simplifies our work.
\begin{thm}
\label{thm:linearindep}
For a set of vectors $\mathcal{\beta} = \{\vec{v}^{(1)}, \vec{v}^{(2)}, \vec{v}^{(3)}, \ldots, \vec{v}^{(q)}\}$ where $\vec{v}^{(j)} \in \mathcal{V}$, $j=1,2,\ldots,q$ that are from the same vector space, they are linearly independent if and only if, the equation 
\begin{align}
c_1\vec{v}^{(1)} + c_2\vec{v}^{(2)} + c_3\vec{v}^{(3)} + \cdots + c_q\vec{v}^{(q)} = \textbf{0}    
\end{align}
has the trivial solution where all the (real [complex] for a real [complex] vector space) coefficients are zeros ($c_j = \textbf{0}$) as its unique solution. Using the language in Properties \ref{proper:linearcombmatrix}, if $\vec{v}^{(j)} \in \mathbb{R}^n$ come from the real $n$-space, it means that the homogeneous linear system $A\vec{x} = \textbf{0}$ where $A = [\vec{v}^{(1)}|\vec{v}^{(2)}|\vec{v}^{(3)}|\cdots|\vec{v}^{(q)}]$ is an $n \times q$ matrix, only has the trivial solution $\vec{x} = \textbf{0}$.
\end{thm}
\begin{proof}
The "if" direction: We need to show that $c_j = \textbf{0}$ being the only solution to $c_1\vec{v}^{(1)} + c_2\vec{v}^{(2)} + c_3\vec{v}^{(3)} + \cdots + c_q\vec{v}^{(q)} = \textbf{0}$ implies that $\vec{v}^{(1)}, \vec{v}^{(2)}, \vec{v}^{(3)}, \ldots, \vec{v}^{(q)}$ are linearly independent. We can prove the contrapositive where the opposite of the conclusion, $\vec{v}^{(1)}, \vec{v}^{(2)}, \vec{v}^{(3)}, \ldots, \vec{v}^{(q)}$ are linearly dependent, implies the opposite of the premise, i.e. there is non-trivial solution to the equation. This requires that at least one of these vectors, without the loss of generality let's say $\vec{v}^{(1)}$, can be written as the linear combination of other vectors in the form of
\begin{align*}
\vec{v}^{(1)} = a_2\vec{v}^{(2)} + a_3\vec{v}^{(3)} + \cdots + a_q\vec{v}^{(q)}
\end{align*}
Rearranging gives 
\begin{align*}
\vec{v}^{(1)} - a_2\vec{v}^{(2)} - a_3\vec{v}^{(3)} - \cdots - a_q\vec{v}^{(q)} = \textbf{0}
\end{align*}
which shows that the coefficients $c_1 = 1, c_2 = -a_2, c_3 = -a_3, \ldots, c_q = -a_q$ is another solution other than $c_j = \textbf{0}$ to $c_1\vec{v}^{(1)} + c_2\vec{v}^{(2)} + c_3\vec{v}^{(3)} + \cdots + c_q\vec{v}^{(q)} = \textbf{0}$ (concerning $c_1 = 1 \neq 0$ particularly). \par
The "only if" direction: We want to show the converse that linear independence of $\vec{v}^{(1)}, \vec{v}^{(2)}, \vec{v}^{(3)}, \ldots, \vec{v}^{(q)}$ only permits $c_j = \textbf{0}$ as the unique solution to $c_1\vec{v}^{(1)} + c_2\vec{v}^{(2)} + c_3\vec{v}^{(3)} + \cdots + c_q\vec{v}^{(q)} = \textbf{0}$. To do so, we can again resort to its contrapositive, i.e.\ the existence of an alternative solution of $c_j = a_j$ which are not all zeros to the equation in question, means that the vectors $\vec{v}^{(1)}, \vec{v}^{(2)}, \vec{v}^{(3)}, \ldots, \vec{v}^{(q)}$ are linearly dependent. Choose one of the $a_j$ that is not zero and denote it by $a_k$, then
\begin{align*}
a_1\vec{v}^{(1)} + \cdots + a_{k-1}\vec{v}^{(k-1)} + a_k\vec{v}^{(k)} + a_{k+1}\vec{v}^{(k+1)} + \cdots + a_q\vec{v}^{(q)} = \textbf{0} \\
\vec{v}^{(k)} = -\frac{a_1}{a_k}\vec{v}^{(1)} - \cdots - \frac{a_{k-1}}{a_k}\vec{v}^{(k-1)} - \frac{a_{k+1}}{a_k}\vec{v}^{(k+1)} - \cdots - \frac{a_q}{a_k}\vec{v}^{(q)} 
\end{align*}
where we have divided the equation by the non-zero $a_k$ to avoid dividing by zero and rearranged it to show that $\vec{v}^{(k)}$ can be written in some linear combination of other vectors $\vec{v}^{(j)}$, $j \neq k$ as shown above, and thus vectors in $\mathcal{\beta}$ are linearly dependent. 
\end{proof}
As a corollary, any set containing the zero vector $\textbf{0}$ must be linearly dependent. (Why?)\footnote{For any such a set $\mathcal{\beta}_0 = \{\vec{u}_1, \vec{u}_2, \ldots, \textbf{0}\}$, the linear system $c_1\vec{u}_1 + c_2\vec{u}_2 + \cdots + c_0\textbf{0} = \textbf{0}$ has a family of infinitely many solution with $c_j = 0$ for $j \neq 0$ and any value of $c_0$, which by Theorem \ref{thm:linearindep} they are linearly dependent.}
\begin{exmp}
\label{exmp:exmplinearindep}
Determine if $\vec{u} = (1,2,1)^T$, $\vec{v} = (3,4,2)^T$, $\vec{w} = (6,8,1)^T$ are linearly independent.
\end{exmp}
By Theorem \ref{thm:linearindep}, this is equivalent to deciding if $A\vec{x} = \textbf{0}$, where $A = [\vec{u}|\vec{v}|\vec{w}]$ has the trivial solution as the only solution. With the help of Theorem \ref{thm:sqlinsysunique}, we know that it is equivalent to checking if $\text{det}(A)$ is zero or not. Since
\begin{align*}
|A| &=
\begin{vmatrix}
1 & 3 & 6\\
2 & 4 & 8 \\
1 & 2 & 1
\end{vmatrix} = 6 \neq 0
\end{align*}
We conclude that $A\vec{x} = \textbf{0}$ only has the trivial solution $\vec{x} = \textbf{0}$ and these three vectors are linearly independent. \par
Short Exercise: Redo the above example with $\vec{u} = (1,1,3)^T$, $\vec{v} = (1,3,2)^T$, $\vec{w} = (2,8,3)^T$.\footnote{The determinant of $A = [\vec{u}|\vec{v}|\vec{w}]$ in the case is $\abs{A} = 0$, and hence by the remark for Theorem \ref{thm:sqlinsysunique} the linear system $A\vec{x} = \textbf{0}$ has infinitely many solutions, and these three vectors are linearly dependent by Theorem \ref{thm:linearindep}.}\par
Including our earlier discussion in Section \ref{subsection:SolLinSysGauss}, Theorem \ref{thm:linearindep} gives some interesting results.
\begin{enumerate}
\item If there are $q$ vectors of $\mathbb{R}^p$ in a set and $p < q$, i.e. the number of vectors is more than their dimension, then $A = [\vec{v}^{(1)}|\vec{v}^{(2)}|\vec{v}^{(3)}|\cdots|\vec{v}^{(q)}]$ is a $p \times q$ matrix which has more columns ($q$) than rows ($p$). In this case, $A\vec{x} = \textbf{0}$ must have at least one free variable and thus infinitely many solutions, hence the vectors must be linearly dependent.
\item Otherwise ($p \geq q$), we can solve $A\vec{x} = \textbf{0}$ by Gaussian Elimination to see if it only has the trivial solution. If so [not], the vectors are linearly independent [dependent]. Alternatively, if $A$ is a square matrix, then we may check if its determinant is non-zero, just like what have been done in Example \ref{exmp:exmplinearindep}. Gaussian Elimination still works for any square matrix, and in the case of linear independence [dependence], $A$ will [not] be reduced to an identity matrix.
\end{enumerate}
In many cases, the number of vectors are indeed not equal to their dimension so the method of using determinant to check linear independence in the last example does not apply and we need to resort to Gaussian Elimination. In fact, Gaussian Elimination can disclose more information than just if a set of vectors is linearly (in)dependent as an entirety in both cases, but also how exactly these vectors are dependent on each other, soon to be explained. Before doing so, we note that the above observations lead to an extension of Theorem \ref{thm:equiv2}.
\begin{thm}[Equivalence Statement, ver.\ 3]
\label{thm:equiv3}
For an $n \times n$ real square matrix $A$, the followings are equivalent:
\begin{enumerate}[label=(\alph*)]
\item $A$ is invertible, i.e.\ $A^{-1}$ exists,
\item $\det(A) \neq 0$,
\item The reduced row echelon form of $A$ is the identity $I$,
\item The linear system $A\vec{x} = \vec{h}$ has a unique solution for any $\vec{h}$, particularly $A\vec{x} = \textbf{0}$ has only the trivial solution $\vec{x} = \textbf{0}$,
\item The $n$ column vectors $\vec{v}^{(1)}, \vec{v}^{(2)}, \vec{v}^{(3)}, \ldots, \vec{v}^{(n)}$ of $\mathbb{R}^n$ as in $A = [\vec{v}^{(1)}|\vec{v}^{(2)}|\vec{v}^{(3)}|\cdots|\vec{v}^{(n)}]$ are linearly independent.
\end{enumerate}
\end{thm}
We now revisit the procedure of Gaussian Elimination and show that it actually explicitly reveals the so-called \index{Dependence Relation}\keywordhl{dependence relations} between vectors (how a vector can be written as some linear combination of other vectors) as a by-product when determining linear (in)dependence. Let's illustrate this with an example: Given
\begin{align*}
\vec{v}^{(1)} &= (1,2,1)^T \\
\vec{v}^{(2)} &= (2,4,2)^T \\
\vec{v}^{(3)} &= (1,-1,-1)^T \\
\vec{v}^{(4)} &= (2,1,0)^T \\
\vec{v}^{(5)} &= (0,-3,-2)^T
\end{align*}
Note that the vectors are related by these dependence relations: $\vec{v}^{(2)} = 2\vec{v}^{(1)}$, $\vec{v}^{(4)} = \vec{v}^{(1)} + \vec{v}^{(3)}$ and $\vec{v}^{(5)} = -\vec{v}^{(1)} + \vec{v}^{(3)}$, while $\vec{v}^{(1)}$ and $\vec{v}^{(3)}$ are themselves linearly independent of each other. Construct 
\begin{align*}
A &= [\vec{v}^{(1)}|\vec{v}^{(2)}|\vec{v}^{(3)}|\vec{v}^{(4)}|\vec{v}^{(5)}] \\
&= 
\begin{bmatrix}
1 & 2 & 1 & 2 & 0 \\
2 & 4 & -1 & 1 & -3\\
1 & 2 & -1 & 0 & -2
\end{bmatrix}
\end{align*}
by concatenating the five vectors column by column. Now we carry out Gaussian Elimination as follows.
\begin{align*}
\left[\begin{array}{@{\,}wc{10pt}wc{10pt}wc{10pt}wc{10pt}wc{10pt}@{\,}}
1 & 2 & 1 & 2 & 0 \\
2 & 4 & -1 & 1 & -3\\
1 & 2 & -1 & 0 & -2
\end{array}\right]
&\to \left[\begin{array}{@{\,}wc{10pt}wc{10pt}wc{10pt}wc{10pt}wc{10pt}@{\,}}
1 & 2 & 1 & 2 & 0 \\
0 & 0 & -3 & -3 & -3\\
0 & 0 & -2 & -2 & -2
\end{array}\right] &
\begin{aligned}
R_2 - 2R_1 &\to R_2 \\
R_3 - R_1 &\to R_3 
\end{aligned} \\
&\to \left[\begin{array}{@{\,}wc{10pt}wc{10pt}wc{10pt}wc{10pt}wc{10pt}@{\,}}
1 & 2 & 1 & 2 & 0 \\
0 & 0 & 1 & 1 & 1 \\
0 & 0 & -2 & -2 & -2
\end{array}\right] &
-\frac{1}{3}R_2 \to R_2 \\
&\to \left[\begin{array}{@{\,}wc{10pt}wc{10pt}wc{10pt}wc{10pt}wc{10pt}@{\,}}
1 & 2 & 1 & 2 & 0 \\
0 & 0 & 1 & 1 & 1 \\
0 & 0 & 0 & 0 & 0
\end{array}\right] &
R_3 + 2R_2 \to R_3 \\
&\to \left[\begin{array}{@{\,}wc{10pt}wc{10pt}wc{10pt}wc{10pt}wc{10pt}@{\,}}
1 & 2 & 0 & 1 & -1 \\
0 & 0 & 1 & 1 & 1 \\
0 & 0 & 0 & 0 & 0
\end{array}\right] &
R_1 - R_2 \to R_1
\end{align*}
The new columns in the above RREF matrix $A_{\text{rref}} = [\vec{v}^{(1)'}|\vec{v}^{(2)'}|\vec{v}^{(3)'}|\vec{v}^{(4)'}|\vec{v}^{(5)'}]$ follow the exact same dependence relation: $\vec{v}^{(2)'} = 2\vec{v}^{(1)'}$, $\vec{v}^{(4)'} = \vec{v}^{(1)'} + \vec{v}^{(3)'}$ and $\vec{v}^{(5)'} = -\vec{v}^{(1)'} + \vec{v}^{(3)'}$. $\vec{v}^{(1)'}$ and $\vec{v}^{(3)'}$ are clearly still linearly independent of each other too. This demonstrates that dependence relations (and by extension linear independent vectors) are preserved under elementary row operations during Gaussian Elimination. (The detailed argument is put in Appendix \ref{section:vecspaceappend}) We now introduce a helper theorem so that we may proceed.
\begin{thm}[Plus/Minus Theorem]
\label{thm:plusminus}
Let $\mathcal{\beta} = \{\vec{v}^{(1)}, \vec{v}^{(2)}, \vec{v}^{(3)}, \ldots, \vec{v}^{(q)}\}$ be a set of vectors in the vector space $\mathcal{V}$, i.e. $\vec{v}^{(j)} \in \mathcal{V}$, $1 \leq j \leq q$, we have the following two results:
\begin{enumerate}[label=(\alph*)]
    \item If $\mathcal{\beta}$ is a linearly independent set and $\vec{v}$ is not in $\text{span}(\mathcal{\beta})$, then $\mathcal{\beta} \cup \{\vec{v}\}$ formed after inserting $\vec{v}$ into the set is still linearly independent,
    \item If $\vec{w}$ is a vector in some other set (also denoted by $\mathcal{\beta}$) that can be expressed as a linear combination of other vectors in the now linearly dependent set, then the new set $\mathcal{\beta} - \{\vec{w}\}$ remained after removing $\vec{w}$ from $\mathcal{\beta}$ has the same span, i.e.
    \begin{align*}
    \text{span}(\mathcal{\beta}) = \mathcal{B} =\text{span}(\mathcal{\beta} - \{\vec{w}\})
    \end{align*}
\end{enumerate}
\end{thm}
\begin{proof}
We include the proof for (a) as a footnote since (a) is less of a concern.\footnote{We will prove the contrapositive that given $\mathcal{\beta}$ is a linearly independent set then, $\mathcal{\beta} \cup \{\vec{v}\}$ is linearly dependent only if $\vec{v} \coloneq \vec{v}^{(q+1)}$ is in $\text{span}(\mathcal{\beta})$: if $\mathcal{\beta} \cup \{\vec{v}\}$ is linearly dependent, then there is non-trivial solution $c_j = d_j$ where $d_j$ are not all zeros to the equation $c_1\vec{v}^{(1)} + c_2\vec{v}^{(2)} + \cdots + c_q\vec{v}^{(q)} + c_{q+1}\vec{v}^{(q+1)} = \textbf{0}$ by Theorem \ref{thm:linearindep}. Since $\mathcal{\beta}$ is required to be linearly independent, $d_{q+1} \neq 0$, for otherwise $d_{q+1} = 0$ and then at least one of the $c_j = d_j$, $j \neq v$, will be non-zero due to the linear dependence of the union set $\mathcal{\beta} \cup \{\vec{v}\}$ and lead to a non-trivial solution to $c_1\vec{v}^{(1)} + c_2\vec{v}^{(2)} + \cdots + c_q\vec{v}^{(q)} = \textbf{0}$ instead, which contradicts the assumed linear independence of $\mathcal{\beta}$ alone, so we have $d_1\vec{v}^{(1)} + d_2\vec{v}^{(2)} + \cdots + d_q\vec{v}^{(q)} + d_{q+1}\vec{v}^{(q+1)} = \textbf{0}$ and because $d_{q+1} \neq 0$ we can obtain
\begin{align*}
\vec{v}^{(q+1)} &= -\frac{1}{d_{q+1}}(d_1\vec{v}^{(1)} + d_2\vec{v}^{(2)} + \cdots + d_q\vec{v}^{(q)})
\end{align*}
showing that $\vec{v}^{(q+1)}$ is a linear combination of $\vec{v}^{(j)} \in \mathcal{\beta}$, $1 \leq j \leq q$.} For (b), assign the linearly dependent vector $\vec{v}^{(k)}$ that is being removed where $1 \leq k \leq q$ as $\vec{w}$. We can write $\vec{w} = a_1\vec{v}^{(1)} + a_2\vec{v}^{(2)} + \cdots + a_{k-1}\vec{v}^{(k-1)} + a_{k+1}\vec{v}^{(k+1)} + \cdots + a_q\vec{v}^{(q)}$ using other vectors in $\mathcal{\beta}$ where $a_j$, $j \neq k$ are some constants. For any vector $\vec{v} = b_1\vec{v}^{(1)} + b_2\vec{v}^{(2)} + \cdots + b_{k-1}\vec{v}^{(k-1)} + b_k\vec{v}^{(k)} + b_{k+1}\vec{v}^{(k+1)} + \cdots + b_q\vec{v}^{(q)}$ in $\text{span}(\mathcal{\beta})$ with $b_j$ being some coefficients, it can be rewritten as a linear combination of the remaining vectors:
\begin{align*}
\vec{v} &= b_1\vec{v}^{(1)} + b_2\vec{v}^{(2)} + \cdots + b_{k-1}\vec{v}^{(k-1)} + b_k\vec{v}^{(k)} + b_{k+1}\vec{v}^{(k+1)} + \cdots + b_q\vec{v}^{(q)} \\
&= b_1\vec{v}^{(1)} + b_2\vec{v}^{(2)} + \cdots + b_{k-1}\vec{v}^{(k-1)} + b_{k+1}\vec{v}^{(k+1)} + \cdots + b_q\vec{v}^{(q)} + b_k\vec{v}^{(k)} \\
&= 
\begin{aligned}
& b_1\vec{v}^{(1)} + b_2\vec{v}^{(2)} + \cdots + b_{k-1}\vec{v}^{(k-1)} + b_{k+1}\vec{v}^{(k+1)} + \cdots + b_q\vec{v}^{(q)} \\
& + b_k(a_1\vec{v}^{(1)} + a_2\vec{v}^{(2)} + \cdots + a_{k-1}\vec{v}^{(k-1)} + a_{k+1}\vec{v}^{(k+1)} + \cdots + a_q\vec{v}^{(q)}) \\    
\end{aligned} \\
&=
\begin{aligned}
&(b_1 + b_ka_1) \vec{v}^{(1)} + (b_2 + b_ka_2) \vec{v}^{(2)} + (b_{k-1} + b_ka_{k-1})\vec{v}^{(k-1)} \\
& + (b_{k+1} + b_ka_{k+1}) \vec{v}^{(k+1)} + \cdots + (b_q + b_ka_q)\vec{v}^{(q)}
\end{aligned} \\
&\in \text{span}(\mathcal{\beta} - \{\vec{v}^{(k)}\}) = \text{span}(\mathcal{\beta} - \{\vec{w}\})
\end{align*}
Therefore for all $\vec{v} \in \text{span}(\mathcal{\beta})$, $\vec{v} \in \text{span}(\mathcal{\beta} - \{\vec{w}\})$ and hence $\text{span}(\mathcal{\beta}) \subseteq \text{span}(\mathcal{\beta} - \{\vec{w}\})$. It is trivial to show $\text{span}(\mathcal{\beta} - \{\vec{w}\}) \subseteq \text{span}(\mathcal{\beta})$, and thus $\text{span}(\mathcal{\beta}) = \text{span}(\mathcal{\beta} - \{\vec{w}\}) = \mathcal{B}$. This part of the theorem is very relevant to the span of sets $\mathcal{\beta}_3$ and $\mathcal{\beta}_4$ in the previous Example \ref{exmp:S3S4}.
\end{proof}
With these results, Gaussian Elimination enables us to carry out the \index{Column-row Factorization}\index{CR Factorization}\keywordhl{Column-row (CR) Factorization} over a matrix. First, note that by part (b) of Theorem \ref{thm:plusminus} above, the column space (Definition \ref{defn:colspace}) of a matrix $A$ can be expressed as the span of a \index{Minimal Generating Set}\keywordhl{minimal generating set} by removing linearly dependent column vectors in $A$ (which does not change the span and still generates the same subspace) and only keeping the linearly independent ones. Meanwhile, the manners of linear (in)dependence over the column vectors of $A$ can be inferred by Gaussian Elimination as just demonstrated in the last example. After obtaining the minimal generating set, we can express any vector in the column space as a unique linear combination of these linearly independent vectors inside the set due to the following properties.
\begin{proper}
\label{proper:lincombofspan}
For a set of vectors $\mathcal{\beta} = \{\vec{v}^{(1)}, \vec{v}^{(2)}, \vec{v}^{(3)}, \ldots, \vec{v}^{(q)}\}$, $\vec{v}^{(j)} \in \mathcal{V}$ for $j = 1,2,\ldots,q$ which are linearly independent, any vector $\vec{v} \in \text{span}(\mathcal{\beta})$ in their span can be written as a unique linear combination of these generating vectors in $\mathcal{\beta}$. Otherwise, if the vectors in $\mathcal{\beta}$ are linearly dependent, there will be infinitely many such linear combinations to assemble $\vec{v}$.
\end{proper}
Again, we will simply provide the proof in a footnote for reference.\footnote{We will show the first part only. Since $\vec{v}$ already belongs to $\text{span}(\mathcal{\beta})$, it must be possible to express $\vec{v}$ as some linear combination(s) of vectors $\vec{v}^{(1)}, \vec{v}^{(2)}, \vec{v}^{(3)}, \ldots, \vec{v}^{(q)}$ in $\mathcal{\beta}$ by Definition \ref{defn:span}. Now it suffices to show that it is unique. Assume the contrary that there are two distinct linear combinations of vectors in $\mathcal{\beta}$ that represent $\vec{v}$, and hence we can express it by
\begin{align*}
\vec{v} &= d_1\vec{v}^{(1)} + d_2\vec{v}^{(2)} + d_3\vec{v}^{(3)} + \cdots + d_q\vec{v}^{(q)} \\
&= g_1\vec{v}^{(1)} + g_2\vec{v}^{(2)} + g_3\vec{v}^{(3)} + \cdots + g_q\vec{v}^{(q)}
\end{align*}
where $d_j$, $g_j$ are two sets of coefficients that are not exactly the same. Subtracting one expression by another leads to
\begin{align*}
\begin{aligned}
&\quad (d_1\vec{v}^{(1)} + d_2\vec{v}^{(2)} + d_3\vec{v}^{(3)} + \cdots + d_q\vec{v}^{(q)}) \\
& -(g_1\vec{v}^{(1)} + g_2\vec{v}^{(2)} + g_3\vec{v}^{(3)} + \cdots + g_q\vec{v}^{(q)})
\end{aligned}
&= \vec{v} - \vec{v} \\
(d_1 - g_1)\vec{v}^{(1)} + (d_2 - g_2)\vec{v}^{(2)} + (d_3 - g_3)\vec{v}^{(3)} + \cdots + (d_q - g_q)\vec{v}^{(q)} &= \textbf{0} 
\end{align*}
Since $d_j$, $g_j$ are assumed to be not completely identical, it is a non-trivial solution to the equation $c_1\vec{v}^{(1)} + c_2\vec{v}^{(2)} + c_3\vec{v}^{(3)} + \cdots + c_q\vec{v}^{(q)} = \textbf{0}$, where $c_j = d_j - g_j$ are not all zeros. This contradicts our assumed linear independence of $\mathcal{\beta}$ and hence the linear combination of $\vec{v}^{(1)}, \vec{v}^{(2)}, \vec{v}^{(3)}, \ldots, \vec{v}^{(q)}$ to generate $\vec{v}$ must be unique.}
%In cases of where $\vec{u}_1, \vec{u}_2, \vec{u}_3, \ldots, \vec{u}_q$ are linearly dependent, start from any linear combination $\vec{v} = d_1\vec{u}_1 + d_2\vec{u}_2 + d_3\vec{u}_3 + \cdots + d_q\vec{u}_q$. By Theorem \ref{thm:linearindep}, we have some non-trivial solution of $c_j$ that are not all zeros to the equation $c_1\vec{u}_1 + c_2\vec{u}_2 + c_3\vec{u}_3 + \cdots + c_q\vec{u}_q = \textbf{0}$. Adding this non-trivial solution times a parameter $t$ to the linear combination we have begun with, gives
%\begin{align*}
%\vec{v} + t\textbf{0} &= 
%\begin{aligned}
%& (d_1\vec{u}_1 + d_2\vec{u}_2 + d_3\vec{u}_3 + \cdots + d_q\vec{u}_q) \\
%&+ t(c_1\vec{u}_1 + c_2\vec{u}_2 + c_3\vec{u}_3 + \cdots + c_q\vec{u}_q)
%\end{aligned} \\
%\vec{v} &= (d_1 + tc_1)\vec{u}_1 + (d_2 + tc_2)\vec{u}_2 + (d_3 + tc_3)\vec{u}_3 + \cdots + (d_q + tc_q)\vec{u}_q
%\end{align*}
%This expresses $\vec{v}$ in infinitely many linear combinations of $\vec{u}_j$ as $t$ is varied.
Return to our example, where
\begin{align*}
A &= [\vec{v}^{(1)}|\vec{v}^{(2)}|\vec{v}^{(3)}|\vec{v}^{(4)}|\vec{v}^{(5)}] \\
&= 
\begin{bmatrix}
1 & 2 & 1 & 2 & 0 \\
2 & 4 & -1 & 1 & -3\\
1 & 2 & -1 & 0 & -2
\end{bmatrix}
\end{align*}
We have found that the corresponding RREF is
\begin{align*}
A_{\text{rref}} =
\begin{bmatrix}
1 & 2 & 0 & 1 & \textcolor{red}{-1} \\
0 & 0 & 1 & 1 & \textcolor{red}{1} \\
0 & 0 & 0 & 0 & 0
\end{bmatrix}
\end{align*}
and thus concluded that the first/third column vectors are linearly independent, and the second/fourth/fifth column vectors are linearly dependent on the first/third ones and can be expressed as a linear combination of them. Now let
\begin{align*}
C = [\vec{v}^{(1)}|\vec{v}^{(3)}] = \begin{bmatrix}
1 & 1 \\
2 & -1 \\
1 & -1
\end{bmatrix}
\end{align*}
using the two linearly independent vectors. By Properties \ref{proper:lincombofspan} and \ref{proper:linearcombmatrix}, each of the $\vec{v}^{(j)}$ can be expressed as a unique linear combination in the form of a matrix product between $C$ and a column vector that contains the coefficients in front of the chosen linear independent vectors that make up the $\vec{v}^{(j)}$. The required column vectors to produce them are exactly the corresponding columns in the RREF which retain the dependence relations, with row(s) of all zeros removed. For instance,
\begin{align*}
\vec{v}^{(5)} &= -\vec{v}^{(1)'} + \vec{v}^{(3)'} \\
\begin{bmatrix}
0 \\
-3 \\
2
\end{bmatrix}
&= -1 
\begin{bmatrix}
1 \\
2 \\
1
\end{bmatrix}
+
\begin{bmatrix}
1 \\
-1 \\
-1
\end{bmatrix} \\
&=
\begin{bmatrix}
1 & 1 \\
2 & -1 \\
1 & -1
\end{bmatrix}
\begin{bmatrix}
\textcolor{red}{-1} \\
\textcolor{red}{1}
\end{bmatrix} & \text{(Properties \ref{proper:linearcombmatrix})} \\
&= [\vec{v}^{(1)}|\vec{v}^{(3)}]\begin{bmatrix}
-1 \\
1
\end{bmatrix} = C \begin{bmatrix}
-1 \\
1
\end{bmatrix}
\end{align*}
Denote
\begin{align*}
R = \begin{bmatrix}
1 & 2 & 0 & 1 & -1 \\
0 & 0 & 1 & 1 & 1 \\
\end{bmatrix}
\end{align*}
which is consisted of the non-zero rows of $A_{\text{rref}}$, then similarly
\begin{align*}
\vec{v}^{(1)} &= 
\begin{bmatrix}
1 \\
2 \\
1
\end{bmatrix}
=
\begin{bmatrix}
1 & 1 \\
2 & -1 \\
1 & -1
\end{bmatrix}
\begin{bmatrix}
1 \\
0
\end{bmatrix}
= CR_1 \\
\vec{v}^{(2)} &= 
\begin{bmatrix}
2 \\
4 \\
2
\end{bmatrix}
=
\begin{bmatrix}
1 & 1 \\
2 & -1 \\
1 & -1
\end{bmatrix}
\begin{bmatrix}
2 \\
0
\end{bmatrix}
= CR_2 \\
\vec{v}^{(3)} &= 
\begin{bmatrix}
1 \\
-1 \\
-1
\end{bmatrix}
=
\begin{bmatrix}
1 & 1 \\
2 & -1 \\
1 & -1
\end{bmatrix}
\begin{bmatrix}
0 \\
1
\end{bmatrix}
= CR_3 \\
\vec{v}^{(4)} &= 
\begin{bmatrix}
2 \\
1 \\
0
\end{bmatrix}
=
\begin{bmatrix}
1 & 1 \\
2 & -1 \\
1 & -1
\end{bmatrix}
\begin{bmatrix}
1 \\
1
\end{bmatrix}
= CR_4
\end{align*}
where $R_j$ is the $j$-th column of $R$. Therefore, 
\begin{align*}
A &= [\vec{v}^{(1)}|\vec{v}^{(2)}|\vec{v}^{(3)}|\vec{v}^{(4)}|\vec{v}^{(5)}] 
= \begin{bmatrix}
1 & 2 & 1 & 2 & 0 \\
2 & 4 & -1 & 1 & -3\\
1 & 2 & -1 & 0 & -2
\end{bmatrix} \\
&= [CR_1|CR_2|CR_3|CR_4|CR_5] \\
&= C([R_1|R_2|R_3|R_4|R_5]) = CR = \begin{bmatrix}
1 & 1 \\
2 & -1 \\
1 & -1
\end{bmatrix}\begin{bmatrix}
1 & 2 & 0 & 1 & -1 \\
0 & 0 & 1 & 1 & 1 \\
\end{bmatrix}
\end{align*}
from the second line to the third line we use the fact that the same matrix multiplied to the left in every column of another matrix can be factored out (why?)\footnote{\label{foot:factorleftmatrix} For an $m \times r$ matrix $A$, and an $r \times n$ matrix $B$, we have, by Definition \ref{defn:matprod} (with a slightly different notation),
\begin{align*}
A[B_1|B_2|\cdots|B_n] &= \begin{bmatrix}
a_{11} & a_{12} & \cdots & a_{1r} \\
a_{21} & a_{22} & & a_{2r} \\
\vdots & & \ddots & \vdots \\
a_{m1} & a_{m2} & \cdots & a_{mr} \\
\end{bmatrix}
\left[\begin{array}{@{\,}c|c|c|c@{\,}}
b_{11} & b_{12} & \cdots & b_{1n} \\
b_{21} & b_{22} & & b_{2n} \\
\vdots & & \ddots & \vdots \\
b_{r1} & b_{r2} & \cdots & b_{rn} \\    
\end{array}\right] \\
&=
\left[\begin{array}{@{\,}c|c|c|c@{\,}}
\sum_{k=1}^r a_{1k}b_{k1} & \sum_{k=1}^r a_{1k}b_{k2} & \cdots & \sum_{k=1}^r a_{1k}b_{kn} \\
\sum_{k=1}^r a_{2k}b_{k1} & \sum_{k=1}^r a_{2k}b_{k2} & & \sum_{k=1}^r a_{2k}b_{kn} \\
\vdots & & \ddots & \vdots \\
\sum_{k=1}^r a_{mk}b_{k1} & \sum_{k=1}^r a_{mk}b_{k2} & \cdots & \sum_{k=1}^r a_{mk}b_{kn} \\
\end{array}\right] \\
&=
[AB_1|AB_2|\cdots|AB_n]
\end{align*}
\vspace{\maxdimen}
where $B_j$ now denotes the $j$-th column of $B$:
\begin{align*}
B_j &= \begin{bmatrix}
b_{1j} \\
b_{2j} \\
\vdots \\
b_{rj} \\
\end{bmatrix}
&
\text{ and hence }
AB_j &= \begin{bmatrix}
a_{11} & a_{12} & \cdots & a_{1r} \\
a_{21} & a_{22} & & a_{2r} \\
\vdots & & \ddots & \vdots \\
a_{m1} & a_{m2} & \cdots & a_{mr} \\
\end{bmatrix}
\begin{bmatrix}
b_{1j} \\
b_{2j} \\
\vdots \\
b_{rj} \\
\end{bmatrix} \\
& & &=
\begin{bmatrix}
\sum_{k=1}^r a_{1k}b_{kj} \\
\sum_{k=1}^r a_{2k}b_{kj} \\
\vdots \\
\sum_{k=1}^r a_{mk}b_{kj} \\
\end{bmatrix} 
\end{align*}} and this is the desired CR Factorization of $A$. In general, for any matrix, its CR Factorization is derived as follows.
\begin{proper}[CR Factorization]
\label{proper:CRFactor}
The Column-Row Factorization of any matrix $A$ that has an RREF of $A_{\text{rref}}$, is given by $A = CR$, where $C$ contains the $r$ linearly independent columns of $A$ at which the $r$ leading $1$s of $A_{\text{rref}}$ are located, and $R$ is simply the first $r$ rows of $A_{\text{rref}}$ that hold these leading $1$s, with all the full-zero rows below removed.
\end{proper}
The $k$-th row of $R$ contains the coefficients in front of the $k$-th column vector in $C$ required to generate each column in the original $A$ matrix.

\begin{exmp}
Show that $\vec{u} = (2,1,-1,1)^T$, $\vec{v} = (1,2,1,-1)^T$, $\vec{w} = (0,1,1,2)^T$ are linearly independent and find the CR Factorization of $A=[\vec{u}|\vec{v}|\vec{w}]$. What if $\vec{w} = (1,-1,-2,2)^T$ instead?
\end{exmp}
\begin{solution}
From Theorem \ref{thm:linearindep}, we need to show that the system $A\vec{x} = \textbf{0}$ has only the trivial solution $\vec{x} = \textbf{0}$, where
\begin{align*}
A = [\vec{u}|\vec{v}|\vec{w}] =
\left[
\begin{array}{@{\,}wc{10pt}wc{10pt}wc{10pt}@{\,}}
2 & 1 & 0 \\
1 & 2 & 1 \\
-1 & 1 & 1 \\
1 & -1 & 2
\end{array}
\right]
\end{align*}
To do so we can apply Gaussian Elimination as below.
\begin{align*}
\left[
\begin{array}{@{\,}wc{10pt}wc{10pt}wc{10pt}|wc{10pt}@{\,}}
2 & 1 & 0 & 0\\
1 & 2 & 1 & 0\\
-1 & 1 & 1 & 0\\
1 & -1 & 2 & 0
\end{array}
\right]
&\to
\left[
\begin{array}{@{\,}wc{10pt}wc{10pt}wc{10pt}|wc{10pt}@{\,}}
1 & -1 & 2 & 0 \\
1 & 2 & 1 & 0\\
-1 & 1 & 1 & 0\\
2 & 1 & 0 & 0\\
\end{array}
\right] & R_1 \leftrightarrow R_4 \\
&\to
\left[
\begin{array}{@{\,}wc{10pt}wc{10pt}wc{10pt}|wc{10pt}@{\,}}
1 & -1 & 2 & 0 \\
0 & 3 & -1 & 0\\
0 & 0 & 3 & 0\\
0 & 3 & -4 & 0\\
\end{array}
\right] & 
\begin{aligned}
R_2 - R_1 &\to R_2 \\
R_3 + R_1 &\to R_3 \\
R_4 - 2R_1 &\to R_4 
\end{aligned}\\
&\to
\left[
\begin{array}{@{\,}wc{10pt}wc{10pt}wc{10pt}|wc{10pt}@{\,}}
1 & -1 & 2 & 0 \\
0 & 1 & -\frac{1}{3} & 0\\
0 & 0 & 3 & 0\\
0 & 3 & -4 & 0\\
\end{array}
\right] & \frac{1}{3}R_2 \to R_2 \\
&\to
\left[
\begin{array}{@{\,}wc{10pt}wc{10pt}wc{10pt}|wc{10pt}@{\,}}
1 & -1 & 2 & 0 \\
0 & 1 & -\frac{1}{3} & 0\\
0 & 0 & 3 & 0\\
0 & 0 & -3 & 0\\
\end{array}
\right] & R_4 - 3R_2 \to R_4 \\
&\to
\left[
\begin{array}{@{\,}wc{10pt}wc{10pt}wc{10pt}|wc{10pt}@{\,}}
1 & -1 & 2 & 0 \\
0 & 1 & -\frac{1}{3} & 0\\
0 & 0 & 1 & 0\\
0 & 0 & -3 & 0\\
\end{array}
\right] & \frac{1}{3}R_3 \to R_3 \\
&\to
\left[
\begin{array}{@{\,}wc{10pt}wc{10pt}wc{10pt}|wc{10pt}@{\,}}
1 & -1 & 2 & 0 \\
0 & 1 & -\frac{1}{3} & 0\\
0 & 0 & 1 & 0\\
0 & 0 & 0 & 0\\
\end{array}
\right] & R_4 + 3R_1 \to R_4 
\end{align*}
(the zero column to the right can be omitted) The forward phase leads to a redundant row and the presence of pivots in every column indicates that the trivial solution of $\vec{x} = 0$ is the only solution, hence the three vectors $\vec{u}, \vec{v}, \vec{w}$ are linearly independent (refer to Section \ref{subsection:SolLinSysGauss}). The backward phase concerning the matrix $A$ itself is instantaneous, yielding its RREF:
\begin{align*}
\left[
\begin{array}{@{\,}wc{10pt}wc{10pt}wc{10pt}@{\,}}
1 & -1 & 2 \\
0 & 1 & -\frac{1}{3} \\
0 & 0 & 1 \\
0 & 0 & 0 \\
\end{array}
\right] \to
\left[
\begin{array}{@{\,}wc{10pt}wc{10pt}wc{10pt}@{\,}}
1 & 0 & 0 \\
0 & 1 & 0 \\
0 & 0 & 1 \\
0 & 0 & 0 \\
\end{array}
\right]
\end{align*}
By Properties \ref{proper:CRFactor}, the CR Factorization of $A$ is then trivially
\begin{align*}
A = 
\left[
\begin{array}{@{\,}wc{10pt}wc{10pt}wc{10pt}@{\,}}
2 & 1 & 0 \\
1 & 2 & 1 \\
-1 & 1 & 1 \\
1 & -1 & 2
\end{array}
\right] 
=
\left[
\begin{array}{@{\,}wc{10pt}wc{10pt}wc{10pt}@{\,}}
2 & 1 & 0 \\
1 & 2 & 1 \\
-1 & 1 & 1 \\
1 & -1 & 2
\end{array}
\right] 
\left[
\begin{array}{@{\,}wc{10pt}wc{10pt}wc{10pt}@{\,}}
1 & 0 & 0 \\
0 & 1 & 0 \\
0 & 0 & 1 \\
\end{array}
\right] = CR
\end{align*}
with $C = A$ and $R = I_3$ as all columns in $A$ are linearly independent. In general, if the $n$ column vectors in an $m \times n$ matrix $A$ are linearly independent\footnote{Bear in mind that it is necessary to have $m \geq n$.}, then its RREF will be in the form of
\begin{align*}
A_{\text{rref}} = 
\begin{bmatrix}
1 & 0 & 0 & \cdots & 0 \\
0 & 1 & 0 & & 0 \\
0 & 0 & 1 & & 0 \\
\vdots & & & \ddots & \vdots \\
0 & 0 & 0 & \cdots & 1 \\
0 & 0 & 0 & \cdots & 0 \\
\vdots & & & & \vdots
\end{bmatrix}
\end{align*}
where the top is an $n \times n$ identity matrix $I_n$, followed by $m-n$ rows of full zeros at the bottom. The CR Factorization of $A$ will then be simply comprised of $C=A$ and $R=I_n$. For the second case where $\vec{w} = (1,-1,-2,2)^T$, we can repeat the same analysis by deriving the RREF of the modified $A$ matrix.
\begin{align*}
\left[
\begin{array}{@{\,}wc{10pt}wc{10pt}wc{10pt}@{\,}}
2 & 1 & 1 \\
1 & 2 & -1 \\
-1 & 1 & -2 \\
1 & -1 & 2 
\end{array}
\right]
&\to
\left[
\begin{array}{@{\,}wc{10pt}wc{10pt}wc{10pt}@{\,}}
1 & -1 & 2 \\
1 & 2 & -1 \\
-1 & 1 & -2 \\
2 & 1 & 1 
\end{array}
\right] & R_1 \leftrightarrow R_4 \\
&\to
\left[
\begin{array}{@{\,}wc{10pt}wc{10pt}wc{10pt}@{\,}}
1 & -1 & 2 \\
0 & 3 & -3 \\
0 & 0 & 0 \\
0 & 3 & -3 
\end{array}
\right] & 
\begin{aligned}
R_2 - R_1 \to R_2 \\
R_3 + R_1 \to R_3 \\
R_4 - 2R_1 \to R_4
\end{aligned} \\
&\to
\left[
\begin{array}{@{\,}wc{10pt}wc{10pt}wc{10pt}@{\,}}
1 & -1 & 2 \\
0 & 1 & -1 \\
0 & 0 & 0 \\
0 & 3 & -3 
\end{array}
\right] & \frac{1}{3}R_2 \to R_2 \\
&\to
\left[
\begin{array}{@{\,}wc{10pt}wc{10pt}wc{10pt}@{\,}}
1 & -1 & 2 \\
0 & 1 & -1 \\
0 & 0 & 0 \\
0 & 0 & 0 
\end{array}
\right] & R_4 - 3R_2 \to R_4 \\
&\to
\left[
\begin{array}{@{\,}wc{10pt}wc{10pt}wc{10pt}@{\,}}
1 & 0 & 1 \\
0 & 1 & -1 \\
0 & 0 & 0 \\
0 & 0 & 0 
\end{array}
\right] & R_1 + R_2 \to R_1 
\end{align*}
The final RREF reveals that $\vec{u}$ and $\vec{v}$ are two linearly independent vectors in the column space of $A$, in addition to the dependence relation of $\vec{w} = \vec{u} - \vec{v}$. Hence its new CR Factorization, by Properties \ref{proper:CRFactor}, is
\begin{align*}
\left[\begin{array}{@{\,}wc{10pt}wc{10pt}wc{10pt}@{\,}}
2 & 1 & 1 \\
1 & 2 & -1 \\
-1 & 1 & -2 \\
1 & -1 & 2 
\end{array}
\right]
=
\left[\begin{array}{@{\,}wc{10pt}wc{10pt}@{\,}}
2 & 1 \\
1 & 2 \\
-1 & 1 \\
1 & -1
\end{array}
\right]
\left[
\begin{array}{@{\,}wc{10pt}wc{10pt}wc{10pt}@{\,}}
1 & 0 & 1 \\
0 & 1 & -1 \\
\end{array}
\right]
\end{align*}
\end{solution}

\section{Coordinate Bases for $\mathbb{R}^n$ and its Subspaces}

\subsection{Coordinate Bases for $\mathbb{R}^n$}
\label{section:6.1.5}
Back in Definition \ref{defn:standardunitvec}, we have introduced the $n$ standard unit vectors $\hat{e}^{(1)}, \hat{e}^{(2)}, \ldots, \hat{e}^{(n)}$ for the real $n$-space $\mathbb{R}^n$. Obviously, the standard unit vectors are linearly independent and their span is exactly $\mathbb{R}^n$. We often refer to the coefficients $x_j$ in front of $\hat{e}^{(j)}$ of a vector $\vec{x} = (x_1, x_2, \ldots, x_n)^T = x_1\hat{e}^{(1)} + x_2\hat{e}^{(2)} + \cdots + x_n\hat{e}^{(n)}$ in $\mathbb{R}^n$ as the \textit{Cartesian} coordinates of $\vec{x}$. The coordinates $x_j$ are unique, guaranteed by Properties \ref{proper:lincombofspan}.\par
However, sometimes we may want to express an $\mathbb{R}^n$ vector in another \index{Coordinate Basis}\keywordhl{coordinate basis (system)} with axes different from the standard unit vectors, that is, other than the \index{Standard Basis}\keywordhl{standard basis} $\mathcal{S} = \{\hat{e}^{(1)}, \hat{e}^{(2)}, \ldots, \hat{e}^{(n)}\}$. Motivated by the properties of the Cartesian coordinate system above, in which the standard unit vectors are linearly independent and span $\mathbb{R}^n$ such that every vector in $\mathbb{R}^n$ can be expressed as a unique linear combination of them, we require all other coordinate bases for $\mathbb{R}^n$ to carry the same properties. The coefficients of this linear combination will then become the coordinates of that vector with respect to this basis.
\begin{defn}[Coordinate Basis for $\mathbb{R}^n$]
\label{defn:coordRn}
A coordinate basis $\mathcal{\beta}$ for $\mathbb{R}^n$ should consists of $n$ vectors $\{\vec{v}^{(1)}, \vec{v}^{(2)}, \ldots, \vec{v}^{(n)}\}$ where $\vec{v}^{(j)} \in \mathbb{R}^n$, which
\begin{enumerate}[label=(\alph*)]
\item are linearly independent, and
\item span (generate) $\mathbb{R}^n$.
\end{enumerate}
\end{defn}
Some may wonder why the definition above has explicitly stated that the number of vectors in a coordinate basis for $\mathbb{R}^n$ is exactly $n$, although many people would probably think it is reasonable and accept this without a doubt. For the sake of completeness, we will explain that this is a result coming naturally from the conditions of linear independence and spanning $\mathbb{R}^n$. We have previously shown that Theorem \ref{thm:linearindep} implies that in $\mathbb{R}^n$ if there are more vectors $q$ than the dimension $n$ then they will be linearly dependent. So linear independence requires $q \leq n$. To span $\mathbb{R}^n$, it is apparent that $q \geq n$.\footnote{
\label{foot:inconsth}
To formally show this, express the span of $q$ $\mathbb{R}^n$ vectors $c_1\vec{v}^{(1)} + c_2\vec{v}^{(2)} + c_3\vec{v}^{(3)} + \cdots + c_q\vec{v}^{(q)}$ by $A\vec{x}$ where $A = [\vec{v}^{(1)}|\vec{v}^{(2)}|\vec{v}^{(3)}|\cdots|\vec{v}^{(q)}]$ is an $n \times q$ matrix and $\vec{x} = (c_1, c_2, c_3, \ldots, c_q)^T$ consists of $q$ coefficients as unknowns (Properties \ref{proper:linearcombmatrix}). If $q < n$, then $A\vec{x} = \smash{\vec{h}}$ is an overdetermined system so that we can always find some row of full zeros in the RREF of $A$ to the left of the augmented matrix as we solve the system by Gaussian Elimination. Along the column vector to the right of the augmented matrix that undergoes the reduction process together, we can always set the number on such a row to some non-zero number (let's say, $1$) if not already, to make sure it is inconsistent. Invert the entire process of Gaussian Elimination over the augmented matrix to recover $A$ from its reduced form. To the right of the augmented matrix will then appear $\smash{\vec{h}_{\text{inconst}}}$. This system $A\vec{x} = \smash{\vec{h}_{\text{inconst}}}$ is inconsistent by the design above (just do the same steps of Gaussian Elimination again and the inconsistent $1$ to the right will reappear), which shows that the span does not include $\smash{\vec{h}_{\text{inconst}}}$ and cannot cover the entire $\mathbb{R}^n$.} Hence the number of vectors $q$ must be equal to $n$. \par
The following theorem shows that we actually only need to check either one of the conditions in Definition \ref{defn:coordRn}.
\begin{thm}
\label{thm:linindspan}
A set of $n$ vectors of $\mathbb{R}^n$ is linearly independent if and only if they span $\mathbb{R}^n$.
\end{thm}
\begin{proof}
Linear Independence $\rightarrow$ Spanning $\mathbb{R}^n$: Assume $\vec{v}^{(1)}, \vec{v}^{(2)}, \ldots, \vec{v}^{(n)}$ are linear independent with $A = [\vec{v}^{(1)}|\vec{v}^{(2)}|\cdots|\vec{v}^{(n)}]$ being an $n \times n$ square matrix. The application of part (e) $\rightarrow$ (d) of Theorem \ref{thm:equiv3} immediately shows that there is always a (unique) solution to $A\vec{x} = \vec{h}$ for any $\vec{h}$ of $\mathbb{R}^n$. Recall that $A\vec{x}$ represents the span of $\{\vec{v}^{(1)}, \vec{v}^{(2)}, \ldots, \vec{v}^{(n)}\}$ (Definition \ref{defn:span}) so it implies that these vectors can generate the entire $\mathbb{R}^n$.\\
\\
Spanning $\mathbb{R}^n$ $\rightarrow$ Linear Independence: Assume the opposite of the implication that $\vec{v}^{(1)}, \vec{v}^{(2)}, \ldots, \vec{v}^{(n)}$ are linearly dependent, then by (c) and (e) of Theorem \ref{thm:equiv3} the reduced row echelon form of $A = [\vec{v}^{(1)}|\vec{v}^{(2)}|\cdots|\vec{v}^{(n)}]$ is not the identity matrix and contains at least one row of full zeros. Following a logic similar to Footnote \ref{foot:inconsth}, these vectors cannot span $\mathbb{R}^n$ and the contrapositive is proved. 
\end{proof}

\begin{exmp}
\label{exmp:basisR3}
Show that $\mathcal{\beta} = \{\vec{v}^{(1)}, \vec{v}^{(2)}, \vec{v}^{(3)}\} = \{\smash{(1,2,1)_S^T}, \smash{(-1,1,0)_S^T}, \allowbreak \smash{(1,-1,2)_S^T}\}$ forms a basis for $\mathbb{R}^3$ and express $\vec{v}$ in the $\mathcal{\beta}$ basis (denoted by $[\vec{v}]_\beta$) where $[\vec{v}]_S = \smash{(2,1,2)_S^T}$, the subscript $S$ [$\beta$] emphasizes that the coordinates are relative to the standard basis $\mathcal{S}$ [the new basis $\mathcal{\beta}$].
\end{exmp}
\begin{solution}
By Definition \ref{defn:coordRn} and Theorem \ref{thm:linindspan}, the first part is equivalent to checking if the three $\mathbb{R}^3$ vectors in $\mathcal{\beta}$ are linearly independent. By (b) to (e) of Theorem \ref{thm:equiv3}, we can simply check if $\det(A)$ is non-zero where 
\begin{align*}
A = \begin{bmatrix}[\vec{v}^{(1)}]_S|[\vec{v}^{(2)}]_S|[\vec{v}^{(3)}]_S
\end{bmatrix} =
\begin{bmatrix}
1 & -1 & 1 \\
2 & 1 & -1 \\
1 & 0 & 2
\end{bmatrix}
\end{align*}
A simple calculation reveals that $\det(A) = 6 \neq 0$ so $\mathcal{\beta}$ is indeed a valid basis for $\mathbb{R}^3$. To express $(2,1,2)_S^T$ in $\mathcal{\beta}$ is to find $[\vec{v}]_\beta = \smash{([v_1]_\beta, [v_2]_\beta, [v_3]_\beta)_\beta^T}$ where $[v_j]_\beta$ is the $j$-th component (coefficient) of $\vec{v}$ in the $\mathcal{\beta}$ coordinate system such that the corresponding linear combination of $\vec{v}^{(j)}$ below produces the desired vector that is consistent in the $\mathcal{S}$ basis as well:
\begin{align*}
[v_1]_\beta(\vec{v}^{(1)}) + [v_2]_\beta(\vec{v}^{(2)}) + [v_3]_\beta(\vec{v}^{(3)}) &= \vec{v} \\
[v_1]_\beta(1,2,1)_S^T + [v_2]_\beta(-1,1,0)_S^T + [v_3]_\beta(1,-1,2)_S^T &= (2,1,2)_S^T \end{align*}
or put in matrix form,
\begin{align*}
[v_1]_\beta[\vec{v}^{(1)}]_S + [v_2]_\beta[\vec{v}^{(2)}]_S + [v_3]_\beta[\vec{v}^{(3)}]_S = [\vec{v}]_S \\
A[\vec{v}]_\beta = \begin{bmatrix}[\vec{v}^{(1)}]_S|[\vec{v}^{(2)}]_S|[\vec{v}^{(3)}]_S
\end{bmatrix}\begin{bmatrix}
[v_1]_\beta \\
[v_2]_\beta \\
[v_3]_\beta
\end{bmatrix}
= 
\begin{bmatrix}
1 & -1 & 1 \\
2 & 1 & -1 \\
1 & 0 & 2
\end{bmatrix}
\begin{bmatrix}
[v_1]_\beta \\
[v_2]_\beta \\
[v_3]_\beta
\end{bmatrix}
&=
\begin{bmatrix}
2 \\
1 \\
2
\end{bmatrix}
\end{align*}
We can either use matrix inverse or Gaussian Elimination to solve for the $[v_j]_\beta$, yielding $[v_1]_\beta = 1, [v_2]_\beta = -\frac{1}{2}, [v_3]_\beta = \frac{1}{2}$, and hence $[\vec{v}]_\beta = \smash{(1, -\frac{1}{2}, \frac{1}{2})^T_\beta}$. Notice that the matrix equation
\begin{align}
A[\vec{v}]_\beta = \begin{bmatrix}[\vec{v}^{(1)}]_S|[\vec{v}^{(2)}]_S|[\vec{v}^{(3)}]_S
\end{bmatrix}\begin{bmatrix}
[v_1]_\beta \\
[v_2]_\beta \\
[v_3]_\beta
\end{bmatrix}
&= [\vec{v}]_S
\label{eqn:PvvbetaS}
\end{align}
shows that the matrix $A$ transforms the coordinate system from the $\mathcal{\beta}$ to $\mathcal{S}$ basis in which a given vector $\vec{v}$ is expressed, and we will write $A = \smash{P_\beta^S}$ (thus $\smash{P_\beta^S} [\vec{v}]_\beta = [\vec{v}]_S$) for clarity in the future. Also, the vector $\vec{v}$ is still the same one despite having different coordinate representations since only the reference coordinate frame/system is changed. For this, we will henceforth use the equivalence symbol to write along the lines of $\vec{v} \equiv [\vec{v}]_\beta \equiv [\vec{v}]_S$. \par
Notice that $\smash{P_\beta^S}[\vec{v}^{(j)}]_\beta = \smash{P_\beta^S} (e^{(j)})_\beta$ returns the $j$-th basis vector in $\mathcal{\beta}$ ($j$-th column of $\smash{P_\beta^S}$) expressed in the original standard basis $\mathcal{S}$, namely $\smash{[\vec{v}^{(j)}]_S}$, where $\smash{(e^{(j)})_\beta} = \smash{[\vec{v}^{(j)}]_\beta}$ is the numeric coordinate representation (emphasized by the absence of the hat symbol over $e$) of the $j$-th basis vector in the $\mathcal{\beta}$ coordinate system with the $j$-th component being $1$ and other being $0$. For instance\footnote{In contrast, the standard unit vector $\hat{e}^{(2)}$ with a hat is deemed as a "true" vector with $[\hat{e}^{(2)}]_S = (0,1,0)_S^T$. Then, $P_\beta^S[\hat{e}^{(2)}]_\beta = [\hat{e}^{(2)}]_S$ and thus
\begin{align*}
[\hat{e}^{(2)}]_\beta =
(P_\beta^S)^{-1} [\hat{e}^{(2)}]_S =
\begin{bmatrix}
\frac{1}{3} & \frac{1}{3} & 0\\ 
-\frac{5}{6} & \frac{1}{6} & \frac{1}{2}\\ 
-\frac{1}{6} & -\frac{1}{6} & \frac{1}{2}
\end{bmatrix}
\begin{bmatrix}
0 \\
1 \\
0
\end{bmatrix}
=
(\frac{1}{3}, \frac{1}{6}, -\frac{1}{6})_\beta^T
\end{align*}},
\begin{align*}
P_\beta^S [\vec{v}^{(2)}]_\beta =
P_\beta^S (e^{(2)})_\beta =
\begin{bmatrix}
1 & -1 & 1 \\
2 & 1 & -1 \\
1 & 0 & 2
\end{bmatrix}
\begin{bmatrix}
0 \\
1 \\
0
\end{bmatrix}   
=
\begin{bmatrix}
-1 \\
1 \\
0
\end{bmatrix}
= [\vec{v}^{(2)}]_S
\end{align*}
From now on, we simply omit the subscript $S$ and write $\vec{v}$ in place of $[\vec{v}]_S$ if not specified, to denote vectors in the standard basis as implicitly assumed before.
\end{solution}

\subsection{Coordinate Bases for Subspaces of $\mathbb{R}^n$}
\label{section:subspacebasis}

Now that we are able to construct a coordinate basis for $\mathbb{R}^n$, it is natural to ask if we can also extend this and come up with some coordinate basis for any subspace of $\mathbb{R}^n$ (since a subspace is itself a vector space too), in the sense that any vector in the subspace can be uniquely expressed by the basis vectors (\textit{linear independence}) and the basis \textit{spans} the subspace exactly, just like any basis for $\mathbb{R}^n$. Actually, we have already done this for the column space of a matrix $A$ back in the derivation of CR Factorization in Section \ref{section:linearind} where we created a minimal generating set from the column vectors that compose $A$. \par
For other subspaces of $\mathbb{R}^n$ the procedure is similar. If we are given a subspace as a span of some vectors, then to find a basis for it, we carry out CR Factorization as if these vectors are the columns of a matrix and retain the linearly independent vectors. These linearly independent vectors still span the subspace by part (b) of Theorem \ref{thm:plusminus}, and any vector in the subspace can be written as a unique linear combination of them by Properties \ref{proper:lincombofspan}. 

\begin{proper}
\label{proper:findgenbasis}
For the subspace generated by a spanning set $\{\vec{v}^{(1)}, \vec{v}^{(2)}, \ldots, \\ \vec{v}^{(q)}\}$, its basis can be found by applying CR Factorization (Properties \ref{proper:CRFactor}) over $A = \begin{bmatrix}
\vec{v}^{(1)}|\vec{v}^{(2)}|\ldots|\vec{v}^{(q)}    
\end{bmatrix}$. Then a possible basis is $\{\vec{v}^{(j)}\}$ for all $j$ such that the $j$-th column of RREF of $A$ contains a leading 1.
\end{proper}

\begin{exmp}
\label{exmp:gentrimbasis}
Find a basis $\mathcal{\beta}$ for the subspace generated by $\mathcal{\gamma} = \{(1,0,2,1)^T, \\ (1,-1,1,-2)^T, (0,0,1,0)^T, (-2,1,0,1)^T\}$. Hence express $(3,-1,4,0)^T$ in this basis. 
\end{exmp}
\begin{solution}
By Properties \ref{proper:findgenbasis}, we need to find the CR Factorization for
\begin{align*}
A = 
\begin{bmatrix}
1 & 1 & 0 & -2\\
0 & -1 & 0 & 1\\
2 & 1 & 1 & 0\\
1 & -2 & 0 & 1
\end{bmatrix}
\end{align*}
We proceed with Gaussian Elimination.
\begin{align*}
\left[\begin{array}{@{\,}wc{10pt}wc{10pt}wc{10pt}wc{10pt}@{\,}}
1 & 1 & 0 & -2\\
0 & -1 & 0 & 1\\
2 & 1 & 1 & 0\\
1 & -2 & 0 & 1 
\end{array}
\right] &\to 
\left[\begin{array}{@{\,}wc{10pt}wc{10pt}wc{10pt}wc{10pt}@{\,}}
1 & 1 & 0 & -2\\
0 & -1 & 0 & 1\\
0 & -1 & 1 & 4\\
0 & -3 & 0 & 3 
\end{array}
\right] & 
\begin{aligned}
R_3 - 2R_1 &\to R_3 \\
R_4 - R_1 &\to R_4
\end{aligned} \\
&\to 
\left[\begin{array}{@{\,}wc{10pt}wc{10pt}wc{10pt}wc{10pt}@{\,}}
1 & 1 & 0 & -2\\
0 & 1 & 0 & -1\\
0 & -1 & 1 & 4\\
0 & -3 & 0 & 3 
\end{array}
\right] & 
-R_2 \to R_2 \\
&\to 
\left[\begin{array}{@{\,}wc{10pt}wc{10pt}wc{10pt}wc{10pt}@{\,}}
1 & 1 & 0 & -2\\
0 & 1 & 0 & -1\\
0 & 0 & 1 & 3\\
0 & 0 & 0 & 0 
\end{array}
\right] & 
\begin{aligned}
R_3 + R_2 &\to R_3 \\
R_4 + 3R_2 &\to R_4
\end{aligned} \\
&\to 
\left[\begin{array}{@{\,}wc{10pt}wc{10pt}wc{10pt}wc{10pt}@{\,}}
1 & 0 & 0 & -1\\
0 & 1 & 0 & -1\\
0 & 0 & 1 & 3\\
0 & 0 & 0 & 0 
\end{array}
\right] & 
R_1 - R_2 \to R_1
\end{align*}
So a possible basis for the subspace: $\text{span}(\mathcal{\gamma})$, contains the first three generating vectors in $\mathcal{\gamma}$, so that $\mathcal{\beta} = \{(1,0,2,1)^T, (1,-1,1,-2)^T, (0,0,1,0)^T\}$. To express $\vec{v} = (3,-1,4,0)^T$ in this basis, we need to find $[\vec{v}]_\beta$ just like in Example \ref{exmp:basisR3}, which is derived by
\begin{align*}
\begin{bmatrix}
1 & 1 & 0 \\
0 & -1 & 0 \\
2 & 1 & 1 \\
1 & -2 & 0
\end{bmatrix}
\begin{bmatrix}
[v_1]_\beta \\
[v_2]_\beta \\
[v_3]_\beta
\end{bmatrix} =
\begin{bmatrix}
3 \\
-1 \\
4 \\
0
\end{bmatrix}
\end{align*}
We repeat the same steps of Gaussian Elimination over the augmented matrix where now the left portion consists of the three linearly independent basis vectors only.
\begin{align*}
\left[\begin{array}{@{\,}wc{10pt}wc{10pt}wc{10pt}|wc{10pt}@{\,}}
1 & 1 & 0 & 3\\
0 & -1 & 0 & -1\\
2 & 1 & 1 & 4\\
1 & -2 & 0 & 0 
\end{array}\right] &\to
\left[\begin{array}{@{\,}wc{10pt}wc{10pt}wc{10pt}|wc{10pt}@{\,}}
1 & 1 & 0 & 3 \\
0 & -1 & 0 & -1\\
0 & -1 & 1 & -2\\
0 & -3 & 0 & -3 
\end{array}\right]
& \begin{aligned}
R_3 - 2R_1 &\to R_3 \\
R_4 - R_1 &\to R_4
\end{aligned} \\
&\to
\left[\begin{array}{@{\,}wc{10pt}wc{10pt}wc{10pt}|wc{10pt}@{\,}}
1 & 1 & 0 & 3 \\
0 & 1 & 0 & 1\\
0 & -1 & 1 & -2\\
0 & -3 & 0 & -3 
\end{array}\right]
& -R_2 \to R_2 \\
&\to
\left[\begin{array}{@{\,}wc{10pt}wc{10pt}wc{10pt}|wc{10pt}@{\,}}
1 & 1 & 0 & 3 \\
0 & 1 & 0 & 1\\
0 & 0 & 1 & -1\\
0 & 0 & 0 & 0 
\end{array}\right]
& 
\begin{aligned}
R_3 + R_2 &\to R_3 \\
R_4 + 3R_2 &\to R_4
\end{aligned} \\
&\to
\left[\begin{array}{@{\,}wc{10pt}wc{10pt}wc{10pt}|wc{10pt}@{\,}}
1 & 0 & 0 & 2 \\
0 & 1 & 0 & 1\\
0 & 0 & 1 & -1\\
0 & 0 & 0 & 0 
\end{array}\right] 
& R_1 - R_2 \to R_1
\end{align*}
The last full-zero row is consistent and $[\vec{v}]_\beta = \smash{([v_1]_\beta, [v_2]_\beta, [v_3]_\beta)_B^T} = \smash{(2,1,-1)_B^T}$. As a final note, since the fourth vector $(-2,1,0,1)^T$ in the generating set can be written as a non-zero linear combination of the first three vectors (with the coefficients of $-1,-1,3$), we can replace any one of the three vectors in the basis by $(-2,1,0,1)^T$.
\end{solution}

In general, any vector $\vec{v}^{(j)}$ in a basis can be replaced by another vector that is the linear combination of the basis vectors where the coefficient corresponding to $\vec{v}^{(j)}$ is particularly not zero.\footnote{\label{foot:ft14}This preserves the span and linear independence. Refer to Properties \ref{proper:elemrowoprowrank} later for the span part (where the vectors are to be viewed as rows and the replacement is effectively multiplications and additions of rows), plus (a) of Properties \ref{proper:linindspanbasisnewver} for linear independence.} Moreover, we can now properly define the "dimension" of any subspace of $\mathbb{R}^n$. It is simply the number of (linearly independent) vectors in its basis. Some may wonder if it is possible for two bases of the same vector space to have different number of vectors so that the notion of its dimension will be problematic. In fact, all bases of a \textit{finite-dimensional} vector (sub)space must possess the same amount of vectors, and we simply note this below.\footnote{It, along with other results below, actually comes from a more general theorem called the \textit{Steinitz Replacement Theorem} which is proved in Appendix \ref{section:vecspaceappend}.} %Extending from the requirement of a basis for $\mathbb{R}^n$, here a basis of a general vector space $\mathcal{V}$ should be linearly independent and spans $\mathcal{V}$ too. To show this, we need an important result called \index{Steinitz Replacement Theorem}\keywordhl{Steinitz Replacement Theorem}.

\begin{proper}
\label{proper:samenvecsbases}
If $\mathcal{V}$ is a vector space with a finite basis, then all bases of $\mathcal{V}$ are finite and have the same number of vectors.
\end{proper}
%\begin{proof}
%Assume $\mathcal{G}$ is a finite basis for $\mathcal{V}$ consists of $n$ vectors, and let $\mathcal{U}$ be any other basis with $k$ vectors for $\mathcal{V}$. If $\mathcal{U}$ contains more than $n$ vectors such that $k > n$, then we can take a subset $\mathcal{S}$ of $\mathcal{U}$ with exactly $m = n+1$ vectors. By Theorem \ref{thm:Steinitz}, as $\mathcal{G}$ generates $\mathcal{V}$, and $\mathcal{S}$ as a subset of the basis $\mathcal{U}$ is linearly independent, $m = n+1 \leq n$ which leads to a contradiction, so it has to be $k \leq n$. Reversing the roles of $\mathcal{G}$ and $\mathcal{U}$, the same arguments requires $k \geq n$, and therefore $k = n$, i.e. every bases of $\mathcal{V}$ have $n$ vectors and are finite.
%\end{proof}
From the statement above, we see that if we can find any basis with exactly $n$ vectors for a vector space $\mathcal{V}$ where $n$ is finite, then $n$ will be the unique integer such that every basis $\mathcal{V}$ is generated by this number of vectors. $n$ is then referred to as the \index{Dimension}\keywordhl{dimension} of $\mathcal{V}$, and we define $\text{dim}(\mathcal{V}) = n$. $\mathcal{V}$ is then known as a \index{Finite-dimensional}\keywordhl{finite-dimensional} vector space. If a vector space is not finite-dimensional, i.e. a finite basis cannot be found, then it is called \index{Infinite-dimensional}\keywordhl{infinite-dimensional}. Moreover,
\begin{proper}
\label{proper:dimWleqV}
For any subspace $\mathcal{W}$ of a vector space $\mathcal{V}$, $\dim(\mathcal{W}) \leq \dim(\mathcal{V})$. If $\dim(\mathcal{W}) = \dim(\mathcal{V})$, $\mathcal{W} = \mathcal{V}$.
\end{proper}
and
\begin{thm}
\label{thm:finitebasissubset}
If a vector space $\mathcal{V}$ is generated by a spanning set $\mathcal{\gamma}$ with a finite amount of vectors, then some subset $\mathcal{\gamma'}$ of $\mathcal{\gamma}$ is a basis for $\mathcal{V}$, and $\mathcal{V}$ has finite bases.
\end{thm}
%\begin{proof}
    %The proof closely parallels that for Properties \ref{proper:genbasis}. If $\mathcal{G}$ has $n$ vectors, then we can choose $\mathcal{B} = \{\vec{u}_1, \vec{u}_2, \cdots, \vec{u}_j\}$, $j \leq n$, such that $\mathcal{B}$ are linearly independent. If $j = n$, then $\mathcal{B} = \mathcal{G}$ is linearly independent, spans $\mathcal{V}$, and hence itself a basis for $\mathcal{V}$. Otherwise, $\text{span}(\mathcal{B}) \subseteq \mathcal{V}$ as $\mathcal{B} \subseteq \mathcal{V}$ by Properties \ref{proper:WcontainsspanS}, and using the logic similar to the second part of proof in Properties \ref{proper:genbasis}, $\mathcal{V} \subseteq \text{span}(\mathcal{B})$ and hence $\text{span}(\mathcal{B}) = \mathcal{V}$. This $\mathcal{B}$, as a linearly independent subset of $\mathcal{G}$, is thus a basis for $\mathcal{V}$.
%\end{proof}
which is a broader restatement of Properties \ref{proper:findgenbasis}.

Finally, we expand Theorem \ref{thm:linindspan} (Equivalent requirements of a basis) to any finite-dimensional vector (sub)space. The results are simply stated below.
\begin{proper}
\label{proper:linindspanbasisnewver}
If $\mathcal{V}$ is a vector space with $\text{dim}(\mathcal{V}) = n$, then
\begin{enumerate}[label=(\alph*)]
    \item Any generating set for $\mathcal{V}$ contains at least $n$ vectors. If, furthermore, it is made of exactly $n$ vectors, then it will be a basis for $\mathcal{V}$;
    \item Any linearly independent subset of $\mathcal{V}$ that has exactly $n$ vectors is a basis for $\mathcal{V}$;
    \item Every linearly independent subset $\mathcal{\gamma}_1$ of $\mathcal{V}$ with $m \leq n$ vectors can be extended to a basis for $\mathcal{V}$, i.e. there exists another subset $\mathcal{\gamma}_2$ of $\mathcal{V}$ with $n-m$ (linearly independent) vectors such that $\mathcal{\beta} = \mathcal{\gamma}_1 \cup \mathcal{\gamma}_2$ is a basis for $\mathcal{V}$.
\end{enumerate}
\end{proper}
%\begin{proof}
%\begin{enumerate}[label=(\alph*)]
    %\item Let $\mathcal{G}$ be a generating set for $\mathcal{V}$. By Theorem \ref{thm:finitebasissubset}, there exists a subset $\mathcal{H}$ of $\mathcal{G}$ as a basis for $\mathcal{V}$. By Properties \ref{proper:samenvecsbases}, $\mathcal{H}$ has exactly $n$ vectors, and $\mathcal{G}$, having $\mathcal{H}$ as a subset, must have at least $n$ vectors. If $\mathcal{G}$ happens to have $n$ vectors as well, then $\mathcal{G} = \mathcal{H}$ is a basis for $\mathcal{V}$.
    %\item Use Theorem \ref{thm:Steinitz} with $\mathcal{S}$ being a linearly independent subset of $\mathcal{V}$ that has exactly $n$ vectors. $\mathcal{G}$ can any basis for $\mathcal{V}$, and in particular it generates $\mathcal{V}$ and is consisted of $n$ vectors as well. Then the Replacement Theorem implies that there exists a subset $\mathcal{H}$ of $\mathcal{G}$ with $n-n = 0$ vectors such that $\mathcal{S} \cup \mathcal{H}$ spans $\mathcal{V}$. But $\mathcal{H}$ having $0$ vectors means it is the empty set $\varnothing$, and it reduces to $\mathcal{S}$ spanning $\mathcal{V}$. Hence $\mathcal{S}$, being linearly independent, is a basis for $\mathcal{V}$.
    %\item Again use Theorem \ref{thm:Steinitz} with $\mathcal{S}$ this time being a linearly independent subset of $\mathcal{V}$ having $m \leq n$ vectors. Subsequently, there is a subset $\mathcal{H}$ of $\mathcal{V}$ (since any generating set $\mathcal{G} \subset \mathcal{V}$ is within the vector space) made up of $n-m$ vectors such that $\mathcal{S} \cup \mathcal{H}$ spans $\mathcal{V}$. As $\mathcal{S}$ and $\mathcal{H}$ have $m$ and $n-m$ vectors, the union $\mathcal{S} \cup \mathcal{H}$ can have at most $m + (n-m) = n$ vectors. Because $\mathcal{S} \cup \mathcal{H}$ generates $\mathcal{V}$, part (a) of the properties then means that $\mathcal{S} \cup \mathcal{H}$ has exactly $n$ vectors and it is a basis for $\mathcal{V}$ as well.
%\end{enumerate}
A point worth mentioning is that part (c) of the properties above allows the possibility of completing a basis from some fragment, which will be used in many arguments from time to time.
%\end{proof}

\subsection{Direct Sum Representation}
\label{section:directsum}

Since we can create subspaces from multiple individual vectors, we may like to know if we can go one step further and make a larger vector space from smaller subspaces by composing them together. This then leads to the \textit{direct sum} representation. Let's begin with the definition of \index{Subspace Sum}\keywordhl{sum of subspaces} first.
\begin{defn}[Subspace Sum]
\label{defn:subspacesum}
Given two subspaces $\mathcal{W}_1, \mathcal{W}_2$, of a vector space $\mathcal{V}$, their subspace sum is
\begin{align}
\mathcal{W}_1 + \mathcal{W}_2 = \{\vec{w}_1 + \vec{w}_2 \mid \text{all } \vec{w}_1 \in \mathcal{W}_1, \vec{w}_2 \in \mathcal{W}_2\}  \label{eqn:subspacesum}
\end{align}
consisted of all possible vectors resulting from addition between any pair of vectors $\vec{w}_1 \in \mathcal{W}_1, \vec{w}_2 \in \mathcal{W}_2$ from $\mathcal{W}_1, \mathcal{W}_2 \subseteq \mathcal{V}$ respectively. Note that $(\mathcal{W}_1 + \mathcal{W}_2) \subseteq \mathcal{V}$ is also a subspace of $\mathcal{V}$.
\end{defn}
For example, if $\mathcal{W}_1 = \text{span}(\{(1,0,1)^T\})$ and $\mathcal{W}_2 = \text{span}(\{(1,1,0)^T, (0,1,1)^T\})$, then according to the definition of spans in Definition \ref{defn:span} and that of a subspace sum above, $\mathcal{W}_1 + \mathcal{W}_2 = \text{span}(\{(1,0,1)^T, (1,1,0)^T, (0,1,1)^T\})$, which is just the span of the union of generating vectors from the two smaller spans, and can be shown to be equal to $\mathbb{R}^3$ following the same idea used when doing Example \ref{exmp:basisR3}. Extending this, we have
\begin{align}
\mathcal{W}_1 + \mathcal{W}_2 + \cdots + \mathcal{W}_n = \{\vec{w}_1 + \vec{w}_2 + \cdots \vec{w}_n \mid \vec{w}_j \in \mathcal{W}_j, 1 \leq j \leq n\}    
\end{align}
In the small example above, $\dim(\mathcal{W}_1) + \dim(\mathcal{W}_2) = 1 + 2 = 3 = \dim(\mathcal{W}_1 + \mathcal{W}_2)$, as the spanning vectors collected from the two subspaces are linearly independent of each other, i.e. the basis vector in $\mathcal{W}_1$ cannot be expressed as the linear combination of those in $\mathcal{W}_2$ and vice versa. In this case, the dimensions of the two subspaces can be \textit{directly} added together, and hence it constitutes a \index{Direct Sum}\keywordhl{direct sum}, whose requirement is given below.
\begin{defn}[Direct Sum]
\label{defn:directsum}
A direct sum between two subspaces $\mathcal{W}_1, \mathcal{W}_2 \subseteq \mathcal{V}$ is their subspace sum $\mathcal{W}_1 + \mathcal{W}_2$ as defined by (\ref{eqn:subspacesum}) in Definition \ref{defn:subspacesum}, which additionally satisfies $\mathcal{W}_1 \cap \mathcal{W}_2 = \{\textbf{0}\}$, and is denoted as $\mathcal{W}_1 \oplus \mathcal{W}_2$ (their intersection is empty), and we have $\dim(\mathcal{W}_1 \oplus \mathcal{W}_2) = \dim(\mathcal{W}_1) + \dim(\mathcal{W}_2)$.
\end{defn}
Here we show that the condition of $\mathcal{W}_1 \cap \mathcal{W}_2 = \{\textbf{0}\}$ is equivalent to the above condition that the basis vectors from $\mathcal{W}_1$ and $\mathcal{W}_2$ combined are linearly independent (sometimes we also simply say that the subspaces $\mathcal{W}_1$ and $\mathcal{W}_2$ are linearly independent). Let $\vec{u}^{(1)}, \vec{u}^{(2)}, \ldots, \vec{u}^{(p)}$ and $\vec{v}^{(1)}, \vec{v}^{(2)}, \ldots, \vec{v}^{(q)}$ be the basis vectors for $\mathcal{W}_1$ and $\mathcal{W}_2$ respectively. If these basis vectors are linearly independent, then by Theorem \ref{thm:linearindep}, the equation
\begin{align*}
c_1\vec{u}^{(1)} + c_2\vec{u}^{(2)} + \cdots + c_p\vec{u}^{(p)} + c_{p+1}\vec{v}^{(1)} + \cdots + c_{p+q}\vec{v}^{(q)} = \textbf{0}
\end{align*}
only has $c_j = 0$ as the trivial solution, $1 \leq j \leq p+q$. Rearranging, we have
\begin{align*}
&\quad c_1\vec{u}^{(1)} + c_2\vec{u}^{(2)} + \cdots + c_p\vec{u}^{(p)} \in \mathcal{W}_1 \\
&= -c_{p+1}\vec{v}^{(1)} - \cdots - c_{p+q}\vec{v}^{(q)} \in \mathcal{W}_2
\end{align*}
But since $c_j = 0$ is the only solution to this, it shows that there is only the zero vector in both $\mathcal{W}_1$ and $\mathcal{W}_2$ at the same time. The converse essentially follows the same argument in reverse. We say that $\mathcal{W}_1$ and $\mathcal{W}_2$ are a \index{Complement}\keywordhl{(subspace) complement}\footnote{A subspace complement is not the same as the set-theoretic complement.} to each other in $\mathcal{W}_1 \oplus \mathcal{W}_2$, as any non-zero vector from the direct sum $\mathcal{W}_1 \oplus \mathcal{W}_2$ that has purely zero components in one of the two smaller subspaces, let's say $\mathcal{W}_1$, will always completely reside in the other subspace, $\mathcal{W}_2$.
\begin{proper}[Subspace Complement]
\label{proper:complement}
If a vector space can be written as a direct sum of two smaller subspaces, i.e. $\mathcal{V} = \mathcal{W}_1 \oplus \mathcal{W}_2$, then $\mathcal{W}_1 = \mathcal{W}_2^C$ and $\mathcal{W}_2 = \mathcal{W}_1^C$ are said to be the (subspace) complement (denoted by the superscript $^C$) to each other in $\mathcal{V}$.
\end{proper}
As a counter-example, consider Example \ref{exmp:gentrimbasis}, suppose that $\mathcal{W}_1 = \text{span}(\mathcal{\beta}_1) = \text{span}(\{(1,0,2,1)^T, (1,-1,1,-2)^T\})$ and $\mathcal{W}_2 = \text{span}(\mathcal{\beta}_2) = \text{span}(\{(0,0,1,0)^T, \\(-2,1,0,1)^T\})$ are the two subspaces spanned by the first/last two vectors in $\mathcal{\gamma}$ respectively. It is not hard to see that $\mathcal{\beta}_1$ and $\mathcal{\beta}_2$ are themselves linearly independent (and hence they are bases for $\mathcal{W}_1$ and $\mathcal{W}_2$ individually), and $\dim(\mathcal{W}_1) = \dim(\mathcal{W}_2) = 2$. \par 
Nevertheless, in that example, we already know that the four vectors, when put together, are not linearly independent: $(-2,1,0,1)^T$ is equal to $-(1,0,2,1)^T-(1,-1,1,-2)^T+3(0,0,1,0)^T$, and they only generate a three-dimensional subspace. Hence $\dim(\mathcal{W}_1 + \mathcal{W}_2) = \dim(\text{span}(\mathcal{\beta}_1) + \text{span}(\mathcal{\beta}_2)) = \dim(\text{span}(\mathcal{\gamma})) = 3 \neq 4 = 2+2 = \dim(\mathcal{W}_1) + \dim(\mathcal{W}_2)$, and therefore they cannot form a direct sum. Geometrically, these two subspaces are like two planes intersecting along a straight line.\par
The direct sum of multiple subspaces is then recursively defined as
\begin{align*}
&\quad \mathcal{W}_1 \oplus \mathcal{W}_2 \oplus \mathcal{W}_3 \oplus \cdots \oplus \mathcal{W}_{n-1} \oplus \mathcal{W}_n \\
&= (\cdots((\mathcal{W}_1 \oplus \mathcal{W}_2) \oplus \mathcal{W}_3) \oplus \cdots \oplus \mathcal{W}_{n-1}) \oplus \mathcal{W}_n
\end{align*}
where we add up the subspaces one by one. Below is an example of this.
\begin{exmp}
\label{exmp:directsum}
Given $\mathcal{W}_1 = \text{span}\{(1,0,2,1,0)^T, (2,1,0,0,-1)^T\}$, $\mathcal{W}_2 = \text{span}\{(0,3,1,0,0)^T, (0,0,-1,-2,1)^T\}$, $\mathcal{W}_3 = \text{span}\{(1,1,-3,0,-1)^T\}$, show that $\mathcal{W}_1 \oplus \mathcal{W}_2 \oplus \mathcal{W}_3$ is a valid direct sum equal to $\mathbb{R}^5$.
\end{exmp}
\begin{solution}
First, let's derive $\mathcal{W}_1 \oplus \mathcal{W}_2$. It is obvious that the two generating vectors from each of $\mathcal{W}_1$ and $\mathcal{W}_2$ are linearly independent themselves as they are not constant multiples of another. Now following  similar ideas in Example \ref{exmp:gentrimbasis}, we are going to show that every column in the matrix formed by combining basis vectors of both $\mathcal{W}_1$ and $\mathcal{W}_2$
\begin{align*}
\left[\begin{array}{@{\,}wc{10pt}wc{10pt}wc{10pt}wc{10pt}@{\,}}
1 & 2 & 0 & 0\\
0 & 1 & 3 & 0\\
2 & 0 & 1 & -1\\
1 & 0 & 0 & -2\\
0 & -1 & 0 & 1
\end{array}\right]    
\end{align*}
is pivotal after Gaussian Elimination, as follows.
\begin{align*}
\left[\begin{array}{@{\,}wc{10pt}wc{10pt}wc{10pt}wc{10pt}@{\,}}
1 & 2 & 0 & 0\\
0 & 1 & 3 & 0\\
2 & 0 & 1 & -1\\
1 & 0 & 0 & -2\\
0 & -1 & 0 & 1
\end{array}\right]  
&\rightarrow
\left[\begin{array}{@{\,}wc{10pt}wc{10pt}wc{10pt}wc{10pt}@{\,}}
1 & 2 & 0 & 0\\
0 & 1 & 3 & 0\\
0 & -4 & 1 & -1\\
0 & -2 & 0 & -2\\
0 & -1 & 0 & 1
\end{array}\right] 
&
\begin{aligned}
R_3 - 2R_1 &\to R_3 \\
R_4 - R_1 &\to R_4
\end{aligned} \\
&\rightarrow
\left[\begin{array}{@{\,}wc{10pt}wc{10pt}wc{10pt}wc{10pt}@{\,}}
1 & 2 & 0 & 0\\
0 & 1 & 3 & 0\\
0 & 0 & 13 & -1\\
0 & 0 & 6 & -2\\
0 & 0 & 3 & 1
\end{array}\right] 
&
\begin{aligned}
R_3 + 4R_2 &\to R_3 \\
R_4 + 2R_2 &\to R_4 \\
R_5 + R_2 &\to R_5
\end{aligned} \\
&\rightarrow
\left[\begin{array}{@{\,}wc{10pt}wc{10pt}wc{10pt}wc{10pt}@{\,}}
1 & 2 & 0 & 0\\
0 & 1 & 3 & 0\\
0 & 0 & 3 & 1\\
0 & 0 & 6 & -2\\
0 & 0 & 13 & -1
\end{array}\right] 
& R_3 \leftrightarrow R_5 \\
&\rightarrow
\left[\begin{array}{@{\,}wc{10pt}wc{10pt}wc{10pt}wc{10pt}@{\,}}
1 & 2 & 0 & 0\\
0 & 1 & 3 & 0\\
0 & 0 & 1 & \frac{1}{3}\\
0 & 0 & 6 & -2\\
0 & 0 & 13 & -1
\end{array}\right] 
& \frac{1}{3}R_3 \to R_3 \\
&\rightarrow
\left[\begin{array}{@{\,}wc{10pt}wc{10pt}wc{10pt}wc{10pt}@{\,}}
1 & 2 & 0 & 0\\
0 & 1 & 3 & 0\\
0 & 0 & 1 & \frac{1}{3}\\
0 & 0 & 0 & -4\\
0 & 0 & 0 & -\frac{16}{3}
\end{array}\right] 
& \begin{aligned}
R_4 - 6R_3 &\to R_4 \\
R_5 - 13R_3 &\to R_5
\end{aligned} \\
&\rightarrow
\left[\begin{array}{@{\,}wc{10pt}wc{10pt}wc{10pt}wc{10pt}@{\,}}
1 & 2 & 0 & 0\\
0 & 1 & 3 & 0\\
0 & 0 & 1 & \frac{1}{3}\\
0 & 0 & 0 & 1\\
0 & 0 & 0 & -\frac{16}{3}
\end{array}\right] 
& -\frac{1}{4}R_4 \to R_4 \\
&\rightarrow
\left[\begin{array}{@{\,}wc{10pt}wc{10pt}wc{10pt}wc{10pt}@{\,}}
1 & 2 & 0 & 0\\
0 & 1 & 3 & 0\\
0 & 0 & 1 & \frac{1}{3}\\
0 & 0 & 0 & 1\\
0 & 0 & 0 & 0
\end{array}\right] 
& R_5 + \frac{16}{3}R_4 \to R_5
\end{align*}
and we are done (the backward phase is optional for this). Therefore, the four column vectors are linearly independent when considered as a whole, and the direct sum $\mathcal{W}_1 \oplus \mathcal{W}_2 = \text{span}(\{(1,0,2,1,0)^T, (2,1,0,0,-1)^T, (0,3,1,0,0)^T, \\ (0,0,-1,-2,1)^T\})$ makes sense, with $\dim(\mathcal{W}_1 \oplus \mathcal{W}_2) = \dim(\mathcal{W}_1) + \dim(\mathcal{W}_2) \\ = 2+2 = 4$, $\mathcal{W}_1 \oplus \mathcal{W}_2 \subset \mathbb{R}^5$. \par
Now, we attempt to compose $\mathcal{W}_1 \oplus \mathcal{W}_2 \oplus \mathcal{W}_3 = (\mathcal{W}_1 \oplus \mathcal{W}_2) \oplus \mathcal{W}_3$, which requires showing that the only generating vector $(1,1,-3,0,-1)^T$ in $\mathcal{W}_3$ is linearly independent from (the basis vectors of) $\mathcal{W}_1 \oplus \mathcal{W}_2$. One way to do this is to show that the augmented system formed by appending $(1,1,-3,0,-1)^T$ to the matrix at the start
\begin{align*}
\left[\begin{array}{@{\,}wc{10pt}wc{10pt}wc{10pt}wc{10pt}|wc{10pt}@{\,}}
1 & 2 & 0 & 0 & 1\\
0 & 1 & 3 & 0 & 1\\
2 & 0 & 1 & -1 & -3\\
1 & 0 & 0 & -2 & 0\\
0 & -1 & 0 & 1 & -1
\end{array}\right]    
\end{align*}
has no solution and thus $(1,1,-3,0,-1)^T$ cannot be written as their linear combination (see part (a) of Theorem \ref{thm:plusminus}). We can simply repeat the exact same reduction steps performed above, which would lead to
\begin{align*}
\left[\begin{array}{@{\,}wc{10pt}wc{10pt}wc{10pt}wc{10pt}@{\,}|wc{10pt}@{\,}}
1 & 2 & 0 & 0 & 1\\
0 & 1 & 3 & 0 & 1\\
0 & 0 & 1 & \frac{1}{3} & 0\\
0 & 0 & 0 & 1 & -\frac{1}{4} \\
0 & 0 & 0 & 0 & -\frac{7}{3}
\end{array}\right]     
\end{align*}
where the last row is inconsistent. Therefore $(1,1,-3,0,-1)^T$ is linearly independent from the preceding four vectors and $\mathcal{W}_1 \oplus \mathcal{W}_2 \oplus \mathcal{W}_3$ is a valid direct sum, and $\dim(\mathcal{W}_1 \oplus \mathcal{W}_2 \oplus \mathcal{W}_3) = \dim(\mathcal{W}_1 \oplus \mathcal{W}_2) + \dim(\mathcal{W}_3) = 4+1=5$. By Properties \ref{proper:dimWleqV}, $\mathcal{W}_1 \oplus \mathcal{W}_2 \oplus \mathcal{W}_3 = \mathbb{R}^5$. 
\end{solution}
The importance of direct sum is that the coordinates of two vectors in the respective bases from the two subspaces can be simply concatenated when we add up both the vectors and bases, and \textit{this representation will be unique}. Going in the opposite direction, we can also "split" the coordinates of a direct sum back into the respective subspaces. Let's illustrate this with $\mathcal{W}_1$ and $\mathcal{W}_2$ in the above example. Using the given sets of generating vectors $\mathcal{\beta}_1 = \{(1,0,2,1,0)^T, (2,1,0,0,-1)^T\}$ and $\mathcal{\beta}_2 = \{(0,3,1,0,0)^T, (0,0,-1,-2,1)^T\}$ as bases for $\mathcal{W}_1$ and $\mathcal{W}_2$, the coordinates $(1,2)_{\beta_1}^T$ and $(1,-1)_{\beta_2}^T$ in the $\mathcal{\beta}_1$ and $\mathcal{\beta}_2$ system, represent the vectors 
\begin{align*}
(1,0,2,1,0)^T + 2(2,1,0,0,-1)^T &= (5,2,2,1,-2)^T \\
\text{ and } (0,3,1,0,0)^T - (0,0,-1,-2,1)^T &= (0,3,2,2,-1)^T
\end{align*}
in $\mathbb{R}^5$ respectively. When they are summed, it yields $(5,2,2,1,-2)^T + (0,3,2,2,-1)^T = (5,5,4,3,-3)^T$. The basis formed by combining $\mathcal{\beta}_1$ and $\mathcal{\beta}_2$ will be
\begin{align*}
\mathcal{\beta}_1 \cup \mathcal{\beta}_2 = \begin{aligned}
&\{(1,0,2,1,0)^T, (2,1,0,0,-1)^T, \\ &(0,3,1,0,0)^T,  (0,0,-1,-2,1)^T\}    
\end{aligned}
\end{align*}
and the merged coordinates $(1,2,1,-1)_{\beta_1+\beta_2}^T$ then correspond exactly to
\begin{align*}
&\quad (1,0,2,1,0)^T + 2(2,1,0,0,-1)^T + (0,3,1,0,0)^T - (0,0,-1,-2,1)^T \\
&= (5,5,4,3,-3)^T \in \mathcal{W}_1 \oplus \mathcal{W}_2 \subset \mathbb{R}^5
\end{align*}
The new coordinate representation $(1,2,1,-1)_{\beta_1+\beta_2}^T$ is unique as $\mathcal{\beta}_1 \cup \mathcal{\beta}_2$ has been shown to be linearly independent in Example \ref{exmp:directsum} and Properties \ref{proper:lincombofspan} applies over the direct sum $\mathcal{W}_1 \oplus \mathcal{W}_2$, and it can be partitioned cleanly as $(1,2,1,-1)_{\beta_1+\beta_2}^T = (1,2)_{\beta_1}^T + (1,-1)_{\beta_2}^T$.

On the other hand, the uniqueness property will not hold if the subspace sum is not a direct sum. Let's use Example \ref{exmp:gentrimbasis} again as a demonstration, where $\mathcal{\beta}_1 = \{(1,0,2,1)^T, (1,-1,1,-2)^T\}$ and $\mathcal{\beta}_2 = \{(0,0,1,0)^T, (-2,1,0,1)^T\}$ and we have already shown that they are not linearly independent when combined. Take
\begin{align*}
(1,2)_{\beta_1}^T &= (1,0,2,1)^T + 2(1,-1,1,-2)^T = (3,-2,4,-3)^T \\
\text{ and } (-1,1)_{\beta_2}^T &= -(0,0,1,0)^T + (-2,1,0,1)^T = (-2,1,-1,1)^T
\end{align*} Their concatenated sum will be 
\begin{align*}
&\quad (1,2,-1,1)_{\beta_1+\beta_2}^T \\ 
&= (1,0,2,1)^T + 2(1,-1,1,-2)^T - (0,0,1,0)^T + (-2,1,0,1)^T \\
&= (3,-2,4,-3)^T + (-2,1,-1,1)^T \\
&= (1,-1,3,-2)^T  
\end{align*}
but 
\begin{align*}
(0,1)_{\beta_1}^T + (2,0)_{\beta_2}^T = (0,1,2,0)_{\beta_1+\beta_2}^T &= (1,-1,1,-2)^T + 2(0,0,1,0)^T \\
&= (1,-1,3,-2)^T = (1,2,-1,1)_{\beta_1+\beta_2}^T 
\end{align*}
is aptly an alternative representation.

Another aspect of direct sum is that all finite-dimensional, particularly $n$-dimensional vector spaces with some basis consisting of $n$ vectors can be regarded to be a direct sum of the $n$ one-dimensional subspaces generated by each of these basis vectors individually. We have provided a schematic (Figure \ref{fig:directsumeachsubspace}) to better illustrate this.

\begin{landscape}
\begin{figure}
    \begin{tikzpicture}
    \draw[blue] (-1,-0.5,0) -> (4,2,0) node[above left]{$\mathcal{W} = \mathcal{W}_1$};
    \draw[line width=1.2, red, ->] (0,0,0) -> (2,1,0) node[above left]{$\vec{v}^{(1)}$};
    \draw[line width=2, ->] (3,1,0) -- (5,1,0);
    \filldraw[Green!30] (3,-0.5,0) -- (6.5,1.25,0) -- (12.5,2.25,0) -- (9,0.5,0) -- cycle;
    \node[Green] at (11,2.5) {$\mathcal{W}^+ = \mathcal{W}_1 \oplus \mathcal{W}_2$};
    \draw[blue] (4,-0.5,0) -> (9,2,0) node[above left]{$\mathcal{W}_1$};
    \draw[line width=1.2, red, ->] (5,0,0) -> (7,1,0) node[left]{$\vec{v}^{(1)}$};
    \draw[blue] (2,-0.5,0) -> (11,1,0) node[above left]{$\mathcal{W}_2$};
    \draw[line width=1.2, red, ->] (5,0,0) -> (8,0.5,0) node[below right]{$\vec{v}^{(2)}$};
    \draw[line width=2, ->] (12,1,0) -- (12,-1,0);
    \end{tikzpicture}\\
    \begin{tikzpicture}
    \draw[black, fill=yellow, opacity=0.3] (3.5,-0.5,1) -- (7,1.25,1) -- (13,2.25,1) -- (9.5,0.5,1) -- cycle;
    \filldraw[Green!30] (3,-0.5,0) -- (6.5,1.25,0) -- (12.5,2.25,0) -- (9,0.5,0) -- cycle;
    \draw[line width=1.2, red, ->] (5,0,0) -> (7,1,0) node[above, yshift=-2]{$\vec{v}^{(1)}$};
    \draw[line width=1.2, red, ->] (5,0,0) -> (8,0.5,0) node[below right, yshift=5]{$\vec{v}^{(2)}$};
    \draw[blue] (3,0,-4) -> (6,0,2) node[right]{$\mathcal{W}_3$};
    \draw[line width=1.2, red, ->] (5,0,0) -> (4,0,-2) node[right]{$\vec{v}^{(3)}$};
    \draw[black, fill=yellow, opacity=0.3] (2,-0.5,-2) -- (5.5,1.25,-2) -- (11.5,2.25,-2) -- (8,0.5,-2) -- cycle;
    \draw[black, fill=yellow, opacity=0.3] (2,-0.5,-2) -- (3.5,-0.5,1) -- (9.5,0.5,1) -- (8,0.5,-2) -- cycle;
    \draw[black, fill=yellow, opacity=0.3] (11.5,2.25,-2) -- (13,2.25,1) -- (9.5,0.5,1) -- (8,0.5,-2) -- cycle;
    \node[Green!50] at (13,1.5) {$\mathcal{W}^+ = \mathcal{W}_1 \oplus \mathcal{W}_2$};
    \node[orange, align=right] at (3,2) {$\mathcal{W}^{++} = \mathcal{W}^+ \oplus \mathcal{W}_3$ \\
    $= (\mathcal{W}_1 \oplus \mathcal{W}_2) \oplus \mathcal{W}_3$};
    \draw[blue] (3,0,-4) -> (4,0,-2);
    \draw[line width=2, ->] (3,1,0) -- (1,1,0) node[left]{$\cdots$};
    \end{tikzpicture}
    \caption{\textit{Iteratively adding one-dimensional subspaces to the direct sum. Note that the lines, plane, and "cuboid" all extend infinitely. We can only visualize up to a three-dimensional direct sum but it goes on even for higher dimensions.}}
    \label{fig:directsumeachsubspace}
\end{figure}
\end{landscape}

%This also completes our previous proof mentioned in Definition \ref{inverseidentity}.
%\begin{thm}
%If $AP = A$, and $A$ is a invertible square matrix, then $P$ must be $I$.
%\paragraph{Proof}
%The assumption implies that $A$ has linearly independent column vectors. As a result, they cannot be expressed by other vectors. Consider any one of the column vector, like $\vec{u_i}$, then the linear system
%\begin{align*}
%A\vec{x} &= ([\vec{u_1}|\cdots|\vec{u_i}|\cdots|\vec{u_n}])\vec{x} = x_1\vec{u_1} + \cdots + x_i\vec{u_i} + \cdots + x_n\vec{u_n} \\
%&= \vec{u_i}
%\end{align*}
%will only have the solution
%\begin{align*}
%x_j &= 1 & \text{if $j = i$} \\
%x_j &= 0 & \text{if $j \neq i$}
%\end{align*}
%This means that $\vec{x} = \hat{e_i}$. Now if we expand $P = [\vec{p_1}|\cdots|\vec{p_i}|\cdots|\vec{p_n}]$, then we can write $AP = A$ as
%\begin{align*}
%AP &= [A\vec{p_1}|\cdots|A\vec{p_i}|\cdots|A\vec{p_n}] \\
%&= A = [\vec{u_1}|\cdots|\vec{u_i}|\cdots|\vec{u_n}]
%\end{align*}
%The readers are encouraged to verify the expression of $AP = [A\vec{p_1}|\cdots|A\vec{p_i}|\cdots|A\vec{p_n}]$ as a mental exercise, as from time to time we will partition such matrix product into columns. We have just found that for $A\vec{p_i} = \vec{u_i}$ to hold, $\vec{p_i}$ must be $\hat{e_i}$. This implies $P = [\hat{e_1}|...|\hat{e_i}|...|\hat{e_n}] = I$.
%\end{thm}

\section{The Four Fundamental Subspaces Induced by Matrices}

\subsection{Row Space, Column Space}

In Definition \ref{defn:colspace}, we have developed the notion of column space. For an $m\times n$ matrix $A = [\vec{v}^{(1)}|\vec{v}^{(2)}|\cdots|\vec{v}^{(n)}]$, its column space is the subspace generated by the $n$ $\mathbb{R}^m$ vectors $\vec{v}^{(j)}$, $j = 1,2,\ldots,n$. Similarly, we can also define the \index{Row Space}\keywordhl{row space} of a matrix. We formally define both of them below.

\begin{defn}[Column/Row Space]
\label{defn:colrowspace}
For an $m \times n$ real matrix $A$, its column space $\mathcal{C}(A)$ is the subspace spanned by its $n$ column vectors, $\vec{v}^{(1)}, \vec{v}^{(2)}, \ldots, \vec{v}^{(n)} \in \mathbb{R}^m$ as in 
\begin{align}
A = [\vec{v}^{(1)}|\vec{v}^{(2)}|\cdots|\vec{v}^{(n)}]    
\end{align}
Meanwhile its row space is the subspace spanned by its $m$ row vectors $\vec{w}^{(1)T}, \vec{w}^{(2)T}, \ldots, \vec{w}^{(m)T} \in \mathbb{R}^n$ as in
\begin{align}
A = 
\left[\begin{array}{c}
\vec{w}^{(1)T} \\
\hline
\vec{w}^{(2)T} \Tstrut\\
\hline
\vdots \\
\hline
\vec{w}^{(m)T} \Tstrut
\end{array}\right]
\end{align}
Notice that the row [column] space of a matrix is just the column [row] space of its transpose, hence we denote the row space of $A$ as $\mathcal{C}(A^T)$.
\end{defn}

For instance, in Example \ref{exmp:gentrimbasis}, the matrix
\begin{align*}
A &= 
\begin{bmatrix}
1 & 1 & 0 & -2\\
0 & -1 & 0 & 1\\
2 & 1 & 1 & 0\\
1 & -2 & 0 & 1
\end{bmatrix}
\end{align*}
actually has a column space of $\mathcal{C}(A) = \text{span}(\{(1,0,2,1)^T, (1,-1,1,-2)^T, \allowbreak (0,0,1,0)^T, (-2,1,0,1)^T\}) = \text{span}(\{(1,0,2,1)^T, (1,-1,1,-2)^T, (0,0,1,0)^T \allowbreak \})$ of dimension $3$ despite the vectors are in $\mathbb{R}^4$. In the middle of deriving this result, we have produced the reduced row echelon form of $A$, which is 
\begin{align*}
\begin{bmatrix}
1 & 0 & 0 & -1\\
0 & 1 & 0 & -1\\
0 & 0 & 1 & 3\\
0 & 0 & 0 & 0 
\end{bmatrix}
\end{align*}
from which we can see the number of pivots, or \textit{rank}, is also $3$. In fact, just like the case above, the \index{Rank}\keywordhl{rank} of a matrix always indicates the dimension of its column space. This is due to Properties \ref{proper:findgenbasis} and \ref{proper:samenvecsbases}, leading to the following equivalent definition.
\begin{defn}[Rank]
\label{defn:rank}
The rank of a matrix $A$ is the number of leading $1$s in its reduced row echelon form, which is also the amount of linearly independent vectors in any basis of its column space, i.e. the dimension of the column space.
\end{defn}
We can approach this from another angle, which involves restating previous results related to Gaussian Elimination. In Section \ref{section:linearind}, we have shown that elementary row operations preserve (the amount of) linearly independent vectors, hence
\begin{proper}
\label{proper:elemrowopcolrank}
Elementary row operations do not change the number of dimensions in the column space of a matrix.
\end{proper}
The matrix $A$ will then have the same number of dimensions in its column space throughout the Gaussian Elimination procedure, which coincides with the number of linearly independent vectors and thus pivots in the final reduced row echelon form, establishing the equivalence in Definition \ref{defn:rank}. However, notice that elementary row operations do change the actual column space. On the other hand, for row space, we have an even stronger result.
\begin{proper}
\label{proper:elemrowoprowrank}
Elementary row operations do not change the row space of a matrix and thus its dimension.
\end{proper}
which is not hard to accept. Swapping rows, and multiplying a row by some constant obviously does not affect the span of rows in the matrix. Adding to/subtracting from a row $R_p$ (also as a row vector $\vec{w}^{(p)T}$) by the constant multiple of another row $R_q$ ($\vec{w}^{(q)T}$) also will not alter it. To see this, observe that the newly resulting row vector is just a linear combination of the two input rows, i.e.\ the new $R_p$ becomes $\vec{w}^{(r)T} = \vec{w}^{(p)T} + c\vec{w}^{(q)T}$ (and hence $\vec{w}^{(p)T} = \vec{w}^{(r)T} - c\vec{w}^{(q)T}$). Using part (b) of Theorem \ref{thm:plusminus} twice, we have
\begin{align*}
&= \text{span}(\{\ldots, \vec{w}^{(p)}, \ldots, \vec{w}^{(q)}, \ldots, \vec{w}^{(r)}\}) \\
\mathcal{C}(A^T) &= \text{span}(\{\ldots, \vec{w}^{(p)}, \ldots, \vec{w}^{(q)}, \ldots\}) \\
&= \text{span}(\{\ldots, \vec{w}^{(r)}, \ldots, \vec{w}^{(q)}, \ldots\}) = \mathcal{C}(A'^T)
\end{align*}
where $A'$ denotes the matrix after the addition/subtraction elementary row operation. Our next key result relies on the observation that the dimensions of row and column space of a matrix in its reduced row echelon form are the same, or in other words,
\begin{proper}
\label{proper:rrefcolrowrank}
A matrix in its reduced row echelon form has the same number of (linearly independent) vectors in the basis of its row and column space.
\end{proper}
We will not read off the detailed arguments in the proof, but instead note that it is essentially an analysis of positions of the leading $1$s and zeros in any reduced row echelon form. However, we will give an example to elucidate how it holds. Take a reduced row echelon form of
\begin{align*}
\begin{bmatrix}
1 & 1 & 0 & 0 & 1 \\
0 & 0 & 1 & 0 & 0 \\
0 & 0 & 0 & 1 & 1 \\
0 & 0 & 0 & 0 & 0 
\end{bmatrix}
\end{align*}
It is obvious that its column space is spanned by the basis $\{(1,0,0,0)^T, \allowbreak (0,1,0,0)^T, (0,0,1,0)^T\}$, while a basis of its row space can be simply formed by the first three non-zero row vectors $\{(1,1,0,0,1), (0,0,1,0,0), (0,0,0,1,1)\}$. In this case, the dimensions of row/column space of the reduced row echelon form are both $3$. With these observations, we can derive the desired result, sometimes referred to as \textit{"Column rank equals row rank"}.
\begin{proper}
\label{proper:samecolrowrank}
For any matrix, the dimension of its column space is equal to that of its row space, i.e.
\begin{align}
\dim(\mathcal{C}(A)) = \dim(\mathcal{C}(A^T))
\end{align}
\end{proper}
\begin{proof}
Any matrix has a unique reduced row echelon form due to Theorem \ref{thm:uniquerref}, whose row/column space has the same number of dimensions by Properties \ref{proper:rrefcolrowrank}. According to Properties \ref{proper:elemrowopcolrank} and \ref{proper:elemrowoprowrank}, the elementary row operations done to convert the matrix to its reduced row echelon form leave both the dimensions of row and column space conserved, and thus the column rank and row rank in the starting matrix are equal.
\end{proof}
\begin{exmp}
\label{exmp:colrowspace}
Given a matrix
\begin{align*}
A = 
\begin{bmatrix}
1 & 1 & -2 & 1 \\
1 & 2 & 1 & -1 \\
1 & 0 & -5 & 3
\end{bmatrix}
\end{align*}
find a basis for its column/row space $\mathcal{C}(A)$ and $\mathcal{C}(A^T)$ and check if Properties \ref{proper:samecolrowrank} holds.
\end{exmp}
\begin{solution}
We first apply Gaussian Elimination to $A$, which leads to
\begin{align*}
\left[\begin{array}{@{\,}wc{10pt}wc{10pt}wc{10pt}wc{10pt}@{\,}}
1 & 1 & -2 & 1 \\
1 & 2 & 1 & -1 \\
1 & 0 & -5 & 3
\end{array}\right]
& \rightarrow
\left[\begin{array}{@{\,}wc{10pt}wc{10pt}wc{10pt}wc{10pt}@{\,}}
1 & 1 & -2 & 1 \\
0 & 1 & 3 & -2 \\
0 & -1 & -3 & 2
\end{array}\right]
& \begin{aligned}
R_2 - R_1 &\to R_2 \\
R_3 - R_1 &\to R_3
\end{aligned} \\
& \rightarrow
\left[\begin{array}{@{\,}wc{10pt}wc{10pt}wc{10pt}wc{10pt}@{\,}}
1 & 1 & -2 & 1 \\
0 & 1 & 3 & -2 \\
0 & 0 & 0 & 0
\end{array}\right]
& R_3 - R_2 \to R_3 \\
& \rightarrow
\left[\begin{array}{@{\,}wc{10pt}wc{10pt}wc{10pt}wc{10pt}@{\,}}
1 & 0 & -5 & 3 \\
0 & 1 & 3 & -2 \\
0 & 0 & 0 & 0
\end{array}\right]
& R_1 - R_2 \to R_1
\end{align*}
The number of pivotal columns is $2$, and from the reduced row echelon form, we obtain the dependence relations where the first two column vectors $(1,1,1)^T$ and $(1,2,0)^T$ are linearly independent while the last two column vectors $(-2,1,-5)^T = -5(1,1,1)^T + 3(1,2,0)^T$ and $(1,-1,3)^T = 3(1,1,1)^T - 2(1,2,0)^T$ are linear combinations of the previous two. Hence $\mathcal{C}(A)$ has a basis of $\{(1,1,1)^T, (1,2,0)^T\}$ and $\dim(\mathcal{C}(A)) = 2$. On the other hand, to find the row space we consider $A^T$ and repeat the elimination process again as follows. However, notice that according to the dependence relations for the column vectors in $A$ above, we can immediately do the corresponding addition/subtraction operations for the rows in $A^T$, to reduce the third/fourth rows, obtaining
\begin{align*}
\left[\begin{array}{@{\,}wc{10pt}wc{10pt}wc{10pt}wc{10pt}@{\,}}
1 & 1 & 1 \\
1 & 2 & 0 \\
-2 & 1 & -5 \\
1 & -1 & 3
\end{array}\right] 
&\rightarrow
\left[\begin{array}{@{\,}wc{10pt}wc{10pt}wc{10pt}wc{10pt}@{\,}}
1 & 1 & 1 \\
1 & 2 & 0 \\
0 & 0 & 0 \\
0 & 0 & 0
\end{array}\right] 
& \begin{aligned}
R_3 + 5R_1 - 3R_2 &\to R_3, \\
R_4 - 3R_1 + 2R_2 &\to R_4
\end{aligned}
\end{align*}
and the next step is straightforward:
\begin{align*}
\left[\begin{array}{@{\,}wc{10pt}wc{10pt}wc{10pt}wc{10pt}@{\,}}
1 & 1 & 1 \\
1 & 2 & 0 \\
0 & 0 & 0 \\
0 & 0 & 0
\end{array}\right] 
&\rightarrow
\left[\begin{array}{@{\,}wc{10pt}wc{10pt}wc{10pt}wc{10pt}@{\,}}
1 & 1 & 1 \\
0 & 1 & -1 \\
0 & 0 & 0 \\
0 & 0 & 0
\end{array}\right] 
& R_2 - R_1 \to R_2 \\
&\rightarrow
\left[\begin{array}{@{\,}wc{10pt}wc{10pt}wc{10pt}wc{10pt}@{\,}}
1 & 0 & 2 \\
0 & 1 & -1 \\
0 & 0 & 0 \\
0 & 0 & 0
\end{array}\right] 
& R_1 - R_2 \to R_1
\end{align*}
which reveals that the first two columns (representing the first two row vectors in $A$) are linearly independent and the third column (the last row vector in $A$) is redundant ($(1,0,-5,3)^T = 2(1,1,-2,1)^T-(1,2,1,-1)^T$). Therefore $\mathcal{C}(A^T)$ has a basis of $\{(1,1,-2,1)^T, (1,2,1,-1)^T\}$, and $\dim(\mathcal{C}(A^T)) = 2 = \dim(\mathcal{C}(A))$, and Properties \ref{proper:samecolrowrank} is true in this case.
\end{solution}
Finally, in view of Definitions \ref{defn:span} and \ref{defn:colrowspace}, the analysis of solving linear systems in Section \ref{section:SolveLinSys} can be summarized as follows.
\begin{proper}
\label{proper:consistentcolspace}
A linear system $A\vec{x} = \vec{h}$ is consistent if and only if $\vec{h}$ is in the column space of $A$.
\end{proper}

\subsection{Null Space, Rank-Nullity Theorem}
\label{section:null}

As we have briefly mentioned at the end of the last chapter, the solution of a linear system $A\vec{x} = \vec{h}$, where $A$ is an $m \times n$ matrix and $\vec{x} \in \mathbb{R}^n$, can be viewed as some sort of solution space. In Section \ref{subsection:SolLinSysGauss} we know that it is made up of the particular and complementary solution, where the latter corresponds to the family of $\vec{x} = \vec{x}_0$ ($= \vec{x}_c$ using the notation in that section) that satisfies the homogeneous part $A\vec{x} = \textbf{0}$. The set $\vec{x}_0 \in \mathbb{R}^n$ can be shown to form a subspace of $\mathbb{R}^n$, and this subspace is then called the \index{Null Space}\keywordhl{null space} of $A$. To verify, we check the two conditions in Theorem \ref{thm:subspacecriteria}. Let $\smash{\vec{x}_0^{(1)}}$ and $\smash{\vec{x}_0^{(2)}}$ be two vectors in the null space $\vec{x}_0$. Then we have: 1. $A\smash{(\vec{x}_0^{(1)} + \vec{x}_0^{(2)})} = A\smash{\vec{x}_0^{(1)}} + A\smash{\vec{x}_0^{(2)}} = \textbf{0} + \textbf{0} = \textbf{0}$, so $\smash{\vec{x}_0^{(1)}} + \smash{\vec{x}_0^{(2)}} \in \vec{x}_0$, and 2. $A(a\smash{\vec{x}_0^{(1)}}) = a(A\smash{\vec{x}_0^{(1)}}) = a\textbf{0} = \textbf{0}$, hence $a\smash{\vec{x}_0^{(1)}} \in \vec{x}_0$.
\begin{defn}[Null Space]
\label{defn:nullspace}
For an $m \times n$ real matrix $A$, its null space $\mathcal{N}(A)$ is the subspace consisting of all solution vectors $\vec{x} = \vec{x}_0 \in \mathbb{R}^n$ to the matrix equation $A\vec{x} = \textbf{0}$. The dimension of null space is called \index{Nullity}\keywordhl{nullity}.
\end{defn}
This definition of nullity as the dimension of null space is consistent with that in Section \ref{subsection:SolLinSysGauss} where nullity is initially given by the number of columns in the matrix minus the number of leading $1$s (rank) in its RREF, or equivalently the number of non-pivotal columns. To see this, observe that any solution $\vec{x}_0$ to $A\vec{x} = \textbf{0}$ is also the solution to $A_{\text{rref}}\vec{x} = \textbf{0}$ and vice versa, via applying elementary matrices. Hence the null space and nullity of $A$ will be the same as that of $A_{\text{rref}}$. Previously we have assigned free variables to the non-pivotal columns (let's say there is $k$ of them) of $A_{\text{rref}}$ and derive $\vec{x}_0$ where they are generated by $k$ pairs of free variables and column vectors ($\smash{\vec{x}_0^{(1)}}, \smash{\vec{x}_0^{(2)}}, \ldots, \smash{\vec{x}_0^{(k)}}$). It is clear that such a procedure will ensure these $k$ vectors are linearly independent as each of them has a component of $1$ in the position corresponding to that particular free variable indicated by the RREF and $0$s in other positions corresponding to other free variables (see Example \ref{exmp:underdetsys} for an instance). We claim that they also span the entire null space of $A_{\text{rref}}$.\footnote{Assume the contrary so that the span of $\{\smash{\vec{x}_0^{(1)}}, \smash{\vec{x}_0^{(2)}}, \ldots, \smash{\vec{x}_0^{(k)}}\}$ does not cover the whole null space, then the dimension of null space has to be greater than $k$ by Properties $\ref{proper:dimWleqV}$. Without loss of generality, let the "correct" dimension of null space to be $k+1$. Then by (c) of Properties \ref{proper:linindspanbasisnewver}, there exists $\smash{\vec{x}_0^{(k+1)}}$ such that $\{\smash{\vec{x}_0^{(1)}}, \smash{\vec{x}_0^{(2)}}, \ldots, \smash{\vec{x}_0^{(k)}}, \smash{\vec{x}_0^{(k+1)}}\}$ is a basis of the null space. This $\smash{\vec{x}_0^{(k+1)}}$ can be made to take the value of $0$ at all $k$ positions where the free variables reside by subtracting it by appropriate multiples of $\smash{\vec{x}_0^{(j)}}$, $j \neq k+1$, without altering the null space (c.f.\ Footnote \ref{foot:ft14}), and by doing so non-zero components of $\smash{\vec{x}_0^{(k+1)}}$ only appear in positions corresponding to leading $1$s in $A_{\text{rref}}$. This causes a contradiction since $A_{\text{rref}}\smash{\vec{x}_0^{(k+1)}} = \textbf{0}$ then implies that there exists a non-trivial dependence relation between the pivotal column vectors themselves.} Hence by the definition given in Section \ref{section:subspacebasis} they form a basis for the null space of $A_{\text{rref}}$ as well as $A$ and by Properties \ref{proper:samenvecsbases} the dimension of null space of $A$ is also $k$.

Using Definitions \ref{defn:colrowspace}, \ref{defn:rank}, and \ref{defn:nullspace} to rephrase, the preceding discussion means that the rank of a matrix plus its nullity equals its number of columns, which leads to the so-called \index{Rank-nullity Theorem}\keywordhl{Rank-nullity Theorem}.
\begin{thm}[Rank-nullity Theorem]
\label{thm:ranknullity}
For a real $m \times n$ matrix $A$, we have
\begin{subequations}
\begin{align}
\dim(\mathcal{C}(A)) + \dim(\mathcal{N}(A)) &= \text{rank}(A) + \text{nullity}(A) = n \\
&= \dim(\mathcal{C}(A^T)) + \dim(\mathcal{N}(A))
\end{align}
\end{subequations}
where $\dim(\mathcal{C}(A)) = \dim(\mathcal{C}(A^T)) = \text{rank}(A)$ by Properties \ref{proper:samecolrowrank}.
\end{thm}
An invertible square matrix has a reduced row echelon form of an identity matrix according to Theorem \ref{thm:equiv3}, and since an identity matrix has \textit{full rank}\footnote{An $m \times n$ matrix $A$ is said to have full rank if $\text{rank}(A) = \text{min}(m,n)$.}, by Definition \ref{defn:rank} and the Rank-nullity Theorem \ref{thm:ranknullity} above, we have
\begin{proper}
\label{proper:invertrank}
A $n \times n$ square matrix is invertible if and only if its rank and nullity are $n$ and $0$. 
\end{proper}
A notable relationship between row space and null space is that any pair of two vectors coming from the respective subspaces will be orthogonal to each other. 
\begin{proper}
\label{proper:rownullortho}
Given a real matrix $A$, any vector in its row space $\mathcal{C}(A^T)$ is orthogonal to all vectors in its null space $\mathcal{N}(A)$ and vice versa.
\end{proper}
\begin{proof}
Let the shape of $A$ be $m \times n$, we can express $A$ in the form of its row vectors as
\begin{align*}
A = 
\left[\begin{array}{c}
\vec{w}^{(1)T} \\
\hline
\vec{w}^{(2)T} \Tstrut \\
\hline
\vdots \\
\hline
\vec{w}^{(m)T} \Tstrut
\end{array}\right]
\end{align*}
and the corresponding homogeneous system $A\vec{x} = \textbf{0}$ then can be written as
\begin{align*}
A\vec{x} =
\left[\begin{array}{c}
\vec{w}^{(1)T} \\
\hline
\vec{w}^{(2)T} \Tstrut \\
\hline
\vdots \\
\hline
\vec{w}^{(m)T} \Tstrut
\end{array}\right]
\vec{x}
=
\left[\begin{array}{c}
\vec{w}^{(1)T} \cdot \vec{x} \\
\vec{w}^{(2)T} \cdot \vec{x} \Tstrut \\
\vdots \\
\vec{w}^{(m)T} \cdot \vec{x} \Tstrut
\end{array}\right]
= \textbf{0}
=
\left[\begin{array}{c}
0 \\
0 \Tstrut \\
\vdots \\
0 \Tstrut
\end{array}\right]
\end{align*}
where for a solution $\vec{x} = \vec{x}_0$ in the null space of $A$, each of the dot products $\vec{w}^{(i)T} \cdot \vec{x}_0 = 0$, $i = 1, 2, \ldots, m$, has to be equal to zero. Any vector in the row space of $A$ can be expressed as $\vec{w} = c_1\vec{w}^{(1)} + c_2\vec{w}^{(2)} + \cdots + c_m\vec{w}^{(m)}$ by Definitions \ref{defn:colrowspace} and \ref{defn:span}, and subsequently, its dot product with $\vec{x}_0$
\begin{align*}
\vec{w}^T \cdot \vec{x}_0 &= (c_1\vec{w}^{(1)} + c_2\vec{w}^{(2)} + \cdots + c_m\vec{w}^{(m)})^T \cdot \vec{x}_0 \\
&= c_1(\vec{w}^{(1)T} \cdot \vec{x}_0) + c_2(\vec{w}^{(2)T} \cdot \vec{x}_0) + \cdots + c_m(\vec{w}^{(m)T} \cdot \vec{x}_0) \\
&= c_1(0) + c_2(0) + \cdots + c_m(0) = 0
\end{align*}
is also zero, therefore they are orthogonal by Properties \ref{proper:dotorth}, which implies that any vector in $\mathcal{C}(A^T)$ is orthogonal to any another vector in $\mathcal{N}(A)$.
\end{proof}
As a corollary, this is equivalent to all vectors in the generating set or basis for the row space for a matrix being orthogonal to all vectors in those for its null space. The following additional observation will be useful later.
\begin{proper}
\label{proper:ortholinind}
Non-zero orthogonal vectors are linearly independent.
\end{proper}
\begin{proof}
We will only prove the case with two vectors in $\mathbb{R}^n$ but those with multiple vectors can be derived in the same essence. Consider $c_1\vec{u}^{(1)} + c_2\vec{u}^{(2)} = \textbf{0}$ where $\vec{u}^{(1)}$ and $\vec{u}^{(2)}$ are orthogonal, i.e. $\vec{u}^{(1)} \cdot \vec{u}^{(2)} = 0$. Taking dot product with $\vec{u}_1$ on both sides gives
\begin{align*}
\vec{u}^{(1)} \cdot (c_1\vec{u}^{(1)} + c_2\vec{u}^{(2)}) = c_1(\vec{u}^{(1)} \cdot \vec{u}^{(1)}) + c_2(\vec{u}^{(1)} \cdot \vec{u}^{(2)}) &= \vec{u}^{(1)} \cdot \textbf{0} \\
c_1 \norm{\vec{u}^{(1)}}^2 + c_2 (0) = c_1 \norm{\vec{u}^{(1)}}^2 &= 0
\end{align*}
Since $\vec{u}_1$ is non-zero, $\norm{\vec{u}_1}^2 > 0$, and $c_1$ must be zero. In a similar vein, we can show that $c_2$ is zero as well. Therefore the only solution to the equation $c_1\vec{u}_1 + c_2\vec{u}_2 = \textbf{0}$ is the trivial solution $c_1 = c_2 = 0$. By Theorem \ref{thm:linearindep}, the two vectors are linearly independent.
\end{proof}
\begin{exmp}
\label{exmp:colrowspace2}
For the matrix in Example \ref{exmp:colrowspace}, find its null space and check if Properties \ref{proper:rownullortho} and Theorem \ref{thm:ranknullity} hold.
\end{exmp}
\begin{solution}
The homogeneous system corresponding to the matrix is
\begin{align*}
\left[\begin{array}{@{\,}wc{10pt}wc{10pt}wc{10pt}wc{10pt}|wc{10pt}@{\,}}
1 & 1 & -2 & 1 & 0\\
1 & 2 & 1 & -1 & 0\\
1 & 0 & -5 & 3 & 0
\end{array}\right]
\end{align*}
which can be reduced, following the same steps in Example \ref{exmp:colrowspace}, to
\begin{align*}
\left[\begin{array}{@{\,}wc{10pt}wc{10pt}wc{10pt}wc{10pt}|wc{10pt}@{\,}}
1 & 0 & -5 & 3 & 0 \\
0 & 1 & 3 & -2 & 0\\
0 & 0 & 0 & 0 & 0
\end{array}\right]
\end{align*}
where there are two non-pivotal columns and hence two free parameters can be assigned to them. Let $x_3 = s$ and $x_4 = t$, then $x_1 = 5s - 3t$ and $x_2 = -3s + 2t$. So the solution to the system is
\begin{align*}
\begin{bmatrix}
x_1 \\
x_2 \\
x_3 \\
x_4
\end{bmatrix}
=
\begin{bmatrix}
5s-3t \\
-3s+2t \\
s \\
t
\end{bmatrix}
=
s
\begin{bmatrix}
5\\
-3\\
1\\
0
\end{bmatrix}
+ t
\begin{bmatrix}
-3 \\
2 \\
0 \\
1
\end{bmatrix}
\end{align*}
and thus a basis for the null space is $\{(5,-3,1,0)^T, (-3,2,0,1)^T\}$ where these two vectors are clearly linearly independent (by observing the $0$ and $1$ of the last two components). As found in Example \ref{exmp:colrowspace}, the basis for its row space is $\{(1,1,-2,1)^T, (1,2,1,-1)^T\}$. Subsequently, checking orthogonality between the two bases is straightforward, and we will only do this for the first vector in the row space basis against the null space basis.
\begin{align*}
(5,-3,1,0)^T \cdot (1,1,-2,1)^T &= (5)(1)+(-3)(1)+(1)(-2)+(0)(1) = 0 \\
(-3,2,0,1)^T \cdot (1,1,-2,1)^T &= (-3)(1)+(2)(1)+(0)(-2)+(1)(1) = 0
\end{align*}
Furthermore, the dimension of null space, or the nullity, is $\dim{\mathcal{N}(A)} = 2$. Previously we have also found that $\dim{\mathcal{C}(A)} = \text{rank}(A) = 2$. So $\text{rank}(A) + \text{nullity}(A) = 2+2 = 4$, and Theorem \ref{thm:ranknullity} is true.
\end{solution}
Short Exercise: Show that\footnote{Replace $A$ by $A^T$ in Theorem \ref{thm:ranknullity} to get $\dim(\mathcal{C}(A^T)) + \dim(\mathcal{N}(A^T)) = m$.}
\begin{align*}
\dim(\mathcal{C}(A^T)) + \dim(\mathcal{N}(A^T)) = \dim(\mathcal{C}(A)) + \dim(\mathcal{N}(A^T)) = m   
\end{align*} $\mathcal{N}(A^T)$ is also known as the \index{Left Null Space}\keywordhl{left null space} of $A$.

By Properties \ref{proper:rownullortho} and \ref{proper:ortholinind}, vectors in the row space and null space of an $m \times n$ matrix $A$ are linearly independent of each other and can form a direct sum $\mathcal{C}(A^T) \oplus \mathcal{N}(A)$ according to Definition \ref{defn:directsum}. Note that they are the complement to each other with respect to this direct sum according to Properties \ref{proper:complement}. Since $\mathcal{C}(A^T) \subseteq \mathbb{R}^n$, $\mathcal{N}(A) \subseteq \mathbb{R}^n$ and hence $\mathcal{C}(A^T) \oplus \mathcal{N}(A) \subseteq \mathbb{R}^n$, from Theorem \ref{thm:ranknullity} and Properties \ref{proper:dimWleqV}, we conclude that $\mathcal{C}(A^T) \oplus \mathcal{N}(A)$ which has a dimension of $n$, is just $\mathbb{R}^n$. In other words, the row space and null space of a matrix can reconstruct the real $n$-space by forming their direct sum. The same can be said for its column and left null space. Furthermore, since all vectors in the row space (column space) are orthogonal to those in the (left) null space (and vice versa) via Properties \ref{proper:rownullortho}, we say that they are actually an \index{Orthogonal Complement}\keywordhl{orthogonal complement} to each other.

\begin{proper}
\label{proper:funsubsortho}
For a real $m \times n$ matrix $A$, we have
\begin{align}
& \mathcal{C}(A^T) \oplus \mathcal{N}(A) = \mathbb{R}^n & & \mathcal{C}(A) \oplus \mathcal{N}(A^T) = \mathbb{R}^m
\end{align}
where $\mathcal{C}(A^T)^\perp = \mathcal{N}(A)$, $\mathcal{N}(A)^\perp = \mathcal{C}(A^T)$, and $\mathcal{C}(A)^\perp = \mathcal{N}(A^T)$, $\mathcal{N}(A^T)^\perp = \mathcal{C}(A)$, with $^\perp$ denoting an orthogonal complement.
\end{proper}

We conclude the relationships between the column, row, null, and left null space, a.k.a the \index{The Four Fundamental Subspaces}\keywordhl{Four fundamental subspaces} induced by a matrix, with a diagram (Figure \ref{fig:foursubspaces}). \par
\begin{figure}
    \centering
    \begin{tikzpicture}[>={Stealth[length=5pt]}]
    \node[opacity=0.1,scale=5] at (0,0) {$\mathbb{R}^n$}; 
    \node[opacity=0.1,scale=5] at (8,0) {$\mathbb{R}^m$};
    \draw[SkyBlue] (0,0) -- (2,2.8) -- (-0.1,4.3) -- (-2.1,1.5) -- cycle; 
    \node[align=left, scale=0.7] at (0,2.3) {Row space \\ $\mathcal{C}(A^T)$};
    \draw[SkyBlue] (0,0) -- (-1, -1.4) -- (1.8,-3.4) -- (2.8,-2) -- cycle;
    \node[align=left, scale=0.7] at (0.9,-1.7) {Null space \\ $\mathcal{N}(A)$};
    \draw (0.2, 0.28) -- (0.48, 0.08) -- (0.28, -0.2);
    \node at (0,5) {$\dim = r$};
    \node at (-1,-3) {$\dim = n-r$};
    \draw[CarnationPink] (8,0) -- (6,2.8) -- (8.1,4.3) -- (10.1,1.5) -- cycle;
    \node[align=left,scale=0.7] at (8.1,2) {Column space \\ $\mathcal{C}(A)$};
    \draw[CarnationPink] (8,0) -- (8.8,-8/7) -- (4.8,-2.8-8/7) -- (4,-2.8) -- cycle;
    \node[align=left,scale=0.7] at (6.6,-2) {Left null space \\ $\mathcal{N}(A^T)$};
    \draw (7.8, 0.28) -- (7.52, 0.08) -- (7.72, -0.2);
    \node at (6.5,4.5) {$\dim = r$};
    \node at (9,-2.5) {$\dim = m-r$};
    \node[scale=2.5, opacity=0.5] at (3.5,4) {$A_{m \times n}$};
    \node[circle,fill,inner sep=2pt,Green] at (0,0) {};
    \node at (-0.5,0) {$\textbf{0}$};
    \node[circle,fill,inner sep=2pt,yellow] at (8,0) {};
    \node at (8.5,0) {$\textbf{0}$};
    \draw[->] (1,0.5) -- (7,2.5) node[sloped, midway, below]{$A\vec{x} = \vec{h}$};
    \draw[dashed, -{Circle}] (1,0.5) -- (-0.4,1.5) node[left]{$\vec{x}_r$};
    \draw[dashed, -{Circle}] (1,0.5) -- (-0.2,-1.18) node[below]{$\vec{x}_n$};
    \node[circle,fill,inner sep=2pt] at (1,0.5) {};
    \node at (1,0.5) [below right]{$\vec{x} = \vec{x}_r + \vec{x}_n$};
    \draw[dashed, ->] (-0.2,-1.18) -- (8,0) node[sloped, pos=0.7, above]{$A\vec{x}_n = \textbf{0}$};
    \draw[dashed, ->] (-0.4,1.5) -- (7,2.5) node[sloped, midway, above]{$A\vec{x}_r = \vec{h}$};
    \node[above] at (7,2.6) {$\vec{h}$};
    \end{tikzpicture}
    \caption{\textit{The relationships between the four fundamental subspaces for an $m \times n$ real matrix $A$ of rank $r$: the row space $\mathcal{C}(A^T)$, null space $\mathcal{N}(A)$, column space $\mathcal{C}(A)$, left null space $\mathcal{N}(A^T)$. Any vector $\vec{x} \in \mathbb{R}^n$ can be partitioned into $\vec{x} = \vec{x}_r + \vec{x}_n$ uniquely, where $\vec{x}_r \in \mathcal{C}(A^T) \subseteq \mathbb{R}^n$ and $\vec{x}_n \in \mathcal{N}(A) \subseteq \mathbb{R}^n$ are in the row/null space of $A$ respectively. The matrix $A$ maps $\vec{x}_n$ to the zero vector in $\mathbb{R}^m$ and $\vec{x}_r$ to some vector $\vec{h} \in \mathcal{C}(A) \subseteq \mathbb{R}^m$ in the column space of $A$. The total effect on $\vec{x}$ multiplied by $A$, is the sum of the two responses: $A\vec{x} = A(\vec{x}_r + \vec{x}_n) = A\vec{x}_r + A\vec{x}_n = \vec{h} + \textbf{0} = \vec{h}$.}}
    \label{fig:foursubspaces}
\end{figure}

Last but not least, we end this chapter with a useful result.
\begin{proper}
\label{proper:rankABsmaller}
For two (real) $m \times r$ and $r \times n$ matrices $A$ and $B$, the rank of $AB$
\begin{align}
\text{rank}(AB) \leq \min(\text{rank}(A),\text{rank}(B))
\end{align}
is capped by the smaller of ranks of $A$ and $B$.
\end{proper}
\begin{proof}
The column space of $AB$ is a subset of that of $A$, i.e.\ $\mathcal{C}(AB) \subseteq \mathcal{C}(A)$, because the columns of $AB$ can be viewed as
\begin{align*}
AB &= \begin{bmatrix}
A_1|A_2|\cdots|A_r
\end{bmatrix}
\begin{bmatrix}
b_{11} & b_{12} & \cdots & b_{1n} \\
b_{21} & b_{22} & & b_{2n} \\
\vdots & & \ddots & \vdots \\
b_{r1} & b_{r2} & \cdots & b_{rn}
\end{bmatrix} \\
&= \begin{bmatrix}
b_{11}A_1 + b_{21}A_2 + \cdots + b_{r1}A_r|\cdots|b_{1n}A_1 + b_{2n}A_2 + \cdots + b_{rn}A_r
\end{bmatrix}
\end{align*}
(where $A_j$ is the $j$-th column of $A$) which shows that the columns of $AB$ are linear combinations of columns in $A$ and hence are in the column space of $A$. By Properties \ref{proper:WcontainsspanS}, the column space of $A$ then contains the column space of $AB$. Applying Properties \ref{proper:dimWleqV} we have $\text{rank}(AB) = \dim(\mathcal{C}(AB)) \leq \dim(\mathcal{C}(A)) = \text{rank}(A)$. The same argument on the row space of $AB$ and $B$ similarly shows that $\text{rank}(AB) \leq \text{rank}(B)$ and the desired inequality follows.
\end{proof}
Short Exercise: Show that $\text{rank}(AB) = \text{rank}(A)$ if $B$ is an invertible square matrix.\footnote{$\text{rank}(A) = \text{rank}(ABB^{-1}) \leq \text{rank}(AB) \leq \text{rank}(A)$ by applying Properties \ref{proper:rankABsmaller} twice. So $\text{rank}(AB)$ is "sandwiched" by and must be equal to $\text{rank}(A)$.}

\section{Python Programming}
To check linear independence and find a basis for columns in a matrix (or in general, any basis from a spanning set), we can use the \verb|columnspace| method in \verb|sympy|. Let's test it with the matrix in Example \ref{exmp:colrowspace}.
\begin{lstlisting}
import sympy

myMatrix = sympy.Matrix([[1., 1., -2., 1.],
                         [1., 2., 1., -1.],
                         [1., 0., -5., 3.]])
print(myMatrix.columnspace())
\end{lstlisting}
which gives
\begin{lstlisting}
[Matrix([        
[1.0],
[1.0],
[1.0]]), 
Matrix([
[1.0],
[2.0],
[  0]])]   
\end{lstlisting}
as expected. The rank can be found in two ways.
\begin{lstlisting}
print(myMatrix.rank()) # or len(myMatrix.columnspace())
\end{lstlisting}
This returns \verb|2| correctly. We can make a basis for the row space similarly by the \verb|rowspace| method. In the same manner, the null space is computed by the \verb|nullspace| method:
\begin{lstlisting}
print(myMatrix.nullspace())
\end{lstlisting}
producing an output of
\begin{lstlisting}
[Matrix([
[ 5.0],
[-3.0],
[   1],
[   0]]), 
Matrix([
[-3.0],
[ 2.0],
[   0],
[   1]])]
\end{lstlisting}
The nullity is then simply calculated by \verb|len(myMatrix.nullspace())|, which gives a right answer of \verb|2|. Finally, CR Factorization is computed via the \verb|rank_decomposition| method, where
\begin{lstlisting}
C, R = myMatrix.rank_decomposition()
print(C, R)    
\end{lstlisting}
gives
\begin{lstlisting}
Matrix([[1.00, 1.00], 
        [1.00, 2.00], 
        [1.00, 0]])
Matrix([[1, 0, -5.00, 3.00], 
        [0, 1, 3.00, -2.00]])        
\end{lstlisting}
where the matrix $C$ is essentially the same as that comes from \verb|columnspace|.

\section{Exercises}

\begin{Exercise}
For $\vec{v}^{(1)} =
\begin{bmatrix}
1\\
1\\
0
\end{bmatrix}$,
$\vec{v}^{(2)} =
\begin{bmatrix}
1\\
0\\
1
\end{bmatrix}$,
$\vec{v}^{(3)} =
\begin{bmatrix}
0\\
1\\
1
\end{bmatrix}$,
find the constants $a$, $b$, $c$ such that their linear combination $a\vec{v}^{(1)} + b\vec{v}^{(2)} + c\vec{v}^{(3)}$ equals to 
\begin{enumerate}[label=(\alph*)]
\item $(3,2,9)^T$, 
\item $(9,1,5)^T$.
\end{enumerate}
\end{Exercise}

\begin{Exercise}
Determine if the following sets of vectors are linearly independent.
\begin{enumerate}[label=(\alph*)]
\item $\vec{u} = (2,-1)^T$, $\vec{v} = (-4,2)^T$,
\item $\vec{u} = (1,2,3)^T$, $\vec{v} = (6,7,9)^T$, $\vec{w} = (4,8,5)^T$, and
\item $\vec{u} = (1,3,3)^T$, $\vec{v}=(3,2,9)^T$, $\vec{w} = (1,-4,3)^T$.
\end{enumerate}
\end{Exercise}

\begin{Exercise}
Given a spanning set $\mathcal{\gamma} = \{\vec{v}^{(1)}, \vec{v}^{(2)}, \vec{v}^{(3)}, \vec{v}^{(4)}\}$ in which $\vec{v}^{(1)} = (1,3,0,1)^T$, $\vec{v}^{(2)} = (1,-1,2,-1)^T$, $\vec{v}^{(3)} = (-1,2,1,2)^T$, $\vec{v}^{(4)} = (3,0,1,-2)^T$, determine if 
\begin{enumerate}[label=(\alph*)]
\item $(1,4,3,2)^T$, and
\item $(1,2,-3,1)^T$.
\end{enumerate}
are in the subspace generated by $\mathcal{\gamma}$.
\end{Exercise}

\begin{Exercise}
For the basis $\mathcal{\beta} = \{\vec{v}^{(1)}, \vec{v}^{(2)}, \vec{v}^{(3)}\}$, where $\vec{v}^{(1)} = (6,1,2)^T$,
$\vec{v}^{(2)} = (1,0,1)^T$,
$\vec{v}^{(3)} = (2,3,3)^T$
(relative to the standard basis $\mathcal{S}$), do the following coordinate conversion.
\begin{enumerate}[label=(\alph*)]
\item Find the coordinates of $(5, 2, 3)^T$ in the $\mathcal{\beta}$ frame,
\item Transform $(1, -1, 1)^T_B$ from the $\mathcal{\beta}$ system back to the the standard basis $\mathcal{S}$.
\end{enumerate}
\end{Exercise}

\begin{Exercise}
Show that $\mathcal{\beta} = \{\vec{w}^{(1)}, \vec{w}^{(2)}, \vec{w}^{(3)}\}$, where $\vec{w}^{(1)} = (1,-1,0,1)^T$,
$\vec{w}^{(2)} = (2,1,1,0)^T$,
$\vec{w}^{(3)} = (1,2,-1,1)^T$ forms a basis for the subspace generated by itself, and find the coordinates of $\vec{v} = (1.-1,-2,3)^T$ with respect to this basis.
\end{Exercise}

\begin{Exercise}
Prove that for any two subspaces $\mathcal{W}_1, \mathcal{W}_2 \subseteq \mathcal{V}$. Their intersection $\mathcal{W}_1 \cap \mathcal{W}_2$ is also a subspace of $\mathcal{V}$. How about their union?
\end{Exercise}

\begin{Exercise}
Show that $\mathcal{W}_1 = \text{span}(\{(1,0,0,1)^T, (0,1,-1,1)^T\})$ and $\mathcal{W}_2 = \text{span}(\{(1,0,1,-1)^T\})$ can be composed to produce a direct sum $\mathcal{W}_1 \oplus \mathcal{W}_2$. Find bases for $\mathcal{W}_1$, $\mathcal{W}_2$ and hence this direct sum. Express $(2,0,1,0)^T$ in the direct sum basis as the combined coordinates of the two smaller subspaces.
\end{Exercise}

\begin{Exercise}
Find (bases for) the column, row, null, and left null space of 
\begin{align*}
A = 
\begin{bmatrix}
1 & 0 & 1 & 1 \\
0 & 1 & -1 & 1 \\
1 & 2 & -1 & 0 \\
1 & 0 & 1 & 0
\end{bmatrix}
\end{align*}
and check if Theorem \ref{thm:ranknullity} and Properties \ref{proper:funsubsortho} hold in this case.
\end{Exercise}
\chapter{More on Coordinate Bases, Linear Transformations}

In this chapter we will go deeper about what actually a matrix represents in the big picture. Matrices by nature is a rule of \textit{linear transformation} (or \textit{linear mapping}) between two vector spaces. We are going to study several special types of linear transformations, which ultimately reveals the relationship between any $n$-dimensional real vector space and the real $n$-space $\mathbb{R}^n$, as an \textit{isomorphism}. We then move to discuss how a change of coordinates works for vectors and matrices, as well as the \textit{Gram-Schmidt} process to make an \textit{orthonormal} basis.

\section{About Linear Transformations}

\subsection{Linear Maps between Vector Spaces}

Consider two vector spaces, we may want to know if vectors in one of the spaces, let's say $\mathcal{U}$, can be associated to or transformed into those in another vector space, $\mathcal{V}$, according to some rules. This is known as a \index{Transformation}\index{Transformation!Mapping}\keywordhl{transformation/mapping} from the vector space $\mathcal{U}$ to $\mathcal{V}$. Of the most concern is the class of \index{Linear Transformation}\index{Linear Transformation!Linear Mapping}\keywordhl{linear transformations/mappings} which obeys the two properties listed below.
\begin{defn}[Linear Transformation/Map]
\label{defn:lintrans}
A linear transformation (or linear map) from a vector space $\mathcal{U}$ to another vector space $\mathcal{V}$ is a mapping: $T: \mathcal{U} \to \mathcal{V}$, such that for all vectors $\vec{u}_1, \vec{u}_2 \in \mathcal{U}$, and any scalar $a$, it satisfies:
\begin{enumerate}
    \item $T(\vec{u}_1 + \vec{u}_2) = T(\vec{u}_1) + T(\vec{u}_2)$ (Additivity), and
    \item $T(a\vec{u}_j) = aT(\vec{u}_j)$ (Homogeneity).
\end{enumerate}
These two properties combined are known as \textit{linearity}. An equivalent condition is $T(a\vec{u}_1 + b\vec{u}_2) = aT(\vec{u}_1) + bT(\vec{u}_2)$, where $b$ is any scalar as well.
\end{defn}
Notice that if $\mathcal{U}$/$\mathcal{V}$ coincides with the real $n$/$m$-space $\mathbb{R}^n$/$\mathbb{R}^m$, and we express any vector $\vec{u}$: $[\vec{u}]_B$ in $\mathcal{U}$ with $n$ coordinates using some basis $\mathcal{B}$ of its (similarly for $\vec{v}$: $ [\vec{v}]_{H}$ in $\mathcal{V}$ having $m$ coordinates from some basis $\mathcal{H}$). Then $T$: $[T]_B^{H} = A$ where $A$ is any $m \times n$ matrix satisfies the requirements of and is a linear transformation from $\mathcal{U}$ to $\mathcal{V}$ according to the rule $T(\vec{u})$: $A[\vec{u}]_B$. (Short Exercise: show this satisfies the conditions outlined in Definition \ref{defn:lintrans}!\footnote{$T(\vec{u}_1+\vec{u}_2)$: $A([\vec{u}_1]_B + [\vec{u}_2]_B) = A[\vec{u}_1]_B + A[\vec{u}_2]_B$: $T(\vec{u}_1)+T(\vec{u}_2)$ and $T(a\vec{u}_1)$: $A(a[\vec{u}_1]_B) = a(A[\vec{u}_1]_B)$: $aT(\vec{u}_1)$}) This implies that all matrices can be considered as some sort of linear mappings (for now, between $\mathbb{R}^n$ and $\mathbb{R}^m$). In fact, the converse, which states that any linear transformation (between finite-dimensional vector spaces) can be represented by a matrix, is also true as well, and will be discussed in the remaining parts of this section.\\
\\
Let's now explicitly fix a basis $\mathcal{B} = \{\vec{u}_1, \vec{u}_2, \ldots, \vec{u}_n\}$ for $\mathcal{U}$ (again, similarly we have $\mathcal{H} = \{\vec{v}_1, \vec{v}_2, \ldots, \vec{v}_m\}$ for $\mathcal{V}$). For each $\vec{u}_j$, denote $\vec{v}^{(j)} = T(\vec{u}_j)$ as the resulting vectors in $\mathcal{V}$ after applying the transformation $T$ over the basis vectors for $\mathcal{U}$. Notice that $[\vec{u}_j]_B = (e_j)_B$ where the $j$-th basis vector of $\mathcal{B}$ is explicitly represented in a numeric tuple form with the $j$-th entry being $1$ and others being $0$ (where the usual hat symbol on $e$ is not present) in the $\mathcal{B}$ system. Due to Definition \ref{defn:coordRn} and Properties \ref{proper:lincombofspan}, $T(\vec{u}_j) = \vec{v}^{(j)}$ can be expressed as a unique linear combination as $\vec{v}^{(j)} = a_1^{(j)}\vec{u}_1 + a_2^{(j)}\vec{u}_2 + \cdots a_m^{(j)}\vec{u}_m = \sum_{i=1}^{m} a_i^{(j)}\vec{u}_i$ of the basis vectors $\vec{v}_i$ from $\mathcal{H}$, i.e.
\begin{align*}
T(\vec{u}_j) = \sum_{i=1}^{m} a_i^{(j)}\vec{v}_i
\end{align*}
The matrix formed by the above coefficients $A = a_i^{(j)}$ is then the desired \index{Linear Transformation!Matrix Representation of Linear Transformation}\keywordhl{matrix representation} of our linear transformation $T$. To see this, compare with what we have taken in the last paragraph, $T(\vec{u})$: $[\vec{u}]_B$. Subsequently,
\begin{align*}
T(\vec{u_j})\text{: } A[\vec{u}_j]_B &= a_i^{(j)}(e_j)_B \\
&=
\begin{bmatrix}
a_1^{(1)} & a_1^{(2)} & \cdots & a_1^{(j)} & \cdots & a_1^{(n)} \\
a_2^{(1)} & a_2^{(2)} & & a_2^{(j)} & & a_2^{(n)} \\
\vdots & & & \vdots & & \vdots \\
a_m^{(1)} & a_m^{(2)} & \cdots & a_m^{(j)} & \cdots & a_m^{(n)}
\end{bmatrix}
\begin{bmatrix}
0 \\
0 \\
\vdots \\
1 {\scriptsize \text{ (the $j$-th entry)}} \\
\vdots \\
0 {\scriptsize \text{ (the last index is $n$)}}
\end{bmatrix} \\
&=
\begin{bmatrix}
a_1^{(j)} \\
a_2^{(j)} \\
\vdots \\
a_m^{(j)}
\end{bmatrix}
\end{align*}
Due to the structure of $(e_j)_B$, this matrix product yields exactly the $j$-th column of $A = a_i^{(j)}$ as shown above (see Properties \ref{proper:linearcombmatrix}). Moreover, the coordinates of $\vec{v}^{(j)}$ in the $\mathcal{H}$ system
\begin{align*}
[\vec{v}^{(j)}]_H = [\sum_{i=1}^{m} a_i^{(j)}\vec{v}_i]_H &= \sum_{i=1}^{m} a_i^{(j)}[\vec{v}_i]_H \\
&= \sum_{i=1}^{m} a_i^{(j)}(e_i)_H \\
&=  a_1^{(j)} \begin{bmatrix}
1 \\
0 \\
\vdots \\
0
\end{bmatrix}
+
a_2^{(j)} \begin{bmatrix}
0 \\
1 \\
\vdots \\
0
\end{bmatrix}
+ \cdots
+
a_m^{(j)} \begin{bmatrix}
0 \\
0 \\
\vdots \\
1 {\scriptsize \text{ (the last index is $m$)}}
\end{bmatrix}
\\
&= \begin{bmatrix}
a_1^{(j)} \\
0 \\
\vdots \\
0
\end{bmatrix}
+
\begin{bmatrix}
0 \\
a_2^{(j)} \\
\vdots \\
0
\end{bmatrix}
+ \cdots
+
\begin{bmatrix}
0 \\
0 \\
\vdots \\
a_m^{(j)}
\end{bmatrix}
=
\begin{bmatrix}
a_1^{(j)} \\
a_2^{(j)} \\
\vdots \\
a_m^{(j)}
\end{bmatrix}
\end{align*}
also gives the same $j$-th column of $A = a_i^{(j)}$. This holds for any $j$. Hence, the association of the matrix $[A]_B^H = a_i^{(j)}$ to the linear transformation $T$ is consistent, where we have now added the subscript $B$ and superscript $H$ to emphasize the transformation are carried out in reference to the two specific coordinate bases. This reasoning also shows that, to construct the matrix representation of a linear transformation, we compute each of the $T(\vec{u}_j) = \vec{v}^{(j)}$ and find its coordinates in the $\mathcal{H}$ frame, namely $[\vec{v}^{(j)}]_H$, which readily become the $j$-th column of the matrix to be found. To be more clear, we have
\begin{align*}
[T]_B^H &= 
\begin{bmatrix}
[T(\vec{u}_1)]_H | [T(\vec{u}_2)]_H | \cdots | [T(\vec{u}_n)]_H
\end{bmatrix} \\
&=
\begin{bmatrix}
[\vec{v}^{(1)}]_H | [\vec{v}^{(2)}]_H | \cdots | [\vec{v}^{(n)}]_H
\end{bmatrix} \\
&= 
\begin{bmatrix}
a_1^{(1)} & a_1^{(2)} & \cdots & a_1^{(n)} \\
a_2^{(1)} & a_2^{(2)} &  & a_2^{(n)} \\
\vdots & & \ddots & \vdots \\
a_m^{(1)} & a_m^{(2)} & \cdots & a_m^{(n)}
\end{bmatrix}
= a_i^{(j)} = [A]_B^H
\end{align*}
Notice that here the $i$/$j$ subscript/superscript has been exchanged when compared to like Properties \ref{proper:linsysmat}.
\begin{defn}[Matrix Representation of a Linear Transformation]
\label{defn:matrixrepoflintrans}
A linear transformation $T: \mathcal{U} \to \mathcal{V}$ as defined in Definition \ref{defn:lintrans}, with respect to the bases $\mathcal{B} = \{\vec{u}_1, \vec{u}_2, \ldots, \vec{u}_n\}$ for $\mathcal{U}$ and $\mathcal{H} = \{\vec{v}_1, \vec{v}_2, \ldots, \vec{v}_m\}$ for $\mathcal{V}$, has a matrix representation of
\begin{align*}
[T]_B^H = 
\begin{bmatrix}
a_1^{(1)} & a_1^{(2)} & \cdots & a_1^{(n)} \\
a_2^{(1)} & a_2^{(2)} &  & a_2^{(n)} \\
\vdots & & \ddots & \vdots \\
a_m^{(1)} & a_m^{(2)} & \cdots & a_m^{(n)}
\end{bmatrix}
\end{align*}
where the entries $a_i^{(j)}$ are those according to the relations $T(\vec{u}_j) = \sum_{i=1}^{m} a_i^{(j)}\vec{v}_i$, or in matrix notation, $[T]_B^H [\vec{u}]_B = [\vec{v}]_H$.
\end{defn}

Let's illustrate how it works out using an easy example using the familiar $\mathbb{R}^n$ and $\mathbb{R}^m$. 
\begin{exmp}
\label{exmp:lineartransmatrixrep}
Let $\mathcal{U} = \mathbb{R}^3$ and $\mathcal{V} = \mathbb{R}^2$, it can be easily verified that $\mathcal{B} = \{(1,2,1)^T, (0,1,-1)^T, (2,-1,1)^T\}$ is a basis for $\mathcal{U}$, and the same goes for $\mathcal{V}$ with a basis $\mathcal{H} = \{(1,2)^T, (2,-1)^T\}$. If a linear transformation $T: \mathbb{R}^3 \to \mathbb{R}^2$ obeys the rule $T((x,y,z)^T) = (x+2y, x-y+z)^T$, (By the way, you should verify if this is really a linear transformation.) find its matrix representation $[T]_B^H$ with respect to $\mathcal{B}$ and $\mathcal{H}$. Then, use the results to compute $T((-1,4,-1)^T)$.
\end{exmp}
\begin{solution}
Following Definition \ref{defn:matrixrepoflintrans}, we set out to find how the linear transformation will apply on the basis vectors in $\mathcal{B}$. For the first one, we have
\begin{align*}
T((1,2,1)^T) = ((1)+2(2), (1)-(2)+(1))^T = (5,0)^T
\end{align*}
which can be subsequently written as a linear combination of the two basis vectors in $\mathcal{H}$:
\begin{align*}
(5,0)^T = 1(1,2)^T + 2(2,-1)^T
\end{align*}
Hence $a_1^{(1)} = 1$, $a_2^{(1)} = 2$, and this gives us the first column of $[T]_B^H$ as
\begin{align*}
\begin{bmatrix}
1 & * & * \\
2 & * & * 
\end{bmatrix}
\end{align*}
We repeat the same procedure on the other two basis vectors $(0,1,-1)^T$ and $(2,-1,0)^T$ of $\mathcal{B}$, where it can be shown that
\begin{align*}
T((0,1,-1)^T) &= ((0)+2(1), (0)-(1)+(-1))^T = (2,-2)^T \\
&= -\frac{2}{5}(1,2)^T + \frac{6}{5}(2,-1)^T \\
T((2,-1,1)^T) &= ((2)+2(-1), (2)-(-1)+(1))^T = (0,4)^T \\
&= \frac{8}{5}(1,2)^T - \frac{4}{5}(2,-1)^T
\end{align*}
Therefore, the required matrix representation is
\begin{align*}
[T]_B^H = 
\begin{bmatrix}
1 & -\frac{2}{5} & \frac{8}{5} \\
2 & \frac{6}{5} & -\frac{4}{5}
\end{bmatrix}
\end{align*}
For the second part, we start by expressing $(-1,4,1)^T$ in the basis $\mathcal{B}$. As $(-1,4,-1)^T = 1(1,2,1)^T + 1(0,1,-1)^T - 1(2,-1,1)^T$, we have $(-1,4,1)^T$: $(1,1,-1)_B^T$, and then
\begin{align*}
[T((1,1,-1)_B^T)]_H &= [T]_B^H (1,1,-1)_B^T \\
&=
\left(\begin{bmatrix}
1 & -\frac{2}{5} & \frac{8}{5} \\
2 & \frac{6}{5} & -\frac{4}{5}
\end{bmatrix}
\begin{bmatrix}
1 \\
1 \\
-1
\end{bmatrix}\right)_H \\
&=
\begin{bmatrix}
-1 \\
4
\end{bmatrix}_H = (-1,4)^T_H
\end{align*}
implying that $T((-1,4,-1)^T) = -1(1,2)^T + 4(2,-1)^T = (7,-6)^T$ in the usual standard basis. This can be cross-checked by directly invoking the given definition of $T$, where $T((-1,4,-1)^T) = ((-1)+2(4), (-1)-(4)+(-1))^T = (7,-6)^T$ as well.
\end{solution}

Up until now, we have been playing around with the simple real $n$-space only, but the real (no pun intended) power of the notion of a general vector space lies in its abstraction: Any mathematical object that satisfies the criteria in Definition \ref{defn:realvecspaceaxiom} is a (real) vector space, and the results that we have already established in the previous parts for the real $n$-space are readily transferable to them. Two prime examples of abstract vector spaces are the set of (real) polynomials $\mathcal{P}^n$ with a degree up to $n$ and\footnote{We shall argue for some criteria in Definition \ref{defn:realvecspaceaxiom} for $\mathcal{P}^n$ here. For instances, condition (1) is obvious as adding up two polynomials with a degree up to $n$ can only result in another polynomial with a maximum degree of $n$. In condition (4), the zero vector for $\mathcal{P}^n$ is simply the constant zero function $0$, which is considered to have a degree of $-1$ by convention.} the family of continuous ($k$-times continuously differentiable) functions $\mathcal{C}^0$ ($\mathcal{C}^k$) over a fixed interval. Now we will see how the concept of linear transformation is laid out when these abstract vector spaces are involved, preparing us for the key insight in the next subsection.

\begin{exmp}
\label{exmp:lineartransderivative}
Consider $\mathcal{U} = \mathcal{P}^2$, and $\mathcal{V} = \mathcal{P}^1$, and let the bases for $\mathcal{U}$ and $\mathcal{V}$ be $\mathcal{B} = \{1, x, x^2\}$ and $\mathcal{H} = \{1, x\}$. (They are known as the standard bases for $\mathcal{P}^2$ and $\mathcal{P}^1$ respectively. In general the standard basis for $\mathcal{P}^n$ is $\{1, x, x^2, \cdots, x^{n-1}, x^n\}$ and thus $n+1$-dimensional. Readers are advised to justify why they constitute a basis for the polynomial spaces.) Let $T: \mathcal{U} \to \mathcal{V}$ be $T[p(x)] = p'(x)$ the differentiation operator and find its matrix representation with respect to $\mathcal{B}$ and $\mathcal{H}$.
\end{exmp}
\begin{solution}
We essentially do the same thing as in Example \ref{exmp:lineartransmatrixrep} but applied over polynomials now. From elementary calculus, we have
\begin{align*}
T(1) &= \frac{d}{dx}(1) = 0 \\
T(x) &= \frac{d}{dx}(x) = 1 \\
T(x^2) &= \frac{d}{dx}(x^2) = 2x
\end{align*}
and by Definition \ref{defn:matrixrepoflintrans}, the desired matrix representation is
\begin{align*}
[T]_B^H = 
\begin{bmatrix}
0 & 1 & 0 \\
0 & 0 & 2
\end{bmatrix}
\end{align*}
Notice that we can express, quite trivially
\begin{align*}
1 &= \begin{bmatrix}
1 \\
0 \\
0
\end{bmatrix}_B
=
\begin{bmatrix}
1 \\
0
\end{bmatrix}_H \\
x &= \begin{bmatrix}
0 \\
1 \\
0
\end{bmatrix}_B
=
\begin{bmatrix}
0 \\
1
\end{bmatrix}_H \\
x^2 &= \begin{bmatrix}
0 \\
0 \\
1
\end{bmatrix}_B
\end{align*}
using vector notation in the given two standard bases. We can verify the form of $[T]_B^H$ by a test polynomial $c_0 + c_1x + c_2x^2$, whose vector representation in $\mathcal{B}$ is clearly $(c_0, c_1, c_2)^T_B$. Then, multiplying $[T]_B^H$ to its left gives
\begin{align*}
[T((c_0, c_1, c_2)^T_B)]_H = 
\begin{bmatrix}
0 & 1 & 0 \\
0 & 0 & 2
\end{bmatrix}_B^H 
\begin{bmatrix}
c_0 \\
c_1 \\
c_2
\end{bmatrix}_B
=
\begin{bmatrix}
c_1 \\
2c_2
\end{bmatrix}_H
\end{align*}
which corresponds to the polynomial $c_1 + 2c_2x$. This coincides with the usual result of differentiation, that is, $\frac{d}{dx}(c_0 + c_1x + c_2x^2) = c_1 + 2c_2x$.
\end{solution}

In each of the previous examples, we consider a linear transformation between two vector spaces that are of the same type (the usual real vectors/polynomials). Below shows what happen when they are mixed together. Actually, due to the abstraction provided by the nature of vector space, the outcome follows easily.
\begin{exmp}
Let $\mathcal{U} = \mathbb{R}^3$ and $\mathcal{V} = \mathcal{P}^2$, while $\mathcal{B} = \{(1,0,0)^T, (0,1,0)^T, (0,0,1)^T\}$ and $\mathcal{H} = \{1, x, x^2\}$ be the standard bases for $\mathcal{U}$ and $\mathcal{V}$ respectively. Show that, the rather trivial linear transformation $T((c_0, c_1, c_2)^T) = c_0 + c_1x + c_2x^2$ has a matrix representation of an identity with respect to $\mathcal{B}$ and $\mathcal{H}$.
\end{exmp}
\begin{solution}
Again, we repeat what we have done in the previous two examples. It is apparent that
\begin{align*}
T((1,0,0)^T_B) &= (1) + (0)x + (0)x^2 = 1 = (1,0,0)^T_H\\
T((0,1,0)^T_B) &= (0) + (1)x + (0)x^2 = x = (0,1,0)^T_H\\
T((0,0,1)^T_B) &= (0) + (0)x + (1)x^2 = x^2 = (0,0,1)^T_H
\end{align*}
So by Definition \ref{defn:matrixrepoflintrans}, the desired matrix representation is simply
\begin{align*}
[T]_B^H = 
\begin{bmatrix}
1 & 0 & 0 \\
0 & 1 & 0 \\
0 & 0 & 1
\end{bmatrix}
\end{align*}
which is the $3 \times 3$ identity matrix. This is expected as the linear transformation is essentially $T[(c_0, c_1, c_2)_B^T] = (c_0, c_1, c_2)_H^T$ where $(c_0, c_1, c_2)_H^T = c_0 + c_1x + c_2x^2$, which means that the numeric representation of vectors in the two spaces is preserved under such a linear transformation between them and the only visible change is the subscript.
\end{solution}
Most of the readers should find it boring in the above example as we are just stating the obvious. It is a straight-forward, "one-to-one" association between the standard bases of the real $n$-space and space of polynomials with degree $n-1$. However, the important message is that given such an association we can always identify any vector of some space as a vector in another space of a completely different class, which is very powerful as many operations become transferable between these two spaces. In this sense, this kind of "one-to-one" mapping is not limited to the identity mapping, or by the bases used for the two vector spaces as we will see in the following subsection.

\subsection{One-to-one and Onto, Kernel and Range}

Continuing our discussion above, to identify a vector (one and only one) from one vector space as another vector in another vector space through a linear mapping, we require it to be \index{One-to-one}\index{Injective}\keywordhl{one-to-one (injective)}. On the other hand, another important property of a linear transformation is that whether it is \index{Onto}\index{Surjective}\keywordhl{onto (surjective)}, which means that every vector in the latter vector space (\textit{image}) is being mapped onto by some vector(s) in the former vector space (\textit{preimage}). The formal definitions of these two properties are given as below.

\begin{proper}[Injective Transformation]
\label{proper:injective}
A transformation $T: \mathcal{U} \to \mathcal{V}$ is called one-to-one if for two vectors $\vec{u}_1, \vec{u}_2 \in \mathcal{U}$, $T(\vec{u}_1) = T(\vec{vu}_2)$ implies $\vec{u}_1 = \vec{u}_2$, i.e. an image has one and only one corresponding preimage. Furthermore, if $T$ is linear, then equivalently $T(\vec{u}) = \textbf{0}$ implies $\vec{u} = \textbf{0}$ only.
\end{proper}
To show the equivalence of the two conditions above, notice that $T(\textbf{0}) = \textbf{0}$ if $T$ is linear. (why?)\footnote{$T(\textbf{0})=T(0\vec{u})=0T(\vec{u})=\textbf{0}$ for arbitrary $\vec{v}$ due to the homogeneity property as required in Definition \ref{defn:lintrans}.} For any $\vec{v}$ such that $T(\vec{v}) = 0$, we have
\begin{align*}
T(\vec{v}) = \textbf{0} = T(\textbf{0})
\end{align*}
and hence $\vec{v}$ must be $\textbf{0}$ if $T(\vec{v}_1) = T(\vec{v_2})$ implies $\vec{v}_1 = \vec{v}_2$. The proof of the converse is left as an exercise.
\begin{proper}[Surjective Transformation]
\label{proper:surjective}
A transformation $T: \mathcal{U} \to \mathcal{V}$ is called onto if for any vector $\vec{v} \in \mathcal{V}$ (image), there exists at least one vector(s) $\vec{u} \in \mathcal{U}$ (preimage) such that $T(\vec{u}) = \vec{v}$.
\end{proper}
As an illustration, in Example \ref{exmp:lineartransderivative}, the differentiation operator $T(p(x)) = p'(x)$ from $\mathcal{P}^2$ to $\mathcal{P}^1$ is onto but not one-to-one. To see these, note that given any image $\vec{v} = d_0 + d_1x \in \mathcal{P}^1$, all preimages in the form of $\vec{u} = K + d_0x + \frac{d_1}{2}x^2\in \mathcal{P}^2$ where $K$ can be any number satisfies $T(\vec{u}) = \vec{v}$ by elementary calculus, and the surjectivity is obvious. To explicitly disprove injectivity, fix an image $\vec{v} = d_0 + d_1x$ with specific $d_0$ and $d_1$, and note that both $\vec{u}_1 = K_1 + d_0x + \frac{d_1}{2}x^2$ and $\vec{u}_2 = K_2 + d_0x + \frac{d_1}{2}x^2$ where $K_1, K_2$ are distinct satisfy $T(\vec{u}_1) = T(\vec{u}_2) = \vec{v}$, but $\vec{u}_1 \neq \vec{u}_2$.

However, in other cases it may not be so easy to check injectivity and surjectivity as directly as above. Therefore, we need a general method to determine if these two properties hold for a transformation between two abstract vector bases. The following theorem links injectivity and surjectivity with their basis vectors, but it requires the transformation to be linear (and here is where the linearity comes to play).

\begin{thm}
\label{thm:oneto_onebasis}
A linear transformation $T: \mathcal{U} \to \mathcal{V}$ between two finite-dimensional vector spaces is one-to-one if and only if given any basis $\mathcal{B} = \{\vec{u}_1, \vec{u}_2, \ldots, \vec{u}_n\}$ for $\mathcal{U}$, $T(\vec{u}_1), T(\vec{u}_2), \ldots, T(\vec{u}_n) \in \mathcal{V}$ are linearly independent.
\end{thm}
\begin{thm}
\label{thm:onto_basis}
A linear transformation $T: \mathcal{U} \to \mathcal{V}$ between two finite-dimensional vector spaces is onto if and only if given any basis $\mathcal{H} = \{\vec{v}_1, \vec{v}_2, \ldots, \vec{v}_m\}$ for $\mathcal{V}$, we can find a vector $\vec{u}_i \in \mathcal{U}$ such that $T(\vec{u}_i) = \vec{v}_i$ for each of the $\vec{v}_i$.
\end{thm}
\begin{proof}
Theorem \ref{thm:oneto_onebasis}: The "if" direction is proved by showing $T(\vec{u}_1), T(\vec{u}_2), \ldots, T(\vec{u}_n)$ are linearly independent implies that, if $T(\vec{u}) = \textbf{0}$ then $\vec{u} = \textbf{0}$ as suggested by the alternative condition in Properties \ref{proper:injective}. By Theorem \ref{thm:linearindep}, the equation $c_1T(\vec{u}_1) + c_2T(\vec{u}_2) + \cdots + c_nT(\vec{u}_n) = \textbf{0}$ only has $c_j = \textbf{0}$ as the trivial solution. Now by linearity from Definition \ref{defn:lintrans}, we have
\begin{align*}
c_1T(\vec{u}_1) + c_2T(\vec{u}_2) + \cdots + c_nT(\vec{u}_n) &= T(c_1\vec{u}_1 + c_2\vec{u}_2 + \cdots + c_n\vec{u}_n) \\
&= \textbf{0}    
\end{align*}
Since $c_j = 0$ is the only possibility, this means that if $T(c_1\vec{u}_1 + c_2\vec{u}_2 + \cdots + c_n\vec{u}_n) = \textbf{0}$ then $\vec{u} = c_1\vec{u}_1 + c_2\vec{u}_2 + \cdots + c_n\vec{u}_n$ must be $\textbf{0}$, hence $T(\vec{u}) = \textbf{0}$ implies $\vec{u} = 0$ and we are done. The converse is similarly proved,  having the argument goes in reverse direction. \\
Theorem \ref{thm:onto_basis}: We compare Theorem \ref{thm:onto_basis} against Properties \ref{proper:surjective} to show the part of "if" direction. Since $H = \{\vec{v}_i\}$ is a basis for $\mathcal{V}$, any $\vec{v} \in \mathcal{V}$ can be written as a linear combination of $\vec{v} = c_1\vec{v}_1 + c_2\vec{v}_2 + \cdots c_m\vec{v}_m$. If we can find $\vec{u_i} \in \mathcal{U}$ such that $T(\vec{u}_i) = \vec{v}_i$ for all $\vec{v}_i$, then
\begin{align*}
\vec{v} &= c_1\vec{v}_1 + c_2\vec{v}_2 + \cdots c_m\vec{v}_m \\
&= c_1T(\vec{u}_1) + c_2T(\vec{u}_2) + \cdots + c_mT(\vec{u}_m) \\
&= T(c_1\vec{u}_1 + c_2\vec{u}_2 + \cdots + c_m\vec{u}_m)
\end{align*}
the last equality uses linearity from Definition \ref{defn:lintrans} again. This shows that $\vec{u} = c_1\vec{u}_1 + c_2\vec{u}_2 + \cdots + c_m\vec{u}_m$ is readily one possible vector in $\mathcal{U}$ such that $T(\vec{u}) = \vec{v}$ and the desired result is established. The converse is trivial as we take $\vec{v} = \vec{v}_i$ in Properties \ref{proper:surjective} for all possible $i$.
\end{proof}

\begin{exmp}
Given a linear transformation $T: \mathcal{U} \to \mathcal{V}$ where $\mathcal{U}$ and $\mathcal{V}$ have a dimension of $3$ and $4$ respectively, if its matrix representation corresponding to some bases $\mathcal{B}$ and $\mathcal{H}$ is
\begin{align*}
[T]_B^H =
\begin{bmatrix}
1 & -1 & 0 \\
0 & 1 & 1 \\
1 & 0 & -1 \\
1 & 1 & 0
\end{bmatrix}
\end{align*}
determine whether it is (a) one-to-one, as well as (b) onto, or not.
\end{exmp}
\begin{solution}
\begin{enumerate}[label=(\alph*)]
    \item By Theorem \ref{thm:oneto_onebasis}, we need to check if $T(\vec{u}_1), T(\vec{u}_2), T(\vec{u}_3)$ are linearly independent, where $\vec{u}_1$, $\vec{u}_2, \vec{u}_3$ are the basis vectors from $\mathcal{B}$. Their numeric representation in the $\mathcal{B}$ system is trivially $[\vec{u}_1]_B = (e_1)_B = (1,0,0)_B^T, [\vec{u}_2]_B = (e_2)_B = (0,1,0)_B^T$ and $[\vec{u}_3]_B = (e_3)_B = (0,0,1)_B^T$, and hence
    \begin{align*}
    [T(\vec{u}_1)]_H &= [T]_B^H(e_1)_B \\
    &= 
    \left(\begin{bmatrix}
    1 & -1 & 0 \\
    0 & 1 & 1 \\
    1 & 0 & -1 \\
    1 & 1 & 0
    \end{bmatrix}
    \begin{bmatrix}
    1 \\
    0 \\
    0
    \end{bmatrix}\right)_H = 
    \begin{bmatrix}
    1 \\
    0 \\
    1 \\
    1
    \end{bmatrix}_H
    \end{align*}
    which is just the first column of $[T]_B^H$. Similarly, $[T(\vec{u}_2)]_H = (-1,1,0,1)_H^T$, $[T(\vec{u}_3)]_H = (0,1,-1,0)_H^T$ are then the second/third column of $[T]_B^H$. From this we see that in general, the coordinates in $\mathcal{H}$ after transformation $[T(\vec{u}_j)]_H$ is just the $j$-th column of $[T]_B^H$. (Actually, this has been observed when we are deriving the matrix representation of linear transformations in the beginning of this chapter.) So the problem is reduced to decide whether the column vectors constituting $[T]_B^H$ are linearly independent or not. By Theorem \ref{thm:linearindep}, we can accomplish this by showing if the solution $[T]_B^H\vec{x} = \textbf{0}$ is consisted of the trivial solution only, and we have
    \begin{align*}
    \left[
    \begin{array}{@{\;}wc{10pt}wc{10pt}wc{10pt}|wc{10pt}@{\;}}
    1 & -1 & 0 & 0 \\
    0 & 1 & 1 & 0\\
    1 & 0 & -1 & 0\\
    1 & 1 & 0 & 0
    \end{array}
    \right] &\to 
    \left[\begin{array}{@{\;}wc{10pt}wc{10pt}wc{10pt}|wc{10pt}@{\;}}
    1 & -1 & 0 & 0 \\
    0 & 1 & 1 & 0\\
    0 & 1 & -1 & 0\\
    0 & 2 & 0 & 0
    \end{array}\right] & 
    \begin{aligned}
    R_3 - R_1 &\to R_3 \\
    R_4 - R_1 &\to R_4 \\
    \end{aligned}\\
    &\to 
    \left[\begin{array}{@{\;}wc{10pt}wc{10pt}wc{10pt}|wc{10pt}@{\;}}
    1 & -1 & 0 & 0 \\
    0 & 1 & 1 & 0\\
    0 & 0 & -2 & 0\\
    0 & 0 & -2 & 0
    \end{array}\right] & 
    \begin{aligned}
    R_3 - R_2 &\to R_3 \\
    R_4 - R_2 &\to R_4 \\
    \end{aligned}\\
    &\to 
    \left[\begin{array}{@{\;}wc{10pt}wc{10pt}wc{10pt}|wc{10pt}@{\;}}
    1 & -1 & 0 & 0 \\
    0 & 1 & 1 & 0\\
    0 & 0 & 1 & 0\\
    0 & 0 & -2 & 0
    \end{array}\right] & 
    -\frac{1}{2}R_3 \to R_3 \\
    &\to 
    \left[\begin{array}{@{\;}wc{10pt}wc{10pt}wc{10pt}|wc{10pt}@{\;}}
    1 & -1 & 0 & 0 \\
    0 & 1 & 1 & 0\\
    0 & 0 & 1 & 0\\
    0 & 0 & 0 & 0
    \end{array}\right] & 
    R_4 + 2R_3 \to R_4
    \end{align*}
    As every column in this homogeneous system contains a pivot, it demonstrates that $[T]_B^H\vec{x} = \textbf{0}$ indeed only has the trivial solution $\vec{x} = \textbf{0}$, and therefore the linear transformation in question is one-to-one.
    \item By Properties \ref{proper:surjective}, it is equivalent to showing that if the $\{T(\vec{u}_j)\}$s span $\mathcal{W}$, or expressed in terms of the $\mathcal{B}$/$\mathcal{H}$ coordinates, whether the three transformed vectors $\{[T(\vec{u}_1)]_H, [T(\vec{u}_2)]_H, [T(\vec{u}_3)]_H\}$ span $\mathbb{R}^4$. However, it is apparent that three vectors can never span a four-dimensional vector space as the number of vectors is fewer then the dimension, and thus the linear transformation is not onto.
\end{enumerate}
Notice that in the above arguments we never explicitly say what the vector spaces $\mathcal{U}$ and $\mathcal{V}$ are and only the matrix representation of the linear transformation is involved. However, some may be skeptical as we have fixed bases for the linear transformation and may ask if the results are basis-dependent. We will address this issue in later parts of this chapter.
\end{solution}
Accompanying injectivity and surjectivity is the ideas of \index{Kernel}\keywordhl{kernel} and \index{Range}\keywordhl{range}. For a (linear) transformation $T: \mathcal{U} \to \mathcal{V}$, its kernel is consisted of vectors in $\mathcal{U}$ that is mapped to the zero vector in $\mathcal{V}$, while its range is made up of all possible vectors in $\mathcal{V}$ that are mapped from $\mathcal{U}$ via $T$.
\begin{defn}
\label{defn:kernelrange}
For a (linear) transformation $T: \mathcal{U} \to \mathcal{V}$, its kernel is defined to be
\begin{align*}
\text{Ker}(T) = \{\vec{u} \in \mathcal{U} | T(\vec{u}) = \textbf{0}_V\}
\end{align*}
whereas its range is
\begin{align*}
R(T) = \{\vec{v} \in \mathcal{V} | T(\vec{u}) = \vec{v} \text{ for some } \vec{u} \in \mathcal{U}\}    
\end{align*}
\end{defn}
Also, notice that the kernel and range are a subspace of $\mathcal{U}$ and $\mathcal{V}$ respectively.\footnote{For $\vec{u}_1, \vec{u}_2 \in \text{Ker}(T) \subset \mathcal{U}$, $T(a\vec{u}_1 + b\vec{u}_2) = aT(\vec{u}_1) + bT(\vec{u}_2) = a\textbf{0}_V + b\textbf{0}_V = \textbf{0}_V$ for any scalar $a$ and $b$ so $a\vec{u}_1 + b\vec{u}_2 \in \text{Ker}(T)$ and by Theorem \ref{thm:subspacecriteria} it is a subspace of $\mathcal{U}$. We leave showing the range is a subspace of $\mathcal{V}$ as an exercise to the readers.} Hence it is reasonable to speak of their dimension or basis and we will discuss this matter later. For now, let's look at how to determine the kernel and range of a linear transformation first. For instance, in Example \ref{exmp:lineartransderivative}, the kernel is $\text{span}(\{1\})$ since the derivative of any constant vanishes, and the range is $\text{span}(\{1, x\}) = \mathcal{V} = \mathcal{P}^1$ because we have already shown that every $\mathcal{P}^1$ polynomial in this case have some corresponding preimage in $\mathcal{U} = \mathcal{P}^2$. Here the dimension of kernel/range is $1$ and $2$.

\begin{exmp}
Given another linear transformation $T: \mathcal{U} \to \mathcal{V}$ where $\mathcal{U}$ and $\mathcal{V}$ are now both having a dimension of $3$, if its matrix representation corresponding to some bases $\mathcal{B}$ and $\mathcal{H}$ is
\begin{align*}
[T]_B^H =
\begin{bmatrix}
1 & 0 & 1 \\
1 & -1 & 1 \\
1 & 1 & 1 
\end{bmatrix}
\end{align*}
find its kernel and range.
\end{exmp}
\begin{solution}
According to Definition \ref{defn:kernelrange}, $\text{Ker}(T)$ is the set of $\vec{u}$ that satisfies $T(\vec{u}) = \textbf{0}$, or using basis representation, $[T]_B^H[\vec{u}]_B = \textbf{0}$. Therefore, it is equivalent to finding the null space of $[T]_B^H$:
\begin{align*}
\left[
\begin{array}{@{\;}wc{10pt}wc{10pt}wc{10pt}|wc{10pt}@{\;}}
1 & 0 & 1 & 0 \\
1 & -1 & 1 & 0 \\
1 & 1 & 1 & 0
\end{array}
\right] &\to
\left[
\begin{array}{@{\;}wc{10pt}wc{10pt}wc{10pt}|wc{10pt}@{\;}}
1 & 0 & 1 & 0 \\
0 & -1 & 0 & 0 \\
0 & 1 & 0 & 0
\end{array}
\right] &
\begin{aligned}
R_2 - R_1 &\to R_2 \\
R_3 - R_1 &\to R_3 \\
\end{aligned} \\
&\to
\left[
\begin{array}{@{\;}wc{10pt}wc{10pt}wc{10pt}|wc{10pt}@{\;}}
1 & 0 & 1 & 0 \\
0 & 1 & 0 & 0 \\
0 & -1 & 0 & 0
\end{array}
\right]
& R_2 \leftrightarrow R_3 \\
&\to
\left[
\begin{array}{@{\;}wc{10pt}wc{10pt}wc{10pt}|wc{10pt}@{\;}}
1 & 0 & 1 & 0 \\
0 & 1 & 0 & 0 \\
0 & 0 & 0 & 0
\end{array}
\right]
& R_3 + R_2 \to R_3
\end{align*}
The nullity is $1$ and we can let $[u_3]_B = t$ be the free variable, and we have $[u_1]_B = -t$ and $[u_2]_B = 0$ from the first two rows. So the kernel takes the form of
\begin{align*}
\text{Ker}(T) = 
\begin{bmatrix}
-t \\
0 \\
t
\end{bmatrix}_B
= t
\begin{bmatrix}
-1 \\
0 \\
1
\end{bmatrix}_B
\end{align*}
where $-\infty < t < \infty$, or in other words, $\text{Ker}(T) = \text{span}(\{(-1,0,1)_B^T\})$ with a dimension of $1$. Similarly, the range of $T$ will be the column space of $[T]_B^H$. From the elimination procedure carried out above, we know that the first two column vectors are linearly independent and the third column is clearly the same as the first column, and thus the range is $R(T) = \text{span}(\{(1,1,1)_B^T, (0,-1,1)_B^T\})$ and has a dimension of $2$, which coincides with the rank of the $[T]_B^H$ matrix. Note that we approach the problem with some bases (albeit unknown) fixed to represent the linear transformation in matrix form just like in the last example. Again, we will soon justify that the results are actually unrelated to the choices of bases such that the dimensions of kernel and range are exactly the nullity and rank of any matrix representation of the linear transformation.
\end{solution}

Finally, we can rewrite Properties \ref{proper:injective} and \ref{proper:surjective} using the notion of kernel and range.
\begin{proper}
A linear transformation $T: \mathcal{U} \to \mathcal{V}$ is one-to-one if and only if the dimension of its kernel $\text{Ker}(T)$ is zero, i.e. $\text{Dim}(\text{Ker}(T)) = 0$. Meanwhile, it is onto if and only if the dimension of range $R(T)$ (rank) is same as the dimension of $\mathcal{V}$.
\end{proper}

\subsection{Vector Space Isomorphism to $\mathbb{R}^n$}

A linear transformation where both injectivity and surjectivity hold is known as \index{Bijective}\index{Isomorphic}\keywordhl{bijective/isomorphic}. As we will immediately see, this property is very central in relating finite-dimensional real vector spaces to the real $n$-space. Combining Properties \ref{proper:injective} and \ref{proper:surjective}, for a linear transformation $T: \mathcal{U} \to \mathcal{V}$ to be bijective, every vector $\vec{v} \in \mathcal{V}$ must be an image which there is one and only one preimage $\vec{u} \in \mathcal{U}$ is mapped onto, i.e. there is a unique $\vec{u} \in \mathcal{U}$ that satisfies $T(\vec{u}) = \vec{v}$ for every $\vec{v} \in \mathcal{V}$, which also means that it is \textit{invertible} in the sense that every $\vec{v} \in \mathcal{V}$ can be traced back to one and only one $\vec{u} \in \mathcal{U}$ via the transformation in reverse direction. Hence it makes sense to say a transformation is bijective \textit{between} two vector spaces. There are two major results regarding invertibility. The first one is
\begin{thm}
\label{thm:bijectivechincoord}
There always exists a bijective linear mapping between $\mathcal{V}$ itself, i.e. $T: \mathcal{V} \to \mathcal{V}$, that transforms the coordinates of any fixed vector in $\mathcal{V}$ between two different bases (denote them by $\mathcal{B}$ and $\mathcal{B}'$) of its. Such a change of coordinates in $\mathcal{V}$ has a matrix representation $[T]_B^{B'} = P_B^{B'}$ that is invertible.
\end{thm}
\begin{proof}
Since it is the same vector space $\mathcal{V}$ but just represented in different bases, the number of dimension will stay the same, let's say $n$, and the bases $\mathcal{B}$ and $\mathcal{B}'$ both are made up of $n$ basis vectors (Properties \ref{proper:samenvecsbases}). Denote them by $\mathcal{B} = \{\vec{v}_{1,B}, \vec{v}_{2,B}, \ldots, \vec{v}_{n,B}\}$ and $\mathcal{B'} = \{\vec{v}_{1,B'}, \vec{v}_{2,B'}, \ldots, \vec{v}_{n,B'}\}$. The desired mapping $T: \mathcal{V} \to \mathcal{V}$ is in fact
\begin{align*}
T(\vec{v}) = \text{id}(\vec{v}) = \vec{v}
\end{align*}
the \index{Identity Transformation}\index{Identity Mapping}\keywordhl{identity transformation/identity mapping} as it is just a change of coordinates where the actual vector stays identical. This transformation is then trivially bijective because any vector is just mapped into itself, and is described by $[\vec{v}]_B' = [T]_B^{B'} [\vec{v}]_B$ following Definition \ref{defn:matrixrepoflintrans} with $\mathcal{U} = \mathcal{V}$ and $\vec{u} = \vec{v}$. Now note that $[\vec{v}]_{B'} = [T]_B^{B'} [\vec{v}]_{B'}$ has a unique solution $[\vec{v}]_B$ for any $[\vec{v}]_{B'}$ as $T$ is bijective and by definition each of $[\vec{v}]_{B'}$ is mapped onto by one and only one $[\vec{v}]_B$. Part (d) to (a) of Theorem \ref{thm:equiv2} then shows that $[T]_B^{B'}$ is an invertible matrix. According to the discussion prior to Definition \ref{defn:matrixrepoflintrans}, $[T]_B^{B'}$ takes the form of
\begin{align*}
P_B^{B'} = [T]_B^{B'} &= \begin{bmatrix}
[\text{id}(\vec{v}_{1,B})]_{B'} | [\text{id}(\vec{v}_{2,B})]_{B'} | \cdots | [\text{id}(\vec{v}_{n,B})]_{B'}
\end{bmatrix} \\
&=
\begin{bmatrix}
[\vec{v}_{1,B}]_{B'} | [\vec{v}_{2,B}]_{B'} | \cdots | [\vec{v}_{n,B}]_{B'}
\end{bmatrix}
\end{align*}
So we have to find how each of the basis vectors in $\mathcal{B}$ is expressed in the $\mathcal{B}'$ system. Conversely,
\begin{align*}
P_{B'}^B = ([T]_B^{B'})^{-1} = [T]_{B'}^B =
\begin{bmatrix}
[\vec{v}_{1,B'}]_B | [\vec{v}_{2,B'}]_B | \cdots | [\vec{v}_{n,B'}]_B
\end{bmatrix}
\end{align*}
Be aware that despite it being an identity mapping, the exact matrix representation is dependent on the bases and will usually not be an identity matrix. Nevertheless, such bijectivity between any two coordinate systems of the same vector space implies that all linear transformation from one vector space to another $T: \mathcal{U} \to \mathcal{V}$, together with its (dimensions of) kernel or range, are independent of the choices of bases for either $\mathcal{U}$ or $\mathcal{V}$ and we can pick whatever bases that suit the situation better. The only thing that is dependent on the coordinate systems will be their numeric representation and we will see how it unfolds in the next part. This justify our fixing of bases during several arguments in the last subsection.
\end{proof}

\begin{exmp}
\label{exmp:changecoord}
Show that $\mathcal{B} = \{(1,0,1)^T, (0,2,1)^T, (-1,1,2)^T\}$ and $\mathcal{B}' = \{(0,0,1)^T, (2,0,1)^T, (1,-1,0)^T\}$ are both bases for $\mathcal{V} = \mathbb{R}^3$ and find the matrix representation of coordinate conversion between them.
\end{exmp}
\begin{solution}
Just like in Example \ref{exmp:basisR3}, we need to check whether the determinants of
\begin{align*}
B &= 
\begin{bmatrix}
1 & 0 & -1\\
0 & 2 & 1 \\
1 & 1 & 2
\end{bmatrix}
& \text{and} &
& B' = 
\begin{bmatrix}
0 & 2 & 1 \\
0 & 0 & -1 \\
1 & 1 & 0
\end{bmatrix}
\end{align*}
are non-zero or not. A simple computation shows that $\det(B) = 5$ and $\det(B') = -2$ and thus both $\mathcal{B}$ and $\mathcal{B}'$ are bases for $\mathbb{R}^3$. By Theorem \ref{thm:bijectivechincoord}, the matrix representation for the change of basis abides
\begin{align*}
[\text{id}]_B^{B'} = [T]_B^{B'} = \begin{bmatrix}
[\vec{v}_{1,B}]_{B'} | [\vec{v}_{2,B}]_{B'} | [\vec{v}_{3,B}]_{B'}
\end{bmatrix}
\end{align*}
where each of $[\vec{v}_{j,B}]_{B'}$ is found via the equation
\begin{align*}
[(v_{j,B})_1]_{B'} (\vec{v}_{1,B'}) + [(v_{j,B})_2]_{B'} (\vec{v}_{2,B'}) + [(v_{j,B})_3]_{B'} (\vec{v}_{3,B'}) = \vec{v}_{j,B}
\end{align*}
just as in Example \ref{exmp:basisR3} with $[(v_{j,B})_i]_{B'}$ being the $i$-th component of $\vec{v}_{j,B}$ in the $\mathcal{B}'$ frame, or equivalently,
\begin{align*}
\begin{bmatrix}
\vec{v}_{1,B'} | \vec{v}_{2,B'} | \vec{v}_{3,B'}
\end{bmatrix}
\begin{bmatrix}
[(v_{j,B})_1]_{B'} \\
[(v_{j,B})_2]_{B'} \\
[(v_{j,B})_3]_{B'}
\end{bmatrix}
&=
\vec{v}_{j,B} \\
[\vec{v}_{j,B}]_{B'} =
\begin{bmatrix}
[(v_{j,B})_1]_{B'} \\
[(v_{j,B})_2]_{B'} \\
[(v_{j,B})_3]_{B'}
\end{bmatrix}
&=
\begin{bmatrix}
\vec{v}_{1,B'} | \vec{v}_{2,B'} | \vec{v}_{3,B'}
\end{bmatrix}^{-1}
\vec{v}_{j,B} \\
&= B'^{-1}\vec{v}_{j,B}
\end{align*} 
Subsequently,
\begin{align*}
[T]_B^{B'} &= \begin{bmatrix}
[\vec{v}_{1,B}]_{B'} | [\vec{v}_{2,B}]_{B'} | [\vec{v}_{3,B}]_{B'}
\end{bmatrix} \\
&= \begin{bmatrix}
B'^{-1}\vec{v}_{1,B} | B'^{-1}\vec{v}_{2,B} | B'^{-1}\vec{v}_{3,B} 
\end{bmatrix} \\
&= B'^{-1}\begin{bmatrix}
\vec{v}_{1,B} | \vec{v}_{2,B} | \vec{v}_{3,B} 
\end{bmatrix} \\
&= B'^{-1}B
\end{align*}
The readers should verify that we can indeed factor out the $B'^{-1}$ from the columns and put it to the left in the third line, and the required matrix representation for the coordinate change is
\begin{align*}
P_B^{B'} = [T]_B^{B'} = B'^{-1}B &= 
\begin{bmatrix}
0 & 2 & 1 \\
0 & 0 & -1 \\
1 & 1 & 0
\end{bmatrix}^{-1}
\begin{bmatrix}
1 & 0 & -1\\
0 & 2 & 1 \\
1 & 1 & 2
\end{bmatrix} \\
&=
\begin{bmatrix}
-\frac{1}{2} & -\frac{1}{2} & 1 \\
\frac{1}{2} & \frac{1}{2} & 0 \\
0 & -1 & 0
\end{bmatrix}
\begin{bmatrix}
1 & 0 & -1\\
0 & 2 & 1 \\
1 & 1 & 2
\end{bmatrix}
=
\begin{bmatrix}
\frac{1}{2} & 0 & 2 \\
\frac{1}{2} & 1 & 0 \\
0 & -2 & -1
\end{bmatrix}
\end{align*}
Let's take $(2,2,3)^T = 2(1,0,1)^T + 1(0,2,1)^T + 0(-1,1,2)^T = (2,1,0)^T_B$ for double-checking:
\begin{align*}
P_B^{B'}(2,1,0)^T_B &= 
\begin{bmatrix}
\frac{1}{2} & 0 & 2 \\
\frac{1}{2} & 1 & 0 \\
0 & -2 & -1
\end{bmatrix}_B^{B'}
\begin{bmatrix}
2 \\
1 \\
0
\end{bmatrix}_B
=
\begin{bmatrix}
1 \\
2 \\
-2
\end{bmatrix}_{B'}
\end{align*}
and indeed $(2,2,3)^T = 1(0,0,1)^T + 2(2,0,1)^T + (-2)(1,-1,0)^T = (1,2,-2)^T_H$.
\end{solution}
For other cases of coordinate transformation, more generally, the relation $P_B^{B'} = [\text{id}]_B^{B'} = B'^{-1}B$ still remains valid where $B$ and $B'$ are matrices composed by the basis vectors from the $\mathcal{B}$ and $\mathcal{B}'$ systems, relative to a third basis (without loss of generality we assume it is the standard basis $\mathcal{S}$\footnote{Unfortunately, as you may notice, there is actually no satisfying "standard" of what really is a standard basis for (real) finite-dimensional vector space other than the real $n$-space since any basis can be regarded to be one with respect to itself. Here we just pretend it is available for the sake of reasoning.}, but the readers are advised to extend this for any other arbitrary basis), that are arranged in columns. To see this from another perspective, take any vector $\vec{v}$ that is expressed in the $\mathcal{B}$ coordinates, $[\vec{v}]_B$. We can view the change in coordinates from $\mathcal{B}$ to $\mathcal{B}'$ in two steps: first from $\mathcal{B}$ to $\mathcal{S}$, and then from $\mathcal{S}$ to $\mathcal{B}'$. From Section \ref{section:6.1.5}, we already know that the former constitutes $[\vec{v}]_S = B[\vec{v}]_B$, and the latter is done by $[\vec{v}]_{B'} = B'^{-1}[\vec{v}]_S$. Combining these two operations together we have $[\vec{v}]_{B'} = B'^{-1}[\vec{v}]_S = B'^{-1}B[\vec{v}]_B$ and hence $[\text{id}]_B^{B'} = B'^{-1}B$.

The second major result in this subsection is
\begin{thm}
\label{thm:isomorphism}
There is always a bijective linear mapping between $\mathcal{V}$ and $\mathbb{R}^n$ where $\mathcal{V}$ is any $n$-dimensional real vector space. In this sense we say $\mathcal{V}$/such a mapping is \index{Isomorphic}\keywordhl{isomorphic}/an \index{Isomorphism}\keywordhl{isomorphism} to $\mathbb{R}^n$. It has an invertible matrix representation. Conversely if a matrix representation of a linear transformation is invertible, it is bijective.
\end{thm}
\begin{proof}
We construct such a mapping explicitly. Note that $\mathcal{V}$ and $\mathbb{R}^n$ are both $n$-dimensional vector spaces and any of their bases will contain $n$ basis vectors. Denote the basis chosen for $\mathcal{V}$ by $\mathcal{B} = \{\vec{v}_1, \vec{v}_2, \ldots, \vec{v}_n\}$ and we use the standard basis $\mathcal{S} = \{\hat{e}_1, \hat{e}_2, \ldots, \hat{e}_n\}$ for $\mathbb{R}^n$. Then the linear mapping $T: \mathcal{V} \to \mathbb{R}^n$ that abides
\begin{align*}
T(\vec{v}_j) = \hat{e}_j    
\end{align*}
where $j = 1,2,\ldots,n$, is bijective as desired. To see this, by Theorem \ref{thm:oneto_onebasis}, as for every $\vec{v}_j$, $T(\vec{v}_j) = \hat{e}_j$ leads to the standard unit vectors that are linearly independent, $T$ is one-to-one. Meanwhile, a direct use of Theorem \ref{thm:onto_basis} over the defined association $T(\vec{v}_j) = \hat{e}_j$ for each of the $\hat{e}_j$ immediately shows that $T$ is onto. Since $T$ is now one-to-one and onto, it is bijective. Again, the bijectivity, in addition to the uniqueness of basis coordinates, implies that for any $\vec{u} \in \mathbb{R}^n$, $[\vec{u}]_S = [T]_B^S[\vec{v}]_B$ has a unique solution $[\vec{v}]_B$, and part (d) to (a) of Theorem \ref{thm:equiv2} then shows that the matrix representation $[T]_B^S$ is invertible. The converse follows the same argument running in opposite direction.
\end{proof}
This theorem enables us to identify and treat any finite-dimensional real vector space $\mathcal{V}$ as the real $n$-space $\mathbb{R}^n$ with $n$ being the dimension of $\mathcal{V}$. Thus we can work with $\mathcal{V}$ as if it is $\mathbb{R}^n$ and the results for $\mathbb{R}^n$ derived in this and the last chapter are all applicable on other $n$-dimensional real vector spaces with an appropriate transformation. Actually, we have been implicitly utilizing this isomorphism relation in many of our previous examples, e.g. writing out the coordinates of a vector from an $n$-dimensional vector space with $n$ components like an $\mathbb{R}^n$ vector. As a corollary,
\begin{proper}
Any two real vector spaces are isomorphic such that there exists a bijective transformation between them, if and only if they have the same number of dimension. Otherwise, there will be no isomorphism between those with different dimensions.
\end{proper}
The "if" direction is easy to see because they are both isomorphic to $\mathbb{R}^n$ and bijectivity is transitive. For the "only if" direction, let the two vector spaces $\mathcal{U}$ and $\mathcal{V}$ have dimensions of $m$ and $n$ respectively, and without loss of generality $m < n$. Then they can never be isomorphic since given any transformation $T: \mathcal{U} \to \mathcal{V}$ the $m$ transformed vectors $T(\vec{u_1}), T(\vec{u_2}), \ldots, T(\vec{u_m})$ will be unable to span the $n$-dimensional $\mathcal{V}$ and by Properties \ref{proper:surjective} all of them are not surjective.
\begin{exmp}
Explicitly show that $\mathcal{U} = \mathcal{P}^3$ and $\mathcal{V} = \text{span}(\mathcal{H})$, where $\mathcal{H} = \{e^x, xe^x, x^2e^x, x^3e^x\}$, are isomorphic by considering $T: \mathcal{U} \to \mathcal{V}$, $T[p(x)] = \int_{-\infty}^x e^x p(x) dx$.
\end{exmp}
\begin{solution}
It is clear that both $\mathcal{U}$ and $\mathcal{V}$ are four-dimensional and by the above corollary they are isomorphic. Take $\mathcal{B} = \{1, x, x^2, x^3\}$ the standard polynomial basis for $\mathcal{U} = \mathcal{P}^3$ and the linearly independent $\mathcal{H}$ is automatically the basis for $\mathcal{V}$. Now we compute the matrix representation $[T]_B^H$ as follows. By elementary calculus,
\begin{align*}
T(1) &= \int_{-\infty}^x e^x dx = e^x \\
T(x) &= \int_{-\infty}^x xe^x dx = xe^x - e^x \\
T(x^2) &= \int_{-\infty}^x x^2e^x dx = x^2e^x - 2xe^x + 2e^x \\
T(x^3) &= \int_{-\infty}^x x^3e^x dx = x^3e^x - 3x^2e^x + 6xe^x - 6e^x \\
\end{align*}
and thus
\begin{align*}
[T]_B^H = 
\begin{bmatrix}
1 & -1 & 2 & -6 \\ 
0 & 1 & -2 & 6 \\
0 & 0 & 1 & -3 \\
0 & 0 & 0 & 1
\end{bmatrix}
\end{align*}
is an upper-triangular matrix and its determinant is simply the product of diagonal entries $(1)^4 = 1 \neq 0$. Therefore, by Theorem \ref{thm:equiv2}, $[T]_B^H$ is invertible and the given transformation, as well as $\mathcal{U}$ and $\mathcal{V}$ themselves, is/are isomorphic according to Theorem \ref{thm:isomorphism}.
\end{solution}
Short Exercise: Redo the above example by considering $T[p(x)] = e^x p(x)$ this time.\footnote{It becomes trivial and the matrix representation is simply the identity matrix.}

\section{More on Coordinate Bases}

\subsection{Linear Change of Coordinates}

In previous parts we have already mentioned about change of coordinates between bases for several times, where such a mapping are confined to be linear just like other transformations discussed. In this section we will drive deeper into the details and address two distinct scenarios: change of coordinates for vectors and linear transformations (matrices).

\subsubsection{Change of Coordinates for Vectors}
The procedure about change of coordinates for vectors have been discussed substantially in Examples \ref{exmp:basisR3}, \ref{exmp:changecoord} and explained through Theorem \ref{thm:bijectivechincoord}. Here we will focus on its geometric interpretation instead, which will be illustrated by the small example below.

\begin{exmp}
\label{exmp:2Dtransform}
Consider the vector space of $\mathbb{R}^2$ as the $x$-$y$ plane. Given a basis $\mathcal{B}$ for $\mathbb{R}^2$ that is consisted of two vectors $\vec{u}_1 = (1,2)^T$ and $\vec{u}_2 = (1,-1)^T$, transform the coordinates of the vector $\vec{v} = (2,1)^T$ from the standard basis $\mathcal{S}$ to $\mathcal{B}$.
\end{exmp}
\begin{solution}
As before, $P_B^S = [\vec{u}_1|\vec{u}_2]$, and it can be seen that
\begin{align*}
&P_B^S =
\begin{bmatrix}
1 & 1 \\
2 & -1
\end{bmatrix}
&P_S^B = (P_B^S)^{-1} =
\begin{bmatrix}
\frac{1}{3} & \frac{1}{3} \\
\frac{2}{3} & -\frac{1}{3}
\end{bmatrix}
\end{align*}
Hence the coordinates of $\vec{v}$ in the $\mathcal{B}$ system is
\begin{align*}
[\vec{v}]_B = P_S^B[\vec{v}]_S = 
\begin{bmatrix}
\frac{1}{3} & \frac{1}{3} \\
\frac{2}{3} & -\frac{1}{3}
\end{bmatrix}_S^B
\begin{bmatrix}
2 \\
1
\end{bmatrix}_S
=
\begin{bmatrix}
1\\
1
\end{bmatrix}_B
\end{align*}
The geometry of this problem is shown in the figure below where each grid line separation represents one unit length of the axis vectors.
\begin{center}
\begin{tikzpicture}[scale = 0.75]
\draw[->] (-5,0)--(5,0) node[right]{$x$};
\draw[->] (0,-5)--(0,5) node[above]{$y$};
\draw[gray,dashed] (-4,-5)--(-4,5);
\draw[gray,dashed] (-3,-5)--(-3,5);
\draw[gray,dashed] (-2,-5)--(-2,5);
\draw[gray,dashed] (-1,-5)--(-1,5);
\draw[gray,dashed] (4,-5)--(4,5);
\draw[gray,dashed] (3,-5)--(3,5);
\draw[gray,dashed] (2,-5)--(2,5);
\draw[gray,dashed] (1,-5)--(1,5);
\draw[gray,dashed] (-5,-4)--(5,-4);
\draw[gray,dashed] (-5,-3)--(5,-3);
\draw[gray,dashed] (-5,-2)--(5,-2);
\draw[gray,dashed] (-5,-1)--(5,-1);
\draw[gray,dashed] (-5,4)--(5,4);
\draw[gray,dashed] (-5,3)--(5,3);
\draw[gray,dashed] (-5,2)--(5,2);
\draw[gray,dashed] (-5,1)--(5,1);
\draw[red,->,line width=1] (-2.2,-4.4)--(2.2,4.4) node[above right]{$u_1$};
\draw[red,->,line width=1] (-4.4,4.4)--(4.4,-4.4) node[below left]{$u_2$};
\draw[red, thick, dashed] (4,-1)--(-1,4) node[right, pos=0, yshift=-5]{$u_1 = 1$};
\draw[red, thick, dashed] (3.5,4)--(-0.5,-4) node[right, pos=0, yshift=5]{$u_2 = 1$};
\draw[red, thick, dashed] (4,2)--(2,4);
\draw[red, thick, dashed] (4,2)--(1,-4);
\draw[red, thick, dashed] (-4,1)--(1,-4);
\draw[red, thick, dashed] (-4,-2)--(-2,-4);
\draw[red, thick, dashed] (-3.5,-4)--(0.5,4);
\draw[red, thick, dashed] (-4,-2)--(-1,4);
\draw[blue,-stealth,line width=2] (0,0)--(2,1) node[right, yshift=10]{$\vec{v} = (2,1)^T = (1,1)^T_B$};
\fill[orange!75, opacity=0.5] (0,0) -- (-1,1) -- (-2,-1) -- (-1,-2) -- cycle;
\fill[Green!75, opacity=0.5] (0,0) -- (-1,0) -- (-1,-1) -- (0,-1) -- cycle;
\draw[orange,->,line width=3] (0,0)--(-1,-2) node[right]{$\vec{u}_1$};
\draw[orange,->,line width=3] (0,0)--(-1,1) node[left]{$\vec{u}_2$};
\draw[Green,->,line width=3] (0,0)--(-1,0) node[left]{$\hat{e}_1$};
\draw[Green,->,line width=3] (0,0)--(0,-1) node[right]{$\hat{e}_2$};
\node[below left]{$O$}; 
\end{tikzpicture}
\end{center}
\end{solution}
In this example, we can see that in the two bases $\mathcal{S}$, $\mathcal{B}$, their axis vectors (reversed in the figure) can be transformed via $T: \mathbb{R}^2 \to \mathbb{R}^2$ with $T(\hat{e}_j) = \vec{u}_j$. The corresponding matrix representation is $\vec{u}_j = [\vec{u}_1|\vec{u}_2]\hat{e}_j = P_B^S\hat{e}_j$. Meanwhile the coordinate transformation follows $[\vec{v}]_B = (P_B^S)^{-1}[\vec{v}]_S = P_S^B[\vec{v}]_S$, where the transformation matrix is the inverse of the former. The former actually alters the vectors themselves and is sometimes known as an \index{Active Transformation}\keywordhl{active (coordinate) transformation}. In contrast, the latter only changes the coordinate frame but keep the vector unchanged and is hence called a \index{Passive Transformation}\keywordhl{passive (coordinate) transformation} (in fact, it is just the identity transformation with a change of basis). We can see that in the example above, after the active transformation the area of square formed by the new two basis vectors is enlarged by a factor of $\abs{\det(P_B^S)}=3$. Such a magnifying factor is a result of Properties \ref{proper:nvolume} and the similar holds for cases of any dimension. Oppositely, with the passive transformation we can say that the value of area of an identical square is shrinked to $\abs{\det(P_S^B)} = \abs{\det((P_B^S)^{-1})} = \abs{\det(P_B^S)}^{-1} = \frac{1}{3}$ of the original, expressed in the new units. Therefore, the appropriate factors in the two scenarios are the inverse of each other.

\subsubsection{Change of Coordinates for Linear Transformations/Matrices}

It is also possible to do a change of coordinates for linear transformations and hence the matrices that represent them. Consider a linear transformation $T: \mathcal{U} \to \mathcal{V}$ that has a matrix representation of $[\vec{v}]_H = [T]_B^H[\vec{u}]_B$ where $\mathcal{B}$ and $\mathcal{H}$ are bases for $\mathcal{U}$ and $\mathcal{V}$ respectively. If we want to change the basis for $\mathcal{U}$ from $\mathcal{B}$ to some other basis $\mathcal{B}'$ (and similarly $\mathcal{H}'$ for $\mathcal{V}$), then the new matrix representation of the linear transformation would be $[\vec{v}]_{H'} = [T]_{B'}^{H'}[\vec{u}]_{B'}$. Since they are the same transformation but only expressed in different coordinate systems, these two matrix equations have to be equivalent. Now, the vectors on both sides of the original equation themselves can undergo changes of coordinates according to the previous Theorem \ref{thm:bijectivechincoord} with $[\vec{u}]_B = [\text{id}]_{B'}^B [\vec{u}]_{B'} = P_{B'}^B [\vec{u}]_{B'}$ and $[\vec{v}]_{H} = [\text{id}]_{H'}^{H} [\vec{v}]_{H'} = Q_{H'}^H [\vec{v}]_{H'}$, where we denote the change of coordinates matrices from $\mathcal{B'}$ to $\mathcal{B}$ by $P_{B'}^{B}$ (and similarly  $\mathcal{H'}$ to $\mathcal{H}$ by $Q_{H'}^{H}$). Subsequently,
\begin{align*}
[\vec{v}]_H &= [T]_B^H[\vec{u}]_B \\
Q_{H'}^H [\vec{v}]_{H'} &= [T]_B^H P_{B'}^B [\vec{u}]_{B'} \\
[\vec{v}]_{H'} &= \left( (Q_{H'}^H)^{-1} [T]_B^H P_{B'}^B \right) [\vec{u}]_{B'}
\end{align*}
Comparing with the latter equation, we can identify $[T]_{B'}^{H'}$ with $(Q_{H'}^H)^{-1} [T]_B^H P_{B'}^B$, and this is the desired formula for change of coordinates over the matrix form of a linear transformation.
\begin{proper}
\label{proper:chcoordsmat}
The change of coordinates for the matrix representation of a linear transformation $T: \mathcal{U} \to \mathcal{V}$ from bases $\mathcal{B}$ and $\mathcal{H}$ to $\mathcal{B}'$ and $\mathcal{H}'$ for $\mathcal{U}$ and $\mathcal{V}$ respectively follows the relation
\begin{align*}
[T]_{B'}^{H'} = (Q_{H'}^H)^{-1} [T]_B^H P_{B'}^B
\end{align*}
where $P_{B'}^{B}$ and $Q_{H'}^{H}$ are matrices for change of coordinates on vectors from $\mathcal{B'}$ to $\mathcal{B}$ and $\mathcal{H'}$ to $\mathcal{H}$ individually.
\end{proper}
Another way to derive the above formula is to consider the linear transformation with respect to the basis $\mathcal{B'}$ to $\mathcal{H'}$ as three smaller steps: firstly, convert the input vector from the basis $\mathcal{B'}$ back to $\mathcal{B}$ ($P_{B'}^B$); subsequently, carry out the transformation in terms of $\mathcal{B}$ and $\mathcal{H}$ ($[T]_B^H$); finally, map the vector from the basis $\mathcal{H}$ to $\mathcal{H'}$ ($Q_H^{H'} = (Q_{H'}^H)^{-1}$). This flow is illustrated in the schematic of Figure \ref{fig:transcoordsmatrix}.
\begin{figure}
    \centering
    \begin{tikzpicture}
    \node[opacity=0.1,scale=5] at (-2,-2) {$\mathcal{U}$}; 
    \node[opacity=0.1,scale=5] at (8,-2) {$\mathcal{V}$};
    \draw [fill=red!15] (-1,-0.75) rectangle (1,0.75);
    \draw [fill=orange!15] (-1,-4.75) rectangle (1,-3.25);
    \draw [fill=blue!15] (5,-0.75) rectangle (7,0.75);
    \draw [fill=Green!15] (5,-4.75) rectangle (7,-3.25);
    \node[scale=2] at (0,0) {$[\vec{u}]_B$};
    \node[scale=2] at (6,0) {$[\vec{v}]_H$};
    \node[scale=2] at (0,-4) {$[\vec{u}]_{B'}$};
    \node[scale=2] at (6,-4) {$[\vec{v}]_{H'}$}; 
    \draw[->, line width=2] (0,-3.25) -- (0,-0.75) node[midway, left]{$P_{B'}^B$};
    \draw[->, line width=2] (6,-0.75) -- (6,-3.25) node[midway, right]{$Q_H^{H'} = (Q_{H'}^H)^{-1}$};
    \draw[->, line width=2] (1,0) -- (5,0) node[midway, above]{$[T]_B^H$};
    \draw[Goldenrod, ->, line width=2] (1,-4) -- (5,-4) node[midway, below]{$[T]_{B'}^{H'}$};
    \end{tikzpicture}
    \caption{A schematic showing how the change of coordinate bases works for linear transformation.}
    \label{fig:transcoordsmatrix}
\end{figure}

\begin{exmp}
Use Properties \ref{proper:chcoordsmat} to redo Example \ref{exmp:lineartransderivative} with respect to new bases $\mathcal{B}' = \{1, x-1, (x-1)^2\}$ and $\mathcal{H}' = \{1, x+1\}$.
\end{exmp}
\begin{solution}
First it is instructive to find $P_{B'}^B$ and $Q_{H'}^H$. We leave to the readers to verify that
\begin{align*}
P_{B'}^B &= 
\begin{bmatrix}
1 & -1 & 1 \\
0 & 1 & -2\\
0 & 0 & 1
\end{bmatrix}
& Q_{H'}^H &=
\begin{bmatrix}
1 & 1 \\
0 & 1 
\end{bmatrix}
\end{align*}
and hence by Properties \ref{proper:chcoordsmat},
\begin{align*}
[T]_{B'}^{H'} &= (Q_{H'}^H)^{-1} [T]_B^H P_{B'}^B \\
&= \begin{bmatrix}
1 & 1 \\
0 & 1 
\end{bmatrix}^{-1}
\begin{bmatrix}
0 & 1 & 0 \\
0 & 0 & 2
\end{bmatrix}
\begin{bmatrix}
1 & -1 & 1 \\
0 & 1 & -2\\
0 & 0 & 1
\end{bmatrix} \\
&= \begin{bmatrix}
1 & -1 \\
0 & 1 
\end{bmatrix}
\begin{bmatrix}
0 & 1 & 0 \\
0 & 0 & 2
\end{bmatrix}
\begin{bmatrix}
1 & -1 & 1 \\
0 & 1 & -2\\
0 & 0 & 1
\end{bmatrix} =
\begin{bmatrix}
0 & 1 & -4 \\
0 & 0 & 2
\end{bmatrix}
\end{align*}
We use a test case to check the answer. Let $p(x) = x^2 - 3x + 1 = (x-1)^2 - (x-1) - 1$. Then its coordinates in the $\mathcal{B}'$ basis is $(-1,-1,1)^T_{B'}$, and the transformation can be described by 
\begin{align*}
[T]_{B'}^{H'} (1,-1,-1)^T_{B'} &= 
\begin{bmatrix}
0 & 1 & -4 \\
0 & 0 & 2
\end{bmatrix}_{B'}^{H'}
\begin{bmatrix}
-1 \\
-1 \\
1
\end{bmatrix}_{B'}
=
\begin{bmatrix}
-5 \\
2
\end{bmatrix}_{H'}
\end{align*}
which corresponds to $-5(1) + 2(x+1) = 2x-3$, which is consistent with the usual calculation $T(p(x)) = p'(x) = (x^2-3x+1)' = 2x-3$ from elementary calculus.
\end{solution}

In most of the times, we are interested in the type of linear transformations that are an \index{Endomorphism}\keywordhl{endomorphism} (sometimes also referred to as a \index{Linear Operator}\keywordhl{linear operator}) in which the mapping is from a vector space $\mathcal{V}$ to itself, i.e. $T: \mathcal{V} \to \mathcal{V}$\footnote{An endomorphism that is at the same time an isomorphism is known as an \index{Automorphism}\keywordhl{automorphism}, e.g. the linear transformation in Example \ref{exmp:endomorphch}.}. Often we also use the same basis $\mathcal{B}$ for the input and output. Subsequently, to change the basis for both of them at the same time, let's say $\mathcal{B}'$, if the matrix for change of coordinates on vectors from $\mathcal{B}'$ to $\mathcal{B}$ is denoted as $P = P_{B'}^B$, then Properties \ref{proper:chcoordsmat} is reduced to $[T]_{B'}^{B'} = (P_{B'}^B)^{-1} [T]_B^B P_{B'}^B = P^{-1}AP$ where $A = [T]_B^B$ is the original matrix representation of the endomorphism. When it is clear from the context, we will simply write $[T]_B^B$ ($[T]_{B'}^{B'}$) as $[T]_B$ ($[T]_{B'}$).
\begin{proper}
\label{proper:endomorph}
For a linear endomorphism $T: \mathcal{V} \to \mathcal{V}$, the change of coordinates for its matrix representation from the old basis $\mathcal{B}$ to the new one $\mathcal{B}'$ is described by the formula
\begin{align*}
[T]_{B'} = (P_{B'}^B)^{-1} [T]_B P_{B'}^B    
\end{align*}
Or speaking loosely, the change of coordinates for a matrix in general takes the form of
\begin{align*}
A' = P^{-1}AP
\end{align*}
\end{proper}

\begin{exmp}
\label{exmp:endomorphch}
For a two-dimensional vector space $\mathcal{V}$ with a basis $\mathcal{B} = \{\vec{v}_1, \vec{v}_2\}$, if a linear endomorphism $T: \mathcal{V} \to \mathcal{V}$ is defined by $T(\vec{v}_1) = \vec{v}_1$, $T(\vec{v}_2) = \vec{v}_1 + \vec{v}_2$, finds its matrix representation with respect to $\mathcal{B}$. Subsequently, if a new basis $\mathcal{B}'$ is formed by $\{\vec{v}'_1, \vec{v}'_2\}$ where $\vec{v}'_1 = 2\vec{v}_1 - \vec{v}_2$ and $\vec{v}'_2 = -\vec{v}_1 + \vec{v}_2$, use Properties \ref{proper:endomorph} to compute the matrix representation of the endomorphism with respect to the new basis.
\end{exmp}
\begin{solution}
By Definition \ref{defn:matrixrepoflintrans}, the linear transformation has a matrix representation of
\begin{align*}
[T]_B = \begin{bmatrix}
[T(\vec{v}_1)]_B|[T(\vec{v}_2)]_B    
\end{bmatrix} &= 
\begin{bmatrix}
[\vec{v}_1]_B|[\vec{v}_1 + \vec{v}_2]_B    
\end{bmatrix} \\
&=
\begin{bmatrix}
1 & 1 \\
0 & 1
\end{bmatrix}
\end{align*}
with respect to the old basis $\mathcal{B}$. The appropriate $P_{B'}^B$ matrix that will be used for Properties \ref{proper:endomorph}, by Theorem \ref{thm:bijectivechincoord}, is
\begin{align*}
P_{B'}^B = 
\begin{bmatrix}
[\vec{v}'_1]_B|[\vec{v}'_2]_B
\end{bmatrix}
&= \begin{bmatrix}
[2\vec{v}_1 - \vec{v}_2]_B|[-\vec{v}_1 + \vec{v}_2]_B
\end{bmatrix} \\
&=
\begin{bmatrix}
2 & -1 \\
-1 & 1
\end{bmatrix}
\end{align*}
and thus the desired new matrix representation of the endomorphism with respect to $\mathcal{B}'$ is
\begin{align*}
[T]_{B'} &= (P_{B'}^B)^{-1} [T]_B P_{B'}^B \\
&= 
\begin{bmatrix}
2 & -1 \\
-1 & 1
\end{bmatrix}^{-1}
\begin{bmatrix}
1 & 1 \\
0 & 1
\end{bmatrix}
\begin{bmatrix}
2 & -1 \\
-1 & 1
\end{bmatrix} \\
&= 
\begin{bmatrix}
1 & 1 \\
1 & 2
\end{bmatrix}
\begin{bmatrix}
1 & 1 \\
0 & 1
\end{bmatrix}
\begin{bmatrix}
2 & -1 \\
-1 & 1
\end{bmatrix}
=
\begin{bmatrix}
0 & 1\\
-1 & 2
\end{bmatrix}
\end{align*}
\end{solution}

\subsection{Gram-Schmidt Orthogonalization, QR Decomposition}

Sometimes the coordinate basis consists of vectors that are linearly independent but not orthogonal to each other, unlike the standard basis. A common way to create an orthogonal basis from the set is to apply the so-called \index{Gram-Schmidt Orthogonalization}\keywordhl{Gram-Schmidt Orthogonalization}. Basically, it is an iterative method. At each step it constructs a vector that are orthogonal to all the previously processed vectors by removing the parallel components projected onto them (blue) while retaining the orthogonal part (red).
\begin{center}
\begin{tikzpicture}[scale=1.3]
\coordinate (0) at (0,0);
\coordinate (vecu) at (4,1);
\coordinate (vecv) at (1,2);
\draw[->](0)--(vecu) node[right]{$\vec{u}_1$, $\vec{v}_1$};
\draw[->](0)--(vecv) node[above]{$\vec{u}_2$};
\draw[red, dashed, thick, ->] (24/17, 6/17)--(1,2) node[midway, right]{$\vec{v_2}$};
\draw[red] (24/17+0.2, 6/17+0.05)--(24/17+0.15, 6/17+0.25)--(24/17-0.05, 6/17+0.2);
\pic[draw, "$\theta$", angle eccentricity=1.5] {angle = vecu--0--vecv};
\draw[blue, very thick] (0,0)--(24/17, 6/17) node[below, shift={(0mm, -2mm)}]{$\text{proj}_{v_1}\vec{u_2}$};
\end{tikzpicture}
\end{center}
\begin{defn}[Algorithm for Gram-Schmidt Orthogonalization]
\label{defn:GSorth}
Given a coordinate basis consisted of $\vec{u}_1, \vec{u}_2, \vec{u}_3, \ldots, \vec{u}_n \in \mathbb{R}^m$, Gram-Schmidt Orthogonalization transforms them into $\vec{v}_1, \vec{v}_2, \vec{v}_3, \ldots, \vec{v}_n \in \mathbb{R}^m$ ($m$ and $n$ are not necessarily equal) according to the following formulae:
\begin{align*}
\vec{v}_1 &= \vec{u}_1 \\
\vec{v}_2 &= \vec{u}_2 - \text{proj}_{v_1}\vec{u}_2 = \vec{u}_2 - \frac{\vec{v}_1 \cdot \vec{u}_2}{\norm{\vec{v}_1}^2} \vec{v}_1 \\
\vec{v}_3 &= \vec{u}_3 - \text{proj}_{v_1}\vec{u}_3 - \text{proj}_{v_2}\vec{u}_3 = \vec{u}_3 - \frac{\vec{v}_1 \cdot \vec{u}_3}{\norm{\vec{v}_1}^2} \vec{v}_1 - \frac{\vec{v}_2 \cdot \vec{u}_3}{\norm{\vec{v}_2}^2} \vec{v}_2 \\
\vdots \\
\vec{v}_n &= \vec{u}_n - \text{proj}_{v_1}\vec{u}_n - \text{proj}_{v_2}\vec{u}_n - \cdots - \text{proj}_{v_{n-1}}\vec{u}_n \\
&= \vec{u}_n - \frac{\vec{v}_1 \cdot \vec{u}_n}{\norm{\vec{v}_1}^2} \vec{v}_1 - \frac{\vec{v}_2 \cdot \vec{u}_n}{\norm{\vec{v}_2}^2} \vec{v}_2 - \cdots - \frac{\vec{v}_{n-1} \cdot \vec{u}_n}{\norm{\vec{v}_{n-1}}^2} \vec{v}_{n-1}
\end{align*}
In general, for $j \geq 2$, the $j$-th new vector is computed by
\begin{align*}
\vec{v}_j &= \vec{u}_j - \sum_{k=1}^{j-1}\text{proj}_{v_k}\vec{u}_j  = \vec{u}_j - \sum_{k=1}^{j-1}\frac{\vec{v}_k \cdot \vec{u}_j}{\norm{\vec{v}_k}^2} \vec{v}_k
\end{align*}
where the expression of projection, Properties \ref{proper:proj}, is used.
\end{defn}
A variant of Gram-Schmidt Orthogonalization is to normalize every vector at each step immediately, such that $\norm{\hat{v_j}} = 1$ for all $j$, and the resulted basis is said to be \index{Orthonormal}\keywordhl{orthonormal} (both orthogonal and of unit length). The formulae in Definition \ref{defn:GSorth} are then reduced to
\begin{defn}[Gram-Schmidt Orthogonalization with Normalization]
\label{defn:GSorth_norm}
\begin{align*}
\hat{v}_1 &= \frac{\vec{u}_1}{\norm{\vec{u}_1}} \\
\hat{v}_2 &= \frac{\vec{u}_2 - (\hat{v}_1 \cdot \vec{u}_2)\hat{v}_1}{\norm{\vec{u}_2 - (\hat{v}_1 \cdot \vec{u}_2)\hat{v}_1}} \\
\hat{v}_3 &= \frac{\vec{u}_3 - (\hat{v}_1 \cdot \vec{u}_3)\hat{v}_1 - (\hat{v}_2 \cdot \vec{u}_3)\hat{v}_2}{\norm{\vec{u}_3 - (\hat{v}_1 \cdot \vec{u}_3)\hat{v}_1 - (\hat{v}_2 \cdot \vec{u}_3)\hat{v}_2}} \\
\vdots \\
\hat{v}_n &= \frac{\vec{u}_n - (\hat{v}_1 \cdot \vec{u}_n)\hat{v}_1 - (\hat{v}_2 \cdot \vec{u}_n)\hat{v}_2 - \cdots - (\hat{v}_{n-1} \cdot \vec{u}_n)\hat{v}_{n-1}}{\norm{\vec{u}_n - (\hat{v}_1 \cdot \vec{u}_n)\hat{v}_1 - (\hat{v}_2 \cdot \vec{u}_n)\hat{v}_2 - \cdots - (\hat{v}_{n-1} \cdot \vec{u}_n)\hat{v}_{n-1}}} 
\end{align*}
For $j \geq 2$, the general formulae is
\begin{align*}
\hat{v}_j &= \frac{\vec{u}_j - \sum_{k=1}^{j-1}(\hat{v}_k \cdot \vec{u}_j)\hat{v}_k}{\norm{\vec{u}_j - \sum_{k=1}^{j-1}(\hat{v}_k \cdot \vec{u}_j)\hat{v}_k}}
\end{align*}
\end{defn}
\begin{exmp}
\label{exmp:GS_ex}
Perform Gram-Schmidt Orthogonalization with normalization on the coordinate basis for $\mathbb{R}^3$ that is consisted of $\vec{u_1} = (1,2,2)^T$, $\vec{u_2} = (1,-1,0)^T$, $\vec{u_3} = (3,-1,1)^T$, using the formula in Definition \ref{defn:GSorth_norm}.
\end{exmp}
\begin{solution}
The first vector is
\begin{align*}
\hat{v}_1 &= \frac{1}{\sqrt{1^2+2^2+2^2}}
\begin{bmatrix}
1 \\
2 \\
2
\end{bmatrix} 
= 
\frac{1}{3}
\begin{bmatrix}
1 \\
2 \\
2
\end{bmatrix} 
=
\begin{bmatrix}
\frac{1}{3} \\
\frac{2}{3} \\
\frac{2}{3}
\end{bmatrix}
\end{align*}
The second vector can be found via
\begin{align*}
\vec{u}_2 - (\hat{v}_1 \cdot \vec{u}_2)\hat{v}_1 &= 
\begin{bmatrix}
1 \\
-1 \\
0
\end{bmatrix} 
-
[(\frac{1}{3})(1) + (\frac{2}{3})(-1) + (\frac{2}{3})(0)]
\begin{bmatrix}
\frac{1}{3} \\
\frac{2}{3} \\
\frac{2}{3}
\end{bmatrix} \\
&= 
\begin{bmatrix}
1 \\
-1 \\
0
\end{bmatrix}
- (-\frac{1}{3})
\begin{bmatrix}
\frac{1}{3} \\
\frac{2}{3} \\
\frac{2}{3}
\end{bmatrix}
=
\begin{bmatrix}
\frac{10}{9} \\
-\frac{7}{9} \\
\frac{2}{9}
\end{bmatrix} \\
\hat{v_2} &= \frac{1}{\sqrt{(\frac{10}{9})^2+(-\frac{7}{9})^2+(\frac{2}{9})^2}}
\begin{bmatrix}
\frac{10}{9} \\
-\frac{7}{9} \\
\frac{2}{9}
\end{bmatrix}
=
\frac{3}{\sqrt{17}}
\begin{bmatrix}
\frac{10}{9} \\
-\frac{7}{9} \\
\frac{2}{9}
\end{bmatrix}
=
\begin{bmatrix}
\frac{10}{3\sqrt{17}} \\
-\frac{7}{3\sqrt{17}}\\
\frac{2}{3\sqrt{17}}
\end{bmatrix}
\end{align*}
By the same essence, we have the third vector as
\begin{align*}
&\quad \vec{u}_3 - (\hat{v}_1 \cdot \vec{u}_3)\hat{v}_1 - (\hat{v}_2 \cdot \vec{u}_3)\hat{v}_2\\
&=
\begin{bmatrix}
3 \\
-1 \\
1
\end{bmatrix}
-
[(\frac{1}{3})(3) + (\frac{2}{3})(-1) + (\frac{2}{3})(1)]
\begin{bmatrix}
\frac{1}{3} \\
\frac{2}{3} \\
\frac{2}{3}
\end{bmatrix} \\
&\quad -
[(\frac{10}{3\sqrt{17}})(3) + (-\frac{7}{3\sqrt{17}})(-1) + (\frac{2}{3\sqrt{17}})(1)]
\begin{bmatrix}
\frac{10}{3\sqrt{17}} \\
-\frac{7}{3\sqrt{17}} \\
\frac{2}{3\sqrt{17}}
\end{bmatrix} \\
&=
\begin{bmatrix}
3 \\
-1 \\
1
\end{bmatrix}
- 1
\begin{bmatrix}
\frac{1}{3} \\
\frac{2}{3} \\
\frac{2}{3}
\end{bmatrix}
-
\textcolor{red}{\frac{13}{\sqrt{17}}}
\begin{bmatrix}
\frac{10}{3\sqrt{17}} \\
-\frac{7}{3\sqrt{17}} \\
\frac{2}{3\sqrt{17}}
\end{bmatrix}
=
\begin{bmatrix}
\frac{2}{17} \\
\frac{2}{17} \\
-\frac{3}{17}
\end{bmatrix}
\\
\hat{v}_3 &=  \frac{1}{\sqrt{(\frac{2}{17})^2 + (\frac{2}{17})^2 + (-(\frac{3}{17}))^2}}
\begin{bmatrix}
\frac{2}{17} \\
\frac{2}{17} \\
-\frac{3}{17}
\end{bmatrix}
=
\textcolor{blue}{\sqrt{17}}
\begin{bmatrix}
\frac{2}{17} \\
\frac{2}{17} \\
-\frac{3}{17}
\end{bmatrix}
=
\begin{bmatrix}
\frac{2}{\sqrt{17}} \\
\frac{2}{\sqrt{17}} \\
-\frac{3}{\sqrt{17}}
\end{bmatrix}
\end{align*}
\end{solution}
Short Exercise: Verify that $\hat{v}_1, \hat{v}_2, \hat{v}_3$ are pairwise orthogonal.\footnote{We will only check $\hat{v}_1$ and $\hat{v}_3$ are orthogonal to each other and leave the remaining two pairs to the readers. $\hat{v}_1 \cdot \hat{v}_3 = (\frac{1}{3}, \frac{2}{3}, \frac{2}{3})^T \cdot (\frac{2}{\sqrt{17}}, \frac{2}{\sqrt{17}}, -\frac{3}{\sqrt{17}})^T = (\frac{1}{3})(\frac{2}{\sqrt{17}}) + (\frac{2}{3})(\frac{2}{\sqrt{17}}) + (\frac{2}{3})(-\frac{3}{\sqrt{17}}) = 0$.} \\
\\
An major application of the Gram-Schmidt process is the \index{QR Decomposition}\keywordhl{QR Decomposition}, which factors a matrix into two matrices, one as its orthogonal basis vectors arranged in columns and another one as a upper-triangular matrix (non-zero elements only found along or above the main diagonal) where the elements take the form of $\vec{u}_j \cdot \hat{v}_i$ as shown below. This is very useful in the processing of large matrices and least-square error fitting.
\begin{proper}
\label{proper:QRdecompose}
For a matrix $A = [\vec{u}_1|\vec{u}_2|\vec{u}_3|\cdots|\vec{u}_n]$, and the matrix $Q = [\hat{v}_1|\hat{v}_2|\hat{v}_3|\cdots|\hat{v}_n]$, where the $\hat{v}_j$ are orthonormal vectors that come from carrying out Gram-Schmidt orthogonalization on the basis vectors $\vec{u}_j$ according to the Definition \ref{defn:GSorth_norm}, we have $A = QR$, where
\begin{align*}
R &= 
\begin{bmatrix}
\hat{v}_1 \cdot \vec{u}_1 &  \hat{v}_1 \cdot \vec{u}_2 & \hat{v}_1 \cdot \vec{u}_3 & \cdots & \hat{v}_1 \cdot \vec{u}_n \\
0 & \hat{v}_2 \cdot \vec{u}_2 &  \hat{v}_2 \cdot \vec{u}_3 &  & \hat{v}_2 \cdot \vec{u}_n \\
0 & 0 & \hat{v}_3 \cdot \vec{u}_3 &  & \hat{v}_3 \cdot \vec{u}_n \\
\vdots & & & \ddots & \vdots\\
0 & 0 & 0 & \cdots & \hat{v}_n\cdot \vec{u}_n \\
\end{bmatrix} \\
\text{i.e. } R_{ij} &= 
\begin{cases}
\hat{v}_i \cdot \vec{u}_j & i \leq j \\
0 & i > j
\end{cases}
& \text{for $1 \leq i, j \leq n$}
\end{align*}
is an upper triangular $n \times n$ invertible matrix.
\end{proper}
\begin{proof}
We will show that every column of $A$ and $QR$ coincides. The $j$-th column of $A$ is simply the $j$-th vector in the starting basis, $\vec{u}_j$. Meanwhile, the $j$-th column of $QR$ is $Q$ times the $j$-th column of $R$, which is
\begin{align*}
QR^{(j)} &=
\begin{bmatrix}
\hat{v}_1|\hat{v}_2|\cdots|\hat{v}_j|\cdots|\hat{v}_n   
\end{bmatrix}
\begin{bmatrix}
\hat{v}_1 \cdot \vec{u}_j \\   
\hat{v}_2 \cdot \vec{u}_j \\   
\vdots \\
\hat{v}_j \cdot \vec{u}_j \\
\vdots \\
0
\end{bmatrix} \\
&= (\hat{v}_1 \cdot \vec{u}_j) \hat{v}_1 + (\hat{v}_2 \cdot \vec{u}_j) \hat{v}_2 + \cdots + (\hat{v}_j \cdot \vec{u}_j) \hat{v}_j + 0 \\
&= \sum_{k=1}^{j}(\hat{v_k} \cdot \vec{u_j})\hat{v_k} = \sum_{k=1}^{j-1}(\hat{v_k} \cdot \vec{u_j})\hat{v_k} + (\hat{v_j} \cdot \vec{u_j})\hat{v_j}
\end{align*}
By Definition \ref{defn:GSorth_norm}, we have
\begin{align*}
\hat{v}_j &= \frac{\vec{u}_j - \sum_{k=1}^{j-1}(\hat{v}_k \cdot \vec{u}_j)\hat{v}_k}{\norm{\vec{u}_j - \sum_{k=1}^{j-1}(\hat{v}_k \cdot \vec{u}_j)\hat{v}_k}}
\end{align*}
\footnote{\label{foot:GSnonzero} Some may ask if $\norm{\vec{u}_j - \sum_{k=1}^{j-1}(\hat{v}_k \cdot \vec{u}_j)\hat{v}_k}$ can be $0$ (or $\vec{u}_j - \sum_{k=1}^{j-1}(\hat{v}_k \cdot \vec{u}_j)\hat{v}_k$ be the zero vector) and $\hat{v}_j$ is not well-defined. However, this will contradict the linear independence of the basis vectors $\vec{u}_k$. We can use induction to show this: (WIP)} which after rearrangement, becomes
\begin{align*}
\vec{u}_j = \sum_{k=1}^{j-1}(\hat{v}_k \cdot \vec{u}_j)\hat{v}_k + {\norm{\vec{u}_j - \sum_{k=1}^{j-1}(\hat{v}_k \cdot \vec{u}_j)\hat{v}_k}}\hat{v}_j
\end{align*}
Therefore, in order to show that $\vec{u}_j = QR^{(j)}$, by comparing the two expressions, we need to check if
\begin{align*}
\hat{v}_j \cdot \vec{u}_j &= {\norm{\vec{u}_j - \sum_{k=1}^{j-1}(\hat{v}_k \cdot \vec{u}_j)\hat{v}_k}} 
\end{align*}
Consider
\begin{align*}
(\vec{u}_j - \sum_{k=1}^{j-1}(\hat{v}_k \cdot \vec{u}_j)\hat{v}_k) \cdot \hat{v}_j &= \hat{v}_j \cdot \vec{u}_j - \sum_{k=1}^{j-1}(\hat{v}_k \cdot \vec{u_j}) (\hat{v}_k \cdot \hat{v}_j)\\
&= \hat{v}_j \cdot \vec{u}_j 
\end{align*}
as $\vec{v}_k \cdot \hat{v}_j = 0$ for $k \neq j$ due to the orthogonality enforced by the Gram-Schmidt process. On the other hand, by Definition \ref{defn:GSorth_norm} again, 
\begin{align*}
\vec{u}_j - \sum_{k=1}^{j-1}(\hat{v}_k \cdot \vec{u}_j)\hat{v}_k = \norm{\vec{u}_j - \sum_{k=1}^{j-1}(\hat{v}_k \cdot \vec{u}_j)\hat{v}_k} \hat{v}_j
\end{align*}
Therefore,
\begin{align*}
(\vec{u}_j - \sum_{k=1}^{j-1}(\hat{v}_k \cdot \vec{u}_j)\hat{v}_k) \cdot \hat{v}_j &= \left(\norm{\vec{u}_j - \sum_{k=1}^{j-1}(\hat{v}_k \cdot \vec{u}_j\hat{v}_k} \hat{v}_j)\right) \cdot \hat{v}_j \\
&= \norm{\vec{u}_j - \sum_{k=1}^{j-1}(\hat{v}_k \cdot \vec{u}_j)\hat{v}_k} (\hat{v}_j \cdot \hat{v}_j) \\
&= \norm{\vec{u}_j - \sum_{k=1}^{j-1}(\hat{v}_k \cdot \vec{u}_j)\hat{v}_k}
\end{align*}
as $\hat{v}_j \cdot \hat{v}_j = \norm{\hat{v}_j}^2 = 1^2 = 1$. The required equality is then established and the result follows. The invertibility of $R$ can be shown by noting that all diagonal elements $\hat{v}_j \cdot \hat{u}_j = {\norm{\vec{u}_j - \sum_{k=1}^{j-1}(\hat{v}_k \cdot \vec{u}_j)\hat{v}_k}} $ of the upper triangular $R$ matrix are non-zero (see Footnote \ref{foot:GSnonzero}). 
\end{proof}

\begin{exmp}
\label{exmp:QRdecom}
Construct a QR decomposition for the case in Example \ref{exmp:GS_ex}.
\end{exmp}
\begin{solution}
The matrix $Q$ is simply
\begin{align*}
Q &= 
\begin{bmatrix}
\frac{1}{3} & \frac{10}{3\sqrt{17}} & \frac{2}{\sqrt{17}} \\
\frac{2}{3} & -\frac{7}{3\sqrt{17}} & \frac{2}{\sqrt{17}} \\
\frac{2}{3} & \frac{2}{3\sqrt{17}} & -\frac{3}{\sqrt{17}}
\end{bmatrix}
\end{align*}
And by Properties \ref{proper:QRdecompose}, the entries in $R$ are
\begin{align*}
R &= 
\begin{bmatrix}
\hat{v}_1 \cdot \vec{u}_1 &  \hat{v}_1 \cdot \vec{u}_2 & \hat{v}_1 \cdot \vec{u}_3 \\
0 & \hat{v}_2 \cdot \vec{u}_2 &  \hat{v}_2 \cdot \vec{u}_3 \\
0 & 0 & \hat{v}_3 \cdot \vec{u}_3 
\end{bmatrix}  \\
&= 
\begin{bmatrix}
3 & -\frac{1}{3} & 1 \\
0 & \frac{\sqrt{17}}{3} & \textcolor{red}{\frac{13}{\sqrt{17}}}  \\
0 & 0 & \textcolor{blue}{\frac{1}{\sqrt{17}}} \\
\end{bmatrix} 
\end{align*}
whose values can be readily inferred from the steps during the orthogonalization process itself in Example \ref{exmp:GS_ex} (highlighted in red/blue). The readers are encouraged to compute the matrix product $QR$ to see if the original matrix $A$ is recovered.\\
\\
We conclude this section with a small remark related to the concept of orthogonal complement discussed in Section \ref{section:null}.
\begin{proper}
For an orthogonal(-normal) basis $\mathcal{B} = \{\vec{v}_1, \vec{v}_2, \vec{v}_3, \cdots, \vec{v}_n\}$ for a finite-dimensional vector space $\mathcal{V}$, the subspaces $\mathcal{V}_G$ and $\mathcal{V}_H$ formed by $\mathcal{G} = \{\vec{v}_I\}$ and $\mathcal{H} = \{\vec{v}_J\}$ respectively, where $I$ and $J$ are mutually exclusive indices that together exhaust all integers from $1$ to $n$, are the orthogonal complement to each other, such that $\mathcal{V}_G^\perp = \mathcal{V}_H$ and $\mathcal{V}_G \oplus \mathcal{V}_H = \mathcal{V}$.
\end{proper}
\end{solution}

\section{Python Programming}
We can define a function to a change in coordinates for vectors or matrices. Let's first write a helper function to produce the change of coordinates matrix $P$ proposed in Theorem \ref{thm:bijectivechincoord}, which equals to $B'^{-1}B$ as discussed in the end of Example \ref{exmp:changecoord}:
\begin{lstlisting}
import numpy as np
from scipy import linalg

def P_matrix(B, B_prime):
    """ Computes the P matrix of change in coordinates. """
    P = linalg.inv(B_prime) @ B
    return(P)
\end{lstlisting}
Then we use Example \ref{exmp:2Dtransform} as an illustration for coordinate change for vectors, where regarding $\mathcal{B}$ we have
\begin{align*}
B = 
\begin{bmatrix}
1 & 1 \\ 
2 & -1
\end{bmatrix}
\end{align*}
and $B' = I$ as $\mathcal{B}' = \mathcal{S}$ implicitly in this case. We define another function for transforming the coordinates of any given vector as
\begin{lstlisting}
def coord_trans_vector(vec, P):
    """ Transforms the coordinates of a vector. """
    trans_vec = linalg.inv(P) @ vec
    return(trans_vec)    
\end{lstlisting}
Then Example \ref{exmp:2Dtransform} can be proceeded as follows.
\begin{lstlisting}
B = np.array([[1.,  1.], 
              [2., -1.]])

P = P_matrix(B, np.identity(2))
old_v = np.array([2., 1.])
new_v = coord_trans_vector(old_v, P)
print(new_v)    
\end{lstlisting}
which returns \verb|[1. 1.]| correctly. Similarly, according to Properties \ref{proper:endomorph}, we can make a function to carry out the change of coordinates for matrices through
\begin{lstlisting}
def coord_trans_matrix(A, P):
    """ Transforms the coordinates of a matrix. """
    trans_matrix = linalg.inv(P) @ A @ P
    return(trans_matrix)    
\end{lstlisting}
Let's use this to redo Example \ref{exmp:endomorphch}, where
\begin{align*}
A &= 
\begin{bmatrix}
1 & 1 \\
0 & 1
\end{bmatrix} &
& P=
\begin{bmatrix}
2 & -1 \\
-1 & 1
\end{bmatrix}
\end{align*}
Subsequently,
\begin{lstlisting}
B = np.array([[2., -1.], 
             [-1.,  1.]])
old_A = np.array([[1.,  1.], 
              [0.,  1.]])

P = P_matrix(B, np.identity(2))
new_A = coord_trans_matrix(old_A, P)
print(new_A)    
\end{lstlisting}
gives
\begin{lstlisting}
[[ 0.  1.]
 [-1.  2.]]    
\end{lstlisting}
as expected. Meanwhile, to apply Gram-Schmidt Orthogonalization for a basis, in addition to deriving the corresponding QR decomposition, we can use the function \verb|qr| in \verb|scipy.linalg|. Let's we use Examples \ref{exmp:GS_ex} and \ref{exmp:QRdecom} as a demonstration:
\begin{lstlisting}
A = np.array([[1.,  1.,  3.],
              [2., -1., -1.],
              [2.,  0.,  1.],])
Q, R = linalg.qr(A)
print("Q = ", Q)
print("R = ", R)
\end{lstlisting}
which yields
\begin{lstlisting}
Q =  [[-0.33333333  0.80845208 -0.48507125]
      [-0.66666667 -0.56591646 -0.48507125]
      [-0.66666667  0.16169042  0.72760688]]
R =  [[-3.          0.33333333 -1.        ]
      [ 0.          1.37436854  3.15296313]
      [ 0.          0.         -0.24253563]]
\end{lstlisting}
The columns in \verb|Q| form the desired orthonormal basis. Notice that the signs of the first/third column vectors in \verb|Q| are flipped when compared to that in Example \ref{exmp:QRdecom}, which leads to corresponding sign switches in \verb|R| as well.

\section{Exercises}

\begin{Exercise}
Let $\mathcal{V} = \mathcal{W} = \mathbb{R}^3$, and take $\mathcal{B} = \{(1,1,1)^T, (1,1,0)^T, (1,0,0)^T\}$ and $\mathcal{H} = \{(1,2,3)^T, (1,-1,0)^T, (2,-1,-1)^T\}$ as bases for $\mathcal{V}$ and $\mathcal{W}$. If a linear transformation $T: \mathbb{R}^3 \to \mathbb{R}^3$ is defined by $T(x,y,z)^T = (x+y+z,2x+y,x-y-3z)^T$, find its matrix representation and decide if it is one-to-one, onto and hence bijective.
\end{Exercise}

\begin{Exercise}
Let $\mathcal{V}$ be the real vector space generated by the basis $\mathcal{B} = \{\cos x, \sin x, x\cos x, x\sin x\}$ and $T: \mathcal{V} \to \mathcal{V}, T[f(x)] = f'(x)$ be the differentiation operator over $\mathcal{V}$. Find the matrix representation of $T$ with respect to $\mathcal{B}$, and determine if $T$ is injective, surjective and hence bijective. 
\end{Exercise}

\begin{Exercise}
Let $\mathcal{V} = \mathcal{P}^2$, $\mathcal{W} = \mathcal{P}^3$ be the polynomial spaces of degree $2$ and $3$ respectively. Define $T: \mathcal{V} \to \mathcal{W}$ by
\begin{align*}
T[p(x)] = \int_1^x p(x) dx
\end{align*}
find its matrix representation with respect to the standard bases and decide if the transformation is isomorphic.
\end{Exercise}

\begin{Exercise}
Show that every identity transformation $T: \mathcal{V} \to \mathcal{V}, T(\vec{v}) = \text{id}(\vec{v}) = \vec{v}$ for a finite-dimensional vector space $\mathcal{V}$ with respect to a fixed basis $\mathcal{B}$ throughout always has a matrix representation of an identity matrix such that $[T]_B = I$. 
\end{Exercise}

\begin{Exercise}
Apply Gram-Schmidt Orthogonalization on the following set of vectors, and then write down their QR Decomposition.
\begin{enumerate}[label=(\alph*)]
\item $\vec{u}_1 = (1,2)^T, \vec{u}_2 = (3,8)^T$,
\item $\vec{u}_1 = (1,2,1)^T, \vec{u}_2 = (1,4,4)^T, \vec{u}_3 = (2,2,5)^T$, and
\item $\vec{u}_1 = (1,-2,2,1)^T, \vec{u}_2 = (1,1,0,2)^T, \vec{u}_3 = (2,3,-1,0)^T$.
\end{enumerate}  
\end{Exercise}
\chapter{Complex Vectors/Matrices and Block Matrices}
\label{chap:complex}

In this chapter, we will take a detour to talk about two auxiliary topics. The first one is the generalization of vectors and matrices to having complex numbers as entries. Eventually, we will mention about \textit{complex vector spaces}, and compare them to real vector spaces that we just learnt in the previous chapters. The second one is about the \textit{block form} of a matrix (or simply referred to as a \textit{block matrix}) that is composed of smaller \textit{submatrices} as the building blocks. Writing a matrix in block form enables efficient manipulation for many situations that we will encounter in the remaining parts of this book.

\section{Definition and Operations of Complex Numbers}
\label{section:complexno}

\subsection{Basic Structure of Complex Numbers} 

The idea of complex numbers initially came from some algebra problems that led to the square root of a negative quantity, which was undefined back in the day. Later, mathematicians addressed this issue by introducing the \index{Imaginary Number}\keywordhl{imaginary number} $i = \sqrt{-1}$, and $i^2 = -1$. For any positive number $b$, we have $\sqrt{-b^2} = \sqrt{b^2}\sqrt{-1} = bi$. \index{Complex Number}\keywordhl{Complex numbers} are then quantities in the form of $a + bi$, where $a$ and $b$ themselves are real. Here $a$ and $b$ are called the \index{Real Part}\keywordhl{real} and \index{Imaginary Part}\keywordhl{imaginary part} respectively. As a small example of how complex numbers arise, note that the solutions to the quadratic equation $(3x+2)^2 = -1$, are $-\frac{2}{3} \pm \frac{1}{3}i$.
\begin{defn}[Complex Number]
Complex numbers are scalars in the form of $z = a + bi$, where $a$ and $b$ are some real numbers. Their real and imaginary parts are denoted by $\Re{z} = a$ and $\Im{z} = b$.
\end{defn}
We also need to consider when two complex numbers are equal. This happens when their real parts, as well as imaginary parts, are equal to each other respectively.
\begin{proper}
Two complex numbers $z_1 = a + bi$, and $z_2 = c + di$, where $a, b, c, d$ are real numbers, are equal if and only if $\Re{z_1} = a = c = \Re{z_2}$ and $\Im{z_1} = b = d = \Im{z_2}$.
\end{proper}
For every complex number, there exists another corresponding complex number known as the \index{Complex Conjugate}\keywordhl{(complex) conjugate} associated with it, formed by flipping the sign of its imaginary part.
\begin{defn}[Complex Conjugate]
For a complex number $z = a + bi$, its complex conjugate is defined as $\overline{z} = a - bi$.
\end{defn}
For example, the conjugate of $2-5i$ is $2+5i$.

\subsection{Complex Number Operations}
Below are some rules about usual operations on two complex numbers.
\subsubsection{Addition and Subtraction}
Addition and subtraction between two complex numbers are carried out over the real parts and the imaginary parts separately.
\begin{defn}
For two complex numbers $z_1 = a + bi$, and $z_2 = c + di$, we have
\begin{align}
z_1 \pm z_2 &= (a + bi) \pm (c + di) \nonumber \\
&= (a \pm c) + (b \pm d)i \nonumber \\
&= (\Re{z_1} \pm \Re{z_2}) + (\Im{z_1} \pm \Im{z_2})i    
\end{align}
\end{defn}
For instance, adding $1 + 3i$ to $2 - 4i$ results in $(1+2) + (3-4)i = 3 - i$.

\subsubsection{Multiplication and Division}
Multiplication of two complex numbers simply works like the usual distributive law where we pretend that $i$ is a variable.
\begin{defn}
Given two complex numbers $a + bi$, and $c + di$, their product is
\begin{align}
(a + bi)(c + di) &= a(c + di) + bi(c + di) \nonumber\\
&= ac + adi + bci + bdi^2 \nonumber \\
&= (ac - bd) + (ad + bc)i & (i^2 = -1)
\end{align}
\end{defn}

\begin{exmp}
Evaluate $(1+2i)(3-4i)$.
\end{exmp}
\begin{solution}
\begin{align*}
(1+2i)(3-4i) &= ((1)(3) - (2)(-4)) + ((1)(-4) + (2)(3))i \\
&= 11 + 2i \qedhere
\end{align*}
\end{solution}

Dividing something by a complex number $a+bi$ can be viewed as multiplication by its complex conjugate $a-bi$, as
\begin{align}
\frac{1}{a+bi} &= \frac{1}{a+bi}\frac{a-bi}{a-bi} \nonumber \\
&= \frac{a-bi}{a^2 - (-b^2) - abi + bai} \nonumber \\
&= \frac{a-bi}{a^2 + b^2}
\end{align}
with an additional factor of $\frac{1}{a^2+b^2}$. It is interesting that this $a^2+b^2$ term coming from multiplying the complex number by its conjugate over the denominator looks like the square of hypotenuse as in the \textit{Pythagoras' Theorem}. Later on, we will see more when we discuss the geometric meaning of complex numbers.

\begin{exmp}
Compute $\frac{1+4i}{2+3i}$.
\end{exmp}
\begin{solution}
Following the idea outlined above, we have
\begin{align*}
\frac{1+4i}{2+3i} &= \frac{1+4i}{2+3i}\frac{2-3i}{2-3i} \\
&= \frac{(1+4i)(2-3i)}{2^2+3^2} \\
&= \frac{((1)(2) - (4)(-3)) + ((1)(-3) + (4)(2))i}{13} = \frac{14}{13}+\frac{5}{13}i
\end{align*}
\end{solution}

\subsection{Geometric Meaning of Complex Numbers}
\label{section:complexnogeo}

\begin{figure}[ht!]
\centering
\begin{tikzpicture}[scale=0.5]
\draw[->] (-5,0)--(5,0) node[right](x){Real Axis};
\draw[->] (0,-5)--(0,5) node[above]{Imaginary Axis};
\draw[blue,-stealth] (0,0)--(3,4) node[anchor=south](z){$z = 3+4i$};
\draw[gray,dashed] (3,4)--(3,0) node[below]{$\Re{z} = 3$};
\draw[gray,dashed] (3,4)--(0,4) node[left]{$\Im{z} = 4$};
\pic[draw, ->, "$\theta$",angle eccentricity=1.5] {angle = x--0--z};
\node[below left]{$O$}; 
\end{tikzpicture}
\caption{\textit{A complex number $z = 3+4i$ represented in the complex plane.}}
\label{fig:argand}
\end{figure}
A complex number can be visualized as a two-dimensional vector, in the so-called \index{Complex Plane}\keywordhl{complex plane} (or sometimes referred to as the \index{Argand Plane}\keywordhl{Argand plane}), where the $x$-axis represents the real part and the $y$-axis represents the imaginary part. These two axes are referred to as the \index{Real Axis}\keywordhl{real axis} and \index{Imaginary Axis}\keywordhl{imaginary axis} respectively.\par

It is obvious that the length of such a vector is 
\begin{align}
\abs{z} = \sqrt{\Re{z}^2 + \Im{z}^2} \label{eqn:modulus}
\end{align}
which is called the \index{Modulus}\keywordhl{modulus} of the corresponding complex number. In the diagram (Figure \ref{fig:argand}) above, the modulus of $z$ is easily seen to be $\abs{z} = 5$. \par
The angle between the real axis and the complex number is called the \index{Argument}\keywordhl{argument}, shown as
\begin{align}
\theta = \arctan(\Im{z}/\Re{z}) \label{eqn:argument} 
\end{align}
in the same figure. Since its complex conjugate $\overline{z}$ has the sign of the imaginary part flipped while the real part remains the same, the argument of the complex conjugate is simply the negative of that of the original complex number $z$. Also, the modulus will be unchanged. \par
Moreover, from elementary trigonometry, we know that $\Re{z} = \abs{z} \cos{\theta}$ and $\Im{z} = \abs{z} \sin{\theta}$. Hence $z$ can be represented as $z = \Re{z} + i\Im{z} = \abs{z} (\cos \theta + i \sin \theta)$. We also have the famous \index{Euler's Formula}\keywordhl{Euler's Formula}, relating the geometry of any complex number with an exponential raised to an imaginary power.
\begin{defn}[Euler's Formula]
\label{defn:Euler}
An exponential raised to an imaginary power is a complex number such that
\begin{align}
e^{i \theta} = \cos \theta + i \sin \theta \label{eqn:Euler}
\end{align}
where $\theta$ is taken to be real.
\end{defn}
Hence $z$ can be further written as $z = \abs{z} e^{i \theta}$, and $\overline{z} = \Re{z} - i\Im{z} = \abs{z} (\cos \theta - i \sin \theta) = \abs{z} (\cos (-\theta) + i \sin (-\theta)) = \abs{z} e^{-i \theta}$. Conversely, the quantity $e^{i \theta}$ can be regarded as a complex number that has a modulus of $1$ and an argument of $\theta$. Additionally, this provides formulae to express sines and cosines with complex exponentials.
\begin{proper}
\label{proper:sincoscomplex}
For any $\theta$ which is confined to be real,
\begin{subequations}
\label{eqn:sincoscomplex}
\begin{align}
\cos \theta &= \frac{e^{i\theta} + e^{-i\theta}}{2} \label{eqn:sincoscomplexa}\\
\sin \theta &= \frac{e^{i\theta} - e^{-i\theta}}{2i}
\end{align}
\end{subequations}
\end{proper}
\begin{proof}
By Definition \ref{defn:Euler},
\begin{align*}
\frac{e^{i\theta} + e^{-i\theta}}{2} &= \frac{1}{2}((\cos \theta + i \sin \theta) + (\cos (-\theta) + i \sin (-\theta))) \\
&= \frac{1}{2}((\cos \theta + i \sin \theta) + (\cos \theta - i \sin \theta)) \\
&= \cos \theta
\end{align*}  
The derivation for $\sin \theta$ is left as an exercise.
\end{proof}
Now we can go back to investigate complex multiplication and division. Multiplication of a complex number $z_1$ by another complex number $z_2$, can be viewed as $z_1z_2 = (\abs{z_1}e^{i \theta_1}) (\abs{z_2} e^{i \theta_2}) = \abs{z_1}\abs{z_2}e^{i (\theta_1+\theta_2)}$.\footnote{We take it for granted that $e^{i \theta_1}e^{i \theta_2} = e^{i (\theta_1+\theta_2)}$.} This can be interpreted as, starting with the complex number $z_1 = \abs{z_1}e^{i \theta_1}$ on the complex plane, rotating it anti-clockwise (i.e.\ in the positive direction) by an angle of $\theta_2$, and scaling its modulus by a factor of $\abs{z_2}$. \par
Similarly, division of $z_1$ by $z_2$, is $z_1/z_2 = (\abs{z_1}/\abs{z_2})e^{i (\theta_1-\theta_2)}$. Notice that for a fraction like $1/z = 1/(a+bi)$, it can be rewritten as
\begin{align*}
\frac{1}{z} = \frac{1}{\abs{z}e^{i \theta}} = \frac{1}{\abs{z}}\frac{e^{-i \theta}}{e^{i \theta}e^{-i \theta}} = \frac{1}{\abs{z}}\frac{e^{-i \theta}}{e^{i (\theta-\theta)}} = \frac{1}{\abs{z}}\frac{e^{-i \theta}}{e^0} &= \frac{1}{\abs{z}} e^{-i \theta} \\
&= \frac{1}{\abs{z}^2} (\abs{z} e^{-i \theta}) \\
&= \frac{1}{\abs{z}^2} \overline{z}
\end{align*}
which is consistent with the discussion about complex division in the last subsection, where $\abs{z}^2 = a^2 + b^2$ arises in the denominator. In addition, we can observe that $\abs{z}^2 = z\overline{z}$. This is not a coincidence, as
\begin{align}
z\overline{z} &= \abs{z} e^{i \theta} \abs{z} e^{-i \theta} \nonumber \\
&= \abs{z}^2 e^{i(\theta-\theta)} = \abs{z}^2e^{0} = \abs{z}^2 \label{eqn:zzbar}
\end{align}
Geometrically, we can think of it as starting with $1$ along the real axis in the complex plane, then we scale it by $\abs{z}$ and rotate it by $\theta$, and finally scale it again by $\abs{z}$ but rotate it by $-\theta$, the same angle but in opposite direction. The results will be a real number $\abs{z}^2$ (the length of $z$ squared), since the two opposite rotations cancel out each other. \par
Below are some properties of modulus and complex conjugate to be remembered.
\begin{proper}
\label{proper:complexnum}
For two complex numbers $z_1$ and $z_2$, we have
\begin{enumerate}[label=(\alph*)]
\item $\overline{z_1 \pm z_2} = \overline{z_1} \pm \overline{z_2}$;
\item $\overline{z_1z_2} = \overline{z_1}\,\overline{z_2}$;
\item $\overline{z_1/z_2} = \overline{z_1}/\overline{z_2}$;
\item $\overline{\overline{z}} = z$;
\item $\overline{\overline{z_1}z_2} = z_1\overline{z_2}$;
\item $\abs{\overline{z}} = \abs{z}$;
\item $\abs{z_1z_2} = \abs{z_1}\abs{z_2}$;
\item $\abs{z_1/z_2} = \abs{z_1}/\abs{z_2}$.
\end{enumerate}
\end{proper}
Another very useful result is the \index{De Moivre's Formula}\keywordhl{De Moivre's Formula} that builds up on the Euler's formula, expressing $e^{i \theta}$ raised to an integer power $n$.
\begin{thm}[De Moivre's Formula]
Given $n$ as an integer, then
\begin{subequations}
\begin{align}
(e^{i \theta})^n &= e^{i (n\theta)} \\
(\cos\theta + i \sin\theta)^n &= \cos(n\theta) + i \sin(n\theta)
\end{align}
\end{subequations}
\end{thm}

\section{Complex Vectors and Complex Matrices}

Our discussion about vectors and matrices in previous chapters is limited to those with real entries. However, we can extend the ideas to include complex entries. A complex vector is simply a vector that have complex numbers as components. An $n$-dimensional complex vector can be somehow viewed as a real vector that is $2n$-dimensional, as each complex entry can be expressed in two parts, real and imaginary. This equivalence will be further clarified at the end of this section. A complex matrix is similarly a matrix with complex entries, or from another perspective, formed by complex column vectors.

\subsection{Operations and Properties of Complex Vectors}
Addition and subtraction for complex vectors are the same as the real counterpart, carried out component-wise. Multiplication by a scalar is also similar, applied to all components. However, the form of complex dot product is slightly different from the real dot product, as defined below.
\begin{defn}[Complex Dot Product]
\label{defn:complexdotproduct}
The dot product of two complex vectors $\vec{u}$ and $\vec{v}$ is computed as the sum of products between each pair of components, but additionally with the conjugate operation applied on the second complex vector beforehand.
\begin{align}
\vec{u} \cdot \vec{v} &= \textbf{u}^T \overline{\textbf{v}} \nonumber \\
&= u_1\overline{v_1} + u_2\overline{v_2} + \cdots + u_n\overline{v_n} = \sum_{k=1}^{n} u_k\overline{v_k}
\end{align}
The bar on $\overline{\textbf{v}}$ means carrying out conjugate on every entry of $\textbf{v}$. If $\textbf{v} = \Re{\textbf{v}} + i\Im{\textbf{v}}$, where $\Re{\textbf{v}}$ and $\Im{\textbf{v}}$ are the vectors consisting of the real/imaginary parts of every entry in $\textbf{v}$, then $\overline{\textbf{v}} = \Re{\textbf{v}} - i\Im{\textbf{v}}$.
\end{defn}
The Euclidean norm, or length of a complex vector, is defined in a similar fashion.
\begin{defn}
The length $\norm{\vec{v}}$ of a complex vector $\vec{v}$ is calculated by
\begin{align}
\norm{\vec{v}} &= \sqrt{\vec{v} \cdot \vec{v}} = \sqrt{\textbf{v}^T \overline{\textbf{v}}} \nonumber \\
&= \sqrt{v_1\overline{v_1} + v_2\overline{v_2} + \cdots + v_n\overline{v_n}} \nonumber\\
&= \sqrt{\abs{v_1}^2 + \abs{v_2}^2 + \cdots + \abs{v_n}^2} & \text{(by (\ref{eqn:zzbar}))} \nonumber \\
&= \sqrt{\sum_{k=1}^{n} \abs{v_k}^2}
\end{align}
\end{defn}
Properties of complex dot product hence also vary slightly from its real counterpart, Properties \ref{proper:dotproper}.
\begin{proper}
\label{proper:complexdot}
For two complex vectors $\vec{u}$ and $\vec{v}$, we have
\begin{align*}
\vec{u} \cdot \vec{v} &= \textcolor{red}{\overline{\vec{v} \cdot \vec{u}}} &\text{Conjugate-symmetric Property} \\
\vec{u} \cdot (\vec{v} \pm \vec{w}) &= \vec{u} \cdot \vec{v} \pm \vec{u} \cdot \vec{w} &\text{Distributive Property} \\
(\vec{u} \pm \vec{v}) \cdot \vec{w} &= \vec{u} \cdot \vec{w} \pm \vec{v} \cdot \vec{w} &\text{Distributive Property} \\
(a\vec{u}) \cdot (b\vec{v}) &= a\textcolor{red}{\bar{b}}(\vec{u} \cdot \vec{v}) &\text{where $a$, $b$ are some complex constants}    
\end{align*}
\end{proper}
There is no complex analog for cross product.
\begin{exmp}
Show that the conjugate-symmetric property in Properties \ref{proper:complexdot} holds for $\vec{u} = (1+2i, 3+i)^T$, $\vec{v} = (2-5i, 1+4i)^T$.
\end{exmp}
\begin{solution}
\begin{align*}
\vec{u} \cdot \vec{v} &= (1+2i)(\overline{2-5i}) + (3+i)(\overline{1+4i}) \\
&= (1+2i)(2+5i) + (3+i)(1-4i) \\
&= (-8+9i) + (7-11i) \\
&= -1-2i 
\end{align*}
\begin{align*}
\vec{v} \cdot \vec{u} &= (2-5i)(\overline{1+2i}) + (1+4i)(\overline{3+i}) \\
&= (2-5i)(1-2i) + (1+4i)(3-i) \\
&= (-8-9i) + (7+11i) \\
&= -1+2i 
\end{align*}
Hence $\vec{u} \cdot \vec{v} = \overline{\vec{v} \cdot \vec{u}}$.
\end{solution}

$\blacktriangleright$ Short Exercise: Find the norm $\norm{\vec{u}}$ and $\norm{\vec{v}}$ respectively.\footnote{$\norm{\vec{u}} = \sqrt{(1+2i)(1-2i) + (3+i)(3-i)} = \sqrt{(1^2 + 2^2) + (3^2 + 1^2)} = \sqrt{15}$. Similarly, $\norm{\vec{v}} = \sqrt{46}$.}

\subsection{Operations and Properties of Complex Matrices}
Matrix multiplication between two complex matrices is carried out in the same way as we have been always doing, according to Definition \ref{defn:matprod}. However, due to the difference in the definition of dot product for real and complex vectors, we can no longer claim like in the discussion below Definition \ref{defn:dotreal} that the entries resulting from a complex matrix product are complex vector dot products between the corresponding rows and columns. To make the statement work again, a minor modification is needed, as we will see soon. 

\subsubsection{Conjugate Transpose}
Transpose can be similarly defined for complex matrices. However, there exists a more useful operation that combines transpose and conjugate.
\begin{defn}[Conjugate Transpose]
\label{defn:conjutrans}
The \index{Conjugate Transpose}\keywordhl{conjugate transpose} of a matrix $A$, denoted as $A^* = \overline{A^T}$, has its entries as $A^*_{pq} = \overline{A}_{qp}$, where $\overline{A}$ is the conjugate of the matrix $A$ produced by changing every entry in $A$ to its complex conjugate. Sometimes $A^*$ is called the \textit{adjoint} or \index{Hermitian Transpose}\keywordhl{Hermitian transpose} of $A$, and alternatively denoted as $A^H$. 
\end{defn}
It means that conjugate transpose is done by conjugating all entries of the matrix and flipping them about its main diagonal. A \index{Hermitian Matrix}\keywordhl{Hermitian matrix} is a complex matrix whose conjugate transpose equals to itself.
\begin{defn}
\label{defn:Hermitian}
A complex square matrix $A$ is called Hermitian if $A^* = A$.
\end{defn}
Note that the diagonal entries of a Hermitian matrix are always real. Properties of conjugate transpose are alike to those for real transpose stated in Properties \ref{proper:transp}. Furthermore, for complex dot product, there is also something parallel to the second half of Properties \ref{proper:dotproper}.
\begin{proper}
\label{proper:complexmat}
For two complex matrices $A$ and $B$, we have
\begin{enumerate}
\item $(cA)^* = \overline{c}A^*$, where $c$ is any complex scalar;
\item $(A^*)^* = A$;
\item $(A \pm B)^* = A^* \pm B^*$, if $A$ and $B$ have the same shape;
\item $(AB)^* = B^*A^*$, if $A$ and $B$ have compatible shapes.
\end{enumerate}
\end{proper}
\begin{proper}
\label{proper:complexdotherm}
For two complex vectors $\vec{u}$ and $\vec{v}$ and a complex matrix $A$, we have 
\begin{align*}
\vec{u} \cdot (A\vec{v}) &= \textbf{u}^T\overline{(A\textbf{v})} = (\overline{A^T}\textbf{u})^T\overline{\textbf{v}} = (A^*\vec{u}) \cdot \vec{v} \\
(A\vec{u}) \cdot \vec{v} &= (A\textbf{u})^T\overline{\textbf{v}} = \textbf{u}^T(A^T\overline{\textbf{v}}) = \vec{u} \cdot (A^*\vec{v})
\end{align*}
where Properties \ref{proper:transp} and Definitions \ref{defn:complexdotproduct}, \ref{defn:conjutrans} are used.
\end{proper}

With the complex conjugate of a matrix defined in passing, we can now say that the complex vector dot products between each of the row and column vectors in a matrix $A$ and another matrix $B$ respectively, are encoded in the entries of the complex matrix product $A\overline{B}$ where a conjugate is applied on the second matrix.
\begin{exmp}
For two complex matrices
\begin{align*}
& A =
\begin{bmatrix}
1 & i \\
-i & 0
\end{bmatrix} 
& B =
\begin{bmatrix}
1 & 2+3i \\
1+i & 1-i
\end{bmatrix}
\end{align*}
Verify that $(AB)^* = B^*A^*$.
\end{exmp}
\begin{solution}
\begin{align*}
A^* &=
\begin{bmatrix}
1 & i \\
-i & 0
\end{bmatrix} \\
B^* &=
\begin{bmatrix}
1 & 1-i \\
2-3i & 1+i
\end{bmatrix} \\
B^*A^* &= 
\begin{bmatrix}
1 & 1-i \\
2-3i & 1+i
\end{bmatrix} 
\begin{bmatrix}
1 & i \\
-i & 0
\end{bmatrix} \\
&=
\begin{bmatrix}
(1)(1) + (1-i)(-i) & (1)(i) + (1-i)(0)  \\
(2-3i)(1) + (1+i)(-i) & (2-3i)(i) + (1+i)(0)
\end{bmatrix} \\
&=
\begin{bmatrix}
-i & i \\
3-4i & 3+2i
\end{bmatrix} 
\end{align*}
\begin{align*}
AB &= 
\begin{bmatrix}
1 & i \\
-i & 0
\end{bmatrix} 
\begin{bmatrix}
1 & 2+3i \\
1+i & 1-i
\end{bmatrix} \\
&= 
\begin{bmatrix}
(1)(1)+(i)(1+i) & (1)(2+3i) + (i)(1-i) \\
(-i)(1)+(0)(1+i) & (-i)(2+3i) + (0)(1-i)
\end{bmatrix} \\
&= 
\begin{bmatrix}
i & 3+4i \\
-i & 3-2i
\end{bmatrix} \\
(AB)^* &= 
\begin{bmatrix}
-i & i \\
3-4i & 3+2i
\end{bmatrix} 
\end{align*}
\end{solution}

\subsubsection{Determinants and Inverses for complex matrices}
Complex matrices also have determinants and inverses, and are calculated in the exact same ways outlined in Sections \ref{section:det} and \ref{section:inv}. We provide a few examples here.

\begin{exmp}
Calculate the determinant for
\begin{align*}
A = 
\begin{bmatrix}
1-i & 3 & 2 \\
1+i & 0 & i \\
2 & -2i & 1
\end{bmatrix}
\end{align*}
\end{exmp}
\begin{solution}
We apply cofactor expansion along the middle row in the way outlined in Properties \ref{proper:cofactorex}, and the result is
\begin{align*}
\det(A) &= -(1+i)
\begin{vmatrix}
3 & 2 \\
-2i & 1
\end{vmatrix}
+ (0)
\begin{vmatrix}
1-i & 2 \\
2 & 1
\end{vmatrix}
- (i)
\begin{vmatrix}
1-i & 3 \\
2 & -2i
\end{vmatrix} \\
&= -(1+i)(3+4i) - (i)(-8-2i) \\
&= -1 + i
\end{align*}  
\end{solution}

\begin{exmp}
Find the inverse of the matrix $A$ in the last example.
\end{exmp}

\begin{solution}
The computation of the inverse follows Properties \ref{proper:invadj}. First, we note that
\begin{align*}
\frac{1}{\det(A)} &= \frac{1}{-1+i} \\
&= \frac{1}{-1+i} \frac{-1-i}{-1-i} \\
&= \frac{-1-i}{1+1} = -\frac{1+i}{2}
\end{align*}
Then, we proceed to compute the cofactor matrix for $A$, which is
\begin{align*}
C &=
\begin{bmatrix*}[r]
\begin{vmatrix}
0 & i \\
-2i & 1
\end{vmatrix} &
-\begin{vmatrix}
1+i & i \\
2 & 1
\end{vmatrix} &
\begin{vmatrix}
1+i & 0 \\
2 & -2i
\end{vmatrix} \\[10pt]
-\begin{vmatrix}
3 & 2 \\
-2i & 1
\end{vmatrix} &
\begin{vmatrix}
1-i & 2 \\
2 & 1
\end{vmatrix} &
-\begin{vmatrix}
1-i & 3 \\
2 & -2i
\end{vmatrix} \\[10pt]
\begin{vmatrix}
3 & 2 \\
0 & i
\end{vmatrix} &
-\begin{vmatrix}
1-i & 2 \\
1+i & i
\end{vmatrix} &
\begin{vmatrix}
1-i & 3 \\
1+i & 0
\end{vmatrix}
\end{bmatrix*} \\
&= 
\begin{bmatrix}
-2 & -1+i & 2-2i \\
-3-4i & -3-i & 8+2i \\
3i & 1+i & -3-3i
\end{bmatrix}
\end{align*}
Thus, by Properties \ref{proper:invadj}, the inverse of $A$ is
\begin{align*}
A^{-1} &= \frac{1}{\det(A)} \text{adj}(A) = \frac{1}{\det(A)} C^T \\
&= -\frac{1+i}{2} 
\begin{bmatrix}
-2 & -3-4i & 3i \\
-1+i & -3-i & 1+i \\
2-2i & 8+2i & -3-3i
\end{bmatrix} \\
&= 
\begin{bmatrix}
1+i & -\frac{1}{2}+\frac{7}{2}i & \frac{3}{2}-\frac{3}{2}i \\
1 & 1+2i & -i \\
-2 & -3-5i & 3i
\end{bmatrix} 
\end{align*}
\end{solution}

$\blacktriangleright$ Short Exercise: Find $A^{-1}$ via Gaussian Elimination.\footnote{Notice that we will now need to multiply rows with complex constants instead when doing elementary row operations. You should be able to get the same answer. Possible first few steps are multiplying the first row by $\frac{1+i}{2}$ to create a leading $1$ and subtracting $1+i$ and $2$ times the new first row from the second and third row respectively.}

Below are some useful properties of determinants and inverses for complex matrices that can be compared to Properties \ref{proper:properdet} and \ref{proper:inverse}.
\begin{proper}
If $A$ is a complex matrix, then
\begin{enumerate}
\item $\text{det}(A^T) = \text{det}(A)$;
\item $\text{det}(A^*) = \overline{\text{det}(A)}$;
\item $\text{det}(kA) = k^n \text{det}(A)$, for any complex constant $k$;
\item $\text{det}(AB) = \text{det}(A)\text{det}(B)$; and
\item $\text{det}(A^{-1}) = \frac{1}{\text{det}(A)}$, given $A$ is invertible ($\det(A) \neq 0$ as the same as before).
\end{enumerate}
Additionally, if $A$ is non-singular, then
\begin{enumerate}
\item $(cA)^{-1} = \frac{1}{c}A^{-1}$, for any complex scalar $c \neq 0$;
\item $(A^{-1})^{-1} = A$;
\item $(A^n)^{-1} = (A^{-1})^n$, for any positive integer $n$;
\item $(AB)^{-1} = B^{-1}A^{-1}$, provided that $B$ is invertible too, plus $A$ and $B$ are conformable;
\item $(A^T)^{-1} = (A^{-1})^T$;
\item $(A^*)^{-1} = (A^{-1})^*$.
\end{enumerate}
\end{proper}

\subsection{The Complex $n$-space $\mathbb{C}^n$}

Similar to the real $n$-space $\mathbb{R}^n$ brought up in Definition \ref{defn:real_nspace}, the set of all complex vectors, now with $n$ complex components, forms the \index{Complex $n$-space}\keywordhl{complex $n$-space} $\mathbb{C}^n$ as follows.
\begin{defn}[The Complex $n$-space $\mathbb{C}^n$]
\label{defn:complex_nspace}
The complex $n$-space $\mathbb{C}^n$ is defined as the set of all possible $n$-tuples in the form of $\vec{v} = (v_1, v_2, v_3, \ldots, v_n)^T$, where $v_i$ can be any \textit{complex} numbers, for $i = 1,2,3,\ldots,n$. They are known as $n$-dimensional complex vectors.
\end{defn}
A very interesting (and perhaps quite confusing) fact about the complex $n$-space $\mathbb{C}^n$, or an $n$-dimensional complex vector, is that it can be considered as $2n$-dimensional when put in the frame of a real vector space. The key lies in Definition \ref{defn:realvecspaceaxiom}, where if the underlying scalar is set to $\mathbb{R}$ or $\mathbb{C}$ so that it becomes a real/complex vector space. Notice the subtle difference between a real/complex vector (that is indicative of its components being real/complex) and real/complex vector space (concerning \textit{the underlying scalar used in scalar multiplication)}. We take $\mathbb{C}$ as a vector space here for illustration. If $\mathbb{C}$ is treated as a complex vector space, i.e.\ \textit{over} $\mathbb{C}$ itself, then $\{1\}$ is a basis for $\mathbb{C}$ since the scalar multiplication of $1$ by any arbitrary \textit{complex scalar} can generate all complex numbers as the scalar itself. Hence, the dimension of $\mathbb{C}$ is $1$ over $\mathbb{C}$ (Properties \ref{proper:samenvecsbases} still holds for complex vector spaces). Otherwise, if $\mathbb{C}$ is taken as a real vector space (\textit{over} $\mathbb{R}$), then $\{1\}$ is not sufficient to be a basis for $\mathbb{C}$ since multiplication of $1$ by only any \textit{real scalar} $a$ can never produce complex numbers with a non-zero imaginary part. On the other hand, $\{1, i\}$ can instead be a basis for $\mathbb{C}$ over $\mathbb{R}$ as linear combinations of $1$ and $i$ with purely real coefficients can produce all complex numbers. So by Properties \ref{proper:samenvecsbases}, the dimension of $\mathbb{C}$ over $\mathbb{R}$ is $2$, and with Theorem \ref{thm:isomorphism}, it is isomorphic to $\mathbb{R}^2$ in this situation. An explicit isomorphism between $\mathbb{C}$ and $\mathbb{R}^2$ over $\mathbb{R}$ is simply
\begin{align*}
T(a+bi) = (a,b)^T
\end{align*}
Extending this observation, $\mathbb{C}^n$ can either be treated as $n$-dimensional over $\mathbb{C}$ or $2n$-dimensional over $\mathbb{R}$ (and is isomorphic to $\mathbb{R}^{2n}$). However, unless mentioned otherwise, we consider any $\mathbb{C}^n$ vector (or complex matrix) to be taken over $\mathbb{C}$ (the former case) onwards. All major results from the last two chapters are still valid if we replace $\mathbb{R}$ and $\mathbb{R}^n$ by $\mathbb{C}$ and $\mathbb{C}^n$ appropriately.\footnote{For example, a linear combination of complex vectors (Definition \ref{defn:linearcomb}) $\vec{v}^{(j)} \in \mathbb{C}^n$ is still in the form of $c_1\vec{v}^{(1)} + c_2\vec{v}^{(2)} + c_3\vec{v}^{(3)} + \cdots + c_q\vec{v}^{(q)}$ but the coefficients $c_j \in \mathbb{C}$ are complex numbers now.} In particular, we note that
\begin{proper}
The angle between two complex vectors $\vec{u}$ and $\vec{v}$ is found according to
\begin{align}
\cos \theta = \frac{\Re{\vec{u} \cdot \vec{v}}}{\norm{u}\norm{v}}
\end{align}
\end{proper}
which should be compared to (\ref{eqn:dotgeo}).

\section{Manipulating Block Matrices}

Moving to our second topic, a \keywordhl{block matrix} is a matrix written in smaller \textit{submatrices} as if they are ordinary entries. For example, a $2 \times 2$ block matrix has the form of
\begin{align*}
M =
\begin{bmatrix}
A & B \\
C & D
\end{bmatrix}
\end{align*}
where $A$, $B$, $C$, $D$ are themselves matrices having the shapes of $m \times p$, $m \times q$, $n \times p$, $n \times q$, and $m, n, p, q$ can be any positive integer. As a more concrete example, we have
\begin{align*}
M = 
\left[\begin{array}{@{\,}wc{10pt}wc{10pt}wc{10pt}|wc{10pt}wc{10pt}}
1 & 0 & 3 & 0 & 4 \\
0 & 1 & 1 & 2 & -1 \\
\hline
2 & -1 & 0 & 1 & -2 \\
\end{array}\right]
\end{align*}
being a $3 \times 5$ matrix at the same time a $2 \times 2$ block matrix where
\begin{align*}
A &= \begin{bmatrix}
1 & 0 & 3 \\
0 & 1 & 1
\end{bmatrix}
& 
B &= \begin{bmatrix}
0 & 4 \\
2 & -1
\end{bmatrix} \\
C &= \begin{bmatrix}
2 & -1 & 0 \\
\end{bmatrix}
& 
D &= \begin{bmatrix}
1 & -2
\end{bmatrix}
\end{align*}
are of the shapes $2 \times 3$, $2 \times 2$, $1 \times 3$ and $1 \times 2$. We can extend this for block matrices of any partition. For instance, a $3 \times 4$ block matrix will be in the form of
\begin{align*}
M =
\begin{bmatrix}
M_{11} & M_{12} & M_{13} & M_{14} \\
M_{21} & M_{22} & M_{23} & M_{24} \\
M_{31} & M_{32} & M_{33} & M_{34} \\
\end{bmatrix}
\end{align*}
where the $M_{ij}$ are submatrices, and for a fixed $i$ [$j$], $M_{ij}$ has the same number of rows [columns].

\subsection{Block Matrix Multiplication}
\label{subsection:blockmul}

With the structure of a block matrix explained, we can now examine how matrix multiplication between two block matrices is done. Let's take a look at the easiest case of two $2 \times 2$ block matrices:
\begin{align*}
M &=
\begin{bmatrix}
A & B \\
C & D
\end{bmatrix} 
& &
N =
\begin{bmatrix}
X & Y \\
Z & W
\end{bmatrix} 
\end{align*}
Of course, from the very beginning (Section \ref{section:matrixdefn}), we know that $M$ and $N$ themselves have to be of the shapes $m \times r$ and $r \times n$ as ordinary matrices, but how about the submatrices? In fact, we just carry out the multiplication as if each of them is a single entry, such that
\begin{align}
MN = 
\begin{bmatrix}
A & B \\
C & D
\end{bmatrix} 
\begin{bmatrix}
X & Y \\
Z & W
\end{bmatrix} 
=
\begin{bmatrix}
AX + BZ & AY + BW \\
CX + DZ & CY + DW
\end{bmatrix}
\label{eqn:22blockmul}
\end{align}
Then, for each resulting block to be valid, the number of columns in $A$ and $C$ [$B$ and $D$] must be the same as that of rows in $X$ and $Y$ [$Z$ and $W$]. So that $A$ and $C$ will have the shapes of $m_1 \times r_1$, $m_2 \times r_1$, $X$ and $Y$ will have the shapes of $r_1 \times n_1$, $r_1 \times n_2$, where $m_1 + m_2 = m$, $n_1 + n_2 = n$. Similarly, $B$ and $D$ need to have the shapes of $m_1 \times r_2$, $m_2 \times r_2$, $Z$ and $W$ need to have the shapes of $r_2 \times n_1$, $r_2 \times n_2$, and $r = r_1 + r_2$. In short, the position of vertical cuts along the column direction of $M$ must coincide with that of horizontal cuts along the row direction of $N$. Below is a walk-through example.

\begin{exmp}
Given 
\begin{align*}
M &=
\begin{bmatrix}
A & B \\
C & D
\end{bmatrix} 
& &
N =
\begin{bmatrix}
X & Y \\
Z & W
\end{bmatrix} 
\end{align*}
as a $3 \times 3$ and $3 \times 2$ matrix respectively, with
\begin{align*}
A &= 
\begin{bmatrix}
1 & 2 \\
0 & 1
\end{bmatrix}
& 
B &=
\begin{bmatrix}
1 \\
-2
\end{bmatrix} \\
C &= 
\begin{bmatrix}
0 & -1 
\end{bmatrix}
&
D &=
\begin{bmatrix}
1
\end{bmatrix} \\
X &= 
\begin{bmatrix}
0 \\
2 
\end{bmatrix}
&
Y &=
\begin{bmatrix}
1 \\
-1
\end{bmatrix} \\
Z &= 
\begin{bmatrix}
-1
\end{bmatrix}
&
W &=
\begin{bmatrix}
1
\end{bmatrix} 
\end{align*}
such that the partitions look like
\begin{align*}
M &=
\left[\begin{array}{@{\,}wc{10pt}wc{10pt}|wc{10pt}}
1 & 2 & 1 \\
0 & 1 & -2 \\
\hline
0 & -1 & 1
\end{array}\right]
& &
N =
\left[\begin{array}{@{\,}wc{10pt}|wc{10pt}}
0 & 1 \\
2 & -1 \\
\hline
-1 & 1
\end{array}\right]   
\end{align*}
Use block matrix multiplication to compute $MN$.
\end{exmp}
\begin{solution}
Note that the cuts along the column/row direction in $M$ and $N$ are both located in-between the $2$nd-$3$rd index. Consequentially, we can use Formula (\ref{eqn:22blockmul}) above:
\begin{align*}
MN = 
\begin{bmatrix}
A & B \\
C & D
\end{bmatrix} 
\begin{bmatrix}
X & Y \\
Z & W
\end{bmatrix} 
=
\begin{bmatrix}
AX + BZ & AY + BW \\
CX + DZ & CY + DW
\end{bmatrix}
\end{align*}
which requires us to compute
\begin{align*}
AX &=
\begin{bmatrix}
1 & 2 \\
0 & 1
\end{bmatrix}
\begin{bmatrix}
0 \\
2 
\end{bmatrix}
=
\begin{bmatrix}
4 \\
2
\end{bmatrix}
&
BZ &=
\begin{bmatrix}
1 \\
-2
\end{bmatrix}
\begin{bmatrix}
-1
\end{bmatrix}
=
\begin{bmatrix}
-1 \\
2
\end{bmatrix} \\
AY &=
\begin{bmatrix}
1 & 2 \\
0 & 1
\end{bmatrix}
\begin{bmatrix}
1 \\
-1
\end{bmatrix}
=
\begin{bmatrix}
-1 \\
-1
\end{bmatrix}
&
BW &=
\begin{bmatrix}
1 \\
-2
\end{bmatrix}
\begin{bmatrix}
1
\end{bmatrix} 
=
\begin{bmatrix}
1 \\
-2
\end{bmatrix} \\
CX &=
\begin{bmatrix}
0 & -1 
\end{bmatrix}
\begin{bmatrix}
0 \\
2 
\end{bmatrix}
=
\begin{bmatrix}
-2
\end{bmatrix}
&
DZ &=
\begin{bmatrix}
1
\end{bmatrix}
\begin{bmatrix}
-1
\end{bmatrix}
=
\begin{bmatrix}
-1
\end{bmatrix} \\
CY &=
\begin{bmatrix}
0 & -1 
\end{bmatrix}
\begin{bmatrix}
1 \\
-1
\end{bmatrix}
=
\begin{bmatrix}
1
\end{bmatrix}
&
DW &=
\begin{bmatrix}
1
\end{bmatrix}
\begin{bmatrix}
1
\end{bmatrix} 
=
\begin{bmatrix}
1
\end{bmatrix}
\end{align*}
Hence
\begin{align*}
MN &= \begin{bmatrix}
AX + BZ & AY + BW \\
CX + DZ & CY + DW
\end{bmatrix} \\
&=
\left[\begin{array}{@{\,}wc{55pt}|wc{55pt}}
\begin{bmatrix}
4 \\
2
\end{bmatrix}+
\begin{bmatrix}
-1 \\
2
\end{bmatrix}
&
\begin{bmatrix}
-1 \\
-1
\end{bmatrix}+
\begin{bmatrix}
1 \\
-2
\end{bmatrix}\\[9pt]
\hline
\begin{bmatrix}
-2
\end{bmatrix}+
\begin{bmatrix}
-1 
\end{bmatrix}
&
\begin{bmatrix}
1
\end{bmatrix}+
\begin{bmatrix}
1 
\end{bmatrix} \Tstrut
\end{array}\right]
=
\left[\begin{array}{@{\,}wc{10pt}|wc{10pt}}
3 & 0 \\
4 & -3 \\
\hline
-3 & 2
\end{array}\right]
\end{align*}
The readers can check the answer by computing the matrix product in the usual way.
\end{solution}

For multiplication involving block matrices with more blocks, the two block matrices $M$ and $N$ must have a partition of $m \times r$ and $r \times n$ blocks, and the block multiplication is carried out as if they are individual entries in usual matrix multiplication as well. Particularly, the positions where the $r$ column [row] partition of $M$ [$N$] occurs must align exactly. Given
\begin{align*}
M &=
\small\begin{bmatrix}
M_{11} & M_{12} & M_{13} & \cdots & M_{1r} \\
M_{21} & M_{22} & M_{23} &  & M_{2r} \\
M_{31} & M_{32} & M_{33} &  & M_{3r} \\
\vdots & & & \ddots & \vdots \\
M_{m1} & M_{m2} & M_{m3} &  & M_{mr} 
\end{bmatrix} & \text{ and } & &
N &=
\small\begin{bmatrix}
N_{11} & N_{12} & N_{13} & \cdots & N_{1n} \\
N_{21} & N_{22} & N_{23} &  & N_{2n} \\
N_{31} & N_{32} & N_{33} &  & N_{3n} \\
\vdots & & & \ddots & \vdots \\
N_{r1} & N_{r2} & N_{r3} &  & N_{rn} 
\end{bmatrix}
\end{align*}
this means that the numbers of columns and rows in $M_{ik}$ and $N_{kj}$ for any fixed $k$ should be equal, such that $M_{ik}$ and $N_{kj}$ are of the shapes $m_i \times r_k$ and $r_k \times n_j$. 

\subsection{Inverse and Determinant of a Block Matrix}
To properly utilize block matrices, we also need to know how to compute some basic quantities related to them, like inverse and determinant. Since most of the situations involve $2 \times 2$ block matrices only, we will handle them exclusively. Specifically, we consider $2 \times 2$ block matrices in the form of
\begin{align*}
M = \begin{bmatrix}
A & B \\
C & D
\end{bmatrix}
\end{align*}
where $A$, $B$, $C$, $D$ are submatrices of the shapes $p \times p$, $p \times q$, $q \times p$ and $q \times q$, such that $A$, $D$ and thus $M$ are square. To proceed, we need the following observations.
\begin{proper}
\label{proper:blockmatinv}
Denote the $p \times p$ and $q \times q$ identity matrices by $I_p$ and $I_q$. Then the matrix
\begin{align*}
\begin{bmatrix}
I_p & [\textbf{0}]_{p \times q} \\
-C & I_q
\end{bmatrix}
\end{align*}
is invertible, particularly having a determinant of $1$. If furthermore, $A$ is invertible, then
\begin{align*}
\begin{bmatrix}
A^{-1} & [\textbf{0}]_{p\times q} \\
[\textbf{0}]_{q\times p} & I_q
\end{bmatrix}
\end{align*}
is also invertible with a determinant of $\det(A^{-1}) = (\det(A))^{-1}$.
\end{proper}
\begin{proof}
For the first matrix, simply note that it is a lower-triangular matrix with all diagonal entries being $1$, and hence it has a determinant of $1$. Therefore, by Properties \ref{proper:invnonzerodet}, it is invertible. Similarly, by repeated cofactor expansions along the bottommost row for $q$ times, the determinant of the second matrix can be seen to be $\det(A^{-1}) = (\det(A))^{-1}$ (Properties \ref{proper:properdet}). If $A$ is invertible, then $\det(A^{-1}) = (\det(A))^{-1}$ is nonzero by Properties \ref{proper:invnonzerodet} again, and 
\begin{align*}
\begin{bmatrix}
A^{-1} & [\textbf{0}]_{p\times q} \\
[\textbf{0}]_{q\times p} & I_q
\end{bmatrix}    
\end{align*}
will also be invertible.
\end{proof}
The above properties imply that these two matrices are the results from elementary row operations (Properties \ref{proper:invseqelement}), and therefore, their product
\begin{align*}
&\quad \begin{bmatrix}
I_p & [\textbf{0}]_{p \times q} \\
-C & I_q
\end{bmatrix}
\begin{bmatrix}
A^{-1} & [\textbf{0}]_{p\times q} \\
[\textbf{0}]_{q\times p} & I_q
\end{bmatrix} \\
&=
\begin{bmatrix}
I_pA^{-1} + [\textbf{0}]_{p\times q}[\textbf{0}]_{q\times p} & I_p[\textbf{0}]_{p\times q} + [\textbf{0}]_{p\times q}I_q \\
(-C)A^{-1} + I_q[\textbf{0}]_{q \times p} & -C[\textbf{0}]_{p \times q} + I_qI_q
\end{bmatrix} \\
&=
\begin{bmatrix}
A^{-1} & [\textbf{0}]_{p\times q} \\
-CA^{-1} & I_q
\end{bmatrix}
\end{align*}
can also be arrived via elementary row operations and is invertible as well (Properties \ref{proper:ABinv}). By multiplying this matrix to $M$, we have
\begin{align*}
\begin{bmatrix}
A^{-1} & [\textbf{0}]_{p\times q} \\
-CA^{-1} & I_q
\end{bmatrix}
\begin{bmatrix}
A & B \\
C & D
\end{bmatrix} &= 
\begin{bmatrix}
A^{-1}A + [\textbf{0}]_{p\times q}C & A^{-1}B + [\textbf{0}]_{p \times q}D \\
-CA^{-1}A + I_qC & - CA^{-1}B + I_qD  
\end{bmatrix} \\
&= 
\begin{bmatrix}
I_p & A^{-1}B \\
-C + C & - CA^{-1}B + D  
\end{bmatrix} \\
&=
\begin{bmatrix}
I_p & A^{-1}B \\
[\textbf{0}]_{q\times p} & D - CA^{-1}B 
\end{bmatrix}
\end{align*}
The bottom right block, $D - CA^{-1}B$, is known as the \index{Schur complement}\keywordhl{Schur complement} of block $A$ in $M$, denoted as $M/A$ and has the same shape $q \times q$ as $D$. The above block multiplication then constitutes a \textit{block Gaussian Elimination} over the matrix $M$ to make it \textit{block upper-triangular}. It is not hard to see that
\begin{align*}
\begin{bmatrix}
I_p & A^{-1}B \\
[\textbf{0}]_{q\times p} & D - CA^{-1}B 
\end{bmatrix}
\begin{bmatrix}
I_p & -A^{-1}B \\
[\textbf{0}]_{q \times p} & I_q
\end{bmatrix}
=
\begin{bmatrix}
I_p & [\textbf{0}]_{p \times q} \\
[\textbf{0}]_{q\times p} & D - CA^{-1}B 
\end{bmatrix}
\end{align*}
Therefore,
\begin{align*}
&\quad \begin{bmatrix}
A^{-1} & [\textbf{0}]_{p\times q} \\
-CA^{-1} & I_q
\end{bmatrix}
\begin{bmatrix}
A & B \\
C & D
\end{bmatrix}
\begin{bmatrix}
I_p & -A^{-1}B \\
[\textbf{0}]_{q \times p} & I_q
\end{bmatrix} \\
&= \begin{bmatrix}
I_p & A^{-1}B \\
[\textbf{0}]_{q\times p} & D - CA^{-1}B 
\end{bmatrix}
\begin{bmatrix}
I_p & -A^{-1}B \\
[\textbf{0}]_{q \times p} & I_q
\end{bmatrix}
=
\begin{bmatrix}
I_p & [\textbf{0}]_{p \times q} \\
[\textbf{0}]_{q\times p} & D - CA^{-1}B 
\end{bmatrix}
\end{align*}
According to the above equation, if the Schur complement $M/A = D-CA^{-1}B$ is also invertible, then the inverse of $M$ will exist, because
\begin{align*}
\small
\begin{bmatrix}
A & B \\
C & D
\end{bmatrix}
&= 
\small
\left(\begin{bmatrix}
A^{-1} & [\textbf{0}]_{p\times q} \\
-CA^{-1} & I_q
\end{bmatrix}\right)^{-1}
\begin{bmatrix}
I_p & [\textbf{0}]_{p \times q} \\
[\textbf{0}]_{q\times p} & D - CA^{-1}B 
\end{bmatrix}
\left(\begin{bmatrix}
I_p & -A^{-1}B \\
[\textbf{0}]_{q \times p} & I_q
\end{bmatrix}\right)^{-1}
\end{align*}
where the three matrices on R.H.S. are all invertible.\footnote{The invertibility of the first and last matrix follows the same arguments in Properties \ref{proper:blockmatinv}, while for the matrix in the middle, we have required that $D-CA^{-1}B$ has to be invertible, and its inverse can be readily seen to be
\begin{align*}
\begin{bmatrix}
I_p & [\textbf{0}]_{p \times q} \\
[\textbf{0}]_{q\times p} & D - CA^{-1}B 
\end{bmatrix}^{-1}
=
\begin{bmatrix}
I_p & [\textbf{0}]_{p \times q} \\
[\textbf{0}]_{q\times p} & (D - CA^{-1}B)^{-1}
\end{bmatrix}
\end{align*}} By Properties \ref{proper:inverse}, we arrive at
\begin{align*}
M^{-1} &=
\begin{bmatrix}
A & B \\
C & D
\end{bmatrix}^{-1} \\
&= \left(\begin{bmatrix}
A^{-1} & [\textbf{0}]_{p\times q} \\
-CA^{-1} & I_q
\end{bmatrix}^{-1}
\begin{bmatrix}
I_p & [\textbf{0}]_{p \times q} \\
[\textbf{0}]_{q\times p} & D - CA^{-1}B 
\end{bmatrix}
\begin{bmatrix}
I_p & -A^{-1}B \\
[\textbf{0}]_{q \times p} & I_q
\end{bmatrix}^{-1} \right)^{-1} \\
&=
\begin{bmatrix}
I_p & -A^{-1}B \\
[\textbf{0}]_{q \times p} & I_q
\end{bmatrix} 
\begin{bmatrix}
I_p & [\textbf{0}]_{p \times q} \\
[\textbf{0}]_{q\times p} & (D - CA^{-1}B)^{-1}
\end{bmatrix}
\begin{bmatrix}
A^{-1} & [\textbf{0}]_{p\times q} \\
-CA^{-1} & I_q
\end{bmatrix} \\
&=
\begin{bmatrix}
I_p & -A^{-1}B (D - CA^{-1}B)^{-1} \\
[\textbf{0}]_{q \times p} & (D - CA^{-1}B)^{-1}
\end{bmatrix}
\begin{bmatrix}
A^{-1} & [\textbf{0}]_{p\times q} \\
-CA^{-1} & I_q
\end{bmatrix} \\
&= \begin{bmatrix}
A^{-1} + A^{-1}B (D - CA^{-1}B)^{-1} CA^{-1} & -A^{-1}B (D - CA^{-1}B)^{-1} \\
-(D - CA^{-1}B)^{-1}CA^{-1} & (D - CA^{-1}B)^{-1}
\end{bmatrix} \\
&= \begin{bmatrix}
A^{-1} + A^{-1}B (M/A)^{-1} CA^{-1} & -A^{-1}B (M/A)^{-1} \\
-(M/A)^{-1}CA^{-1} & (M/A)^{-1}
\end{bmatrix}
\end{align*}
To summarize, we have the following statements.
\begin{proper}
\label{proper:schurinv}
For the $2 \times 2$ block matrix
\begin{align*}
M = 
\begin{bmatrix}
A & B \\
C & D
\end{bmatrix}
\end{align*}
where $A$ and $D$ are square submatrices, if $A$ and its Schur complement $M/A = D - CA^{-1}B$ of block $A$ are both invertible, then $M$ is invertible with
\begin{align}
M^{-1} = \begin{bmatrix}
A^{-1} + A^{-1}B (M/A)^{-1} CA^{-1} & -A^{-1}B (M/A)^{-1} \\
-(M/A)^{-1}CA^{-1} & (M/A)^{-1}
\end{bmatrix}
\label{eqn:schurinv1}
\end{align}
\end{proper}
\begin{proper}
\label{proper:schurdet}
The determinant of the $2 \times 2$ block matrix in Properties \ref{proper:schurinv} is
\begin{align}
\det(M) = \det(A)\det(D-CA^{-1}B) = \det(A)\det(M/A)
\end{align}
if $A^{-1}$ is well-defined.
\end{proper}
\begin{proof}
From the derivation above, we have
\begin{align*}
\begin{bmatrix}
A^{-1} & [\textbf{0}]_{p\times q} \\
-CA^{-1} & I_q
\end{bmatrix}
\begin{bmatrix}
A & B \\
C & D
\end{bmatrix}
\begin{bmatrix}
I_p & -A^{-1}B \\
[\textbf{0}]_{q \times p} & I_q
\end{bmatrix} 
=
\begin{bmatrix}
I_p & [\textbf{0}]_{p \times q} \\
[\textbf{0}]_{q\times p} & D - CA^{-1}B 
\end{bmatrix}
\end{align*}
Evaluating the determinants of both sides (Properties \ref{proper:properdet}) leads to
\begin{align*}
&\quad \det(\begin{bmatrix}
A^{-1} & [\textbf{0}]_{p\times q} \\
-CA^{-1} & I_q
\end{bmatrix})
\det(M)
\det(\begin{bmatrix}
I_p & -A^{-1}B \\
[\textbf{0}]_{q \times p} & I_q
\end{bmatrix}) \\
&=
\det(\begin{bmatrix}
I_p & [\textbf{0}]_{p \times q} \\
[\textbf{0}]_{q\times p} & D - CA^{-1}B 
\end{bmatrix})
\end{align*}
In the same vein of Properties \ref{proper:blockmatinv}, we then have
\begin{align*}
(\det(A))^{-1}\det(M)(1) &= \det(D-CA^{-1}B) \\
\det(M) &= \det(A)\det(D-CA^{-1}B) = \det(A)\det(M/A)
\end{align*}
\end{proof}

\begin{exmp}
Use Properties \ref{proper:schurinv} and \ref{proper:schurdet} to compute the inverse and determinant of the following matrix
\begin{align*}
M = 
\left[\begin{array}{@{\,}wc{10pt}wc{10pt}|wc{10pt}}
1 & 2 & 0 \\
0 & 1 & 1 \\
\hline
-2 & -1 & 2
\end{array}\right]
\end{align*}
via the partition above.
\end{exmp}
\begin{solution}
To use Properties \ref{proper:schurinv}, we need to first compute $A^{-1}$ and $M/A = D - CA^{-1}B$. We leave it to the readers to verify that
\begin{align*}
A^{-1} &= 
\begin{bmatrix}
1 & -2 \\
0 & 1
\end{bmatrix} \\
M/A = D - CA^{-1}B &=
\begin{bmatrix}
2
\end{bmatrix}
-
\begin{bmatrix}
-2 & -1
\end{bmatrix}
\begin{bmatrix}
1 & -2 \\
0 & 1    
\end{bmatrix}
\begin{bmatrix}
0 \\
1
\end{bmatrix}
= \begin{bmatrix}
-1
\end{bmatrix}
\end{align*}
Then by Properties \ref{proper:schurinv}, we have
\begin{align*}
M^{-1} &= \begin{bmatrix}
A^{-1} + A^{-1}B (M/A)^{-1} CA^{-1} & -A^{-1}B (M/A)^{-1} \\
-(M/A)^{-1}CA^{-1} & (M/A)^{-1}
\end{bmatrix} \\
&= \small\left[\begin{array}{@{\,}wc{210pt}|wc{84pt}}
\begin{bmatrix}
1 & -2 \\
0 & 1
\end{bmatrix}
+
\begin{bmatrix}
1 & -2 \\
0 & 1
\end{bmatrix}
\begin{bmatrix}
0 \\
1
\end{bmatrix}
\begin{bmatrix}
-1
\end{bmatrix}
\begin{bmatrix}
-2 & -1
\end{bmatrix} 
\begin{bmatrix}
1 & -2 \\
0 & 1
\end{bmatrix}
&
-\begin{bmatrix}
1 & -2 \\
0 & 1
\end{bmatrix}
\begin{bmatrix}
0 \\
1
\end{bmatrix}
\begin{bmatrix}
-1
\end{bmatrix}
\\[9pt]
\hline
-
\begin{bmatrix}
-1
\end{bmatrix}
\begin{bmatrix}
-2 & -1
\end{bmatrix}
\begin{bmatrix}
1 & -2 \\
0 & 1
\end{bmatrix}
&
\begin{bmatrix}
-1
\end{bmatrix}
\Tstrut
\end{array}\right] \\
&=
\left[\begin{array}{@{\,}wc{10pt}wc{10pt}|wc{10pt}}
-3 & 4 & -2 \\
2 & -2 & 1 \\
\hline
-2 & 3 & -1
\end{array}\right]
\end{align*}
Meanwhile, by Properties \ref{proper:schurdet},
\begin{align*}
\det(M) &= \det(A)\det(M/A) \\
&= \det(\begin{bmatrix}
1 & -2 \\
0 & 1
\end{bmatrix})
\det([-1]) = (1)(-1) = -1
\end{align*}
\end{solution}

Similar results are also available in terms of the Schur complement using block $D$ instead of $A$.
\begin{proper}
\label{proper:schurinvD}
For the $2 \times 2$ block matrix
\begin{align*}
M = 
\begin{bmatrix}
A & B \\
C & D
\end{bmatrix}
\label{eqn:schurinv2}
\end{align*}
where $A$ and $D$ are square submatrices, if $D$ and its Schur complement $M/D = A - BD^{-1}C$ of block $D$ are both invertible, then $M$ is invertible with
\begin{align}
M^{-1} = \begin{bmatrix}
(M/D)^{-1} & -(M/D)^{-1}BD^{-1}  \\
-D^{-1}C (M/D)^{-1} & D^{-1} + D^{-1}C (M/D)^{-1} BD^{-1}
\end{bmatrix}
\end{align}
and its determinant can be computed by
\begin{align}
\det(M) = \det(D)\det(A - BD^{-1}C) = \det(D)\det(M/D)
\end{align}
\end{proper}
\begin{proof}
See Exercise \ref{ex:schurinvD}.
\end{proof}

\subsection{Restriction of a Linear Transformation, Direct Sum of a Matrix}
In the last chapter, we have discussed linear transformations between two vector spaces, let's say, from $\mathcal{U}$ to $\mathcal{V}$. Sometimes we only care about how the linear transformation works on some specific subspace $\mathcal{W}$ of $\mathcal{U}$. This leads to the idea of \index{Restriction of a Linear Transformation}\keywordhl{restriction} of a linear transformation as follows.
\begin{defn}[Restriction of a Linear Transformation]
\label{defn:restrictionTW}
Given a linear transformation $T: \mathcal{U} \to \mathcal{V}$ and a proper subspace $\mathcal{W} \subset \mathcal{U}$, the restriction of $T$ to $\mathcal{W}$ is defined as
\begin{align*}
T|_W: \mathcal{W} \to \mathcal{V}, T|_W(\vec{w}) = T(\vec{w}) \text{ for any $\vec{w} \in \mathcal{W}$.}
\end{align*}
\end{defn}
In simpler terms, $T|_W$ works exactly as $T$ but is only defined on $\mathcal{W}$. Assume the vector spaces involved are all finite-dimensional, and $\dim(\mathcal{W}) = r < n = \dim(\mathcal{U})$. $\mathcal{W}$ then has a basis $\mathcal{\beta}_W$ with $r$ generating vectors (Properties \ref{proper:samenvecsbases}), which by part (c) of Properties \ref{proper:linindspanbasisnewver} can be extended to a new basis $\mathcal{\beta}' = \mathcal{\beta}_W \cup \mathcal{\gamma}$ for $\mathcal{U}$, where $\mathcal{\gamma}$ contains $n - r$ vectors and $\mathcal{\beta}' = \mathcal{\beta}_W \cup \mathcal{\gamma}$ has exactly $n$ linearly independent vectors by construction. Some may wonder why we suddenly talk about the restriction of a linear transformation here, and the reason is that its related principles can be viewed from the standpoint of a block matrix.\par
To see this, let $\mathcal{\beta}_W = \{\vec{u}^{(1)}, \vec{u}^{(2)}, \ldots, \vec{u}^{(r)}\}$ and $\mathcal{\gamma} = \{\vec{u}^{(r+1)}, \ldots, \vec{u}^{(n)}\}$, and thus $\mathcal{\beta}' = \{\vec{u}^{(1)}, \vec{u}^{(2)}, \ldots, \vec{u}^{(r)}, \vec{u}^{(r+1)}, \ldots, \vec{u}^{(n)}\}$. By Definition \ref{defn:directsum}, since the vectors in $\mathcal{\beta}' = \mathcal{\beta}_W \cup \mathcal{\gamma}$ are designed to be linearly independent, the subspace $\mathcal{W}^C$ generated by $\mathcal{\gamma}$ will be the complement of $\mathcal{W}$ as a result of $\mathcal{W} \oplus \mathcal{W}^C$ being a direct sum that produces the $n$-dimensional $\mathcal{U}$ (Properties \ref{proper:complement}). Any $\vec{w} \in \mathcal{W} \subset \mathcal{U}$ will then have a coordinate representation of
\begin{align*}
(w_1, w_2, \ldots, w_r, 0, \ldots, 0)^T
\end{align*}
in the $\mathcal{\beta}'$ basis where components beyond the $r$-th index are all zeros. From the perspective of direct sum, it is the same as $\vec{w} \oplus \textbf{0}_{n-r} = (w_1, w_2, \ldots, w_r)_{\beta_W}^T \oplus (0, \ldots, 0)_{\gamma}^T$, i.e. the components of $\vec{w}$ for $\mathcal{W}^C$ are all zeros. By Definition \ref{defn:matrixrepoflintrans}, writing out the matrix representation of $[T]_{\beta'}^\eta$ where $\eta$ is an arbitrary basis for the $m$-dimensional $\mathcal{V}$ results in
\begin{align*}
[T]_{\beta'}^\eta = \begin{bmatrix}
a_1^{(1)} & a_1^{(2)} & \cdots & a_1^{(r)} & a_1^{(r+1)} & \cdots & a_1^{(n)} \\
a_2^{(1)} & a_2^{(2)} & & a_2^{(r)} & a_2^{(r+1)} & & a_2^{(n)} \\
\vdots & & & \vdots & \vdots & & \vdots \\
a_m^{(1)} & a_m^{(2)} & \cdots & a_m^{(r)} & a_m^{(r+1)} & \cdots & a_m^{(n)}
\end{bmatrix}
\end{align*}
Since we are only concerned about $\vec{w} \in \mathcal{W} \subset \mathcal{U}$ (or $\vec{w} \oplus \textbf{0} \in \mathcal{W} \oplus \mathcal{W}^C$) when dealing with $T|_W$, when we apply $T$ on $\vec{w}$, which is
\begin{align}
[T]_{\beta'}^\eta[\vec{w}]_{\beta'}
&=
\left[\begin{array}{@{\,}wc{15pt}wc{15pt}wc{15pt}wc{15pt}|@{\quad}wc{15pt}wc{15pt}wc{15pt}}
a_1^{(1)} & a_1^{(2)} & \cdots & a_1^{(r)} & a_1^{(r+1)} & \cdots & a_1^{(n)} \\
a_2^{(1)} & a_2^{(2)} & & a_2^{(r)} & a_2^{(r+1)} & & a_2^{(n)} \\
\vdots & & & \vdots & \vdots & & \vdots \\
a_m^{(1)} & a_m^{(2)} & \cdots & a_m^{(r)} & a_m^{(r+1)} & \cdots & a_m^{(n)}
\end{array}\right]_{\beta_W+\gamma}^\eta
\left[\begin{array}{c}
w_1 \\
w_2 \\
\vdots \\
w_r \\
\hline
0 \\
\vdots \\
0
\end{array}\right]_{\beta_W+\gamma} \nonumber \\
&= [(T|_W)\mid(T|_{W^C})]_{\beta_W+\gamma}^\eta [\vec{w} \oplus \textbf{0}]^T_{\beta_W+\gamma} 
\end{align}
We can simply ignore $[T|_{W^C}]_\gamma^\eta$, the block at the right of the $[T]_{\beta'}^\eta$ partition as well as discard the all-zero components of $[\vec{w}]_{\beta'}$ starting from the $(r+1)$-th index, and keep only the other block $[T|_{W}]_{\beta_W}^\eta$ at the left and the $[\vec{w}]_{\beta_W}$ part. The output of the truncated multiplication
\begin{align}
[T|_{W}]_{\beta_W}^\eta [\vec{w}]_{\beta_W} = 
\left[\begin{array}{@{\,}wc{15pt}wc{15pt}wc{15pt}wc{15pt}}
a_1^{(1)} & a_1^{(2)} & \cdots & a_1^{(r)} \\
a_2^{(1)} & a_2^{(2)} & & a_2^{(r)} \\
\vdots & & & \vdots \\
a_m^{(1)} & a_m^{(2)} & \cdots & a_m^{(r)} 
\end{array}\right]
\begin{bmatrix}
w_1 \\
w_2 \\
\vdots \\
w_r \\
\end{bmatrix}
\end{align}
will represent the same vector in $\mathcal{W} \subset \mathcal{U}$ as that mapped from the full form $[T]_{\beta'}^\eta[\vec{w}]_{\beta'}$.
\begin{proper}
\label{proper:restrictmat}
For a linear transformation $T: \mathcal{U} \to \mathcal{V}$ between two finite-dimensional spaces, if a proper subspace $\mathcal{W}$ of $\mathcal{U}$ is generated by a basis $\mathcal{\beta}_W = \{\vec{w}^{(1)}, \vec{w}^{(2)}, \ldots, \vec{w}^{(r)}\}$, then the matrix representation of the restriction of $T$ to $\mathcal{W}$ with respect to $\mathcal{\beta}_W$ and $\mathcal{\eta}$ will be given by
\begin{align}
[T|_{W}]_{\beta_W}^\eta = \begin{bmatrix}
[T(\vec{w}^{(1)})]_\eta | [T(\vec{w}^{(2)})]_\eta | \cdots | [T(\vec{w}^{(r)})]_\eta
\end{bmatrix}    
\end{align}
where $\mathcal{\eta}$ is any basis for $\mathcal{V}$. (This can be compared to Definition \ref{defn:matrixrepoflintrans}.)
\end{proper}
In general, the effect of a linear transformation $T: \mathcal{U} \to \mathcal{V}$ applied to $\vec{u} \in \mathcal{U}$ is equivalent to the sum of responses from the restrictions of $T$ to a set of subspaces $\mathcal{W}_1, \mathcal{W}_2, \cdots, \mathcal{W}_s$ where they constitute a direct sum $\mathcal{W}_1 \oplus \mathcal{W}_2 \oplus \cdots \oplus \mathcal{W}_s = \mathcal{U}$, applied on the corresponding components $\vec{w}_1 \in \mathcal{W}_1, \vec{w}_2 \in \mathcal{W}_2, \cdots, \vec{w}_s \in \mathcal{W}_s$ of $\vec{u} = \vec{w}_1 + \vec{w}_2 + \cdots + \vec{w}_s$ in these smaller subspaces: $T(\vec{u}) = T(\vec{w}_1) + T(\vec{w}_2) + \cdots + T(\vec{w}_s)$.
\begin{exmp}
Given a linear transformation $T: \mathcal{U} \to \mathcal{V}$ that has a matrix representation of
\begin{align*}
[T]_\beta^\eta = 
\begin{bmatrix}
1 & 1 & 2 \\
2 & 0 & 1 \\
1 & -1 & 1
\end{bmatrix}
\end{align*}
with respect to some bases $\mathcal{\beta} = \{\vec{u}^{(1)}, \vec{u}^{(2)}, \vec{u}^{(3)}\}$ and $\mathcal{\eta} = \{\vec{v}^{(1)}, \vec{v}^{(2)}, \vec{v}^{(3)}\}$ for $\mathcal{U}$ and $\mathcal{V}$, find the restriction of $T$ to $\mathcal{W}$, where $\mathcal{W} \subset \mathcal{U}$ has a basis of $\beta_W = \{\vec{w}^{(1)}, \vec{w}^{(2)}\}$, with $\vec{w}^{(1)} = \vec{u}^{(1)} + \vec{u}^{(2)}$ and $\vec{w}^{(2)} = \vec{u}^{(1)} + \vec{u}^{(2)} + \vec{u}^{(3)}$.
\end{exmp}
\begin{solution}
We will take an indirect approach of reconstructing the basis first by finding a third vector generating $\mathcal{W}^C$ and producing the direct sum $\mathcal{W} \oplus \mathcal{W}^C = \mathcal{U}$. The change of coordinates matrix $\smash{P_{\beta_W}^{\beta}}$ from $\beta_W$ to $\beta$ as devised in Theorem \ref{thm:bijectivechincoord} appropriate in this situation is a $3 \times 2$ matrix instead since there are only two basis vectors in $\mathcal{\beta}_W$, and it can be easily seen to be
\begin{align*}
P_{\beta_W}^{\beta} &= \begin{bmatrix}
[\vec{w}^{(1)}]_\beta | [\vec{w}^{(2)}]_\beta
\end{bmatrix} \\
&= \begin{bmatrix}
[\vec{u}^{(1)} + \vec{u}^{(2)}]_\beta | [\vec{u}^{(1)} + \vec{u}^{(2)} + \vec{u}^{(3)}]_\beta
\end{bmatrix} \\
&= \begin{bmatrix}
1 & 1 \\
1 & 1 \\
0 & 1
\end{bmatrix}
\end{align*}
and we are to find $[\vec{w}^{(3)}]_\beta$ to complete a basis $\mathcal{\beta}' = \mathcal{\beta}_W \cup \{\vec{w}^{(3)}\}$ and hence $\smash{P_{\beta'}^{\beta}}$. An algorithm to do so, motivated by Footnote \ref{foot:inconsth} in Chapter \ref{chap:vec_space}, is to apply Gaussian Elimination to $\smash{P_{\beta_W}^{\beta}}$ first and then append 
$(0,0,1)^T$ to the right of the RREF to make an identity matrix, and reverse the entire reduction procedure as follows.
\begin{align*}
\left[\begin{array}{@{\,}wc{10pt}wc{10pt}@{\,}}
1 & 1 \\
1 & 1 \\
0 & 1
\end{array}\right] &\to
\left[\begin{array}{@{\,}wc{10pt}wc{10pt}@{\,}}
1 & 1 \\
0 & 0 \\
0 & 1
\end{array}\right] 
& R_2 - R_1 \to R_2 \\
&\to \left[\begin{array}{@{\,}wc{10pt}wc{10pt}@{\,}}
1 & 1 \\
0 & 1 \\
0 & 0
\end{array}\right]
& R_2 \leftrightarrow R_3 \\
&\to \left[\begin{array}{@{\,}wc{10pt}wc{10pt}@{\,}}
1 & 0 \\
0 & 1 \\
0 & 0
\end{array}\right]
& R_1 - R_2 \to R_1
\end{align*}
\begin{align*}
\left[\begin{array}{@{\,}wc{10pt}wc{10pt}|wc{10pt}@{\,}}
1 & 0 & 0 \\
0 & 1 & 0 \\
0 & 0 & 1
\end{array}\right] &\to
\left[\begin{array}{@{\,}wc{10pt}wc{10pt}|wc{10pt}@{\,}}
1 & 1 & 0 \\
0 & 1 & 0 \\
0 & 0 & 1
\end{array}\right]
& R_1 + R_2 \to R_1 \\
&\to \left[\begin{array}{@{\,}wc{10pt}wc{10pt}|wc{10pt}@{\,}}
1 & 1 & 0 \\
0 & 0 & 1 \\
0 & 1 & 0
\end{array}\right]
& R_2 \leftrightarrow R_3 \\
&\to \left[\begin{array}{@{\,}wc{10pt}wc{10pt}|wc{10pt}@{\,}}
1 & 1 & 0 \\
1 & 1 & 1 \\
0 & 1 & 0
\end{array}\right]
& R_2 + R_1 \to R_2
\end{align*}
So $[\vec{w}^{(3)}]_\beta = (0,1,0)_\beta^T$ is a possible choice. This simple algorithm remains straightforward when the number of dimensions and vectors to be appended becomes much larger.\footnote{In fact, we only need to keep track of the row-swapping operations with full-zero rows.} Now
\begin{align*}
P_{\beta'}^\beta = 
\begin{bmatrix}
1 & 1 & 0 \\
1 & 1 & 1 \\
0 & 1 & 0
\end{bmatrix}
\end{align*}
and by Properties \ref{proper:chcoordsmat}
\begin{align*}
[T]_{\beta'}^\eta &= [T]_\beta^\eta P_{\beta'}^\beta \\
&= \begin{bmatrix}
1 & 1 & 2 \\
2 & 0 & 1 \\
1 & -1 & 1
\end{bmatrix}
\begin{bmatrix}
1 & 1 & 0 \\
1 & 1 & 1 \\
0 & 1 & 0
\end{bmatrix} \\
&=
\begin{bmatrix}
2 & 4 & 1 \\
2 & 3 & 0 \\
0 & 1 & -1
\end{bmatrix}
\end{align*}
The matrix representation of the restriction of $T$ to $\mathcal{W}$ with respect to $\mathcal{\beta}_W$ then agrees with the first two columns of $\smash{[T]_{\beta'}^\eta}$. The third column of $[T]_{\beta'}^\eta$ that characterizes the action of $T|_{W_C}$, is removed. These lead to
\begin{align*}
[T|_W]_{\beta_W}^\eta =
\begin{bmatrix}
2 & 4 \\
2 & 3 \\
0 & 1 
\end{bmatrix}
\end{align*}
\end{solution}
$\blacktriangleright$ Short Exercise: Directly apply Properties \ref{proper:restrictmat} to redo the example above.\footnote{$[T(\vec{w}_1)]_\eta = [T(\vec{u}_1+\vec{u}_2)]_\eta = [T]_\beta^\eta(1,1,0)_\beta^T =
\begin{bmatrix}
1 & 1 & 2 \\
2 & 0 & 1 \\
1 & -1 & 1
\end{bmatrix}_\beta^\eta
\begin{bmatrix}
1 \\
1 \\
0
\end{bmatrix}_\beta
= \begin{bmatrix}
2 \\
2 \\
0
\end{bmatrix}_\eta
$ and this will be the first column of $\smash{[T|_W]_{\beta_W}^\eta}$. The second column is derived similarly by evaluating $[T(\vec{w}_2)]_\eta$.}\par
With the concept of restriction, we can now introduce the matrix analog of a direct sum. For a linear transformation $T: \mathcal{U} \to \mathcal{V}$, if the vector spaces (finite-dimensional) involved are direct sums such that $\mathcal{U} = \mathcal{W} \oplus \mathcal{W}_C$ and $\mathcal{V} = \mathcal{Y} \oplus \mathcal{Y}^C$, and the ranges
\begin{align*}
\mathcal{R}(T|_W) \subseteq \mathcal{Y} & & \mathcal{R}(T|_{W^C}) \subseteq \mathcal{Y}^C
\end{align*}
of the two restrictions are such that vectors in $\mathcal{W}$ and $\mathcal{W}^C$ are mapped by $T$ to vectors in $\mathcal{Y}$ and $\mathcal{Y}^C$ separately, then $T = T|_W \oplus T|_{W^C}$ is a \index{Matrix Direct Sum}\keywordhl{matrix direct sum} in the sense that the linear transformation $T$ maps each of the complement subspaces in a direct sum of $\mathcal{U}$ into the corresponding complement subspaces in a direct sum of $\mathcal{V}$. If we write the input vector $\vec{u} = \vec{w} + \vec{w}^C$ as a direct sum where $\vec{w} \in \mathcal{W}$ and $\vec{w}^C \in \mathcal{W}^C$, then the output vector will also be a direct sum $\vec{v} = \vec{y} + \vec{y}^C$ where $\vec{y} = T|_W(\vec{w}) = T(\vec{w})$ and $\vec{y}^C = T|_{W^C}(\vec{w}^C) = T(\vec{w}^C)$, which are obtained by first computing $T(\vec{w})$ and $T(\vec{w}^C)$ individually, and then directly concatenating them together.
\begin{defn}[Matrix Direct Sum]
\label{defn:matdirectsum}
The direct sum of two matrices acting as linear transformations $T_1: \mathcal{U}_1 \to \mathcal{V}_1$ and $T_2: \mathcal{U}_2 \to \mathcal{V}_2$ is $T = T_1 \oplus T_2$ such that for any vector direct sum $\vec{u} = \vec{u}_1 + \vec{u}_2$ in $\mathcal{U} = \mathcal{U}_1 \oplus \mathcal{U}_2$, applying $T$ on $\vec{u}$ will yield an output of a vector direct sum $T(\vec{u}) = \vec{v} = \vec{v}_1 + \vec{v}_2$ in $\mathcal{V}_1 \oplus \mathcal{V}_2$ as well, where $\vec{v}_1 = T_1(\vec{u}_1) = T|_{U_1}(\vec{u}_1) \in \mathcal{V}_1$ and $\vec{v}_2 = T_2(\vec{u}_2) = T|_{U_2}(\vec{u}_2) \in \mathcal{V}_2$. The matrix direct sum is then the matrix representation of $T = T_1 \oplus T_2$ with respect to the direct sum basis for $\mathcal{U}_1 \oplus \mathcal{U}_2$ and $\mathcal{V}_1 \oplus \mathcal{V}_2$.
\end{defn}
Using the above definition, if $\mathcal{U}_1$ and $\mathcal{U}_2$ has a basis $\mathcal{\beta}_1 = \{\vec{w}^{(1)}, \vec{w}^{(2)}, \ldots, \vec{w}^{(r)}\}$ and $\mathcal{\beta}_2 = \{\vec{w}^{(r+1)}, \vec{w}^{(r+2)}, \ldots, \vec{w}^{(n)}\}$, and $\mathcal{V}_1$ and $\mathcal{V}_2$ has a basis $\mathcal{\eta}_1 = \{\vec{y}^{(1)}, \vec{y}^{(2)}, \allowbreak \ldots, \vec{y}^{(s)}\}$ and $\mathcal{\eta}_2 = \{\vec{y}^{(s+1)}, \vec{y}^{(s+2)}, \ldots, \vec{y}^{(m)}\}$, where $r, n, s, m$ are some integers, then $T = T_1 \oplus T_2$ will have a \textit{block diagonal} matrix representation of
\begin{align}
[T]_{\beta_1 + \beta_2}^{\eta_1 + \eta_2}
\equiv
\begin{bmatrix}
([T_1]_{\beta_1}^{\eta_1})_{s \times r} & [\textbf{0}]_{s \times (n-r)} \\
[\textbf{0}]_{(m-s) \times r} & ([T_2]_{\beta_2}^{\eta_2})_{(m-s) \times (n-r)}
\end{bmatrix}_{\beta_1 + \beta_2}^{\eta_1 + \eta_2}
\end{align}
with respect to the $\mathcal{\beta}_1 \cup \mathcal{\beta}_2$ and $\mathcal{\eta}_1 \cup \mathcal{\eta}_2$ bases. To see this, let $\vec{u} = \vec{u}^{(1)} \oplus \vec{u}^{(2)}$, $\vec{u}^{(1)} \in \mathcal{U}_1$ and $\vec{u}^{(2)} \in \mathcal{U}_2$, then
\begin{align*}
T(\vec{u}) &= [T]_{\beta_1 + \beta_2}^{\eta_1 + \eta_2}[\vec{u}]_{\beta_1 + \beta_2} \\ 
&= \begin{bmatrix}
([T_1]_{\beta_1}^{\eta_1})_{s \times r} & [\textbf{0}]_{s \times (n-r)} \\
[\textbf{0}]_{(m-s) \times r} & ([T_2]_{\beta_2}^{\eta_2})_{(m-s) \times (n-r)}
\end{bmatrix}_{\beta_1 + \beta_2}^{\eta_1 + \eta_2}
\begin{bmatrix}
([\vec{u}_1]_{\beta_1})_r \\
([\vec{u}_2]_{\beta_2})_{n-r}
\end{bmatrix}_{\beta_1 + \beta_2} \\
&= 
\begin{bmatrix}
([T_1]_{\beta_1}^{\eta_1})_{s \times r}([\vec{u}_1]_{\beta_1})_r + [\textbf{0}]_{s \times (n-r)}([\vec{u}_2]_{\beta_2})_{n-r} \\
[\textbf{0}]_{(m-s) \times r}([\vec{u}_1]_{\beta_1})_r + ([T_2]_{\beta_2}^{\eta_2})_{(m-s) \times (n-r)}([\vec{u}_2]_{\beta_2})_{n-r}
\end{bmatrix}_{\eta_1 + \eta_2} \\
&= 
\begin{bmatrix}
([T_1]_{\beta_1}^{\eta_1}[\vec{u}_1]_{\beta_1})_s \\
([T_2]_{\beta_2}^{\eta_2}[\vec{u}_2]_{\beta_2})_{m-s}
\end{bmatrix}_{\eta_1 + \eta_2} \\
&\equiv
T_1(\vec{u}_1) \oplus T_2(\vec{u}_2)
\end{align*}
where the image is a direct sum composed of $T_1(\vec{u}_1) \equiv \smash{[T_1]_{\beta_1}^{\eta_1}}[\vec{u}_1]_{\beta_1} \in \mathcal{V}_1$ and $T_2(\vec{u}_2) \equiv \smash{[T_2]_{\beta_2}^{\eta_2}}[\vec{u}_2]_{\beta_2} \in \mathcal{V}_2$ from applying $T_1$ and $T_2$ separately to the preimages $\vec{u}_1 \in \mathcal{U}_1$ and $\vec{u}_2 \in \mathcal{U}_2$ in the two subspaces.
% we will consider linear operators such that $T: \mathcal{V} \to \mathcal{V}$ is a linear transformation within the vector space $\mathcal{V}$ itself. A special case for the restriction of a linear operator, originally $T|_W: \mathcal{W} \to \mathcal{V}$ with $\mathcal{W} \subset \mathcal{V}$, is that the range of $T|_W$ is a subspace of $\mathcal{W}$ so that it can be reduced to $T|_W: \mathcal{W} \to \mathcal{W}$. $\mathcal{W}$ is then referred to as an \index{Invariant Subspace}\keywordhl{invariant subspace} under $T$, since the mapping $T|_W$ and hence $T$ does not alter the subspace in the sense that any vector in $\mathcal{W}$ is never mapped to other vectors outside $\mathcal{W}$ via $T$, i.e. $T(\vec{w}) \notin \mathcal{W}^C$ except the zero vector. In this case, we can replace $T|_W$ by $T$.
%\begin{defn}[Invariant Subspaces]
%A subspace $\mathcal{W}$ of $\mathcal{V}$ is known as an invariant subspace under the linear transformation $T$ if for any $\vec{w} \in \mathcal{W}$, $T(\vec{w}) \in \mathcal{W}$.
%\end{defn}
%If $\dim(\mathcal{V}) = n$, $\dim(\mathcal{W}) = r$ are both finite-dimensional, then the matrix representation of $T: \mathcal{V} \to \mathcal{V}$, with respect to the basis $\mathcal{B}' = \mathcal{B}_W \cup \mathcal{G}$ as put during the derivation for Properties \ref{proper:restrictmat}, can be expressed as a block matrix:
%\begin{align*}
%[T]_{B'} = \begin{bmatrix}
%A_{W,{r \times r}} & *_{r\times(n-r)} \\
%0_{(n-r) \times r} & *_{r\times(n-r)}
%\end{bmatrix}
%=
%\begin{bmatrix}
%[T|_W]_{B_W} \, |\,*_{n\times(n-r)}
%\end{bmatrix}
%\end{align*}
%where the $r$ bottommost rows of $[T|_W]_{B_W} = \begin{bmatrix}
%A_{W,{r \times r}} \\
%0_{(n-r) \times r}
%\end{bmatrix}$ are all zeros and $A_W$ is an $r \times r$ matrix. Note that for any $\vec{w} \in \mathcal{W}$, its mapped image $T(\vec{w})$
%\begin{align*}
%[T]_{B'}[\vec{w}]_{B'} &= \begin{bmatrix}
%A_{W,{r \times r}} & *_{r\times(n-r)} \\
%0_{(n-r) \times r} & *_{r\times(n-r)}
%\end{bmatrix}_{B'}
%\begin{bmatrix}
%w_1 \\
%w_2 \\
%\vdots \\
%w_r \\
%0 \\
%\vdots \\
%0
%\end{bmatrix}_{B'}
%=
%\begin{bmatrix}
%\sum_{j=1}^r (A_W)_{1j} w_j \\
%\sum_{j=1}^r (A_W)_{2j} w_j \\
%\vdots \\
%\sum_{j=1}^r (A_W)_{rj} w_j \\
%0 \\
%\vdots \\
%0
%\end{bmatrix}_{B'}
%\in \mathcal{W}
%\end{align*}
%is still in $\mathcal{W}$ so that the block matrix representation above is consistent. If the complement $\mathcal{W}^C$ generated by $\mathcal{G}$ itself is also a invariant subspace under $T$, then by a similar logic, we have
%\begin{align*}
%[T]_{B'} = \begin{bmatrix}
%A_{W,{r \times r}} & 0_{r\times(n-r)} \\
%0_{(n-r) \times r} & A_{W^C,{r \times r}}
%\end{bmatrix}
%\end{align*}
For example, the matrix direct sum of $A \oplus B$ given
\begin{align*}
A &= 
\begin{bmatrix}
1 & 2 & 3 \\
4 & 5 & 6
\end{bmatrix}
& 
B &=
\begin{bmatrix}
1 & 8 \\
1 & 1 \\
4 & 0
\end{bmatrix}
\end{align*}
is
\begin{align*}
A \oplus B =
\left[\begin{array}{@{\,}wc{10pt}wc{10pt}wc{10pt}|wc{10pt}wc{10pt}}
1 & 2 & 3 & 0 & 0 \\
4 & 5 & 6 & 0 & 0 \\
\hline
0 & 0 & 0 & 1 & 8 \\
0 & 0 & 0 & 1 & 1 \\
0 & 0 & 0 & 4 & 0
\end{array}\right]
\end{align*}
in which $A$ and $B$ are matrices representing linear transformations of $T_1: \mathcal{U}_1 \to \mathcal{V}_1$ and $T_2: \mathcal{U}_2 \to \mathcal{V}_2$, where $\mathcal{U}_1$, $\mathcal{U}_2$, $\mathcal{V}_1$, $\mathcal{V}_2$ have dimensions of $3,2,2,3$. Subsequently, $A \oplus B$ is a matrix corresponding to a mapping $T_1 \oplus T_2$ from $\mathcal{U} = \mathcal{U}_1 \oplus \mathcal{U}_2$ to $\mathcal{V} = \mathcal{V}_1 \oplus \mathcal{V}_2$. Finally, the matrix direct sum of more than two matrices $A_1, A_2, A_3, \ldots, A_{n-1}, A_n$ are defined recursively just like a vector direct sum as
\begin{align*}
&\quad A_1 \oplus A_2 \oplus A_3 \oplus \cdots \oplus A_{n-1} \oplus A_n \\
&= (\cdots ((A_1 \oplus A_2) \oplus A_3) \oplus \cdots \oplus A_{n-1}) \oplus A_n
\end{align*}
As another example, sometimes we may regard a matrix that does not look like a direct sum to be effectively one with respect to appropriate coordinate systems in a broader sense.
\begin{exmp}
For a linear transformation $T: \mathcal{U} \to \mathcal{V}$ that has a matrix representation of
\begin{align*}
[T]_\beta^\eta =
\begin{bmatrix}
1 & 0 & 2 & -2 \\
0 & 0 & 1 & 0 \\
1 & -2 & 1 & 0
\end{bmatrix} 
\end{align*}
with respect to some bases $\mathcal{\beta} = \{\vec{u}^{(1)}, \vec{u}^{(2)}, \vec{u}^{(3)}, \vec{u}^{(4)}\}$, $\mathcal{\eta} = \{\vec{v}^{(1)}, \vec{v}^{(2)}, \vec{v}^{(3)}\}$, show that it can turn into an apparent matrix direct sum if the coordinate systems are changed according to $\mathcal{\beta}' = \{\vec{u}^{(1)'}, \vec{u}^{(2)'}, \vec{u}^{(3)'}, \vec{u}^{(4)'}\}$, $\mathcal{\eta}' = \{\vec{v}^{(1)'}, \vec{v}^{(2)'}, \vec{v}^{(3)'}\}$, where
\begin{align*}
\vec{u}^{(1)'} &= \vec{u}^{(1)} & \vec{v}^{(1)'} &= \vec{v}^{(1)} + \vec{v}^{(2)} \\
\vec{u}^{(2)'} &= \vec{u}^{(3)} & \vec{v}^{(2)'} &= -\vec{v}^{(2)} + \vec{v}^{(3)} \\
\vec{u}^{(3)'} &= \vec{u}^{(1)} + \vec{u}^{(2)} & \vec{v}^{(3)'} &= \vec{v}^{(1)} - \vec{v}^{(3)} \\
\vec{u}^{(4)'} &= \vec{u}^{(1)} + \vec{u}^{(4)}
\end{align*}
\end{exmp}
\begin{solution}
The change of coordinate matrices suggested by Properties \ref{proper:chcoordsmat} are
\begin{align*}
P_{\beta'}^\beta &= 
\begin{bmatrix}
1 & 0 & 1 & 1 \\
0 & 0 & 1 & 0 \\
0 & 1 & 0 & 0 \\
0 & 0 & 0 & 1
\end{bmatrix}_{\beta'}^\beta
&
Q_{\eta'}^\eta &= 
\begin{bmatrix}
1 & 0 & 1 \\
1 & -1 & 0 \\
0 & 1 & -1
\end{bmatrix}_{\eta'}^\eta
\end{align*}
and the new matrix representation of $T$ is
\begin{align*}
[T]_{\beta'}^{\eta'} &= (Q_{\eta'}^\eta)^{-1} [T]_\beta^\eta P_{\beta'}^\beta \\
&= \left(\begin{bmatrix}
1 & 0 & 1 \\
1 & -1 & 0 \\
0 & 1 & -1
\end{bmatrix}_{\eta'}^\eta\right)^{-1}
\begin{bmatrix}
1 & 0 & 2 & -2 \\
0 & 0 & 1 & 0 \\
1 & -2 & 1 & 0
\end{bmatrix}_\beta^\eta
\begin{bmatrix}
1 & 0 & 1 & 1 \\
0 & 0 & 1 & 0 \\
0 & 1 & 0 & 0 \\
0 & 0 & 0 & 1
\end{bmatrix}_{\beta'}^\beta \\
&= \begin{bmatrix}
\frac{1}{2} & \frac{1}{2} & \frac{1}{2} \\
\frac{1}{2} & -\frac{1}{2} & \frac{1}{2} \\
\frac{1}{2} & -\frac{1}{2} & -\frac{1}{2}
\end{bmatrix}_{\eta}^{\eta'}
\begin{bmatrix}
1 & 0 & 2 & -2 \\
0 & 0 & 1 & 0 \\
1 & -2 & 1 & 0
\end{bmatrix}_\beta^\eta
\begin{bmatrix}
1 & 0 & 1 & 1 \\
0 & 0 & 1 & 0 \\
0 & 1 & 0 & 0 \\
0 & 0 & 0 & 1
\end{bmatrix}_{\beta'}^\beta \\
&=
\begin{bmatrix}
1 & 2 & 0 & 0 \\
1 & 1 & 0 & 0 \\
0 & 0 & 1 & -1
\end{bmatrix}_{\beta'}^{\eta'}
\end{align*}
where it can be seen that
\begin{align*}
\begin{bmatrix}
1 & 2 & 0 & 0 \\
1 & 1 & 0 & 0 \\
0 & 0 & 1 & -1
\end{bmatrix} =
\begin{bmatrix}
1 & 2 \\
1 & 1
\end{bmatrix}
\oplus 
\begin{bmatrix}
1 & -1
\end{bmatrix}
\end{align*}
\end{solution}

\section{Python Programming}
Complex numbers in Python are written as \verb|a+bj|. For example,
\begin{lstlisting}
z_1 = 1 - 2j
z_2 = 3 + 1j
print(z_1, z_2)
\end{lstlisting}
returns \verb|(1-2j) (3+1j)| (\texttt{1} in front of \texttt{j} is needed). Conjugate, modulus, and argument can be found by
\begin{lstlisting}
import numpy as np
from scipy import linalg

print(np.conjugate(z_1))
print(np.abs(z_2))    
print(np.angle(z_1))
\end{lstlisting}
which yields \verb|(1+2j)|, \verb|3.162278| ($\sqrt{10}$), and \verb|-1.10715| (in radians). Addition, subtraction, multiplication, and division of complex numbers in Python are coded just like if they are ordinary numbers.
\begin{lstlisting}
print(3*z_1 + z_2) # (6-5j)
print(z_1 - 2*z_2) # (-5-4j)
print(z_1 * z_2) # (5-5j)
print(z_1 / z_2) # (0.1-0.7000000000000001j), floating-point error
\end{lstlisting}
The same goes for complex matrices and their multiplication. As an example,
\begin{lstlisting}
A = np.array([[4.  , 2.+1.j],
              [-3.j, 1-2.j]])
B = np.array([[3-1.j, 0],
              [2-5.j, 4.j]])
print(A @ B)
\end{lstlisting}
produces
\begin{lstlisting}
[[ 21.-12.j  -4. +8.j]
 [-11.-18.j   8. +4.j]]    
\end{lstlisting}
Conjugate and Hermitian transpose of a complex matrix is retrieved by
\begin{lstlisting}
print(np.conjugate(A))
print(np.conjugate(B).T) # Hermitian transpose = conjugate + transpose, or just .H if np.matrix is used instead of np.array
\end{lstlisting}
resulting in
\begin{lstlisting}
[[ 4.-0.j  2.-1.j]
 [-0.+3.j  1.+2.j]]
[[3.+1.j 2.+5.j]
 [0.-0.j 0.-4.j]]
\end{lstlisting}
The usual functions for inverse and determinant also work on complex matrices. The lines
\begin{lstlisting}
print(linalg.det(A))
print(linalg.inv(B))
\end{lstlisting}
give
\begin{lstlisting}
(1-2j)
[[3.00000000e-01+0.1j   0.00000000e+00+0.j   ]
 [3.25000000e-01+0.275j 1.38777878e-17-0.25j ]] # Again, round-off error    
\end{lstlisting}
We can use 1D complex matrices as complex vectors.
\begin{lstlisting}
u = np.array([1+1.j, -3.j, 2])
v = np.array([5, 1+2.j, 1-4.j])
\end{lstlisting}
The complex dot product between two complex vectors are then found by \verb|vdot|
\begin{lstlisting}
print(np.vdot(u,v).conj())
\end{lstlisting}
notice that a conjugate is needed since \verb|numpy| defines complex dot product with a different convention such that the first complex vector is conjugated instead of the second one. It then outputs the correct answer of \verb|(1+10j)|. The \verb|norm| function still works fine, e.g. \verb|print(linalg.norm(u))| gives \verb|3.87298| ($\sqrt{(1+i)(1-i) + (-3i)(3i) + (2)^2} = \sqrt{15}$). Finally, for the discussion in the last section, to build a block matrix using submatrices, we can use the \verb|block| function as
\begin{lstlisting}
C = np.array([1, 3+2.j])
D = np.array([-1.j, 2])

print(np.block([[A, B],
                [C, D]]))
\end{lstlisting}
which outputs
\begin{lstlisting}
[[ 4.+0.j  2.+1.j  3.-1.j  0.+0.j]
 [-0.-3.j  1.-2.j  2.-5.j  0.+4.j]
 [ 1.+0.j  3.+2.j -0.-1.j  2.+0.j]]    
\end{lstlisting}
Another example of constructing a block diagonal matrix is
\begin{lstlisting}
print(np.block([[A, np.zeros([2,3])],
                [np.zeros([3,2]), np.identity(3)]]))
\end{lstlisting}
generating
\begin{lstlisting}
[[ 4.+0.j  2.+1.j  0.+0.j  0.+0.j  0.+0.j]
 [-0.-3.j  1.-2.j  0.+0.j  0.+0.j  0.+0.j]
 [ 0.+0.j  0.+0.j  1.+0.j  0.+0.j  0.+0.j]
 [ 0.+0.j  0.+0.j  0.+0.j  1.+0.j  0.+0.j]
 [ 0.+0.j  0.+0.j  0.+0.j  0.+0.j  1.+0.j]]    
\end{lstlisting}

\section{Exercises}

\begin{Exercise}
By considering Euler's formula stated in Definition \ref{defn:Euler}, we have for any $\theta$, $\phi$
\begin{align*}
e^{i \theta} &= \cos \theta + i \sin \theta \\
e^{i \phi} &= \cos \phi + i \sin \phi \\
e^{i (\theta+\phi)} &= \cos (\theta+\phi) + i \sin (\theta+\phi)
\end{align*}
If we take the product of the first two equations, we also have
\begin{align*}
e^{i (\theta+\phi)} &= (\cos \theta + i \sin \theta)(\cos \phi + i \sin \phi)
\end{align*}
By equating the two expressions of $e^{i (\theta+\phi)}$, expand and compare the real and imaginary parts, prove the famous angle sum identities, which are
\begin{align*}
\cos(\theta+\phi) &= \cos\theta \cos\phi - \sin\theta \sin\phi \\
\sin(\theta+\phi) &= \sin\theta \cos\phi + \cos\theta \sin\phi 
\end{align*}
Hence, by either using the results above or De Moivre's Formula, prove the double angle formula shown below.
\begin{align*}
\cos(2\theta) &= \cos^2\theta - \sin^2\theta \\
\sin(2\theta) &= 2\sin\theta \cos\theta  
\end{align*}
\end{Exercise}
\begin{Answer}
By De Moivre's Formula:
\begin{align*}
\cos(2\theta) + i \sin(2\theta) &= (\cos\theta + i \sin\theta)^2 \\
&= (\cos^2\theta - \sin^2 \theta) + 2i \sin\theta\cos\theta
\end{align*}
Now simply compare the real and imaginary parts.
\end{Answer}

\begin{Exercise}
Evaluate
\begin{enumerate}[label=(\alph*)]
\item $(1+i)(3-2i)$,
\item $\overline{(2-i)/(4+i)}$,
\item $(3+5i)\overline{(1+i)/(2-3i)}$
\end{enumerate}
as well as their modulus and argument.
\end{Exercise}
\begin{Answer}
\begin{enumerate}[label=(\alph*)]
\item $5+i$, $\sqrt{26}$, $0.1974$;
\item $\frac{7}{17}+\frac{6}{17}i$, $\sqrt{\frac{5}{17}}$, $0.7086$;
\item $\frac{22}{13}-\frac{20}{13}i$, $\frac{2\sqrt{17}}{\sqrt{13}}$, $-0.7378$. (All arguments are in radian.)
\end{enumerate}
\end{Answer}

\begin{Exercise}
For $\vec{u} = (1+i, 2-i, 3)^T$, $\vec{v} = (2+i, 1-2i, i)^T$, and $\vec{w} = (-i, 3, 1-i)^T$, find
\begin{enumerate}[label=(\alph*)]
\item $\vec{u} \cdot \vec{v}$;
\item $(\vec{u} + \vec{v}) \cdot (\vec{u} - \vec{w})$;
\item $\norm{\vec{u}} \vec{v} - \norm{\vec{v}} \vec{w}$.
\end{enumerate}
\end{Exercise}
\begin{Answer}
\begin{enumerate}[label=(\alph*)]
\item $7-i$;
\item $(3+2i, 3-3i, 3+i)^T \cdot (1+2i,-1-i,2+i)^T = 14+i$;
\item $4(2+i, 1-2i, i)^T - \sqrt{11}(-i, 3, 1-i)^T \\
= (8+(4+\sqrt{11})i, (4-3\sqrt{11})-8i, -\sqrt{11}+(4+\sqrt{11})i)^T $
\end{enumerate}
\end{Answer}

\begin{Exercise}
For the two complex matrices below,
\begin{align*}
& A=
\begin{bmatrix}
1+i & -i & 3 \\
0 & 2-i & 1 \\
-1 & i & 2
\end{bmatrix}
& B=
\begin{bmatrix}
1 & 2-i & i \\
-i & 3+i & 1-i \\
0 & 1 & 2i
\end{bmatrix}
\end{align*}
compute $AB^*$, and verify $(AB^*)^* = BA^*$.
\end{Exercise}
\begin{Answer}
\begin{align*}
AB^* &=
\begin{bmatrix}
1+i & -i & 3 \\
0 & 2-i & 1 \\
-1 & i & 2
\end{bmatrix}
\begin{bmatrix}
1 & i & 0 \\ 
2+i & 3-i & 1 \\ 
-i & 1+i & -2i
\end{bmatrix} \\
&=
\begin{bmatrix}
2-4i&1+i&-7i\\ 
5-i&6-4i&2-3i\\ 
-2&3+4i&-3i
\end{bmatrix}
\end{align*}
and
\begin{align*}
BA^* &= 
\begin{bmatrix}
1 & 2-i & i \\
-i & 3+i & 1-i \\
0 & 1 & 2i
\end{bmatrix}
\begin{bmatrix}
1-i & 0 & -1\\
i & 2+i & -i\\
3 & 1 & 2
\end{bmatrix} \\
&= 
\begin{bmatrix}
2+4i&5+i&-2\\ 
1-i&6+4i&3-4i\\ 
7i&2+3i&3i
\end{bmatrix} = (AB^*)^*
\end{align*}
\end{Answer}

\begin{Exercise}
For the matrix
\begin{align*}
A=
\begin{bmatrix}
1-4i&-3i&2+i\\ 
2-3i&0&4i\\ 
-2&1&3-i
\end{bmatrix}    
\end{align*}
find its determinant and inverse.
\end{Exercise}
\begin{Answer}
\begin{align*}
\det(A) &= i \\
A^{-1} &= 
\begin{bmatrix}
-4&10-5i&-12i\\ 
3i+3&-11-3i&-8+9i\\ 
-3-2i&10+i&6-9i
\end{bmatrix}
\end{align*}    
\end{Answer}

\begin{Exercise}
\phantomsection
\label{ex:schurinvD}
Prove the formulae in Properties \ref{proper:schurinvD}, by noting that
\begin{align*}
\begin{bmatrix}
I_p & [\textbf{0}]_{p \times q} \\
-D^{-1}C & D^{-1}
\end{bmatrix}  
=
\begin{bmatrix}
I_p & [\textbf{0}]_{p \times q} \\
[\textbf{0}]_{q \times p} & D^{-1}
\end{bmatrix} 
\begin{bmatrix}
I_p & [\textbf{0}]_{p \times q} \\
-C & I_q
\end{bmatrix} 
\end{align*}
and
\begin{align*}
\begin{bmatrix}
A & B \\
C & D
\end{bmatrix}
\begin{bmatrix}
I_p & [\textbf{0}]_{p \times q} \\
-D^{-1}C & D^{-1}
\end{bmatrix} =
\begin{bmatrix}
A - BD^{-1}C & BD^{-1} \\
[\textbf{0}]_{q \times p} & I_q
\end{bmatrix}
\end{align*}
\end{Exercise}
\begin{Answer}
Notice that
\begin{align*}
\begin{bmatrix}
I_p & -BD^{-1} \\
[\textbf{0}]_{q \times p} & I_q
\end{bmatrix}
\begin{bmatrix}
A - BD^{-1}C & BD^{-1} \\
[\textbf{0}]_{q \times p} & I_q
\end{bmatrix}
=
\begin{bmatrix}
A - BD^{-1}C & [\textbf{0}]_{p \times q} \\
[\textbf{0}]_{q \times p} & I_q
\end{bmatrix}
\end{align*}
then
\begin{align*}
&\begin{bmatrix}
I_p & -BD^{-1} \\
[\textbf{0}]_{q \times p} & I_q
\end{bmatrix}
\begin{bmatrix}
A & B \\
C & D
\end{bmatrix}
\begin{bmatrix}
I_p & [\textbf{0}]_{p \times q} \\
-D^{-1}C & D^{-1}
\end{bmatrix} \\
=&
\begin{bmatrix}
I_p & -BD^{-1} \\
[\textbf{0}]_{q \times p} & I_q
\end{bmatrix}
\begin{bmatrix}
A - BD^{-1}C & BD^{-1} \\
[\textbf{0}]_{q \times p} & I_q
\end{bmatrix} =
\begin{bmatrix}
A - BD^{-1}C & [\textbf{0}]_{p \times q} \\
[\textbf{0}]_{q \times p} & I_q
\end{bmatrix}
\end{align*}
The determinant formula can easily be derived from this equation. Moreover,
\begin{align*}
& \begin{bmatrix}
A & B \\
C & D
\end{bmatrix}^{-1} \\
=& 
\left(
\begin{bmatrix}
I_p & -BD^{-1} \\
[\textbf{0}]_{q \times p} & I_q
\end{bmatrix}^{-1}
\begin{bmatrix}
A - BD^{-1}C & [\textbf{0}]_{p \times q} \\
[\textbf{0}]_{q \times p} & I_q
\end{bmatrix}
\begin{bmatrix}
I_p & [\textbf{0}]_{p \times q} \\
-D^{-1}C & D^{-1}
\end{bmatrix}^{-1}\right)^{-1} \\
=& 
\begin{bmatrix}
I_p & [\textbf{0}]_{p \times q} \\
-D^{-1}C & D^{-1}
\end{bmatrix}
\begin{bmatrix}
(A - BD^{-1}C)^{-1} & [\textbf{0}]_{p \times q} \\
[\textbf{0}]_{q \times p} & I_q
\end{bmatrix}
\begin{bmatrix}
I_p & -BD^{-1} \\
[\textbf{0}]_{q \times p} & I_q
\end{bmatrix} \\
=& 
\begin{bmatrix}
(A - BD^{-1}C)^{-1} & [\textbf{0}]_{p \times q} \\
-D^{-1}C(A - BD^{-1}C)^{-1} & D^{-1}
\end{bmatrix}
\begin{bmatrix}
I_p & -BD^{-1} \\
[\textbf{0}]_{q \times p} & I_q
\end{bmatrix} \\
=& 
\begin{bmatrix}
(A - BD^{-1}C)^{-1} & -(A - BD^{-1}C)^{-1}BD^{-1} \\
-D^{-1}C(A - BD^{-1}C)^{-1} & D^{-1} + D^{-1}C(A - BD^{-1}C)^{-1}BD^{-1}
\end{bmatrix} \\
=&
\begin{bmatrix}
(M/D)^{-1} & -(M/D)^{-1}BD^{-1}  \\
-D^{-1}C (M/D)^{-1} & D^{-1} + D^{-1}C (M/D)^{-1} BD^{-1}
\end{bmatrix}
\end{align*}
\end{Answer}

\begin{Exercise}
Write down the direct sum of the following three matrices.
\begin{align*}
A &= 
\begin{bmatrix}
2 & 1 \\
0 & 4 \\
-1 & 3
\end{bmatrix} &
C &= 
\begin{bmatrix}
1 & 4 & 0 & -3 \\
0 & 2 & -1 & 1
\end{bmatrix} \\
B &= \begin{bmatrix}
1
\end{bmatrix}
\end{align*}
\end{Exercise}
\begin{Answer}
\begin{align*}
A \oplus B \oplus C =
\begin{bmatrix}
2 & 1 & 0 & 0 & 0 & 0 & 0\\
0 & 4 & 0 & 0 & 0 & 0 & 0\\
-1 & 3 & 0 & 0 & 0 & 0 & 0\\
0 & 0 & 1 & 0 & 0 & 0 & 0\\
0 & 0 & 0 & 1 & 4 & 0 & -3 \\
0 & 0 & 0 & 0 & 2 & -1 & 1
\end{bmatrix}
\end{align*}
\end{Answer}

\begin{Exercise}
Show that given two bases $\mathcal{\beta} = \{\cos x, \sin x, 1, x, x^2\}$ and $\mathcal{\eta} = \{\cos x, \sin x, 1, x\}$ which generate vector spaces $\mathcal{U}$ and $\mathcal{V}$ respectively, the differentiation operator $T(f(x)) = f'(x): \mathcal{U} \to \mathcal{V}$ has a $2 \times 2$ block matrix direct sum representation.
\end{Exercise}
\begin{Answer}
It is simply
\begin{align*}
[T]_{\beta}^{\eta} &=
\begin{bmatrix}
0 & 1 & 0 & 0 &0\\
-1 & 0 & 0 & 0 & 0\\
0 & 0 & 0 & 1 & 0\\
0 & 0 & 0 & 0 & 2
\end{bmatrix} \\
&=
\begin{bmatrix}
0 & 1 \\
-1 & 0 
\end{bmatrix}
\oplus
\begin{bmatrix}
0 & 1 & 0\\    
0 & 0 & 2
\end{bmatrix}
\end{align*}
\end{Answer}
\chapter{Eigenvalues and Eigenvectors}

In this section we will discuss a very important topic in Linear Algebra, the \textit{eigenvalue-eigenvector} problem. By finding the eigenvectors of a square matrix which span subspaces that are \textit{invariant} under the corresponding linear operator, it is sometimes possible to obtain a coordinate basis such that the matrix can be \textit{diagonalized}, i.e. become a diagonal matrix under that particular change of coordinates. One of the practical usages of \textit{diagonalization} is to solve systems of linear ordinary differential equations (ODEs) which is also commonly seen in many areas of Earth Science. In the end of this chapter, we will build up from the idea of invariant subspaces and introduce the concept of \textit{cyclic subspaces}, leading to a famous related result called the \textit{Cayley-Hamilton Theorem}.

\section{Eigenvalues and Eigenvectors of a Square Matrix}
\label{section:eigensection}

\subsection{Definition of Eigenvalues and Eigenvectors}

Consider a linear operator/endomorphism $T: \mathcal{V} \to \mathcal{V}$, an interesting question is about if a vector $\vec{v} \in \mathcal{V}$ under this mapping will remain stationary in direction such that the image $T(\vec{v}) = \lambda v$ is a scalar multiple of the original vector, or in other words, the effect of $T$ on $\vec{v}$ is simply a rescaling. In this situation, the vector $\vec{v}$ is known as an \keywordhl{eigenvector} of $T$ and the factor $\lambda$ is the corresponding \keywordhl{eigenvalue}. Since a linear operator is a mapping between a vector space itself, it has a square matrix representation under any basis. This fact extends the ideas of eigenvalues and eigenvectors to square matrices.

\begin{defn}
\label{defn:eigen}
Given a linear operator $T: \mathcal{V} \to \mathcal{V}$, we call $\lambda$ and $\vec{v}_\lambda$ its eigenvalue and eigenvector if
\begin{align*}
T(\vec{v}_\lambda) = \lambda\vec{v}_\lambda
\end{align*}
Similarly, given an $n \times n$ square matrix $A$, $\lambda$ and $\vec{v}_\lambda$ will be an eigenvalue and eigenvector for it when
\begin{align*}
A\vec{v}_\lambda = \lambda\vec{v}_\lambda
\end{align*}
This is a special case in which a vector space $\mathcal{V}$ is finite-dimensional, $\dim(\mathcal{V}) = n$, and $A = [T]_B$ is just the matrix representation of $T$ with respect to some basis $\mathcal{B}$.
\end{defn}
Notice that there can be more than one eigenvalues and eigenvectors.
An example is given by the matrix
\begin{align*}
A =
\begin{bmatrix}
1 & \frac{1}{2} \\
2 & 1
\end{bmatrix}
\end{align*}
It can be seen that the vector $\vec{v}_1 = (1,2)^T$ is an eigenvector of $A$, as
\begin{align*}
\begin{bmatrix}
1 & \frac{1}{2} \\
2 & 1
\end{bmatrix}
\begin{bmatrix}
1 \\
2
\end{bmatrix}
=
\begin{bmatrix}
2 \\
4
\end{bmatrix}
=
2
\begin{bmatrix}
1 \\
2 
\end{bmatrix}
\end{align*}
that corresponds to an eigenvalue of $\lambda = 2$. Meanwhile, $\vec{v}_2 = (1,-2)^T$ is another eigenvector that has an eigenvalue of $\lambda = 0$, since
\begin{align*}
\begin{bmatrix}
1 & \frac{1}{2} \\
2 & 1
\end{bmatrix}
\begin{bmatrix}
1 \\
-2
\end{bmatrix}
=
\begin{bmatrix}
0 \\
0
\end{bmatrix}
=
0
\begin{bmatrix}
1 \\
-2
\end{bmatrix}
\end{align*}
We emphasize that a zero eigenvalue is perfectly valid. \par
Short Exercise: Prove that all vectors in form of $s(1,2)^T$, where $s$ is any number, are eigenvectors for the matrix $A$ above with $\lambda = 2$.\footnote{
$
\begin{bmatrix}
1 & \frac{1}{2} \\
2 & 1
\end{bmatrix}
\left(s
\begin{bmatrix}
1 \\
2 
\end{bmatrix}\right)
=
s\begin{bmatrix}
1 & \frac{1}{2} \\
2 & 1
\end{bmatrix}
\begin{bmatrix}
1 \\
2 
\end{bmatrix}
=
s
\begin{bmatrix}
2 \\
4
\end{bmatrix}
=
2\left(s
\begin{bmatrix}
1 \\
2 
\end{bmatrix}\right)
$. In general, if $A\vec{v}_\lambda = \lambda\vec{v}_\lambda$ so that $\vec{v}_\lambda$ is some non-zero eigenvector, then $A(s\vec{v}_\lambda) = sA\vec{v}_\lambda = s\lambda\vec{v}_\lambda = \lambda(s\vec{v}_\lambda)$ and therefore all of its non-zero scalar multiples $s\vec{v}_\lambda$ is also an eigenvector.}
\begin{center}
\begin{tikzpicture}
\draw[->] (-2,0)--(2,0) node[right]{$x$};
\draw[->] (0,-2)--(0,2) node[above]{$y$};
\draw[blue,-stealth,line width=1.5] (0,0)--(1,2) node[align=left, right]{$A\vec{v}_\lambda = \lambda\vec{v}_\lambda$\\ $= 2(1,2)^T = (2,4)^T$};
\draw[red,-stealth,line width=1.5] (0,0)--(1/2,1) node[anchor=west]{$\vec{v}_\lambda = (1,2)^T$};
\node[below left]{$O$}; 
\end{tikzpicture}
\begin{tikzpicture}
\draw[->] (-2,0)--(2,0) node[right]{$x$};
\draw[->] (0,-2)--(0,2) node[above]{$y$};
\draw[red,-stealth,line width=1.5] (0,0)--(1/2,-1) node[anchor=west]{$\vec{v}_\lambda = (1,-2)^T$};
\draw[blue, fill=blue] (0,0) circle[radius=2pt] node[align=left, above]{$A\vec{v}_\lambda = \lambda\vec{v}_\lambda$\\ $= 0(1,-2)^T = (0,0)^T$};
\node[below left]{$O$}; 
\end{tikzpicture}\\
Illustrations for the example above with $\lambda = 2 > 1$ (Extension), and $\lambda = 0$ (Vanished).
\end{center}
There are infinitely many eigenvectors which are oriented in the same direction for a single eigenvalue as seen in the remark of the last short exercise. Particularly, they are actually the span of any one of these eigenvectors that is non-zero. Thus, along a single direction, only one of them is needed for representation, and its span is at the same time a subspace by Properties \ref{proper:subspace_n_span}. This subspace is known as the \keywordhl{eigenspace} corresponding to that eigenvalue. Moreover, there may be more than one linearly independent eigenvectors for the same eigenvalue, and the dimension of eigenspace generated by them will be greater than one as well. In addition, the zero vector, technically, can be the eigenvectors of any matrix since $A\vec{0} = \vec{0} = \lambda\vec{0}$ for any matrix $A$ and scalar $\lambda$. However, it is a trivial solution, plus more importantly the zero vector is always linearly dependent by definition, and will not be taken into consideration (unlike the totally fine eigenvalue of zero).\\
\\
Below is the visualization of some other possibilities of eigenvector rescaling.
\begin{center}
\begin{tikzpicture}
\draw[->] (-1.5,0)--(1.5,0) node[right]{$x$};
\draw[->] (0,-1.5)--(0,1.5) node[above]{$y$};
\draw[red,-stealth] (0,0)--(0.75,1.5);
\draw[blue,-stealth] (0,0)--(0.25,0.5);
\node[below left]{$O$}; 
\end{tikzpicture}
\begin{tikzpicture}
\draw[->] (-1.5,0)--(1.5,0) node[right]{$x$};
\draw[->] (0,-1.5)--(0,1.5) node[above]{$y$};
\draw[red,-stealth] (0,0)--(1,-0.5);
\draw[blue,-stealth] (0,0)--(-0.8,0.4);
\node[below left]{$O$}; 
\end{tikzpicture}\\
Contraction ($0 < \lambda < 1$), Reversal ($\lambda < 0$). \\
\begin{tikzpicture}
\draw[->] (-1.5,0)--(1.5,0) node[right]{$x$};
\draw[->] (0,-1.5)--(0,1.5) node[above]{$y$};
\draw[Green,-stealth] (0,0)--(-1.25,1.25);
\node[below left]{$O$}; 
\end{tikzpicture}
\begin{tikzpicture}
\draw[->] (-1.5,0)--(1.5,0) node[right]{$x$};
\draw[->] (0,-1.5)--(0,1.5) node[above]{$y$};
\draw[red,-stealth] (0,0)--(1,1);
\draw[fill=blue] (0,0) circle[radius=2pt];
\node[below left]{$O$}; 
\end{tikzpicture}\\
Unchanged ($\lambda = 1$), Vanished ($\lambda = 0$). 
\end{center}

The eigenspace actually belongs to a broader class of subspaces known as the invariant subspaces.

\subsection{Finding Eigenvalues and Eigenvectors with Characteristic Polynomials}

To find eigenvalues, rearrange the equation in Definition \ref{eigen} relating eigenvalues and eigenvectors to obtain
\begin{align*}
A\vec{v}_\lambda &= \lambda\vec{v}_\lambda \\
A\vec{v}_\lambda &= \lambda I\vec{v}_\lambda  &\text{($I\vec{u} = \vec{u}$ for any $\vec{u}$)} \\
(A-\lambda I)\vec{v}_\lambda &= \vec{0}
\end{align*}
The last line constitutes a homogeneous linear system $B\vec{v}_\lambda = \vec{0}$ where $B = A - \lambda I$. For this system to have a non-trivial solution and hence an eigenvector, it is required that $\det(B) = \det(A - \lambda I) = 0$ from Theorem \ref{LinSysUnique}. The relationship $\det(A - \lambda I) = 0$ is called the characteristic equation. The roots for $\lambda$ of the characteristic polynomial are then the desired eigenvalues. \\
Short Exercise: By inspection, find all three eigenvalues of the matrix
\begin{align*}
\begin{bmatrix}
1 & 0 & 0 \\
0 & 2 & 0 \\
0 & 0 & 3
\end{bmatrix}
\end{align*}
For each eigenvalue there corresponds at least one eigenvectors. The number of eigenvectors for an eigenvalue depend on the number of the root appearing in the characteristic equation. The eigenvectors are then the general solution of the matrix equation $B\vec{v}_\lambda = (A - \lambda I)\vec{v}_\lambda = \vec{0}$, which exists because of the condition $\det(A - \lambda I) = 0$. The number of eigenvectors obeys the following theorem.
\begin{thm}
For each particular eigenvalue $\lambda_j$ for a matrix $A$ computed from its characteristic equation $\det(A - \lambda I) = 0$, the amount of $\lambda_j$ appearing as its root, or equivalently the power $n$ of the factor $(x-\lambda_j)^n$ in the characteristic polynomial, is called the algebraic multiplicity. \\
\\
Meanwhile, the amount of eigenvectors $\vec{v}_{\lambda_j}$ corresponding to $\lambda_j$ is called the geometric multiplicity. With these two quantities defined, we have
\begin{align*}
1 \leq \text{Geometric Multiplicity} \leq \text{Algebraic Multiplicity}
\end{align*}
for every eigenvalue $\lambda_j$.
\end{thm}

\begin{exmp}
Find all eigenvalues and eigenvectors for the matrix
\begin{align*}
A &=
\begin{bmatrix}
1 & -1 \\
0 & 1
\end{bmatrix}
\end{align*}
The characteristic equation is
\begin{align*}
\det(A - \lambda I) &= 
\begin{vmatrix}
1-\lambda & -1 \\
0 & 1-\lambda
\end{vmatrix} \\
&= (1-\lambda)^2 = 0
\end{align*}
Apparently, there is only one eigenvalue $\lambda = 1$, which has an algebraic multiplicity of $2$. Possible eigenvectors are then found by solving
\begin{align*}
\left[\begin{array}{@{}cc|c@{}}
1-1 & -1 & 0 \\
0 & 1-1 & 0
\end{array}\right] 
= 
\left[\begin{array}{@{}cc|c@{}}
0 & -1 & 0 \\
0 & 0 & 0
\end{array}\right]
\end{align*}
where the general solution is easily seen to be $t(1,0)^T$. So for the eigenvalue $\lambda = 1$, there is only one eigenvector $(1,0)^T$, which implies a geometric multiplicity of $1$.
\end{exmp}

\begin{exmp}
For the matrix
\begin{align*}
A &= 
\begin{bmatrix}
1 & 3 & 1 \\
0 & 1 & 0 \\
1 & 0 & 2
\end{bmatrix}
\end{align*}
the characteristic polynomial is
\begin{align*}
\begin{vmatrix}
1-\lambda & 3 & 1 \\
0 & 1-\lambda & 0 \\
1 & 0 & 2-\lambda 
\end{vmatrix} &=
(1-\lambda)(1-\lambda)(2-\lambda) - (1)(1-\lambda)(1) \\
&= (1-\lambda)((2-3\lambda+\lambda^2) - 1) \\
&= (1-\lambda)(1-3\lambda+\lambda^2)
\end{align*}
The roots and thus eigenvalues are $\lambda = 1$, as well as
\begin{align*}
\lambda &= \frac{-(-3) \pm \sqrt{(-3)^2 - 4(1)(1)}}{2} \\
&= \frac{3}{2} \pm \frac{\sqrt{5}}{2}
\end{align*}
Particularly, for the eigenvalue $\lambda = \frac{3}{2} + \frac{\sqrt{5}}{2}$, the eigenvector is inferred from the homogeneous system
\begin{align*}
\left[\begin{array}{@{}ccc|c@{}}
-\frac{1}{2}-\frac{\sqrt{5}}{2} & 3 & 1 & 0 \\
0 & -\frac{1}{2}-\frac{\sqrt{5}}{2} & 0 & 0 \\
1 & 0 & \frac{1}{2}-\frac{\sqrt{5}}{2} & 0
\end{array}\right] &\to
\left[\begin{array}{@{}ccc|c@{}}
1 & 0 & \frac{1}{2}-\frac{\sqrt{5}}{2} & 0 \\
0 & 1 & 0 & 0 \\
-\frac{1}{2}-\frac{\sqrt{5}}{2} & 3 & 1 & 0
\end{array}\right]  \\
&\to
\left[\begin{array}{@{}ccc|c@{}}
1 & 0 & \frac{1}{2}-\frac{\sqrt{5}}{2} & 0 \\
0 & 1 & 0 & 0 \\
0 & 0 & 0 & 0
\end{array}\right]
\end{align*}
whose general solution and hence the eigenvector is $(-\frac{1}{2}+\frac{\sqrt{5}}{2},0,1)^T$ for $\lambda = \frac{3}{2} + \frac{\sqrt{5}}{2}$.\\
Short Exercise: Find the eigenvectors for other remaining eigenvalues.
\end{exmp}

This section is ended with some notable properties of eigenvalue and eigenvector.
\begin{proper}
For a square matrix $A$,
\begin{enumerate}
\item $A^T$ shares the same eigenvalues, but for each eigenvalue the eigenvector is not guaranteed to be the same,
\item The eigenvalues for the inverse $A^{-1}$, provided that it exists, are the reciprocals of the eigenvalues of $A$, but the eigenvectors are the same. This can be proved by starting with $A\vec{v}_\lambda = \lambda\vec{v}_\lambda$, and multiplying to the left on both sides by $A^{-1}$.
\end{enumerate}
\end{proper}

\subsection*{Cayley-Hamilton Theorem}
Here we introduce an important theorem, Cayley-Hamilton Theorem, stating that, for every square matrix, it satisfies its own characteristic equation, which means that substituting the matrix into the characteristic polynomial as the variable results in a zero matrix.
\begin{thm}
By Cayley-Hamilton Theorem, for any $n \times n$ square matrix $A$, if the characteristic polynomial is
\begin{align*}
p(\lambda) = \det(A-\lambda I) = \sum_{k=0}^{n} c_k \lambda^k
\end{align*}
then we have
\begin{align*}
p(A) = \sum_{k=0}^{n} c_k A^k = [\textbf{0}]
\end{align*}
which is a $n \times n$ zero matrix.
\end{thm}
One may be tempted to substitute $\lambda = A$ into $\det(A-\lambda I)$ to prove the Cayley-Hamilton Theorem. However, since $\lambda$ is a scalar but $A$ is a matrix, it is not a rigorous proof. Correct proofs require advanced knowledge, which will not be presented here.

\begin{exmp}
With the matrix
\begin{align*}
A = 
\begin{bmatrix}
1 & -1 \\
3 & 5
\end{bmatrix}
\end{align*}
verify the Cayley-Hamilton Theorem, and use the Cayley-Hamilton Theorem to evaluate $A^2 - 7A + 6I$.\\
\\
The characteristic polynomial is
\begin{align*}
\begin{vmatrix}
1-\lambda & -1 \\
3 & 5-\lambda
\end{vmatrix}  
&= (1-\lambda)(5-\lambda) - (3)(-1) \\
&= 5 - 6 \lambda + \lambda^2 + 3 \\
&= \lambda^2 - 6\lambda + 8
\end{align*}
Replacing all $\lambda^k$ terms in the characteristic polynomial with $A^k$ (Notice that the constant term $c_0$ becomes $c_0 I$), we have
\begin{align*}
A^2 - 6A + 8I &= 
\begin{bmatrix}
1 & -1 \\
3 & 5
\end{bmatrix}^2
- 6
\begin{bmatrix}
1 & -1 \\
3 & 5
\end{bmatrix} 
+ 8
\begin{bmatrix}
1 & 0 \\
0 & 1
\end{bmatrix} \\
&=
\begin{bmatrix}
-2 & -6 \\
18 & 22
\end{bmatrix}
+
\begin{bmatrix}
-6 & 6 \\
-18 & -30
\end{bmatrix} 
+
\begin{bmatrix}
8 & 0 \\
0 & 8
\end{bmatrix} \\
&=
\begin{bmatrix}
0 & 0\\
0 & 0
\end{bmatrix}
\end{align*}
So Cayley-Hamilton Theorem holds in this case. We can quickly compute $A^2 - 7A + 6I$ by
\begin{align*}
A^2 - 7A + 6I &= (A^2 - 7A + 6I) - (A^2 - 6A + 8I) \\
&= -A-2I \\
&= -\begin{bmatrix}
1 & -1 \\
3 & 5
\end{bmatrix} 
-2
\begin{bmatrix}
1 & 0 \\
0 & 1
\end{bmatrix} \\
&=
\begin{bmatrix}
-3 & 1 \\
-3 & -7
\end{bmatrix}
\end{align*}
since $A^2 - 6A + 8I$ is a zero matrix.
\end{exmp}

\section{Diagonalization}

\subsection{Ideas and Properties of Diagonalization}
The properties of eigenvectors allow us to carry out diagonalization which helps us to solve linear algebra related problem. A matrix $P$ is said to diagonalize another matrix $A$ if the product $P^{-1}AP$ results in a diagonal matrix $D$, where non-zero entries are only found along the main diagonal.
\begin{defn}
A square matrix $A$ is diagonalizable, if there exists some invertible square matrix $P$, such that
\begin{align*}
P^{-1}AP = D
\end{align*}
where $D$ is a diagonal matrix.
\end{defn}
Particularly, the matrix $P$ required for diagonalizing matrix $A$ is formed by combining all the eigenvectors of $A$ column by column. This is only possible if the amount of distinct eigenvectors is equal to the size of A.
\begin{proper}
\label{diagonalize}
A $n \times n$ square matrix $A$ can be diagonalized by another matrix $P$, if $A$ has $n$ linearly independent eigenvectors, and the column vectors of $P$ are those eigenvectors. A equivalent condition is that for every eigenvalue, its geometric multiplicity is equal to the algebraic multiplicity. The diagonal entries of $P^{-1}AP = D$, are the eigenvalues $\lambda_j$ corresponding to the eigenvectors $\vec{v}_{\lambda_j}$ in the same column of $P$. 
\paragraph{Proof}
Consider two matrix dot products $AP$ and $PD$, where $P = [\vec{v}_{\lambda_1}|\cdots|\vec{v}_{\lambda_n}]$
\begin{align*}
AP &= A[\vec{v}_{\lambda_1}|\cdots|\vec{v}_{\lambda_n}] \\
&= [A\vec{v}_{\lambda_1}|\cdots|A\vec{v}_{\lambda_n}] \\
&= [\lambda_1\vec{v}_{\lambda_1}|\cdots|\lambda_n \vec{v}_{\lambda_n}]
\end{align*}
where we have used the Definition \ref{eigen} and the second step can be compared to the last paragraph in Section \ref{6.1.1}. Also
\begin{align*}
PD &= [\vec{v}_{\lambda_1}|\cdots|\vec{v}_{\lambda_n}]
\begin{bmatrix}
\lambda_1 & \cdots & 0 \\
\vdots & \ddots & \vdots \\
0 & \cdots & \lambda_n
\end{bmatrix} \\
&= [\lambda_1\vec{v}_{\lambda_1}|\cdots|\lambda_n \vec{v}_{\lambda_n}]
\end{align*}
So $AP = PD$. Notice that $P$ is invertible by Theorem \ref{equiv3}, since $P$ is made up of linearly independent eigenvectors, and thus $P^{-1}AP = D$.
\end{proper}

The original matrix $A$ and its diagonalization form $P^{-1}AP$ share some similarities sometimes called invariants. 
\begin{proper}
If there are a diagonalizable matrix $A$, and its diagonalization form $D = P^{-1}AP$, then $A$ and $D$ have the
\begin{enumerate}
\item Same determinant, 
\item Same trace, 
\item Same eigenvalues, 
\item Same characteristic equation.
\end{enumerate}
\end{proper}
Short Exercise: Prove the invariant property for determinant.

\subsection{Diagonalization for Real Eigenvalues}

For a diagonalizable matrix with real eigenvalues, diagonalization is straight forward by the use of Properties \ref{diagonalize}. Below shows a simple example.
\begin{exmp}
For the matrix 
\begin{align*}
A &= 
\begin{bmatrix}
3 & -1 & 1 \\
-2 & 4 & 2 \\
-1 & 1 & 5
\end{bmatrix}
\end{align*}
It is given that its eigenvectors are $(1,1,0)^T, (1,0,1)^T, (0,1,1)^T$ for $\lambda = 2,4,6$ respectively. Concatenating them column by column yields
\begin{align*}
P &=
\begin{bmatrix}
1 & 1 & 0 \\
1 & 0 & 1 \\
0 & 1 & 1
\end{bmatrix}
\end{align*}
The matrix product
\begin{align*}
D &= P^{-1}AP \\
&=
\begin{bmatrix}
1 & 1 & 0 \\
1 & 0 & 1 \\
0 & 1 & 1
\end{bmatrix}^{-1}
\begin{bmatrix}
3 & -1 & 1 \\
-2 & 4 & 2 \\
-1 & 1 & 5
\end{bmatrix}
\begin{bmatrix}
1 & 1 & 0 \\
1 & 0 & 1 \\
0 & 1 & 1
\end{bmatrix} \\
&=
\begin{bmatrix}
\frac{1}{2} & \frac{1}{2} & -\frac{1}{2} \\
\frac{1}{2} & -\frac{1}{2} & \frac{1}{2} \\
-\frac{1}{2} & \frac{1}{2} & \frac{1}{2}
\end{bmatrix}
\begin{bmatrix}
2 & 4 & 0 \\
2 & 0 & 6 \\
0 & 4 & 6
\end{bmatrix} \\
&=
\begin{bmatrix}
2 & 0 & 0 \\
0 & 4 & 0 \\
0 & 0 & 6
\end{bmatrix}
\end{align*}
This is expected from Properties \ref{diagonalize}.\\
Short Exercise: Confirm the provided eigenvalues and eigenvectors. Also, reverse the diagonalization to recover the original matrix.\\
Short Exercise: Verify the invariant properties for this example.
\end{exmp}

\subsection{Diagonalization for Complex Eigenvalues}
It is not uncommon for a real square matrix $A$ to have complex eigenvalues. In such cases, to perform diagonalization, one possible approach is following what we have done in the last section to generate $D = P^{-1}AP$, which can be useful sometimes, but the downsides are that $P$ and $D$ are comprised of complex numbers, despite $A$ being a real matrix. There is, indeed, another method uses the property of complex numbers introduced in the last chapter, that avoids the appearance of complex numbers. But first of all, we need to introduce a basic theorem about complex roots of an equation.
\begin{thm}
For a real polynomial equation with order $n$
\begin{align*}
p(x) = \sum_{k=0}^{n} c_k x^k
\end{align*}
If $x_0 = a+b\imath$ is a complex root so that $p(x_0) = \sum_{k=0}^{n} c_k x_0^k = 0$, where $a$ and $b$ are real constants, then
$\overline{x_0} = a-b\imath$ is also a root for the equation.
\paragraph{Proof}
By Properties \ref{complexnum}, if we take the complex conjugate on both sides of the polynomial equation with $x = x_0$, then
\begin{align*}
\overline{p(x_0)} &= \overline{\sum_{k=0}^{n} c_k x_0^k} = \overline{0} \\
{p(\overline{x_0})} &= \sum_{k=0}^{n} c_k \overline{x_0}^k = 0
\end{align*}
\end{thm}
Since the characteristic equation is a real polynomial equation for the real matrix $A$, by the theorem we have just proved, we know that complex roots for the characteristic equation and hence complex eigenvalues always come in a conjugate pair. For a pair of complex eigenvalues, their eigenvectors are also the conjugate of each other.
\begin{proper}
If $\lambda_0$ is an eigenvalue for a real matrix $A$ with an eigenvector of $\vec{v}_{\lambda_0}$, then $A$ also has $\overline{\lambda_0}$ and $\overline{\vec{v}_{\lambda_0}}$ as another eigenvalue and eigenvector.
\paragraph{Proof}
By definition,
\begin{align*}
A\vec{v}_{\lambda_0} &= \lambda_0\vec{v}_{\lambda_0}    
\end{align*}
Taking complex conjugate on both sides, we have
\begin{align*}
\overline{A\vec{v}_{\lambda_0}} &= \overline{\lambda_0\vec{v}_{\lambda_0}} \\
A \overline{\vec{v}_{\lambda_0}} &= \overline{\lambda_0}\; \overline{\vec{v}_{\lambda_0}}
\end{align*}
with the use of Properties \ref{complexnum}, and noting that $\overline{A} = A$ as $A$ is a real matrix.
\end{proper}
Now as we know that complex eigenvalues and eigenvectors always appear as a pair of complex conjugates, and conjugates share the same real and imaginary part except a sign difference, we are encouraged to use their real and imaginary part like two eigenvalues and eigenvectors for making up. The following theorem shows that it is possible to do so with some tweaks.

\begin{thm}
\label{diagonalize2}
The procedure in Properties \ref{diagonalize} can be extended for complex eigenvalue $\lambda_0 = \Re{\lambda_0} + \imath \Im{\lambda_0}$, the corresponding eigenvector $\vec{v}_{\lambda_0} = \Re{\vec{v}_{\lambda_0}} + \imath \Im{\vec{v}_{\lambda_0}}$, as well as their complex conjugates, by replacing the corresponding columns in
\begin{align*}
&P = [\cdots|\vec{v_{\lambda_0}}|\overline{\vec{v_{\lambda_0}}}|\cdots]
&D =
\begin{bmatrix}
\ddots & 0 & 0 & \\
& \lambda_0 & 0 & \\
& 0 & \overline{\lambda_0} & \\
& 0 & 0 & \ddots
\end{bmatrix}
\end{align*}
by
\begin{align*}
&P = [\cdots|\Re{\vec{v}_{\lambda_0}}|\Im{\vec{v}_{\lambda_0}}|\cdots]
&D =
\begin{bmatrix}
\ddots & 0 & 0 & \\
& \Re{\lambda_0} & -\Im{\lambda_0} & \\
& \Im{\lambda_0} & \Re{\lambda_0} & \\
& 0 & 0 & \ddots
\end{bmatrix}
\end{align*}
\paragraph{Proof}
As in the proof for Properties \ref{diagonalize}, we set to prove that $AP = PD$ for the columns concerned. To make it easier to read, denote $\Re{\lambda_0} = a$ and $\Im{\lambda_0} = b$. It is easy to see that
\begin{align*}
PD &=
[\cdots|\Re{\vec{v}_{\lambda_0}}|\Im{\vec{v}_{\lambda_0}}|\cdots]
\begin{bmatrix}
\ddots & 0 & 0 & \\
& a & b & \\
& -b & a & \\
& 0 & 0 & \ddots
\end{bmatrix} \\
&= [\cdots|a\Re{\vec{v}_{\lambda_0}} - b\Im{\vec{v}_{\lambda_0}} | b\Re{\lambda_0} + a\Im{\vec{v}_{\lambda_0}}|\cdots]
\end{align*}
Notice that the relation between eigenvalue and eigenvector can be expanded as
\begin{align*}
A\vec{v}_{\lambda_0} &= \lambda_0\vec{v}_{\lambda_0} \\
A(\Re{\vec{v}_{\lambda_0}} + \imath \Im{\vec{v}_{\lambda_0}}) &= (a+b\imath)(\Re{\vec{v}_{\lambda_0}} + \imath \Im{\vec{v}_{\lambda_0}}) \\
A\Re{\vec{v}_{\lambda_0}} + \imath A\Im{\vec{v}_{\lambda_0}} &= (a\Re{\vec{v}_{\lambda_0}} - b\Im{\vec{v}_{\lambda_0}}) + \imath(b\Re{\vec{v}_{\lambda_0}} + a\Im{\vec{v}_{\lambda_0}})
\end{align*}
By equating real and imaginary parts, we have
\begin{align*}
A\Re{\vec{v}_{\lambda_0}} &= a\Re{\vec{v}_{\lambda_0}} - b\Im{\vec{v}_{\lambda_0}} \\
A\Im{\vec{v}_{\lambda_0}} &= b\Re{\vec{v}_{\lambda_0}} + a\Im{\vec{v}_{\lambda_0}}
\end{align*}
Hence
\begin{align*}
AP &= A[\cdots|\Re{\vec{v}_{\lambda_0}}|\Im{\vec{v}_{\lambda_0}}|\cdots] \\
&= [\cdots|A\Re{\vec{v}_{\lambda_0}}|A\Im{\vec{v}_{\lambda_0}}|\cdots] \\
&= [\cdots|a\Re{\vec{v}_{\lambda_0}} - b\Im{\vec{v}_{\lambda_0}} | b\Re{\lambda_0} + a\Im{\vec{v}_{\lambda_0}}|\cdots] \\
&= PD
\end{align*}
The only caveat is to prove that $P$ is invertible, or equivalently $\Re{\vec{v}_{\lambda_0}}$ and $\Im{\vec{v}_{\lambda_0}}$ are linearly independent. We can prove that by assuming they are linearly dependent, so $\Re{\vec{v}_{\lambda_0}} = k\Im{\vec{v}_{\lambda_0}}$ for some $k$, and plugging this into the expressions of $A\Re{\vec{v}_{\lambda_0}}$ and $A\Im{\vec{v}_{\lambda_0}}$ to derive a contradiction.
\end{thm}

The block in the form
\begin{align*}
\begin{bmatrix}
\ddots & 0 & 0 & \\
& a & b & \\
& -b & a & \\
& 0 & 0 & \ddots
\end{bmatrix}    
\end{align*}
generally represents a rotation by a degree of $\theta = \arctan(b/a)$. It means that a complex eigenvalue entails a rotation. This will be discussed more thoroughly in the next chapter.

\begin{exmp}
\label{ex8.2.2}
Using Theorem \ref{diagonalize2} to convert
\begin{align*}
A &= 
\begin{bmatrix}
1 & 1 \\
-2 & 1
\end{bmatrix}
\end{align*}
into the form $D = P^{-1}AP$.\\
The characteristic equation is
\begin{align*}
\begin{bmatrix}
1-\lambda & 1 \\
-2 & 1-\lambda
\end{bmatrix} 
&=
(1-\lambda)^2 + 2 \\
&= \lambda^2 - 2\lambda + 3 = 0
\end{align*}
The roots are
\begin{align*}
\lambda &= \frac{-(-2) \pm \sqrt{(-2)^2 - 4(1)(3)}}{2} \\
&= 1 \pm \sqrt{2}\imath
\end{align*}
Since the eigenvectors occur in a conjugate pair, we only need to find one of them. We can find the eigenvector for $\lambda = 1 - \sqrt{2}\imath$, by solving the system
\begin{align*}
\left[\begin{array}{@{}cc|c@{}}
\sqrt{2}\imath & 1 & 0 \\
-2 & \sqrt{2}\imath & 0
\end{array}\right] &\to  
\left[\begin{array}{@{}cc|c@{}}
\sqrt{2}\imath & 1 & 0 \\
0 & 0 & 0
\end{array}\right]
\end{align*}
So the eigenvector for $\lambda = 1 - \sqrt{2}\imath$ is $(\imath, \sqrt{2})^T$, and the eigenvector for $\lambda = 1 + \sqrt{2}\imath$ is $(-\imath, \sqrt{2})^T$.
Applying Theorem \ref{diagonalize2}, with $a = 1$, $b = \sqrt{2}$, $\Re{\vec{v}_{\lambda_0}} = (0,\sqrt{2})^T$, $\Im{\vec{v}_{\lambda_0}} = (-1,0)^T$, we have
\begin{align*}
\begin{bmatrix}
1 & \sqrt{2} \\
-\sqrt{2} & 1
\end{bmatrix}
&= 
\begin{bmatrix}
0 & -1 \\
\sqrt{2} & 0
\end{bmatrix}^{-1}
\begin{bmatrix}
1 & 1 \\
-2 & 1
\end{bmatrix}
\begin{bmatrix}
0 & -1 \\
\sqrt{2} & 0
\end{bmatrix}
\end{align*}
\end{exmp}

\section{System of Ordinary Differential Equations}
Ordinary Differential Equations appear frequently in the area of Physics and Earth Science. The easiest class of Ordinary Differential Equations is the first-order, linear, homogeneous Ordinary Differential Equations.
\begin{defn}
A first-order linear ordinary differential equation is a differential equation in the form of
\begin{align*}
\frac{dy}{dx} + P(x)y = Q(x)
\end{align*}
where $x$ is the indepedent variable, $y(x)$ is the dependent variable, $P(x)$ and $Q(x)$ is some function of $x$. 
\end{defn}
First-order means that the highest order derivative involved is the first derivative. Linearity means that there are no cross product terms between the dependent variable and its derivatives, e.g. $y\frac{dy}{dx}, y^2$. Finally, if it is of constant coefficients, and homogeneous, then it implies that $P(x)$ is a constant and $Q(x) = 0$ respectively.
\begin{thm}
\label{ODEsol}
For a first-order, linear, constant coefficients, homogenenous ordinary differential equation in the form of
\begin{align*}
\frac{dy}{dx} = \beta y
\end{align*}
where $\beta$ is a constant, the general solution is
\begin{align*}
y(x) = ce^{\beta x}
\end{align*}
with $c$ as the integration constant. A direct substitution can verify the solution.
\end{thm}
However, for some real-life problems, there are multiple ordinary differential equations that are related to each other to be solved. This especially happens in the area of Earth System Science where we have to consider the interactions, such as mass flow, between different units. This gives rise to a system of ordinary differential equations.
\begin{defn}
A system of ordinary differential equations that are first-order, linear, constant coefficients, and homogeneous has the form of
\begin{align*}
\begin{cases}
dy_1/dx &= \alpha_1 y_1 + \beta_1 y_2 + \gamma_1 y_3 + \cdots \\
dy_2/dx &= \alpha_2 y_1 + \beta_2 y_2 + \gamma_2 y_3 + \cdots \\
dy_3/dx &= \alpha_3 y_1 + \beta_3 y_2 + \gamma_3 y_3 + \cdots \\
\cdots &= \cdots
\end{cases}
\end{align*}
and so on, up to $dy_n/dx$, or more compactly
\begin{align*}
\frac{dy_i}{dx} &= \alpha_i y_1 + \beta_i y_2 + \gamma_i y_3 + \cdots
\end{align*}
or in matrix notation
\begin{align*}
\textbf{y}' &= A\textbf{y}
\end{align*}
where $\alpha_i, \beta_i, \gamma_i, \cdots$ are constants, $A$ is a square $n \times n$ matrix, and
\begin{align*}
&\textbf{y} =
\begin{bmatrix}
y_1 \\
y_2 \\
y_3
\end{bmatrix}
&\textbf{y}' =
\begin{bmatrix}
dy_1/dx \\
dy_2/dx \\
dy_3/dx
\end{bmatrix} \\
&A=
\begin{bmatrix}
\alpha_1 & \beta_1 & \gamma_1 & \cdots \\
\alpha_2 & \beta_2 & \gamma_2 & \\
\alpha_3 & \beta_3 & \gamma_3 &  \\
\vdots & & & \ddots
\end{bmatrix}
\end{align*}
\end{defn}
An example is the system
\begin{align*}
\begin{cases}
dy_1/dx &= 3y_1 - y_2 \\
dy_2/dx &= 2y_1
\end{cases} 
\end{align*}
which can be rewritten into
\begin{align*}
\textbf{y}' =
\begin{bmatrix}
3 & -1 \\
2 & 0
\end{bmatrix}
\textbf{y}
\end{align*}
Since each ordinary differential equations for $dy_i/dx$ now involves multiple dependent variables $y_j$, where $j = 1,2,3,\cdots$, at the right hand side, we cannot directly use the result from Theorem \ref{ODEsol} to solve the system, unless we can find a way to transform the system so that each equation is in terms of a single dependent variable only. Notice that if we make a change of variables such that $\textbf{y} = P\textbf{z}$ and hence $\textbf{y}' = P\textbf{z}'$, then the system $\textbf{y}' = A\textbf{y}$ becomes
\begin{align*}
P\textbf{z}' &= AP\textbf{z} \\
\textbf{z}' &= (P^{-1}AP)\textbf{z}
\end{align*}
if we assume $P$ is invertible. The term $P^{-1}AP$ immediately tells us a hint in the sense that it resembles a diagonalization described in \ref{diagonalize}. If $P^{-1}AP = D$, which is a diagonal matrix, then the system becomes $\textbf{z}' = D\text{z}$, written more explicitly, is
\begin{align*}
\begin{cases}
dz_1/dx &= D_{11}z_1 \\
dz_2/dx &= D_{22}z_2 \\
dz_3/dx &= D_{33}z_3 \\
\cdots &= \cdots
\end{cases}     
\end{align*}
which are all solvable in their own. Subsequently, we have the following conclusion.
\begin{thm}
For a system of ordinary differential equations in the form of 
\begin{align*}
\textbf{y}' &= A\textbf{y}
\end{align*}
where $A$ is a square $n \times n$ matrix with constant entries, it is solvable if $A$ is diagonalizable, and we make the change of variables
\begin{align*}
\textbf{y} = P\textbf{z}
\end{align*}
where $P$ is the eigenvector matrix outlined in Properties \ref{diagonalize}. So that the system becomes
\begin{align*}
\textbf{z}' &= (P^{-1}AP)\textbf{z} = D\textbf{z}
\end{align*}
After solving for $\textbf{z}$, we can recover the required solution by computing $\textbf{y} = P\textbf{z}$.
\end{thm}

\begin{exmp}
Solve the following system of differential equations.
\begin{align*}
\begin{cases}
dy_1/dx &= \frac{5}{2} y_1 + y_2 + \frac{1}{2} y_3 \\
dy_2/dx &= \frac{3}{2} y_1 + 3 y_2 - \frac{1}{2} y_3 \\
dy_3/dx &= \frac{3}{2} y_1 + y_2 + \frac{3}{2} y_3 
\end{cases}    
\end{align*}
The system written in matrix notation is
\begin{align*}
\begin{bmatrix}
dy_1/dx \\
dy_2/dx \\
dy_3/dx
\end{bmatrix}
=
\begin{bmatrix}
\frac{5}{2} & 1 & \frac{1}{2} \\
\frac{3}{2} & 3 & -\frac{1}{2} \\
\frac{3}{2} & 1 & \frac{3}{2}
\end{bmatrix}
\begin{bmatrix}
y_1 \\
y_2 \\
y_3
\end{bmatrix}
\end{align*}
We leave to the readers to check that the eigenvectors for the coefficient matrix are $(-1,1,1)^T$, $(1,-1,1)^T$, $(1,1,1)^T$ for $\lambda = 1, 2, 4$. Hence by the theorem we have just derived, we can make the change of variables
\begin{align*}
\begin{bmatrix}
y_1 \\
y_2 \\
y_3
\end{bmatrix}
=
\begin{bmatrix}
-1 & 1 & 1 \\
1 & -1 & 1 \\
1 & 1 & 1
\end{bmatrix}
\begin{bmatrix}
z_1 \\
z_2 \\
z_3
\end{bmatrix}
\end{align*}
so that the system becomes
\begin{align*}
\begin{bmatrix}
dz_1/dx \\
dz_2/dx \\
dz_3/dx
\end{bmatrix}
&=
\begin{bmatrix}
-1 & 1 & 1 \\
1 & -1 & 1 \\
1 & 1 & 1
\end{bmatrix}^{-1}
\begin{bmatrix}
\frac{5}{2} & 1 & \frac{1}{2} \\
\frac{3}{2} & 3 & -\frac{1}{2} \\
\frac{3}{2} & 1 & \frac{3}{2}
\end{bmatrix}
\begin{bmatrix}
-1 & 1 & 1 \\
1 & -1 & 1 \\
1 & 1 & 1
\end{bmatrix}
\begin{bmatrix}
z_1 \\
z_2 \\
z_3
\end{bmatrix} \\
&=
\begin{bmatrix}
1 & 0 & 0\\
0 & 2 & 0\\
0 & 0 & 4
\end{bmatrix}
\begin{bmatrix}
z_1 \\
z_2 \\
z_3
\end{bmatrix}
\end{align*}
The solution in terms of $z_i$ is $z_1 = c_1e^x$, $z_2 = c_2e^{2x}$, $z_3 = c_3e^{4x}$. The integration constants $c_i$ can be determined by the so-called initial condition. Assume that the initial condition reads $y_1 = 3$, $y_2 = 1$, $y_3 = 5$, for $x = 0$, then substitution into the change of variables gives
\begin{align*}
\begin{bmatrix}
y_1(0) \\
y_2(0) \\
y_3(0)
\end{bmatrix}
&=
\begin{bmatrix}
-1 & 1 & 1 \\
1 & -1 & 1 \\
1 & 1 & 1
\end{bmatrix}
\begin{bmatrix}
z_1(0) \\
z_2(0) \\
z_3(0)
\end{bmatrix} \\
\begin{bmatrix}
3 \\
1 \\
5
\end{bmatrix}
&=
\begin{bmatrix}
-1 & 1 & 1 \\
1 & -1 & 1 \\
1 & 1 & 1
\end{bmatrix}
\begin{bmatrix}
c_1 \\
c_2 \\
c_3
\end{bmatrix}
\end{align*}
It is not hard to see that $c_1 = 1$, $c_2 = 2$, $c_3 = 2$, $z_1 = e^x$, $z_2 = 2e^{2x}$, $z_3 = 2e^{4x}$. So the full solution for $y_i$ is
\begin{align*}
\begin{cases}
y_1 &= -e^{x} + 2e^{2x} + 2e^{4x} \\
y_2 &= e^{x} - 2e^{2x} + 2e^{4x} \\
y_3 &= e^{x} + 2e^{2x} + 2e^{4x}
\end{cases}
\end{align*}
or in vector notation
\begin{align*}
\begin{bmatrix}
y_1 \\
y_2 \\
y_3
\end{bmatrix}
=
e^{x}
\begin{bmatrix}
-1 \\
1 \\
1
\end{bmatrix}
+
e^{2x}
\begin{bmatrix}
2 \\
-2 \\
2
\end{bmatrix}
+
e^{4x}
\begin{bmatrix}
2 \\
2 \\
2
\end{bmatrix}
\end{align*}
Short Exercise: Derive another solution if the initial condition is $y_1(x=0) = 0$, $y_2(x=0) = 4$, $y_3(x=0) = 6$ instead.
\end{exmp}

\begin{exmp}
Solve the system $\textbf{y}' = A\textbf{y}$, where
\begin{align*}
A = 
\begin{bmatrix}
1 & 1 \\
-2 & 1
\end{bmatrix}    
\end{align*}
is given in Example \ref{ex8.2.2}.\\
\\
From the previous work we know that the eigenvectors are $(\imath,\sqrt{2})^T$ and $(-\imath,\sqrt{2})^T$ for $\lambda = 1-\sqrt{2}\imath, 1+\sqrt{2}\imath$ respectively. In this situation it is advantageous to work with complex numbers as we will see soon. Similar to the example above, letting $\textbf{y} = P\textbf{z}$, where
\begin{align*}
P &=
\begin{bmatrix}
\imath & -\imath \\
\sqrt{2} & \sqrt{2} 
\end{bmatrix}
\end{align*}
then we have
\begin{align*}
\textbf{z}' &= (P^{-1}AP)\textbf{z} = D\textbf{z}
\end{align*}
with 
\begin{align*}
D &=
\begin{bmatrix}
1 - \sqrt{2}\imath & 0 \\
0 & 1 + \sqrt{2}\imath
\end{bmatrix}
\end{align*}
The solution in Theorem \ref{ODEsol} is valid even when $\beta$ is complex. Hence the solution for $z$ will be
\begin{align*}
\begin{cases}
z_1 &= c_1e^{(1-\sqrt{2}\imath)x} = c_1e^{x}(\cos(\sqrt{2}x) - \imath\sin\sqrt{2}x)\\
z_2 &= c_2e^{(1+\sqrt{2}\imath)x} = c_2e^{x}(\cos(\sqrt{2}x) + \imath\sin\sqrt{2}x)
\end{cases}
\end{align*}
where Euler's formula from Definition \ref{Euler} is applied. Hence 
\begin{align*}
\begin{bmatrix}
y_1 \\
y_2
\end{bmatrix}
&=
\begin{bmatrix}
\imath & -\imath \\
\sqrt{2} & \sqrt{2} 
\end{bmatrix}
\begin{bmatrix}
c_1e^{x}(\cos(\sqrt{2}x) - \imath\sin(\sqrt{2}x)) \\
c_2e^{x}(\cos(\sqrt{2}x) + \imath\sin(\sqrt{2}x))
\end{bmatrix} \\
&= 
\begin{bmatrix}
(c_1 + c_2)e^{x}\sin(\sqrt{2}x) + \imath(c_1-c_2)e^{x}\cos(\sqrt{2}x) \\
\sqrt{2}(c_1+c_2)e^{x}\cos(\sqrt{2}x) - \imath\sqrt{2}(c_1-c_2)e^{x}\sin(\sqrt{2}x)
\end{bmatrix} \\
&= 
\begin{bmatrix}
C_1e^{x}\sin(\sqrt{2}x) + \imath C_2e^{x}\cos(\sqrt{2}x) \\
\sqrt{2}C_1e^{x}\cos(\sqrt{2}x) - \imath \sqrt{2}C_2e^{x}\sin(\sqrt{2}x)
\end{bmatrix}
\end{align*}
if we write $C_1 = c_1 + c_2$, $C_2 = c_1 - c_2$. Both the real and imaginary parts of $\textbf{y}$ will satisfy the system, and they form the general solution. It is because $A$ is a real matrix, hence we can rewrite
\begin{align*}
\vec{y'} &= A\vec{y} \\
\Re{\vec{y'}} + \imath\Im{\vec{y'}} &= A(\Re{\vec{y}} + \imath\Im{\vec{y}}) \\
\Re{\vec{y'}} + \imath\Im{\vec{y'}} &= A\Re{\vec{y}} + \imath A\Im{\vec{y}}
\end{align*}
Equating the real and imaginary parts gives
\begin{align*}
\vec{y'}_{Re} &= A\vec{y}_{Re} \\
\vec{y'}_{Im} &= A\vec{y}_{Im}
\end{align*}
So the final answer, expressed in real values, is
\begin{align*}
\begin{cases}
y_1 &= C_1e^{x}\sin(\sqrt{2}x) + C_2e^{x}\cos(\sqrt{2}x) \\
y_2 &= \sqrt{2}C_1e^{x}\cos(\sqrt{2}x) - \sqrt{2}C_2e^{x}\sin(\sqrt{2}x)
\end{cases}
\end{align*}
where $C_1, C_2$ are to be decided by initial condition.
\end{exmp}
This will be the same solution basis we will obtain if we consider only one eigenvalue $\lambda_0$ out of the conjugate pair and compute $\vec{y} = \vec{v}_{\lambda_0}e^{\lambda_0 x}$. 

\section{Invariant Subspaces, Cayley-Hamilton Theorem}

\paragraph{Remark} If the coefficient matrix for a system of ordinary differential equation is not diagonalizable, then we need to employ other methods to solve the system. The most common method, generalized from diagonalization, is the Jordan Normal Form. Interested readers are invited to search about it.

\section{Earth Science Applications}

\section{Python Programming}

\section{Exercises}

\begin{Exercise}
If a matrix $A$ has a determinant of zero, show that it must have $\lambda = 0$ as an eigenvalue.
\end{Exercise}

\begin{Exercise}
Show that if the eigenvalues of a matrix $A$ all have an algebraic multiplicity of $1$, or in other words, no repeated root for the characteristic equation, then $A$ is diagonalizable. 
\end{Exercise}

\begin{Exercise}
Find the eigenvalues of the following matrices.
\begin{align*}
&A =
\begin{bmatrix}
1 & 3 & 3\\
4 & 0 & 4\\
0 & 1 & 2
\end{bmatrix}
&B =
\begin{bmatrix}
1 & 2 & 3 & 4\\
0 & 5 & 6 & 7\\
0 & 0 & 8 & 9\\
0 & 0 & 0 & 10
\end{bmatrix}
\end{align*}
as well as their transpose.
\end{Exercise}

\begin{Exercise}
Find the eigenvalues and corresponding eigenvectors for
\begin{align*}
A =
\begin{bmatrix}
3 & 1 & 4\\
1 & 3 & 0\\
0 & 0 & 1
\end{bmatrix}
\end{align*}
Repeat the calculation for its inverse.
\end{Exercise}

\begin{Exercise}
Find the eigenvalues and corresponding eigenvectors for
\begin{align*}
A =
\begin{bmatrix}
2 & 0 & 1\\
1 & 1 & 2\\
3 & 0 & 0
\end{bmatrix}
\end{align*}
and perform Gram-Schmidt orthogonalization on the eigenvectors found.
\end{Exercise}

\begin{Exercise}
For the matrix 
\begin{align*}
A = 
\begin{bmatrix}
1 & -1 & 1\\
1 & 1 & -1\\
1 & -1 & 1
\end{bmatrix}
\end{align*}
compute its characteristic polynomial. By applying Cayley-Hamilton Theorem,
\begin{enumerate}[label=(\alph*)]
\item Find $A^{3}$,
\item Prove 
\begin{align*}
A^{n} = \begin{bmatrix}
1 & -(2^n-1) & 2^n-1\\
1 & 1 & -1\\
1 & -(2^n-1) & 2^n-1
\end{bmatrix}
\end{align*}
for $n > 3$ by Mathematical Induction (Assume this holds for $n = k$, then prove for $n = k+1$), and
\item Prove inverse of $A$ does not exist (Hint: Assume $A^{-1}$ exists and multiply both sides by the inverse on the equation given by Cayley-Hamilton Theorem).
\end{enumerate}
\end{Exercise}

\begin{Exercise}
Apply diagonalization for the following matrix, and prove that the characteristic polynomial remains unchanged, if possible.
\begin{align*}
&A =
\begin{bmatrix}
2 & 0 & 1\\
1 & 1 & 2\\
0 & 0 & 2
\end{bmatrix} 
&B =
\begin{bmatrix}
1 & 0 & 0\\
2 & 3 & 2\\
0 & 0 & 1
\end{bmatrix} \\
&C = 
\begin{bmatrix}
3 & 0 & 0\\
1 & 5 & 1\\
1 & 0 & 1
\end{bmatrix} 
\end{align*}
\end{Exercise}

\begin{Exercise}
Solve the following system of ordinary differential equations.
\begin{align*}
\begin{cases}
y_1' &= -y_1 + y_2 + 3y_3\\
y_2' &= -2y_1 + 3y_2 + 2y_3\\
y_3' &= -2y_1 + y_2 + 4y_3
\end{cases}
\end{align*}
with the initial condition $y_1(0) = 3$, $y_2(0) = 2$, $y_3(0) = 9$.
\end{Exercise}

\begin{Exercise}
Given an idealized situation where there are three chemical gases $P, Q, R$. Denote their concentrations by $[P], [Q], [R]$ respectively. If they undergo reactions in a closed pathway $P \to Q \to R \to P$ that can be regarded as first-order, so that the governing equations about their concentrations are
\begin{align*}
\begin{cases}
d[P]/dt &= k_{31}[R] - k_{12}[P] \\
d[Q]/dt &= k_{12}[P] - k_{23}[Q] \\
d[R]/dt &= k_{23}[Q] - k_{31}[R] \\
\end{cases}
\end{align*}
where $k_{mn}$ are all constants. Derive the time evolution of the concentrations of the three gases. What happens when $t \to \infty$?
\end{Exercise}
\chapter{Orthogonal and Normal Matrices}
\label{chap:normalmat}

We have discussed orthonormal bases in Section \ref{section:GSortho} and it is logical to go one step further and make a coordinate transformation matrix with these orthonormal basis vectors just as if they are ordinary basis vectors. It is not surprising that such a matrix is called an \textit{orthonormal(-gonal) matrix}, but it also turns out that they carry some important properties which will be explored in this chapter. Particularly, we may want to ask if there exists an orthogonal change of coordinates matrix, which represents rotation and reflection geometrically, for making a given square matrix become diagonal, a.k.a.\ \textit{orthogonal diagonalization} which is a stronger version of diagonalization introduced in the last chapter. Eventually, this will lead to a major result known as the \textit{Spectral Theorem}. These concepts will also be promoted to complex vectors and matrices, where the complex counterpart of orthogonal(-normal) is now known as \textit{unitary}, accompanied by a less stringent condition known as \textit{normal}. 

\section{Orthogonal Matrices}

\subsection{Definition of Orthogonal Matrices}
In Section \ref{section:dotprod}, we have talked about how two $\mathbb{R}^n$ vectors are orthogonal to each other when their dot product is zero. We can extend the concept of orthogonality to a matrix. An \index{Orthogonal Matrix}\keywordhl{orthogonal matrix} (sometimes called \index{Orthonormal Matrix}\keywordhl{orthonormal}) is a matrix, each column of which is an $\mathbb{R}^n$ vector of unit length and orthogonal to all other columns. We further require the number of columns to be $n$ too so that they form an \textit{orthonormal basis} by Properties \ref{proper:ortholinind} and (b) of Properties \ref{proper:linindspanbasisnewver}. Hence it will be a square matrix.

\begin{defn}[Orthogonal Matrix]
\label{defn:orthomatrix}
An $n \times n$ square matrix
\begin{align*}
P = \begin{bmatrix}
\vec{v}^{(1)}|\cdots|\vec{v}^{(j)}|\cdots|\vec{v}^{(n)}   
\end{bmatrix}
\end{align*}
is said to be an orthogonal matrix, if all of its columns $\vec{v}^{(j)} \in \mathbb{R}^n$ satisfy the relationship:
\begin{align*}
\vec{v}^{(p)} \cdot \vec{v}^{(q)} =
\begin{cases}
1 &\text{ if } p = q \quad \text{ i.e.\ } \norm{\vec{v}^{(p)}} = 1 \\
0 &\text{ if } p \neq q    
\end{cases}
\end{align*}
so that the vectors $\vec{v}^{(j)}$, $j=1,2,\ldots,n$ form an orthonormal basis.
\end{defn}
The orthonormal basis vectors in an orthogonal matrix can be generated through the procedure of Gram-Schmidt orthogonalization with normalization, introduced in Section \ref{section:GSortho}.\par
Short Exercise: Verify that 
\begin{align*}
\begin{bmatrix}
\frac{1}{\sqrt{2}} & \frac{1}{\sqrt{3}} & -\frac{1}{\sqrt{6}} \\
\frac{1}{\sqrt{2}} & -\frac{1}{\sqrt{3}} & \frac{1}{\sqrt{6}} \\
0 & \frac{1}{\sqrt{3}} & \frac{\sqrt{2}}{\sqrt{3}}
\end{bmatrix}
\end{align*}
is an orthogonal matrix.\footnote{There are $3$ $\mathbb{R}^3$ column vectors in the matrix, where $\vec{v}^{(1)} \cdot \vec{v}^{(2)} = (\frac{1}{\sqrt{2}})(\frac{1}{\sqrt{3}}) + (\frac{1}{\sqrt{2}})(-\frac{1}{\sqrt{3}}) + (0)(\frac{1}{\sqrt{3}}) = 0$, $\vec{v}^{(1)} \cdot \vec{v}^{(3)} = (\frac{1}{\sqrt{2}})(-\frac{1}{\sqrt{6}}) + (\frac{1}{\sqrt{2}})(\frac{1}{\sqrt{6}}) + (0)(\frac{\sqrt{2}}{\sqrt{3}}) = 0$. We leave to the readers to check $\vec{v}^{(2)} \cdot \vec{v}^{(3)} = 0$ as well. Also, $\norm{\vec{v}^{(1)}} = \sqrt{(\frac{1}{\sqrt{2}})^2 + (\frac{1}{\sqrt{2}})^2 + (0)^2} = 1$ and $\norm{\vec{v}^{(2)}} = \sqrt{(\frac{1}{\sqrt{3}})^2 + (-\frac{1}{\sqrt{3}})^2 + (\frac{1}{\sqrt{3}})^2} = 1$. Again we let the readers to check $\norm{\vec{v}^{(3)}} = 1$ too.}\par
Due to its definition, an orthogonal matrix $P$ has the property $P^TP = I$, since the resulting entries of this matrix product $P^TP$ are basically dot products of the column vectors of $P$ (see the explanation below Definition \ref{defn:dotreal}), where explicitly
\begin{align*}
P^TP &= 
\begin{bmatrix}
\vec{v}^{(1)}|\cdots|\vec{v}^{(j)}|\cdots|\vec{v}^{(n)}   
\end{bmatrix}^T
\begin{bmatrix}
\vec{v}^{(1)}|\cdots|\vec{v}^{(j)}|\cdots|\vec{v}^{(n)}   
\end{bmatrix} \\
&= 
\left[
\begin{array}{c}
\vec{v}^{(1)T}\\
\hline
\vdots \\
\hline
\vec{v}^{(j)T} \Tstrut \\
\hline
\vdots \\
\hline
\vec{v}^{(n)T} \Tstrut 
\end{array}
\right]
\begin{bmatrix}
\vec{v}^{(1)}|\cdots|\vec{v}^{(j)}|\cdots|\vec{v}^{(n)}
\end{bmatrix} \\ 
&=
\begin{bmatrix}
\vec{v}^{(1)} \cdot \vec{v}^{(1)} & \cdots & \vec{v}^{(1)} \cdot \vec{v}^{(j)} & \cdots & \vec{v}^{(1)} \cdot \vec{v}^{(n)} \\
\vdots & \ddots & \vdots & & \vdots \\
\vec{v}^{(j)} \cdot \vec{v}^{(1)} & \cdots & \vec{v}^{(j)} \cdot \vec{v}^{(j)} & \cdots & \vec{v}^{(j)} \cdot \vec{v}^{(n)} \\
\vdots & & \vdots & \ddots & \vdots \\
\vec{v}^{(n)} \cdot \vec{v}^{(1)} & \cdots & \vec{v}^{(n)} \cdot \vec{v}^{(j)} & \cdots & \vec{v}^{(n)} \cdot \vec{v}^{(n)}
\end{bmatrix} \\
&=
\begin{bmatrix}
1 & \cdots & 0 & \cdots & 0 \\
\vdots & \ddots & \vdots & & \vdots \\
0 & \cdots & 1 & \cdots & 0 \\
\vdots & & \vdots & \ddots & \vdots \\
0 & \cdots & 0 & \cdots & 1
\end{bmatrix} = I_n
\end{align*}
By Definition \ref{defn:inverse}, $P^TP = I$ means that $PP^T = P^TP = I$ where the transpose of the orthogonal matrix $P^T = P^{-1}$ is exactly its inverse. The argument works in both directions such that $PP^T = I$ also means that $P$ is orthogonal.
\begin{proper}
\label{proper:orthoinvT}
$P$ is an orthogonal matrix if and only if $PP^T = P^TP = I$ so that its inverse is simply its transpose $P^{-1} = P^T$.
\end{proper}
The equivalence between $P^TP=I$ and $PP^T = I$ for an orthogonal matrix extends Definition \ref{defn:orthomatrix} where the former equality indicates that the rows of $P$ also have to form an orthonormal basis. So a parallel definition of orthogonal matrices is
\begin{defn}
An $n \times n$ square matrix
\begin{align*}
P = \left[
\begin{array}{c}
\vec{w}^{(1)T}\\
\hline
\vdots \\
\hline
\vec{w}^{(i)T} \Tstrut \\
\hline
\vdots \\
\hline
\vec{w}^{(n)T} \Tstrut 
\end{array}
\right]
\end{align*}
is an orthogonal matrix, also if all the row vectors $\vec{w}^{(i)} \in \mathbb{R}^n$ satisfy the similar condition:
\begin{align*}
\vec{w}^{(p)} \cdot \vec{w}^{(q)} =
\begin{cases}
1 &\text{ if } p = q \quad \text{ i.e.\ } \norm{\vec{w}^{(p)}} = 1 \\
0 &\text{ if } p \neq q    
\end{cases}
\end{align*}
so that the vectors $\vec{w}^{(i)}$, $i=1,2,\ldots,n$ constitute an orthonormal basis too.
\end{defn}

Short Exercise: Confirm Properties \ref{proper:orthoinvT} for the matrix in the last short exercise.\footnote{It is a direct computation that shows
\begin{align*}
\left[\begin{array}{@{}wr{14pt}wr{14pt}wr{14pt}@{\,}}
\frac{1}{\sqrt{2}} & \frac{1}{\sqrt{3}} & -\frac{1}{\sqrt{6}} \\
\frac{1}{\sqrt{2}} & -\frac{1}{\sqrt{3}} & \frac{1}{\sqrt{6}} \\
0 & \frac{1}{\sqrt{3}} & \frac{\sqrt{2}}{\sqrt{3}}
\end{array}\right]^T
\left[\begin{array}{@{}wr{14pt}wr{14pt}wr{14pt}@{\,}}
\frac{1}{\sqrt{2}} & \frac{1}{\sqrt{3}} & -\frac{1}{\sqrt{6}} \\
\frac{1}{\sqrt{2}} & -\frac{1}{\sqrt{3}} & \frac{1}{\sqrt{6}} \\
0 & \frac{1}{\sqrt{3}} & \frac{\sqrt{2}}{\sqrt{3}}
\end{array}\right] &=
\left[\begin{array}{@{\,}wr{14pt}wr{14pt}wr{14pt}@{\,}}
\frac{1}{\sqrt{2}} & \frac{1}{\sqrt{2}} & 0 \\
\frac{1}{\sqrt{3}} & -\frac{1}{\sqrt{3}} & \frac{1}{\sqrt{3}} \\
-\frac{1}{\sqrt{6}} & \frac{1}{\sqrt{6}} & \frac{\sqrt{2}}{\sqrt{3}}
\end{array}\right]
\left[\begin{array}{@{}wr{14pt}wr{14pt}wr{14pt}@{\,}}
\frac{1}{\sqrt{2}} & \frac{1}{\sqrt{3}} & -\frac{1}{\sqrt{6}} \\
\frac{1}{\sqrt{2}} & -\frac{1}{\sqrt{3}} & \frac{1}{\sqrt{6}} \\
0 & \frac{1}{\sqrt{3}} & \frac{\sqrt{2}}{\sqrt{3}}
\end{array}\right] \\
&=
\begin{bmatrix}
1 & 0 & 0 \\
0 & 1 & 0 \\
0 & 0 & 1
\end{bmatrix}
\end{align*}
and $PP^T = I$ similarly.}

\subsection{Geometric Implications of Orthogonal Matrices}
\label{section:orthogeometricsub}
All invertible matrices represent some sort of coordinate transformation as highlighted in Section \ref{section:coordchange}. Orthogonal matrices further belong to a special class of them. Before going into details, we need to start with some simple observations.
\begin{proper}
\label{proper:orthopm1det}
An orthogonal matrix $P$ has a determinant of either $1$ or $-1$.
\end{proper}
\begin{proof}
Using the relationship from Properties \ref{proper:orthoinvT} and taking determinant on both sides of it, we have
\begin{align*}
\det(P^TP) &= \det(I) \\
\det(P^T)\det(P) = (\det(P))^2 &= 1\\
\det(P) &= \pm 1
\end{align*}
where Properties \ref{proper:properdet} is used.    
\end{proof} 
The non-zero determinant reaffirms that the $n$ $\mathbb{R}^n$ row/column vectors in an orthogonal matrix $P$ are linearly independent ((b) to (e) of Theorem \ref{thm:equiv3}). 
\begin{proper}
\label{proper:ortholengthpreserve}
Coordinate transformation by an orthogonal matrix $P$ on a vector is length-preserving.
\end{proper}
\begin{proof}
Assume the coordinate transformation is the passive type as described in Section \ref{section:coordchange}. Then the new coordinates of the vector $\vec{v}$ will be $P^{-1}\vec{v} = P^T\vec{v}$ (Properties \ref{proper:orthoinvT}). The length of this vector in the new coordinate frame is
\begin{align*}
\norm{P^T\vec{v}} &= \sqrt{(P^T\vec{v})\cdot(P^T\vec{v})} & \text{(Properties \ref{proper:lengthdot})} \\
&= \sqrt{(P^T\textbf{v})^T(P^T\textbf{v})} & \text{(Properties \ref{proper:dotproper})}\\
&= \sqrt{(\textbf{v}^T PP^T \textbf{v})} & \text{(Properties \ref{proper:dotproper})} \\
&= \sqrt{(\textbf{v}^T I \textbf{v})} & \text{(Properties \ref{proper:orthoinvT})} \\
&= \sqrt{(\textbf{v}^T\textbf{v})} = \norm{\vec{v}} & \text{(Properties \ref{proper:lengthdot} again)}
\end{align*}
So the length of the vector remains the same when the coordinate system is changed. For active transformation that actually changes the vector in space instead of just the frame, the argument is very similar and the new vector will also have the same length as the old one. In fact, Properties \ref{proper:ortholengthpreserve} implies Properties \ref{proper:orthopm1det} (the discussion about the geometric interpretation of the scaling factor for a coordinate transformation in Section \ref{section:coordchange} will be helpful).
\end{proof}

Now we are ready to see what kind of coordinate transformation an orthogonal matrix means.
\begin{thm}
\label{thm:orthodet}
For an orthogonal matrix $P$, if $P$ has a determinant of $1$, then it is a rotation. On the other hand, if $P$ has a determinant of $-1$, then it implies a reflection.
\end{thm}
Just as in Section \ref{section:coordchange}, depending on the situation, rotation and reflection can be viewed as (a) rotation and reflection of the vector physically while keeping the coordinate frame unchanged (active transformation), or (b) rotation and reflection of the coordinate system, while keeping the vector in the original place (passive transformation). By the previous length-preserving properties there will be no stretching/compression or shearing, and the rotation/reflection will be pure.

\begin{center}
\begin{tikzpicture}
\draw[->] (0,0)--(3,0) node[right]{$x$};
\draw[->] (0,0)--(0,3) node[above]{$y$};
\draw[red,line width=1.5,-stealth] (0,0)--(2,1);
\draw[blue,line width=1.5,-stealth] (0,0)--(1,2);
\draw[gray,-stealth] (1.9,1.1) to [out=120,in=-30](1.1,1.9);
\node[below left]{$O$}; 
\end{tikzpicture}
\begin{tikzpicture}
\draw[gray,->] (0,0)--(3,0) node[right]{$x$};
\draw[gray,->] (0,0)--(0,3) node[above](vecu){$y$};
\draw[->] (0,0)--(2.719, 1.268) node[right]{$x'$};
\draw[->] (0,0)--(-1.268, 2.719) node[above](vecv){$y'$};
\draw[Green,-stealth, line width=1.5] (0,0)--(2.5,0.5);
\node[below left]{$O$}; 
\pic[draw, ->, "$\theta$", angle eccentricity=1.5, angle radius=0.75cm] {angle = vecu--0--vecv};
\end{tikzpicture}\\
\textit{Left: case (a), active transformation, and right: case (b), passive transformation, for a 2D rotation.} 
\end{center}

The construction of a rotational or reflectional matrix requires the representation of the new unit axes in the old coordinate system as suggested in Theorem \ref{thm:bijectivechincoord}, but with the extra condition that they are orthogonal to each other. Once they are found, they can be combined column by column to form the \index{Transition Matrix}\keywordhl{transition matrix}. For an anti-clockwise (positive) rotation like the one to the right side of the figure above, we have
\begin{subequations}
\label{eqn:2drotate}
\begin{empheq}[left={\empheqlbrace}]{alignat=1}
\hat{x}' &= (\cos \theta) \hat{x} + (\sin \theta) \hat{y} \\
\hat{y}' &= (-\sin \theta) \hat{x} + (\cos \theta) \hat{y}
\end{empheq}
\end{subequations}
The corresponding transition matrix is then
\begin{align*}
P_\beta^S &= \begin{bmatrix}
[\hat{x}']_S|[\hat{y}']_S    
\end{bmatrix} \\
&= \begin{bmatrix}
\cos \theta & -\sin \theta \\
\sin \theta & \cos \theta
\end{bmatrix}
\end{align*}
where $S$ represents the standard basis, and $\beta$ indicates the rotated coordinates system. Alternatively,
\begin{align*}
P_S^\beta &= (P_\beta^S)^{-1} \\
&= (P_\beta^S)^T & \text{(Properties \ref{proper:orthoinvT})}\\
&= \begin{bmatrix}
\cos \theta & \sin \theta \\
-\sin \theta & \cos \theta
\end{bmatrix}
\end{align*}
These two matrices can be compared to the one we see in the last chapter when we study the real variant of diagonalization for complex eigenvalues. \par
If we have an orthogonal matrix $P$, and a vector $\vec{v}_0$ in the standard basis, then computing $\vec{v}_n = P\vec{v}_0$ in a forward direction can be viewed as directly applying the corresponding rotation or reflection on the vector $\vec{v}_0$ to produce $\vec{v}_n$ while staying in the same coordinate frame. This is the case (a) we have discussed above. For case (b), it is the exact opposite. Rotation or reflection of the coordinate system, where the concerned vector $\vec{v}_0$ is fixed physically, is achieved by solving $\vec{v}_0 = P\vec{v}_n$ in a backward manner, or equivalently $\vec{v}_n = P^{-1}\vec{v}_0 = P^T\vec{v}_0$. Taking transpose, it becomes $\vec{v}_n^T = \vec{v}_0^T P$. In such a situation, the vector is still the same one in a physical sense, but represented in the new coordinate system. Most situations belong to case (b), on which thereafter we will focus our discussion.
\par
However, as a reminder, these two cases are much related. For example, an anti-clockwise/positive rotation of a vector in case (a), can be viewed as a clockwise/negative rotation of the coordinate system in case (b), and vice versa. The two operations are only differed by a transpose. For successive rotations and reflections, the net transition matrix $P_f$ is produced by taking the product of individual rotational and reflectional matrices each by each. One common convention has the order from right to left, where $\vec{v}_n = (P_n^T\cdots P_3^TP_2^TP_1^T)\vec{v}_0$, and thus $P_f^T = P_n^T\cdots P_3^TP_2^TP_1^T$. Another equivalent option is to do it from left to right by applying a transpose, as in
$\vec{v}_n^T = \vec{v}_0^T(P_1P_2P_3\cdots P_n)$, $P_f = P_1P_2P_3\cdots P_n$.

\begin{exmp}
Find the net transition matrix, if a rotation about $y$-axis of $40$ degree in the positive direction is done first to produce an intermediate coordinate system $x', y', z'$, and then a reflection across the $x'$-$y'$ plane is made to generate the final coordinate system $x'', y'', z''$ (so the new $z''$ axis is the negative of $z'$ axis).
\end{exmp}
\begin{solution}
The first transition matrix is
\begin{align*}
P_1
&= 
\begin{bmatrix}
\cos(40^\circ) & 0 & \sin(40^\circ) \\
0 & 1 & 0 \\
-\sin(40^\circ) & 0 & \cos(40^\circ)
\end{bmatrix}
\end{align*}
The readers should verify this. The second transition matrix is simply
\begin{align*}
P_2
&= 
\left[\begin{array}{@{\,}wc{10pt}wc{10pt}wc{10pt}@{\,}}
1 & 0 & 0 \\
0 & 1 & 0 \\
0 & 0 & -1
\end{array}\right]    
\end{align*}
So the net transition matrix is
\begin{align*}
P_f &= P_1P_2 \\
&= 
\begin{bmatrix}
\cos(40^\circ) & 0 & \sin(40^\circ) \\
0 & 1 & 0 \\
-\sin(40^\circ) & 0 & \cos(40^\circ)
\end{bmatrix}
\left[\begin{array}{@{\,}wc{10pt}wc{10pt}wc{10pt}@{\,}}
1 & 0 & 0 \\
0 & 1 & 0 \\
0 & 0 & -1
\end{array}\right] \\
&= 
\begin{bmatrix}
\cos(40^\circ) & 0 & -\sin(40^\circ) \\
0 & 1 & 0 \\
-\sin(40^\circ) & 0 & -\cos(40^\circ)
\end{bmatrix}
\end{align*}
\end{solution}
Short Exercise: What happens if we reverse the order of rotation/reflection?\footnote{The transition matrix becomes
\begin{align*}
\left[\begin{array}{@{\,}wc{8pt}wc{8pt}wc{8pt}@{\,}}
1 & 0 & 0 \\
0 & 1 & 0 \\
0 & 0 & -1
\end{array}\right]   
\begin{bmatrix}
\cos(40^\circ) & 0 & \sin(40^\circ) \\
0 & 1 & 0 \\
-\sin(40^\circ) & 0 & \cos(40^\circ)
\end{bmatrix}
&=
\begin{bmatrix}
\cos(40^\circ) & 0 & \sin(40^\circ) \\
0 & 1 & 0 \\
\sin(40^\circ) & 0 & -\cos(40^\circ)
\end{bmatrix} \\
&\neq 
\begin{bmatrix}
\cos(40^\circ) & 0 & -\sin(40^\circ) \\
0 & 1 & 0 \\
-\sin(40^\circ) & 0 & -\cos(40^\circ)
\end{bmatrix}
\end{align*}
In general, finite rotations/reflections are not commutative and the order matters.
}

\begin{exmp}
For a vector $\vec{v}_0$ in the three-dimensional standard basis $S$, if the coordinate system undergoes a positive rotation about the straight line $x = 0, y = z$ by a degree of $\theta$, find its representation $\vec{v}_n$ in the new system $\beta$.
\end{exmp}
\begin{solution}
The common way to construct a transition matrix is to find the new axes in terms of the old coordinates. However, in this case it is harder because the axis about which the rotation occurs is not along any of the main axes. Here, we can do another rotation 
\begin{align*}
P_1 = 
\begin{bmatrix}
1 & 0 & 0 \\
0 & \cos(\SI{45}{\degree}) & \sin(\SI{45}{\degree}) \\
0 & -\sin(\SI{45}{\degree}) & \cos(\SI{45}{\degree})
\end{bmatrix}
=
\left[\begin{array}{@{\,}wc{12pt}wc{12pt}wc{12pt}@{\,}}
1 & 0 & 0 \\
0 & \frac{1}{\sqrt{2}} & \frac{1}{\sqrt{2}} \\
0 & -\frac{1}{\sqrt{2}} & \frac{1}{\sqrt{2}}
\end{array}\right]
\end{align*}
about the $x$-axis first with a degree of $\SI{45}{\degree}$ clockwise so that the intermediate $z'$ axis is oriented along the desired line, while the $y'$ axis points in the $x = 0, y = -z$ direction relative to the original coordinate frame. Then we can apply the required rotation in the intermediate coordinate system about that $z'$ axis, which has a simple transition matrix representation of
\begin{align*}
P_2 =
\begin{bmatrix}
\cos \theta & -\sin \theta & 0 \\
\sin \theta & \cos \theta & 0 \\
0 & 0 & 1
\end{bmatrix}
\end{align*}
Finally, we undo the effect of the first rotation via multiplying by its inverse $P_1^{-1}$ which is equal to $P_1^T$ due to orthogonality (Properties \ref{proper:orthoinvT}). 
\begin{center}
\begin{tikzpicture}[x={(-0.4cm, -0.9cm)}, y={(1cm, 0cm)}, z={(0cm, 1cm)}]
\node[above left]{$O$}; 
\draw[thick,->] (0,0,0) -- (0,0,2) node[anchor=north west]{$z$};
\draw[thick,->] (0,0,0) -- (2,0,0) node[anchor=north east]{$x$};
\draw[thick,->] (0,0,0) -- (0,2,0) node[anchor=west]{$y$};
\draw[gray,dashed] (0,0,0) -- (0,2,2);
\node[above left] at (0,0,-5) {$O$}; 
\draw[thick,->] (0,0,-5) -- (0,1.414,1.414-5);
\draw[thick,->] (0,0,-5) -- (2,0,-5);
\draw[thick,->] (0,0,-5) -- (0,1.414,-1.414-5);
\draw[gray,dashed] (0,0,-5) -- (0,0,-5+2);
\draw[->] (0,0,-5+1) to[out=0,in=120] (0,1.414/2,1.414/2-5);
\node[above left] at (0,5,-5) {$O$}; 
\draw[thick,->] (0,5,-5) -- (0,1.414+5,1.414-5);
\draw[thick,->] (0,5,-5) -- (1.414,1+5,-1-5);
\draw[thick,->] (0,5,-5) -- (-1.414,1+5,-1-5);
\draw[gray,dashed] (0,5,-5) -- (2,5,-5);
\draw[->] (1,5,-5) to[out=240,in=210] (1.414/2,1/2+5,-1/2-5);
\node at (0,5,-6.5) {$\theta$};
\node[above left] at (0,5,0) {$O$}; 
\draw[thick,->] (0,5,0) -- (1.414,1+5,1) node[below]{$x'$};
\draw[thick,->] (0,5,0) -- (-1,0.707+1+5,-0.707+1) node[right]{$y'$};
\draw[thick,->] (0,5,0) -- (1,-0.707+1+5,0.707+1) node[above]{$z'$};
\draw[gray,dashed] (0,5,0) -- (0,5+2,2);
\draw[red,->] (0,0.75+5,1.35) to[out=240,in=210] (0,1.25+5,0.85);
\draw[red,thick,->] (0,3,0) -- (0,4,0) node[midway,above]{$P_f$};
\draw[blue,thick,->] (0,3,-5) -- (0,4,-5) node[midway,below]{$P_2$};
\draw[blue,thick,->] (0,1,-2) -- (0,1,-3) node[midway,left]{$P_1$};
\draw[blue,thick,->] (0,6,-3) -- (0,6,-2) node[midway,right]{$P_1^{-1}$};
\end{tikzpicture}     
\end{center}
Hence the net transition matrix is
\begin{align*}
P_f = P_1 P_2 P_1^{-1} &= P_1 P_2 P_1^T \\
&= \left[\begin{array}{@{\,}wc{12pt}wc{12pt}wc{12pt}@{\,}}
1 & 0 & 0 \\
0 & \frac{1}{\sqrt{2}} & \frac{1}{\sqrt{2}} \\
0 & -\frac{1}{\sqrt{2}} & \frac{1}{\sqrt{2}}
\end{array}\right]
\begin{bmatrix}
\cos \theta & -\sin \theta & 0 \\
\sin \theta & \cos \theta & 0 \\
0 & 0 & 1
\end{bmatrix}
\left[\begin{array}{@{\,}wc{12pt}wc{12pt}wc{12pt}@{\;}}
1 & 0 & 0 \\
0 & \frac{1}{\sqrt{2}} & -\frac{1}{\sqrt{2}} \\
0 & \frac{1}{\sqrt{2}} & \frac{1}{\sqrt{2}}
\end{array}\right] \\
&= 
\begin{bmatrix}
\cos{\theta} & -\frac{\sin{\theta}}{\sqrt{2}} & \frac{\sin{\theta}}{\sqrt{2}} \\
\frac{\sin{\theta}}{\sqrt{2}} & \frac{(\cos{\theta}) + 1}{2} & \frac{-(\cos{\theta}) + 1}{2} \\
-\frac{\sin{\theta}}{\sqrt{2}} & \frac{-(\cos{\theta}) + 1}{2} & \frac{(\cos{\theta}) + 1}{2}
\end{bmatrix}
\end{align*}
For any vector $\vec{v}_0 = (x_0,y_0,z_0)^T$ expressed in the standard coordinate basis, the new coordinates after rotation are then
\begin{align*}
\vec{v}_0 &= P_f \vec{v}_n \\
\vec{v}_n &= (P_f)^T\vec{v}_0 \\
&=
\begin{bmatrix}
\cos{\theta} & \frac{\sin{\theta}}{\sqrt{2}} & -\frac{\sin{\theta}}{\sqrt{2}} \\
-\frac{\sin{\theta}}{\sqrt{2}} & \frac{(\cos{\theta}) + 1}{2} & \frac{-(\cos{\theta}) + 1}{2} \\
\frac{\sin{\theta}}{\sqrt{2}} & \frac{-(\cos{\theta}) + 1}{2} & \frac{(\cos{\theta}) + 1}{2}
\end{bmatrix}
\begin{bmatrix}
x_0 \\
y_0 \\
z_0
\end{bmatrix} \\
&=
\begin{bmatrix}
(\cos{\theta})x_0 + (\frac{\sin{\theta}}{\sqrt{2}})y_0 + (-\frac{\sin{\theta}}{\sqrt{2}})z_0 \\
(-\frac{\sin{\theta}}{\sqrt{2}})x_0 + (\frac{(\cos{\theta}) + 1}{2})y_0 + (\frac{-(\cos{\theta}) + 1}{2})z_0 \\
(\frac{\sin{\theta}}{\sqrt{2}})x_0 + (\frac{-(\cos{\theta}) + 1}{2})y_0 + (\frac{(\cos{\theta}) + 1}{2})z_0
\end{bmatrix}
\end{align*}
\end{solution}
Short Exercise: For the example above, if $(x_0, y_0, z_0)^T = (0,1,1)^T$ and $\theta = \frac{\pi}{6}$, find $\vec{v}_n = (x_n, y_n, z_n)^T$.\footnote{It is
\begin{align*}
\renewcommand\arraystretch{1.41}
\begin{bmatrix}
\cos{\frac{\pi}{6}} & \frac{\sin{\frac{\pi}{6}}}{\sqrt{2}} & -\frac{\sin{\frac{\pi}{6}}}{\sqrt{2}} \\
-\frac{\sin{\frac{\pi}{6}}}{\sqrt{2}} & \frac{(\cos{\frac{\pi}{6}}) + 1}{2} & \frac{-(\cos{\frac{\pi}{6}}) + 1}{2} \\
\frac{\sin\frac{\pi}{6}}{\sqrt{2}} & \frac{-(\cos{\frac{\pi}{6}}) + 1}{2} & \frac{(\cos{\frac{\pi}{6}}) + 1}{2}
\end{bmatrix}
\begin{bmatrix}
0 \\
1 \\
1
\end{bmatrix}
=
\begin{bmatrix}
(\frac{\sqrt{3}}{2})(0) + (\frac{1}{2\sqrt{2}})(1) + (-\frac{1}{2\sqrt{2}})(1) \\
(-\frac{1}{2\sqrt{2}})(0) + (\frac{\frac{1}{2} + 1}{2})(1) + (\frac{-\frac{1}{2} + 1}{2})(1) \\
(\frac{1}{2\sqrt{2}})(0) + (\frac{-\frac{1}{2} + 1}{2})(1) + (\frac{\frac{1}{2} + 1}{2})(1)
\end{bmatrix}
=
\begin{bmatrix}
0 \\
1 \\
1
\end{bmatrix}
\end{align*}
In fact, all vectors in the form of $s(0,1,1)^T$ where $s$ is any number will keep the same coordinates after transformation, no matter what the value $\theta$ takes, because the vector is oriented exactly along the rotational axis in question.}

\section{Orthogonal Diagonalization}
\label{section:orthogonaldiagreal}

\index{Orthogonal Diagonalization}\keywordhl{Orthogonal diagonalization} is a special case of diagonalization, in which the transformation matrix $P$ used to diagonalize the target matrix $A$ is an orthogonal matrix. Since from Section \ref{section:diagonalizeidea} we know that diagonalization boils down to asking if there exists a basis, represented by the columns of $P$, such that $D = P^{-1}AP$ is diagonal, orthogonal diagonalization is equivalent to demanding further that such a basis, comprised of the eigenvectors of $A$, is orthonormal, i.e.\ $P$ is an orthogonal matrix. We will discuss the case where $A$ is a real matrix first.
\begin{defn}[Orthogonal Diagonalization]
\label{defn:orthodiagonal}
Orthogonal diagonalization on a real $n \times n$ square matrix $A$ is to find an orthogonal matrix $P$ that consists of columns being the $n$ orthonormal $\mathbb{R}^n$ eigenvectors of $A$ (Definition \ref{defn:orthomatrix}) such that $P^{-1}AP = P^TAP = D$ is a real diagonal matrix (Properties \ref{proper:diagonalize} and \ref{proper:orthoinvT}).
\end{defn}
Note that the requirements of eigenvectors being $\mathbb{R}^n$ and hence orthogonality defined by the real dot product implicitly constrain us to work over $\mathbb{R}$ and only real eigenvalues are allowed. Not all real square matrices can be orthogonally diagonalized because orthogonal diagonalization is a stronger version of ordinary diagonalization and the requirement is unsurprisingly stricter. However, it turns out that there is a really simple criterion: whether the matrix is symmetric.
\begin{thm}
\label{thm:symdiag}
A real square matrix $A$ can be orthogonally diagonalized if and only if $A$ is symmetric.
\end{thm}
As a corollary, all symmetric real matrices have only real eigenvalues. The "only if" part is very easy to show.\footnote{Note that if $A$ is orthogonally diagonalizable then $A = PDP^{-1} = PDP^T$. Then $A^T = (PDP^T)^T = PD^TP^T = PDP^T = A$ since the transpose of any diagonal matrix $D^T = D$ equals to itself.} On the other hand, the "if" part is much more difficult to derive and requires us to build some intermediate results first, which will be done sequentially in the remainder of this section. Since the transformation matrix $P$ used to diagonalize a given matrix $A$ is formed from the eigenvectors of $A$ as pointed out by Properties \ref{proper:diagonalize}, for $P$ to be an orthogonal matrix, obviously we need to show that these eigenvectors are orthogonal to each other if $A$ is symmetric. To this end, we have the following observation.
\begin{proper}
\label{proper:symortho}
Any two eigenvectors of a real, symmetric matrix corresponding to two distinct eigenvalues are always orthogonal to each other.
\end{proper}
\begin{proof}
Denote the two eigenvectors as $\vec{v}_{\lambda_1}$ and $\vec{v}_{\lambda_2}$ which have two different eigenvalues $\lambda_1$ and $\lambda_2$. Then consider the quantity $\vec{v}_{\lambda_1} \cdot (A\vec{v}_{\lambda_2})$ where $A$ is the real, symmetric matrix. By Definition \ref{defn:eigen}, it is equal to $\vec{v}_{\lambda_1} \cdot (\lambda_{2}\vec{v}_{\lambda_2}) = \lambda_{2}(\vec{v}_{\lambda_1} \cdot \vec{v}_{\lambda_2})$. At the same time, using Properties \ref{proper:dotproper} too, we have
\begin{align*}
\vec{v}_{\lambda_1}\cdot (A\vec{v}_{\lambda_2}) &=    (A^T\vec{v}_{\lambda_1})\cdot \vec{v}_{\lambda_2} \\
&= (A\vec{v}_{\lambda_1})\cdot \vec{v}_{\lambda_2} &\text{($A$ is symmetric)} \\
&= (\lambda_1\vec{v}_{\lambda_1})\cdot \vec{v}_{\lambda_2} = \lambda_1(\vec{v}_{\lambda_1}\cdot \vec{v}_{\lambda_2}) &\text{(Definition \ref{defn:eigen} again)}
\end{align*}
Therefore
\begin{align*}
\vec{v}_{\lambda_1}\cdot (A\vec{v}_{\lambda_2}) =  \lambda_1(\vec{v}_{\lambda_1}\cdot \vec{v}_{\lambda_2}) &= \lambda_{2}(\vec{v}_{\lambda_1}\cdot \vec{v}_{\lambda_2}) \\
(\lambda_{1}-\lambda_{2})(\vec{v}_{\lambda_1}\cdot \vec{v}_{\lambda_2}) &= 0
\end{align*}
Since the two eigenvalues are required to be distinct, $\lambda_{1} \neq \lambda_{2}$, and $\vec{v}_{\lambda_1}\cdot \vec{v}_{\lambda_2}$ must be $0$. By Properties \ref{proper:dotorth}, the two eigenvectors $\vec{v}_{\lambda_1}$ and $\vec{v}_{\lambda_2}$ are orthogonal to each other.
\end{proof}
Even when there are multiple eigenvectors (a geometric multiplicity/an eigenspace of dimension $\geq 2$) associated with the same eigenvalue, we can mediate the problem by using the Gram-Schmidt Orthogonalization procedure introduced in Section \ref{section:GSortho} to produce an orthonormal basis to that eigenspace. Since all vectors in the eigenspace are subjected to the same eigenvalue, the choice of new eigenvectors will not disrupt the diagonalization process. So, with these, we have solved the "orthogonal" part of "orthogonal diagonalization", and the remaining half of the problem is to justify why symmetric matrices are always "diagonalizable", i.e.\ the geometric multiplicities of all their eigenvalues are equal to the corresponding algebraic multiplicities as suggested by Properties \ref{proper:diagonalize}\footnote{The another condition requires the characteristic polynomial to split over $\mathbb{R}$, but it is immediately satisfied due to the previous corollary to Theorem \ref{thm:symdiag} which points out that the eigenvalues of any symmetric matrix are always real.}, or in other words, there is no \textit{"deficient"} eigenvalue.
\begin{proper}
\label{proper:symnodefic}
The geometric multiplicity of any distinct eigenvalue of a real symmetric matrix is always strictly equal to its algebraic multiplicity.
\end{proper}
\begin{proof}
Let $A$ be some $n \times n$ real symmetric matrix. Denote the geometric multiplicity of any distinct eigenvalue $\lambda_{J_i}$ of $A$ by $n_{J_i}$ and its algebraic multiplicity by $k_{J_i}$. We will assume that $n_{J_i} < k_{J_i}$ and derive a contradiction. The geometric multiplicity of $n_{J_i}$ implies that there can be $n_{J_i}$ orthonormal (see the discussion above) eigenvectors $\vec{v}^{(1)}_{\lambda_{J_i}}, \vec{v}^{(2)}_{\lambda_{J_i}}, \ldots, \vec{v}^{(n_{J_i})}_{\lambda_{J_i}} \in \mathbb{R}^n$ corresponding to $\lambda_{J_i}$ in the eigenspace $\mathcal{E}_{J_i}$. By (c) of Properties \ref{proper:linindspanbasisnewver}, we can complete a basis $\mathcal{E}_{J_i} \oplus (\mathcal{E}_{J_i})^C$ for $\mathbb{R}^n$ by filling them with some other orthonormal $n-n_{J_i}$ vectors. Let $P$ be the orthogonal coordinate matrix that represents this basis in its columns, such that
\begin{align*}
P = \begin{bmatrix}
\vec{v}^{(1)}_{\lambda_{J_i}} | \vec{v}^{(2)}_{\lambda_{J_i}} | \cdots | \vec{v}^{(n_{J_i})}_{\lambda_{J_i}} | \text{other vectors used to fill the basis for $\mathbb{R}^n$}
\end{bmatrix}
\end{align*}
similar to the derivation of Theorem \ref{thm:geolessalgebra}. With the same logic, we can apply a change of coordinates for $A$ with $P$, such that
\begin{align*}
A' &= P^{-1}AP \\
&= \begin{bmatrix}
\lambda_{J_i} I_{n_{J_i}} & *_{n_{J_i}\times(n-n_{J_i})} \\
[\textbf{0}]_{(n-n_{J_i})\times n_{J_i}} & *_{(n-n_{J_i})\times(n-n_{J_i})}
\end{bmatrix}
\end{align*}
However, this time, as $P$ has been declared to be an orthogonal matrix, $A' = P^{-1}AP = P^TAP$ (Properties \ref{proper:orthoinvT}) can be easily shown to be symmetric like $A$ as well.\footnote{$(P^TAP)^T = P^TA^TP = P^TAP$.} Therefore, the upper-right block of $A'$ also has to be a zero submatrix (as the transpose of the zero submatrix to the lower left):
\begin{align*}
A' &=  
\begin{bmatrix}
\lambda_j I_{n_{J_i}} & [\textbf{0}]_{n_{J_i}\times(n-n_{J_i})} \\
[\textbf{0}]_{(n-n_{J_i})\times n_{J_i}} & *_{(n-n_{J_i})\times(n-n_{J_i})}
\end{bmatrix}
\end{align*}
which is now in the form of a matrix direct sum (see Definition \ref{defn:matdirectsum}). This shows that we can treat it as $A' = \lambda_j I_{n_{J_i}} \oplus A^C$ with respect to the $\mathcal{E}_{J_i} \oplus (\mathcal{E}_{J_i})^C$ vector direct sum basis following Definition \ref{defn:matdirectsum} where $A^C$ is an $(n-n_{J_i})\times(n-n_{J_i})$ matrix that makes up the bottom right asterisked block. From the perspective of a linear transformation, the restriction of $A'$ to $(\mathcal{E}_{J_i})^C$ is simply $A^C: (\mathcal{E}_{J_i})^C \to (\mathcal{E}_{J_i})^C$ which is a linear operator by itself. Now if the algebraic multiplicity of $\lambda_{J_i}$ is $k_{J_i}$, then $A$ will have a characteristic polynomial of 
\begin{align*}
p_A(\lambda) = (\lambda_{J_i}-\lambda)^{k_{J_i}} p^-(\lambda)    
\end{align*} where $p^-(\lambda)$ is some other polynomial. By Properties \ref{proper:similarinvariant}, $A'$ will also have this same characteristic polynomial $p_{A'}(\lambda) = (\lambda_{J_i}-\lambda)^{k_{J_i}} p^-(\lambda)$. At the same time, from the structure of $A'$ derived above, we know that 
\begin{align*}
p_{A'}(\lambda) = (\lambda_{J_i}-\lambda)^{n_{J_i}} p^C(\lambda)    
\end{align*} by repeated cofactor expansions along the first $n_{J_i}$ columns, with $p^C(\lambda)$ denoting the characteristic polynomial for $A^C$. Thus comparing the two expressions we know that 
\begin{align*}
p^C(\lambda) = (\lambda_j-\lambda)^{k_{J_i} - n_{J_i}}p^-(\lambda)    
\end{align*} Since we assume $n_{J_i} < k_{J_i}$, we know that $k_{J_i} - n_{J_i} \geq 1$, and $p^C(\lambda)$ will then contain some $(\lambda_j-\lambda)$ factor. Hence $A^C$, as a square matrix in its own right, must have an eigenvector $\vec{v}^{(n_{J_i}+1)}_{\lambda_{J_i}}$ in the $(\mathcal{E}_{J_i})^C$ basis corresponding to $\lambda_{J_i}$ since its geometric multiplicity in $A^C$ must be at least $1$. This $\vec{v}^{(n_{J_i}+1)}_{\lambda_{J_i}}$ is also an eigenvector of the entire matrix $A$ and will be linearly independent of $\vec{v}^{(1)}_{\lambda_{J_i}}, \vec{v}^{(2)}_{\lambda_{J_i}}, \ldots, \vec{v}^{(n_{J_i})}_{\lambda_{J_i}}$ because $\mathcal{E}_{J_i} \oplus (\mathcal{E}_{J_i})^C$ is a direct sum (see Definition \ref{defn:directsum}). Therefore, this shows that the geometric multiplicity of $\lambda_{J_i}$ in $A$ is actually $n_{J_i}+1$ which contradicts our hypothesis that it is $n_{J_i}$, whenever $n_{J_i} < k_{J_i}$. (We can also inductively use this argument until the geometric multiplicity adds up to $k_{J_i}$.) Hence the only reasonable conclusion is $n_{J_i} = k_{J_i}$. 
\end{proof}
\par
Properties \ref{proper:symortho} and \ref{proper:symnodefic} together imply the "if" part of Theorem \ref{thm:symdiag} and the proof is completed. Recall that the essence of orthogonal diagonalization is to search for an orthonormal basis such that the linear operator or square matrix is diagonal with respect to it. These orthonormal basis vectors are at the same time the eigenvectors of the symmetric matrix following Properties \ref{proper:diagonalize}. Hence as a corollary,
\begin{proper}
\label{proper:orthobasissym}
A real $n \times n$ matrix has $n$ linearly independent eigenvectors that form an orthonormal basis for $\mathbb{R}^n$ if and only if it is symmetric if and only if it is orthogonally diagonalizable.
\end{proper}

\begin{exmp}
\label{exmp:orthodiag}
Carry out orthogonal diagonalization on the matrix
\begin{align*}
A =
\begin{bmatrix}
1 & 0 & 0 \\
0 & 2 & 1 \\
0 & 1 & 2
\end{bmatrix}
\end{align*}
\end{exmp}
\begin{solution}
First, we observe that $A$ is real and symmetric, and can be orthogonally diagonalized according to Theorem \ref{thm:symdiag}. It can be found that the eigenvectors, after normalization, are
\begin{align*}
&\vec{v}_\lambda = 
\left[\renewcommand{\arraystretch}{1.2}
\begin{array}{wc{8pt}}
1 \\
0 \\
0
\end{array}\right] \text{ and }
\left[\begin{array}{wc{8pt}}
0 \\
-\frac{1}{\sqrt{2}} \\
\frac{1}{\sqrt{2}}
\end{array}\right]
& \text{for } \lambda = 1 \\
&\vec{v}_\lambda = 
\left[\begin{array}{wc{8pt}}
0 \\
\frac{1}{\sqrt{2}} \\
\frac{1}{\sqrt{2}}
\end{array}\right]
& \text{for } \lambda = 3
\end{align*}
Hence we can construct
\begin{align*}
P =
\left[\begin{array}{@{}wc{12pt}wc{12pt}wc{12pt}@{\,}}
1 & 0 & 0 \\
0 & -\frac{1}{\sqrt{2}} & \frac{1}{\sqrt{2}} \\
0 & \frac{1}{\sqrt{2}} & \frac{1}{\sqrt{2}}
\end{array}\right]
\end{align*}
So that
\begin{align*}
P^TAP &=
\left[\begin{array}{@{}wc{12pt}wc{12pt}wc{12pt}@{\,}}
1 & 0 & 0 \\
0 & -\frac{1}{\sqrt{2}} & \frac{1}{\sqrt{2}} \\
0 & \frac{1}{\sqrt{2}} & \frac{1}{\sqrt{2}}
\end{array}\right]^T
\begin{bmatrix}
1 & 0 & 0 \\
0 & 2 & 1 \\
0 & 1 & 2
\end{bmatrix}
\left[\begin{array}{@{}wc{12pt}wc{12pt}wc{12pt}@{\,}}
1 & 0 & 0 \\
0 & -\frac{1}{\sqrt{2}} & \frac{1}{\sqrt{2}} \\
0 & \frac{1}{\sqrt{2}} & \frac{1}{\sqrt{2}}
\end{array}\right] \\
&= 
\begin{bmatrix}
1 & 0 & 0\\
0 & 1 & 0\\
0 & 0 & 3 
\end{bmatrix} = D
\end{align*}
\end{solution}
Short Exercise: Confirm that $P$ is orthogonal.\footnote{It is simply checking
\begin{align*}
\begin{bmatrix}
1 & 0 & 0 \\
0 & -\frac{1}{\sqrt{2}} & \frac{1}{\sqrt{2}} \\
0 & \frac{1}{\sqrt{2}} & \frac{1}{\sqrt{2}}
\end{bmatrix}^T
\begin{bmatrix}
1 & 0 & 0 \\
0 & -\frac{1}{\sqrt{2}} & \frac{1}{\sqrt{2}} \\
0 & \frac{1}{\sqrt{2}} & \frac{1}{\sqrt{2}}
\end{bmatrix} = 
\begin{bmatrix}
1 & 0 & 0 \\
0 & 1 & 0 \\
0 & 0 & 1
\end{bmatrix}
\end{align*}
}

\subsubsection{Remark} From Properties \ref{proper:endomorph} we know that orthogonal diagonalization $D = P^{-1}AP = P^TAP$ is simply a special case of coordinate transformation for a square matrix, where $P$ is now orthogonal and only pure rotations/reflections are involved (Theorem \ref{thm:orthodet}).

\section{Orthogonal Projections and Spectral Theorem}

\subsection{Projections onto a Subspace}

Another important concept that also involves orthogonality is \textit{orthogonal projections}. Back in Section \ref{section:proj} we have defined the projection of a vector $\vec{v}$ onto another vector $\vec{u}$, which is implicitly an orthogonal projection since the orthogonal component of $\vec{v}$ normal to $\vec{u}$ is removed while the parallel component of $\vec{v}$ along $\vec{u}$ is retained, i.e.\ $\vec{v} = \overrightarrow{\text{proj}}_u v + (\vec{v} - \overrightarrow{\text{proj}}_u v)$ and $\vec{u} \cdot (\vec{v} - \overrightarrow{\text{proj}}_u v) = 0$. Here we can treat $\vec{u}$ as the one-dimensional subspace generated by itself, and the projection of any vector $\vec{v}$ onto $\vec{u}$ is an operation to project the vector onto this subspace of $\text{span}(\{\vec{u}\})$. However, before digging deep into orthogonal projections, we have to first generalize the notion of projections involving multi-dimensional subspaces. Given two subspaces $\mathcal{W}_1$ and $\mathcal{W}_2$ of a vector space $\mathcal{V}$ which forms a direct sum $\mathcal{W}_1 \oplus \mathcal{W}_2 = \mathcal{V}$, the \textit{projection of a vector onto $\mathcal{W}_1$ along $\mathcal{W}_2$} is a linear operator $T:\mathcal{V} \to \mathcal{V}$ such that for any $\vec{v} = \vec{w}_1 + \vec{w}_2$\footnote{This decomposition of $\vec{v}$ into $\vec{w}_1$ and $\vec{w}_2$ is unique as $\mathcal{W}_1 \oplus \mathcal{W}_2$ is required to be a direct sum. (see Section \ref{section:directsum})} where $\vec{w}_1 \in \mathcal{W}_1$ and $\vec{w}_2 \in \mathcal{W}_2$, $T(\vec{v}) = \vec{w}_1$, so that only the $\mathcal{W}_1$ component is kept (projected onto) while the $\mathcal{W}_2$ component is discarded. Clearly, $\mathcal{R}(T) = \mathcal{W}_1$ and $\mathcal{N}(T) = \mathcal{W}_2$ is the range and kernel of $T$, and $\mathcal{R}(T) \oplus \mathcal{N}(T) = \mathcal{V}$. Here we give another equivalent definition of a projection.
\begin{proper}[Projection]
\label{proper:matrixproj}
A linear operator $T: \mathcal{V} \to \mathcal{V}$ is a projection if and only if $T^2 = T$.
\end{proper}
\begin{proof}
If $T^2 = T$, then for any $\vec{v} \in \mathcal{V}$, we have $T^2(\vec{v}) = T(T(\vec{v})) = T(\vec{v})$, therefore for any vector $T(\vec{v}) = \vec{w}_1 \in \mathcal{W}_1 = \mathcal{R}(T)$ in the range of $T$, $T(\vec{w}_1) = \vec{w}_1$, and the $\mathcal{W}_1$ component remains unchanged. Now, we demand $\mathcal{W}_2$ such that we can write $\vec{v} = \vec{w}_1 + (\vec{v} - \vec{w}_1) = \vec{w}_1 + \vec{w}_2$ for any $\vec{v} \in \mathcal{V}$, where $\vec{v} - \vec{w}_1 = \vec{w}_2 \in \mathcal{W}_2$ and $\mathcal{W}_1 \oplus \mathcal{W}_2 = \mathcal{V}$\footnote{$\mathcal{W}_2$ can be shown to be a subspace and is linearly independent of $\mathcal{W}_1$. The actual projection matrix $T$ is not only determined by $\mathcal{W}_1$ but also $\mathcal{W}_2$, since the choice of $\mathcal{W}_2$ will dictate $\vec{v} - \vec{w}_1 \in \mathcal{W}_2$ and hence $\vec{w}_1$ too.}. Applying $T$ on both sides gives 
\begin{align*}
T(\vec{v}) &= T(\vec{w}_1 + \vec{v} - \vec{w}_1) \\
T(\vec{v}) &= T(\vec{w}_1) + T(\vec{v} - \vec{w}_1) & \text{(Linearity from Definition \ref{defn:lintrans})}\\
\vec{w}_1 &= \vec{w}_1 + T(\vec{v} - \vec{w}_1) \\
\implies T(\vec{v} - \vec{w}_1) &= \textbf{0}
\end{align*}
So $T(\vec{v} - \vec{w}_1) = T(\vec{w}_2) = \textbf{0}$ for any $\vec{v} \in \mathcal{V}$ and $\vec{w}_2 \in \mathcal{W}_2$ as well. This shows that $\mathcal{W}_2 = \mathcal{N}(T)$ is the kernel of $T$ and any $\mathcal{W}_2$ component is annihilated. The "only if" part is simpler: given the effect of projection $T$ as defined in the beginning, for any arbitrary $\vec{v}$, $T^2(\vec{v}) = T(T(\vec{v})) = T(\vec{w}_1) = \vec{w}_1 = T(\vec{v})$, so by Properties \ref{proper:sametrans} it must be that $T^2 = T$.
\end{proof}
A linear operator/matrix $T$ satisfying the condition $T^2 = T$ is known as \index{Idempotent}\keywordhl{idempotent}, and we have $T^n = T$ for any positive integer $n$.\footnote{For $n \geq 2$, $T^n = (T^2) (T^{n-2}) = (T) (T^{n-2}) = T^{n-1}$ which can be recursively used.} So in other words, a linear operator is a projection if and only if it is idempotent.

\begin{exmp}
\label{exmp:xyproj}
Show that the linear operator $T$ which has a matrix representation of
\begin{align*}
[T] =
\begin{bmatrix}
0 & 1 \\
0 & 1 
\end{bmatrix}
\end{align*}
is a projection. Onto/along which subspace this projection is? 
\end{exmp}
\begin{solution}
By Properties \ref{proper:matrixproj}, we have to check if $[T]^2 = [T]$. A simple calculation yields
\begin{align*}
[T]^2 &=
\begin{bmatrix}
0 & 1 \\
0 & 1 
\end{bmatrix}
\begin{bmatrix}
0 & 1 \\
0 & 1 
\end{bmatrix} \\
&= 
\begin{bmatrix}
(0)(0) + (1)(0) & (0)(1) + (1)(1) \\
(0)(0) + (1)(0) & (0)(1) + (1)(1)
\end{bmatrix} \\
&= 
\begin{bmatrix}
0 & 1 \\
0 & 1 
\end{bmatrix} = [T]
\end{align*}
So it is indeed a projection. The subspace onto which the linear operator projects, is simply equal to its range, or equivalently the column space of $[T]$, which can be immediately identified as $\text{span}(\{(1,1)^T\})$, that is, the straight line $y=x$. Similarly, the projection is along its kernel/null space, which is also easily seen to be $\text{span}(\{(1,0)^T\})$, the $x$-axis.\\
\begin{tikzpicture}
\draw[->] (-3,0)--(3,0) node[right]{$x$};
\draw[->] (0,-3)--(0,3) node[above]{$y$};
\draw[dashed, Green, line width=1.2] (-2.5,-2.5) -- (2.75,2.75) node[right]{$y=x$};
\draw[->, red, line width=1.5] (0,0) -- (0,2) node[left]{$(0,2)^T$};
\draw[->, blue, line width=1.5] (0,0) -- (2,2) node[below right, align=right]{$T((0,2)^T)$\\ $= (2,2)^T$};
\draw[->, red, line width=1.5] (0,0) -- (2,-1) node[right]{$(2,-1)^T$};
\draw[->, blue, line width=1.5] (0,0) -- (-1,-1) node[above left, yshift=-4, align=right]{$T((2,-1)^T)$ \\ $= (-1,-1)^T$};
\draw[dashed, gray, line width=1.2] (0,2) -- (2,2);
\draw[dashed, gray, line width=1.2] (2,-1) -- (-1,-1);
\node[below left]{$O$}; 
\end{tikzpicture}
\end{solution}

\subsection{Orthogonal Projections}
\label{subsection:orthoproj}

After knowing how projections onto a subspace look like, we can now properly generalize the \index{Orthogonal Projection}\keywordhl{orthogonal projection} (not to be confused with an orthogonal matrix) of a vector onto another vector, to a (possibly multi-dimensional) subspace. For a projection onto a subspace $\mathcal{W}_1$ along another subspace $\mathcal{W}_2$ to be an orthogonal projection, these two subspaces have to be the orthogonal complement to each other, such that for any $\vec{v} = \vec{w}_1 + \vec{w}_2$, $\vec{w}_1 \in \mathcal{W}_1$ and $\vec{w}_2 \in \mathcal{W}_2$, $\vec{w}_1$ and $\vec{w}_2$ are orthogonal (i.e.\ $\vec{w}_1 \cdot \vec{w}_2 = 0$, $\mathcal{W}_1^\perp = \mathcal{W}_2$ and $\mathcal{W}_2^\perp = \mathcal{W}_1$), and $T(\vec{v}) = \vec{w}_1$ means that the orthogonal component $\vec{w}_2 \in \mathcal{W}_2$ normal to $\mathcal{W}_1$ is removed. As there is only one orthogonal complement for any (finite-dimensional) subspace, the range $\mathcal{R}(T) = \mathcal{W}_1$ uniquely determines $\mathcal{N}(T) = \mathcal{W}_2$ and hence $T$, where $\mathcal{R}(T)^\perp = \mathcal{N}(T)$ and $\mathcal{N}(T)^\perp = \mathcal{R}(T)$ now. An equivalent condition of an orthogonal projection is that $[T] = [T]^T$ is symmetric given we are working over $\mathbb{R}$.\footnote{Some may wonder why in Properties \ref{proper:matrixproj} we use $T$ directly but here we circumvent by employing the matrix representation $[T]$ instead. It is because we haven't defined the "transpose" or "symmetric" equivalent for a linear operator, which will be introduced as the adjoint in Chapter \ref{chap:innerchap}.}
\begin{proper}[Orthogonal Projection]
\label{proper:orthogonalproj}
A \textit{real}, \textit{finite-dimensional} linear projection operator $T: \mathcal{V} \to \mathcal{V}$ is an orthogonal projection (with respect to the usual real dot product) if and only if $[T] = [T]^T$ in terms of its matrix representation.  
\end{proper}
\begin{proof}
The "if" part: We need to show that $[T] = [T]^T$ implies $\mathcal{R}(T)^\perp = \mathcal{N}(T)$ (and $\mathcal{N}(T)^\perp = \mathcal{R}(T)$)\footnote{This two are equivalent as $\mathcal{V}$ is finite-dimensional.}. Let $\vec{w}_1 \in \mathcal{R}(T)$ and $\vec{w}_2 \in \mathcal{N}(T)$, then 
\begin{align*}
\vec{w}_1 \cdot \vec{w}_2 &= ([T]\vec{w}_1) \cdot \vec{w}_2 & \text{($T(\vec{w}_1) = \vec{w}_1$ by the definition of a projection)} \\
&= ([T]^T\vec{w}_1) \cdot \vec{w}_2 \\
&= \vec{w}_1 \cdot ([T]\vec{w}_2) & \text{(Properties \ref{proper:dotproper})} \\
&= \vec{w}_1 \cdot \textbf{0} = 0 & \text{($T(\vec{w}_2) = \textbf{0}$ by the definition of a projection)}
\end{align*}
So any $\vec{w}_2 \in \mathcal{N}(T)$ will be orthogonal to all $\vec{w}_1 \in \mathcal{R}(T)$ and $\mathcal{N}(T) \subseteq \mathcal{R}(T)^\perp$. Now let $\vec{w}_3 \in \mathcal{R}(T)^\perp$ and we want to show $\vec{w}_3 \in \mathcal{N}(T)$, i.e.\ $T(\vec{w}_3) = \textbf{0}$, such that $\mathcal{R}(T)^\perp \subseteq \mathcal{N}(T)$ and thus $\mathcal{R}(T)^\perp = \mathcal{N}(T)$. Consider $\vec{w}_3 \cdot T^2(\vec{w}_3)$ where $T^2(\vec{w}_3) \in \mathcal{R}(T)$ since $T^2 = T$ represents the action of projection (Properties \ref{proper:matrixproj}), and therefore $\vec{w}_3 \cdot T^2(\vec{w}_3) = 0$ as $\vec{w}_3 \in \mathcal{R}(T)^\perp$, but also
\begin{align*}
\vec{w}_3 \cdot T^2(\vec{w}_3) = \vec{w}_3 \cdot ([T]^2\vec{w}_3) &= \vec{w}_3 \cdot ([T]^T[T]\vec{w}_3) & \text{(By assumption)} \\
&= ([T]\vec{w}_3) \cdot ([T]\vec{w}_3) & \text{(Properties \ref{proper:dotproper})} \\
&= \norm{[T]\vec{w}_3}^2 = \norm{T(\vec{w}_3)}^2 
\end{align*}
Since this quantity is shown to be zero, $\norm{T(\vec{w}_3)} = 0$ and $T(\vec{w}_3) = \textbf{0}$ (see the remark below Properties \ref{proper:lengthdot}), and we are done.
\\
The "only if" part: Let $T$ be an orthogonal projection, $\vec{u} = \vec{w}_1^{(1)} + \vec{w}_2^{(1)}$ and $\vec{v} = \vec{w}_1^{(2)} + \vec{w}_2^{(2)}$ where $\vec{w}_1^{(1)}, \vec{w}_1^{(2)} \in \mathcal{W}_1 = \mathcal{R}(T)$ and $\vec{w}_2^{(1)}, \vec{w}_2^{(2)} \in \mathcal{W}_2 = \mathcal{N}(T)$. Consider 
\begin{align*}
\vec{u} \cdot T(\vec{v}) = (\vec{w}_1^{(1)} + \vec{w}_2^{(1)}) \cdot (\vec{w}_1^{(2)}) &= \vec{w}_1^{(1)} \cdot \vec{w}_1^{(2)} + \vec{w}_2^{(1)} \cdot \vec{w}_1^{(2)} \\
&= \vec{w}_1^{(1)} \cdot \vec{w}_1^{(2)}
\end{align*}
where $\vec{w}_2^{(1)} \cdot \vec{w}_1^{(2)} = 0$ because $\mathcal{N}(T) = \mathcal{R}(T)^\perp$. Similarly we have $[T]\vec{u} \cdot \vec{v} = T(\vec{u}) \cdot \vec{v} = \vec{w}_1^{(1)} \cdot \vec{w}_1^{(2)} + \vec{w}_1^{(1)} \cdot \vec{w}_2^{(2)} = \vec{w}_1^{(1)} \cdot \vec{w}_1^{(2)} = \vec{u} \cdot T(\vec{v})$. At the same time
\begin{align*}
\vec{u} \cdot T(\vec{v}) &= \vec{u} \cdot ([T]\vec{v}) \\
&= ([T]^T\vec{u}) \cdot \vec{v} & \text{(Properties \ref{proper:dotproper})}
\end{align*}
This shows that $([T]\vec{u}) \cdot \vec{v} = ([T]^T\vec{u}) \cdot \vec{v}$ for any $\vec{u}, \vec{v} \in \mathcal{V}$. Particularly, since this equality holds for any $\vec{v} \in \mathcal{V}$, it is always true that $[T]\vec{u} = [T]^T\vec{u}$, which further holds for any $\vec{u} \in \mathcal{V}$ and we conclude that $[T] = [T]^T$.
\end{proof}

\begin{exmp}
Example \ref{exmp:xyproj} has illustrated a projection in $\mathbb{R}^2$ onto the straight line $y = x$ along the $x$-axis. Continuing from this example, find the unique orthogonal projection onto the same line of $y = x$ correspondingly.
\end{exmp}
\begin{solution}
The matrix representation of $T$ is derived following Definition \ref{defn:matrixrepoflintrans}, during which the standard coordinates are used:
\begin{align*}
[T] =
\begin{bmatrix}
T(\hat{e}^{(1)})|T(\hat{e}^{(2)})
\end{bmatrix}
\end{align*}
Its columns are the outputs of applying the orthogonal projection on the standard unit vectors. Geometrically (see the figure below), it can be easily deduced that $T(\hat{e}^{(1)}) = T(\hat{e}^{(2)}) = (\frac{1}{2}, \frac{1}{2})^T$, and hence
\begin{align*}
[T] =
\begin{bmatrix}
\frac{1}{2} & \frac{1}{2} \\
\frac{1}{2} & \frac{1}{2} 
\end{bmatrix}
\end{align*}
As a small example, given $\vec{v}=(1,-2)^T$, then the orthogonal projection of $\vec{v}$ onto the straight $y=x$ is computed as
\begin{align*}
T(\vec{v}) = [T](1,-2)^T = 
\begin{bmatrix}
\frac{1}{2} & \frac{1}{2} \\
\frac{1}{2} & \frac{1}{2}     
\end{bmatrix}
\begin{bmatrix}
1 \\
-2
\end{bmatrix}
=
\begin{bmatrix}
-\frac{1}{2} \\
-\frac{1}{2}
\end{bmatrix}
\end{align*}
\par \begin{tikzpicture}[scale=1.5]
\draw[->] (-2,0)--(2,0) node[right]{$x$};
\draw[->] (0,-2)--(0,2) node[above]{$y$};
\draw[dashed, Green, line width=1.2] (-1.75,-1.75) -- (1.875,1.875) node[right]{$y=x$};
\draw[->, Gray, line width=1.5] (0,0) -- (1,0) node[below right]{$\hat{e}^{(1)}$};
\draw[->, Gray, line width=1.5] (0,0) -- (0,1) node[left]{$\hat{e}^{(2)}$};
\draw[->, Green, line width=1.5] (0,0) -- (0.5,0.5) node[yshift=-2, right]{$T(\hat{e}^{(1)}) = T(\hat{e}^{(2)}) = (\frac{1}{2}, \frac{1}{2})^T$};
\draw[->, red, line width=1.5] (0,0) -- (1,-2) node[right]{$\vec{v} = (1,-2)^T$};
\draw[->, blue, line width=1.5] (0,0) -- (-0.5,-0.5) node[left, yshift=2, align=left]{$T(\vec{v}) = T((1,-2)^T)$ \\ $= (-\frac{1}{2},-\frac{1}{2})^T$};
\draw[dashed, gray, line width=1.2] (1,0) -- (0.5,0.5);
\draw[dashed, gray, line width=1.2] (0,1) -- (0.5,0.5);
\draw[dashed, gray, line width=1.2] (1,-2) -- (-0.5,-0.5);
\draw[gray, line width=1.2] (-0.35,-0.35) -- (-0.2,-0.5) -- (-0.35,-0.65);
\node[below left]{$O$}; 
\end{tikzpicture}
\end{solution}

\begin{exmp}
\label{exmp:planeorthoproj}
Find the matrix for the orthogonal projection in $\mathbb{R}^3$ onto the subspace described by the plane $x + 2y - 4z = 0$.
\end{exmp}
\begin{solution}
In a higher-dimensional scenario, it is beneficial to derive an orthonormal basis for both the range $\mathcal{R}(T)$ and kernel $\mathcal{N}(T)$ of the orthogonal projection $T$ first, and work out the projection transformation in the coordinate system according to this basis. We can follow a similar approach as in Example \ref{exmp:vecgeohighdim}, and one possible basis for the plane is simply $\{(4,0,1)^T, (0,2,1)^T\}$. Subsequently, we can apply Gram-Schmidt orthogonalization (Definition \ref{defn:GSorth_norm}) to obtain an orthonormal basis for the plane. We leave to the readers to check that it is $\{(\frac{4}{\sqrt{17}}, 0, \frac{1}{\sqrt{17}})^T, (-\frac{2}{\sqrt{357}}, \frac{17}{\sqrt{357}}, \frac{8}{\sqrt{357}})^T\}$. Motivated by Properties \ref{proper:orthodirectsum}, we will derive the third vector to complete an orthonormal basis for $\mathbb{R}^3$ and it will represent the kernel $\mathcal{N}(T) = \mathcal{R}(T)^\perp$ as the orthogonal complement to the plane. A quick way is to utilize cross product which yields $(\frac{4}{\sqrt{17}}, 0, \frac{1}{\sqrt{17}})^T \times (-\frac{2}{\sqrt{357}}, \frac{17}{\sqrt{357}}, \frac{8}{\sqrt{357}})^T = (-\frac{1}{\sqrt{21}}, -\frac{2}{\sqrt{21}}, \frac{4}{\sqrt{21}})^T$. The projection in this basis $\beta = \{(\frac{4}{\sqrt{17}}, 0, \frac{1}{\sqrt{17}})^T, (-\frac{2}{\sqrt{357}}, \frac{17}{\sqrt{357}}, \frac{8}{\sqrt{357}})^T, (-\frac{1}{\sqrt{21}}, -\frac{2}{\sqrt{21}}, \frac{4}{\sqrt{21}})^T\}$ will have a matrix representation of
\begin{align*}
[T]_\beta = 
\begin{bmatrix}
1 & 0 & 0 \\
0 & 1 & 0 \\
0 & 0 & 0
\end{bmatrix}
\end{align*}
as the first two basis vectors are in the range $\mathcal{R}(T)$ and will be projected into itself, while the last basis vector is in the kernel $\mathcal{N}(T)$ and hence annihilated. Now we can transform the projection matrix back into the standard basis system according to Properties \ref{proper:endomorph}, where
\begin{align*}
P_\beta^S = 
\renewcommand\arraystretch{1.25}
\begin{bmatrix}
\frac{4}{\sqrt{17}} & -\frac{2}{\sqrt{357}} & -\frac{1}{\sqrt{21}} \\
0 & \frac{17}{\sqrt{357}} & -\frac{2}{\sqrt{21}} \\
\frac{1}{\sqrt{17}} & \frac{8}{\sqrt{357}} & \frac{4}{\sqrt{21}}
\end{bmatrix}
\end{align*}
and
\begin{align*}
[T]_S &= (P_S^\beta)^{-1} [T]_\beta P_S^\beta \\
&= P_\beta^S [T]_\beta (P_\beta^S)^T \text{\quad (Properties \ref{proper:orthoinvT} for the orthogonal $P$ matrix)} \\
&=
\renewcommand\arraystretch{1.25}
\begin{bmatrix}
\frac{4}{\sqrt{17}} & -\frac{2}{\sqrt{357}} & -\frac{1}{\sqrt{21}} \\
0 & \frac{17}{\sqrt{357}} & -\frac{2}{\sqrt{21}} \\
\frac{1}{\sqrt{17}} & \frac{8}{\sqrt{357}} & \frac{4}{\sqrt{21}}
\end{bmatrix}
\begin{bmatrix}
1 & 0 & 0 \\
0 & 1 & 0 \\
0 & 0 & 0
\end{bmatrix}
\begin{bmatrix}
\frac{4}{\sqrt{17}} & -\frac{2}{\sqrt{357}} & -\frac{1}{\sqrt{21}} \\
0 & \frac{17}{\sqrt{357}} & -\frac{2}{\sqrt{21}} \\
\frac{1}{\sqrt{17}} & \frac{8}{\sqrt{357}} & \frac{4}{\sqrt{21}}
\end{bmatrix}^T \\
&=
\renewcommand\arraystretch{1.25}
\begin{bmatrix}
\frac{20}{21}&-\frac{2}{21}&\frac{4}{21}\\ 
-\frac{2}{21}&\frac{17}{21}&\frac{8}{21}\\ 
\frac{4}{21}&\frac{8}{21}&\frac{5}{21}
\end{bmatrix}
\end{align*}
To cross-check that $[T]_S$ indeed represents an orthogonal projection, it is obvious that $[T]_S$ is symmetric and the readers can verify that $[T]_S^2 = [T]_S$. Another cross-checking method is to pick a vector in the range (kernel), e.g. $\vec{v}_R = (6,1,2)^T$ and confirm that the projection $[T]_S\vec{v}_R = [T]_S(6,1,2)^T = (6,1,2)^T = \vec{v}_R$ leaves it unchanged (vanished).
\end{solution}
Now we can revisit the projection of a vector $\vec{v}$ onto another vector $\vec{u}$ as a special case of orthogonal projection in this section. In Definition \ref{proper:proj}, the projection acting on $\vec{v}$ is given as \begin{align*}
\overrightarrow{\text{proj}}_u v = \frac{\vec{v} \cdot \vec{u}}{\norm{\vec{u}}^2} \vec{u}    
\end{align*} which can be rewritten as
\begin{align*}
T(\vec{v}^T) &= \frac{1}{\norm{\textbf{u}}^2}(\textbf{v}^T\textbf{u})\textbf{u}^T \\
\implies T(\vec{v}) &= \frac{1}{\norm{\textbf{u}}^2}(\textbf{v}^T\textbf{u}\textbf{u}^T)^T \\
&= (\frac{1}{\norm{\textbf{u}}^2}\textbf{u}\textbf{u}^T)\textbf{v} \\
&= (\frac{\textbf{u}}{\norm{\textbf{u}}}(\frac{\textbf{u}}{\norm{\textbf{u}}})^T)\textbf{v} = [T]\textbf{v}
\end{align*}
where the orthogonal projection matrix is 
\begin{align*}
[T] = \frac{\textbf{u}}{\norm{\textbf{u}}}(\frac{\textbf{u}}{\norm{\textbf{u}}})^T = \hat{\textbf{u}}\hat{\textbf{u}}^T    
\end{align*} which is clearly symmetric and can be shown that $[T]^2 = [T]$.\footnote{$(\hat{\textbf{u}}\hat{\textbf{u}}^T)(\hat{\textbf{u}}\hat{\textbf{u}}^T) = \hat{\textbf{u}}(\hat{\textbf{u}}^T\hat{\textbf{u}})\hat{\textbf{u}}^T = \hat{\textbf{u}}(1)\hat{\textbf{u}}^T = \hat{\textbf{u}}\hat{\textbf{u}}^T$ as $\hat{\textbf{u}}$ is a unit vector and the dot product $\hat{\textbf{u}}^T\hat{\textbf{u}} = \vec{u}\cdot\vec{u} = \norm{\vec{u}}^2 = 1$ is its square length which simply equals $1$.} For an orthogonal projection $T$ in $\mathbb{R}^n$ onto some $r$-dimensional subspace, which becomes its range $\mathcal{R}(T)$ with an orthonormal basis $\{\vec{w}^{(1)}, \vec{w}^{(2)}, \ldots, \vec{w}^{(r)}\}$, complete this orthonormal basis for $\mathbb{R}^n$ by appending another orthonormal basis $\{\vec{w}^{(r+1)}, \ldots, \vec{w}^{(n)}\}$ of its kernel $\mathcal{N}(T) = \mathcal{R}(T)^\perp$ (possible due to Definition \ref{defn:GSorth_norm} and Properties \ref{proper:orthodirectsum}) just like Example \ref{exmp:planeorthoproj} above. By the same argument in that example, we know that
\begin{align}
[T] &= 
\footnotesize
\begin{bmatrix}
\vec{w}^{(1)}|\cdots|\vec{w}^{(r)}|\vec{w}^{(r+1)}|\cdots|\vec{w}^{(n)}
\end{bmatrix}
\begin{bmatrix}
I_r & [\textbf{0}]_{r\times(n-r)} \\
[\textbf{0}]_{(n-r)\times r} & [\textbf{0}]_{(n-r)\times(n-r)} \\
\end{bmatrix}
\left[\begin{array}{c} 
\vec{w}^{(1)T} \\
\hline
\vdots \\
\hline
\vec{w}^{(r)T} \Tstrut \\
\hline
\vec{w}^{(r+1)T} \Tstrut\\
\hline 
\vdots \\
\hline
\vec{w}^{(n)T} \Tstrut
\end{array}\right] \nonumber \\
&= \hat{\textbf{w}}^{(1)}\hat{\textbf{w}}^{(1)T} + \hat{\textbf{w}}^{(2)}\hat{\textbf{w}}^{(2)T} + \cdots + \hat{\textbf{w}}^{(r)}\hat{\textbf{w}}^{(r)T} \label{eqn:projrank1sum}
\end{align}
which generalizes the form of an orthogonal projection matrix above in the case of a multi-dimensional range. Furthermore, geometrically, the shortest distance of a point to a subspace is found by the orthogonal projection of that point onto the subspace.
\begin{proper}
\label{proper:shortestorthoproj}
The shortest distance of a subspace $\mathcal{W} \subset \mathcal{V}$ to some point $\vec{v} \in \mathcal{V}$ is achieved by $\vec{v} - T(\vec{v})$ where $T$ is the orthogonal projection operator onto $\mathcal{W}$.
\end{proper}
\begin{proof}
For any other point $\vec{w} \in \mathcal{W}$, we have
\begin{align*}
\norm{\vec{v} - \vec{w}}^2 &= \norm{(\vec{v} - T(\vec{v})) + (T(\vec{v}) - \vec{w})}^2 \\
&= \norm{\vec{v} - T(\vec{v})}^2 + \norm{T(\vec{v}) - \vec{w}}^2 + 2 (\vec{v} - T(\vec{v})) \cdot (T(\vec{v}) - \vec{w})
\end{align*}
Note that $\vec{v} - T(\vec{v}) \in \mathcal{N}(T) = \mathcal{R}(T)^\perp$ and $T(\vec{v}) - \vec{w} \in \mathcal{R}(T)$ as $T$ is an orthogonal projection. Therefore, $(\vec{v} - T(\vec{v})) \cdot (T(\vec{v}) - \vec{w}) = 0$, and 
\begin{align*}
\norm{\vec{v} - \vec{w}}^2 &= \norm{\vec{v} - T(\vec{v})}^2 + \norm{T(\vec{v}) - \vec{w}}^2 \\
& > \norm{\vec{v} - T(\vec{v})}^2
\end{align*}
as $\vec{w} \neq T(\vec{v})$, $\norm{T(\vec{v}) - \vec{w}}^2 > 0$.
\end{proof}
\subsection{Spectral Theorem}
With orthonormal matrices and projections defined, we have come to the most important theorem in this chapter, the \index{Spectral Theorem}\keywordhl{Spectral Theorem}.
\begin{thm}[Spectral Theorem]
\label{thm:spectral}
For a linear operator $T: \mathcal{V} \to \mathcal{V}$ where $\mathcal{V}$ is a real, finite($n$)-dimensional vector space and the matrix representation of $T$ is symmetric, i.e.\ $[T]=[T]^T$, denote its eigenvalues by $\lambda_j$, $j = 1,2,\ldots,n$ (counting repeated ones). Assume there are $k$ distinct eigenvalues out of them. For every such a distinct eigenvalue, collect all the indices $j$ where $\lambda_j$ points to that same eigenvalue, let's say, $\lambda_{J_i}$, and group them into an index set $J_i$, $i = 1,2,\ldots,k$. Let the associated eigenspace be $\mathcal{E}_{J_i}$ for the $k$ sets of $J_i$ and denote the orthogonal projection onto $\mathcal{E}_{J_i}$ as $T_{J_i}$, then we have:
\begin{enumerate}[label=(\alph*)]
\item $\mathcal{V} = \mathcal{E}_{J_1} \oplus \mathcal{E}_{J_2} \oplus \cdots \oplus \mathcal{E}_{J_k} = \bigoplus_{i=1}^{k} \mathcal{E}_{J_i}$;
\item $(\bigoplus_{i \in \{I\}} \mathcal{E}_{J_i})^\perp = \bigoplus_{i \notin \{I\}} \mathcal{E}_{J_i}$ where $I$ is an index set containing some integers between $1$ and $k$;
\item $T_{J_i} T_{J_{i'}} = 
\begin{cases}
T_{J_i} & \text{if $i = i'$} \\
0 & \text{if $i \neq i'$}
\end{cases}$;
\item $I = T_{J_1} + T_{J_2} + \cdots + T_{J_k} = \sum_{i=1}^{k} T_{J_i}$; and
\item $T = \lambda_{J_1}T_{J_1} + \lambda_{J_2}T_{J_2} + \cdots + \lambda_{J_k}T_{J_k} = \sum_{i=1}^{k} \lambda_{J_i}T_{J_i}$.
\end{enumerate}
\end{thm}
\begin{proof}
\begin{enumerate}[label=(\alph*)]  
\item By Theorem \ref{thm:symdiag}, $[T]$ is (orthogonally) diagonalizable. The prior discussion in Section \ref{section:diagonalproper} has shown that
\begin{align*}
\bigoplus_{i=1}^{k} \mathcal{E}_{J_i} = \bigoplus_{j=1}^n \mathcal{E}_j = \mathcal{V} 
\end{align*}
where $\mathcal{E}_{J_i} = \bigoplus_{j \in J_i} \mathcal{E}_j$ is the direct sum of one-dimensional subspaces generated by each of the linearly independent eigenvector corresponding to the same eigenvalue $\lambda_{J_i}$.
\item This is a consequence of applying Properties \ref{proper:orthodirectsum} on (a) where $\bigoplus_{i \in \{I\}} \mathcal{E}_{J_i}$ and $\bigoplus_{i \notin \{I\}} \mathcal{E}_{J_i}$ are derived from the two mutually exclusive portions of the orthonormal basis for $\mathcal{V}$ that contains the $n$ orthonormal eigenvectors of the symmetric $[T]$, available due to the implication of Theorem \ref{proper:orthobasissym}.
\item $T_{J_i}^2 = T_{J_i}$ by the definition of a projection (Properties \ref{proper:matrixproj}). Also, we can express any $\vec{v} = \vec{v}_{J_1} + \vec{v}_{J_2} + \cdots + \vec{v}_{J_k}$ by (a) where $\vec{v}_{J_i} \in \mathcal{E}_{J_i}$. Then $T_{J_{i'}}(\vec{v}) = \vec{v}_{J_{i'}}$ due to the definition of an orthogonal projection and (b) as $\mathcal{E}_{J_{i'}}^\perp = \bigoplus_{i \neq i'} \mathcal{E}_{J_i}$. Using the same argument again means that $T_{J_i}(T_{J_i'}(\vec{v})) = T_{J_i}(\vec{v}_{J_{i'}}) = \textbf{0}$ if $i \neq i'$. Since this holds for any $\vec{v}$, $T_{J_i} T_{J_{i'}} = 0$.
\item Again for any $\vec{v}$ we can rewrite it as $\vec{v} = \vec{v}_{J_1} + \vec{v}_{J_2} + \cdots + \vec{v}_{J_k}$, $\vec{v}_{J_i} \in \mathcal{E}_{J_i}$. Since $T_{J_i}$ is an orthogonal projection onto $\mathcal{E}_{J_i}$ and also $\mathcal{E}_{J_{i'}}^\perp = \bigoplus_{i \neq i'} \mathcal{E}_{J_i} = \mathcal{N}(T_{J_i})$ as derived in (c), $T_{J_i}(\vec{v}) = \vec{v}_{J_i}$. Thus $I(\vec{v}) = \vec{v} = \vec{v}_{J_1} + \vec{v}_{J_2} + \cdots + \vec{v}_{J_k} = T_{J_1}(\vec{v}) + T_{J_2}(\vec{v}) + \cdots + T_{J_k}(\vec{v}) = (T_{J_1} + T_{J_2} + \cdots + T_{J_k})(\vec{v})$ and it must be that $I = T_{J_1} + T_{J_2} + \cdots + T_{J_k}$ by Properties \ref{proper:sametrans}.
\item As usual, $\vec{v} = \vec{v}_{J_1} + \vec{v}_{J_2} + \cdots + \vec{v}_{J_k}$ and
\begin{align*}
T(\vec{v}) &= T(\vec{v}_{J_1}) + T\vec{v}_{J_2} + \cdots + T(\vec{v}_{J_k}) \\
&= \lambda_{J_1}\vec{v}_{J_1} + \lambda_{J_2}\vec{v}_{J_2} + \cdots + \lambda_{J_k}\vec{v}_{J_k} \quad \begin{aligned}
\text{(Definition \ref{defn:eigen} for} \\
\text{eigenvalues/eigenvectors)}
\end{aligned} \\
&= \lambda_{J_1}T_{J_1}(\vec{v}) + \lambda_{J_2}T_{J_2}(\vec{v}) + \cdots + \lambda_{J_k}T_{J_k}(\vec{v}) \text{\; (Just as (d))} \\
&= (\lambda_{J_1}T_{J_1} + \lambda_{J_2}T_{J_2} + \cdots + \lambda_{J_k}T_{J_k})(\vec{v})
\end{align*}
and thus the desired relation follows, again by Properties \ref{proper:sametrans}.
\end{enumerate}
\end{proof}
The set of eigenvalues $\{\lambda_{J_i}\}$, $i = 1,2,\ldots,k$ is known as the \index{Spectrum}\keywordhl{spectrum} of $T$, hence comes the name of the theorem. The expression in (d) is called the \index{Resolution of the Identity}\keywordhl{resolution of the identity} induced by $T$, and that in (e) is referred to as the \index{Spectral Decomposition}\keywordhl{spectral decomposition} of $T$. The implications of these two parts of the Spectral Theorem are as follows: The resolution of the identity in (d) means that any vector $\vec{v}$ can be decomposed into components in the orthogonal coordinate system derived from the orthonormal eigenvectors of a symmetric $T$, where the corresponding coordinates are given by simply projecting it onto each of these eigenspaces; The spectral decomposition in (e) means that any symmetric linear operator $T$, when applied on a vector $\vec{v}$, can be treated as equivalent to first decomposing $\vec{v}$ into components in the orthogonal coordinate system as suggested by (d), and then scaling each of these components according to the spectrum (eigenvalues), which represents the "intensity" of $T$ in the respective eigenspace. Since only projections and scalings are involved, it implies that $T$ constitutes no rotation. Finally, notice that an orthogonal projection is a special case of (e) where the spectrum consists of $1$ (range) and $0$ (kernel) only.\par
Short Exercise: Verify statements (d) and (e) in the Spectral Theorem for the symmetric matrix in Example \ref{exmp:orthodiag}.\footnote{We will just show (d) here and leave (e) to the readers. By Equation (\ref{eqn:projrank1sum}), R.H.S. of (d) reads
\begin{align*}
T_{J_1} + T_{J_2} &= (\textbf{v}_{J_1}^{(1)T}\textbf{v}_{J_1}^{(1)T} + \textbf{v}_{J_1}^{(2)T}\textbf{v}_{J_1}^{(2)T}) + \textbf{v}_{J_2}^{(1)T}\textbf{v}_{J_2}^{(1)T} \\
&= \begin{bmatrix}
1 \\
0 \\
0
\end{bmatrix}
\begin{bmatrix}
1 & 0 & 0
\end{bmatrix} + 
\begin{bmatrix}
0 \\
-\frac{1}{\sqrt{2}} \\
\frac{1}{\sqrt{2}}
\end{bmatrix}
\begin{bmatrix}
0 & -\frac{1}{\sqrt{2}} & \frac{1}{\sqrt{2}}
\end{bmatrix} +
\begin{bmatrix}
0 \\
\frac{1}{\sqrt{2}} \\
\frac{1}{\sqrt{2}}
\end{bmatrix}
\begin{bmatrix}
0 & \frac{1}{\sqrt{2}} & \frac{1}{\sqrt{2}}
\end{bmatrix} \\
&=
\begin{bmatrix}
1 & 0 & 0 \\
0 & 0 & 0 \\
0 & 0 & 0
\end{bmatrix}
+
\begin{bmatrix}
0 & 0 & 0 \\
0 & \frac{1}{2} & -\frac{1}{2} \\
0 & -\frac{1}{2} & \frac{1}{2}
\end{bmatrix}
+
\begin{bmatrix}
0 & 0 & 0 \\
0 & \frac{1}{2} & \frac{1}{2} \\
0 & \frac{1}{2} & \frac{1}{2}
\end{bmatrix}\
=
\begin{bmatrix}
1 & 0 & 0 \\
0 & 1 & 0 \\
0 & 0 & 1
\end{bmatrix}
\end{align*}
where the set indices $J_1$ and $J_2$ indicate the eigenvectors for the eigenvalues of $\lambda = 1$ and $3$ respectively.
}

\section{Normal Matrices and Unitary Diagonalization}

\subsection{Unitary and Normal Matrices} 
We now generalize the ideas of orthogonal matrices/diagonalization to complex vector space. The complex counterpart of orthogonal matrices is known as \index{Unitary Matrix}\keywordhl{unitary matrices}.
\begin{defn}[Unitary Matrix]
\label{defn:unitary}
A complex matrix $A$ is said to be unitary if
\begin{align*}
A^*A = AA^* = I
\end{align*}
where the superscript $^*$ denotes conjugate transpose (Definition \ref{defn:conjutrans}).
\end{defn}
Along the same line, the complex equivalent of orthogonal diagonalization is \index{Unitary Diagonalization}\keywordhl{unitary diagonalization}.
\begin{defn}
\label{defn:unitarydiag}
A complex square matrix $A$ is said to be unitarily diagonalizable if there is a unitary matrix $U$ such that $U^* AU = D$ yields a complex diagonal matrix whose diagonal entries are the eigenvalues of $A$.
\end{defn}
This can be compared to Definition \ref{defn:orthodiagonal}. In addition, since we know that the availability of orthogonal diagonalization depends on whether the matrix is symmetric, we may want to know if there is a comparable criterion for unitary diagonalization. A reasonable guess is the matrix $A$ being \textit{Hermitian} (Definition \ref{defn:Hermitian}) such that $A^* = A$ since it is the direct complex analog of a symmetric matrix. However, the criterion is actually looser where the matrix only needs to be \index{Normal}\keywordhl{normal}.
\begin{defn}[Normal Matrix]
A matrix $A$ is known as normal whenever
\begin{align*}
A^*A = AA^*    
\end{align*}
which may or may not be equal to the identity.
\end{defn}
So normal is a less strict condition than unitary. It is also very easy to see that a Hermitian matrix is always normal. In the next part, we are going to derive how unitary diagonalization can happen on a normal matrix.

\subsection{Unitary Diagonalization}

As in Properties \ref{proper:diagonalize}, the unitary matrix $U$ that diagonalizes some $n \times n$ complex, normal matrix $A$ is formed by the $n$ linearly independent eigenvectors of $A$ arranged in columns, which also like in Definition \ref{defn:orthodiagonal} these eigenvectors have to be orthonormal. Note that the eigenvectors are now $\mathbb{C}^n$ and orthogonality is defined with respect to the complex dot product (Definition \ref{defn:complexdotproduct}) which has a complex conjugate applied on the second input vector. To show the equivalence between a matrix being unitarily diagonalizable and normal, we can of course follow the similar steps in the proof of Theorem \ref{thm:symdiag} that uses Properties \ref{proper:symortho} and \ref{proper:symnodefic}. However, an alternative approach of utilizing the \keywordhl{Schur's Triangularization Theorem} will be adopted here.

\begin{thm}[Schur's Triangularization Theorem]
\label{thm:schurtrig}
For any square complex (real) matrix $A$, there exists a unitary (orthogonal) matrix $U$ ($P$) such that $S = U^*AU$ ($P^TAP$) is upper-triangular, where the eigenvalues of $A$, counting algebraic multiplicities, are located along the main diagonal of $S$, if the characteristic polynomial of $A$ splits over $\mathbb{C}$ ($\mathbb{R}$).
\end{thm}
Notice the requirement at the end of the theorem (see Footnote \ref{foot:split} of Chapter \ref{chap:eigen}) and recall that every polynomial splits over $\mathbb{C}$. So in other words, every square matrix is \textit{unitariliy similar} to an upper-triangular matrix. (But not always orthogonally similar to a real diagonal matrix as the characteristic polynomial may not split over $\mathbb{R}$\footnote{Its splitting over $\mathbb{R}$ is equivalent to all eigenvalues being real.}.)
\begin{proof}
We will use mathematical induction on the size of $A$ to show the theorem for the complex case. If $A$ is $1 \times 1$ then the result is automatic. Then assume $A$ is $n \times n$ and the theorem is true for all $r \times r$ matrices where $r < n$, particularly for $r = n-1$. As the characteristic polynomial splits, there is always an available (complex) root to the polynomial as one of the eigenvalues of $A$, let's say $\lambda_1$, and a corresponding eigenvector $\vec{v}_\lambda^{(1)}$. Use the Gram-Schmidt process (with respect to the complex dot product now) to generate another set of $n-1$ vectors $\vec{w}^{(2)}, \ldots, \vec{w}^{(n)}$ to complete an orthonormal basis for $\mathbb{C}^n$. The matrix formed by arranging these orthonormal basis vectors (normalized into unit length) in columns,
\begin{align*}
Q = \begin{bmatrix}
\vec{v}_\lambda^{(1)} | \vec{w}^{(2)} | \cdots | \vec{w}^{(n)}
\end{bmatrix} =
\begin{bmatrix}
\vec{v}_\lambda^{(1)} | W
\end{bmatrix}
\end{align*}
satisfies $Q^*Q = QQ^* = I$ and is unitary, so $Q^{-1} = Q^*$. A change of coordinates on $A$ by $Q$ as in Properties \ref{proper:endomorph} is
\begin{align*}
Q^{-1}AQ = Q^* AQ &= Q^*A\begin{bmatrix}
\vec{v}_\lambda^{(1)} | W
\end{bmatrix} \\
&= Q^*\begin{bmatrix}
A\vec{v}_\lambda^{(1)} | AW
\end{bmatrix} \\
&= Q^*\begin{bmatrix}
\lambda_1\vec{v}_\lambda^{(1)} | AW
\end{bmatrix} \quad \begin{aligned}
\text{(Definition \ref{defn:eigen} for} \\
\text{an eigenvalue/eigenvector)}
\end{aligned} \\
&= 
\left[\begin{array}{c}
\vec{v}_\lambda^{(1)*} \Bstrut \\
\hline
W^* \Tstrut
\end{array}\right]
\begin{bmatrix}
\lambda_1\vec{v}_\lambda^{(1)} | AW
\end{bmatrix} \\
&= 
\left[\begin{array}{c|c}
\left[\begin{array}{c}
\vec{v}_\lambda^{(1)*} \Bstrut \\
\hline
W^* \Tstrut
\end{array}\right]\lambda_1\vec{v}_\lambda^{(1)} &
\left[\begin{array}{c}
\vec{v}_\lambda^{(1)*} \Bstrut \\
\hline
W^* \Tstrut
\end{array}\right]AW
\end{array}\right] \\
&=
\left[\begin{array}{cc}
\lambda_1 (\vec{v}_\lambda^{(1)*} \cdot \vec{v}_\lambda^{(1)}) & \vec{v}_\lambda^{(1)*}AW \\
\textbf{0} & W^*AW
\end{array}\right]
=
\left[\begin{array}{cc}
\lambda_1 & \vec{v}_\lambda^{(1)*}AW \\
\textbf{0} & W'
\end{array}\right]
\end{align*}
which is expressed in the form of a $2\times 2$ block matrix. The leftmost column takes its form because $\vec{v}_\lambda^{(1)}$ is now a unit vector and hence $\textbf{v}_\lambda^{(1)*} \textbf{v}_\lambda^{(1)} = 1$ and it is orthogonal to other vectors in $W$ by construct. Now we can use the induction hypothesis on the bottom right block $W' = W^*AW$ which is an $(n-1) \times (n-1)$ complex matrix, such that there exists a unitary matrix $R$ so that $R^*W'R$ is upper-triangular. Then, consider
\begin{align*}
U = 
Q 
\left[\begin{array}{cc}
1 & \textbf{0}^* \\
\textbf{0} & R
\end{array}\right]
\end{align*}
This matrix is unitary (Definition \ref{defn:unitary}), as
\begin{align*}
U^*U &= 
\left[\begin{array}{cc}
1 & \textbf{0}^* \\
\textbf{0} & R^*
\end{array}\right]
Q^* Q 
\left[\begin{array}{cc}
1 & \textbf{0}^* \\
\textbf{0} & R
\end{array}\right] \\
&= \left[\begin{array}{cc}
1 & \textbf{0}^* \\
\textbf{0} & R^*
\end{array}\right]
(I)
\left[\begin{array}{cc}
1 & \textbf{0}^* \\
\textbf{0} & R
\end{array}\right] & \text{(Q is unitary)} \\
&= \left[\begin{array}{cc}
1 & \textbf{0}^* \\
\textbf{0} & R^*
\end{array}\right]
\left[\begin{array}{cc}
1 & \textbf{0}^* \\
\textbf{0} & R
\end{array}\right] \\
&=
\left[\begin{array}{cc}
1 & \textbf{0}^* \\
\textbf{0} & R^*R
\end{array}\right] 
=
\left[\begin{array}{cc}
1 & \textbf{0}^* \\
\textbf{0} & I_{n-1}
\end{array}\right] 
= I & \text{(R is unitary)}
\end{align*}
Subsequently,
\begin{align*}
S = U^*AU &= \left[\begin{array}{cc}
1 & \textbf{0}^* \\
\textbf{0} & R^*
\end{array}\right]
Q^* AQ 
\left[\begin{array}{cc}
1 & \textbf{0}^* \\
\textbf{0} & R
\end{array}\right] \\
&=
\left[\begin{array}{cc}
1 & \textbf{0}^* \\
\textbf{0} & R^*
\end{array}\right]
\left[\begin{array}{cc}
\lambda_1 & \vec{v}_\lambda^{(1)*}AW \\
\textbf{0} & W'
\end{array}\right]
\left[\begin{array}{cc}
1 & \textbf{0}^* \\
\textbf{0} & R
\end{array}\right] & \text{(from above)} \\
&=
\left[\begin{array}{cc}
\lambda_1 & \vec{v}_\lambda^{(1)*}AW \\
\textbf{0} & R^*W'
\end{array}\right]
\left[\begin{array}{cc}
1 & \textbf{0}^* \\
\textbf{0} & R
\end{array}\right] \\
&=
\left[\begin{array}{cc}
\lambda_1 & \vec{v}_\lambda^{(1)*}AWR \\
\textbf{0} & R^*W'R
\end{array}\right]
\end{align*}
which is an upper-triangular matrix as desired since the induction hypothesis demands the bottom-right block $R^*W'R$ to be upper-triangular too. And therefore, the diagonal elements of $S$ are exactly its eigenvalues, which will be the same as those of $A$ since they are unitarily similar (Properties \ref{proper:similarinvariant} adapted to complex matrices). 
\end{proof}

With Schur's Triangularization Theorem, the fact that normal matrices are unitarily diagonalizable is attainable.
\begin{thm}
\label{thm:normalunidiag}
A square matrix $A$ can be unitarily diagonalized such that $U^*AU = D$ if and only if $A$ is normal.
\end{thm}
We will prove the "only if" part first. This is a bit harder than that in Theorem \ref{thm:symdiag}:
\begin{align*}
A^*A = (UDU^*)^*(UDU^*) = UD^*(U^*U) DU^* = UD^*DU^* 
\end{align*}
and
\begin{align*}
AA^* = (UDU^*)(UDU^*)^* = UD(U^*U) DU^* = UDD^*U^*  
\end{align*}
$U^*U = I$ by Definition \ref{defn:unitary} as $U$ is unitary. But also $D^*D = DD^*$ since they are diagonal and commute, thus $A^*A = AA^*$. The "if" part is also a bit more tricky. By Theorem \ref{thm:schurtrig}, we can always write $S = U^*AU$ where $S$ is upper-triangular (and $U$ is unitary), and hence $S^* = U^*A^*U$ is lower-triangular. Notice that
\begin{align*}
S^*S = U^*A^*U U^*AU &= U^*A^* AU &\text{($U$ is unitary)} \\
&= U^*AA^*U &\text{($A$ is normal)} \\
&= U^*A(UU^*)A^*U &\text{(again, $U$ is unitary)} \\
&= (U^*AU) (U^*A^*U) = SS^*
\end{align*}
With
\begin{align*}
S &= 
\begin{bmatrix}
s_{11} & s_{12} & s_{13} & \cdots & s_{1n} \\
0 & s_{22} & s_{23} & \cdots & s_{2n} \\
0 & 0 & s_{33} & \cdots & s_{3n} \\
\vdots & \vdots & \vdots & \ddots & \vdots \\
0 & 0 & 0 & \cdots & s_{nn}
\end{bmatrix}
& \text{ and } 
S^* &= 
\begin{bmatrix}
\overline{s_{11}} & 0 & 0 & \cdots & 0 \\
\overline{s_{12}} & \overline{s_{22}} & 0 & \cdots & 0 \\
\overline{s_{13}} & \overline{s_{23}} & \overline{s_{33}} & \cdots & 0 \\
\vdots & \vdots & \vdots & \ddots & \vdots \\
\overline{s_{1n}} & \overline{s_{2n}} & \overline{s_{3n}} & \cdots & \overline{s_{nn}}
\end{bmatrix}
\end{align*}
by considering
\begin{align*}
(S^*S)_{11} &= (SS^*)_{11} \\
\overline{s_{11}}s_{11} &= s_{11}\overline{s_{11}} + s_{12}\overline{s_{12}} + \cdots + s_{1n}\overline{s_{1n}} \\
0 &= \abs{s_{12}}^2 + \cdots + \abs{s_{1n}}^2 \geq 0
\end{align*}
we must have $s_{12} = \cdots = s_{1n} = 0$. Similarly, consider $(S^*S)_{22} = (SS^*)_{22}, \ldots, \\ (S^*S)_{nn} = (SS^*)_{nn}$ in sequential order, we conclude that $s_{ij} = 0$ when $i < j$, all off-diagonal elements are zero, and thus $S$ is diagonal. Hence $A$ is actually unitarily diagonalized into $D = S = U^*AU$ by the matrix $U$.

\begin{exmp}
Unitarily diagonalize the following matrix.
\begin{align*}
A =
\begin{bmatrix}
1 & 1+i \\
1-i & 2
\end{bmatrix}
\end{align*}
\end{exmp}
\begin{solution}
The matrix $A$ can be seen to be Hermitian, and therefore unitary diagonalization is possible by Theorem \ref{thm:normalunidiag}. The characteristic equation is 
\begin{align*}
(1-\lambda)(2-\lambda) - (1+i)(1-i) &= 0 \\
(2 - 3\lambda + \lambda^2) - 2 &= 0 \\
\lambda^2 - 3\lambda &= 0 \\
\lambda &= 0 \text{ or } 3
\end{align*}
The eigenvalues being real for a Hermitian matrix is not a mere coincidence and will be proved afterward. In fact, it is an extension of the fact that the eigenvalues of any symmetric matrix are all real. The eigenvectors can be found to be $(-1-i, 1)^T$ for $\lambda = 0$ and $(1+i, 2)^T$ for $\lambda = 3$. After normalization, they are $\frac{1}{\sqrt{3}}(-1-i, 1)^T$ and $\frac{1}{\sqrt{6}}(1+i, 2)^T$ respectively. Now define
\begin{align*}
U =
\begin{bmatrix}
\frac{-1-i}{\sqrt{3}} & \frac{1+i}{\sqrt{6}} \\
\frac{1}{\sqrt{3}} & \frac{\sqrt{2}}{\sqrt{3}}
\end{bmatrix}
\end{align*}
Then the unitary diagonalization has the form of
\begin{align*}
U^* AU &= D \\
\begin{bmatrix}
\frac{-1+i}{\sqrt{3}} & \frac{1}{\sqrt{3}} \\
\frac{1-i}{\sqrt{6}} & \frac{\sqrt{2}}{\sqrt{3}}
\end{bmatrix}
\begin{bmatrix}
1 & 1+i \\
1-i & 2
\end{bmatrix}
\begin{bmatrix}
\frac{-1-i}{\sqrt{3}} & \frac{1+i}{\sqrt{6}} \\
\frac{1}{\sqrt{3}} & \frac{\sqrt{2}}{\sqrt{3}}
\end{bmatrix}
&=
\begin{bmatrix}
0 & 0 \\
0 & 3
\end{bmatrix}
\end{align*}
\end{solution}
\begin{proper}
\label{proper:hermrealeig}
The eigenvalues of any Hermitian matrix $A = A^*$ must be real.
\end{proper}
\begin{proof}
Consider $\vec{v}_\lambda \cdot (A\vec{v}_\lambda)$ where $\vec{v}_\lambda$ is a complex eigenvector of $A$ and the complex dot product is used in place. Then
\begin{align*}
\vec{v}_\lambda \cdot (A\vec{v}_\lambda) &= \vec{v}_\lambda \cdot (\lambda \vec{v}_\lambda) & \text{(Definition \ref{defn:eigen})} \\
&= \overline{\lambda} (\vec{v}_\lambda \cdot \vec{v}_\lambda) & \text{(Properties \ref{proper:complexdot})} \\
&= \overline{\lambda} \norm{\vec{v}_\lambda}^2
\end{align*}
But also
\begin{align*}
\vec{v}_\lambda \cdot (A\vec{v}_\lambda) &= (A^*\vec{v}_\lambda) \cdot \vec{v}_\lambda & \text{(Properties \ref{proper:complexdotherm})} \\
&= (A\vec{v}_\lambda) \cdot \vec{v}_\lambda & \text{($A$ is Hermitian, Definition \ref{defn:Hermitian})} \\
&= (\lambda\vec{v}_\lambda) \cdot \vec{v}_\lambda & \text{(Definition \ref{defn:eigen})} \\
&= \lambda (\vec{v}_\lambda \cdot \vec{v}_\lambda) & \text{(Properties \ref{proper:complexdot})} \\
&= \lambda \norm{\vec{v}_\lambda}^2
\end{align*}
Therefore, $\overline{\lambda}\norm{\vec{v}_\lambda}^2 = \lambda \norm{\vec{v}_\lambda}^2$, and since $\vec{v}_\lambda \neq \textbf{0}$, $\norm{\vec{v}_\lambda} \neq 0$, $\overline{\lambda} = \lambda$ and the eigenvalue $\lambda$ has to be real.
\end{proof}

\begin{exmp}
\label{exmp:unitarydiagnormal}
Check
\begin{align*}
A = 
\begin{bmatrix}
\frac{5}{3}-\frac{1}{3}i &\frac{1}{3}+i \\ 
-1+\frac{1}{3}i & \frac{4}{3}+\frac{1}{3}i
\end{bmatrix}
\end{align*}
is normal and carry out unitary diagonalization on it.
\end{exmp}
\begin{solution}
First,
\begin{align*}
A^* A=
\begin{bmatrix}
\frac{5}{3}+\frac{1}{3}i & -1-\frac{1}{3}i\\ 
\frac{1}{3}-i & \frac{4}{3}-\frac{1}{3}i
\end{bmatrix}
\begin{bmatrix}
\frac{5}{3}-\frac{1}{3}i &\frac{1}{3}+i \\ 
-1+\frac{1}{3}i & \frac{4}{3}+\frac{1}{3}i
\end{bmatrix} =
\begin{bmatrix}
4 & -1+i \\
-1-i & 3
\end{bmatrix}
\end{align*}
and
\begin{align*}
A A^* =
\begin{bmatrix}
\frac{5}{3}-\frac{1}{3}i & \frac{1}{3}+i \\ 
-1+\frac{1}{3}i & \frac{4}{3}+\frac{1}{3}i
\end{bmatrix}
\begin{bmatrix}
\frac{5}{3}+\frac{1}{3}i & -1-\frac{1}{3}i \\ 
\frac{1}{3}-i & \frac{4}{3}-\frac{1}{3}i
\end{bmatrix} =
\begin{bmatrix}
4 & -1+i \\
-1-i & 3
\end{bmatrix} = A^* A
\end{align*}
So $A$ is indeed normal. Now the characteristic equation is
\begin{align*}
(\frac{5}{3}-\frac{1}{3}i-\lambda)(\frac{4}{3}+\frac{1}{3}i-\lambda) - (-1+\frac{1}{3}i)(\frac{1}{3}+i) &= 0 \\
\lambda^2 - 3\lambda + (3+i) &= 0 
\end{align*}
which can be checked to have the complex roots of $\lambda_1 = 1+i$ and $\lambda_2 = 2-i$ as the two eigenvalues. The corresponding orthonormal eigenvectors are $\vec{v}_{\lambda_1} = (\frac{1}{\sqrt{3}}, \frac{1+i}{\sqrt{3}})^T$ and $\vec{v}_{\lambda_2} = (\frac{-1+i}{\sqrt{3}}, \frac{1}{\sqrt{3}})^T$, and thus the unitary diagonalization reads
\begin{align*}
D &= U^*AU \\
&= 
\begin{bmatrix}
\frac{1}{\sqrt{3}} & \frac{1-i}{\sqrt{3}} \\ 
\frac{-1-i}{\sqrt{3}}  & \frac{1}{\sqrt{3}}
\end{bmatrix}
\begin{bmatrix}
\frac{5}{3}-\frac{1}{3}i &\frac{1}{3}+i \\ 
-1+\frac{1}{3}i & \frac{4}{3}+\frac{1}{3}i
\end{bmatrix}
\begin{bmatrix}
\frac{1}{\sqrt{3}} & \frac{-1+i}{\sqrt{3}} \\ 
\frac{1+i}{\sqrt{3}} & \frac{1}{\sqrt{3}}
\end{bmatrix} \\
&= 
\begin{bmatrix}
1+i & 0 \\
0 & 2-i
\end{bmatrix}
\end{align*}
where
\begin{align*}
U = \begin{bmatrix}
\frac{1}{\sqrt{3}} & \frac{-1+i}{\sqrt{3}} \\ 
\frac{1+i}{\sqrt{3}} & \frac{1}{\sqrt{3}}
\end{bmatrix} 
\end{align*}
\end{solution}

\section{Python Programming}

Orthogonal diagonalization is also performed by the \verb|diagonalize| method in \texttt{sympy} as long as the matrix is symmetric. Let's try this with Example \ref{exmp:orthodiag}:
\begin{lstlisting}
import numpy as np
import sympy

A = np.array([[1., 0., 0.],
              [0., 2., 1.],
              [0., 1., 2.]])
A_sympy = sympy.Matrix(A)
print(A_sympy.is_symmetric())
P, D = A_sympy.diagonalize()
print(P, D)    
\end{lstlisting}
which returns \verb|true| for the symmetricity checking and
\begin{lstlisting}
Matrix([[1.0000,  0,       0], 
        [0,  0.7071, -0.7071], 
        [0, -0.7071, -0.7071]])
Matrix([[1.0000, 0, 0], 
        [0, 1.0000, 0], 
        [0, 0, 3.0000]])    
\end{lstlisting}
as expected. We can confirm the $P$ matrix is orthogonal by
\begin{lstlisting}
print(np.allclose(np.array(P.T @ P, dtype=float), np.identity(3))) # True
\end{lstlisting}
which is slightly complicated since there will be numerical errors when computing $P$ and $P^T P$ so we need to first convert the matrix product to \verb|np.array| and use \verb|np.allclose| which does not check the exact but close equality against $I$ over all entries. The same idea applies to unitary diagonalization and let's use Example \ref{exmp:unitarydiagnormal} to demonstrate. First, we check if the matrix is normal by 
\begin{lstlisting}
A = np.array([[5/3 - (1/3)*1j, 1/3 + 1j],
              [-1 + (1/3)*1j, 4/3 + (1/3)*1j]])

print(np.allclose(np.conjugate(A).T @ A, A @ np.conjugate(A).T))
\end{lstlisting}
which outputs \verb|true|. Next,
\begin{lstlisting}
A_sympy = sympy.Matrix(A)

U, D = A_sympy.diagonalize()
print(U, D)
print(np.allclose(np.array(U.H @ U, dtype=complex), np.identity(2)))   
\end{lstlisting}
gives the matrices $U$ and $D$:
\begin{lstlisting}
Matrix([[-0.4898 - 0.6531*I, 0.0816 + 0.571*I], 
        [-0.0816 + 0.5715*I, -0.4898 + 0.6531*I]]) 
Matrix([[2.0 - 1.0*I, 0], 
        [0, 1.0 + 1.0*I]])  
\end{lstlisting}
and returns \verb|true| as we check if $U$ is unitary. The form of $U$ may seem to be different from Example \ref{exmp:unitarydiagnormal} but they are actually equivalent, only differed by a complex multiplicative factor. To see this, we can do
\begin{lstlisting}
U_np = np.array(U, dtype=complex)
U_np[:,0] = U_np[:,0] * np.conjugate(U_np[1,0])
U_np[:,1] = U_np[:,1] * np.conjugate(U_np[0,1])
print(U_np)    
\end{lstlisting}
which produces
\begin{lstlisting}
[[-0.33333333+3.33333333e-01j  0.33333333-2.58092536e-18j]
 [ 0.33333333-3.97842146e-19j  0.33333333+3.33333333e-01j]]    
\end{lstlisting}
which is essentially the same as in Example \ref{exmp:unitarydiagnormal} but with the two columns of eigenvectors interchanged plus some very small round-off errors.

\section{Exercises}

\begin{Exercise}
Determine if the following matrices are orthogonal. If so, also determine if they represent a rotation or reflection.
\begin{enumerate}[label=(\alph*)]
\item $\renewcommand\arraystretch{1.3}\left[\begin{array}{wc{15pt}wc{15pt}wc{15pt}}
\frac{1}{\sqrt{2}} & 0 & \frac{1}{\sqrt{2}}\\
\frac{3}{2\sqrt{6}} & \frac{1}{2} & -\frac{3}{2\sqrt{6}}\\
-\frac{1}{2\sqrt{2}} & \frac{\sqrt{3}}{2} & \frac{1}{2\sqrt{2}}
\end{array}\right]$
\item $\renewcommand\arraystretch{1.3}\left[\begin{array}{wc{15pt}wc{15pt}wc{15pt}}
-\frac{\sqrt{3}}{2} & \frac{1}{2} & 0\\
\frac{1}{4} & \frac{\sqrt{3}}{4} & -\frac{\sqrt{3}}{2}\\
\frac{\sqrt{3}}{4} & \frac{3}{4} & \frac{1}{2}
\end{array}\right]$
\item $\renewcommand\arraystretch{1.3}\left[\begin{array}{wc{15pt}wc{15pt}wc{15pt}}
\frac{1}{2}&-\frac{1}{2}&\frac{1}{4}\\
1&-1&-\frac{1}{2}\\ 
1&1&2 \end{array}\right]$
\item $\renewcommand\arraystretch{1.3}\left[\begin{array}{wc{15pt}wc{15pt}wc{15pt}}
\frac{1}{\sqrt{2}} & -\frac{1}{\sqrt{3}} & \frac{1}{\sqrt{3}}\\
\frac{1}{\sqrt{2}} & \frac{1}{\sqrt{3}} & -\frac{1}{\sqrt{3}}\\
0 & \frac{1}{\sqrt{3}} & 0
\end{array}\right]$
\end{enumerate}
\end{Exercise}

\begin{Exercise}
Represent the following operations on a three-dimensional $xyz$ coordinate system by a transitional matrix $P$. State clearly the relationship between the vectors in the old and new coordinate systems ($\vec{v}_0$ and $\vec{v_n}$), as well as the matrix $P$.
\begin{enumerate}[label=(\alph*)]
\item Rotation about the z-axis (x-axis and y-axis revolving around the z-axis) counter-clockwise by 45 degrees,
\item Reflection of the x-axis across the y-z plane and subsequently rotation about the intermediate y-axis counter-clockwise by 30 degrees,
\item Rotation about the y-axis clockwise by 45 degrees, and then rotation about the new z-axis counter-clockwise by 60 degrees.
\end{enumerate}
It is emphasized that the order of operations is important.
\end{Exercise}

\begin{Exercise}
Argue that
\begin{align*}
P = 
\begin{bmatrix}
\cos{\theta} & \sin{\theta} \\
\sin{\theta} & -\cos{\theta}
\end{bmatrix}
\end{align*}
is a transition matrix that represents a reflection about the infinitely long straight line with an angle of $\frac{\theta}{2}$ passing through the origin on the $xy$ plane.
\end{Exercise}

\begin{Exercise}
For the symmetric matrix 
\begin{align*}
\begin{bmatrix}
1 & a & a\\
a & 5 & 3\\
a & 3 & 5
\end{bmatrix}
\end{align*}
It is given that one of the eigenvalues is zero and the product of the other two eigenvalues is $18$. Using the knowledge that the trace and characteristic polynomial are invariants, i.e. remain unchanged after diagonalization. Find 
\begin{enumerate}[label=(\alph*)]
\item The two remaining eigenvalues,
\item The possible values of $a$,
\item For every possible case, find the eigenvector corresponding to the eigenvalue of zero, then carry out orthogonal diagonalization.
\end{enumerate}
\end{Exercise}

\begin{Exercise}
Orthogonally diagonalize the following matrix, if possible.
\begin{enumerate}[label=(\alph*)]
\item $\left[\begin{array}{wc{10pt}wc{10pt}wc{10pt}}
7 & 8 & 10\\
1 & 6 & 3\\
5 & 1 & 2
\end{array}\right]$
\item $\left[\begin{array}{wc{10pt}wc{10pt}wc{10pt}}
2 & 0 & 1\\
0 & 2 & 1\\
1 & 1 & 1
\end{array}\right]$
\item $\left[\begin{array}{wc{10pt}wc{10pt}wc{10pt}}
3 & 1 & 1\\
1 & 3 & 1\\
1 & 1 & 3
\end{array}\right]$
\end{enumerate}
\end{Exercise}

\begin{Exercise}
Find the eigenvalues and corresponding eigenspaces for the matrix $A = I_n - \textbf{u}\textbf{u}^T$ where $\vec{u} \in \mathbb{R}^n$ is a unit vector such that $\norm{\textbf{u}} = 1$. Hint: use the Spectral Theorem and note that $\vec{u}$ alone is linearly independent and can be completed to an orthonormal basis.
\end{Exercise}

\begin{Exercise}
Find the orthogonal projection operator in $\mathbb{R}^4$ on the subspace intersected by the two three-dimensional hyperplanes $x + 2y - 3z + 4w = 0$ and $3x - y + z - 2w = 0$.
\end{Exercise}

\begin{Exercise}
Check the Spectral Theorem (Theorem \ref{thm:spectral}) for the symmetric matrix below:
\begin{align*}
A = 
\renewcommand\arraystretch{1.2}
\begin{bmatrix}
\frac{5}{3}&0&-\frac{\sqrt{2}}{3}\\ 
0&2&0\\ 
-\frac{\sqrt{2}}{3}&0&\frac{4}{3}
\end{bmatrix}
\end{align*}
\end{Exercise}

\begin{Exercise}
Show that the matrix below is Hermitian and unitarily diagonalize it.
\begin{align*}
A &=
\begin{bmatrix}
1 & -i & 0 \\
i & 2 & 1+i \\
0 & 1-i & 3 
\end{bmatrix}
\end{align*}
\end{Exercise}

\begin{Exercise}
Hooke's law states that the force acting on a mass by spring is given by $F = -kx$ where $k$ is the spring constant and $x$ is the displacement (extension or compression) from the equilibrium position. By considering Newton's second law, $F = ma$, we have
\begin{equation*}
ma = m\frac{d^2x}{dt^2} = -kx\text{ and hence }x'' = \frac{d^2x}{dt^2} = -\frac{k}{m}x
\end{equation*}
The general solution is
\begin{equation*}
x = C\cos(\omega t - \theta)
\end{equation*}
where $\omega = \sqrt{\frac{k}{m}}$ and $C$, $\theta$ are some arbitrary constants to be determined from the initial condition. Consider the situation shown in the figure below. Find the two equations describing motions of the two masses $m_1, m_2$ respectively, in terms of the displacements $\textbf{x} = (x_1, x_2)^T$, and re-written them into matrix form. Carry out orthogonal diagonalization to simplify the equations and find the general solutions for the motions.
\begin{center}
\begin{tikzpicture}
[wall/.style = {gray,fill=gray},
mass/.style = {draw,circle,ball color=blue},
spring/.style = {decorate,decoration={zigzag, pre length=.3cm,post length=.3cm,segment length=#1}},]
\draw[wall] (-.5,-1) rectangle (0,1);
\coordinate (l) at (0,0);
\node[mass,label={above:$m_1$}] (m1) at (2,0) {};
\node[mass,label={above:$m_2$}] (m2) at (4,0) {};
\coordinate (r) at (6,0);
\draw[wall] (6,-1) rectangle (6.5,1);

\draw[spring=4pt] (l) -- node[above] {$k_s$} (m1);
\draw[spring=4pt] (m1) -- node[above] {$k_s$} (m2);
\draw[spring=4pt] (m2) -- node[above] {$k_s$} (r);
\end{tikzpicture}  
\end{center}
\end{Exercise}
\chapter{Quadratic Forms}
\label{chap:quad}

Symmetric matrices often see their appearance in the areas of Geometry, Statistics, Physics, and more, as \textit{quadratic forms}. They can be used to describe a family of geometric shapes known as \textit{conic sections}, including ellipses and hyperbola. A more common application of quadratic forms in Earth Science will be obtaining a \textit{covariance matrix} between different variables, eventually leading to \textit{Principal Component Analysis (PCA)}, which breaks down the variables into uncorrelated modes that explain the spread of their distribution. In Atmospheric Science, it is more commonly known as \textit{Empirical Orthogonal Functions (EOF)} and has been widely used to analyze prominent, large-scale climate patterns like \textit{El Niño-Southern Oscillation (ENSO)}. 

\section{Mathematical and Geometric Ideas of Quadratic Forms}
\subsection{Definition of Quadratic Forms}

The word \textit{quadratic} is commonly associated with \textit{quadratic equations} in the form of $y = ax^2 + bx + c$. \index{Quadratic Form}\keywordhl{Quadratic forms} are the generalization of quadratic equations when there are multiple variables $x_1, x_2, x_3, \ldots$: The possible quadratic forms will be made up of the usual quadratic terms $x_1^2, x_2^2, x_3^2, \ldots$, as well as the \textit{cross-product terms} (again, not to be confused with the vector cross product in Section \ref{section:crossprod}) $x_px_q$, $p \neq q$. The usual quadratic terms can then be seen as just another kind of cross-product terms when $p = q$. We will limit our discussion to real, finite-dimensional vector spaces first.
\begin{defn}[Quadratic Form]
\label{defn:quadform}
Real quadratic forms in multiple variables $\vec{x} = (x_1, x_2, x_3, \ldots, x_n)^T$ has a structure of
\begin{align}
Q(\vec{x}) = \sum_{p=1}^{n}\sum_{q=1}^{n} b_{pq} x_px_q \label{eqn:quadform}
\end{align}
Note that $b_{pq}$ are real and they produce a real scalar.
\end{defn}
For example, in a two-variable situation, $x^2 + 3xy + y^2$ and $3x^2 - 4xy$ are quadratic forms, while $x^2 + y$ and $xy + xy^2$ are not. Notice that $x_px_q$ and $x_qx_p$ are actually the same term, and we will replace $b_{pq}$ or $b_{qp}$ by $\frac{1}{2}(b_{pq} + b_{qp})$ as a single coefficient for both of them. 
\begin{proper}[Matrix Representation of a Quadratic Form]
\label{proper:quadformmatrix}
All quadratic forms as given by (\ref{eqn:quadform}) in Definition \ref{defn:quadform} for any real vector $\vec{x} = (x_1, x_2, x_3, \ldots, x_n)^T \in \mathcal{V}$ can be expressed as
\begin{align}
Q: \mathcal{V} \to \mathbb{R},\, Q(\vec{x}) = \textbf{x}^TB\textbf{x}
\end{align}
where $B$ is real symmetric and has the form of
\begin{align}
\left[\begin{array}{@{}ccccc@{}}
b_{11} & \frac{1}{2}(b_{12} + b_{21}) & \frac{1}{2}(b_{13} + b_{31}) & \cdots & \frac{1}{2}(b_{1n} + b_{n1}) \\
\frac{1}{2}(b_{12} + b_{21}) & b_{22} & \frac{1}{2}(b_{23} + b_{32}) & & \frac{1}{2}(b_{2n} + b_{n2}) \\
\frac{1}{2}(b_{13} + b_{31}) & \frac{1}{2}(b_{23} + b_{32}) & b_{33} & \cdots & \frac{1}{2}(b_{3n} + b_{n3})\\
\vdots &  & \vdots & \ddots & \vdots \\
\frac{1}{2}(b_{1n} + b_{n1}) & \frac{1}{2}(b_{2n} + b_{n2}) & \frac{1}{2}(b_{3n} + b_{n3}) & \cdots & b_{nn}
\end{array}\right]
\end{align}
\end{proper}
The readers can check that the above matrix expression indeed leads to the desired quadratic form by a direct expansion.\footnote{\scriptsize\begin{align*}
&\quad \begin{bmatrix}
x_1 & x_2 & x_3 & \cdots
\end{bmatrix}
\left[\begin{array}{@{}cccc@{}}
b_{11} & \frac{1}{2}(b_{12} + b_{21}) & \frac{1}{2}(b_{13} + b_{31}) & \cdots \\
\frac{1}{2}(b_{12} + b_{21}) & b_{22} & \frac{1}{2}(b_{23} + b_{32}) &  \\
\frac{1}{2}(b_{13} + b_{31}) & \frac{1}{2}(b_{23} + b_{32}) & b_{33} &  \\
\vdots & & & \ddots
\end{array}\right]
\begin{bmatrix}
x_1 \\
x_2 \\
x_3 \\
\vdots
\end{bmatrix} \\
&=
\begin{bmatrix}
x_1 & x_2 & x_3 & \cdots
\end{bmatrix}
\left[\begin{array}{@{}cccc@{}}
b_{11}x_1 + \frac{1}{2}(b_{12} + b_{21})x_2 + \frac{1}{2}(b_{13} + b_{31})x_3 + \cdots \\
\frac{1}{2}(b_{12} + b_{21})x_1 + b_{22}x_2 + \frac{1}{2}(b_{23} + b_{32})x_3 + \cdots \\
\frac{1}{2}(b{13} + b_{31})x_1 + \frac{1}{2}(b_{23} + b_{32})x_2 + b_{33}x_3 + \cdots \\
\vdots 
\end{array}\right] \\
&= x_1\left(b_{11}x_1 + \frac{1}{2}(b_{12} + b_{21})x_2 + \frac{1}{2}(b_{13} + b_{31})x_3 + \cdots\right) \quad = b_{11}x_1^2 + b_{12}x_1x_2 + b_{13}x_1x_3 + \cdots \\
&\quad + x_2\left(\frac{1}{2}(b_{12} + b_{21})x_1 + b_{22}x_2 + \frac{1}{2}(b_{23} + b_{32})x_3 + \cdots\right) \quad + b_{21}x_2x_1 + b_{22}x_2^2 + b_{23}x_2x_3 + \cdots \\
&\quad + x_3\left(\frac{1}{2}(b_{13} + b_{31})x_1 + \frac{1}{2}(b_{23} + b_{32})x_2 + b_{33}x_3 + \cdots\right) \quad + b_{31}x_3x_1 + b_{32}x_3x_2 + b_{33}x_3^2 + \cdots
\end{align*}} In the same essence, we can always express any quadratic form using the symmetric part of a matrix. For instance, the quadratic form $x^2 - 2xy + 3y^2$ can be rewritten as
\begin{align*}
\begin{bmatrix}
x & y
\end{bmatrix}
\begin{bmatrix}
1 & -1 \\
-1 & 3
\end{bmatrix}
\begin{bmatrix}
x \\
y
\end{bmatrix}
\end{align*}
$\blacktriangleright$ Short Exercise: Verify the quadratic form by expanding it.\footnotemark \par
Even if we are given $\textbf{x}^TA\textbf{x}$ where $A$ is not symmetric to begin with, we can extract the symmetric part of $A$ (see Exercise \ref{ex:symskew}), that is, $B = \frac{1}{2}(A + A^T)$. Reconstruction of the quadratic form by $\textbf{x}^TB\textbf{x}$ will still be equivalent to the original $\textbf{x}^TA\textbf{x}$:
\begin{align*}
\textbf{x}^TB\textbf{x} &= \frac{1}{2}\textbf{x}^T(A + A^T)\textbf{x} \\
&= \frac{1}{2}\textbf{x}^TA\textbf{x} + \frac{1}{2}\textbf{x}^TA^T\textbf{x} \\
&= \frac{1}{2}\textbf{x}^TA\textbf{x} + \frac{1}{2}(\textbf{x}^TA^T\textbf{x})^T & \begin{aligned}
&\text{($\textbf{x}^TA^T\textbf{x} = (\textbf{x}^TA^T\textbf{x})^T$ since it is just} \\
&\text{a scalar in a $1 \times 1$ singleton block)}  
\end{aligned} \\
&= \frac{1}{2}\textbf{x}^TA\textbf{x} + \frac{1}{2}\textbf{x}^TA\textbf{x} & \text{(Properties \ref{proper:transp})} \\
&= \textbf{x}^TA\textbf{x}
\end{align*}
\footnotetext{{\begin{align*}
\begin{bmatrix}
x & y
\end{bmatrix}
\begin{bmatrix}
1 & -1 \\
-1 & 3
\end{bmatrix}
\begin{bmatrix}
x \\
y
\end{bmatrix} 
&= 
\begin{bmatrix}
x & y
\end{bmatrix}
\begin{bmatrix}
x - y \\
-x + 3y
\end{bmatrix} \\
&= x(x-y) + y(-x+3y) \\
&= x^2 - xy - xy + 3y^2 = x^2 - 2xy + 3y^2
\end{align*}}}Similarly the skew-symmetric part does not contribute anything to the quadratic form\footnote{$\frac{1}{2}\textbf{x}^T(A - A^T)\textbf{x} = \frac{1}{2}\textbf{x}^TA\textbf{x} - \frac{1}{2}\textbf{x}^TA^T\textbf{x} = \frac{1}{2}\textbf{x}^TA\textbf{x} - \frac{1}{2}(\textbf{x}^TA^T\textbf{x})^T = \frac{1}{2}\textbf{x}^TA\textbf{x} - \frac{1}{2}\textbf{x}^TA\textbf{x} = 0$}, and we will characterize all quadratic forms using symmetric matrices henceforth.
Such a symmetric matrix $B$ is also frequently considered to behave as a \index{Bilinear Form}\index{Symmetric Bilinear Form}\keywordhl{symmetric bilinear form} in a more general setting.
\begin{defn}[(Symmetric) Bilinear Form]
A real bilinear form $B(\vec{x}, \vec{y}): \mathcal{V} \times \mathcal{V} \to \mathbb{R}$ over a real vector space $\mathcal{V}$ takes two vectors $\vec{x}$, $\vec{y} \in \mathcal{V}$ from it as inputs and returns a real scalar, where \textit{bilinearity} means it is linear in both components:
\begin{subequations}
\begin{align}
B(a\vec{x}^{(1)}+b\vec{x}^{(2)},\vec{y}) &= B(a\vec{x}^{(1)},\vec{y}) + B(b\vec{x}^{(2)},\vec{y}) = aB(\vec{x}^{(1)},\vec{y}) + bB(\vec{x}^{(2)},\vec{y}) \\
B(x,a\vec{y}^{(1)}+b\vec{y}^{(2)}) &= B(x,a\vec{y}^{(1)}) + B(x,b\vec{y}^{(2)}) = aB(x,\vec{y}^{(1)}) + bB(x,\vec{y}^{(2)}) 
\end{align}
\end{subequations}
A symmetric bilinear form is one that satisfies 
\begin{align}
B(\vec{x}, \vec{y}) = B(\vec{y}, \vec{x})    
\end{align}
for any pair of $\vec{x}$ and $\vec{y}$. For finite-dimensional cases, the expression becomes
\begin{align}
B(\vec{x}, \vec{y}) = \textbf{x}^T B \textbf{y}    
\end{align}
where $B^T = B$ is an $n \times n$ real symmetric matrix and $n$ is the dimension of $\mathcal{V}$.
\end{defn}
It is not hard to see that the form $\textbf{x}^TB\textbf{y}$ fulfills the requirement of $B(\vec{x}, \vec{y}) = B(\vec{y}, \vec{x})$.\footnote{$B(\vec{x}, \vec{y}) = \textbf{x}^TB\textbf{y} = (\textbf{x}^TB\textbf{y})^T = \textbf{y}^TB^T\textbf{x} = \textbf{y}^TB\textbf{x} = B(\vec{y}, \vec{x})$, where $\textbf{x}^TB\textbf{y} = (\textbf{x}^TB\textbf{y})^T$ as it is just a single real number.} It is also not hard to see that the pattern of quadratic forms proposed in Properties \ref{proper:quadformmatrix} can be derived from a symmetric bilinear form as we set $\vec{y} = \vec{x}$. In this sense, we say that a quadratic form is characterized by the symmetric bilinear form induced by the appropriate symmetric matrix $B$.

\subsection{(Semi)Definiteness and Congruence}
\label{subsection:definiteness}

An important attribute of quadratic forms is their \index{Definiteness}\index{Semidefiniteness}\keywordhl{(semi)definiteness}. If a quadratic form $Q(\vec{x})$ is \index{Positive-Definite}\index{Negative-Definite}\keywordhl{positive/negative-definite}, it means that it always outputs positive/negative numbers no matter what the input vector $\vec{x}$ is, as long as $\vec{x} \neq \textbf{0}$ is a non-zero vector. Semidefiniteness relaxes the restriction such that the quadratic form can also return zero for some non-zero $\vec{x}$, in other words, a positive[negative]-semidefinite quadratic form always gives non-negative [non-positive] numbers. Now we will show that definiteness is related to the eigenvalues of the symmetric matrix that characterizes the quadratic form.
\begin{defn}
\label{defn:quaddefinite}
For any real quadratic form $Q(x) = \textbf{x}^T B\textbf{x}$, $Q$ (or $B$) is called
\begin{enumerate}[label=(\alph*)]
\item positive-definite, if for any $\vec{x} \neq \vec{0}$, $\textbf{x}^T B\textbf{x} > 0$ (positive-semidefinite if $\textbf{x}^T B\textbf{x} \geq 0$);
\item negative-definite, if for any $\vec{x} \neq \vec{0}$, $\textbf{x}^T B\textbf{x} < 0$ (negative-semidefinite if $\textbf{x}^T B\textbf{x} \leq 0$); 
\item indefinite if $\textbf{x}^T B\textbf{x}$ can take both positive and negative values.
\end{enumerate}
\end{defn}
\begin{thm}
\label{thm:quaddefinite}
The quadratic form $Q(x) = \textbf{x}^T B\textbf{x}$ where $B$ is real symmetric, is
\begin{enumerate}[label=(\alph*)]
\item positive-definite, if and only if all eigenvalues of $B$ are positive (positive-semidefinite if and only if all eigenvalues of $B$ are non-negative);
\item negative-definite, if and only if all eigenvalues of $B$ are negative (negative-semidefinite if and only if all eigenvalues of $B$ are non-positive); 
\item indefinite when there are both positive and negative eigenvalues for $B$.
\end{enumerate}
\end{thm}
\begin{proof}
We will only show the positive-definite part, but the other cases essentially follow the same logic. Since we are given a symmetric matrix, the Spectral Theorem \ref{thm:spectral} naturally comes off as handy. Assume we are working in an $n$-dimensional real vector space $\mathcal{V}$. Part (d) of the Spectral Theorem shows that any vector $\vec{x} \in \mathcal{V}$ can be rewritten into $\vec{x} = \vec{x}_{J_1} + \vec{x}_{J_2} + \cdots + \vec{x}_{J_k}$ where each of $\vec{x}_{J_i} \in \mathcal{E}_{J_i}$ belongs to the respective eigenspace of $B$. Then as in the derivation for part (e) of the Spectral Theorem, we have
\begin{align*}
B\textbf{x} &= \lambda_{J_1}\vec{x}_{J_1} + \lambda_{J_2}\vec{x}_{J_2} + \cdots + \lambda_{J_k}\vec{x}_{J_k}
\end{align*}
Subsequently,
\begin{align*}
Q(x) &= \textbf{x}^T B\textbf{x} \\
&= (\vec{x}_{J_1} + \vec{x}_{J_2} + \cdots + \vec{x}_{J_k}) \cdot (\lambda_{J_1}\vec{x}_{J_1} + \lambda_{J_2}\vec{x}_{J_2} + \cdots + \lambda_{J_k}\vec{x}_{J_k}) \\
&= \lambda_{J_1}(\vec{x}_{J_1} \cdot \vec{x}_{J_1}) + \lambda_{J_2}(\vec{x}_{J_2} \cdot \vec{x}_{J_2}) + \cdots + \lambda_{J_k}(\vec{x}_{J_k} \cdot \vec{x}_{J_k}) \\
&= \lambda_{J_1}\norm{\vec{x}_{J_1}}^2 + \lambda_{J_2}\norm{\vec{x}_{J_2}}^2 + \cdots + \lambda_{J_k}\norm{\vec{x}_{J_k}}^2
\end{align*}
where $\smash{\vec{x}_{J_i} \cdot \vec{x}_{J_i'}} = 0$ whenever $i \neq i'$ (Properties \ref{proper:symortho}) so we have the second to third line. Now, since for any $\vec{x}$, all the square length quantities $\smash{\norm{\vec{x}_{J_i}}^2} \geq 0$ are greater than or equal to $0$, if all $\lambda_{J_i} > 0$ are positive and $\vec{x}$ is not a zero vector such that at least one of the $\smash{\norm{\vec{x}_{J_i}}^2} > 0$ is positive, then the quadratic form $Q(x) > 0$ will always take a positive value. The converse can be established by considering the contrapositive and following the same line of reasoning.
\end{proof}

\begin{exmp}
Show that the symmetric matrix
\begin{align*}
B = 
\begin{bmatrix}
4 & 1 \\
1 & 4
\end{bmatrix}
\end{align*}
is positive-definite.
\end{exmp}
\begin{solution}
By Theorem \ref{thm:quaddefinite}, we simply check whether all its eigenvalues are positive. Its characteristic polynomial
\begin{align*}
\det(B-\lambda I) =
\begin{vmatrix}
4 - \lambda & 1 \\
1 & 4 - \lambda
\end{vmatrix}
&= (4-\lambda)(4-\lambda) - (1)(1) \\
&= (16 - 8\lambda + \lambda^2) - 1 \\
&= 15 - 8\lambda + \lambda^2 = (3-\lambda)(5-\lambda)
\end{align*}
has $\lambda = 3,5$ as its roots which are both positive. Hence $B$ is positive-definite. We can double-check by the explicit method of completing the square:
\begin{align*}
\begin{bmatrix}
x & y
\end{bmatrix}
\begin{bmatrix}
4 & 1 \\
1 & 4
\end{bmatrix}
\begin{bmatrix}
x \\
y 
\end{bmatrix}
&= 4x^2 - 2xy + 4y^2 \\
&= (x^2 - 2xy + y^2) + 3x^2 + 3y^2 \\
&= (x-y)^2 + 3x^2 + 3y^2 > 0
\end{align*}
which is always positive as long as $x$ and $y$ are not both zeros.
\end{solution}

Since it has been shown that symmetric matrices can undergo orthogonal diagonalization, which is essentially a change of coordinates to make the matrix representation of a linear operator diagonal, we may ask in general how a coordinate transformation works for a quadratic/symmetric bilinear form. However, bear in mind that the previous way of changing coordinates (Properties \ref{proper:endomorph}, $A' = P^{-1}AP$) is based on regarding the matrix to be a linear transformation, and hence it is reasonable that the rule of coordinate transformation will be somehow different when the matrix acts like a quadratic form instead. Let's take a step back and consider the change of coordinates for a vector in (\ref{eqn:bijectvTv}) under Theorem \ref{thm:bijectivechincoord}: $[\vec{x}]_\beta = P_{\beta'}^\beta [\vec{x}']_{\beta'}$, hence
\begin{align*}
Q(\vec{x}) &\equiv [\vec{x}]_\beta^T B [\vec{x}]_\beta \\
&= (P_{\beta'}^\beta [\vec{x}']_{\beta'})^T B (P_{\beta'}^\beta [\vec{x}']_{\beta'}) \\
&= [\vec{x}']_{\beta'}^T ((P_{\beta'}^\beta)^T B P_{\beta'}^\beta) [\vec{x}']_{\beta'} = [\vec{x}']_{\beta'}^T B' [\vec{x}']_{\beta'} 
\end{align*}
so we identify the coordinate transformation of a quadratic form by $B' = P^TBP$ where $P$ is some invertible coordinate basis matrix. In this case, $B'$ and $B$ are known as \index{Congruent}\keywordhl{congruent}.
\begin{defn}
\label{defn:coordtransquad}
The coordinate transformation of a symmetric matrix $B$ as a quadratic form follows
\begin{align}
B' = P^TBP
\end{align}
where $P$ is invertible and consists of column vectors in the new basis expressed relative to the old basis. Any pair of $B'$ and $B$ related in this way are referred to as \textit{congruent}.
\end{defn}
Fortunately, previously during orthogonal diagonalization, we have $P^{-1} = P^T$, and hence the corresponding coordinate transformation of a symmetric matrix to its orthonormal eigenbasis by either treating it as a linear operator or quadratic form coincides. Hence the quadratic form $B$ can be transformed into (and congruent to) a diagonal matrix $D = P^TBP$ where the columns of $P$ are all the orthonormal eigenvectors of $B$ (Properties \ref{proper:orthobasissym})\footnote{Notice that this is not the only way to make a quadratic form diagonal (unlike a linear transformation, dictated by its eigenvectors) and there exist many other $P$ that can do it. (However, an important observation about them is Theorem \ref{thm:sylvester}, to be introduced soon.)}. Furthermore, such a diagonal matrix $D$ contains the eigenvalues of $B$ and let's say $r$ of them are positive: $\lambda_1^+, \lambda_2^+, \cdots, \lambda_r^+$, $s$ of them are negative $\lambda_1^-,  \lambda_2^-, \cdots, \lambda_s^-$, and the remaining eigenvalues are zeros. Arranging them in the order of positive-negative-zero, and via an extra real diagonal factor matrix $F$, where
\begin{align}
F_{kk} = 
\begin{cases}
\frac{1}{\sqrt{\lambda_{k}^+}} & 1 \leq k \leq r \\
\frac{1}{\sqrt{-\lambda_{k}^-}} & r+1 \leq k \leq r+s \\
1 & r+s+1 \leq k \leq n
\end{cases}
\end{align}
it is easy to see that $F$ is invertible, and we can further transform the quadratic form into
\begin{align*}
C &= F^TDF \\
&= \left[\begin{smallmatrix}
\frac{1}{\sqrt{\lambda_{1}^+}} & & 0 & & & & &\\
 & \ddots & & & & & & \\
0 & & \frac{1}{\sqrt{\lambda_{r}^+}} & 0 & & & &\\
 & & 0 & \frac{1}{\sqrt{-\lambda_{1}^-}} & & 0 & & \\
 & & & & \ddots & & & \\
 & & & 0 & & \frac{1}{\sqrt{-\lambda_{s}^-}} & 0 &  \\
 & & & & & 0 & 1 & 0\\
 & & & & & & 0 & \ddots
\end{smallmatrix}\right]^T
\left[\begin{smallmatrix}
\lambda_1^+ & & 0 & & & & &\\
 & \ddots & & & & & & \\
0 & & \lambda_{r}^+ & 0 & & & &\\
 & & 0 & \lambda_{1}^- & & 0 & & \\
 & & & & \ddots & & & \\
 & & & 0 & & \lambda_{s}^- & 0 &  \\
 & & & & & 0 & 0 & 0\\
 & & & & & & 0 & \ddots
\end{smallmatrix}\right] \\
& \quad \left[\begin{smallmatrix}
\frac{1}{\sqrt{\lambda_{1}^+}} & & 0 & & & & &\\
 & \ddots & & & & & & \\
0 & & \frac{1}{\sqrt{\lambda_{r}^+}} & 0 & & & &\\
 & & 0 & \frac{1}{\sqrt{-\lambda_{1}^-}} & & 0 & & \\
 & & & & \ddots & & & \\
 & & & 0 & & \frac{1}{\sqrt{-\lambda_{s}^-}} & 0 &  \\
 & & & & & 0 & 1 & 0\\
 & & & & & & 0 & \ddots
\end{smallmatrix}\right] \\
&=
\begin{bmatrix}
1 & & 0 & & & & &\\
 & \ddots & & & & & & \\
0 & & 1 & 0 & & & &\\
 & & 0 & -1 & & 0 & & \\
 & & & & \ddots & & & \\
 & & & 0 & & -1 & 0 &  \\
 & & & & & 0 & 0 & 0\\
 & & & & & & 0 & \ddots
\end{bmatrix} =
\begin{bmatrix}
I_r & & \\
& -I_s & \\
  & & [\textbf{0}]
\end{bmatrix}
\end{align*}
which is known as the \index{Canonical Quadratic Form}\keywordhl{canonical quadratic form} for $B$. To summarize, the matrix product $PF$ converts $B$ into such a form by $(PF)^TB(PF) = F^T (P^T BP)F = F^TDF = C$, and therefore $B$ is congruent to its canonical quadratic form. The following theorem shows that two different canonical quadratic forms cannot be congruent, and hence the canonical quadratic form of any matrix is unique.
\begin{thm}
\label{thm:prepsylvester}
Two canonical quadratic forms of the same extent $n$
\begin{align*}
C &=
\begin{bmatrix}
I_r & & \\
& -I_s & \\
  & & [\textbf{0}]
\end{bmatrix} & 
& C' = \begin{bmatrix}
I_{r'} & & \\
& -I_{s'} & \\
  & & [\textbf{0}]
\end{bmatrix}
\end{align*}
are congruent if and only if $r = r'$ and $s = s'$. As a corollary, this shows the uniqueness of the canonical quadratic form for any quadratic form.
\end{thm}
\begin{proof}
The "if" part is trivial. For the "only if" part, without loss of generality, assume $r' > r$. If the congruence relation $C' = P^TCP$ has to hold where $P$ is some matrix, then consider
\begin{align*}
\textbf{e}^{(j)T}C'\textbf{e}^{(j)} &= 1 > 0 & \text{ where $1 \leq j \leq r'$}
\end{align*}
which is also equal to 
\begin{align*}
\textbf{e}^{(j)T}P^TCP\textbf{e}^{(j)} = \textbf{p}^{(j)T}C\textbf{p}^{(j)}
\end{align*}
where $P = \begin{bmatrix}
\textbf{p}^{(1)} | \cdots | \textbf{p}^{(n)}    
\end{bmatrix}$ is consisted of $n$ column vectors $\textbf{p}^{(j)}$ and $P\textbf{e}^{(j)} = \textbf{p}^{(j)}$. We claim that there exists a non-trivial linear combination of $\textbf{p}^{(j)}$, $1 \leq j \leq r'$, i.e.\ $\textbf{q} = \sum_{j=1}^{r'} c_j\textbf{p}^{(j)}$, such that $\textbf{q}_i = 0$ for $1 \leq i \leq r$.\footnote{The corresponding system is
\begin{align*}
\begin{bmatrix}
\textbf{p}_1^{(1)} & \cdots &\textbf{p}_1^{(r)} & \cdots & \textbf{p}_1^{(r')} \\
\vdots & & \vdots & & \vdots \\
\textbf{p}_r^{(1)} & \cdots & \textbf{p}_r^{(r)} & \cdots & \textbf{p}_r^{(r')} \\
\vdots & & \vdots & & \vdots \\
\textbf{p}_n^{(1)} & \cdots & \textbf{p}_n^{(r)} & \cdots & \textbf{p}_n^{(r')}
\end{bmatrix}
\begin{bmatrix}
c_1 \\
\vdots \\
c_r \\
\vdots \\
c_{r'}
\end{bmatrix} = 
\begin{bmatrix}
0 \\
\vdots \\
0 {\scriptsize \text{ (the $r$-th entry)}}\\
* 
\end{bmatrix}
\end{align*}
The part below the $r$-th row does not matter since the constraints are for the first $r$ rows and so it is effectively an $r \times r'$ underdetermined homogeneous linear system. By the discussion in Section \ref{subsection:SolLinSysGauss}, we know that there will be non-trivial solutions for the $c_j$, $1 \leq j \leq r'$.} Subsequently, consider $\textbf{x} = \sum_{j=1}^{r'} c_j \textbf{e}^{(j)}$, and
\begin{align*}
\textbf{x}^T C'\textbf{x} &= 
\begin{bmatrix}
c_1 & \cdots & c_r' & 0 & \cdots  
\end{bmatrix}
\begin{bmatrix}
I_{r'} & & \\
& -I_{s'} &  \\
& & [\textbf{0}]
\end{bmatrix}
\begin{bmatrix}
c_1 \\
\vdots \\
c_r' \\
0 \\
\vdots
\end{bmatrix} \\
&= c_1^2 + \cdots + c_r'^2 = \sum_{j=1}^{r'} c_j^2 > 0   
\end{align*}
but also $P\textbf{x} = \sum_{j=1}^{r'} c_j P\textbf{e}^{(j)} = \sum_{j=1}^{r'} c_j \textbf{p}^{(j)} = \textbf{q}$, thus similarly
\begin{align*}
\textbf{x}^T P^T C P\textbf{x} &= (P\textbf{x})^T C P\textbf{x} \\
&= \textbf{q}^T C \textbf{q} \\
&= 
\begin{bmatrix}
0 & \cdots & 0 {\scriptsize \text{ (the $r$-th entry)}} & *
\end{bmatrix}
\begin{bmatrix}
I_r & & \\
& -I_s &  \\
& & [\textbf{0}]
\end{bmatrix}
\begin{bmatrix}
0 \\
\vdots \\
0 {\scriptsize \text{ (the $r$-th entry)}} \\
*
\end{bmatrix} \leq 0
\end{align*}
Hence $0 < \textbf{x}^T C'\textbf{x} = \textbf{x}^T P^T CP\textbf{x} \leq 0$ which is a contradiction, and it must be that $r' = r$.\footnote{The same argument in opposite direction will show that it is also not possible to have $r' < r$.} By the same logic, we have $s = s'$ as well.
\end{proof}
An immediate result from this is \index{Sylvester's Law of Inertia}\keywordhl{Sylvester's Law of Inertia}.
\begin{thm}[Sylvester's Law of Inertia]
\label{thm:sylvester}
All diagonalized representations of any quadratic form have the same numbers of positive, negative, and zero diagonal entries. They are collectively known as the \index{Signature}\keywordhl{signature} of the quadratic form. Furthermore, if two diagonalized quadratic forms have the same signature, they are congruent, and vice versa.
\end{thm}
\begin{proof}
If two diagonalized representations of a quadratic form have different signatures, then they can be transformed into two canonical quadratic forms with those two sets of signatures using suitable factor matrices as introduced previously. However, this violates the uniqueness of canonical quadratic form in Theorem \ref{thm:prepsylvester}, and hence the two diagonalized representations of a quadratic form must have the same signature. The last statement follows as they will have the same canonical quadratic form and both are congruent to it.
\end{proof}

\begin{exmp}
Show that
\begin{align*}
B &= \begin{bmatrix}
2 & 3 \\
3 & 0
\end{bmatrix}
& \text{and} & 
& B' = 
\begin{bmatrix}
1 & 0 \\
0 & -2
\end{bmatrix}
\end{align*}
are congruent.
\end{exmp}
\begin{solution}
By Sylvester's Law of Inertia above, we simply count if the two (symmetric) quadratic forms have the same numbers of positive/negative/zero eigenvalues as they will be the diagonal entries when converted via orthogonal diagonalization. The eigenvalues of $B$ are found by
\begin{align*}
\det(B - \lambda I) = 
\begin{vmatrix}
2 - \lambda & 3 \\
3 & - \lambda
\end{vmatrix} &= 0 \\
(2-\lambda)(-\lambda) - (3)(3) = -9 - 2\lambda + \lambda^2 &= 0
\end{align*}
whose solution is 
\begin{align*}
\lambda &= \frac{-(-2) \pm \sqrt{(-2)^2 - 4(1)(-9)}}{2} \\
&= 1 \pm \sqrt{10}
\end{align*}
so there will be one positive and one negative eigenvalue for $B$. It is obvious that the eigenvalues for $B'$ are $\lambda = 1, -2$ so one of them is positive and another is negative as well. Therefore, they are congruent.\footnote{One possible choice of $P$ as in $B' = P^TBP$ is $P = 
\begin{bmatrix}
\frac{1}{\sqrt{2}}&-1\\ 
0&\frac{2}{3}
\end{bmatrix}$.}
\end{solution}

\subsection{Conic Sections}
\label{Conic}
\index{Conic Section}\keywordhl{Conic Sections} are the name given to three types of geometric curves in a two-dimensional space, \textit{ellipses/circles}, \textit{parabola}, and \textit{hyperbola}. The name originates from the fact that they can be obtained by intersecting a plane with a \textit{double cone} (see Figure \ref{fig:conicsecs}). They can be described by the general equation form as below.
\begin{figure}[ht!]
\centering
% Copyleft 2015 | Ridlo W. Wibowo
% ridlo.w.wibowo@gmail.com
\begin{tikzpicture}
\draw[black] (-3,3) ellipse (2 and 0.5);
\draw[black] (3,3) ellipse (2 and 0.5);
\draw[] (-4.95,2.88) -- (-1.05,-2.88);
\draw[] (-4.95,-2.88) -- (-1.05,2.88);
\draw[] (4.95,2.88) -- (1.05,-2.88);
\draw[] (4.95,-2.88) -- (1.05,2.88);

\draw[blue, thick] (-3,-0.6) ellipse (0.4 and 0.1);
\draw[] (-3.8,-0.8) -- (-2.2,-0.8);
\draw[] (-3.8,-0.8) -- (-3.4,-0.4);
\draw[] (-2.2,-0.8) -- (-2.6,-0.4);
\draw[] (-3.4,-0.4) -- (-3.3,-0.4);
\draw[] (-2.6,-0.4) -- (-2.7,-0.4);
\draw[dashed] (-2.7,-0.4) -- (-3.3,-0.4);
\node[] at (-1.8,-0.4) {\textcolor{blue}{Circle}};

\draw[rotate around={20:(-3.26,-1.72)}, red, thick] (-3.26,-1.72) ellipse (1.2 and 0.2);
\draw[] (-4.5,-2.68) -- (-1.5,-1.5);
\draw[] (-4.5,-2.68) -- (-5,-2.06);
\draw[] (-5,-2.06) -- (-4.1,-1.7);
\draw[] (-1.5,-1.5) -- (-2,-0.88);
\draw[] (-2,-0.88) -- (-2.3,-1);
\draw[dashed] (-2.3,-1) -- (-4.1,-1.7);
\node[] at (-1.6,-2) {\textcolor{red}{Ellipse}};

\draw[Green, thick] (-4,3.4) .. controls (-4.5,1.2) and (-4.2,1.5) .. (-3,2.5);
\draw[dotted, Green, thick] (-4,3.4) -- (-3,2.5);
\draw[] (-2.6,2.45) -- (-3.8,-0.15);
\draw[] (-2.6,2.45) -- (-4.1,3.8) coordinate (A);
\draw[] (-3.8,-0.15) -- (-5.3,1.1) coordinate (B);
\draw[dashed] (A) -- (B);
\draw[] (A) -- (-4.28,3.4);
\draw[] (B) -- (-4.7,2.45);
\node[] at (-2.7,1.7) {\textcolor{Green}{Parabola}};

\draw[] (4.5,-3.8) coordinate (A) -- (4.5,2.8) coordinate (B);
\draw[dashed] (A) -- (3.2,-2.63) coordinate (C);
\draw[] (B) -- (3.2,3.97) coordinate (D);
\draw[dashed] (C) -- (3.2, 3.6);
\draw[] (3.2, 3.6) -- (D);
\draw[purple, thick] (4.3,-3.4) coordinate (E) .. controls (4,-0.9) and (3.8,-0.6) .. (3.3,-2.5) coordinate (F);
\draw[dotted, purple, thick] (E) -- (F); 
\draw[purple, thick] (4.3,2.6) coordinate (G) .. controls (4,1) and (3.8,0.7) .. (3.3,3.5) coordinate (H);
\draw[dotted, purple, thick] (G) -- (H); 
\draw[] (A) -- (4.1,-3.44);
\node[] at (5,1) {\textcolor{purple}{Hyperbola}};

\pgfpathmoveto{\pgfpoint{-1cm}{-3cm}}
\pgfpatharcto{2cm}{0.5cm}{0}{0}{0}{\pgfpoint{-3cm}{-3.5cm}}
\pgfpathmoveto{\pgfpoint{-5cm}{-3cm}}
\pgfpatharcto{2cm}{0.5cm}{0}{0}{-1}{\pgfpoint{-3cm}{-3.5cm}}
\pgfstroke

\pgfsetdash{{3pt}{3pt}}{0pt}
\pgfpathmoveto{\pgfpoint{-1cm}{-3cm}}
\pgfpatharcto{2cm}{0.5cm}{0}{0}{-1}{\pgfpoint{-3cm}{-2.5cm}}
\pgfpathmoveto{\pgfpoint{-5cm}{-3cm}}
\pgfpatharcto{2cm}{0.5cm}{0}{0}{0}{\pgfpoint{-3cm}{-2.5cm}}
\pgfstroke

\pgfsetdash{{3pt}{0pt}}{0pt}
\pgfpathmoveto{\pgfpoint{1cm}{-3cm}}
\pgfpatharcto{2cm}{0.5cm}{0}{0}{-1}{\pgfpoint{3cm}{-3.5cm}}
\pgfpathmoveto{\pgfpoint{5cm}{-3cm}}
\pgfpatharcto{2cm}{0.5cm}{0}{0}{0}{\pgfpoint{3cm}{-3.5cm}}
\pgfstroke

\pgfsetdash{{3pt}{3pt}}{0pt}
\pgfpathmoveto{\pgfpoint{1cm}{-3cm}}
\pgfpatharcto{2cm}{0.5cm}{0}{0}{0}{\pgfpoint{3cm}{-2.5cm}}
\pgfpathmoveto{\pgfpoint{5cm}{-3cm}}
\pgfpatharcto{2cm}{0.5cm}{0}{0}{-1}{\pgfpoint{3cm}{-2.5cm}}
\pgfstroke
\end{tikzpicture}
\caption{\textit{The creation of conic sections. (Adapted from the code of Ridlo W. Wibowo)}}
\label{fig:conicsecs}
\end{figure}
\begin{defn}[Conic Sections]
\label{defn:conic}
Conic Sections (circles, ellipses, parabola, hyperbola) are the curves generated by a second-degree polynomial in two variables $(x, y)$ that take the general form of
\begin{align}
ax^2 + bxy + cy^2 + mx + ny - h = 0
\end{align}
where $a$, $b$, $c$, $m$, $n$ and $h$ are all constants. It can be expressed as a quadratic form:
\begin{align}
\textbf{x}^T B\textbf{x} = 
\begin{bmatrix}
x & y & 1
\end{bmatrix}
\begin{bmatrix}
a & \frac{b}{2} & \frac{m}{2} \\
\frac{b}{2} & c & \frac{n}{2} \\
\frac{m}{2} & \frac{n}{2} & -h
\end{bmatrix}
\begin{bmatrix}
x \\
y \\
1
\end{bmatrix} = 0
\end{align}
where $\textbf{x}^T = (x,y,1)^T$.
\end{defn}
To see what type of conic sections a quadratic form represents, we can examine the determinants of $B$ and its minor $B_{33}$. We simply state the results below.
\begin{proper}
\label{proper:quadgentype}
The quadratic form constructed in Definition \ref{defn:conic} represents a degenerate conic if $\det(B) = 0$. Otherwise, if $\det(B) \neq 0$, it indicates
\begin{itemize}
    \item a hyperbola if $\det(B_{33}) < 0$;
    \item a parabola if $\det(B_{33}) = 0$;
    \item an ellipse if $\det(B_{33}) > 0$.
\end{itemize}
where
\begin{align*}
B_{33} = 
\begin{bmatrix}
a & \frac{b}{2} \\
\frac{b}{2} & c 
\end{bmatrix}
\end{align*}
In the case of an ellipse so that $\det(B_{33}) > 0$, if $a = c$ and $b = 0$, then it is further reduced to a circle.
\end{proper}
However, for simplicity, we will only discuss the \index{Central Conic}\keywordhl{central conics} where the linear terms $mx$ and $ny$ do not appear. This excludes the case of a parabola, only keeping the ellipses and hyperbola. The quadratic form can then be simplified as follows.
\begin{proper}[Central Conics]
\label{proper:quadcentraltype}
Ellipses (plus circles) and hyperbola, centered at the origin, are called \textit{central conics} and have the form of
\begin{align}
ax^2 + bxy + cy^2 = h
\end{align}
or can be expressed as a quadratic form of
\begin{align}
\textbf{x}^T B_{33} \textbf{x} = h \label{eqn:xB33x}
\end{align}
where now $\textbf{x} = (x,y)^T$ only and $B_{33}$ is as defined in Properties \ref{proper:quadgentype}.
\end{proper}
By Properties \ref{proper:quadgentype}, they can be classified by the discriminant $\Delta = b^2 - 4ac$, which is easily seen to be equal to $-4\det(B_{33})$: The discriminant is positive [negative] when the graph is a hyperbola [ellipse] and $\det(B_{33})$ is negative [positive]. A zero discriminant actually represents a "parabola", but the removal of linear terms in the conic section equation reduces the parabola to degenerate straight lines. \par
$\blacktriangleright$ Short Exercise: Identify the types of curve generated by $x^2 - xy + 2y^2 = 3$ and $x^2 + xy - y^2 = 1$.\footnotemark\par
Notice that sometimes an "ellipse" where $\det(B_{33}) > 0$ may not produce a real graph and is imaginary. (Take $x^2 + 2y^2 = -3$ as an example.) To address this, we can link the definiteness property of quadratic forms to arrive at an equivalent classification:
\begin{thm}
\label{thm:quadcentraltypealt}
Given $\textbf{x}^TB_{33}\textbf{x} = h$ as of (\ref{eqn:xB33x}) in Properties \ref{proper:quadcentraltype}, where $h$ is chosen to be $1$ for the scaling, then it represents
\begin{itemize}
\item an ellipse if $B_{33}$ is positive-definite,
\item a hyperbola if $B_{33}$ is indefinite,
\item no real graph if $B_{33}$ is negative-definite.
\end{itemize}
\end{thm}
The above works because if the central conic is a hyperbola and $\det(B_{33})$ is negative, then from the viewpoint of orthogonal diagonalization the $2 \times 2$ $B_{33}$ matrix must have one positive and one negative eigenvalue, which by Theorem \ref{thm:quaddefinite} is the same as being indefinite. It is similar for an ellipse where $B_{33}$ being positive-definite means that its two eigenvalues are both positive and $\det(B_{33})$ is positive as well. Meanwhile, when $B_{33}$ is negative-definite, the two eigenvalues are both negative and $\det(B_{33})$ is still positive. Nevertheless, as $h$ is chosen to be $1$, the negative-definiteness means that $\textbf{x}$ has no real solution.
\begin{figure}[ht!]
\centering
\begin{tikzpicture}
\draw[thick, ->] (-3,0) -- (3,0) node[right]{$x$};
\draw[thick, ->] (0,-3) -- (0,3) node[above]{$y$};
\draw[Green,rotate=30] plot[domain=-1.2:1.3] ({1*cosh(\x)},{1*sinh(\x)});
\draw[Green,rotate=30] plot[domain=-1.2:1.3] ({-1*cosh(\x)},{1*sinh(\x)});
\draw[red,rotate=30] plot[domain=-3:3] ({\x},{0});
\draw[blue,rotate=-60] plot[domain=-3:3] ({\x},{0});
\end{tikzpicture}
\begin{tikzpicture}
\draw[thick, ->] (-3,0) -- (3,0) node[right]{$x$};
\draw[thick, ->] (0,-3) -- (0,3) node[above]{$y$};
\draw[Green, rotate=-30] (0,0) ellipse (2 and 1);
\draw[red,rotate=-30] plot[domain=-3:3] ({\x},{0});
\draw[blue,rotate=60] plot[domain=-3:3] ({\x},{0});
\end{tikzpicture}
\caption{\textit{Left: A hyperbola ($x^2 + 2\sqrt{3}xy - y^2 = 1$), Right: An ellipse ($\frac{7}{4}x^2 + \frac{3}{2}\sqrt{3}xy + \frac{13}{4}y^2 = 1$). Both of them (green) are rotated from their standard position so that their major axis (red) and minor axis (blue) are not aligned with the $x$/$y$ axes and make an angle of 30 degrees.}}
\label{fig:hyperellip}
\end{figure}
For example, the quadratic equation represented by the quadratic form $\textbf{x}^TB\textbf{x} = 1$, where
\begin{align*}
B &=
\begin{bmatrix}
1 & -2 \\
-2 & 3
\end{bmatrix}
\end{align*}
is just $x^2 - 4xy + 3y^2 = 1$. $B$ can be shown to have an eigenvalue of $\lambda = 2 \pm \sqrt{5}$. As $\lambda_+ = 2 + \sqrt{5} > 0$ and $\lambda_- = 2 - \sqrt{5} < 0$, $B$ is indefinite and the curves are a pair of hyperbola by Theorem \ref{thm:quadcentraltypealt}. \par
Figure \ref{fig:hyperellip} shows that hyperbola and ellipses can be rotated from their \index{Standard Position}\textit{standard position}. The effect on a quadratic equation resulting from the coordinate transformation by an orthogonal matrix is to produce cross-product terms ($xy$ in two-dimensional cases), which can be eliminated by an inverse rotation to restore the curves so that the major and minor axes are again oriented along the $x$ and $y$ axes. If the graphs start with being tilted by an angle of $\theta$, we can make a rotation by the same angle $\theta$ but in an opposite direction to recover the standard position. It is equivalent to rotating the coordinate system by an angle of $\theta$ in the same sense as the initial tilting. The readers can refer back to Section \ref{section:orthogeometricsub} and Definition \ref{defn:coordtransquad} about the rotation of a coordinate system for a quadratic form.\footnotetext{The first one is an ellipse ($\Delta = (-1)^2 - 4(1)(2) = -7 < 0$) and the second one is a hyperbola ($\Delta = (1)^2 - 4(1)(-1) = 5 > 0$).}

\begin{exmp}
Rotate the quadratic equation $x^2 - xy + y^2 = 1$ so that the major axis lies along the $x$-axis.
\label{exmp:quadgraphrotate}
\end{exmp}
First, we cast the equation into the quadratic form $\textbf{x}^T B\textbf{x} = 1$, with
\begin{align*}
B &=
\begin{bmatrix}
1 & -\frac{1}{2} \\
-\frac{1}{2} & 1
\end{bmatrix}
\end{align*}
We first find the eigenvalues of $B$, and the characteristic equation is
\begin{align*}
\begin{vmatrix}
1-\lambda & -\frac{1}{2} \\
-\frac{1}{2} & 1-\lambda
\end{vmatrix} = (1-\lambda)^2 - (-\frac{1}{2})^2 &= 0 \\
\lambda^2 - 2\lambda + \frac{3}{4} &= 0 \\
\lambda &= \frac{1}{2} \text{ or } \frac{3}{2}
\end{align*}
So by Theorem \ref{thm:quadcentraltypealt}, $A$ is positive-definite and it is an ellipse. The smaller [larger] eigenvalue corresponds to the major [minor] axis. Now we consider an orthogonal matrix $P$ to perform a rotation on the coordinate system, with the old coordinates related to the new coordinates by $\textbf{x} = P\textbf{x}'$. So the quadratic form is transformed to
\begin{align*}
(P\textbf{x}')^T B (P\textbf{x}') &= \textbf{x}'^T (P^T BP) \textbf{x}'
\end{align*}
We immediately identify $P^T BP$ as a rotation of the coordinate system for the matrix $B$, as noted by Definition \ref{defn:coordtransquad}. Section \ref{section:orthogonaldiagreal} tells us that we can deal with the cross-product terms by orthogonal diagonalization, which turns the off-diagonal entries in $B$ into zeros. The normalized eigenvectors of $B$ are found to be
\begin{align*}
&\vec{v}_\lambda = \begin{bmatrix}
\frac{1}{\sqrt{2}} \\
\frac{1}{\sqrt{2}}
\end{bmatrix}
\text{ for } \lambda = \frac{1}{2}
& \begin{bmatrix}
-\frac{1}{\sqrt{2}} \\
\frac{1}{\sqrt{2}}
\end{bmatrix}
\text{ for } \lambda = \frac{3}{2}
\end{align*}
Hence we can set 
\begin{align*}
P =
\begin{bmatrix}
\frac{1}{\sqrt{2}} & -\frac{1}{\sqrt{2}} \\
\frac{1}{\sqrt{2}} & \frac{1}{\sqrt{2}}
\end{bmatrix}
\end{align*}
So that
\begin{align*}
P^T BP = 
\begin{bmatrix}
\frac{1}{\sqrt{2}} & \frac{1}{\sqrt{2}} \\
-\frac{1}{\sqrt{2}} & \frac{1}{\sqrt{2}}
\end{bmatrix}
\begin{bmatrix}
1 & -\frac{1}{2} \\
-\frac{1}{2} & 1
\end{bmatrix}
\begin{bmatrix}
\frac{1}{\sqrt{2}} & -\frac{1}{\sqrt{2}} \\
\frac{1}{\sqrt{2}} & \frac{1}{\sqrt{2}}
\end{bmatrix}
=
\begin{bmatrix}
\frac{1}{2} & 0\\
0 & \frac{3}{2}
\end{bmatrix}
= D
\end{align*}
The new equation is seen to be $\textbf{x}'^T D\textbf{x} = 1$, or $\frac{1}{2}(x')^2 + \frac{3}{2}(y')^2 = 1$. Below are the graphs before and after the rotation. The major (minor) axis now matches the $x$[$y$]-axis as we place the smaller [larger] eigenvalue in the first [second] diagonal entry in $D$.
\begin{figure}[ht!]
\centering
\begin{subfigure}{0.49\textwidth}
\begin{tikzpicture}
\draw[thick, ->] (-3,0) -- (3,0) node[right](vecu){$x$};
\draw[thick, ->] (0,-3) -- (0,3) node[above]{$y$};
\draw[Green, thick, rotate=45] (0,0) ellipse ({1.2*sqrt(2)} and {1.2*sqrt(2/3)});
\draw[red, ->] (0,0) -- (2,2) node[above, rotate=45](vecv){\small $x'$, Major axis, $\lambda = 1/2$};
\draw[blue, ->] (0,0) -- (-1,1) node[below, rotate=-45]{\small $y'$, Minor axis, $\lambda = 3/2$};
\node[Green] at (2,-2.5) {\small $x^2 - xy + y^2 = 1$};
\pic[draw, ->, "$45^\circ$", angle eccentricity=1.875] {angle = vecu--0--vecv};
\end{tikzpicture}
\caption{\textit{Before Rotation.}}
\end{subfigure}
\begin{subfigure}{0.49\textwidth}
\begin{tikzpicture}
\draw[red, thick, ->] (-3,0) -- (3,0) node[right](vecu){$x'$};
\draw[blue, thick, ->] (0,-3) -- (0,3) node[above]{$y'$};
\draw[Green, thick, rotate=0] (0,0) ellipse ({1.2*sqrt(2)} and {1.2*sqrt(2/3)});
\node[Green] at (2,-2.5) {\small $\frac{1}{2}(x')^2 + \frac{3}{2}(y')^2 = 1$};
\end{tikzpicture}
\caption{\textit{After Rotation.}}
\end{subfigure}
\caption{\textit{Illustration for Example \ref{exmp:quadgraphrotate}.}}
\end{figure}
The degree of tilting can be found to be exactly $\pi/4 = 45^{\circ}$, by comparing the general two-dimensional rotation matrix
\begin{align*}
\begin{bmatrix}
\cos \theta & -\sin \theta \\
\sin \theta & \cos \theta
\end{bmatrix}
\end{align*}
against $P$: $\cos \theta = \frac{1}{\sqrt{2}}$ and $\sin \theta = \frac{1}{\sqrt{2}}$ implies that $\tan \theta = 1$, and $\theta = \pi/4$. The possibility of eliminating the cross-product terms in quadratic forms is formally known as the \index{Principal Axes Theorem}\keywordhl{Principal Axes Theorem}.

\begin{thm}[Principal Axes Theorem]
For a quadratic form $\textbf{x}^TB\textbf{x}$, where $B$ is a real symmetric matrix, we can always make an orthogonal change of variables $\textbf{x}' = P^T\textbf{x}$ (or equivalently $\textbf{x} = P\textbf{x}'$) such that it turns into $\textbf{x}'^TD\textbf{x}' = \lambda_1 x_1'^2 + \lambda_2 x_2'^2 + \cdots$ which contains purely quadratic terms and no cross-product terms. The primed coordinates $\textbf{x}'$ then represent the \textit{principal axes}. $P$ is formed by the set of orthonormal column eigenvectors of $B$, and $D$ is a diagonal matrix with entries being the eigenvalues of $B$.
\end{thm}
This is simply a rephrasing of Definition \ref{defn:orthodiagonal} and Properties \ref{proper:orthobasissym}. In general, for a two-dimensional quadratic form
\begin{align*}
\begin{bmatrix}
a & \frac{b}{2} \\
\frac{b}{2} & c
\end{bmatrix}
\end{align*}
It can undergo a rotation of the coordinate system by an angle $\theta$ such that
\begin{align}
\begin{bmatrix}
\cos \theta & \sin \theta \\
-\sin \theta & \cos \theta
\end{bmatrix}
\begin{bmatrix}
a & \frac{b}{2} \\
\frac{b}{2} & c
\end{bmatrix}
\begin{bmatrix}
\cos \theta & -\sin \theta \\
\sin \theta & \cos \theta
\end{bmatrix}
=
\begin{bmatrix}
* & 0\\
0 & *
\end{bmatrix}
\label{eqn:rotate2dquadmat}
\end{align}
the off-diagonal elements become zero. The required $\theta$ is found by expanding the L.H.S. and equating both sides, which gives
\begin{align}
-\sin \theta (a \cos\theta + \frac{b}{2}\sin \theta) + \cos\theta (\frac{b}{2} \cos \theta + c\sin \theta) &= 0 \nonumber \\
\frac{c-a}{2} \sin (2\theta) + \frac{b}{2}\cos(2\theta) &= 0 \nonumber \\
\cot(2\theta) &= \frac{a-c}{b} \label{eqn:rotate2dquad}
\end{align}
where we have applied the familiar double-angle formulae from the first to second line.

\subsubsection{Generalizing to the Three-dimensional Space}
Since physically we are living in a three-dimensional world, it is normal to ask if we can extend the idea of geometrically quadratic shapes from two spatial axes to three. This is possible and we only need to modify the $\textbf{x}$ in the quadratic form to encompass the third axis so that $\textbf{x} = (x,y,z)^T$ and $B$ in $\textbf{x}^TB\textbf{x}$ is now a $3 \times 3$ symmetric matrix. Now the quadratic shapes include \textit{ellipsoids} and \textit{hyperboloids}, and the change of coordinates to convert them into the standard position follows the exact same orthogonal diagonalization procedure. We ask the readers to try working with them in Exercise \ref{ex:ellipsoid}.

\subsection{Hermitian Forms}
\label{section:hermform}

The concept of quadratic forms/symmetric bilinear forms can be readily promoted to complex vector spaces. Since Hermiticity is the complex equivalent of symmetry, it is not surprising that we have the \index{Hermitian Form}\keywordhl{Hermitian forms} as the complex counterpart of symmetric bilinear forms.
\begin{defn}[Hermitian Form]
A (complex) Hermitian form $H(\vec{x}, \vec{y}): \mathcal{V} \times \mathcal{V} \to \mathbb{C}$ takes two vectors $\vec{x}$, $\vec{y} \in \mathcal{V}$ from a complex vector space and returns a complex scalar, that satisfies 
\begin{align}
H(\vec{x}, \vec{y}) = \overline{H(\vec{y}, \vec{x})}
\end{align} for any pair of $\vec{x}$ and $\vec{y}$ (aware of the conjugation!). For finite-dimensional cases, it takes the general form of
\begin{align}
H(\vec{x}, \vec{y}) = \textbf{x}^T \overline{H \textbf{y}} \label{eqn:hermform}
\end{align}
where $H^* = H$ is an $n \times n$ Hermitian matrix and $n$ is the dimension of $\mathcal{V}$.
\end{defn}
Note that the second argument is conjugated just like the complex dot product given in Definition \ref{defn:complexdotproduct}. Due to Properties \ref{proper:hermrealeig}, the eigenvalues of a Hermitian form/matrix $H$ are always real, hence the logic of Theorem \ref{thm:quaddefinite} is still valid and it can be applied when $\vec{x} = \vec{y}$. We simply note the transferred results below.
\begin{thm}
\label{thm:hermdefinite}
The Hermitian form $H(\vec{x}) = \textbf{x}^T \overline{H\textbf{x}}$ by (\ref{eqn:hermform}) where the two input complex vectors are now identical and $H$ is Hermitian, is
\begin{enumerate}[label=(\alph*)]
\item positive-definite, if and only if all eigenvalues of $H$ are positive (positive-semidefinite if and only if all eigenvalues of $H$ are non-negative;
\item negative-definite, if and only if all eigenvalues of $H$ are negative (negative-semidefinite if and only if all eigenvalues of $H$ are non-positive); 
\item indefinite when there are both positive and negative eigenvalues for $H$.
\end{enumerate}
\end{thm}

\begin{exmp}
Show that the Hermitian form characterized by
\begin{align*}
H =
\begin{bmatrix}
-3 & 1+\sqrt{3}i\\ 
1-\sqrt{3}i & -3
\end{bmatrix}
\end{align*}
is negative-definite.
\end{exmp}
\begin{solution}
The readers should check that $H$ is indeed Hermitian. By Theorem \ref{thm:hermdefinite}, we just need to show that the eigenvalues of $H$ are all negative. So we now solve the characteristic polynomial
\begin{align*}
&\quad \begin{vmatrix}
-3-\lambda & 1+\sqrt{3}i\\ 
1-\sqrt{3}i & -3-\lambda
\end{vmatrix} \\
&= (-3-\lambda)(-3-\lambda) - (1-\sqrt{3}i)(1+\sqrt{3}i) \\
&= (9 + 6\lambda + \lambda^2) - 4 \\
&= \lambda^2 + 6\lambda + 5 = (\lambda + 1)(\lambda + 5)
\end{align*}
which yields two negative roots $\lambda = -1, -5$ and we are done.
\end{solution}

\section{Statistics with Quadratic Form}

\subsection{Variance and Covariance}
\label{section:variancesec}
One important quantity in the world of Statistics is the \index{Variance}\keywordhl{variance} of a \index{Random Variable}\keywordhl{random variable} or \textit{time series}. Variance can be viewed as the spread of distribution of the random variable. The larger the variance, the more dispersed the data points are. In Earth Science, we often use it to quantify the variability of certain phenomena or patterns, e.g.\ the variance of spacetime-filtered winds can tell us how active the corresponding wave type is.

\subsubsection{Single Distribution}
We start with the simplest case, the definition of variance for the distribution of a single random variable first. Since in real life, we can only do a finite amount of sampling, the variance of a random variable is always approximated and inferred from the data points if we do not know the underlying statistical distribution.
\begin{defn}
\label{defn:variance}
For a distribution $X$, with $m$ data $x_1, x_2, x_3, \ldots, x_m$, its \index{Population Variance}\keywordhl{population variance} is
\begin{subequations}
\label{eqn:variance}
\begin{align}
\sigma^2 &= \text{Var}(X) \nonumber\\
&= \frac{1}{m} ((x_1 - \mu)^2 + (x_2 - \mu)^2 + (x_3 - \mu)^2 + \cdots + (x_m - \mu)^2) \\
&= \frac{1}{m} \sum_{k=1}^m (x_k - \mu)^2 \label{eqn:varianceb}
\end{align}    
\end{subequations}
where $\mu$ is the \index{Mean}\keywordhl{mean}, or \index{Expected Value}\keywordhl{expected value} of $X$, and is computed by
\begin{align}
\mu = E[X] = \frac{1}{m} (x_1 + x_2 + x_3 + \cdots + x_m)
\end{align}
that is, the average of all data. Hence, variance is the average of squares of differences between the data and their mean, equivalent to $E[(X-\mu)^2]$.
\end{defn}
A simpler formula for computing the population variance is
\begin{align}
\sigma^2 &= E[(X-\mu)^2] \nonumber \\
&= E[(X-E[X])^2] \nonumber \\
&= E[X^2-2XE[X]+(E[X])^2] \nonumber \\
&= E[X^2] - 2E[X]E[X] + (E[X])^2 \nonumber & \begin{aligned} &\text{(Note that $E[X]$ is a constant}\\ &\text{and $E[E[X]] = E[X]$,} \\
&\text{$E[XE[X]] = E[X]E[X]$.)}\end{aligned} \\
&= E[X^2] - (E[X])^2 = E[X^2] - \mu^2 \label{eqn:varshortcut}
\end{align}
As said before, we always have a finite sample size. To account for this, we have to use the sample variance $s^2$, which is the same as population variance but with the $\frac{1}{m}$ factor replaced by $\frac{1}{m-1}$.\footnote{It is because we have used one degree of freedom in estimating the mean.} Hence $s^2 = \frac{m}{m-1}\sigma^2$. As an example, given a dataset $X$, with $5$ data $\vec{x} = (1, 3, 6, 9, 11)^T$, then their mean is
\begin{align*}
\mu = \frac{1}{5}(1 + 3 + 6 + 9 + 11) = 6
\end{align*}
and the population variance is
\begin{align*}
\sigma^2 = \frac{1}{5}((1-6)^2 + (3-6)^2 + (6-6)^2 + (9-6)^2 + (11-6)^2) = 13.6
\end{align*}
We can also use the aforementioned short-cut Formula (\ref{eqn:varshortcut}).
\begin{align*}
\sigma^2 &= E[X^2] - \mu^2 \\
&= \frac{1}{5} (1^2 + 3^2 + 6^2 + 9^2 + 11^2) - 6^2 \\
&= 49.6 - 36 = 13.6
\end{align*}
$\blacktriangleright$ Short Exercise: Find the sample variance of $X$.\footnotemark\par
Note that Formula (\ref{eqn:variance}) for variance in Definition \ref{defn:variance} can be written as a dot product shown below.
\begin{proper}
Given a distribution $X$, with $m$ data $\vec{x} = (x_1, x_2, x_3, \cdots, x_m)^T$, and a mean of $\mu$, the population variance can be written as
\begin{align}
\frac{1}{m} (\vec{x}'\cdot\vec{x}') = \frac{1}{m} \textbf{x}'^T \textbf{x}'
\end{align}
where $\textbf{x}' = \vec{x}' = \vec{x} - \mu$ is the centered distribution with the mean $\mu$ removed.
\end{proper}
It is simply a matter of observing that 
\begin{align*}
\text{Var}(X) &= \frac{1}{m} ((x_1 - \mu)^2 + (x_2 - \mu)^2 + \cdots + (x_m - \mu)^2) \\ &= \frac{1}{m} (x_1 - \mu, x_2 - \mu, \ldots, x_n - \mu)^T \cdot (x_1 - \mu, x_2 - \mu, \ldots, x_m - \mu)^T    
\end{align*}
Notice that variance, as a sum of squares, can never be negative, and is a positive-semidefinite quantity. \par
$\blacktriangleright$ Short Exercise: Discuss under what situation the variance will be zero.\footnotemark

\subsubsection{Linear Combination of Multiple Distributions}
Sometimes we may need to consider the "overall" distribution of the sum of multiple variables. More generally, given any linear combination of multiple ($n$) distributions, like $Z = c_1X^{(1)} + c_2X^{(2)} + \cdots + c_nX^{(n)}$, we may want to know how to compute its mean and variance. The mean will be simply 
\begin{align}
\mu_Z &= E[c_1X^{(1)} + c_2X^{(2)} + \cdots + c_nX^{(n)}] \nonumber \\
&= c_1E[X^{(1)}] + c_2E[X^{(2)}] + \cdots + c_nE[X^{(n)}] \nonumber \\
&= c_1\mu_1 + c_2\mu_2 + \cdots + c_n\mu_n     
\end{align}
where $E$ is linear and $E[X^{(j)}] = \mu_j$ is the mean of $X^{(j)}$. The variance $\text{Var}(Z)$ is a bit more complicated. First, we need to introduce the concept of \index{Covariance}\keywordhl{covariance} between any two variables, which indicates how they change together.\footnotetext[\numexpr\value{footnote}-1]{It is $s^2 = \frac{1}{5-1}((1-6)^2 + (3-6)^2 + (6-6)^2 + (9-6)^2 + (11-6)^2) = 17$.\\(Or simply compute $\frac{5}{5-1}\sigma^2$.)}\footnotetext{When all data are equal (to the mean).}
\begin{defn}
\label{defn:covariance}
For two distributions $X$ and $Y$ consisted of $m$ pairs of data, their \index{Population Covariance}\keywordhl{population covariance} is
\begin{align}
\text{Cov}(X,Y) &= \frac{1}{m}((x_1-\mu_x)(y_1-\mu_y) + (x_2-\mu_x)(y_2-\mu_y)) \nonumber \\
&\quad + \cdots + (x_m-\mu_x)(y_m-\mu_y)) \nonumber \\
&= \frac{1}{m}\sum_{k=1}^{m} (x_k-\mu_x)(y_k-\mu_y) \label{eqn:covsumprod}
\end{align}
where $\mu_x$ and $\mu_y$ are the population means of $X$ and $Y$ respectively. It can be easily seen that $\text{Cov}(X,Y) = \text{Cov}(Y,X)$ so the order does not matter. If $\vec{x}'$ and $\vec{y}'$ are the centered data with their respective mean subtracted away, then their covariance can be denoted by a dot product as
\begin{align}
\frac{1}{m} \vec{x}' \cdot \vec{y}' = \frac{1}{m} \textbf{x}'^T \textbf{y}' \label{eqn:covdot}
\end{align}
For \index{Sample Covariance}\keywordhl{sample covariance}, it is
\begin{align}
q_{xy} = \frac{1}{m-1} \sum_{k=1}^{m} (x_k-\bar{x})(y_k-\bar{y})
\end{align}
where the $\frac{1}{m-1}$ factor replaces $\frac{1}{m}$ as for sample variance, $\bar{x}$ and $\bar{y}$ are the \index{Sample Mean}\keywordhl{sample means} of $X$ and $Y$ which happen to have the same values as $\mu_x$ and $\mu_y$.
\end{defn}
There is also a short-cut formula very similar to that for variance:
\begin{align}
\text{Cov}(X,Y) &= E[(X-E[X])(Y-E[Y])] \nonumber \\
&= E[XY - E[X]Y - XE[Y] + E[X]E[Y]] \nonumber \\
&= E[XY] - E[X]E[Y] - E[X]E[Y] + E[X]E[Y] \nonumber \\
&= E[XY] - E[X]E[Y] \label{eqn:covshort}
\end{align}
In general, if $\text{Cov}(X,Y)$ is positive (negative), it means that when $X$ increases, $Y$ tends to increase (decrease) together. Finally, a direct comparison reveals that $\text{Cov}(X,X) = \text{Var}(X)$ for any distribution $X$. 

\begin{exmp}
Two time series of measured zonal and meridional wind speeds $U$ and $V$ at a weather station are shown in the table below.
\begin{center}
\begin{tabular}{|c|c|c|}
\hline
(in \si{\m \per \s}) & $U$ & $V$\\
\hline
1st Measurement & $4.4$ & $-3.5$ \\
\hline
2nd Measurement & $3.8$ & $-2.6$ \\
\hline
3rd Measurement & $3.3$ & $-2.7$ \\
\hline
4th Measurement & $2.8$ & $-1.4$ \\
\hline
5th Measurement & $2.9$ & $-1.2$ \\
\hline
6th Measurement & $1.7$ & $-0.8$ \\
\hline
7th Measurement & $2.1$ & $-1.1$ \\
\hline
\end{tabular}
\end{center}
Find the covariance of $U$ and $V$.
\end{exmp}
\begin{solution}
It is not hard to get $\mu_U = 3.0$ and $\mu_V = -1.9$. By Definition \ref{defn:covariance}, we have
\begin{align*}
\text{Cov}(U,V) &= \frac{1}{7} [(4.4-3.0)((-3.5)-(-1.9))+(3.8-3.0)((-2.6)-(-1.9)) \\
&\quad+(3.3-3.0)((-2.7)-(-1.9))+(2.8-3.0)((-1.4)-(-1.9)) \\
&\quad+(2.9-3.0)((-1.2)-(-1.9))+(1.7-3.0)((-0.8)-(-1.9)) \\
&\quad+(2.1-3.0)((-1.1)-(-1.9))] \\
&= \frac{-5.36}{7} = \SI{-0.77}{\square\m \per \square\s}
\end{align*}
Alternatively, the short-cut Formula (\ref{eqn:covshort}) gives
\begin{align*}
\text{Cov}(U,V) &= E[UV] - \mu_U \mu_V \\
&= \frac{1}{7}[(4.4)(-3.5) + (3.8)(-2.6) + (3.3)(-2.7) + (2.8)(-1.4) \\
&\quad + (2.9)(-1.2) + (1.7)(-0.8) + (2.1)(-1.1)] - (3.0)(-1.9) \\
&= (-6.466) - (-5.7) = \SI{-0.77}{\square\m \per \square\s}
\end{align*}
\end{solution}
There are two take-away observations from the above example. First, if $X$ and $Y$ both have the same unit $a$, then the unit of their covariance, or the variance for each of them individually, will have a unit of $a^2$. If $Y$ has a unit of $b$ instead then their covariance will have a unit of $ab$. Also, covariance can take negative values, which is different from variance which is always non-negative.\par
Another useful measure related to variance and covariance is \index{Correlation}\keywordhl{correlation}. For two distributions $X$ and $Y$, the correlation is defined by the following formula.
\begin{defn}[Correlation]
\label{defn:correlation}
The correlation of two distributions $X$ and $Y$ is
\begin{subequations}
\begin{align}
\rho_{xy} &= \frac{\text{Cov}(X,Y)}{\sqrt{\text{Var}(X) \text{Var}(Y)}} \\
&= \frac{\text{Cov}(X,Y)}{\sqrt{\text{Cov}(X,X) \text{Cov}(Y,Y)}}
\end{align}    
\end{subequations}
where \text{Var} and \text{Cov} are computed as given in Definitions \ref{defn:variance} and \ref{defn:covariance}.
\end{defn}
Moreover,
\begin{proper}
The correlation between any two distributions $X$ and $Y$ falls in the range between $-1$ and $1$, i.e. $-1 \leq \rho_{xy} \leq 1$.
\end{proper}
\begin{proof}
We can rewrite the correlation using the vector notation for covariance, (\ref{eqn:covdot}) in Definition \ref{defn:covariance}, which gives
\begin{align*}
\rho_{xy} &= \frac{(\vec{x}' \cdot \vec{y}')}{\sqrt{(\vec{x}'\cdot\vec{x}')(\vec{y}'\cdot\vec{y}')}} \\
&= \frac{(\vec{x}' \cdot \vec{y}')}{\sqrt{{\norm{\vec{x}'}}\norm{\vec{y}'}}}
\end{align*}
where $\vec{x}' = \vec{x} - \mu_x$ and $\vec{y}' = \vec{y} - \mu_y$ are centered by removing the mean from the original distributions. Observe that this quantity takes the same form as the one in the Cauchy-Schwarz Inequality (Theorem \ref{thm:CauchySch}), and by that, we promptly know that $\abs{\rho_{xy}} \leq 1$.   
\end{proof}

Correlation between two distributions $X$ and $Y$ indicates how their data varies together just like covariance, but normalized by their variances so that it is dimensionless and will not depend on the units used. Therefore, correlation can be considered as a standardized version of covariance that can be compared across different pairs of variables and is more interpretable. If the correlation is positive, then $X$ and $Y$ will generally increase or decrease together. On the other hand, if the correlation is negative, then when one of them increases, one of them will tend to decrease, and vice versa. The higher the magnitude of correlation, the stronger the \textit{linear} relationship. Notice the word "linear" here. If the correlation is close to zero, it simply means that there is no clear linear relationship between them, but this does not exclude the possibility of having other relationships, e.g.\ exponential or quadratic.\par
In the last example, $\text{Cov}(U,V) = \SI{-0.77}{\square\m \per \square\s}$, $\text{Var}(U) = \SI{0.75}{\square\m \per \square\s}$, $\text{Var}(V) = \SI{0.90}{\square\m \per \square\s}$, and $\rho_{uv} = \smash{\frac{-0.77}{\sqrt{(0.75)(0.90)}}} \approx -0.94$. We have used the population variance and covariance for the computation, but they can be replaced by the sample counterparts. It may be tempting to claim that a strong negative relationship exists in this case, however, the sample size here is a bit small for this result to be meaningful. \par
$\blacktriangleright$ Short Exercise: When will $\rho_{xy}$ take the value of $1$ (or $-1$)?\footnotemark\par

We are now prepared to derive the variance formula for linear combinations of multiple variables.
\begin{proper}
\label{proper:variancemul}
For a distribution constructed by a linear combination of multiple random variables, in the form of $Z = c_1X^{(1)} + c_2X^{(2)} + \cdots + c_nX^{(n)}$, where the coefficients $\vec{c} = (c_1, c_2, \cdots, c_n)^T$ are all constants, the variance $\text{Var}(Z)$ can be expressed as a quadratic form $\textbf{c}^TQ\textbf{c}$, where
\begin{subequations}
\begin{align}
Q &=
\begin{bmatrix}
\text{Cov}(X^{(1)}, X^{(1)}) & \text{Cov}(X^{(1)}, X^{(2)}) & \cdots & \text{Cov}(X^{(1)}, X^{(n)}) \\
\text{Cov}(X^{(2)}, X^{(1)}) & \text{Cov}(X^{(2)}, X^{(2)}) & & \text{Cov}(X^{(2)}, X^{(n)}) \\
\vdots & & \ddots & \vdots \\
\text{Cov}(X^{(n)}, X^{(1)}) & \text{Cov}(X^{(n)}, X^{(2)}) & \cdots & \text{Cov}(X^{(n)}, X^{(n)}) \\
\end{bmatrix} \\
&=
\begin{bmatrix}
\text{Var}(X^{(1)}) & \text{Cov}(X^{(1)}, X^{(2)}) & \cdots & \text{Cov}(X^{(1)}, X^{(n)}) \\
\text{Cov}(X^{(2)}, X^{(1)}) & \text{Var}(X^{(2)}) & & \text{Cov}(X^{(2)}, X^{(n)}) \\
\vdots & & \ddots & \vdots \\
\text{Cov}(X^{(n)}, X^{(1)}) & \text{Cov}(X^{(n)}, X^{(2)}) & \cdots & \text{Var}(X^{(n)}) \\
\end{bmatrix} 
\end{align}    
\end{subequations}
is the so-called \index{Covariance Matrix}\keywordhl{covariance matrix} so that $Q_{ij} = \text{Cov}(X^{(i)}, X^{(j)})$. If $[X'] = [X'^{(1)}|X'^{(2)}|\cdots|X'^{(n)}]$ is the matrix consisted of the centered variables $X'^{(j)} = X^{(j)} - E[X^{(j)}]$ in columns, then we have $Q = \frac{1}{m-1}[X']^T[X']$ where $m$ is the number of data.
\end{proper}
\begin{proof}
Let's say we have $m$ data for $Z$: $z_1, z_2, \ldots, z_m$, as well as each of the $X^{(j)}$: $x_1^{(j)}, x_2^{(j)}, \ldots, x_m^{(j)}$. Denote the mean of $X^{(j)}$ by $\mu_{j}$. Starting from the Expression (\ref{eqn:varianceb}) in Definition \ref{defn:variance}, we have
\begin{align*}
\text{Var}(Z) &= \frac{1}{m-1} \sum_{k=1}^m (z_k - \mu_z)^2 \\
&= \frac{1}{m-1} \sum_{k=1}^m \left(\sum_{j=1}^{n} c_jx^{(j)}_k - \sum_{j=1}^{n} c_j\mu_{j}\right)^2  \\
&= \frac{1}{m-1} \sum_{k=1}^m \left(\sum_{j=1}^{n} (c_jx^{(j)}_k - c_j\mu_{j})\right)^2 \\
&= \frac{1}{m-1} \sum_{k=1}^m \left[\left(\sum_{i=1}^{n} c_i(x^{(i)}_k - \mu_{i})\right)\left(\sum_{j=1}^{n} c_j(x^{(j)}_k - \mu_{j})\right)\right]  \\
&\quad \text{(Changing to a new dummy summation variable)} \\
&= \frac{1}{m-1} \sum_{k=1}^m \left(\sum_{i=1}^{n}\sum_{j=1}^{n} c_ic_j (x^{(i)}_k - \mu_{i})(x^{(j)}_k - \mu_{j})\right) \\
&= \sum_{i=1}^{n}\sum_{j=1}^{n} c_ic_j \left(\frac{1}{m-1} \sum_{k=1}^m (x^{(i)}_k - \mu_{i})(x^{(j)}_k - \mu_{j})\right) \\
&\quad \text{(Switching the order of summation)} \\
&= \sum_{i=1}^{n}\sum_{j=1}^{n} c_ic_j\text{Cov}(X^{(i)}, X^{(j)}) \quad \text{((\ref{eqn:covsumprod}) in Definition \ref{defn:covariance})} \\
&= \textbf{c}^T Q\textbf{c}
\end{align*}
\end{proof}
\footnotetext{It will happen if $X$ and $Y$ have a perfect linear positive [negative] relationship so they appear as a straight line $Y = aX + b$ on the $xy$-plane, $a > 0$ [$a < 0$].}
For the two-variable situation, it reduces to
\begin{align}
\textbf{c}^TQ\textbf{c} =
\begin{bmatrix}
c_1 & c_2
\end{bmatrix}
\begin{bmatrix}
\text{Cov}(X^{(1)}, X^{(1)}) & \text{Cov}(X^{(1)}, X^{(2)}) \\
\text{Cov}
(X^{(2)}, X^{(1)}) & \text{Cov}(X^{(2)}, X^{(2)}) 
\end{bmatrix}
\begin{bmatrix}
c_1 \\
c_2
\end{bmatrix}
\label{eqn:cov2var}
\end{align}

\begin{exmp}
From the previous wind speed observations example, if $W = 0.8U-0.6V$, find $\text{Var}(W)$.
\end{exmp}
\begin{solution}
Earlier calculations showed that $\text{Var}(U) = \SI{0.75}{\square\m \per \square\s}$,
$\text{Var}(V) = \SI{0.90}{\square\m \per \square\s}$, $\text{Cov}(U,V) = \text{Cov}(V,U) = \SI{-0.77}{\square\m \per \square\s}$. Inserting the values into Formula  (\ref{eqn:cov2var}) for Properties \ref{proper:variancemul}, we have
\begin{align*}
\text{Var}(W) &=
\begin{bmatrix}
c_u & c_v
\end{bmatrix}
\begin{bmatrix}
\text{Var}(U) & \text{Cov}(U,V) \\
\text{Cov}(U,V) & \text{Var}(V)
\end{bmatrix}
\begin{bmatrix}
c_u \\
c_v
\end{bmatrix} \\
&=
\begin{bmatrix}
0.8 & -0.6
\end{bmatrix}
\begin{bmatrix}
0.75 & -0.77 \\
-0.77 & 0.90
\end{bmatrix}
\begin{bmatrix}
0.8 \\
-0.6
\end{bmatrix} \\
&= \SI{1.54}{\square\m \per \square\s}
\end{align*} 
The readers can cross-check by computing all the $W$ data first and finding their variance directly.
\end{solution}
Finally, recall that variance is always a positive-semidefinite quantity, and since it can be calculated as a quadratic form via the covariance matrix according to Properties \ref{proper:variancemul}, any covariance matrix will also be a positive-semidefinite quadratic form.

\subsection{Principal Component Analysis (PCA)}
\label{subsection:PCA}
A common practice in Earth Science, as well as the growing field of Machine Learning, is to compress the dimensions of a large dataset having many variables/features (\index{Dimensionality Reduction}\textit{dimensionality reduction}). Given a high number of variables (for example, concentrations of various biochemical substances like blood cells or ions in the blood samples of some hospital patients) in measurements, we want to process them to extract and retain the most important patterns or signals. \index{Principal Component Analysis (PCA)}\keywordhl{Principal Component Analysis (PCA)}, which is also known as \index{Empirical Orthogonal Functions (EOFs)}\keywordhl{Empirical Orthogonal Functions (EOFs)} in Atmospheric Sciences, is the most common technique for this purpose, by finding the mode(s), or more precisely, the linear combination of features, which maximizes the variance of the data along that direction. \par
Consider the simplest case with two variables, or time series $X$ and $Y$ first. Assume they have $m$ pairs of data, from $\textbf{x}_1 = (x_1, y_1)$ to $\textbf{x}_m = (x_{m}, y_{m})$. We can compute the covariance matrix
\begin{align*}
Q = \begin{bmatrix}
\text{Cov}(X, X) & \text{Cov}(X, Y) \\
\text{Cov}(Y, X) & \text{Cov}(Y, Y) 
\end{bmatrix}
\end{align*}
which is introduced in the last section. Principal Component Analysis sets out to find a unit vector $\textbf{e}$ so that the variance of data projected onto the direction indicated by $\textbf{e}$, which is $\textbf{e}^T Q \textbf{e}$ following from Properties \ref{proper:variancemul}, is maximized. \par
\begin{figure}[ht!]
    \centering
    \includegraphics{graphics/PCA_first_demo.png}
    \caption{\textit{The two principal axes/directions (PCs) found by Principal Component Analysis for a random set of data (green). The longer, red [shorter, blue] arrow represents the direction of the largest [smallest] variance. The data points can be seen to spread more [less] along that direction.}}
\end{figure}
Now the problem is to find, under what situation $\textbf{e}^T Q \textbf{e}$ will assume its largest value. Here, we introduce a famous technique, called \index{Lagrange Multiplier}\keywordhl{Lagrange Multiplier}, coming from elementary Multivariable Calculus. We will proceed with the case of two variables for brevity.
\begin{thm}[Lagrange Multiplier]
\label{thm:LagrangeMul}
To find the extreme values attained by a function $f(u,v,\ldots)$, under the constraint $g(u,v,\ldots) = 0$, we consider the expression
\begin{align}
h(u,v,\ldots) = f(u,v,\ldots) - \lambda g(u,v,\ldots)
\end{align}
where $\lambda$ is a constant so that the system below has a solution:
\begin{align}
\left\{\begin{alignedat}{2}
&\partial h/\partial u & &= 0 \\
&\partial h/\partial v & &= 0 \\
&\vdots & &= 0
\end{alignedat}\right.
\end{align}
$\partial/\partial u$ ($\partial/\partial v$) means differentiating with respect to $u$ ($v$) only while treating other variables as constants. The values of $u$ and $v$ (as well as other variables) required to attain the extrema for $f$ are determined by solving the system of equations above. 
\end{thm}
We are now going to find the value of $x'$ and $y'$ so that $\textbf{e}^T Q \textbf{e}$ obtains the maximum for $\textbf{e}^T = (x',y')$. The constraint is that $\textbf{e}$ is a unit vector as a direction, and hence by the method of Lagrange Multiplier outlined above, we have
\begin{align}
g(x',y') = x'^2 + y'^2 - 1 = 0
\end{align}
and with $f(x',y') = \textbf{e}^T Q \textbf{e}$
\begin{subequations}
\begin{align}
h(x',y') &= (\textbf{e}^T Q \textbf{e}) - \lambda(x'^2 + y'^2 - 1) \\
&= \begin{bmatrix}
x' & y' \\
\end{bmatrix}
\begin{bmatrix}
\text{Cov}(X, X) & \text{Cov}(X, Y) \\
\text{Cov}(Y, X) & \text{Cov}(Y, Y) 
\end{bmatrix}
\begin{bmatrix}
x' \\
y'
\end{bmatrix}
- \lambda(x'^2 + y'^2 - 1) \nonumber \\
&= x'^2 \text{Cov}(X, X) + 2x'y' \text{Cov}(X, Y) + y'^2\text{Cov}(Y, Y) - \lambda(x'^2 + y'^2 - 1)
\end{align}
\end{subequations}
according to (\ref{eqn:cov2var}). Carrying out the differentiation gives
\begin{subequations}
\begin{empheq}[left={\empheqlbrace}]{alignat=2}
\partial h/\partial x' &= 2x'\text{Cov}(X, X) + 2y'\text{Cov}(X, Y) - 2\lambda x' & &= 0 \\
\partial h/\partial y' &= 2x'\text{Cov}(X, Y) + 2y'\text{Cov}(Y, Y) - 2\lambda y' & &= 0
\end{empheq}
\end{subequations}
This system can be immediately simplified and recognized as
\begin{subequations}
\begin{align}
2 \begin{bmatrix}
\text{Cov}(X, X)-\lambda & \text{Cov}(X, Y) \\
\text{Cov}(Y, X) & \text{Cov}(Y, Y)-\lambda
\end{bmatrix}
\begin{bmatrix}
x' \\
y'
\end{bmatrix} &= 0\\
(Q-\lambda I)\textbf{e} &= 0
\end{align}
\end{subequations}
which is an eigenvalue problem as introduced in Section \ref{section:eigensection}. Hence we conclude that $f(x',y') = \textbf{e}^T Q \textbf{e}$ can attain an extreme value when $\textbf{e} = (x',y')^T$ is a unit eigenvector of $Q$. Notice that since $Q$ is a symmetric matrix, the eigenvectors of $Q$ form an orthonormal basis and are orthogonal to each other by Properties \ref{proper:orthobasissym}. The corresponding magnitude of variance ${\textbf{e}^{(j)T}} Q \textbf{e}^{(j)}$ for the $j$-th eigenvector is
\begin{align}
\textbf{e}^{(j)T} (Q \textbf{e}^{(j)}) &= \textbf{e}^{(j)T} (\lambda_j \textbf{e}^{(j)}) \nonumber \\
&= \lambda_j (\textbf{e}^{(j)T} \textbf{e}^{(j)}) \nonumber \\
&= \lambda_j \norm{\textbf{e}^{(j)}}^2 \nonumber \\
&= \lambda_j 
\end{align}
where we have used the facts that $Q \textbf{e}^{(j)} = \lambda_j \textbf{e}^{(j)}$ as per Definition \ref{defn:eigen} and the length of a unit vector is $1$. This means that the variance along the direction of the eigenvector is exactly equal to the corresponding eigenvalue. Note that orthogonal diagonalization (see the discussion below Definition \ref{defn:coordtransquad}) transforms the quadratic form to a diagonal matrix consisting of the eigenvalues $\lambda_j$, with respect to the coordinate system made up of the orthonormal eigenvectors $\textbf{e}^{(j)}$. Therefore, from this perspective we can come to the same conclusion that the variance of a transformed variable along the direction indicated by each eigenvector is equal to its eigenvalue, and further, the covariance between two orthogonal directions represented by any pair of distinct eigenvectors is zero, and hence the transformed variables are made uncorrelated. 
\begin{thm}[Principal Component Analysis]
\label{thm:PCA}
For a covariance matrix $Q = \frac{1}{m-1}[X']^T[X']$ (which happens to be symmetric), the variance $\textbf{e}^T Q \textbf{e}$ achieves its maximum value $\lambda_1$ along the direction $\textbf{e}^{(1)}$, the largest eigenvalue of $Q$ and the associated unit eigenvector. Generalizing, as $Q$ has $n$ orthonormal eigenvectors $\textbf{e}^{(1)}, \textbf{e}^{(2)}, \cdots, \textbf{e}^{(n)}$, arranged by the magnitude of corresponding eigenvalues $\lambda_1 > \lambda_2 > \cdots > \lambda_n$ from largest to smallest, the largest variance will be $\lambda_1$ when the direction is along $\textbf{e}^{(1)}$, the second largest will be $\lambda_2$ for $\textbf{e}^{(2)}$ and so on, with the smallest variance being $\lambda_n$ for $\textbf{e}^{(n)}$. This set of orthonormal eigenvectors is called the \index{Principal Axes/Directions}\keywordhl{principal axes/directions} in Principal Component Analysis.
\end{thm}
As a side note, any quadratic form $\textbf{e}^TB\textbf{e}$ will attain its maximum and minimum when $\textbf{e}$ is the eigenvector that represents the largest and smallest eigenvalue, which will also be the value that $\textbf{e}^TB\textbf{e}$ takes when it happens. As before, this is under the constraint that $\textbf{e}$ is a unit vector, and this result bears the name of \index{Constrained Extremum Theorem}\textit{Constrained Extremum Theorem}.\par
Going back to the problem of Principal Component Analysis, for each principal direction $\textbf{e}^{(j)}$ and the variance $\lambda_{j}$, we can compute the ratio of \index{Explained Variance}\keywordhl{explained variance}, which is the fraction of $\lambda_{j}$ over the total variance, the sum of eigenvalues/variances from all eigenvectors for the covariance matrix. This quantity allows us to assess how well the principal direction contributes to the total variance. In the coordinate system constructed by the orthonormal eigenvectors of $Q$, $[\textbf{e}] = [\textbf{e}^{(1)}|\textbf{e}^{(2)}|\cdots|\textbf{e}^{(n)}]$, the new coordinates for any ($i$-th) data point are \begin{align}
\vec{u}_i = [\textbf{e}]^T \vec{x}_i \label{eqn:PCAcoordchange} 
\end{align} (see Section \ref{section:orthogeometricsub}). By the Spectral Theorem \ref{thm:spectral}, this $\vec{u}_i$ can be regarded to be the projection of $\vec{x_i} = (x_i^{(1)}, x_i^{(2)}, \ldots, x_i^{(n)})^T$, the $i$-th set of data, onto the principal directions and are called the \index{Principal Components (PCs)}\keywordhl{principal components (PCs)}. \par
Usually, at the start of PCA, we will detrend the data and remove the mean from each variable, such that $x'_i = x_i - \bar{x}$ and $y'_i = y_i - \bar{y}$ are used to replace $x_i$ and $y_i$. This enables us to express the covariance matrix as $Q = \frac{1}{n}[X'|Y']^T[X'|Y']$ following the end remark of Properties \ref{proper:variancemul}. 
\begin{exmp}
\label{exmp:tempPCA}
The temperature data of two cities $X$ and $Y$ are as follows.
\begin{center}
\begin{tabular}{|c|c|c|c|c|c|}
\hline
(in \si{\degree C}) & $X$ & $Y$ & & $X$ & $Y$ \\
\hline
1st Day & $21.0$ & $22.3$ & 8th Day & $22.1$ & $22.4$ \\
\hline
2nd Day & $21.8$ & $21.6$ & 9th Day & $21.5$ & $21.7$ \\
\hline
3rd Day & $20.9$ & $21.2$ & 10th Day & $22.8$ & $22.5$ \\
\hline
4th Day & $21.6$ & $21.7$ & 11th Day & $22.2$ & $21.6$ \\
\hline
5th Day & $23.4$ & $23.2$ & 12th Day & $23.0$ & $23.3$ \\
\hline 
6th Day & $24.7$ & $24.1$ & 13th Day & $24.2$ & $24.7$ \\
\hline 
7th Day & $22.0$ & $23.0$ & 14th Day & $23.8$ & $23.1$ \\
\hline
\end{tabular}
\end{center}
Perform Principal Component Analysis over them and find the most important principal direction, and extract the time series of the corresponding PC.
\end{exmp}
\begin{solution}
The means of $X$ and $Y$ are $\SI{22.5}{\degree C}$ and $\SI{22.6}{\degree C}$ respectively. After detrending by subtracting the respective means, the new data are
\begin{center}
\begin{tabular}{|c|c|c|c|c|c|}
\hline
(in \si{\degree C}) & $X'$ & $Y'$ & & $X'$ & $Y'$ \\
\hline
1st Day & $-1.5$ & $-0.3$ & 8th Day & $-0.4$ & $-0.2$ \\
\hline
2nd Day & $-0.7$ & $-1.0$ & 9th Day & $-1.0$ & $-0.9$ \\
\hline
3rd Day & $-1.6$ & $-1.4$ & 10th Day & $0.3$ & $-0.1$ \\
\hline
4th Day & $-0.9$ & $-0.9$ & 11th Day & $-0.3$ & $-1.0$ \\
\hline
5th Day & $0.9$ & $0.6$ & 12th Day & $0.5$ & $0.7$ \\
\hline 
6th Day & $2.2$ & $1.5$ & 13th Day & $1.7$ & $2.1$ \\
\hline 
7th Day & $-0.5$ & $0.4$ & 14th Day & $1.3$ & $0.5$ \\
\hline
\end{tabular}
\end{center}
From Properties \ref{proper:variancemul}, the sample covariance matrix is
\begin{align*}
Q &= 
\frac{1}{14-1}
\begin{bmatrix}
X' \cdot X' & X' \cdot Y' \\
Y' \cdot X' & Y' \cdot Y' 
\end{bmatrix} \\
&= \frac{1}{13}
\begin{bmatrix}
-1.5 & -0.7 & -1.6 & \cdots \\
-0.3 & -1.0 & -1.4 & \cdots
\end{bmatrix}
\begin{bmatrix}
-1.5 & -0.3 \\
-0.7 & -1.0 \\
-1.6 & -1.4 \\
\vdots & \vdots
\end{bmatrix} \\
&= \frac{1}{13}
\begin{bmatrix}
18.18 & 13.66 \\
13.66 & 13.64
\end{bmatrix} = 
\begin{bmatrix}
1.398 & 1.051 \\
1.051 & 1.049
\end{bmatrix}
\end{align*}
The unit eigenvectors for $Q$ and thus the principal directions can be found to be $\textbf{e}^{(1)} = (0.763, 0.647)^T$ with a larger variance $\lambda_1 = \SI{2.289}{\square {(\degree C)}}$, and $\textbf{e}^{(2)} = (-0.647, 0.763)^T$ of a smaller variance $\lambda_2 = \SI{0.159}{\square {(\degree C)}}$. The first principal component accounts for $\frac{2.289}{2.289+0.159} \approx 93.5\%$ of the total variance.\par
We can project every pair of data $\vec{x}'_i = (x'_i, y'_i)^T$ onto the principal directions by computing $\vec{u}_i = [\textbf{e}]^T\vec{x}'_i$, where $[\textbf{e}] = [\textbf{e}^{(1)}|\textbf{e}^{(2)}]$. The resulting time series for the principal components are 
\begin{center}
\begin{tabular}{|c|c|c|c|c|c|c|c|}
\hline
 & D-1 & D-2 & D-3 & D-4 & D-5 & D-6 & D-7 \\
\hline
$u^{(1)}$ & $-1.338$ & $-1.181$ & $-2.126$ & $-1.268$ & $1.075$ & $2.648$ & $-0.123$ \\
\hline
$u^{(2)}$ & $0.741$ & $-0.310$ & $-0.034$ & $-0.105$ & $-0.124$ & $-0.278$ & $0.628$ \\
\hline
 & D-8 & D-9 & D-10 & D-11 & D-12 & D-13 & D-14 \\
\hline
$u^{(1)}$ & $-0.434$ & $-1.345$ & $0.164$ & $-0.875$ & $0.834$ & $2.655$ & $1.315$ \\
\hline
$u^{(2)}$ & $0.106$ & $-0.040$ & $-0.270$ & $-0.569$ & $0.211$ & $0.503$ & $-0.459$ \\
\hline
\end{tabular}
\end{center}
In detail, the two principal components for the first day are computed by
\begin{align*}
\begin{bmatrix}
u_1^{(1)} \\
u_1^{(2)} 
\end{bmatrix}
=
\begin{bmatrix}
0.763 & 0.647 \\
-0.647 & 0.763
\end{bmatrix}
\begin{bmatrix}
-1.5 \\
-0.3
\end{bmatrix}
=
\begin{bmatrix}
-1.338\\
0.741
\end{bmatrix}
\end{align*}
The original (detrended) data can be recovered by $\vec{x}'_i = [\textbf{e}]\vec{u}'_i$. If we want to extract the signal originating from the first principal direction only, we can simply remove the other column eigenvector(s) in $[\textbf{e}]$ as well as discard the other principal value(s) in $\smash{\vec{u}'_i}$. The time series reconstructed by the first PC mode is hence computed by $\smash{\vec{x}'_i = \textbf{e}^{(1)}u_i^{(1)}}$:
\begin{center}
\begin{tabular}{|c|c|c|c|c|c|c|c|}
\hline
(in \si{\degree C}) & D-1 & D-2 & D-3 & D-4 & D-5 & D-6 & D-7 \\
\hline
$X'$ & $-1.021$ & $-0.901$ & $-1.622$ & $-0.968$ & $0.820$ & $2.020$ & $-0.094$ \\
\hline
$Y'$ & $-0.865$ & $-0.763$ & $-1.374$ & $-0.820$ & $0.695$ & $1.712$ & $-0.079$ \\
\hline
 & D-8 & D-9 & D-10 & D-11 & D-12 & D-13 & D-14 \\
\hline
$X'$ & $-0.331$ & $-1.026$ & $0.125$ & $-0.668$ & $0.636$ & $2.025$ & $1.003$ \\
\hline
$Y'$ & $-0.281$ & $-0.869$ & $0.106$ & $-0.566$ & $0.539$ & $1.716$ & $0.850$ \\
\hline
\end{tabular}
\end{center}
For example, on the third day, PC1 contributes
\begin{align*}
\vec{x}'_3 =
\begin{bmatrix}
x'_3 \\
y'_3
\end{bmatrix}
=
\textbf{e}^{(1)}u_3^{(1)}
=
\begin{bmatrix}
0.763 \\
0.647
\end{bmatrix}
\begin{bmatrix}
-2.126
\end{bmatrix}
=
\begin{bmatrix}
-1.622 \\
-1.374  
\end{bmatrix}
\end{align*}
\end{solution}
\begin{figure}[h!]
\centering
\includegraphics[scale = 0.65]{graphics/PCA_exmp_1.png}\\
\includegraphics[scale = 0.65]{graphics/PCA_exmp_2.png}
\caption{\textit{The data (green) before and after rotation to the principal axes, with the one with a larger/smaller variance shown as a red/blue arrow, for Example \ref{exmp:tempPCA}.}}
\end{figure}
\begin{figure}[h!]
\centering
\includegraphics[scale = 0.8]{graphics/PCA_exmp_3.png}
\caption{\textit{Comparison between the original time series and the one reconstructed using only the first PC for $Y'$ in Example \ref{exmp:tempPCA}.}}
\end{figure}

\section{Python Programming}
Since doing Empirical Orthogonal Functions (EOFs) as an Earth Science application is essentially a PCA, which has to rely on a computer when the dataset is huge, we will get into the Python programming part first in this chapter. We will need the scikit-learn package (\texttt{sklearn}) for this, and let's use Example \ref{exmp:tempPCA} for demonstration.
\begin{lstlisting}
import numpy as np
from sklearn.decomposition import PCA

X = np.array([[21.0, 22.3],
              [21.8, 21.6],
              [20.9, 21.2],
              [21.6, 21.7],
              [23.4, 23.2],
              [24.7, 24.1],
              [22.0, 23.0],
              [22.1, 22.4],
              [21.5, 21.7],
              [22.8, 22.5],
              [22.2, 21.6],
              [23.0, 23.3],
              [24.2, 24.7],
              [23.8, 23.1]])
\end{lstlisting}
We have to prepare the data where each column represents the time series of one variable, so the shape of \verb|X| is $(\text{Number of samples}, \text{Number of features})$. Now define a \verb|PCA| object and fit it with the data. We can choose how many PCs to use, and for the present, we will keep all of them so that \verb|n_components| $=2$.
\begin{lstlisting}
pca = PCA(n_components=2)
pca.fit(X)
\end{lstlisting}
We can retrieve the principal directions and variances by
\begin{lstlisting}
print(pca.components_)
print(pca.explained_variance_)
\end{lstlisting}
which gives
\begin{lstlisting}
[[ 0.76286648  0.64655605]
 [-0.64655605  0.76286648]]
[2.28902525 0.15866706] 
\end{lstlisting}
Notice that the principal directions are arranged in rows so that to get the first one, we write \verb|pca.components_[0,:]| that returns \verb|[0.763 0.647]|. The percentages of explained variances are simply computed by
\begin{lstlisting}
print(pca.explained_variance_/np.sum(pca.explained_variance_))
\end{lstlisting}
that returns \verb|[0.9352 0.0648]|. To obtain the time series of transformed PCs, we simply use the \verb|transform| method:
\begin{lstlisting}
Z = pca.transform(X)
print(Z)
\end{lstlisting}
yielding the expected output of
\begin{lstlisting}
[[-1.33826654  0.74097414]
 [-1.18056259 -0.31027725]
 [-2.12576485 -0.03352339]
 ...
 [ 1.31500445 -0.45908963]]    
\end{lstlisting}
To reconstruct the time series using only some (the first) PC, we can do the following.
\begin{lstlisting}
Z_trimmed = np.copy(Z)
Z_trimmed[:,1:] = 0
X_inv = pca.inverse_transform(Z_trimmed)
print(X_inv)
\end{lstlisting}
These generate
\begin{lstlisting}
[[21.47908131 21.73473567]
 [21.59938837 21.83670011]
 [20.87832525 21.22557387]
 ...
 [23.50317282 23.45022409]]    
\end{lstlisting}
equivalent to the last table in Example \ref{exmp:tempPCA} but with the original means included. The following code paragraphs produce the main part of the three plots shown in the example.
\begin{lstlisting}
W = X - np.mean(X, axis=0)
lambda_1 = pca.explained_variance_[0]
lambda_2 = pca.explained_variance_[1]

import matplotlib.pyplot as plt

plt.scatter(W[:,0], W[:,1], color="g")
plt.xlim([-3,3])
plt.ylim([-3,3])
plt.xlabel("X' (deg C)")
plt.ylabel("Y' (deg C)")
plt.arrow(0,0,lambda_1**0.5*pca.components_[0,0],lambda_1**0.5*pca.components_[0,1], color="r", width=0.05)
plt.arrow(0,0,lambda_2**0.5*pca.components_[1,0],lambda_2**0.5*pca.components_[1,1], color="b", width=0.05)
plt.grid()
plt.gca().set_aspect("equal")
plt.show()

plt.scatter(Z[:,0], Z[:,1], color="g")
plt.xlim([-3,3])
plt.ylim([-3,3])
plt.xlabel("PC1")
plt.ylabel("PC2")
plt.arrow(0,0,lambda_1**0.5,0, color="r", width=0.05)
plt.arrow(0,0,0,lambda_2**0.5, color="b", width=0.05)
plt.grid()
plt.gca().set_aspect("equal")
plt.show()

plt.plot(np.arange(1,14+1), W[:,1], color="g", label="Original Y'")
plt.plot(np.arange(1,14+1), X_inv[:,1] - np.mean(X, axis=0)[1], color="r", label="Y' reconstructed from PC1")
plt.legend()
plt.grid()
plt.xticks(np.arange(1,14+1))
plt.xlabel("Day")
plt.ylabel("Temperature (deg C)")
plt.show()
\end{lstlisting}

\section{Earth Science Applications: Empirical Orthogonal Functions (EOFs)}
\label{section:EOF}

Here we will read the ERA5 dataset for sea surface temperature (SST) and find the dominant patterns of global SST variability. The data can be retrieved from \href{https://cds.climate.copernicus.eu/datasets/reanalysis-era5-single-levels?tab=download}{https://cds.climate.copernicus.eu/datasets/reanalysis-era5-single-levels?tab=download} and selecting the options "Type: Reanalysis, Variable: Sea surface temperature, Time: 00:00, Data format: NetCDF4". We will choose the time period from 1991 to 2020, every month/day, and a spatial domain from \SI{45}{\degree N} to \SI{45}{\degree S} and \SI{110}{\degree E} to \SI{80}{\degree W}. We will also need the land-sea mask that can be downloaded via \href{https://confluence.ecmwf.int/download/attachments/140385202/lsm_1279l4_0.1x0.1.grb_v4_unpack.nc?version=1&modificationDate=1591983422208&api=v2}{https://confluence.ecmwf.int/download/attachments/\\140385202/lsm\_1279l4\_0.1x0.1.grb\_v4\_unpack.nc?version=1\&modification\\Date=1591983422208\&api=v2}. Now, import the required packages.
\begin{lstlisting}
import matplotlib.pyplot as plt
import numpy as np
import pandas as pd
import xarray as xr
from sklearn.decomposition import PCA
import cartopy.crs as ccrs
\end{lstlisting}
Load the SST data, work on coarse grids with a $\SI{2.5}{\degree} \times \SI{2.5}{\degree}$ spatial resolution (optional to reduce the computational cost), and exclude leap days.
\begin{lstlisting}
SST_nc = xr.open_dataset("ERA5_SST.nc")
coarse_lat = np.array(SST_nc["latitude"][::10]) # 0.25 deg * 10 = 2.5 deg
coarse_lon = np.array(SST_nc["longitude"][::10])
time = pd.Series(SST_nc["valid_time"])
no_leapdays = ~((time.dt.month == 2) & (time.dt.day == 29))
\end{lstlisting}
Also, prepare the land-sea mask to exclude land grid points and flatten it for indexing later.
\begin{lstlisting}
land_sea_mask_nc = xr.open_dataset("lsm_1279l4_0.1x0.1.grb_v4_unpack.nc")
land_sea_mask = np.array(land_sea_mask_nc["lsm"].sel({"latitude": coarse_lat, "longitude": coarse_lon}, method="nearest")[0, ...])
land_sea_mask = np.where(land_sea_mask > 0, 1, 0).astype(bool)
land_sea_mask_flatten = land_sea_mask.flatten()
\end{lstlisting}
Preprocessing the SST data by subtracting the yearly climatology to acquire the anomaly fields, scale them by the square roots of cosines of the latitudes to account for area weighting and keep only the valid over-water grid points using the land-sea mask:
\begin{lstlisting}
SST_data_arr = np.array(SST_nc["sst"].sel({"latitude": coarse_lat, "longitude": coarse_lon}))
SST_data_arr_no_leap = SST_data_arr[no_leapdays, ...].reshape(30,365,len(coarse_lat),len(coarse_lon))
# The SST array now has the shape of 30 years * 365 days * nlat * nlon
SST_clim = np.mean(SST_data_arr_no_leap, axis=0)
SST_anomaly = SST_data_arr_no_leap - SST_clim

cos_factor_root = np.cos(np.deg2rad(coarse_lat))**0.5
SST_anomaly_weighted = SST_anomaly * cos_factor_root[None,None,:,None]

SST_flatten = SST_anomaly_weighted.reshape(30*365, len(coarse_lat)*len(coarse_lon))
SST_valid = SST_flatten[:, ~land_sea_mask_flatten]
\end{lstlisting}
Call the PCA and fit it with the prepared, flattened SST data:
\begin{lstlisting}
SST_PCA = PCA(n_components=3) # any number will be fine
SST_PCA.fit(SST_valid)
\end{lstlisting}
Then, recover the latitude-longitude structure of the first PC:
\begin{lstlisting}
PC1 = np.full(len(coarse_lat)*len(coarse_lon), np.nan)
PC1[~land_sea_mask_flatten] = SST_PCA.components_[0,:] # Fill PC1 at appropriate overwater entries
PC1_2D = PC1.reshape(len(coarse_lat), len(coarse_lon)) / cos_factor_root[:,None] # Invert the cosine factor
\end{lstlisting}
and plot it on the map.
\begin{lstlisting}
plt.figure()
plt.subplot(111, projection=ccrs.PlateCarree(central_longitude=180)) # the central_longitude option is needed to plot across the International Date Line
plt.pcolormesh(coarse_lon, coarse_lat, PC1_2D, cmap="coolwarm", vmin=-0.08, vmax=0.08, transform=ccrs.PlateCarree())
plt.gca().coastlines()
gl = plt.gca().gridlines(draw_labels=True)
gl.top_labels = False
plt.title("First EOF of Sea Surface Temperature (SST) - ENSO")
plt.colorbar(orientation="horizontal", label="Temperature Anomaly (deg C)")
plt.savefig("SST_EOF", dpi=300, bbox_inches="tight")
\end{lstlisting}
You should be able to get Figure \ref{fig:ENSOEOF} as below.
\begin{figure}[h!]
\centering
\includegraphics[scale=0.75]{graphics/SST_EOF.png}
\caption{\textit{The first EOF pattern of the Pacific Sea Surface Temperature (SST) anomalies.}}
\label{fig:ENSOEOF}
\end{figure}
This EOF pattern represents the \index{El Niño-Southern Oscillation (ENSO)}\keywordhl{El Niño-Southern Oscillation (ENSO)} phenomenon where the western Pacific gets cooler [warmer] and the eastern Pacific becomes warmer [cooler] during the El Niño [La Niña] phase.

\section{Exercises}

\begin{Exercise}
Identify the following conic sections and eliminate the cross-product terms by an appropriate rotation.
\begin{enumerate}[label=(\alph*)]
\item $x^2 + 5xy + 3y^2 = 1$,
\item $x^2 - xy + 2y^2 = 4$,
\item $x^2 + 2xy + y^2 = 1$, what is the graph generated? 
\end{enumerate}
\end{Exercise}
\begin{Answer}
\begin{enumerate}[label=(\alph*)]
\item The discriminant is $\Delta = 5^2 - 4(1)(3) = 13 > 0$ so it is a hyperbola. The required degree of rotation is found by $\cot(2\theta) = \frac{1-3}{5} = -\frac{2}{5}$. The eigenvalues of
\begin{align*}
\begin{bmatrix}
1 & \frac{5}{2} \\
\frac{5}{2} & 3
\end{bmatrix}
\end{align*}
are $\lambda = \frac{4 \pm \sqrt{29}}{2}$, so the transformed equation will be \\$\frac{4 - \sqrt{29}}{2}(x')^2 + \frac{4 + \sqrt{29}}{2}(y')^2 = 1$.
\item The discriminant is $\Delta = (-1)^2 - 4(1)(2) = -7 > 0$ so it is an ellipse. The required degree of rotation is found by $\cot(2\theta) = \frac{1-2}{-1} = 1$ ($\theta = \frac{\pi}{8}$). The eigenvalues (where we have to divide the equation by $4$ to standardize the coefficient on R.H.S.) of
\begin{align*}
\begin{bmatrix}
\frac{1}{4} & -\frac{1}{8} \\
-\frac{1}{8} & \frac{1}{2}
\end{bmatrix}
\end{align*}
are $\lambda = \frac{3\pm\sqrt{2}}{8}$, so the new equation will be $\frac{3 - \sqrt{2}}{8}(x')^2 + \frac{3 + \sqrt{2}}{8}(y')^2 = 1$.
\item The discriminant is $\Delta = (2)^2 - 4(1)(1) = 0$ so it will be a degenerate case. In fact, the graphs are a pair of parallel straight lines $y = -x \pm 1$.
\end{enumerate}
\end{Answer}

\begin{Exercise}
Find a new expression for the standard hyperbola $y^2 - x^2 = 1$ if an anti-clockwise rotation of $60$ degrees for the coordinate frame is done. What about a reflection along the $x$-axis?
\end{Exercise}
\begin{Answer}
It is to compute
\begin{align*}
& \begin{bmatrix}
\cos (\SI{60}{\degree}) & \sin (\SI{60}{\degree}) \\
-\sin (\SI{60}{\degree}) & \cos (\SI{60}{\degree})
\end{bmatrix}
\begin{bmatrix}
-1 & 0 \\
0 & 1
\end{bmatrix}
\begin{bmatrix}
\cos (\SI{60}{\degree}) & -\sin (\SI{60}{\degree}) \\
\sin (\SI{60}{\degree}) & \cos (\SI{60}{\degree})
\end{bmatrix} \\
=& \begin{bmatrix}
\frac{1}{2} & \frac{\sqrt{3}}{2} \\
-\frac{\sqrt{3}}{2}  & \frac{1}{2}
\end{bmatrix}
\begin{bmatrix}
-1 & 0 \\
0 & 1
\end{bmatrix}
\begin{bmatrix}
\frac{1}{2} & -\frac{\sqrt{3}}{2}  \\
\frac{\sqrt{3}}{2}  & \frac{1}{2}
\end{bmatrix}
=
\begin{bmatrix}
\frac{1}{2}&\frac{\sqrt{3}}{2}\\ 
\frac{\sqrt{3}}{2}&-\frac{1}{2}
\end{bmatrix}
\end{align*}
so the new hyperbola equation is $\frac{1}{2}(x')^2+ \sqrt{3}x'y' -\frac{1}{2}(y')^2 = 1$. Reflection about the $x$-axis should leave the graph unchanged due to symmetry.
\end{Answer}

\begin{Exercise}
\phantomsection
\label{ex:ellipsoid}
Three-dimensional quadrics can also be treated in a similar fashion to two-dimensional conic sections. Find the length of three axes for an ellipsoid $x^2 + y^2 + z^2 + 0.5xy - yz + 0.5xz = 1$ by doing an orthogonal coordinate transformation. What is the requirement for a three-dimensional quadratic form to represent an ellipsoid?
\end{Exercise}
\begin{Answer}
The corresponding quadratic form is
\begin{align*}
\begin{bmatrix}
1 & \frac{1}{4} & \frac{1}{4} \\
\frac{1}{4} & 1 & -\frac{1}{2}\\
\frac{1}{4} & -\frac{1}{2} & 1
\end{bmatrix}
\end{align*}
and its eigenvalues are $\lambda \approx 0.3170, 1.1803$ and $\lambda = \smash{\frac{3}{2}}$. The length of the three main axes will be the square roots of their reciprocals: $\smash{\sqrt{1/0.3170}} = 1.776$, $\smash{\sqrt{1/1.1803}} = 0.920$, and $\smash{\sqrt{\frac{2}{3}}}$. The requirement for a three-dimensional quadratic form to represent an ellipsoid is that the three eigenvalues of that quadratic form are all positive.
\end{Answer}

\begin{Exercise}
Prove that for any matrix $A$, $B = I + (A - A^T)$ is invertible. Hint: Consider $B\vec{x} = \textbf{0}$ and show that $\vec{x} = \textbf{0}$ must be the zero vector, inferred from the quadratic form $\textbf{x}^TB\textbf{x}$.
\end{Exercise}
\begin{Answer}
If $B\vec{x} = \textbf{0}$, then $\textbf{x}^TB\textbf{x} = 0$, and also
\begin{align*}
\textbf{x}^TB\textbf{x} &= \textbf{x}^T[I + (A - A^T)]\textbf{x} \\
&= \textbf{x}^TI\textbf{x} + \textbf{x}^T(A - A^T)\textbf{x} \\
&= \textbf{x}^T\textbf{x} + 0 = \norm{\vec{x}}^2
\end{align*}
where from the second to last line we have used the fact that $A - A^T$ is skew-symmetric and does not contribute to the quadratic form. This implies that $\norm{\vec{x}}^2 = 0$ so by positivity $\vec{x} = 0$ has to be the zero vector, and so the only solution to $B\vec{x} = \textbf{0}$ is the trivial solution and by the equivalence statements $B = I + (A - A^T)$ is invertible.
\end{Answer}

\begin{Exercise}
\phantomsection
\label{ex:sylvesterdefinite}
Use Sylvester's Law of Inertia to argue that a symmetric matrix $B$ is positive-definite if and only if $B = P^TP$ for some invertible matrix $P$. Hint: Consider $P^T IP$.
\end{Exercise}
\begin{Answer}
A positive-definite symmetric matrix has all positive eigenvalues and thus a canonical quadratic form of an identity matrix $I$ according to Sylvester's Law of Inertia. By congruence, it must be $B = P^T IP = P^TP$ for some (invertible) $P$.
\end{Answer}

\begin{Exercise}
Find the covariance matrix for three sea level pressure time series over $10$ days at cities $X$, $Y$, $Z$ (in hPa, relative to $1000$ hPa), which take the values
\begin{center}
\begin{tabular}{|c|c|c|}
\hline
$X$ & $Y$ & $Z$ \\
\hline
$17.1$ & $19.2$ & $22.0$ \\
\hline
$18.5$ & $16.9$ & $25.4$ \\
\hline
$14.8$ & $15.3$ & $17.3$ \\
\hline
$19.7$ & $21.6$ & $23.5$ \\
\hline
$24.1$ & $22.3$ & $26.8$ \\
\hline
$21.6$ & $20.9$ & $23.2$ \\
\hline 
$28.0$ & $26.7$ & $29.5$ \\
\hline
$24.3$ & $25.0$ & $22.$5 \\
\hline
$20.3$ & $21.5$ & $27.2$ \\
\hline
$23.4$ & $22.4$ & $24.6$ \\
\hline
\end{tabular}
\end{center}
Find the variance of $W = X - 0.5Y - 0.5Z$.
\end{Exercise}
\begin{Answer}
The numerical values of the covariance matrix should be
\begin{align*}
\begin{bmatrix}
\text{Cov}(X,X) & \text{Cov}(X,Y) & \text{Cov}(X,Z) \\
\text{Cov}(Y,X) & \text{Cov}(Y,Y) & \text{Cov}(Y,Z) \\
\text{Cov}(Z,X) & \text{Cov}(Z,Y) & \text{Cov}(Z,Z)
\end{bmatrix}
=
\begin{bmatrix}
15.264 & 12.3984 & 9.6756 \\
12.3984 & 11.664 & 7.4333 \\
9.6756 & 7.4333 & 11.3644
\end{bmatrix}
\end{align*}
and the required variance of $W$ is
\begin{align*}
\begin{bmatrix}
1 & -0.5 & -0.5
\end{bmatrix}
\begin{bmatrix}
15.264 & 12.3984 & 9.6756 \\
12.3984 & 11.664 & 7.4333 \\
9.6756 & 7.4333 & 11.3644
\end{bmatrix}
\begin{bmatrix}
1 \\
-0.5 \\
-0.5
\end{bmatrix} = 2.664
\end{align*}
\end{Answer}

\begin{Exercise}
Carry out Principal Component Analysis on the dataset above. Find the Principal Directions and the ratio of explained variance for each of them. Reconstruct the data using the first principal component with the largest variance only.
\end{Exercise}
\begin{Answer}
The three principal directions should be approximately $(0.665, \allowbreak 0.561, 0.493)^T$, $(0.259, 0.445, -0.857)^T$, and $(-0.700, 0.698, 0.151)^T$, accounting for $85.9\%/11.9\%/2.2\%$ of the explained variance. For checking, the reconstruction of the first data point should give $(X_1, Y_1, Z_1) = (17.92, 18.43, 21.78)$.
\end{Answer}

\begin{Exercise}
\label{ex:MJO}
Download the ERA5 datasets for \SI{200}{hPa} and \SI{850}{hPa} zonal winds, as well as total column water over some $10$ to $20$ years from \SI{15}{\degree N} to \SI{15}{\degree S} and \SI{60}{\degree E} to \SI{160}{\degree E}. Standardize the three variables via dividing them by their standard deviations respectively, concatenate them, and follow the procedure outlined in Section \ref{section:EOF} to do the so-called \textit{combined EOFs}. Recover the physical patterns through multiplying the entries of the EOF modes back by the standard deviations of the corresponding variables. You should be able to observe the appearance of \textit{Madden-Julian Oscillation (MJO)} from the first two EOFs. 
\end{Exercise}


\chapter{Inner Product Spaces}

\section{Definition of Inner Product Spaces}

\subsection{Requirements of Inner Products}

\subsection{Generalization of Length and Orthogonality via Inner Products}

\subsection{Infinite-dimensional Inner Product Spaces}
\chapter{Least-Square Approximation}
\label{chap:leastsq}

As discussed in Chapter \ref{chap:SolLinSys}, given a linear system, if the number of equations (rows) is greater than the number of unknowns (columns), then it is overdetermined. Generally, it will be inconsistent and there will be no solution. However, we can salvage this by finding an approximated solution such that the so-called \textit{squared error} is minimized. This is known as the \textit{least-square approximation}. Its most common application is \textit{linear regression}, in which we predict a dependent variable using an optimized linear equation in terms of some independent variable(s). For example, in Earth Science, we may regress the frequency of tropical cyclones in a basin against the large-scale vorticity field or wind shear.

\section{Mathematical Ideas of Least-Square Approximation}

For a linear system $A\vec{x} = \vec{h}$ where $\vec{h}$ does not lie in $\mathcal{C}(A)$ the column space of $A$, by Properties \ref{proper:consistentcolspace} it is inconsistent and there will be no exact solution. Nevertheless, a best-fit vector $\vec{x}_f$ can be found, where $A\vec{x}_f = \smash{\vec{h}_f}$, in the sense that $\smash{\vec{h}_f}$ will be the closest vector in the column space of $A$ to $\smash{\vec{h}}$ in terms of distance, i.e.\ the squared error
\begin{align}
\norm{\vec{h}-\vec{h}_f}^2 = \norm{\vec{h}-A\vec{x}_f}^2    
\end{align}
is minimized, and $\smash{\vec{h}_f} = A\vec{x}_f$ (or $\vec{x}_f$) is then referred to as the \index{Least-square Approximation}\keywordhl{least-square approximation} to the system. From Properties \ref{proper:shortestorthoproj}, we know that the shortest distance will be achieved by the orthogonal projection of $\vec{h}$ onto the column space of $A$. Notice that the distance and orthogonality can now be defined with respect to a general inner product other than the usual dot product. Therefore, $\vec{h}-A\vec{x}_f$ will be in the orthogonal complement $\mathcal{C}(A)^\perp$\footnote{which may not be equal to $\mathcal{N}(A^T)$ or $\mathcal{N}(A^*)$ but rather $\mathcal{N}(A^\dag)$ if an inner product other than the standard one is used.} of the column space $\mathcal{C}(A)$ of $A$, and we have
\begin{align}
\langle A\vec{x}, \vec{h}-A\vec{x}_f \rangle = \langle \vec{x}, A^\dag(\vec{h}-A\vec{x}_f) \rangle = 0 
\end{align}
Since this holds for any $\vec{x}$, by the last item of Properties \ref{proper:innerprod2} we have
\begin{align}
A^\dag(\vec{h}-A\vec{x}_f) &= \textbf{0} \nonumber \\
A^\dag A\vec{x}_f &= A^\dag \vec{h}
\end{align}
This is called the \index{Normal Equation}\keywordhl{normal equation} due to the appearance of the $A^\dag A$ factor. So any $\vec{x}_f$\footnote{The existence of at least one of such a vector is guaranteed by the uniqueness of orthogonal projection of $\smash{\vec{h}}$ onto $\mathcal{C}(A)$.} satisfying this equation will produce the least-square error. We may be tempted to multiply both sides of the equation by the inverse $(A^\dag A)^{-1}$ to arrive at a formula for the least-square solution $\vec{x}_f = (A^\dag A)^{-1}A^\dag \smash{\vec{h}}$. However, this is only allowed when this inverse indeed exists, and we now discuss under what condition it will happen.\par
\begin{figure}[ht!]
    \centering
    \begin{tikzpicture}[rotate=0]
    \filldraw[fill=Green!20]
    (0,0,0) -- (4,0,0) -- (4,0,4) -- (0,0,4) -- cycle;
    \draw[red, thick, ->] (1,0,1) -- (3,2,2) node[right]{$\vec{h}$};
    \draw[blue, thick, ->] (1,0,1) -- (3,0,2) node[right]{$A\vec{x_f} = \vec{h_f}$};
    \draw[gray, thick, dashed] (3,2,2) -- (3,0,2);
    \node[Green] at (-2,0,1) {Column space of $A$};
    \end{tikzpicture}
    \caption{\textit{Geometric view for the least-square approximation problem, where $\smash{\vec{h}}$ and the orthogonal projection of $\smash{\vec{h}}$ onto the two-dimensional column space of the $3 \times 2$ matrix $A$, that is, $\smash{\vec{h_f}} = A\vec{x_f}$, are $\mathbb{R}^3$ vectors.}}
\end{figure}

\begin{proper}
For an $m \times n$ matrix $A$, $A^\dag A$ and $A$ have the same null space.
\end{proper}
\begin{proof}
It is obvious that if $\vec{x} \in \mathcal{N}(A)$, $A\vec{x} = \textbf{0}$ then $A^\dag A\vec{x} = \textbf{0}$ so $\vec{x} \in \mathcal{N}(A^\dag A)$ and $\mathcal{N}(A) \subseteq \mathcal{N}(A^\dag A)$. Now we only need to show that $\mathcal{N}(A^\dag A) \subseteq \mathcal{N}(A)$. For $\vec{x} \in \mathcal{N}(A^\dag A)$, we have $A^\dag A\vec{x} = \textbf{0}$ and hence $\langle \vec{x}, A^\dag A\vec{x} \rangle = 0$. Subsequently by the definition of an adjoint (Definition \ref{defn:adjoint}), we have $\langle A\vec{x}, A\vec{x} \rangle = \norm{A\vec{x}}^2 = 0$. By Definition \ref{defn:innerprod}, we conclude that it must be $A\vec{x} = \textbf{0}$ so $\vec{x} \in \mathcal{N}(A)$, and thus $\mathcal{N}(A^\dag A) \subseteq \mathcal{N}(A)$. Hence $\mathcal{N}(A^\dag A) = \mathcal{N}(A)$, the null space of $A^\dag A$ and $A$ coincides.
\end{proof}

\begin{proper}
\label{proper:AdagArank}
For an $m \times n$ matrix $A$, $A^\dag A$ has the same rank as $A$. As a corollary, if $A$ has linearly independent columns such that $\text{rank}(A) = n$, then $A^\dag A$ is invertible.
\end{proper}
\begin{proof}
By the Rank-nullity Theorem \ref{thm:ranknullity}, $\text{rank}(A^\dag A) + \text{nullity}(A^\dag A) = n = \text{rank}(A) + \text{nullity}(A)$ but by the previous properties $\text{nullity}(A^\dag A) = \text{nullity}(A)$ so $\text{rank}(A^\dag A) = \text{rank}(A)$ where $A^\dag A$ is an $n \times n$ matrix. Therefore if $\text{rank}(A^\dag A) = \text{rank}(A) = n$, then $A^\dag A$ is full-rank and by Properties \ref{proper:invertrank} it is invertible.
\end{proof}
Therefore, we have the following result.
\begin{thm}
\label{thm:bestfit}
If $A$ is a $m \times n$ matrix with $m \geq n$ and all its $n$ column vectors are linearly independent, then for the system $A\vec{x} = \smash{\vec{h}}$, there exists a unique best-fit solution
\begin{align}
\vec{x}_f &= (A^\dag A)^{-1}A^\dag \vec{h} \label{eqn:bestfit}
\end{align}
such that the squared error $\norm{\vec{h}-\vec{h_f}}^2 = \norm{\vec{h}-A\vec{x_f}}^2$ is minimized.
\end{thm}
Notice that if $\vec{h}$ already lies in the column space of $A$, then $A\vec{x} = \vec{h}$ will have an exact solution, and the best-fit solution will be identical to this exact solution. However, on the other extreme, if the column vectors in $A$ are not linearly independent, then the best-fit solution will not be unique. Rather, the normal equation will still be consistent, but there are infinitely many possible solutions, each having the same least-square error. Also, if the standard real inner product is used so that $A^\dag = A^T$ and we by chance have the QR decomposition of $A$, then
\begin{align}
\vec{x_f} &= (A^TA)^{-1}A^T\vec{h} \nonumber \\
&= ((QR)^T(QR))^{-1}(QR)^T\vec{h} \nonumber \\
&= (R^TQ^TQR)^{-1} (QR)^T\vec{h} \nonumber \\
&= (R^TR)^{-1} R^TQ^T \vec{h} \nonumber \\
&= R^{-1} (R^T)^{-1} R^TQ^T \vec{h} \nonumber \\
&= R^{-1} Q^T\vec{h}
\end{align}
where $Q^TQ = I$ since $Q$ is an orthogonal matrix in which the column vectors form an orthonormal basis as indicated by Properties \ref{proper:QRdecompose}.

\begin{exmp}
Find the least-square solution to the overdetermined linear system
\begin{align*}
\begin{bmatrix}
1 & 2 \\
3 & 4 \\
1 & 4
\end{bmatrix}
\begin{bmatrix}
x \\
y
\end{bmatrix}
=
\begin{bmatrix}
3 \\
8 \\
7
\end{bmatrix}
\end{align*}
where the error is calculated with respect to the usual Euclidean distance, and also relative to the inner product as defined in Example \ref{exmp:R3innerGS}.
\end{exmp}
\begin{solution}
If the standard inner product is used for defining lengths, then the least-square solution in Theorem \ref{thm:bestfit} is reduced to
\begin{align*}
\vec{x}_f &= (A^TA)^{-1}A^T\vec{h} \\
&= 
\left(\begin{bmatrix}
1 & 2 \\
3 & 4 \\
1 & 4
\end{bmatrix}^T
\begin{bmatrix}
1 & 2 \\
3 & 4 \\
1 & 4
\end{bmatrix}\right)^{-1}
\begin{bmatrix}
1 & 2 \\
3 & 4 \\
1 & 4
\end{bmatrix}^T
\begin{bmatrix}
3 \\
8 \\
7
\end{bmatrix} \\
&=
\begin{bmatrix}
11 & 18 \\
18 & 36 
\end{bmatrix}^{-1}
\begin{bmatrix}
1 & 3 & 1\\
2 & 4 & 4
\end{bmatrix}
\begin{bmatrix}
3 \\
8 \\
7
\end{bmatrix} \\
&=
\begin{bmatrix}
\frac{1}{2}&-\frac{1}{4}\\
-\frac{1}{4}&\frac{11}{72}
\end{bmatrix}
\begin{bmatrix}
1 & 3 & 1\\
2 & 4 & 4
\end{bmatrix}
\begin{bmatrix}
3 \\
8 \\
7
\end{bmatrix}
=
\begin{bmatrix}
\frac{1}{2} \\
\frac{19}{12}
\end{bmatrix}
\end{align*}
and the least-square error is
\begin{align*}
\norm{\vec{h}-A\vec{x_f}}^2 &=
\norm{\begin{bmatrix}
3 \\
8 \\
7
\end{bmatrix}-
\begin{bmatrix}
1 & 2 \\
3 & 4 \\
1 & 4
\end{bmatrix}
\begin{bmatrix}
\frac{1}{2} \\
\frac{19}{12}
\end{bmatrix}
}^2 \\
&= 
\norm{\begin{bmatrix}
3 \\
8 \\
7
\end{bmatrix}-
\begin{bmatrix}
\frac{11}{3}\\
\frac{47}{6}\\ 
\frac{41}{6}
\end{bmatrix}
}^2 \\
&= 
\norm{\begin{bmatrix}-\frac{2}{3}\\ 
\frac{1}{6}\\ 
\frac{1}{6}\end{bmatrix}}^2 = \left(-\frac{2}{3}, \frac{1}{6}, \frac{1}{6}\right)^T \cdot \left(-\frac{2}{3}, \frac{1}{6}, \frac{1}{6}\right)^T = \frac{1}{2}
\end{align*}
For the inner product in Example \ref{exmp:R3innerGS}, an appropriate adjoint, adapted from Section \ref{section:adjointdef}, to be used in this situation is
\begin{align*}
A^\dag &= C^{-1} A^T B \\
&=
\begin{bmatrix}
1 & 0 \\
0 & 1
\end{bmatrix}^{-1}
\begin{bmatrix}
1 & 3 & 1\\
2 & 4 & 4
\end{bmatrix}
\begin{bmatrix}
2&1&0\\ 
1&2&1\\
0&1&2
\end{bmatrix} \\
&=
\begin{bmatrix}
5&8&5\\ 
8&14&12
\end{bmatrix}
\end{align*}
where $C$ can be picked to be any positive-definite matrix\footnote{The choice of $C$ actually matters if there are multiple least-square solutions and $\vec{x}_f$ is also required to have a minimal norm with respect to some other inner product. This will be explored in Section \ref{section:SVDgeneral}.} and we choose $C = I$ for simplicity. Subsequently, the least-square solution is
\begin{align*}
\vec{x}_f &= (A^\dag A)^{-1}A^\dag \vec{h} \\
&=
\left(\begin{bmatrix}
5&8&5\\ 
8&14&12
\end{bmatrix}
\begin{bmatrix}
1 & 2 \\
3 & 4 \\
1 & 4
\end{bmatrix}\right)^{-1}
\begin{bmatrix}
5&8&5\\ 
8&14&12
\end{bmatrix}
\begin{bmatrix}
3 \\
8 \\
7
\end{bmatrix} \\
&=
\begin{bmatrix}
34&62\\ 
62&120
\end{bmatrix}^{-1}
\begin{bmatrix}
5&8&5\\ 
8&14&12
\end{bmatrix}
\begin{bmatrix}
3 \\
8 \\
7
\end{bmatrix} \\
&= 
\begin{bmatrix}
\frac{30}{59}&-\frac{31}{118}\\ 
-\frac{31}{118}&\frac{17}{118}
\end{bmatrix}
\begin{bmatrix}
5&8&5\\ 
8&14&12
\end{bmatrix}
\begin{bmatrix}
3 \\
8 \\
7
\end{bmatrix}
=
\begin{bmatrix}
\frac{10}{59} \\
\frac{103}{59}
\end{bmatrix}
\end{align*}
with the least-square error being
\begin{align*}
\norm{\vec{h}-A\vec{x_f}}^2 &=
(\textbf{h}-A\textbf{x}_f)^T B (\textbf{h}-A\textbf{x}_f)\\
&=
\left(\begin{bmatrix}
3 \\
8 \\
7
\end{bmatrix}-
\begin{bmatrix}
1 & 2 \\
3 & 4 \\
1 & 4
\end{bmatrix}
\begin{bmatrix}
\frac{10}{59} \\
\frac{103}{59}
\end{bmatrix}\right)^T
\begin{bmatrix}
2&1&0\\ 
1&2&1\\
0&1&2
\end{bmatrix}
\left(\begin{bmatrix}
3 \\
8 \\
7
\end{bmatrix}-
\begin{bmatrix}
1 & 2 \\
3 & 4 \\
1 & 4
\end{bmatrix}
\begin{bmatrix}
\frac{10}{59} \\
\frac{103}{59}
\end{bmatrix}\right) \\
&=
\begin{bmatrix}
-\frac{39}{59}\\ 
\frac{30}{59}\\ 
-\frac{9}{59}
\end{bmatrix}^T
\begin{bmatrix}
2&1&0\\ 
1&2&1\\
0&1&2
\end{bmatrix}
\begin{bmatrix}
-\frac{39}{59}\\ 
\frac{30}{59}\\ 
-\frac{9}{59}
\end{bmatrix}
= \frac{36}{59}
\end{align*}
\end{solution}
We close this section by confirming that the matrix $T = A(A^\dag A)^{-1}A^\dag$ as in $\vec{h}_f = A\vec{x}_f = A(A^\dag A)^{-1}A^\dag \vec{h}$ indeed represents an orthogonal projection (of $\vec{h}$ onto the column space of $A$). By Properties \ref{proper:orthoprojadjoint}, we just need to check if $T^2 = T = T^\dag$. For the first equality, we have
\begin{align*}
T^2 &= (A(A^\dag A)^{-1}A^\dag)(A(A^\dag A)^{-1}A^\dag) \\
&= A(A^\dag A)^{-1}(A^\dag A)(A^\dag A)^{-1}A^\dag \\
&= A(A^\dag A)^{-1}(I)A^\dag = A(A^\dag A)^{-1}A^\dag = T
\end{align*}
and for the second equality, we can use Properties \ref{proper:adjoints} to get
\begin{align*}
T^\dag &= (A(A^\dag A)^{-1}A^\dag)^\dag \\
&= (A^\dag)^\dag((A^\dag A)^{-1})^\dag A^\dag \\
&= A ((A^\dag A)^\dag)^{-1} A^\dag \\
&= A (A^\dag A)^{-1} A^\dag = T
\end{align*}

\section{Linear Regression}
\subsection{Linear Regression for One Predictor Variable}

\index{Linear Regression}\keywordhl{Linear regression} is a very important tool in Statistics that helps identify any linear trend in data and is based on least-square approximation. The simplest type of linear regression is fitting a straight line $y = \alpha + \beta x$, where $\alpha$ and $\beta$ are its intercept and slope, to $m$ pairs of observation, $(x_1, y_1), (x_2, y_2), \cdots, (x_m, y_m)$ such that the sum of squared errors $\sum_{k=1}^m (y_k - (\alpha + \beta x_k))^2$ is minimized. In this context, $x$ and $y$ are known as the \index{Explanatory Variable}\textit{explanatory} and \index{Response Variable}\textit{response variable} respectively.\par

\begin{figure}[ht!]
\centering
\begin{tikzpicture}
\draw[thick, ->] (-1,0) -- (5,0) node[right]{$x$};
\draw[thick, ->] (0,-1) -- (0,5) node[above]{$y$};
\node[circle, inner sep=1pt, fill=blue] at (1,1.2) {};
\node at (0.8,1.6) {$(1,1.2)$};
\node[circle, inner sep=1pt, fill=blue] at (2,1.6) {};
\node at (2.2,1.2) {$(2,1.6)$};
\node[circle, inner sep=1pt, fill=blue] at (3,2.8) {};
\node at (3.2,2.4) {$(3,2.8)$};
\node[circle, inner sep=1pt, fill=blue] at (4,4.4) {};
\node at (3.8,4.8) {$(4,4.4)$};
\draw[thick, red] (-0.5,-0.74) -- (5,5.2) node[right]{$y = -0.2+1.08x$};
\node[below left]{$O$}; 
\draw[thick, Green] (1,1.2) -- (1,0.88);
\draw[thick, Green] (2,1.6) -- (2,1.96);
\draw[thick, Green] (3,2.8) -- (3,3.04);
\draw[thick, Green] (4,4.4) -- (4,4.12);
\end{tikzpicture}
\caption{\textit{A toy linear regression model for 4 data points. The red straight line represents the best linear fit, and the green lines are the distances, or errors between the actual data and the regression line, whose sum of squares is minimized.}}
\end{figure}

To see how we can apply the results in the last section, we first rewrite the system into matrix form. The actual sampled values are given by $\vec{y} = (y_1, y_2, \ldots, y_m)^T$, while the values predicted by the best-fit straight line will be in the form of $\vec{y}^{\text{pred}} = (\alpha\textbf{1} + \beta \vec{x})^T = (\alpha + \beta x_1, \alpha + \beta x_2, \ldots, \alpha + \beta x_m)^T$, where $\textbf{1}$ is a column vector filled with ones, $\alpha$ and $\beta$ are the intercept and slope of the best-fit line to be determined. In other words, we are trying to find an optimal solution for the unknown parameters $\alpha$ and $\beta$ in the matrix system
\begin{align}
\alpha\textbf{1} + \beta \vec{x} &= \vec{y}
\end{align}
or alternatively
\begin{align}
\begin{bmatrix}
1 & x_1 \\
1 & x_2 \\
\vdots & \vdots \\
1 & x_m
\end{bmatrix}
\begin{bmatrix}
\alpha \\
\beta
\end{bmatrix}
=
\begin{bmatrix}
y_1 \\
y_2 \\
\vdots \\
y_m
\end{bmatrix}
\end{align}
so that the squared error $\smash{\sum_{k=1}^m (y_k - y^{\text{pred}}_k)^2} = \sum_{k=1}^m (y_k - (\alpha + \beta x_k))^2$ is minimized. Such a system is conveniently denoted as $[X]\vec{\beta} = \vec{y}$, where the first and second column of $[X] = [\textbf{1}|\vec{x}]$ represent the two predictors, the constant term which is a "hidden" predictor that gives rise to the $y$-intercept, and the linear term of $x$. Now, the sum of squared errors
\begin{align}
\sum_{k=1}^m (y_k - (\alpha + \beta x_k))^2 = \norm{\vec{y} - [X]\vec{\beta}}^2    
\end{align}
will be minimized by the best-fit parameters which are computed via
\begin{align}
\vec{\beta_f} = ([X]^T[X])^{-1}[X]^T \vec{y}
\end{align}
according to Theorem \ref{thm:bestfit} where $A = [X]$, and simply $A^\dag = [X]^T$ as the error is computed based on the usual dot product/Euclidean distance. Writing out the matrices in the formula, the parameters for single variable linear regression are
\begin{align}
\begin{bmatrix}
\alpha_f \\
\beta_f
\end{bmatrix}
&=
\left(
\begin{bmatrix}
1 & 1 & \cdots & 1 \\
x_1 & x_2 & \cdots & x_m
\end{bmatrix}
\begin{bmatrix}
1 & x_1 \\
1 & x_2 \\
\vdots & \vdots \\
1 & x_m
\end{bmatrix} 
\right)^{-1}
\begin{bmatrix}
1 & 1 & \cdots & 1 \\
x_1 & x_2 & \cdots & x_m
\end{bmatrix}
\begin{bmatrix}
y_1 \\
y_2 \\
\vdots \\
y_m
\end{bmatrix} \nonumber \\
&= 
\begin{bmatrix}
m & \sum X \\
\sum X & \sum X^2
\end{bmatrix}^{-1}
\begin{bmatrix}
\sum Y \\
\sum XY
\end{bmatrix} \nonumber  \\
&=
\frac{1}{m(\sum X^2) - (\sum X)^2}
\begin{bmatrix}
\sum X^2 & -\sum X \\
-\sum X & m
\end{bmatrix}
\begin{bmatrix}
\sum Y \\
\sum XY
\end{bmatrix} \nonumber \\
&=
\frac{1}{m(\sum X^2) - (\sum X)^2}
\begin{bmatrix}
(\sum Y) (\sum X^2) - (\sum X) (\sum XY) \\
m (\sum XY) - (\sum X) (\sum Y)
\end{bmatrix}
\end{align}
where 
\begin{subequations}
\begin{align}
\sum X &= x_1 + x_2 + \cdots + x_m \\
\sum X^2 &= x_1^2 + x_2^2 + \cdots + x_m^2 \\
\sum Y &= y_1 + y_2 + \cdots + y_m \\
\sum XY &= x_1y_1 + x_2y_2 + \cdots + x_my_m   
\end{align}
\end{subequations}
and we have used the results (\ref{eqn:2times2inv}) in Example \ref{exmp:2x2} to calculate the inverse from the second to third line.
\begin{proper}
\label{proper:bestfit2}
The best-fit parameters for the single predictor variable linear regression are found by
\begin{subequations}
\label{eqn:bestfit2}
\begin{align}
\alpha_f &= \frac{(\sum Y) (\sum X^2) - (\sum X) (\sum XY)}{m(\sum X^2) - (\sum X)^2} \\
\beta_f &= \frac{m (\sum XY) - (\sum X) (\sum Y)}{m(\sum X^2) - (\sum X)^2}
\end{align}    
\end{subequations}
such that the regression line $y = \alpha_f + \beta_f x$ achieves the least-square error, i.e.\ the value of $\sum_{k=1}^m (y_k - (\alpha_f + \beta_f x_k))^2$ is the smallest possible.
\end{proper}
\begin{exmp}
\label{ex11.1.1}
Find the best linear fit for five $(x,y)$ data points, which are $(2,4)$, $(3,6)$, $(4,7)$, $(5,9)$, $(7,11)$.
\end{exmp}
\begin{solution}
We first compute the following quantities.
\begin{align*}
\sum X &= 2+3+4+5+7 = 21 \\
\sum(X^2) &= 2^2+3^2+4^2+5^2+7^2 = 103 \\
\sum Y &= 4+6+7+9+11 = 37 \\
\sum (XY) &= (2)(4)+(3)(6)+(4)(7)+(5)(9)+(7)(11) = 176
\end{align*}
By Formula (\ref{eqn:bestfit2}) in Properties \ref{proper:bestfit2}, with $m=5$, the required parameters are
\begin{align*}
\alpha_f &= \frac{\sum Y \sum (X^2) - \sum X \sum (XY)}{m\sum(X^2) - (\sum X)^2} = \frac{(37)(103)-(21)(176)}{(5)(103)-(21)^2} \approx 1.554 \\
\beta_f &= \frac{m \sum(XY) - (\sum X) (\sum Y)}{m\sum(X^2) - (\sum X)^2} = \frac{(5)(176)-(21)(37)}{(5)(103)-(21)^2} \approx 1.392
\end{align*}
So the best linear fit is around $y = 1.554 + 1.392x$.
\end{solution}

\subsection{Linear Regression for Multiple Predictor Variables}

Sometimes we may need to predict a variable $y$ with multiple predictor variables $x^{(1)}, x^{(2)}, \ldots, x^{(n)}$ as there can be different factors contributing to a phenomenon in any Earth Science scenario. Linear regression for multiple predictor variables will then produce a best-fit equation in the form of $y = \alpha + \beta_1x^{(1)} + \beta_2x^{(2)} + \cdots + \beta_nx^{(n)}$. Extending the earlier results from Theorem \ref{thm:bestfit}, the computation of the parameters utilizes the same formula
\begin{align*}
\vec{\beta}_f = ([X]^T[X])^{-1}[X]^T \vec{y}
\end{align*}
where in this case the quantities now involve more predictor variables, $\smash{\vec{\beta}_f^T} = \smash{(\alpha_f, {\beta_1}_f, {\beta_2}_f, \cdots, {\beta_n}_f)^T}$, and the matrix $[X] = [\textbf{1}|\vec{x}^{(1)}|\vec{x}^{(2)}|\cdots|\vec{x}^{(n)}]$ holds the observed values of those multiple predictor variables column by column.

\begin{exmp}
\label{exmp:GPAregress}
The university carried out an experiment and tried to quantify the effects of IQ and studying time on the GPA of students. 5 students participated and the data are shown below.
\begin{center}
\begin{tabular}{|c|c|c|c|}
\hline
Students & IQ & Studying Time per Week (Hours) & GPA \\
\hline
Lily & $104$ & $5.2$ & $3.56$ \\
\hline
Christy & $111$ & $6.1$ & $3.71$ \\
\hline
Sabrina & $107$ & $8.3$ & $3.73$ \\
\hline
Julia & $106$ & $3.4$ & $3.34$ \\
\hline
Emily & $109$ & $9.6$ & $3.88$ \\
\hline
\end{tabular}
\end{center} 
Do a linear regression on the students' GPA against their IQ and studying time.
\end{exmp}
\begin{solution}
Denote the IQ and studying hours of the students by the variables $x^{(1)}$ and $x^{(2)}$, and $y$ for their GPA, we want to fit a linear relationship $y = \alpha + \beta_1 x^{(1)} + \beta_2 x^{(2)}$. The matrix $[X] = [\textbf{1}|\vec{x}^{(1)}|\vec{x}^{(2)}] $, consisting of the observed predictors, is then
\begin{align*}
[X] = \left[
\begin{array}{ccc}
1 & 104 & 5.2 \\
1 & 111 & 6.1 \\
1 & 107 & 8.3 \\
1 & 106 & 3.4 \\
1 & 109 & 9.6
\end{array}
\right]
\end{align*}
The first column represents the intercept term. Subsequently, the fit is derived by
\begin{align*}
\vec{\beta}_f &= ([X]^T [X])^{-1} [X]^T \vec{y} \\
&= \left(\left[
\begin{array}{ccc}
1 & 104 & 5.2 \\
1 & 111 & 6.1 \\
1 & 107 & 8.3 \\
1 & 106 & 3.4 \\
1 & 109 & 9.6
\end{array}
\right]^T
\left[
\begin{array}{ccc}
1 & 104 & 5.2 \\
1 & 111 & 6.1 \\
1 & 107 & 8.3 \\
1 & 106 & 3.4 \\
1 & 109 & 9.6
\end{array}
\right]\right)^{-1}
\left[
\begin{array}{ccc}
1 & 104 & 5.2 \\
1 & 111 & 6.1 \\
1 & 107 & 8.3 \\
1 & 106 & 3.4 \\
1 & 109 & 9.6
\end{array}
\right]^T
\left[
\begin{array}{c}
3.56 \\
3.71 \\
3.73 \\
3.34 \\
3.88
\end{array}
\right] \\
&\approx
\left[
\begin{array}{c}
1.4463 \\
0.0162 \\
0.0710 \\
\end{array}
\right] 
\end{align*}
So the regression model is $\text{GPA} = 1.4463 + 0.0162(\text{IQ}) + 0.0710(\text{Studying Hrs})$. 
\end{solution}

\begin{exmp}
Find a quadratic fit for Example \ref{ex11.1.1}.
\end{exmp}
\begin{solution}
The predictor variables are the constant term, the linear term $x$, as well as the newly added quadratic term $x^2$. The desired parameters are
\begin{align*}
\vec{\beta_f} &= ([X]^T[X])^{-1}[X]^T \vec{y} \\
&=
\left(
\begin{bmatrix}
1 & 1 & 1 & 1 & 1 \\
2 & 3 & 4 & 5 & 7 \\
2^2 & 3^2 & 4^2 & 5^2 & 7^2
\end{bmatrix}
\begin{bmatrix}
1 & 2 & 2^2 \\
1 & 3 & 3^2 \\
1 & 4 & 4^2 \\
1 & 5 & 5^2 \\
1 & 7 & 7^2 
\end{bmatrix}
\right)^{-1}
\begin{bmatrix}
1 & 1 & 1 & 1 & 1 \\
2 & 3 & 4 & 5 & 7 \\
2^2 & 3^2 & 4^2 & 5^2 & 7^2
\end{bmatrix}
\begin{bmatrix}
4 \\
6 \\
7 \\
9 \\
11
\end{bmatrix} \\
& \approx
\begin{bmatrix}
-0.009 \\
2.201 \\
-0.089 
\end{bmatrix}
\end{align*}
The quadratic fit is thus $y = -0.009 + 2.201x - 0.089x^2$.
\end{solution}
Notice that in the above example, although we use the quadratic term $x^2$ as a predictor, the regression is still \textit{linear} in the sense that the regression model is a \textit{linear} combination of these predictors. Usually, fitting a degree $n$ polynomials in a single variable $x$ to $m$ points, $n < m$, will involve a matrix in the form like the one above:
\begin{align}
[X]
&= 
\begin{bmatrix}
1 & x_1 & x_1^2 & \cdots & x_1^n \\
1 & x_2 & x_2^2 & \cdots & x_2^n \\
\vdots & \vdots & \vdots & & \vdots \\
1 & x_m & x_m^2 & \cdots & x_m^n \\
\end{bmatrix} 
\end{align}
This class of matrices is called \index{Vandermonde Matrix}\keywordhl{Vandermonde matrices}.\footnote{As long as $n < m$ and the sampled points $x_i$ are distinct, the columns of a Vandermonde matrix will be linearly independent.}

\subsection{Properties of Linear Regression}

Before going to the next topic, we derive some features of linear regression. The errors, or called the \index{Residual}\keywordhl{residuals} $e_i = h_i - (h_f)_i$, have a mean of zero. Using matrix notation, it means that the sum of components in the error vector $\vec{e} = \vec{h} - \vec{h_f}$ is zero. This can be seen from the very beginning of our derivation for the best-fit problem $A\vec{x} = \vec{h}$, where the condition has been $A^\dag (\vec{h} - A\vec{x}_f) = A^\dag (\vec{h} - \vec{h}_f) = \textbf{0}$. For linear regression, $A^\dag =[X]^T$ and it becomes $[X]^T\vec{e} = \textbf{0}$. However, since the first column in $[X]$ is $\textbf{1}$, the first entry in $[X]^T\vec{e}$ is just $\textbf{1} \cdot \vec{e}$, which is just the sum of errors. As $[X]^T\vec{e} = \textbf{0}$, the sum and hence the mean of errors must be zero.\par

Another property is that, if we denote the mean of $y$ as $\overline{y}$, each actual data and predicted values as $y_i$ and $\hat{y}_i = y^{\text{pred}}_i$ respectively, then
\begin{align}
\sum_i (y_i - \overline{y})^2 &= \sum_i (y_i - \hat{y}_i)^2 + \sum_i (\hat{y}_i - \overline{y})^2
\end{align}
The term on L.H.S. is called the \index{Sum of Squares Total (SST)}\keywordhl{SST/Sum of Squares Total}, while the two terms on R.H.S. are known as the \index{Sum of Squares Error (SSE)}\keywordhl{SSE/Sum of Squares Error} and \index{Sum of Squares Regression (SSR)}\keywordhl{SSR/Sum of Squares Regression}. To prove this, we expand the SST, which gives
\begin{align}
\text{SST} &= \sum_i (y_i - \overline{y})^2 \nonumber \\
&= \sum_i ((y_i - \hat{y}_i) + (\hat{y}_i - \overline{y}))^2 \nonumber \\
&= \sum_i (y_i - \hat{y}_i)^2 + \sum_i (\hat{y}_i - \overline{y})^2 + 2\sum_i ((y_i - \hat{y}_i) (\hat{y}_i - \overline{y})) \nonumber \\
&= \text{SSE} + \text{SSR} + 2\sum_i ((y_i - \hat{y}_i) (\hat{y}_i - \overline{y}))
\end{align}
The remaining step is to show that the last term equals zero. For simplicity, we work with the case of a single predictor variable, so there are two parameters $\alpha$ and $\beta$ only, but it can be easily generalized to multiple predictors and $\beta_j$. Expanding the product gives
\begin{align}
\sum_i ((y_i - \hat{y}_i) (\hat{y}_i - \overline{y})) &= \sum_i (\hat{y}_i(y_i - \hat{y}_i)) - \overline{y} \sum_i (y_i - \hat{y}_i) \nonumber \\
&= \sum_i ((\alpha + \beta x_i)(y_i - \hat{y}_i)) - \overline{y} \sum_i (y_i - \hat{y}_i) \nonumber \\
&= \beta \sum_i (x_i(y_i - \hat{y}_i)) - (\overline{y} - \alpha) \sum_i (y_i - \hat{y}_i) \nonumber \\
&= \beta \sum_i (x_i e_i) - (\overline{y} - \alpha) \sum_i e_i
\end{align}
Using the same logic when we are investigating $[X]^T\vec{e} = \textbf{0}$ before, we know that
\begin{subequations}
\begin{align}
\textbf{1} \cdot \vec{e} &= \sum_i e_i = 0 \\
\vec{x} \cdot \vec{e} &= \sum_i (x_i e_i) = 0
\end{align}    
\end{subequations}
These two relations can also be derived by setting $\partial (\sum_i e_i^2)/\partial \alpha = 0$ and $\partial (\sum_i e_i^2)/\partial \beta = 0$ as the sum of squared errors reaches the minimum at the point of best-fit. Substituting this two equations readily implies that the concerned quantity $\sum_i ((y_i - \hat{y}_i) (\hat{y}_i - \overline{y})) = 0$ is zero, and thus $\text{SST} = \text{SSE} + \text{SSR}$. We repeat these two results as follows.
\begin{proper}
The mean error of any linear regression is zero. Moreover, its sum of squares total (SST) is equal to the sum of squares error (SSE) plus the sum of squares regression (SSR), i.e.\ 
\begin{subequations}
\begin{align}
\sum_i (y_i - \overline{y})^2 &= \sum_i (y_i - \hat{y}_i)^2 + \sum_i (\hat{y}_i - \overline{y})^2 \\
\text{SST} &= \text{SSE} + \text{SSR}   
\end{align}    
\end{subequations}
\end{proper}
The ratio of SSR to SST is known as the \index{Coefficient of Determination}\keywordhl{coefficient of determination} and is denoted by $R^2$ (hence sometimes we also simply call it \index{$R$-squared}\keywordhl{$R$-squared}). It is the proportion of variance in the response variable that can be explained by the predictors and thus indicates how good the linear fit is. It ranges from $0$ to $1$. The larger the value of $R^2$, the better the linear regression at predicting the target variable. If $R^2$ is close to one, then the fit is almost perfect. However, it is cautioned that there may be the problem of \textit{overfitting}\footnote{For $n$ distinct sets of data, it is always possible to achieve a perfect fit ($R^2 = 1$) using $n-1$ predictors (plus the intercept term) in the linear regression even when they may not be really related to the predictand and this is an extreme case of overfitting.}, which means that the regression does well on the dataset it is trained on but fails horribly when extrapolated to data points outside this training set. On the other hand, if $R^2$ is low, then the linear regression is ineffective, but it simply means that there is no apparent linear trend, and the possibility of other forms of relationship, like a logarithmic one, existing between the explanatory variables and response variable, cannot be ruled out.

\begin{exmp}
\label{exmp:GPAregress2}
Find the $R$-squared for the linear regression in Example \ref{exmp:GPAregress}.
\end{exmp}
\begin{solution}
We compute the SST and SSR as follows.
\begin{align*}
\overline{y} &= \frac{1}{5}(3.56 + 3.71 + 3.73 + 3.34 + 3.88) \\
&= 3.644 \\
\text{SST} &= \sum_i (y_i - \overline{y})^2 \\
&= (3.56 - 3.644)^2 + (3.71 - 3.644)^2 + (3.73 - 3.644)^2 \\
&\quad + (3.34 - 3.644)^2 + (3.88 - 3.644)^2 \\
&\approx 0.167 \\
\text{SSR} &= \sum_i (\hat{y}_i - \overline{y})^2 \\
&= ((1.4463 + 0.0162(104)+ 0.0710(5.2)) - 3.644)^2 \\
&\quad + ((1.4463 + 0.0162(111)+ 0.0710(6.1)) - 3.644)^2 \\
&\quad + ((1.4463 + 0.0162(107)+ 0.0710(8.3)) - 3.644)^2 \\
&\quad + ((1.4463 + 0.0162(106)+ 0.0710(3.4)) - 3.644)^2 \\
&\quad + ((1.4463 + 0.0162(109)+ 0.0710(9.6)) - 3.644)^2 \\
&= (3.500 - 3.644)^2 + (3.678 - 3.644)^2 + (3.769 - 3.644)^2 \\
&\quad + (3.405 - 3.644)^2 + (3.894 - 3.644)^2 \\
&\approx 0.157
\end{align*}
Hence $R^2 \approx 0.157/0.167 = 0.94$ which indicates a good fit.
\end{solution}

\section{Earth Science Applications}

\begin{exmp}
Find the least-square approximation to the temperature measurement problem in Example \ref{exmp:weatherstats}.
\end{exmp}
\begin{solution}
In Example \ref{exmp:weatherstats2}, we have found that the corresponding overdetermined system has no exact solution. To find the least-square solution, we invoke the formula in Theorem \ref{thm:bestfit} where
\begin{align*}
\vec{x}_f &= (A^TA)^{-1}A^T\vec{h}
\end{align*}
with 
\begin{align*}
A &= 
\begin{bmatrix}
10 & 20 \\
25 & 15 \\
-10 & 5
\end{bmatrix}
& \text{and} &
& \vec{h} &= \begin{bmatrix}
0.2 \\
0.3 \\
-0.2
\end{bmatrix}
\end{align*}
So
\begin{align*}
\vec{x}_f &= 
\left(\begin{bmatrix}
10 & 20 \\
25 & 15 \\
-10 & 5
\end{bmatrix}^T
\begin{bmatrix}
10 & 20 \\
25 & 15 \\
-10 & 5
\end{bmatrix}\right)^{-1}
\begin{bmatrix}
10 & 20 \\
25 & 15 \\
-10 & 5
\end{bmatrix}^T
\begin{bmatrix}
0.2 \\
0.3 \\
-0.2
\end{bmatrix} \\
&\approx 
\begin{bmatrix}
0.01357 \\
0.00058
\end{bmatrix}
\end{align*}
So the approximated solution of the temperature gradients that minimizes the squared errors is $\partial T/\partial x = \SI{1.36e-2}{\degree C \per \km} $ and $\partial T/\partial y = \SI{0.58e-3}{\degree C \per \km}$.
\end{solution}

\section{Python Programming}

We will use Example \ref{exmp:GPAregress} to demonstrate how to do a linear regression in Python. The \verb|statsmodels| module is recommended for this. First, let's provide the data.
\begin{lstlisting}
import numpy as np
import statsmodels.api as sm

X = np.array([[104, 5.2],
              [111, 6.1],
              [107, 8.3],
              [106, 3.4],
              [109, 9.6]]) 
Y = np.array([3.56, 3.71, 3.73, 3.34, 3.88])
\end{lstlisting}
We need to append the constant term by using the \verb|sm.add_constant| function.
\begin{lstlisting}
Xc = sm.add_constant(X, prepend=False)
\end{lstlisting}
Subsequently, create an \verb|sm.OLS| (stands for "Ordinary Least Squares") object and call the \verb|fit| method that takes the predictand/predictors in the first/second argument.
\begin{lstlisting}
mod = sm.OLS(Y, Xc)
res = mod.fit()
\end{lstlisting}
The \verb|summary| method will output a table that provides relevant parameters about the linear regression like the $R$-squared and p-values.
\begin{lstlisting}
print(res.summary())
\end{lstlisting}
and the \verb|predict| method will return the predicted values from an input dataset according to the linear regression.
\begin{lstlisting}
print(res.predict(Xc))    
\end{lstlisting}
This gives \verb|[3.495 3.672 3.764 3.400 3.888]| which is slightly different from what we have obtained in Example \ref{exmp:GPAregress2} as our manual calculation of the regression coefficients inevitably contains more round-off errors.
\begin{figure}[ht!]
    \centering
    \includegraphics[width=0.99\textwidth]{graphics/OLS_table.png}
    \caption{\textit{The output of the Python program that calculates the OLS for Example \ref{exmp:GPAregress}.}}
\end{figure}

\section{Exercise}

\begin{Exercise}
Show that if $A$ is an invertible $n \times n$ square matrix then the least-square solution of $A\vec{x} = \vec{h}$ given in Theorem \ref{thm:bestfit}
\begin{align*}
\vec{x}_f &= (A^\dag A)^{-1}A^\dag \vec{h}    
\end{align*}
will be reduced to simply
\begin{align*}
\vec{x}_f &= A^{-1}\vec{h}    
\end{align*}
the exact solution as derived in Section \ref{subsection:SolLinSysInv}.
\end{Exercise}
\begin{Answer}
If $A$ (and hence $A^\dag$) is invertible, then
\begin{align*}
\vec{x}_f &= (A^\dag A)^{-1}A^\dag \vec{h} \\
&= A^{-1}(A^\dag)^{-1}A^\dag \vec{h} = A^{-1}\vec{h}  
\end{align*}
\end{Answer}

\begin{Exercise}
Find the least-square solution to the overdetermined linear system below.
\begin{align*}
\left\{\begin{alignedat}{2}
&x + 2y - 3z &&= 4 \\
&x - y + 3z &&= 5 \\
&2x + y - z &&= 1 \\
&x + y + z &&= 2
\end{alignedat}\right.
\end{align*}
\end{Exercise}
\begin{Answer}
\begin{align*}
\vec{x}_f &= (A^TA)^{-1}A^T\vec{h} \\
&= 
\left(\begin{bmatrix}
1 & 2 & -3 \\
1 & -1 & 3 \\
2 & 1 & -1 \\
1 & 1 & 1
\end{bmatrix}^T
\begin{bmatrix}
1 & 2 & -3 \\
1 & -1 & 3 \\
2 & 1 & -1 \\
1 & 1 & 1
\end{bmatrix}\right)^{-1}
\begin{bmatrix}
1 & 2 & -3 \\
1 & -1 & 3 \\
2 & 1 & -1 \\
1 & 1 & 1
\end{bmatrix}^T
\begin{bmatrix}
4 \\
5 \\
1 \\
2
\end{bmatrix} \\
&=
\begin{bmatrix}
7&4&-1\\ 
4&7&-9\\ 
-1&-9&20
\end{bmatrix}^{-1}
\begin{bmatrix}
1 & 1 & 2 & 1 \\
2 & -1 & 1 & 1 \\
-3 & 3 & -1 & 1
\end{bmatrix}
\begin{bmatrix}
4 \\
5 \\
1 \\
2
\end{bmatrix} \\
&=
\begin{bmatrix}
\frac{59}{158}&-\frac{71}{158}&-\frac{29}{158}\\
-\frac{71}{158}&\frac{139}{158}&\frac{59}{158}\\ 
-\frac{29}{158}&\frac{59}{158}&\frac{33}{158}
\end{bmatrix}
\begin{bmatrix}
1 & 1 & 2 & 1 \\
2 & -1 & 1 & 1 \\
-3 & 3 & -1 & 1
\end{bmatrix}
\begin{bmatrix}
4 \\
5 \\
1 \\
2
\end{bmatrix} 
=
\begin{bmatrix}
\frac{225}{158}\\ 
\frac{147}{158}\\ 
\frac{109}{158}
\end{bmatrix}
\end{align*}    
\end{Answer}

\begin{Exercise}
Find a linear fit for the following data about sea level pressure and temperature measured at a weather station.
\begin{center}
\begin{tabular}{|c|c|c|c|c|c|c|}
\hline
Temperature ($^\circ$ C) & $10$ & $12$ & $12$ & $13$ & $16$ & $17$\\
\hline
Pressure (hPa) & $1022.1$ & $1019.5$ & $1018.9$ & $1017.6$ & $1014.3$ & $1013.5$\\
\hline
\end{tabular}
\end{center}
\end{Exercise}
\begin{Answer}
We have $m=6$, $\sum X=80$, $\sum (X^2)=1102$, $\sum (XY)=81368.9$, $\sum Y=6105.9$. So
\begin{align*}
\alpha_f &= \frac{(\sum Y) (\sum X^2) - (\sum X) (\sum XY)}{m(\sum X^2) - (\sum X)^2} \\
&= \frac{(6105.9) (1102) - (80) (81368.9)}{6(1102) - (80)^2} \approx 1033.9 \\
\beta_f &= \frac{m (\sum XY) - (\sum X) (\sum Y)}{m(\sum X^2) - (\sum X)^2} \\
&= \frac{6 (81368.9) - (80) (6105.9)}{6(1102) - (80)^2} \approx -1.220
\end{align*}
Therefore the regression formula is $P = 1033.9 - 1.220T$. 
\end{Answer}

\begin{Exercise}
\phantomsection
\label{ex:carbontrend}
Find a linear fit and a quadratic fit for the following atmospheric data regarding the global carbon dioxide level. Also, calculate the root-mean-square error for each fit.
\begin{figure}[h!]
\centering
\fbox{\includegraphics[scale = 0.32]{carbon.png}}
\caption{\textit{The trend of atmospheric carbon dioxide level measured at Mauna Loa Observatory for Example \ref{ex:carbontrend}.}}
\end{figure}
\begin{center}
\begin{tabular}{|c|c|c|c|c|c|c|}
\hline
Years passed since 1960 & $0$ & $5$ & $10$ & $15$ & $20$ & $25$ \\
\hline
CO$_2$ level (ppm) & $316.9$ & $320.0$ & $325.7$ & $331.1$ & $338.8$ & $346.4$ \\
\hline
Years passed since 1960 & $30$ & $35$ & $40$ & $45$ & $50$ & $55$\\
\hline
CO$_2$ level (ppm) & $354.4$ & $361.0$ & $369.7$ & $380.0$ & $390.1$ & $401.0$\\
\hline
\end{tabular}
\end{center}
(Data from: \href{https://gml.noaa.gov/webdata/ccgg/trends/co2/co2_mm_mlo.csv}{NOAA (https://gml.noaa.gov/webdata/ccgg/trends/co2/\\co2\_annmean\_mlo.csv)})
\end{Exercise}
\begin{Answer}
\begin{enumerate}[label=(\alph*)]
\item We have $m=12$, $\sum X=330$, $\sum (X^2)=12650$, $\sum (XY)=121977.0$, $\sum Y=4235.2$. So
\begin{align*}
\alpha_f &= \frac{(\sum Y) (\sum X^2) - (\sum X) (\sum XY)}{m(\sum X^2) - (\sum X)^2} \\
&= \frac{(4235.2) (12650) - (330) (121977.0)}{12(12650) - (330)^2} \approx 310.6 \\
\beta_f &= \frac{m (\sum XY) - (\sum X) (\sum Y)}{m(\sum X^2) - (\sum X)^2} \\
&= \frac{12 (121977.0) - (330) (4235.2)}{12(12650) - (330)^2} \approx 1.541
\end{align*}
So the linear fit is $[\text{CO}_2] = 310.6 + 1.541t$, RMSE is $\smash{\sqrt{\frac{126.5}{12}}} \approx 3.25$.
\item For a quadratic fit $[\text{CO}_2] = a + bt + ct^2$, we compute
\begin{align*}
\begin{bmatrix}
a\\
b\\
c  
\end{bmatrix}
=
\left(
\begin{bmatrix}
1 & 1 & 1 & \cdots \\
0 & 5 & 10 & \cdots \\
0 & 5^2 & 10^2 & \cdots
\end{bmatrix}
\begin{bmatrix}
1 & 0 & 0\\
1 & 5 & 5^2\\
1 & 10 & 10^2\\
\vdots & \vdots & \vdots
\end{bmatrix}\right)^{-1}
\begin{bmatrix}
1 & 1 & 1 & \cdots \\
0 & 5 & 10 & \cdots \\
0 & 5^2 & 10^2 & \cdots
\end{bmatrix}
\begin{bmatrix}
316.9\\
320.0\\
325.7\\
\vdots
\end{bmatrix}
\end{align*}
which yields $a = 316.1, b = 0.880, c= 0.0120$, $[\text{CO}_2] = 316.1 + 0.880t + 0.0121t^2$. RMSE is $\smash{\sqrt{\frac{5.2}{12}}} \approx 0.66$.
\end{enumerate}
\end{Answer}

\begin{Exercise}
\phantomsection
\label{ex:radioactive}
Radioactive decay is modeled by $N = N_0e^{-kt}$, where $N_0$ and $k$ are the initial concentration and the decay constant respectively. While the formula is exponential, not linear, the technique of linear regression can still be applied if the data undergoes \textit{linearization}. Show that by the substitution $n = \ln N$, the equation can be transformed into a linear equation $n = \ln N = \ln N_0 - kt = n_0 - kt$. Hence find the best linear fit on $(t, n)$ by finding the parameters $(n_0, k)$ from the experimental data on the radioactive isotope Sodium-24 below and recover the decay constant and initial mass.
\begin{center}
\begin{tabular}{|c|c|c|c|c|c|c|}
\hline
Time passed (hr) & $6$ & $8$ & $12$ & $24$ & $36$ & $48$\\
\hline
Mass (g) & $75.8$ & $69.1$ & $57.3$ & $33.0$ & $18.8$ & $10.8$\\
\hline
\end{tabular}
\end{center}
\end{Exercise}
\begin{Answer}
\begin{center}
\begin{tabular}{|c|c|c|c|c|c|c|}
\hline
t & $6$ & $8$ & $12$ & $24$ & $36$ & $48$\\
\hline
$n = \ln N$ & $4.3281$ & $4.2356$ & $4.0483$ & $3.4965$ & $2.9339$ & $2.3795$ \\
\hline
\end{tabular}
\end{center}
The parameters are $m=6$, $\sum X=134$, $\sum (X^2)=4420$, $\sum (XY)=412.1854$, $\sum Y=21.4219$, and thus
\begin{align*}
\alpha_f &= \frac{(\sum Y) (\sum X^2) - (\sum X) (\sum XY)}{m(\sum X^2) - (\sum X)^2} \\
&= \frac{(21.4219) (4420) - (134) (412.1854)}{6(4420) - (134)^2} \approx 4.607 \\
\beta_f &= \frac{m (\sum XY) - (\sum X) (\sum Y)}{m(\sum X^2) - (\sum X)^2} \\
&= \frac{6(412.1854) - (134) (21.4219)}{6(4420) - (134)^2} \approx -0.0464
\end{align*}
As a result, $\ln N = 4.607 - 0.0464t$, hence $N = e^{4.607 - 0.0464t} = 100.18e^{-0.0464t}$, roughly corresponding to a half-life of $\frac{\ln2}{0.0464} \approx 14.94$ hrs.
\end{Answer}

\begin{Exercise}
A commercial study investigates eight companies that sell the same type of products and are also similar in size. The following table summarizes their revenues, R\&D expenses, employee wages, and amounts of advertisement (the last three items are normalized scores).
\begin{center}
\begin{tabular}{|c|c|c|c|c|}
\hline
& Revenues & R\&D & Wages & Advertisement \\
\hline
Company 1 & $135\%$ & $2.5$ & $1.7$ & $0.8$ \\
\hline
Company 2 & $128\%$ & $1.6$ & $1.8$ & $1.5$ \\
\hline
Company 3 & $119\%$ & $1.8$ & $0.5$ & $2.4$ \\
\hline
Company 4 & $121\%$ & $0.3$ & $1.5$ & $1.2$ \\
\hline
Company 5 & $126\%$ & $1.9$ & $1.6$ & $1.3$ \\
\hline
Company 6 & $112\%$ & $0.8$ & $1.1$ & $0.2$ \\
\hline
Company 7 & $143\%$ & $2.2$ & $2.4$ & $1.1$ \\
\hline
Company 8 & $135\%$ & $1.5$ & $2.2$ & $2.3$ \\
\hline
\end{tabular}
\end{center}
Construct a linear regression model for the revenue against the three factors that follow. What is the $R^2$ of this regression?
\end{Exercise} 
\begin{Answer}
$M = 94.96 + 6.24D + 12.25W + 2.22A$, $R^2 = 0.94$.
\end{Answer}
\chapter{Discrete Fourier Transform (DFT)}
\label{chap:DFT}
We now discuss a powerful mathematical tool that can be approached from a least-square approximation point of view. In the last chapter, we saw how to fit a polynomial curve to some given data. It is then natural to further ask, whether there are other suitable types of functions that can be used for data fitting as well. Recall that in Chapter \ref{chap:innerchap}, we have derived the Fourier series into which any reasonable function can be expanded using sines and cosines. Coincidentally, in the area of Earth Science, many phenomena can be described by the notion of \textit{waves}, which are often assumed to be \textit{sinusoidal} during their derivation, e.g.\ atmospheric gravity waves, and seismic waves. However, in real life, our data sampling will be discrete. Therefore, we may want to know if we can do something similar and interpolate a discrete time series by fitting sines and cosines to it, and this is the central idea of what is known as the \textit{Discrete Fourier Transform (DFT)}. An example usage is to filter out the dominant frequency in different climate modes (e.g. ENSO and PDO-Pacific Decadal Oscillation).

\section{Mathematical Ideas of DFT}
\label{section:DFT}

\subsection{From Fourier Series to a Prototype of DFT}
\label{section:FouriertoDFT}

By Properties \ref{proper:fourierseries}, we know that any reasonable function $f(x)$ in the interval $[-\pi, \pi]$ can be written as 
\begin{align}
f(x) = \frac{a_0}{2} + \sum_{m=1}^{\infty} a_m \cos(mx) + \sum_{n=1}^{\infty} b_n \sin(nx) \label{eqn:fourierseri14}
\end{align} where the Fourier coefficients $a_m$ and $b_n$ are given by Formulae (\ref{eqn:fouriera}) and (\ref{eqn:fourierb}). By a change of variable
\begin{align}
x = \frac{2\pi}{N}t - \pi    
\end{align}
the interval is scaled to $t \in [0, N]$ where the cosines and sines are now in the form of
\begin{subequations}
\begin{align}
\cos(m(\frac{2\pi}{N}t - \pi)) &= \cos(\frac{2m\pi}{N}t) \\
\sin(n(\frac{2\pi}{N}t - \pi)) &= \sin(\frac{2n\pi}{N}t)
\end{align}    
\end{subequations}
The negative sign arising from the $m\pi$ or $n\pi$ term inside is absorbed whenever needed. Since it involves a linear variable transformation only, these cosines and sines
\begin{align}
\begin{aligned}
&\left\{\sqrt{\frac{2}{N}}\sin(\frac{2\pi}{N}t), \sqrt{\frac{2}{N}}\sin(\frac{2\pi(2)}{N}t), \sqrt{\frac{2}{N}}\sin(\frac{2\pi(3)}{N}t), \ldots, \frac{1}{\sqrt{N}}, \right. \\ 
&\left.\sqrt{\frac{2}{N}}\cos(\frac{2\pi}{N}t), \sqrt{\frac{2}{N}}\cos(\frac{2\pi(2)}{N}t), \sqrt{\frac{2}{N}}\cos(\frac{2\pi(3)}{N}t), \ldots \right\}     
\end{aligned} 
\end{align}\footnote{\label{foot:fouriernorm} The normalization factor of $\smash{\sqrt{2/N}}$ can be deduced by calculating the squared norm of just the cosines themselves
\begin{align*}
\int_0^N \abs{\cos(\frac{2m\pi}{N}t)}^2 dt &= \int_0^N \frac{1}{2} \left(1+ \cos(\frac{4m\pi}{N}t)\right) dt \\
&= \frac{1}{2}(N) + (0) = \frac{N}{2}
\end{align*} and the same goes for the sines.}
are still an orthonormal basis to the new $L^2[0, N]$ space, mapped from the initial $L^2[-\pi, \pi]$\footnote{Denote the original Fourier basis functions by $\varphi_j(x) \in L^2[-\pi, \pi]$. We will only show the part of linear independence and omit the justification for span and orthogonality. By Theorem \ref{thm:linearindep}, since they form a basis for $L^2[-\pi, \pi]$ as given, $c_1\varphi_j(x) + c_2\varphi_j(x) + \cdots = 0(x) \equiv 0$ where $0(x)$ is the zero function over $x \in [-\pi,\pi]$, has the trivial solution $c_j = 0$ as its only solution. Assume the form of the linear variable change is $x = \alpha t + \beta = X(t)$. Replacing $x$ by $X(t)$ (possible since $\alpha \neq 0$) then immediately produces the equality $c_1\varphi_j(X(t)) + c_2\varphi_j(X(t)) + \cdots = 0(X(t)) \equiv 0$ which automatically inherits the desired property of possessing only the trivial solution $c_j = 0$ for the zero function $0(X(t)) = 0(t)$ as well and shows that the new basis functions are now $\hat{\varphi}_j(t) = \varphi_j(X(t))$.}. The Fourier coefficients are now computed by 
\begin{subequations}
\begin{align}
a_m &= \frac{1}{\pi} \int_{-\pi}^{\pi} f(x)\cos(\frac{2m\pi}{N}t) d(\frac{2\pi}{N}t) = \frac{2}{N} \int_{0}^{N} f(t)\cos(\frac{2m\pi}{N}t) dt \label{eqn:fouriersca} \\
b_n &= \frac{1}{\pi} \int_{-\pi}^{\pi} f(x)\sin(\frac{2n\pi}{N}t) d(\frac{2\pi}{N}t) = \frac{2}{N} \int_{0}^{N} f(t)\sin(\frac{2n\pi}{N}t) dt \label{eqn:fourierscb}
\end{align}   
\end{subequations}
where we have written $f(t)$ in place of $f(x) = f(\frac{2\pi}{N}t - \pi)$. The partial sum of the Fourier expansion of $f(t)$ up to degree $p$
\begin{subequations}
\begin{align}
S_p[f](t) &= \frac{a_0}{2} + \sum_{m=1}^{p} a_m \cos(\frac{2m\pi}{N}t) + \sum_{n=1}^{p} b_n \sin(\frac{2n\pi}{N}t) \label{eqn:fourierpart} \\
&= \frac{a_0}{2} + a_1 \cos(\frac{2\pi}{N}t) + a_2 \cos(\frac{2\pi(2)}{N}t) + \cdots + a_p \cos(\frac{2\pi p}{N}t) \nonumber \\
&\quad + b_1 \sin(\frac{2\pi}{N}t) + b_2 \sin(\frac{2\pi(2)}{N}t) + \cdots + b_p \sin(\frac{2\pi p}{N}t) 
\end{align}
\end{subequations}
will then be the best approximation of the function $f(t)$ using sines and cosines up to order $p$ with distance defined with respect to the inner product in Equation (\ref{eqn:integralinner}), a.k.a.\ the best-fit trigonometric (Fourier) polynomial of degree $p$. This is known as the \index{Best Approximation Theorem}\keywordhl{Best Approximation Theorem} which directly follows from Properties \ref{proper:shortestorthoproj} and the related discussion about orthogonal projections onto the Fourier basis in Section \ref{subsection:spectralthmherm}: Such a Fourier partial sum $S_p[f](t)$ is essentially the orthogonal projection of $f(t)$ onto the subspace spanned by the corresponding orthonormal trigonometric basis up to order $p$. Hence, the higher the degree of the partial sum of the Fourier expansion, the more closer the approximation to the given function. We will use Example \ref{exmp:fourierx} as an illustration: Note that the appropriate Fourier series of $x = f(t) = \frac{2\pi}{N}t - \pi$ is now
\begin{align*}
f(t) &= \sum_{n=1}^{\infty} b_n \sin(\frac{2n\pi}{N}t) \\
&= -2 \sum_{n=1}^{\infty} \frac{1}{n} \sin(\frac{2n\pi}{N}t)
\end{align*}
where $b_n = -\frac{2}{n}$. The following plot (Figure \ref{fig:fourierlint}) then shows the original function and the partial sums of its Fourier expansion up to degrees $3,10,20$ respectively. From this, we can see that as the degree $p$ goes up, $S_p[f](t)$ becomes a better approximation to the straight line $y = \frac{2\pi}{N}t - \pi$.\footnote{Notice that at the two end-points, the values of Fourier series of $f(t) = \frac{2\pi}{N}t - \pi$ stay at $0$. It is because the Fourier series behaves as the periodic extension of the original function (see Footnote \ref{foot:fourier} of Chapter \ref{chap:innerchap}) and will converge to the average value of the left and right limits $(f_-(t) + f_+(t))/2$ at point of discontinuity. Therefore, as every end-point is essentially a discontinuity where one of the side limits is $\pi$ while the other is $-\pi$, it converges to $((-\pi) + \pi)/2 = 0$.}
\begin{figure}[t!]
    \centering
    \includegraphics[scale=0.7]{graphics/fourierxapprox.png}
    \caption{\textit{The Fourier expansion of the linear function up to degrees $3$, $10$, and $20$.}}
    \label{fig:fourierlint}
\end{figure}
Eventually, as $p \to \infty$ and the expression tends to the full Fourier series, $S_p[f]$ will converge to $f$ in the $L^2$ sense\footnote{This is also called \textit{convergence in mean} and is different from \textit{pointwise convergence} that is more intuitive for most people. The rigorous treatment to this requires Measure Theory and, again, Functional Analysis and will not be pursued here.}, with respect to the inner product in Equation (\ref{eqn:integralinner}). To show this heuristically, we need the \index{Bessel's Inequality}\keywordhl{Bessel's Inequality} (and \index{Parseval's Identity}\textit{Parseval's Identity}) for partial sums regarding a complete orthonormal basis in any $L^2$ space, which includes the case of the Fourier series:
\begin{align}
\lim_{p \to \infty}\norm{f - S_p[f]}^2 = 0
\end{align}
The detailed derivation is placed in Appendix \ref{section:DFTappend}. \par
Returning to practices in Earth Science, and other fields like Engineering, we are often not given a function that has a closed form to work with. Instead, we collect data from measurements at a fixed sampling rate, and what we obtain is a \textit{discrete time series}. However, we can still apply the idea of Fourier to approximate and interpolate the finite time series with sinusoidal functions. Assume that we have $N$ data points collected for the time series $f(t)$, evenly spaced at time $t = 0, 1, 2, \ldots, N-1$. Further assume that for now, we only use a few sines and cosines for the approximation, such that the degree $p$ is much less than the number of data points $N$, specifically, $2p+1 \leq N$. The suitable Fourier approximation then will be in the form of Equation (\ref{eqn:fourierpart}) where $N$ is the period of the time series. Since we only have finite data points, we cannot carry out the integrations needed to compute (\ref{eqn:fouriersca}) and (\ref{eqn:fourierscb}). Nevertheless, we can borrow the idea in the last chapter and do the approximation in a way such that the Fourier partial sum will achieve the least-square error when summed over the sampling points. Subsequently, the system to be optimized will be
\begin{empheq}[left={\empheqlbrace}]{alignat=1}
\scriptsize
\begin{aligned}
C_0 + A_1 \cos(\frac{2\pi}{N}(0)) + A_2 \cos(\frac{2\pi(2)}{N}(0)) + \cdots + A_p \cos(\frac{2\pi p}{N}(0)) \\
+ B_1 \sin(\frac{2\pi}{N}(0)) + B_2 \sin(\frac{2\pi(2)}{N}(0)) + \cdots + B_p \sin(\frac{2\pi p}{N}(0))
\end{aligned} &= f(0) \nonumber \\
\scriptsize
\begin{aligned} 
C_0 + A_1 \cos(\frac{2\pi}{N}(1)) + A_2 \cos(\frac{2\pi(2)}{N}(1)) + \cdots + A_p \cos(\frac{2\pi p}{N}(1)) \\
+ B_1 \sin(\frac{2\pi}{N}(1)) + B_2 \sin(\frac{2\pi(2)}{N}(1)) + \cdots + B_p \sin(\frac{2\pi p}{N}(1))
\end{aligned} &= f(1) \nonumber \\
\scriptsize
\begin{aligned} 
C_0 + A_1 \cos(\frac{2\pi}{N}(2)) + A_2 \cos(\frac{2\pi(2)}{N}(2)) + \cdots + A_p \cos(\frac{2\pi p}{N}(2)) \\
+ B_1 \sin(\frac{2\pi}{N}(2)) + B_2 \sin(\frac{2\pi(2)}{N}(2)) + \cdots + B_p \sin(\frac{2\pi p}{N}(2))
\end{aligned} &= f(2) \\
& \vdots \nonumber \\
\tiny
\begin{aligned} 
C_0 + A_1 \cos(\frac{2\pi}{N}(N-1)) + A_2 \cos(\frac{2\pi(2)}{N}(N-1)) + \cdots + A_p \cos(\frac{2\pi p}{N}(N-1)) \\
+ B_1 \sin(\frac{2\pi}{N}(N-1)) + B_2 \sin(\frac{2\pi(2)}{N}(N-1)) + \cdots + B_p \sin(\frac{2\pi p}{N}(N-1))
\end{aligned} &= f(N-1) \nonumber
\end{empheq}
where we have replaced the small letters $a_k, b_k$ by capital letters $A_k, B_k$ and $a_0$ by $C_0$. It can be expressed in the form of a matrix system
\begin{align}
G\vec{\beta} = \vec{d}    
\end{align}
where
\begin{align}
G =
\tiny
\begin{bmatrix}
1 & 1 & 0 & 1 & \cdots & 1 & 0 \\
1 & \cos(\frac{2\pi}{N}) & \sin(\frac{2\pi}{N}) & \cos(\frac{2\pi(2)}{N}) & \cdots & \cos(\frac{2\pi p}{N}) & \sin(\frac{2\pi p}{N}) \\
1 & \cos(\frac{2\pi(2)}{N}) & \sin(\frac{2\pi(2)}{N}) & \cos(\frac{2\pi(2)(2)}{N}) & \cdots & \cos(\frac{2\pi p (2)}{N}) & \sin(\frac{2\pi p (2)}{N}) \\
\vdots & \vdots & \vdots & \vdots & & \vdots & \vdots \\
1 & \cos(\frac{2\pi(N-1)}{N}) & \sin(\frac{2\pi(N-1)}{N}) & \cos(\frac{2\pi(2)(N-1)}{N}) & \cdots & \cos(\frac{2\pi p(N-1)}{N}) & \sin(\frac{2\pi p(N-1)}{N}) \\
\end{bmatrix}
\end{align}
is a $N \times (2p+1)$ matrix and
\begin{align}
&\vec{\beta} = 
\begin{bmatrix}
C_0 \\
A_1 \\
B_1 \\
\vdots \\
A_p \\
B_p
\end{bmatrix}
&\vec{d} = 
\begin{bmatrix}
f(0) \\
f(1) \\
f(2) \\
\vdots \\
f(N-1)
\end{bmatrix}
\end{align}
are vectors with $2p+1$ and $N$ components respectively. The least-square method then sets out to find the best-fit parameters for $\vec{\beta} = (C_0, A_1, B_1, \cdots, A_p, B_p)^T$ to this system of equations. From Theorem \ref{thm:bestfit}, we know that the best parameters are found by
\begin{align}
\vec{\beta}_f = (G^TG)^{-1}G^T\vec{d}
\end{align}
However, we can greatly simplify the expression, by noticing that every column vector of a sine/cosine series is orthogonal to each other. If we write each sine/cosine term in a series as a complex exponential using Euler's formula (\ref{eqn:Euler}), then the two column vectors of a sine-cosine pair in $G$ with a particular frequency of $2\pi k/N$, where both $k$ and $N$ are integers, can be expressed by the real and imaginary parts of
\begin{subequations}
\begin{align}
\left\{\exp(i \frac{2\pi k}{N} t)\right\}_{t = 0,1,2,\ldots,N-1}
&= \left\{\cos(\frac{2\pi k}{N} t) + i\sin(\frac{2\pi k}{N} t) \right\}_{t = 0,1,2,\ldots,N-1}
\label{eqn:complexsinecosa}
\end{align}
or alternatively,
\begin{align}
\left\{\exp(-i \frac{2\pi k}{N} t)\right\}_{t = 0,1,2,\ldots,N-1}
&= \left\{\cos(\frac{2\pi k}{N} t) - i\sin(\frac{2\pi k}{N} t) \right\}_{t = 0,1,2,\ldots,N-1}
\label{eqn:complexsinecosb}
\end{align}
\end{subequations}
We now first prove that the two column vectors of a sine-cosine pair that have the same frequency are orthogonal. Taking the sum of squares over (\ref{eqn:complexsinecosa}) gives
\begin{align}
\small
\begin{aligned}
\sum_{t=0}^{N-1} \left(\exp(i \frac{2\pi k}{N} t)\right)^2    
\end{aligned} &= 
\small
\begin{aligned}
\sum_{t=0}^{N-1} \left(\cos(\frac{2\pi k}{N} t) + i\sin(\frac{2\pi k}{N} t)\right)^2
\end{aligned} \nonumber \\
\small
\begin{aligned}
\sum_{t=0}^{N-1} \exp(i \frac{4\pi k}{N} t)
\end{aligned} &= 
\small
\begin{aligned}
\sum_{t=0}^{N-1} \left(\cos^2(\frac{2\pi k}{N} t) + 2i \sin(\frac{2\pi k}{N} t)\cos(\frac{2\pi k}{N} t) - \sin^2(\frac{2\pi k}{N} t)\right)
\end{aligned}
\end{align}
Notice that the L.H.S. is a geometric sequence with a common ratio $r = \exp(i 4\pi k/N)$\footnote{which will not be equal to $1$ as we have demanded that $p, k \leq \frac{N-1}{2}$.}, the sum of which is seen to be
\begin{align}
\frac{1-r^N}{1-r} &= \frac{1 - \exp(i 4\pi k)}{1 - \exp(i 4\pi k/N)} = 0
\end{align}
as $\exp(i 4\pi k)$ is just $1$. By comparing the real and imaginary parts, we know that
\begin{subequations}
\begin{align}
\sum_{t=0}^{N-1} \cos^2(\frac{2\pi k}{N} t) &= \sum_{t=0}^{N-1} \sin^2(\frac{2\pi k}{N} t) \label{eqn:sumofsincossq} \\
\sum_{t=0}^{N-1} \sin(\frac{2\pi k}{N} t)\cos(\frac{2\pi k}{N} t) &= 0 \label{eqn:sumofsincosprod}
\end{align}
\end{subequations}
The second equation (\ref{eqn:sumofsincosprod}) shows that the two column vectors representing sine and cosine waves of the same frequency have a dot product of zero and hence are orthogonal. \par 

Utilizing the complex formulation, we can also prove that column vectors of sine (or cosine) functions with different frequencies are orthogonal as well. Here we prove one of the cases, where the first series is a sine with a frequency of $2\pi k/N$,  the second series is also a sine, of a frequency of $2\pi l/N$, and $k \neq l$ are distinct integers. We start by considering the sum of products between
\begin{subequations}
\begin{align}
\left\{\exp(i \frac{2\pi k}{N} t)\right\}_{t = 0,1,2,\ldots,N-1} = \left\{\cos(\frac{2\pi k}{N} t) + i\sin(\frac{2\pi k}{N} t) \right\}_{t = 0,1,2,\ldots,N-1} 
\end{align}
and
\begin{align}
\left\{\exp(i \frac{2\pi l}{N} t)\right\}_{t = 0,1,2,\ldots,N-1} = \left\{\cos(\frac{2\pi l}{N} t) + i\sin(\frac{2\pi l}{N} t) \right\}_{t = 0,1,2,\ldots,N-1} 
\end{align}   
\end{subequations}
The analysis is similar to the one above. Particularly, if we take the sum of products over these two equations, the L.H.S. will be zero, and by considering the real part of the expression on R.H.S., we have
\begin{align}
\sum_{t=0}^{N-1} \cos(\frac{2\pi k}{N} t)\cos(\frac{2\pi l}{N} t) - \sum_{t=0}^{N-1} \sin(\frac{2\pi k}{N} t)\sin(\frac{2\pi l}{N} t) &= 0 \label{eqn:cossinmixed1}
\end{align}
as long as $k + l$ is not the integer multiples of $N$. (why?)\footnote{If $k + l = qN$, the sum of complex exponentials on L.H.S. will then become
\begin{align*}
\sum_{t=0}^{N-1} \exp(i \frac{2\pi k}{N} t) \exp(i \frac{2\pi l}{N} t) &= \sum_{t=0}^{N-1} \exp(i \frac{2\pi k}{N} t) \exp(i \frac{2\pi (qN - k)}{N} t) \\
&= \sum_{t=0}^{N-1} \exp(i 2\pi q t) = \sum_{t=0}^{N-1} 1 = N \neq 0
\end{align*}
and the argument fails.} We can also consider another sum of products between
\begin{subequations}
\begin{align}
\left\{\exp(i \frac{2\pi k}{N} t)\right\}_{t = 0,1,2,\ldots,N-1} = \left\{\cos(\frac{2\pi k}{N} t) + i\sin(\frac{2\pi k}{N} t) \right\}_{t = 0,1,2,\ldots,N-1} 
\end{align}
and
\begin{align}
\left\{\exp(-i \frac{2\pi l}{N} t)\right\}_{t = 0,1,2,\cdots,N-1} = \left\{\cos(\frac{2\pi l}{N} t) - i\sin(\frac{2\pi l}{N} t) \right\}_{t = 0,1,2,\cdots,N-1}       
\end{align}
\end{subequations}
Again by looking at the real part, this yields another relation as\footnote{Similarly we have the constraint that $k$ and $l$ are not differed by an integer multiple of $N$.}
\begin{align}
\sum_{t=0}^{N-1} \cos(\frac{2\pi k}{N} t)\cos(\frac{2\pi l}{N} t) + \sum_{t=0}^{N-1} \sin(\frac{2\pi k}{N} t)\sin(\frac{2\pi l}{N} t) &= 0 \label{eqn:cossinmixed2}
\end{align}
From the two equations (\ref{eqn:cossinmixed1}) and (\ref{eqn:cossinmixed2}) just derived, we can hence conclude that
\begin{subequations}
\begin{align}
\sum_{t=0}^{N-1} \cos(\frac{2\pi k}{N} t)\cos(\frac{2\pi l}{N} t) &= 0 \\
\sum_{t=0}^{N-1} \sin(\frac{2\pi k}{N} t)\sin(\frac{2\pi l}{N} t) &= 0        
\end{align}    
\end{subequations}
The orthogonality relations can be proven between sine and cosine of different frequencies as well in a very similar essence and we leave it to the readers. We will now establish the last result about the dot product of any cosine (or sine) column vector with a specific frequency $2\pi k/N$ with itself. We can consider the sum of products between
\begin{align*}
&\left\{\exp(i\frac{2\pi k}{N} t)\right\}_{t = 0,1,2,\ldots,N-1} & \text{ and } & &\left\{\exp(-i \frac{2\pi k}{N} t)\right\}_{t = 0,1,2,\ldots,N-1}
\end{align*} 
This time, the L.H.S. is not a geometric series, but rather $N$ terms of $1$. The relation is then
\begin{align}
N = \sum_{t=0}^{N-1} (1) &= \sum_{t=0}^{N-1} \exp(i \frac{2\pi k}{N} t)\exp(-i \frac{2\pi k}{N} t) \nonumber \\
&= \sum_{t=0}^{N-1} \left(\cos(\frac{2\pi k}{N} t) + i\sin(\frac{2\pi k}{N} t)\right)\left(\cos(\frac{2\pi k}{N} t) - i\sin(\frac{2\pi k}{N} t)\right) \nonumber \\
&= \sum_{t=0}^{N-1} \cos^2(\frac{2\pi k}{N} t) + \sum_{t=0}^{N-1} \sin^2(\frac{2\pi k}{N} t) \label{eqn:samefreqcossin}
\end{align}
Actually, it can also be inferred from the fact that the sum of a sine-cosine square pair is $1$. Solving the two equations (\ref{eqn:sumofsincossq}) and (\ref{eqn:samefreqcossin}) then yields
\begin{align}
\sum_{t=0}^{N-1} \cos^2(\frac{2\pi k}{N} t) &= \sum_{t=0}^{N-1} \sin^2(\frac{2\pi k}{n} t) = \frac{N}{2} \label{eqn:sqcossineNhalf}   
\end{align}
Hence the product $G^TG$, where each entry will be the dot product between the corresponding series of sines and cosines, will be
\begin{align}
G^TG =
\begin{bmatrix}
N & 0 & 0 & \cdots \\
0 & \frac{N}{2} & 0 & \\
0 & 0 & \frac{N}{2} & \\
\vdots & & & \ddots
\end{bmatrix}
\end{align}
and
\begin{align}
(G^TG)^{-1} = \frac{1}{N}
\begin{bmatrix}
1 & 0 & 0 & \cdots \\
0 & 2 & 0 & \\
0 & 0 & 2 & \\
\vdots & & & \ddots
\end{bmatrix}    
\end{align}
So the best-fit parameters are
\begin{align}
\vec{\beta_f} &= (G^TG)^{-1}G^T\vec{d} \nonumber \\
\begin{bmatrix}
C_0 \\
A_1 \\
B_1 \\
\vdots \\
\end{bmatrix}
&= \frac{1}{N} 
\begin{bmatrix}
1 & 0 & 0 & \cdots \\
0 & 2 & 0 & \\
0 & 0 & 2 & \\
\vdots & & & \ddots
\end{bmatrix} 
\begin{bmatrix}
1 & 1 & 1 & \cdots \\
1 & \cos(\frac{2\pi}{N}) & \cos(\frac{2\pi(2)}{N}) & \\
0 & \sin(\frac{2\pi}{N}) & \sin(\frac{2\pi(2)}{N}) & \\
\vdots & & & \ddots
\end{bmatrix}
\begin{bmatrix}
f(0)\\
f(1)\\
f(2)\\
\vdots
\end{bmatrix} \nonumber \\
\begin{bmatrix}
C_0 \\
A_1 \\
B_1 \\
\vdots \\
\end{bmatrix}
&= 
\frac{1}{N}
\begin{bmatrix}
1 & 1 & 1 & \cdots \\
2 & 2\cos(\frac{2\pi}{N}) & 2\cos(\frac{2\pi(2)}{N}) & \\
0 & 2\sin(\frac{2\pi}{N}) & 2\sin(\frac{2\pi(2)}{N}) & \\
\vdots & & & \ddots
\end{bmatrix}
\begin{bmatrix}
f(0)\\
f(1)\\
f(2)\\
\vdots
\end{bmatrix} 
\end{align}
Detailed expressions are
\begin{subequations}
\begin{align}
C_0 &= \frac{1}{N}(f(0) + f(1) + f(2) + \cdots + f(N-1)) \nonumber \\
&= \frac{1}{N}\sum_{t=0}^{n-1} f(t) \label{eqn:protoDFTc}  \\
A_k &= \frac{2}{N}\left(f(0) + f(1)\cos(\frac{2\pi k}{N}) + \cdots + f(N-1)\cos(\frac{2\pi k(N-1)}{N})\right) \nonumber \\
&= \frac{2}{N}\sum_{t=0}^{N-1} f(t)\cos(\frac{2\pi kt}{N}) \label{eqn:protoDFTa} \\
B_k &= \frac{2}{N}\left(f(1)\sin(\frac{2\pi k}{N}) + \cdots + f(N-1)\sin(\frac{2\pi k (N-1)}{N})\right) \nonumber \\
&= \frac{2}{N}\sum_{t=0}^{N-1} f(t)\sin(\frac{2\pi kt}{N}) \label{eqn:protoDFTb}
\end{align}
\end{subequations}
That makes sense, at least they seem to be. But in the worked short exercise that follows the example below, you will immediately notice that there is a big caveat to this prototype.
\begin{exmp}
\label{exmp:ex11.2.1}
Fit the following time series with the Fourier trigonometric basis up to order $p=1$, where $f(0) = 4, f(1) = 1, f(2) = 2, f(3) = 3, f(4) = 1$.
\end{exmp}
\begin{solution}
The degree is $p=1$ and it means that there are only three components, which are the constant term, and a sine-cosine pair with a frequency of $2\pi k/N$ where $k=1$, $N=5$. The best-fit parameters will be
\begin{align*}
\small
\begin{bmatrix}
C_0 \\
A_1 \\
B_1
\end{bmatrix}
&= 
\frac{1}{5}
\small
\begin{bmatrix}
1 & 1 & 1 & 1 & 1 \\
2 & 2\cos(\frac{2\pi}{5}) & 2\cos(\frac{2\pi(2)}{5}) & 2\cos(\frac{2\pi(3)}{5}) & 2\cos(\frac{2\pi(4)}{5}) \\
0 & 2\sin(\frac{2\pi}{5}) & 2\sin(\frac{2\pi(2)}{5}) & 2\sin(\frac{2\pi(3)}{5}) & 2\sin(\frac{2\pi(4)}{5})
\end{bmatrix}
\small
\begin{bmatrix}
f(0)\\
f(1)\\
f(2)\\
f(3)\\
f(4)
\end{bmatrix} 
\end{align*}
\begin{align*}
C_0 &= \frac{1}{5} (4+1+2+3+1) = \frac{11}{5} \\
A_1 &= \frac{2}{5} \left(4 + 1\cos(\frac{2\pi}{5}) + 2\cos(\frac{2\pi(2)}{5}) + 3\cos(\frac{2\pi(3)}{5}) + 1\cos(\frac{2\pi(4)}{5})\right) \\
&\approx 0.229 \\
B_1 &= \frac{2}{5} \left(0 + 1\sin(\frac{2\pi}{5}) + 2\sin(\frac{2\pi(2)}{5}) + 3\sin(\frac{2\pi(3)}{5}) + 1\sin(\frac{2\pi(4)}{5})\right) \\
&\approx -0.235
\end{align*}
So the best trigonometric fit of order $p=1$ for the time series concerned is
\begin{align*}
f(t) \approx \frac{11}{5} + 0.229 \cos(\frac{2\pi}{5}t) - 0.235 \sin(\frac{2\pi}{5}t)
\end{align*}
\end{solution}
\begin{exmp}
As a follow-up short exercise for the last example, find an improved approximation with order $p = 2$. What happens if $p = 3$?    
\end{exmp}
\begin{solution}
For $p = 2$, the best-fit coefficients are
\begin{align*}
\footnotesize
\begin{bmatrix}
C_0 \\
A_1 \\
B_1 \\
A_2 \\
B_2
\end{bmatrix}
&= 
\frac{1}{5}
\footnotesize
\begin{bmatrix}
1 & 1 & 1 & 1 & 1 \\
2 & 2\cos(\frac{2\pi}{5}) & 2\cos(\frac{2\pi(2)}{5}) & 2\cos(\frac{2\pi(3)}{5}) & 2\cos(\frac{2\pi(4)}{5}) \\
0 & 2\sin(\frac{2\pi}{5}) & 2\sin(\frac{2\pi(2)}{5}) & 2\sin(\frac{2\pi(3)}{5}) & 2\sin(\frac{2\pi(4)}{5}) \\
2 & 2\cos(\frac{2\pi(2)}{5}) & 2\cos(\frac{2\pi(2)(2)}{5}) & 2\cos(\frac{2\pi(2)(3)}{5}) & 2\cos(\frac{2\pi(2)(4)}{5}) \\
0 & 2\sin(\frac{2\pi(2)}{5}) & 2\sin(\frac{2\pi(2)(2)}{5}) & 2\sin(\frac{2\pi(2)(3)}{5}) & 2\sin(\frac{2\pi(2)(4)}{5})
\end{bmatrix}
\footnotesize
\begin{bmatrix}
f(0)\\
f(1)\\
f(2)\\
f(3)\\
f(4)
\end{bmatrix}     
\end{align*}
It is not hard to see that $C_0, A_1, B_1$ will be the same and
\begin{align*}
A_2 &= \frac{2}{5} \left(4 + 1\cos(\frac{2\pi(2)}{5}) + 2\cos(\frac{2\pi(2)(2)}{5}) \right. \\
&\quad \left.{} + 3\cos(\frac{2\pi(2)(3)}{5}) + 1\cos(\frac{2\pi(2)(4)}{5})\right) \approx 1.571 \\
B_2 &= \frac{2}{5} \left(0 + 1\sin(\frac{2\pi(2)}{5}) + 2\sin(\frac{2\pi(2)(2)}{5}) \right. \\
&\quad \left.{} + 3\sin(\frac{2\pi(2)(3)}{5}) + 1\sin(\frac{2\pi(2)(4)}{5})\right) \approx 0.380
\end{align*}
Hence the new approximation will be 
\begin{align*}
f(t) &\approx \frac{11}{5} + 0.229 \cos(\frac{2\pi}{5}t) - 0.235 \sin(\frac{2\pi}{5}t) \\
&\quad + 1.571 \cos(\frac{2\pi(2)}{5}t) + 0.380 \sin(\frac{2\pi(2)}{5}t)    
\end{align*}
When $p$ is increased to $3$, the matrix $G$ involved in the best-fit formula becomes
\begin{align*}
\scriptsize
G = 
\begin{bmatrix}
1 & 1 & 0 & 1 & 0 & 1 & 0 \\
1 & \cos(\frac{2\pi}{5}) & \sin(\frac{2\pi}{5}) & \cos(\frac{2\pi(2)}{5}) & \sin(\frac{2\pi(2)}{5}) & \cos(\frac{2\pi(3)}{5}) & \sin(\frac{2\pi(3)}{5})\\
1 & \cos(\frac{2\pi(2)}{5}) & \sin(\frac{2\pi(2)}{5}) & \cos(\frac{2\pi(2)(2)}{5}) & \sin(\frac{2\pi(2)(2)}{5}) & \cos(\frac{2\pi(3)(2)}{5}) & \sin(\frac{2\pi(3)(2)}{5}) \\
1 & \cos(\frac{2\pi(3)}{5}) & \sin(\frac{2\pi(3)}{5}) & \cos(\frac{2\pi(2)(3)}{5}) & \sin(\frac{2\pi(2)(3)}{5}) & \cos(\frac{2\pi(3)(3)}{5}) & \sin(\frac{2\pi(3)(3)}{5})\\
1 & \cos(\frac{2\pi(4)}{5}) & \sin(\frac{2\pi(4)}{5}) & \cos(\frac{2\pi(2)(4)}{5}) & \sin(\frac{2\pi(2)(4)}{5}) & \cos(\frac{2\pi(3)(4)}{5}) & \sin(\frac{2\pi(3)(4)}{5})
\end{bmatrix}    
\end{align*}
and a routine computation will show that
\begin{align*}
G^T G = 
\begin{bmatrix}
5 & 0 & 0 & 0 & 0 & 0 & 0 \\
0 & \frac{5}{2} & 0 & 0 & 0 & 0 & 0 \\
0 & 0 & \frac{5}{2} & 0 & 0 & 0 & 0 \\
0 & 0 & 0 & \frac{5}{2} & 0 & \frac{5}{2} & 0 \\
0 & 0 & 0 & 0 & \frac{5}{2} & 0 & -\frac{5}{2} \\
0 & 0 & 0 & \frac{5}{2} & 0 & \frac{5}{2} & 0 \\
0 & 0 & 0 & 0 & -\frac{5}{2} & 0 & \frac{5}{2} \\
\end{bmatrix}
\end{align*}
is not invertible (the fourth/sixth rows are equal and the fifth/last rows are the negative of each other), so that the formula $\vec{\beta}_f = (G^TG)^{-1}G^T\vec{d}$ will fail. If we forcefully use the expressions in (\ref{eqn:protoDFTa}) and (\ref{eqn:protoDFTb}) to compute $A_3$ and $B_3$ we will get
\begin{align*}
A_3 &= \frac{2}{5} \left(4 + 1\cos(\frac{2\pi(3)}{5}) + 2\cos(\frac{2\pi(3)(2)}{5}) \right. \\
&\quad \left.{} + 3\cos(\frac{2\pi(3)(3)}{5}) + 1\cos(\frac{2\pi(3)(4)}{5})\right) = A_2 \\
B_3 &= \frac{2}{5} \left(0 + 1\sin(\frac{2\pi(3)}{5}) + 2\sin(\frac{2\pi(3)(2)}{5}) \right. \\
&\quad \left.{} + 3\sin(\frac{2\pi(3)(3)}{5}) + 1\sin(\frac{2\pi(3)(4)}{5})\right) = -B_2    
\end{align*}
So $A_2$ [$B_2$] and $A_3$ [$B_3$] carry the same information and one of them will be redundant.
\end{solution}

\subsection{Nyquist Frequency and Real DFT}

While we may want to make the approximation by the Fourier basis as good as possible, we have to know how high the order $p$ needs to be. On the other hand, as we can see in the short exercise of the last example, if $p$ is set too large, the best-fit formula will become problematic and the approximation will contain duplicated terms, in the sense that they take equal values (or with an opposite sign) at all sampling points. For example, if $k + l = N$, then for the integer time steps $t = 0, 1, \ldots, N-1$
\begin{align}
\cos(\frac{2\pi l}{N} t) &= \cos(\frac{2\pi (N-k)}{N} t) \nonumber \\
&= \cos(2\pi t - \frac{2\pi k}{N} t) \nonumber \\
&= \cos(\frac{2\pi k}{N} t) \label{eqn:Nyquistdupcos}
\end{align}
so the two cosine waves, despite having different frequencies, coincide at every data point and will be indistinguishable within the time series. The same problem arises similarly for the sine signals. The condition $k + l = N$ above offers a hint that the maximum value of $p$ should be $N/2$. This corresponds to an angular frequency of 
\begin{align}
\omega_{\text{Ny}} = \frac{2\pi}{N}\left(\frac{N}{2}\right) = \pi    
\end{align}
which is known as the \index{Nyquist Frequency}\keywordhl{Nyquist frequency}. Consequentially, we have the \index{Nyquist Sampling Theorem}\keywordhl{Nyquist Sampling Theorem} as follows.
\begin{thm}[Nyquist Sampling Theorem]
\label{thm:Nyquist}
For an evenly spaced time series that has a time step of $\Delta t = 1$, any sinusoidal wave with an angular frequency exceeding the Nyquist frequency $\omega > \omega_{\text{Ny}} = \pi$ cannot be properly detected. 
\end{thm}
So, it means that the highest resolvable frequency in the time series is the Nyquist frequency $\omega_{\text{Ny}} = \pi$, or in other words, the minimum period length has to cover at least two time steps. As a result, we only need the sine/cosine terms up to order $\lfloor N/2 \rfloor$\footnote{The floor symbol $\lfloor \, \rfloor$ returns the largest integer smaller than or equal to the argument.}. We have just shown a part of the theorem that if $k + l = N$, then the sinusoidal series of a higher order $l = \lfloor N/2 \rfloor + 1, \ldots, N$ will be redundant in the derivation (\ref{eqn:Nyquistdupcos}) above. We will complete the theorem by verifying that the sine/cosine terms of an order $l = N+1, N+2, \ldots$ beyond will also lead to duplicated modes: let $l = k + qN$ where $q$ is a positive integer, then
\begin{align}
\cos(\frac{2\pi l}{N} t) &= \cos(\frac{2\pi (k + qN)}{N} t) \nonumber \\
&= \cos(\frac{2\pi k}{N} t + 2\pi q t) \nonumber \\
&= \cos(\frac{2\pi k}{N} t)
\end{align}
for all time steps $t = 0,1,\ldots,N-1$, and the same goes for the sines. Another perspective to see the problem is that, if the "ground truth" function to be approximated by DFT indeed contains some sinusoidal component with a frequency higher than the Nyquist frequency, then its signal will "spill" into a corresponding lower frequency. Again, using cosine and the case $k + l = N$ as an illustration, if $k$ represents the lower frequency mode and $l$ represents the higher one (assumed to have a Fourier coefficient of $a_l$ in the "ground truth"), then (\ref{eqn:protoDFTa}) will yield
\begin{align}
A_{k \leftarrow l} &= \frac{2}{N}\sum_{t=0}^{N-1} \left(a_l\cos(\frac{2\pi lt}{N})\right) \cos(\frac{2\pi kt}{N})\nonumber \\
&= \frac{2}{N}\sum_{t=0}^{N-1} a_l\cos(\frac{2\pi (N-k)t}{N}) \cos(\frac{2\pi kt}{N})\nonumber \\
&= \frac{2}{N}\sum_{t=0}^{N-1} a_l\cos(2\pi t - \frac{2\pi k}{N} t) \cos(\frac{2\pi kt}{N})\nonumber \\
&= \frac{2}{N}\sum_{t=0}^{N-1} a_l\cos(\frac{2\pi k}{N} t) \cos(\frac{2\pi kt}{N}) \nonumber \\
&= \frac{2}{N}\sum_{t=0}^{N-1} a_l \cos^2(\frac{2\pi kt}{N}) \nonumber \\
&= \frac{2}{N} \left(a_l \left(\frac{N}{2}\right)\right) = a_l & \text{(by (\ref{eqn:sqcossineNhalf}))} 
\end{align}
Hence any signal with a frequency higher than the Nyquist frequency will contribute to the respective DFT frequency and contaminate it.\footnote{On the other hand, if the true function only contains the lower frequency mode, then by the same steps we can show that the prototype DFT coefficient $A_k$ will coincide exactly with the usual Fourier coefficients $a_k$.}\par

Now we can formally derive the \index{Discrete Fourier Transform (DFT, real)}\keywordhl{(real) Discrete Fourier Transform} for a time series which is given by (\ref{eqn:protoDFTc}), (\ref{eqn:protoDFTa}), and (\ref{eqn:protoDFTb}). In the last part, we have already worked with an odd $N$ (see Example \ref{exmp:ex11.2.1}) where the maximum resolvable degree will be $p = \frac{N}{2} - 1$. When $N$ is even, we will have frequencies from zero to $k = \frac{N}{2}$. Notice that the constant term of zero frequency contributes a single parameter, every other frequency contributes two coefficients via a pair of sine and cosine, and the maximum frequency $k_{\text{Ny}} = \frac{N}{2}$ only gives rise to one coefficient from the cosine series which takes the form of alternating $(1,-1,1,-1,\ldots,1,-1)$ (the corresponding diagonal entry in the $(G^TG)^{-1}$ term of the least-square formula will hence be $1/N$ instead of $2/N$), as the sine function of the Nyquist frequency will be always zero at the sampling points. In both cases, there can be at most $N$ sinusoidal curves for fitting $N$ data points and $G$ will be square (and hence invertible). Since the amount of parameters is the same as the number of data, it becomes a unique interpolation that passes through all the given data points. By convention, we will omit the factor of $1/N$ in the computation of DFT.

\begin{exmp}
Find the Discrete Fourier Transform of the time series $(1,2,1,-1,3,0,2,-2)$.
\end{exmp}
\begin{solution}
According to Theorem \ref{thm:Nyquist}, the highest resolvable degree will be $N/2 = 4$. Its DFT is then given by
\begin{align*}
\scriptsize
\begin{bmatrix}
C_0 \\
A_1 \\
B_1 \\
A_2 \\
B_2 \\
A_3 \\
B_3 \\
A_4
\end{bmatrix}
&= 
\tiny
\begin{bmatrix}
1 & 1 & 1 & 1 & \cdots & 1 & 1 \\
2 & 2\cos(\frac{2\pi}{8}) & 2\cos(\frac{2\pi(2)}{8}) & 2\cos(\frac{2\pi(3)}{8}) & \cdots & 2\cos(\frac{2\pi(6)}{8}) & 2\cos(\frac{2\pi(7)}{8}) \\
0 & 2\sin(\frac{2\pi}{8}) & 2\sin(\frac{2\pi(2)}{8}) & 2\sin(\frac{2\pi(3)}{8}) & \cdots & 2\sin(\frac{2\pi(6)}{8}) & 2\sin(\frac{2\pi(7)}{8}) \\
2 & 2\cos(\frac{2\pi(2)}{8}) & 2\cos(\frac{2\pi(2)(2)}{8}) & 2\cos(\frac{2\pi(2)(3)}{8}) & \cdots & 2\cos(\frac{2\pi(2)(6)}{8}) & 2\cos(\frac{2\pi(2)(7)}{8}) \\
0 & 2\sin(\frac{2\pi(2)}{8}) & 2\sin(\frac{2\pi(2)(2)}{8}) & 2\sin(\frac{2\pi(2)(3)}{8}) & \cdots & 2\sin(\frac{2\pi(2)(6)}{8}) & 2\sin(\frac{2\pi(2)(7)}{8}) \\
2 & 2\cos(\frac{2\pi(3)}{8}) & 2\cos(\frac{2\pi(3)(2)}{8}) & 2\cos(\frac{2\pi(3)(3)}{8}) & \cdots & 2\cos(\frac{2\pi(3)(6)}{8}) & 2\cos(\frac{2\pi(3)(7)}{8}) \\
0 & 2\sin(\frac{2\pi(3)}{8}) & 2\sin(\frac{2\pi(3)(2)}{8}) & 2\sin(\frac{2\pi(3)(3)}{8}) & \cdots & 2\sin(\frac{2\pi(3)(6)}{8}) & 2\sin(\frac{2\pi(3)(7)}{8}) \\
1 & -1 & 1 & -1 & \cdots & 1 & -1
\end{bmatrix}
\scriptsize
\begin{bmatrix}
f(0)\\
f(1)\\
f(2)\\
f(3)\\
f(4)\\
f(5)\\
f(6)\\
f(7)
\end{bmatrix}     
\end{align*}
A direct computation then gives 
\begin{align*}
&\quad (C_0, A_1, B_1, A_2, B_2, A_3, B_3, A_4)^T \\
&= (6, -2.586, 2.243, 2, 10, -5.414, 6.243, 8)^T
\end{align*}
\end{solution}

\subsection{Complex DFT}

The real DFT approach above has the drawback of using sine-cosine pairs for computation. In Section \ref{section:complexno}, we have learned the power of complex exponentials to simultaneously represent sines and cosines through Euler's formula, which has also been exploited to derive the relations between the various sinusoidal functions when we are developing the DFT prototype. Therefore, there is an incentive to explore the possibility of using complex exponentials to define Discrete Fourier Transform. This has the benefits of simplicity and also convenience when programming.

Continuing from the ideas built in the last subsection, we propose an interpolation scheme that uses $\exp(i (2\pi k/N) t)$, for a time series with $N$ data, evenly spaced by a time step of $\Delta t = 1$. The range of $k$ will be from 
\begin{align*}
-\frac{N}{2}, -\left(\frac{N}{2}-1\right), \ldots, -1, \allowbreak 0, 1, \ldots, \frac{N}{2}-1  
\end{align*}for even $N$ and
\begin{align*}
-\frac{N-1}{2}, -\left(\frac{N-1}{2}-1\right), \dots, -1, 0, 1, \ldots, \frac{N-1}{2} 
\end{align*} for odd $N$. Both ranges ensure that the total number of complex exponentials used in the fitting process is exactly $N$. Each pair of complex exponentials with the wavenumber $k$ and $-k$, i.e.\ $\exp(i (2\pi k/N) t)$ and $\exp(-i (2\pi k/N) t)$ in combination, gives rise to $\cos( (2\pi k/N) t)$ and $\sin((2\pi k/N) t)$ by (\ref{eqn:sincoscomplex}), and so we can expect the correspondence between the real and complex version of Discrete Fourier Transform.\par
Now, the matrix $G$ will take the form of
\begin{align}
G = \tiny
\renewcommand\arraystretch{1.33}
\left[\begin{array}{@{\,}wc{5pt}wc{40pt}wc{40pt}wc{8pt}wc{56pt}wc{56pt}wc{8pt}wc{40pt}}
1 & 1 & 1 & \cdots & 1 & 1 & \cdots & 1\\
1 & \exp(i\frac{2\pi}{N}) & \exp(i\frac{2\pi(2)}{N}) & \cdots & \exp(i\frac{2\pi(\frac{N}{2}-1)}{N}) & \exp(i\frac{2\pi(-\frac{N}{2})}{N}) & \cdots & \exp(-i\frac{2\pi}{N}) \\
1 & \exp(i\frac{2\pi(2)}{N}) & \exp(i\frac{2\pi(2)(2)}{N}) & \cdots & \exp(i\frac{2\pi(\frac{N}{2}-1)(2)}{N}) & \exp(i\frac{2\pi(-\frac{N}{2})(2)}{N}) & \cdots & \exp(-i\frac{2\pi(2)}{N})\\
\vdots & \vdots & \vdots &  & \vdots & \vdots &  & \vdots \\
1 & \exp(i\frac{2\pi(N-1)}{N}) & \exp(i\frac{2\pi(2)(N-1)}{N}) & \cdots & \exp(i\frac{2\pi(\frac{N}{2}-1)(N-1)}{N}) & \exp(i\frac{2\pi(-\frac{N}{2})(N-1)}{N}) & \cdots & \exp(-i\frac{2\pi(N-1)}{N}) \\
\end{array}\right]
\end{align}
for even $N$, where each column represents frequencies corresponding to 
\begin{align*}
k = 0, 1, \ldots, \frac{N}{2}-1, -\frac{N}{2}, \ldots, -1    
\end{align*}
This is a common convention where we start from $k = 0$, and it increases to the largest positive $k = \frac{N}{2}-1$, then we flip the sign and resume from the most negative $k = -\frac{N}{2}$, finally go all the way back to $k = -1$. For odd $N$, the matrix $G$ is essentially the same, except the $k$ is replaced by the appropriate range of integers, and in particular, the flipping is from $k = \frac{N-1}{2}$ to $k = -\frac{N-1}{2}$. \par
The entries of $G^*G$ in Formula (\ref{eqn:bestfit}) from Theorem \ref{thm:bestfit} are then the complex dot products between the sequences of complex exponentials appearing as the column vectors of $G$. The orthogonality relation between different column vectors of $G$ is very easy to find. The procedure is similar to what we have done when proving the orthogonality for the real case, but even less tedious. The key point is to write the complex dot product over any pair of two columns as a geometric sequence, that when the values of $k$ are different, will be evaluated to zero. The readers are invited to verify this result, in addition to the fact that the complex dot product between any such a column vector and itself is $N$.\footnotemark{} Thus, the expression $G^*G$ is just $N$ times the identity $I$, and $(G^*G)^{-1} = \frac{1}{N} I$. \par
Now we denote the new, complex best-fit parameters, or coefficients by $C_k$. Subsequently,
\begin{align}
\vec{\beta}_f &= (G^*G)^{-1}G^*\vec{d} = \left(\frac{1}{N}I\right)G^*\vec{d} \nonumber \\
\small
\begin{bmatrix}
C_0 \\
C_1 \\
\vdots \\
C_{\frac{N}{2}-1} \\
C_{-\frac{N}{2}} \\
\vdots \\
C_{-1}
\end{bmatrix}
&= 
\frac{1}{N}
\footnotesize
\begin{bmatrix}
1 & 1 & 1 & \cdots \\
1 & \exp(-i\frac{2\pi}{N}) & \exp(-i\frac{2\pi(2)}{N}) & \cdots \\
\vdots & \vdots & \vdots & \\
1 & \exp(-i\frac{2\pi(\frac{N}{2}-1)}{N}) & \exp(-i\frac{2\pi(\frac{N}{2}-1)(2)}{N}) & \cdots \\
1 & \exp(-i\frac{2\pi(-\frac{N}{2})}{N}) & \exp(-i\frac{2\pi(-\frac{N}{2})(2)}{N}) & \cdots \\
\vdots & \vdots & \vdots & \\
1 & \exp(-i\frac{2\pi(-1)}{N}) & \exp(-i\frac{2\pi(-1)(2)}{N}) & \cdots
\end{bmatrix}
\small
\begin{bmatrix}
f(0) \\
f(1) \\
f(2) \\
\vdots
\end{bmatrix}
\end{align}
Again this is for even $N$, and we ought to replace the indices for coefficients and complex exponentials appropriately for odd $N$. The $\frac{1}{N}$ factor will also be ignored. We now conclude the method of \index{Discrete Fourier Transform (DFT, complex)}\keywordhl{(complex) Discrete Fourier Transform} in a compact way as follows.\footnotetext{When $k \neq l$, we have \begin{align*}
\sum_{t=0}^{N-1} \exp(i (2\pi k/N) t)\overline{\exp(i (2\pi l/N) t)} &= \sum_{t=0}^{N-1} \exp(i (2\pi k/N) t)\exp(-i (2\pi l/N) t) \\
&= \sum_{t=0}^{N-1} \exp(i (2\pi (k-l)/N) t) \\
&= \frac{1-\exp(i (2\pi (k-l)/N))^N}{1-\exp(i (2\pi (k-l)/N))} \\
&= \frac{1-\exp(i (2\pi (k-l)))}{1-\exp(i (2\pi (k-l)/N))} \\
&= \frac{1-1}{1-\exp(i (2\pi (k-l)/N))} = 0
\end{align*}
The $\exp(i (2\pi (k-l)/N))$ term in the denominator will not equal to $1$ and the denominator will not be zero as long as $k \neq l$ and the indices are in the specified range. If $k = l$, then $\exp(i (2\pi k/N) t)\overline{\exp(i (2\pi l/N) t)} =  \exp(i (2\pi k/N) t)\exp(-i (2\pi k/N) t) = 1$, and the sum will be $N$.}
\begin{defn}[Discrete Fourier Transform (DFT)]
\label{defn:complexDFT}
The coefficients, or amplitudes, of the DFT in complex form are computed by
\begin{align}
C_k = \sum_{t=0}^{N-1} f(t)\exp(-i\frac{2\pi k}{N}t) \label{eqn:complexDFTCk}
\end{align}
where $N$ is the number of data. $k$ are all integers ranging from $[-\frac{N}{2}, \frac{N}{2}-1]$ for even $N$, and $[-\frac{N-1}{2}, \frac{N-1}{2}]$ for odd $N$. For a time step $\Delta t$ different from $1$ ($t = 0, \Delta t, 2\Delta t, \ldots, (N-1)\Delta t$), the appropriate expression is
\begin{align}
C_k = \sum_{s=0}^{N-1} f(s\Delta t)\exp(-i\frac{2\pi k}{N}s)
\end{align}
We will sometimes denote the series of $C_k$ by $F(k)$ or $\hat{f}(k)$.
\end{defn}
The negative sign inside the complex exponentials in the formula comes from the conjugate transpose required to produce $G^*$. The relation between $C_k$ and the parameters $A_k$, $B_k$ in the real counterpart are inferred from (\ref{eqn:sincoscomplex}) of Properties \ref{proper:sincoscomplex} and comparing the expression for the real case\footnote{\begin{align*}
\frac{C_k + C_{-k}}{N} &= \frac{1}{N}\left(\sum_{t=0}^{N-1} f(t)\exp(-i\frac{2\pi k}{N}t) + \sum_{t=0}^{N-1} f(t)\exp(-i\frac{2\pi (-k)}{N}t)\right) \\ 
&= \frac{1}{N}\left(\sum_{t=0}^{N-1} f(t)\left(\exp(-i\frac{2\pi k}{N}t) + \exp(i\frac{2\pi k}{N}t)\right)\right) \\
&= \frac{2}{N} \sum_{t=0}^{N-1} f(t) \cos(\frac{2\pi k}{N}t) \qquad \text{(by (\ref{eqn:sincoscomplexa}))}
\end{align*}
which is just equal to $A_k$ in Equation (\ref{eqn:protoDFTa}). The derivation is similar for $B_k$.}, that is
\begin{subequations}
\begin{align}
A_k &= \frac{C_k + C_{-k}}{N} \\
B_k &= -\frac{(C_k - C_{-k})}{iN}
\end{align}    
\end{subequations}
for $k \neq 0$. If $N$ is even, then we define $C_{N/2} = 0$ for convenience. Moreover, since sine is an odd function, and cosine is an even function, $\Re(C_k) = \Re(C_{-k})$, and $\Im(C_k) = -\Im(C_{-k})$ if the input signal $f(t)$ is real-valued, or in other words, $C_k$ and $C_{-k}$ are a pair of complex conjugates. And so an alternative relationship between the real and complex DFT is
\begin{proper}
\label{proper:FTrealcomplex}
Given a real time series, the amplitudes $A_k$, $B_k$ in real DFT (\ref{eqn:protoDFTa}) and (\ref{eqn:protoDFTb}), and $C_k$ in complex DFT (\ref{eqn:complexDFTCk}) satisfy the relations
\begin{subequations}
\begin{align}
A_k &= 2\Re(C_k)/N \\
B_k &= -2\Im(C_k)/N
\end{align}    
\end{subequations}
for any $0 \neq \abs{k} < \frac{N}{2}$. Meanwhile, $C_0$ is the same in both types of DFT and when $N$ is even, \smash{$A_{N/2} = \frac{1}{N}C_{-N/2}$}. The $\frac{1}{N}$ factor is optional and depends on if any convention is used.
\end{proper}

\begin{exmp}
\label{exmp:ex11.2.2}
Find the complex DFT for the time series in example \ref{exmp:ex11.2.1}.
\end{exmp}
\begin{solution}
Using Formula (\ref{eqn:complexDFTCk}) in Definition \ref{defn:complexDFT}, we have
\begin{align*}
C_0 &= 4+1+2+3+1 = 11 \\
C_1 &= 4 + 1\exp(-i\frac{2\pi}{5}) + 2\exp(-i\frac{2\pi(2)}{5}) \\
&\quad+ 3\exp(-i\frac{2\pi(3)}{5}) + 1\exp(-i\frac{2\pi(4)}{5}) = 0.573 + 0.588i \\
C_2 &= 4 + 1\exp(-i\frac{2\pi(2)}{5}) + 2\exp(-i\frac{2\pi(2)(2)}{5}) \\
&\quad+ 3\exp(-i\frac{2\pi(2)(3)}{5}) + 1\exp(-i\frac{2\pi(2)(4)}{5}) = 3.927 - 0.951i
\end{align*}
Either by the property just mentioned or a direct computation, $C_{-1} = \overline{C_1}$ and $C_{-2} = \overline{C_2}$ are seen to be the complex conjugates of $C_1$ and $C_2$ respectively. 
\end{solution}
$\blacktriangleright$ Short Exercise: Check if Properties \ref{proper:FTrealcomplex} is true in this example.\footnotemark

\subsection{Inverse DFT}

During the derivation of complex DFT, we have found that $G^*G = I$ (omitting the $N$ factor), and thus the square matrix $G$ is unitary by Definition \ref{defn:unitary}. This further implies that $G^* = G^{-1}$ (and thus $(G^*)^{-1} = G$) is invertible. And since the complex DFT coefficients are given by $\hat{\textbf{f}} = G^*\vec{d}$, we can undo the DFT and recover the original time series by multiplying to the left with the inverse $(G^*)^{-1}$ so that $\vec{d} = (G^*)^{-1}\hat{\textbf{f}} = G \hat{\textbf{f}}$. This means that
\begin{align}
\scriptsize
\begin{bmatrix}
f(0) \\
f(1) \\
f(2) \\
\vdots
\end{bmatrix}    
&= 
\frac{1}{N}
\tiny
\left[\begin{array}{@{\,}wc{6pt}wc{28pt}wc{8pt}wc{50pt}wc{50pt}wc{8pt}wc{38pt}}
1 & 1 & \cdots & 1 & 1 & \cdots & 1 \\[5pt]
1 & \exp(i\frac{2\pi}{N}) & \cdots & \exp(i\frac{2\pi(\frac{N}{2}-1)}{N}) & \exp(i\frac{2\pi(-\frac{N}{2})}{N}) & \cdots & \exp(i\frac{2\pi(-1)}{N}) \\[7pt]
1 & \exp(i\frac{2\pi(2)}{N}) & \cdots & \exp(i\frac{2\pi(\frac{N}{2}-1)(2)}{N}) & \exp(i\frac{2\pi(-\frac{N}{2})(2)}{N}) & \cdots & \exp(i\frac{2\pi(-1)(2)}{N}) \\[7pt]
\vdots & \vdots &  & \vdots & \vdots & & \vdots \\
\end{array}\right]
\scriptsize
\renewcommand\arraystretch{1.15}
\begin{bmatrix}
\hat{f}_0 \\
\hat{f}_1 \\
\vdots \\
\hat{f}_{\frac{N}{2}-1} \\
\hat{f}_{-\frac{N}{2}} \\
\vdots \\
\hat{f}_{-1}
\end{bmatrix}
\end{align}
for even $N$, and can be adapted for odd $N$ with small tweaks similar to those before. It can also be written in a summation form just like the one in Definition \ref{defn:complexDFT} as follows.\footnotetext{We will only check $A_1$ and $B_1$ and leave $A_2$ and $B_2$ to the readers. $2\Re(C_1)/N = 2(0.573)/5 \approx 0.229 = A_1$, $-2\Im(C_1)/N = -2(0.588)/5 \approx -0.235 = B_1$.}
\begin{defn}[Inverse DFT]
\label{defn:iDFT}
The inverse Discrete Fourier Transform is computed by
\begin{align}
f(t) &= \frac{1}{N}\sum_{k= -\lfloor N/2 \rfloor}^{\lfloor (N-1)/2 \rfloor} F(k)\exp(i \frac{2\pi k}{N}t) \label{eqn:invDFT}
\end{align}
Sometimes we use the operator symbol $F^{-1}$ to denote the inverse DFT operation.
\end{defn}
This can also be verified by a direct substitution. Plugging in Equation (\ref{eqn:complexDFTCk}) from Definition \ref{defn:complexDFT} (with a dummy variable $t'$) into R.H.S., we have
\begin{align*}
&\quad \frac{1}{N}\sum_{k= -\lfloor N/2 \rfloor}^{\lfloor (N-1)/2 \rfloor} \left(\sum_{t'=0}^{N-1} f(t')\exp(-i\frac{2\pi k}{N}t')\right) \exp(i\frac{2\pi k}{N}t) \\
&= \frac{1}{N} \sum_{t'=0}^{N-1} f(t') \left(\sum_{k= -\lfloor N/2 \rfloor}^{\lfloor (N-1)/2 \rfloor} \exp(i\frac{2\pi k}{N}(t-t'))\right)
\end{align*}
where
\begin{align}
\sum_{k= -\lfloor N/2 \rfloor}^{\lfloor (N-1)/2 \rfloor} \exp(i\frac{2\pi k}{N}(t-t')) = 
\begin{cases}
N & \text{if $t = t'$} \\
0 & \text{if $t \neq t'$}
\end{cases} \label{eqn:DFTdoublesum}
\end{align}
The first case should be obvious, while the second case is derived in the footnote below.\footnote{
For an integer $t - t' = \Delta t \neq 0$, we have
\begin{align*}
\sum_{k= -\lfloor N/2 \rfloor}^{\lfloor (N-1)/2 \rfloor} \exp(i \frac{2\pi k}{N}\Delta t) 
&= \exp(i \frac{2\pi (-\lfloor \frac{N}{2} \rfloor)}{N}(\Delta t)) \frac{1 - \exp(i \frac{2\pi \Delta t}{N})^N}{1 - \exp(i \frac{2\pi \Delta t}{N})} \\
&= \exp(i \frac{-2\pi \lfloor \frac{N}{2} \rfloor}{N}(\Delta t)) \frac{1 - \exp(i 2\pi \Delta t)}{1 - \exp(i\frac{2\pi \Delta t}{N})} \\
&= \exp(i\frac{-2\pi \lfloor \frac{N}{2} \rfloor}{N}(\Delta t)) \frac{1 - 1}{1 - \exp(i\frac{2\pi \Delta t}{N})} = 0
\end{align*}
as a geometric sum that evaluates to zero. $\Delta t$ cannot be an integer multiple of $N$ as well, otherwise, the denominator will be problematic, but the range of summation prevents this.} Hence
\begin{align*}
&\quad \frac{1}{N} \sum_{t'=0}^{N-1} f(t') \left(\sum_{k= -\lfloor N/2 \rfloor}^{\lfloor (N-1)/2 \rfloor} \exp(i \frac{2\pi k}{N}(t-t'))\right) \\
&= \frac{1}{N} ((0) + \cdots + f(t) (N) + \cdots + (0)) \\
&= f(t)   
\end{align*}
so the R.H.S. indeed reproduces the original function on L.H.S.

\begin{exmp}
Apply inverse DFT on the complex DFT coefficients $C_{-3} = 1, C_{-2} = 1 + i, C_{-1} = 2 - i, C_0 = 3, C_1 = 2 + i, C_2 = 1 - i$ to retrieve the physical time series.
\end{exmp}
\begin{solution}
The time series $f(t)$ will have a period of $6$ where $t = 0,1,2,3,4,5$. By Formula (\ref{eqn:invDFT}) in Definition \ref{defn:iDFT}, we have
\begin{align*}
f(0) &= \frac{1}{N}\sum_{k=-3}^{2} F(k)\exp(i\frac{2\pi k}{N}(0)) \\
&= \frac{1}{6}[C_{-3}\exp(i\frac{2\pi (-3)}{6}(0)) + C_{-2}\exp(i\frac{2\pi (-2)}{6}(0)) \\
&\quad + C_{-1}\exp(i\frac{2\pi (-1)}{6}(0)) + C_{0}\exp(i\frac{2\pi (0)}{6}(0)) \\
&\quad + C_{1}\exp(i\frac{2\pi (1)}{6}(0)) + C_{2}\exp(i\frac{2\pi (2)}{6}(0))] \\
&= \frac{1}{6}[(1) + (1+i) + (2-i) + 3 + (2+i) + (1-i)] = \frac{10}{6} \\
f(1) &= \frac{1}{N}\sum_{k=-3}^{2} F(k)\exp(i\frac{2\pi k}{N}(1)) \\
&= \frac{1}{6}[C_{-3}\exp(i\frac{2\pi (-3)}{6}(1)) + C_{-2}\exp(i\frac{2\pi (-2)}{6}(1)) \\
&\quad + C_{-1}\exp(i\frac{2\pi (-1)}{6}(1)) + C_{0}\exp(i\frac{2\pi (0)}{6}(1)) \\
&\quad + C_{1}\exp(i\frac{2\pi (1)}{6}(1)) + C_{2}\exp(i\frac{2\pi (2)}{6}(1))] \\
&= \frac{1}{6}[(1)(-1) + (1+i)e^{-i \frac{2\pi}{3}} + (2-i)e^{-i \frac{\pi}{3}} + 3 + (2+i)e^{i \frac{\pi}{3}} + (1-i)e^{i \frac{2\pi}{3}}] \\
&= \frac{1}{2} \\
f(2) &= \frac{1}{N}\sum_{k=-3}^{2} F(k)\exp(i\frac{2\pi k}{N}(2)) \\
&= \frac{1}{6}[C_{-3}\exp(i\frac{2\pi (-3)}{6}(2)) + C_{-2}\exp(i\frac{2\pi (-2)}{6}(2)) \\
&\quad + C_{-1}\exp(i\frac{2\pi (-1)}{6}(2)) + C_{0}\exp(i\frac{2\pi (0)}{6}(2)) \\
&\quad + C_{1}\exp(i\frac{2\pi (1)}{6}(2)) + C_{2}\exp(i\frac{2\pi (2)}{6}(2))] \\
&= \frac{1}{6}[(1)(1) + (1+i)e^{-i \frac{4\pi}{3}} + (2-i)e^{-i \frac{2\pi}{3}} + 3 + (2+i)e^{i \frac{2\pi}{3}} + (1-i)e^{i \frac{4\pi}{3}}] \\
&\approx -0.4107 
\end{align*}
We leave the calculations for the remaining three data points to the readers. They are $f(3) = 0$, $f(4) \approx 0.7440$, and $f(5) = \frac{1}{2}$.
\end{solution}

\section{Properties of DFT}

\subsection{Power Spectrum and Parseval's Theorem}

The complex DFT coefficients actually store the relevant information about the sinusoidal signals in the corresponding frequencies, namely the \textit{phase} and \textit{amplitude}. To see this, go back to Properties \ref{proper:FTrealcomplex}, where
\begin{subequations}
\label{eqn:FTrealcomplex2}
\begin{align}
A_k &= 2\Re(C_k)/N \\
B_k &= -2\Im(C_k)/N 
\end{align}    
\end{subequations}
but also, the wave signal of that particular frequency takes the form of
\begin{align}
\label{eqn:FTwave}
A_k \cos(\frac{2\pi k}{N}t) + B_k \sin(\frac{2\pi k}{N}t)
\end{align}
Plugging (\ref{eqn:FTrealcomplex2}) into (\ref{eqn:FTwave}) above gives
\begin{align}
&\quad \frac{2}{N} \left(\Re(C_k)\cos(\frac{2\pi k}{N}t) - \Im(C_k) \sin(\frac{2\pi k}{N}t)\right) \nonumber \\
&= \frac{2}{N} \left(\hat{A}_k \cos \phi_k \cos(\frac{2\pi k}{N}t) - \hat{A}_k\sin\phi_k \sin(\frac{2\pi k}{N}t)\right) \nonumber  \\
&= \frac{2}{N} \hat{A}_k \cos(\frac{2\pi k}{N}t + \phi_k)
\end{align}
where we let $-\pi \leq \phi_k < \pi$ and $\hat{A}_k \geq 0$ as real-valued quantities, in a way such that 
\begin{subequations}
\begin{align}
\Re(C_k) &= \hat{A}_k \cos \phi_k \\
\text{and} \quad \Im(C_k) &= \hat{A}_k \sin \phi_k   
\end{align}   
\end{subequations}
and then apply the trigonometric identity $\cos(\alpha + \beta) = \cos\alpha \cos\beta - \sin\alpha \sin\beta$. $\phi_k$ and $\hat{A}_k$ will be the \index{Phase}\textit{phase} (relative to a cosine wave) and the \index{Amplitude}\textit{amplitude} of the signal (again putting the $\frac{2}{N}$ factor aside) at the $k$-th frequency respectively. The required values of $\phi_k$ and $\hat{A}_k$ are then derived according to
\begin{subequations}
\begin{align}
\frac{\Im(C_k)}{\Re(C_k)} &= \frac{\sin \phi_k}{\cos \phi_k} = \tan \phi_k \\
\Re(C_k)^2 + \Im(C_k)^2 &= \hat{A}_k^2 \cos^2 \phi_k + \hat{A}_k^2 \sin^2 \phi_k = \hat{A}_k^2 
\end{align}
\end{subequations}
Thus 
\begin{subequations}
\begin{align}
\phi_k &= \arctan(\frac{\Im(C_k)}{\Re(C_k)}) \\
\text{and} \quad \hat{A}_k &= \sqrt{\Re(C_k)^2 + \Im(C_k)^2}  
\end{align}   
\end{subequations}
Recalling (\ref{eqn:modulus}) and (\ref{eqn:argument}), these are exactly the argument and modulus of $C_k$. Therefore, by simply looking at the complex DFT coefficient $C_k$ we can readily extract the phase and amplitude of the corresponding DFT signal. However, in the area of signal processing, we often report the \index{Power}\textit{power} of the signal instead of its amplitude for convenience, which is just the square of the amplitude. It can be easily obtained from
\begin{align}
\abs{C_k}^2 = C_k\overline{C_k} = C_kC_{-k}
\end{align}
due to Equation (\ref{eqn:zzbar}) and the fact that $C_k$ has $C_{-k}$ as its complex conjugate (if the input time series is real-valued). The powers over all frequencies are collectively referred to as the \index{Power Spectrum}\keywordhl{power spectrum}.

\begin{exmp}
Find the phase and power at each frequency bin for the complex DFT computed in Example \ref{exmp:ex11.2.2}.   
\end{exmp}
\begin{solution}
The phase of the zeroth frequency (i.e.\ constant term) signal is trivially zero, and its power is simply \smash{$C_0^2 = (11)^2 = 121$}. For the first (base) frequency, the phase is
\begin{align*}
\phi_1 = \arctan(\frac{\Im(C_1)}{\Re(C_1)}) = \arctan(\frac{0.588}{0.573}) = \SI{0.80}{\radian}
\end{align*}
and the power is
\begin{align*}
\abs{C_1}^2 = \abs{C_{-1}}^2 = C_1C_{-1} &= (0.573 + 0.588i)(0.573 - 0.588i) \\
&= 0.67
\end{align*}
Similarly, for the second frequency, the phase is
\begin{align*}
\phi_2 = \arctan(\frac{\Im(C_2)}{\Re(C_2)}) = \arctan(\frac{-0.951}{3.927}) = \SI{-0.24}{\radian}
\end{align*}
and the power is
\begin{align*}
\abs{C_2}^2 = \abs{C_{-2}}^2 = C_2C_{-2} &= (3.927 - 0.951i)(3.927 + 0.951i) \\
&= 16.33
\end{align*}
\end{solution}
The full power spectrum across all frequencies is related to the initial time series according to the \index{Parseval's Theorem}\keywordhl{Parseval's Theorem}.
\begin{thm}[Parseval's Theorem]
\label{thm:DFTParseval}
The sum of powers in each DFT coefficient, divided by the length $N$, is equal to the sum of all data values in the (real) time series squared, i.e.\
\begin{align}
\sum_{t=0}^{N-1} f(t)^2 &= \frac{1}{N} \sum_{k= -\lfloor N/2 \rfloor}^{\lfloor (N-1)/2 \rfloor} \abs{F(k)}^2 \label{eqn:Parseval}
\end{align}
\end{thm}
\begin{proof}
By substituting the form of inverse DFT given as (\ref{eqn:invDFT}) in Definition \ref{defn:iDFT}, the L.H.S. becomes 
\begin{align*}
&\quad \sum_{t=0}^{N-1} f(t)^2 \\
&= \sum_{t=0}^{N-1} \left[ \frac{1}{N} \sum_{k= -\lfloor N/2 \rfloor}^{\lfloor (N-1)/2 \rfloor} F(k)\exp(i\frac{2\pi k}{N}t) \left( \overline{ \frac{1}{N}\sum_{l= -\lfloor N/2 \rfloor}^{\lfloor (N-1)/2 \rfloor} F(l)\exp(i\frac{2\pi l}{N}t)} \right) \right] \\
&= \frac{1}{N^2}\sum_{k= -\lfloor N/2 \rfloor}^{\lfloor (N-1)/2 \rfloor} F(k) \left(\sum_{t=0}^{N-1} \exp(i\frac{2\pi k}{N}t) \left( \sum_{l= -\lfloor N/2 \rfloor}^{\lfloor (N-1)/2 \rfloor} \overline{F(l)}\exp(-i\frac{2\pi l}{N}t) \right) \right) \\
&= \frac{1}{N^2} \sum_{k= -\lfloor N/2 \rfloor}^{\lfloor (N-1)/2 \rfloor} F(k) \left( \sum_{l= -\lfloor N/2 \rfloor}^{\lfloor (N-1)/2 \rfloor}  \sum_{t=0}^{N-1} \overline{F(l)}\exp(i \frac{2\pi (k-l)}{N}t) \right) 
\end{align*}
where we have used two different dummy indices $k$ and $l$ initially. Similar to (\ref{eqn:DFTdoublesum}), we have
\begin{align}
\sum_{t=0}^{N-1} \exp(i \frac{2\pi (k-l)}{N}t) = 
\begin{cases}
N & \text{if $k = l$} \\
0 & \text{if $k \neq l$}
\end{cases}     
\end{align}
Thus
\begin{align*}
&\quad \frac{1}{N^2}  \sum_{k= -\lfloor N/2 \rfloor}^{\lfloor (N-1)/2 \rfloor} F(k) \left( \sum_{l= -\lfloor N/2 \rfloor}^{\lfloor (N-1)/2 \rfloor}  \sum_{t=0}^{N-1} \overline{F(l)}\exp(i \frac{2\pi (k-l)}{N}t) \right) \\
&= \frac{1}{N^2}  \sum_{k= -\lfloor N/2 \rfloor}^{\lfloor (N-1)/2 \rfloor} F(k) [(0) + \cdots + (\overline{F(k)})(N) + \cdots + (0)] \\
&= \frac{1}{N^2}  \sum_{k= -\lfloor N/2 \rfloor}^{\lfloor (N-1)/2 \rfloor} N F(k)\overline{F(k)} \\
&= \frac{1}{N} \sum_{k= -\lfloor N/2 \rfloor}^{\lfloor (N-1)/2 \rfloor} \abs{F(k)}^2
\end{align*} 
is equal to the R.H.S.
\end{proof}

\begin{exmp}
Verify Parseval's Theorem for Examples \ref{exmp:ex11.2.1} and \ref{exmp:ex11.2.2}.
\end{exmp}
\begin{solution}
The L.H.S. of Equation (\ref{eqn:Parseval}) in Theorem \ref{thm:DFTParseval} applied to the time series of Example \ref{exmp:ex11.2.1}, is simply
\begin{align*}
\sum_{t=0}^{4} f(t)^2 &= f(0)^2 + f(1)^2 + f(2)^2 + f(3)^2 + f(4)^2 \\
&= (4)^2 + (1)^2 + (2)^2 + (3)^2 + (1)^2 = 31    
\end{align*}
and from Example \ref{exmp:ex11.2.2}, the R.H.S. is
\begin{align*}
\frac{1}{5}\sum_{k=-2}^{2} \abs{F(k)}^2 &= \frac{1}{5}(\abs{C_{-2}}^2 + \abs{C_{-1}}^2 + \abs{C_{0}}^2 + \abs{C_{1}}^2 + \abs{C_{2}}^2) \\
&= \frac{1}{5}(16.33 + 0.67 + 121 + 0.67 + 16.33) \\
&= \frac{1}{5} (155) = 31
\end{align*}
which are indeed equal.
\end{solution}

\subsection{Convolution Theorem}

\index{Convolution}\keywordhl{Convolution} often appears along with Fourier Transform due to an elegant theorem that will be introduced soon. In the area of Earth Science, as well as Physics and Statistics, convolution commonly has a place in describing the solution of problems, e.g.\ via \textit{Green's Function}. Now we are going to introduce how convolution in a discrete sense is defined first, and a daily example will be provided as an illustration. It always involves two time series or functions.
\begin{defn}[Discrete Convolution]
\label{defn:convolution}
The convolution $h(t)$ between two discrete time series $f(t)$ and $g(t)$ is written as $f(t) * g(t)$, defined by
\begin{align}
h(t) = f(t) * g(t) = \sum_{\tau=-\infty}^{\infty} f(\tau) g(t-\tau) \label{eqn:convolutiondefn}
\end{align}
if $\tau$ takes a value so that the index in either $f$ or $g$ is out of their range, then the term is treated as zero.
\end{defn}
A schematic diagram of convolution is provided as Figure \ref{fig:convschm}. Note that the formula for convolution is symmetric such that we can also define it as $\sum_{\tau=-\infty}^{\infty} f(t-\tau) g(\tau)$. Now, for instance, if $f(t)$ is defined from $t = 0$ and $t = 4$ and $g(t)$ is defined from $t = 0$ and $t = 6$ so that they have a length of $5$ and $7$ respectively, then
\begin{align*}
h(0) &= f(0)g(0) \\
h(1) &= f(0)g(1-0) + f(1)g(1-1) = f(0)g(1) + f(1)g(0) \\
h(4) &= f(0)g(4-0) + f(1)g(4-1) + f(2)g(4-2) \\
&\quad + f(3)g(4-3) + f(4)g(4-4) \\
&= f(0)g(4) + f(1)g(3) + f(2)g(2) + f(3)g(1) + f(4)g(0)
\end{align*}
and
\begin{align*}
h(6) &= f(0)g(6-0) + f(1)g(6-1) + f(2)g(6-2) \\
&\quad + f(3)g(6-3) + f(4)g(6-4) \\
&= f(0)g(6) + f(1)g(5) + f(2)g(4) + f(3)g(3) + f(4)g(2) \\
h(9) &=  f(3)g(6-0) + f(4)g(6-1) = f(3)g(6) + f(4)g(5) \\
h(10) &= f(4)g(6)
\end{align*}
Moreover, $h(t) = 0$ for $t \geq 11$ or $t < 0$ so the effective length of $h(t)$ is $11$. It is not hard to deduce that the resulting convolution will have a length of $m + n - 1$ if the two input time series have a length of $m$ and $n$.

\begin{figure}
    \centering
    \begin{tikzpicture}
    \draw[blue, thick] (-3,4+0.5) -- (0,4+0.5) -- (0,6+0.5) -- (2,6+0.5) -- (2,4+0.5) -- (5,4+0.5);
    \node[blue] at (-1,6+0.5) {$f$};
    \draw[red, thick] (-3,0+0.5) -- (0,0+0.5) -- (2,2+0.5) -- (2,0+0.5) -- (5,0+0.5);
    \node[red] at (-1,2+0.5) {$g$};
    \draw[Green, thick] (-3,-4+0.5) -- (-1,-4+0.5) -- plot[domain=-1:1,samples=50] (\x, {0.5*(\x+1)^2 - 4 +0.5}) -- plot[domain=1:3,samples=50] (\x, {-0.5*(\x-1)^2 - 2 +0.5}) -- (5,-4+0.5);
    \node[Green] at (-1,-2+0.5) {$f*g$};
    \draw[blue, thick] (-1,-9) -- (0.5,-9) -- (0.5,-8) -- (1.5,-8) -- (1.5,-9) -- (3,-9);
    \draw[red, thick] (-1,-9) -- (0.5,-9) -- (0.5,-8) -- (1.5,-9) -- (3,-9);
    \fill[Green!50] (0.5,-9) -- (0.5,-8) -- (1.5,-9) -- cycle;
    \draw[blue, thick] (-1-1.5,-9+1.75) -- (0.5-1.5,-9+1.75) -- (0.5-1.5,-8+1.75) -- (1.5-1.5,-8+1.75) -- (1.5-1.5,-9+1.75) -- (2-1.5,-9+1.75);
    \draw[red, thick] (-1-1.5-0.5,-9+1.75) -- (0.5-1.5-0.5,-9+1.75) -- (0.5-1.5-0.5,-8+1.75) -- (1.5-1.5-0.5,-9+1.75) -- (2-1.5-0.5,-9+1.75);
    \fill[Green!50] (0.5-1.5,-9+1.75) -- (0.5-1.5,-8+1.75-0.5) -- (1.5-1.5-0.5,-9+1.75) -- cycle;
    \draw[blue, thick] (0+1.5,-9+1.75) -- (0.5+1.5,-9+1.75) -- (0.5+1.5,-8+1.75) -- (1.5+1.5,-8+1.75) -- (1.5+1.5,-9+1.75) -- (3+1.5,-9+1.75);
    \draw[red, thick] (0+1.5+0.5,-9+1.75) -- (0.5+1.5+0.5,-9+1.75) -- (0.5+1.5+0.5,-8+1.75) -- (1.5+1.5+0.5,-9+1.75) -- (3+1.5+0.5,-9+1.75);
    \fill[Green!50] (0.5+1.5+0.5,-9+1.75) -- (0.5+1.5+0.5,-8+1.75) -- (1.5+1.5,-8+1.75-0.5) -- (1.5+1.5,-9+1.75) -- cycle;
    \draw[blue, thick] (-1-2,-9+3.5) -- (0.5-2,-9+3.5) -- (0.5-2,-8+3.5) -- (1.5-2,-8+3.5) -- (1.5-2,-9+3.5) -- (2-1.5,-9+3.5);
    \draw[red, thick] (-1-2-1,-9+3.5) -- (0.5-2-1,-9+3.5) -- (0.5-2-1,-8+3.5) -- (1.5-2-1,-9+3.5) -- (2.5-2-1,-9+3.5);
    \draw[blue, thick] (-0.5+2,-9+3.5) -- (0.5+2,-9+3.5) -- (0.5+2,-8+3.5) -- (1.5+2,-8+3.5) -- (1.5+2,-9+3.5) -- (3+2,-9+3.5);
    \draw[red, thick] (-0.5+2+1,-9+3.5) -- (0.5+2+1,-9+3.5) -- (0.5+2+1,-8+3.5) -- (1.5+2+1,-9+3.5) -- (3+2+1,-9+3.5);
    \draw[Green, -{Latex[length=4mm, width=2mm]}] (1, -7.9) -- (1,-2+0.5);
    \draw[Green, -{Latex[length=2mm, width=1mm]}] (-1, -4.4) -- (-1,-4+0.5);
    \draw[Green, -{Latex[length=2mm, width=1mm]}] (3, -4.4) -- (3,-4+0.5);
    \draw[Green, -{Latex[length=3mm, width=1.5mm]}] (0, -6.15) -- (0,-3.5+0.5);
    \draw[Green, -{Latex[length=3mm, width=1.5mm]}] (2, -6.15) -- (2,-2.5+0.5);
    \end{tikzpicture}
    \caption{\textit{An example schematic of convolution.}}
    \label{fig:convschm}
\end{figure}

\begin{exmp}
The probability of Mary getting married within some years can be described by the time series $q(t)$, where
\begin{align*}
q(t) = 
\begin{cases}
0.2 & 0 \leq t \leq 2 \\
0.1 & 3 \leq t \leq 6 \\
0 & t \geq 7
\end{cases}
\end{align*}
The probability $r(t)$ that Mary gives birth to a baby some years $t$ after marriage is
\begin{align*}
r(t) = 
\begin{cases}
0 & t = 0 \\
0.15 & 1 \leq t \leq 4 \\
0.08 & 5 \leq t \leq 9 \\
0 & t \geq 10
\end{cases}    
\end{align*}
Find the net probability of Mary getting a baby some years $t$ from now on, assuming she will only get pregnant after marriage. 
\end{exmp}
\begin{solution}
The required probability $p(t) = q(t) * r(t)$ is actually given by the convolution between $q(t)$ and $r(t)$, which effectively have a length of $7$ and $9$. Particularly, taking $t = 5$ as an example, then we have
\begin{align*}
p(t = 5) &= P(\text{Birth 5 yrs later}|\text{Married now})P(\text{Married now}) \\
&\quad+ P(\text{Birth 4 yrs later}|\text{Married 1 yr later})P(\text{Married 1 yr later})\\
&\quad+ \cdots \\
&\quad+ P(\text{Birth 1 yr later}|\text{Married 4 yrs later})P(\text{Married 4 yrs later}) \\
&\quad+ P(\text{Birth now}|\text{Married 5 yrs later})P(\text{Married 5 yrs later}) \\
&= q(0)r(5) + q(1)r(4) + q(2)r(3) \\
&\quad + q(3)r(2) + q(4)r(1) + q(5)r(0) \\
&= (0.2)(0.08) + (0.2)(0.15) + (0.2)(0.15) \\
&\quad + (0.1)(0.15) + (0.1)(0.15) + (0.1)(0) \\
&= 0.106
\end{align*}
where $P(A|B)$ is the conditional probability of $A$ occurring when $B$ has happened.
\end{solution}
$\blacktriangleright$ Short Exercise: Find the chance of Mary getting a baby at $8$ years later by convolution.\footnote{It is $q(0)r(8) + q(1)r(7) + q(2)r(6) + q(3)r(5) + q(4)r(4) + q(5)r(3) + q(6)r(2) = (0.2)(0.08) + (0.2)(0.08) + (0.2)(0.08) + (0.1)(0.08) + (0.1)(0.15) + (0.1)(0.15) + (0.1)(0.15) = 0.101$.}\par

Since Fourier analysis essentially treats the functions or time series as periodic, it happens that it is more useful to define the \index{Circular Convolution}\keywordhl{circular convolution} for two time series of the same length, where we wrap either one of the input time series in a cyclic manner (see Figure \ref{fig:circconvschm}).
\begin{defn}[Circular Convolution]
The circular convolution \smash{$h_c(t)$} between two discrete time series $f(t)$ and $g(t)$ that are of the same length $N$ is another time series also has a length of $N$, defined as
\begin{align}
h_c(t) = f(t) \circledast g(t) = \sum_{\tau=0}^{N-1} f(\tau) g((t-\tau) \bmod N)
\end{align}
where $m \bmod N$ is the \textit{modulo} operation that returns an integer $0 \leq n < N$ between $0$ and $N-1$ such that $m + qN = n$ for some integer $q$. In this context
\begin{align}
(t-\tau) \bmod N = 
\begin{cases}
t - \tau & \text{if $N > t \geq \tau$} \\
t + N - \tau & \text{if $0\leq t < \tau$}
\end{cases}
\end{align}
Alternatively, one can manipulate the periodic extension of $f(t)$ and $g(t)$ by repeating them indefinitely to produce $f_\text{px}(t)$ and $g_\text{px}(t)$ such that 
\begin{align}
h_c(t) = f(t) \circledast g(t) = \sum_{\tau=-\infty}^{\infty} f_\text{px}(\tau) g_\text{px}(t-\tau)
\end{align}
so that it coincides with Definition \ref{defn:convolution} and we only take the values from $h_c(t)$ within $0 \leq t < N$.
\end{defn}
Circular convolution is also symmetric so that $\sum_{\tau=0}^{N-1} f((t-\tau) \bmod N) g(\tau)$ works fine as well. With circular convolution properly defined, we can go ahead to derive the main result, the \index{Convolution Theorem (Circular)}\keywordhl{(Circular) Convolution Theorem}.
\begin{figure}[t!]
    \centering
    \begin{tikzpicture}
    \draw[blue, thick] plot[domain=-2:2,samples=50] (\x, {2-0.5*(\x)^2});
    \node[blue] at (-2.5,2) {$f$};
    \draw[red, thick] (-2,-3) -- (2,-1) -- (2,-3);
    \node[red] at (-2.5,-1) {$g$};
    \draw[Green, thick, ->] (3,6) -- (9,6) node[midway, above]{$f \circledast g$, Cyclic Direction};
    \draw[blue, thick] plot[domain=4:8,samples=50] (\x, {5-0.5*(\x-6)^2});
    \draw[red, thick] (4,3) -- (4,5) -- (8,3);
    \draw[blue, thick] plot[domain=4:8,samples=50] (\x, {2-0.5*(\x-6)^2});
    \draw[red, thick] (4,0.5) -- (5,0) -- (5,2) -- (8,0.5);
    \draw[blue, thick] plot[domain=4:8,samples=50] (\x, {-1-0.5*(\x-6)^2});
    \draw[red, thick] (4,-2) -- (6,-3) -- (6,-1) -- (8,-2);
    \draw[blue, thick] plot[domain=4:8,samples=50] (\x, {-4-0.5*(\x-6)^2});
    \draw[red, thick] (4,-4.5) -- (7,-6) -- (7,-4) -- (8,-4.5);
    \end{tikzpicture}
    \caption{\textit{An illustration for circular convolution.}}
    \label{fig:circconvschm}
\end{figure}
\begin{thm}[Circular Convolution Theorem]
\label{thm:circonv}
For two time series $f(t)$ and $g(t)$ of the same length $N$, denote their DFT by $\hat{f}(k)$ and $\hat{g}(k)$. Then
\begin{align}
F[f \circledast g](k) = \hat{f}(k)\hat{g}(k)
\end{align}
\end{thm}
This means that convolution in the physical/time domain is equivalent to component-wise multiplication in the frequency domain.
\begin{proof}
We will start from the L.H.S. and show that it is the same as R.H.S.
\begin{align*}
&\quad F[f \circledast g](k) \\
&= \sum_{t=0}^{N-1} \left(\sum_{\tau=0}^{N-1} f(\tau) g((t-\tau) \bmod N)\right) \exp(-i\frac{2\pi k}{N}t) \\
&= \sum_{\tau=0}^{N-1} \sum_{t=0}^{N-1} \left[f(\tau) g((t-\tau) \bmod N) \exp(-i\frac{2\pi k}{N}(t-\tau)) \exp(-i\frac{2\pi k}{N}\tau)\right] \\
&= \sum_{\tau=0}^{N-1} \sum_{t'=-\tau}^{N-1-\tau}  f(\tau) g(t' \bmod N) \exp(-i\frac{2\pi k}{N}t') \exp(-i\frac{2\pi k}{N}\tau) \\
&= \sum_{\tau=0}^{N-1} f(\tau) \left(\sum_{t'=-\tau}^{N-1-\tau} g(t' \bmod N) \exp(-i\frac{2\pi k}{N}t')\right) \exp(-ih\frac{2\pi k}{N}\tau)
\end{align*}
where we have made a change of variable from $t$ to $t' = t - \tau$ with $\tau$ fixed. Note that the bracket term
\begin{align*}
&\quad \sum_{t'=-\tau}^{N-1-\tau} g(t' \bmod N) \exp(-i\frac{2\pi k}{N}t') \\
&= \sum_{t'=-\tau}^{-1} g(t' \bmod N) \exp(-i\frac{2\pi k}{N}t') +  \sum_{t'=0}^{N-1-\tau} g(t' \bmod N) \exp(-i\frac{2\pi k}{N}t') \\
&= \sum_{t'=N-\tau}^{N-1} g((t'-N) \bmod N) \exp(-i\frac{2\pi k}{N}(t'-N)) \\
&\quad + \sum_{t'=0}^{N-\tau-1} g(t') \exp(-i\frac{2\pi k}{N}t') \\
&= \sum_{t'=N-\tau}^{N-1} g(t') \exp(-i\frac{2\pi k}{N}t') + \sum_{t'=0}^{N-\tau-1} g(t') \exp(-i\frac{2\pi k}{N}t') \\
&= \sum_{t'=0}^{N-1} g(t') \exp(-i\frac{2\pi k}{N}t') = \hat{g}(k)
\end{align*}
where from the second line to third line we replace $t'$ by $t' - N$ in the first term and note that $(t' - N) \bmod N = t'$, $\smash{\exp(-i\frac{2\pi k}{N}(t'-N))} = \smash{\exp(-i(\frac{2\pi k}{N}t' - 2\pi k))} = \smash{\exp(-i\frac{2\pi k}{N}t')}$. In the last line, we simply invoke Definition \ref{defn:complexDFT}. Subsequently,
\begin{align*}
&\quad \sum_{\tau=0}^{N-1} f(\tau) \left(\sum_{t'=-\tau}^{N-1-\tau} g(t' \bmod N) \exp(-i\frac{2\pi k}{N}t')\right) \exp(-i\frac{2\pi k}{N}\tau) \\
&= \sum_{\tau=0}^{N-1} f(\tau) \hat{g}(k) \exp(-i\frac{2\pi k}{N}\tau) \\
&= \left(\sum_{\tau=0}^{N-1} f(\tau) \exp(-i\frac{2\pi k}{N}\tau)\right) \hat{g}(k) = \hat{f}(k)\hat{g}(k)
\end{align*}
where we use Definition \ref{defn:complexDFT} again and the desired equality is established.
\end{proof}

\begin{exmp}
Let $f(t)$ be the time series in Example \ref{exmp:ex11.2.1}, and $g(t)$ be another time series with $g(0) = 1$, $g(1) = 4$, $g(2) = 0$, $g(3) = 2$, $g(4) = -1$. Verify the Circular Convolution Theorem for these two time series.
\end{exmp}
\begin{solution}
$\hat{f}(k)$ has been computed in Example \ref{exmp:ex11.2.2} as $(11, 0.573+0.588i, \allowbreak 3.927-0.951i, 3.927+0.951i, 0.573-0.588i)$. Meanwhile,
\begin{align*}
\hat{g}(0) &= 1+4+0+2+(-1) = 6 \\
\hat{g}(1) &= 1 + 4\exp(-i\frac{2\pi}{5}) + 0\exp(-i\frac{2\pi(2)}{5}) \\
&\quad + 2\exp(-i\frac{2\pi(3)}{5}) + (-1)\exp(-i\frac{2\pi(4)}{5}) = 0.309-3.580i \\
\hat{g}(2) &= 1 + 4\exp(-i\frac{2\pi(2)}{5}) + 0\exp(-i\frac{2\pi(2)(2)}{5}) \\
&\quad+ 2\exp(-i\frac{2\pi(2)(3)}{5}) + (-1)\exp(-i\frac{2\pi(2)(4)}{5}) = -0.809 - 4.841i
\end{align*}
and hence $\hat{g}(k) = (6, 0.309-3.580i, -0.809-4.841i, -0.809+4.841i, \allowbreak 0.309+3.580i)$. The circular convolution $f(t) \circledast g(t)$ is found by
\begin{align*}
(f \circledast g)(0) &= f(0)g(0) + f(1)g(4) + f(2)g(3) + f(3)g(2) + f(4)g(1) \\
&= (4)(1) + (1)(-1) + (2)(2) + (3)(0) + (1)(4) = 11 \\
(f \circledast g)(1) &= f(0)g(1) + f(1)g(0) + f(2)g(4) + f(3)g(3) + f(4)g(2) \\
&= (4)(4) + (1)(1) + (2)(-1) + (3)(2) + (1)(0) = 21 \\
(f \circledast g)(2) &= f(0)g(2) + f(1)g(1) + f(2)g(0) + f(3)g(4) + f(4)g(3) \\
&= (4)(0) + (1)(4) + (2)(1) + (3)(-1) + (1)(2) = 5
\end{align*}
We leave to the readers to obtain $(f \circledast g)(3) = 18$ and $(f \circledast g)(4) = 11$. Thus $(f \circledast g)(t) = (11, 21, 5, 18, 11)$ and by Definition \ref{defn:complexDFT} its DFT is
\begin{align*}
(\widehat{f \circledast g})(0) &= 11+21+5+18+11 = 66 \\
(\widehat{f \circledast g})(1) &= 11 + 21\exp(-i\frac{2\pi}{5}) + 5\exp(-i\frac{2\pi(2)}{5}) \\
&\quad + 18\exp(-i\frac{2\pi(3)}{5}) + 11\exp(-i\frac{2\pi(4)}{5}) = 2.281 - 1.869i \\
(\widehat{f \circledast g})(2) &= 11 + 21\exp(-i\frac{2\pi(2)}{5}) + 5\exp(-i\frac{2\pi(2)(2)}{5}) \\
&\quad+ 18\exp(-i\frac{2\pi(2)(3)}{5}) + 11\exp(-i\frac{2\pi(2)(4)}{5}) \\
&= -7.781 - 18.242i
\end{align*}
Therefore, we can verify that
\begin{align*}
&\quad F[f \circledast g](k) \\
&= (66, 2.281 - 1.869i, -7.781 - 18.242i, -7.781 + 18.242i, 2.281 + 1.869i) \\
&= (11, 0.573+0.588i, 3.927-0.951i, 3.927+0.951i, 0.573-0.588i) \\
&\quad \odot (6, 0.309-3.580i, -0.809-4.841i, -0.809+4.841i, 0.309+3.580i) \\
&= \hat{f}(k)\hat{g}(k)
\end{align*}
Theorem \ref{thm:circonv} holds, where $\odot$ represents component-wise multiplication.
\end{solution}

\section{Fast Fourier Transform (FFT)}

The naive way to compute DFT following Definition \ref{defn:complexDFT} is essentially evaluating the matrix product $G^* \vec{d}$ aforementioned. If there are $N$ data points then $G^*$ will be an $n \times n$ matrix (and $\vec{d}$ will be a vector of length $N$) and the calculation will involve $N^2$ multiplications and $N(N-1)$ additions, ultimately leading to a \textit{time complexity} of $O(N^2)$. This means that as the number of data $N$ goes up, the amount of computation required increases quadratically, which is quite inefficient. Therefore, people have been developing alternative methods to compute DFT more efficiently, and these algorithms are collectively known as \index{Fast Fourier Transform (FFT)}\keywordhl{Fast Fourier Transform (FFT)}. The first and the most famous one among them is the \index{Radix-2 Algorithm}\keywordhl{Radix-2 Algorithm}, which utilizes the strategy of \textit{bisection} to cut the calculation into halves recursively. To understand how it works, first notice that we can rewrite the matrix $G^*$ into a nicer form as
\begin{align}
G^* = 
\begin{bmatrix}
1 & 1 & 1 & 1 & \cdots & 1 \\
1 & \omega_N & \omega_N^2 & \omega_N^3 & \cdots & \omega_N^{N-1} \\
1 & \omega_N^2 & \omega_N^4 & \omega_N^6 & \cdots & \omega_N^{2(N-1)} \\
\vdots & \vdots & \vdots & \vdots & & \vdots \\
1 & \omega_N^{N-1} & \omega_N^{2(N-1)} & \omega_N^{3(N-1)} & \cdots & \omega_N^{(N-1)^2}
\end{bmatrix}
\end{align}
where $\omega_N = \smash{\exp(-i\frac{2\pi}{N})}$ is the \index{Fundamental Frequency}\textit{base/fundamental frequency} (see the coming Footnote \ref{foot:DFTcyclicexp}). The algorithm works most effectively when $N$ is the power of $2$. For an easier explanation, we consider a smaller value of $N = 2^3 = 8$. Then we denote
\begin{align}
\renewcommand\arraystretch{1.25}
G^* = F_8 = 
\footnotesize
\begin{bmatrix}
1 & 1 & 1 & 1 & 1 & 1 & 1 & 1 \\
1 & \omega_8 & \omega_8^2 & \omega_8^3 & \omega_8^4 & \omega_8^5 & \omega_8^6 & \omega_8^7 \\
1 & \omega_8^2 & \omega_8^4 & \omega_8^6 & \omega_8^8 & \omega_8^{10} & \omega_8^{12} & \omega_8^{14} \\
1 & \omega_8^3 & \omega_8^6 & \omega_8^9 & \omega_8^{12} & \omega_8^{15} & \omega_8^{18} & \omega_8^{21} \\
1 & \omega_8^4 & \omega_8^8 & \omega_8^{12} & \omega_8^{16} & \omega_8^{20} & \omega_8^{24} & \omega_8^{28} \\
1 & \omega_8^5 & \omega_8^{10} & \omega_8^{15} & \omega_8^{20} & \omega_8^{25} & \omega_8^{30} & \omega_8^{35} \\
1 & \omega_8^6 & \omega_8^{12} & \omega_8^{18} & \omega_8^{24} & \omega_8^{30} & \omega_8^{36} & \omega_8^{42} \\
1 & \omega_8^7 & \omega_8^{14} & \omega_8^{21} & \omega_8^{28} & \omega_8^{35} & \omega_8^{42} & \omega_8^{49} 
\end{bmatrix}
\end{align}
and the DFT is given by $\hat{\textbf{f}} = F_8 \textbf{f}$. However, we can shuffle $\textbf{f}$ such that the interleaving even and odd indices are split into two groups, and it can be shown that
\begin{align}
\hat{\textbf{f}} &= 
\begin{bmatrix}
I_4 & D_{4 \to 8} \\
I_4 & -D_{4 \to 8}
\end{bmatrix}
\begin{bmatrix}
F_4 & 0 \\\
0 & F_4
\end{bmatrix}
\begin{bmatrix}
\textbf{f}_{\text{even}} \\
\textbf{f}_{\text{odd}}
\end{bmatrix}
\label{eqn:FFT48}
\end{align}
where
\begin{align}
D_{4 \to 8} = 
\begin{bmatrix}
1 & 0 & 0 & 0 \\
0 & \omega_8 & 0 & 0 \\
0 & 0 & \omega_8^2 & 0 \\
0 & 0 & 0 & \omega_8^3
\end{bmatrix}
\end{align}
and 
\begin{align}
F_4 = 
\begin{bmatrix}
1 & 1 & 1 & 1 \\
1 & \omega_4 & \omega_4^2 & \omega_4^3 \\
1 & \omega_4^2 & \omega_4^4 & \omega_4^6 \\
1 & \omega_4^3 & \omega_4^6 & \omega_4^9 \\
\end{bmatrix}
=
\begin{bmatrix}
1 & 1 & 1 & 1 \\
1 & \omega_8^2 & \omega_8^4 & \omega_8^6 \\
1 & \omega_8^4 & \omega_8^8 & \omega_8^{12} \\
1 & \omega_8^6 & \omega_8^{12} & \omega_8^{18} \\
\end{bmatrix}
\end{align}
where $\omega_4 = \smash{\exp(-i\frac{2\pi}{4}) = \exp(-i\frac{2\pi}{8}(2))} = \omega_8^2$. Subsequently,
\begin{align}
&\quad \begin{bmatrix}
I_4 & D_{4 \to 8} \\
I_4 & -D_{4 \to 8}
\end{bmatrix}
\begin{bmatrix}
F_4 & 0 \\\
0 & F_4
\end{bmatrix} =
\begin{bmatrix}
I_4F_4 & D_{4 \to 8}F_4 \\
I_4F_4 & -D_{4 \to 8}F_4
\end{bmatrix} \nonumber \\
&=
\renewcommand\arraystretch{1.25}
\footnotesize
\begin{bmatrix}
1 & 1 & 1 & 1 & 1 & 1 & 1 & 1 \\
1 & \omega_8^2 & \omega_8^4 & \omega_8^6 & \omega_8 & \omega_8^3 & \omega_8^5 & \omega_8^7 \\
1 & \omega_8^4 & \omega_8^8 & \omega_8^{12} & \omega_8^2 & \omega_8^6 & \omega_8^{10} & \omega_8^{14} \\
1 & \omega_8^6 & \omega_8^{12} & \omega_8^{18} & \omega_8^3 & \omega_8^9 & \omega_8^{15} & \omega_8^{21} \\
1 & 1 & 1 & 1 & -1 & -1 & -1 & -1 \\
1 & \omega_8^2 & \omega_8^4 & \omega_8^6 & -\omega_8 & -\omega_8^3 & -\omega_8^5 & -\omega_8^7 \\
1 & \omega_8^4 & \omega_8^8 & \omega_8^{12} & -\omega_8^2 & -\omega_8^6 & -\omega_8^{10} & -\omega_8^{14} \\
1 & \omega_8^6 & \omega_8^{12} & \omega_8^{18} & -\omega_8^3 & -\omega_8^9 & -\omega_8^{15} & -\omega_8^{21} \\
\end{bmatrix}
\end{align}
Note that $\omega_N^{k + qN} = \omega_N^k$\footnote{\label{foot:DFTcyclicexp}$\omega_N^{k + qN} = \exp(-i\frac{2\pi}{N}(k+qN)) = \exp(-i(\frac{2\pi}{N}k + 2\pi q)) = \exp(-i\frac{2\pi}{N}k) = \omega_N^{k}$.} for any integer $q$ and $-1 = \omega_N^{N/2}$, hence the matrix can be further rewritten into
\begin{align}
\renewcommand\arraystretch{1.25}
\footnotesize
\begin{bmatrix}
1 & 1 & 1 & 1 & 1 & 1 & 1 & 1 \\
1 & \omega_8^2 & \omega_8^4 & \omega_8^6 & \omega_8 & \omega_8^3 & \omega_8^5 & \omega_8^7 \\
1 & \omega_8^4 & \omega_8^8 & \omega_8^{12} & \omega_8^2 & \omega_8^6 & \omega_8^{10} & \omega_8^{14} \\
1 & \omega_8^6 & \omega_8^{12} & \omega_8^{18} & \omega_8^3 & \omega_8^9 & \omega_8^{15} & \omega_8^{21} \\
1 & \omega_8^8 & \omega_8^{16} & \omega_8^{24} & \omega_8^4 & \omega_8^{12} & \omega_8^{20} & \omega_8^{28} \\
1 & \omega_8^{10} & \omega_8^{20} & \omega_8^{30} & \omega_8^5 & \omega_8^{15} & \omega_8^{25} & \omega_8^{35} \\
1 & \omega_8^{12} & \omega_8^{24} & \omega_8^{36} & \omega_8^6 & \omega_8^{18} & \omega_8^{30} & \omega_8^{42} \\
1 & \omega_8^{14} & \omega_8^{28} & \omega_8^{42} & \omega_8^7 & \omega_8^{21} & \omega_8^{35} & \omega_8^{49} \\
\end{bmatrix}    
\end{align}
which is essentially the same as $F_8$ except the columns have been rearranged into even and odd just like what we have done to $\textbf{f}$. Hence the FFT formulation (\ref{eqn:FFT48})
\begin{align*}
\hat{\textbf{f}} &= 
\begin{bmatrix}
I_4 & D_{4 \to 8} \\
I_4 & -D_{4 \to 8}
\end{bmatrix}
\begin{bmatrix}
F_4 & 0 \\\
0 & F_4
\end{bmatrix}
\begin{bmatrix}
\textbf{f}_{\text{even}} \\
\textbf{f}_{\text{odd}}
\end{bmatrix}
\end{align*}
will give the same DFT result as the direct formula $\hat{\textbf{f}} = F_8\textbf{f}$. We can repeat the same procedure to similarly split $F_4$ into two $F_2$ blocks and shuffle the indices again. A schematic flowchart is shown as Figure \ref{fig:FFT} below. For a bigger $N$, e.g.\ $N = 2^{10} = 1024$, the method proceeds in the same principle such that $F_{1024} \to F_{512} \to F_{256} \to \cdots$. The FFT time complexity is \textit{log-linear} $O(n \log n)$, where the $\log n$ factor has replaced $n$ thanks to its bisection nature. In fact, it requires $\frac{N}{2}\log_2(N)$ multiplications and $N\log_2(N)$ additions. This is much faster than $O(n^2)$ of the naive DFT when $n$ becomes very large. For instance, the naive DFT of size $N = 2^{12} = 4096$ will require $33550336$ operations but the Radix-2 FFT algorithm only needs $73728$ which is around $450$ times more efficient. For this reason, any actual computer implementation of DFT will always use FFT underneath.

\begin{landscape}
\begin{figure}
    \centering
    \begin{tikzpicture}[squarednode/.style={rectangle, draw=blue!, fill=blue!20, very thick, align=center, minimum size=10mm},
    squaredrnode/.style={rectangle, draw=red!, fill=red!20, very thick, align=center, minimum size=10mm},
    rectnode/.style={rectangle, draw=Green!, fill=Green!20, very thick, align=center, minimum size=20mm}]
    \node[squarednode] at (0,0) (f0) {$f_0$};
    \node[squarednode] at (0,-1.5) (f1) {$f_1$};
    \node[squarednode] at (0,-3) (f2) {$f_2$};
    \node[squarednode] at (0,-4.5) (f3) {$f_3$};
    \node[squarednode] at (0,-6) (f4) {$f_4$};
    \node[squarednode] at (0,-7.5) (f5) {$f_5$};
    \node[squarednode] at (0,-9) (f6) {$f_6$};
    \node[squarednode] at (0,-10.5) (f7) {$f_7$};
    \node[rectnode, minimum height=25mm] at (3, -0.75) (F21) {\huge $F_2$};
    \node[rectnode, minimum height=25mm] at (3, -3.75) (F22) {\huge $F_2$};
    \node[rectnode, minimum height=25mm] at (3, -6.75) (F23) {\huge $F_2$};
    \node[rectnode, minimum height=25mm] at (3, -9.75) (F24) {\huge $F_2$};
    \draw[orange, ->] (f0.east) -- (F21.west);
    \draw[gray, ->] (f4.east) -- (F21.west);
    \draw[orange, ->] (f2.east) -- (F23.west);
    \draw[gray, ->] (f6.east) -- (F23.west);
    \draw[orange, ->] (f1.east) -- (F22.west);
    \draw[gray, ->] (f5.east) -- (F22.west);
    \draw[orange, ->] (f3.east) -- (F24.west);
    \draw[gray, ->] (f7.east) -- (F24.west);
    \node[rectnode, minimum height=55mm] at (6, -2.25) (F41) {\huge $F_4$};
    \node[rectnode, minimum height=55mm] at (6, -8.25) (F42) {\huge $F_4$};
    \draw[orange, ->] (F21.east) -- (F41.west);
    \draw[gray, ->] (F23.east) -- (F41.west);
    \draw[orange, ->] (F22.east) -- (F42.west);
    \draw[gray, ->] (F24.east) -- (F42.west);
    \node[rectnode, minimum height=115mm] at (9, -5.25) (F8) {\huge $F_8$};
    \draw[orange, ->] (F41.east) -- (F8.west);
    \draw[gray, ->] (F42.east) -- (F8.west);
    \node[squaredrnode] at (12,0) (fh0) {$\hat{f}_0$};
    \node[squaredrnode] at (12,-1.5) (fh1) {$\hat{f}_1$};
    \node[squaredrnode] at (12,-3) (fh2) {$\hat{f}_2$};
    \node[squaredrnode] at (12,-4.5) (fh3) {$\hat{f}_3$};
    \node[squaredrnode] at (12,-6) (fh4) {$\hat{f}_{-4}$};
    \node[squaredrnode] at (12,-7.5) (fh5) {$\hat{f}_{-3}$};
    \node[squaredrnode] at (12,-9) (fh6) {$\hat{f}_{-2}$};
    \node[squaredrnode] at (12,-10.5) (fh7) {$\hat{f}_{-1}$};
    \draw[red!50] (fh0.west) --++ (-1.5,0);
    \draw[red!50] (fh1.west) --++ (-1.5,0);
    \draw[red!50] (fh2.west) --++ (-1.5,0);
    \draw[red!50] (fh3.west) --++ (-1.5,0);
    \draw[red!50] (fh4.west) --++ (-1.5,0);
    \draw[red!50] (fh5.west) --++ (-1.5,0);
    \draw[red!50] (fh6.west) --++ (-1.5,0);
    \draw[red!50] (fh7.west) --++ (-1.5,0);
    \end{tikzpicture}
    \caption{\textit{A schematic diagram outlining the principle of FFT.}}
    \label{fig:FFT}
\end{figure}
\end{landscape}

\section{Python Programming and Earth System Applications}

Since in practice, the time series to be processed are often very long, it is impossible to do DFT/FFT manually and we will combine the application part with the programming tutorial into one single section. We will use the Niño 3.4 SST Index which is an indicator of the ENSO phenomenon discussed in Section \ref{section:EOF} for demonstration. Now, download the data file from \href{https://psl.noaa.gov/data/timeseries/month/data/nino34.long.anom.csv}{https://psl.noaa.gov/data/timeseries/month/data/nino34.long.anom.csv} and use the following code to read the time series.
\begin{lstlisting}
import numpy as np
import pandas as pd

Nino34 = pd.read_csv("nino34.long.anom.csv", header=0, names=["Date", "Nino34"])
print(Nino34)
\end{lstlisting}
Next, import the required functions from the \verb|scipy.fft| module and apply FFT on the Niño 3.4 time series over the 120-year time period of 1901-2020.
\begin{lstlisting}
from scipy.fft import fft, fftfreq, fftshift

Nino34_120yrs = Nino34[(Nino34["Date"] >= "1901-01-01") & (Nino34["Date"] <= "2020-12-31")]
print(Nino34_120yrs)

Nino34_fft = fft(Nino34_120yrs["Nino34"].values) 
\end{lstlisting}
Compute the power spectrum by multiplying the FFT-transformed data by its complex conjugate.
\begin{lstlisting}
Nino34_power = np.real(Nino34_fft*np.conjugate(Nino34_fft)) # Call the function np.real to remove negligible imaginary parts due to round-off error. Alternatively, write np.abs(Nino34_fft)**2.
\end{lstlisting}
Use the \verb|fftfreq| function to produce the frequency and period bins.
\begin{lstlisting}
Nino34_freq = fftfreq(len(Nino34_power), 1/12) # 1 month = 1/12 yrs
Nino34_period = 1/Nino34_freq
\end{lstlisting}
Finally, we can plot the power spectrum as follows (Figure \ref{fig:Nino1}).
\begin{lstlisting}
import matplotlib.pyplot as plt

def reciprocal(x): # Function to transform between frequency and period for the secondary axis.
    return(1/x)

plt.plot(Nino34_freq[:len(Nino34_power)//2], Nino34_power[:len(Nino34_power)//2])
plt.xlabel("Frequency (per yr)")
period_ax = plt.gca().secondary_xaxis('top', functions=(reciprocal, reciprocal))
plt.xlim([1/60,1])
period_ax.set_ticks([1,2,3,4,5,6,8,12])
period_ax.set_xlabel("Period (yr)")
plt.savefig("NinoFFT", dpi=300, bbox_inches="tight")
\end{lstlisting}
\begin{figure}[ht!]
    \centering
    \includegraphics[scale=0.75]{graphics/NinoFFT.png}
    \caption{\textit{Power Spectrum of the 120-year Niño 3.4 SST.}}
    \label{fig:Nino1}
\end{figure}
It can be seen that the strongest signals are located over the periods from $2.5$ to $7$ years, which coincides with the typical time scale of ENSO. We can carry out a simple filtering to extract the Niño 3.4 signals corresponding to ENSO by zeroing out the FFT array at all other frequencies and then applying an inverse FFT: 
\begin{lstlisting}
from scipy.fft import ifft

Nino34_fft_ENSO = np.copy(Nino34_fft)
Nino34_fft_ENSO[~((2.5 <= np.abs(Nino34_period)) & (np.abs(Nino34_period) <= 7))] = 0
Nino34_ENSO = np.real(ifft(Nino34_fft_ENSO))
\end{lstlisting}
Let's make a plot to compare the filtered time series with the original one.
\begin{lstlisting}
plt.plot(Nino34_120yrs["Date"], Nino34_120yrs["Nino34"].values, label="raw")
plt.plot(Nino34_120yrs["Date"], Nino34_ENSO, label="FFT-filtered")
plt.xticks(np.arange(0,len(Nino34_120yrs["Date"]), 60), rotation=20)
plt.xlim(["1981-01-01", "2020-12-31"])
plt.xlabel("Date")
plt.ylabel("Nino 3.4 Index")
plt.legend()
plt.title("ENSO-filtered Nino 3.4 time series")
plt.savefig("NinoENSOFilter", dpi=300, bbox_inches="tight")
\end{lstlisting}
However, note that this "zeroing-out" filtering method is not a very good idea to be implemented in practice and we do it here only for a heuristic purpose. For more information, search about the \textit{"Gibbs Phenomenon"} and also read the comprehensive discussion in   \href{https://stackoverflow.com/questions/31256252/why-does-numpy-linalg-solve-offer-more-precise-matrix-inversions-than-numpy-li}{this DSP StackExchange post} (6220).
\begin{figure}[ht!]
    \centering
    \includegraphics[scale=0.75]{graphics/NinoENSOFilter.png}
    \caption{\textit{Raw and ENSO-filtered Niño 3.4 signals from 1981 to 2020 data.}}
    \label{fig:nino2}
\end{figure}

\section{Exercise}
\begin{Exercise}
Compute the Discrete Fourier Transform for the following data.
\begin{center}
\begin{tabular}{|c|c|c|c|c|c|c|}
\hline
unit time & $0$ & $1$ & $2$ & $3$ & $4$ \\
\hline
$f(t)$ & $4.5$ & $6.2$ & $7.8$ & $1.1$ & $3.4$  \\
\hline
unit time & $5$ & $6$ & $7$ & $8$ & $9$ \\
\hline
$f(t)$ & $2.5$ & $3.6$ & $5.9$ & $2.9$ & $6.0$ \\
\hline
\end{tabular}
\end{center}
Find the amplitude/power and phase of the sinusoidal wave signal corresponding to the third frequency bin, i.e.\ with an angular frequency of $\omega = \smash{\frac{2\pi(3)}{10}}$. 
\end{Exercise}
\begin{Answer}
Its DFT is $(43.9, 7.350-0.095i, -1.386-6.082i, -2.600+0.059i, -3.064+8.990i, 0.5,  -3.064-8.990i, -2.600-0.059i, -1.386+6.082i, \allowbreak 7.350+0.095i)$. The required power and phase for the third frequency are 
\begin{align*}
\abs{C_3}^2 &= C_3C_{-3} = (-2.600+0.059i)(-2.600-0.059i) = 6.764 \\
\phi_3 &= \arctan(\frac{\Im(C_3)}{\Re(C_3)}) = \arctan(\frac{0.059}{-2.600}) = \SI{-0.022}{\radian}
\end{align*}
\end{Answer}

\begin{Exercise}
Download an \href{https://cds.climate.copernicus.eu/datasets/reanalysis-era5-single-levels?tab=download}{ERA5 Temperature Dataset} over any time period of $20$ years. Select any location as you like, and extract the temperature time series there. Apply DFT to the time series and identify any dominant frequency or period with a large power. Explain the peaks with Earth Science knowledge.
\end{Exercise}

\begin{Exercise}
Perform DFT on the time series for the two MJO EOF modes derived in Exercise \ref{ex:MJO} and deduce the characteristic time scale of MJO by plotting the power spectrum against periods.
\end{Exercise}

\begin{Exercise}
Find the circular convolution of two time series $f(t) = (1,4,2,4, \allowbreak 3,0,-1,2)$ and $g(t) = (2,3,-2,1,-1,0,4,3)$ by definition, as well as via the Convolution Theorem to check the consistency.
\end{Exercise}
\begin{Answer}
The desired circular convolution is
\begin{align*}
f \circledast g = (27, 28, 41, 14, 13, 4, 4, 19)
\end{align*}
and the respective DFTs are
\begin{align*}
& F[f \circledast g](k) \\
=&\, (150, 34.5-50.4i, -5+i, -6.5+23.6i, \\
& 20, -6.5-23.6i, -5-i, 34.5+50.4i) \\
=&\, (15, -0.59-7.24i, 3+2i, -3.41-1.24i, \\
& {-5}, -3.41+1.24i, 3-2i, -0.59+7.24i) \\
& \odot (10, 6.54+5.29i, -1+i, -0.54-6.71i, \\
& {-4}, -0.54+6.71i, -1-i, 6.54-5.29i) = \hat{f}(k)\hat{g}(k)
\end{align*}
\end{Answer}

\begin{Exercise}
Write your own FFT function in Python and compare it with the one in the \verb|scipy.fft| library by testing them on any time series.
\end{Exercise}
\chapter{Markov Chains}
\label{chapter:Markov}

In Earth Science, we often look at the evolution of variables in time, like temperature, rainfall, etc. Sometimes, the variables can be modeled as \textit{random variables}. A simple daily example will be tossing a coin (no matter if it is fair or not), where the outcomes (head/tail) are probabilistic. Some Earth System examples are the chances of extreme weather and slow geological processes. Such processes involving random variables are known as \textit{Stochastic Processes}, and we will visit the related Statistical concepts. Particularly, we will investigate the so-called \textit{Markov Chains}, which assumes that the present state of a stochastic process only depends on the past. Modeling a stochastic process with a Markov Chain can give more satisfactory results when it is known that the process inherently correlates strongly with previous states, or is said to have \textit{memory}.

\section{Statistical Prerequisites for Markov Chains}

\subsection{Lagged Auto-correlation}

First, let's talk about how to determine if a \index{Time Series}\keywordhl{time series} possesses some sort of \index{Memory}\textit{memory} as mentioned in the introduction. Memory causes the past of a time series to influence the present, and hence will leave a mark when we try to correlate the past and present segments of the time series. This leads to the idea of \index{Lagged (Lag-$k$) Auto-correlation}\keywordhl{lagged auto-correlation}, which is the correlation between a time series and itself, but one of them is lagged and shifted by a certain amount of days, let's say $k$ (the direction of the shifting does not matter as it is an auto-correlation, and we only care about the overlapping part). Then, it is more specifically known as the \textit{Lag-$k$ Auto-correlation}.

\begin{defn}[Lagged Auto-correlation]
\label{defn:autocorr}
The lag-$k$ auto-correlation $r_k$ of a time series $\{x_i\}_{i=1}^{n}$ is defined as 
\begin{align}
r_k &= \frac{\sum_{i=1}^{n-k}(x_i - \overline{x_{-}})(x_{i+k} - \overline{x_{+}})}{\sqrt{(\sum_{i=1}^{n-k}(x_i - \overline{x_{-}})^2) (\sum_{j=k+1}^{n}(x_j - \overline{x_{+}})^2)}}
\end{align}
the correlation (Definition \ref{defn:correlation}) between $\{x_{-}\} = \{x_i\}_{i=1}^{n-k}$ and $\{x_{+}\} = \{x_i\}_{i=k+1}^{n}$ where they are the two sub-sequences extracted from the earliest and latest $n-k$ data of the original time series, and the overline denotes an average. Alternatively, the formula can be written as
\begin{subequations}
\label{eqn:autocorrvar}
\begin{align}
r_k &= \frac{\text{Cov}(\{x_{-}\},\{x_{+}\})}{\sqrt{\text{Var}(\{x_{-}\}) \text{Var}(\{x_{+}\})}} \\
&= \frac{\text{Cov}(\{x_{-}\},\{x_{+}\})}{\sqrt{\text{Cov}(\{x_{-}\}, \{x_{-}\}) \text{Cov}(\{x_{+}\}, \{x_{+}\})}}
\end{align}   
\end{subequations}
where the definitions of variance and covariance are based on Definitions \ref{defn:variance} and \ref{defn:covariance}.
\end{defn}
If the lag-$k$ auto-correlation of a time series is close to $1$ or $-1$, it means that the state at a certain time will likely lead to a similar/opposite state $k$ time steps later. This may be used to infer causality but the implication is not always definite. There may be situations when the generated time series is regarded to be infinitely long, and in such cases, we simply apply the shifting and compute the lagged summation over the entire time axis.

\begin{exmp}
For the traffic flow data over some seven days of a highway below, find its lag-$1$ auto-correlation.
\begin{center}
\begin{tabular}{|c|c|c|c|c|c|c|c|}
\hline
Day & $1$ & $2$ & $3$ & $4$ & $5$ & $6$ & $7$ \\
\hline
Vehicles per Hour & $680$ & $820$ & $760$ & $790$ & $840$ & $1030$ & $1080$ \\
\hline
\end{tabular}
\end{center}
\end{exmp}
We first locate the first and last $7-1 = 6$ data and construct the two smaller time series, which are $\{x_-\} = \{680, 820, 760, 790, 840, 1030\}$ and $\{x_+\} = \{820, 760, 790, 840, 1030, 1080\}$. The means of the two time series are easily found to be $\overline{x_{-}} = 820$ and $\overline{x_{+}} = 2660/3$ (the readers can verify these numbers), and the sample variances of the two time series are thus
\begin{align*}
\text{Var}(\{x_-\}) &= \frac{1}{6-1}[(680 - 820)^2 + (820 - 820)^2 + (760 - 820)^2 \\
& \quad + (790 - 820)^2 + (840 - 820)^2 + (1030 - 820)^2] \\
&=13720 \text{ (Vehicles per Hour)}^2 \\
\text{Var}(\{x_+\}) &= \frac{1}{6-1}\left[\left(820 - \frac{2660}{3}\right)^2 + \left(760 - \frac{2660}{3}\right)^2 + \left(790 - \frac{2660}{3}\right)^2\right. \\
&\quad \left. + \left(840 - \frac{2660}{3}\right)^2 + \left(1030 - \frac{2660}{3}\right)^2 + \left(1080 - \frac{2660}{3}\right)^2\right] \\
&= 17987 \text{ (Vehicles per Hour)}^2
\end{align*}
and their sample covariance is
\begin{align*}
&\quad \text{Cov}(\{x_{-}\},\{x_{+}\}) \\
&= \frac{1}{6-1}\left[(680 - 820)\left(820 - \frac{2660}{3}\right) + (820 - 820)\left(760 - \frac{2660}{3}\right)\right. \\
&\quad + (760 - 820)\left(790 - \frac{2660}{3}\right)  + (790 - 820)\left(840 - \frac{2660}{3}\right) \\
&\quad \left. + (840 - 820)\left(1030 - \frac{2660}{3}\right) + (1030 - 820)\left(1080 - \frac{2660}{3}\right)\right] \\
&= 12000 \text{ (Vehicles per Hour)}^2
\end{align*}
Hence by Formula (\ref{eqn:autocorrvar}) in Definition \ref{defn:autocorr} above, the lag-$1$ auto-correlation is
\begin{align*}
r_1 &= \frac{12000}{\sqrt{(13720)(17987)}} = 0.7639
\end{align*}
It means that heavy (quiet) traffic will likely be followed by more or less heavy (quiet) traffic the next day, which is quite reasonable.\par
$\blacktriangleright$ Short Exercise: Compute the lag-$2$ auto-correlation for the same dataset.\footnotemark

\subsection{Conditional Probabilities, Stochastic Matrices}
To understand the idea of Markov Chains, we also need to know what \textit{conditional probabilities} are. The \index{Conditional Probability}\keywordhl{conditional probability} $P(A|B)$ is the probability of event $A$ occurring, given event $B$ has occurred. For an evolving system that has a finite number of states $A_i$, $i = 1,2,\ldots,n$, and can only possess one state at a time, e.g.\ a binary on-and-off, the conditional probability \smash{$P(A_i^{[k+1]}|A_j^{[k]})$} represents the probability of state $A_i$ occurring at time step $k+1$ if the state is $A_j$ at time step $k$. A simple Atmospheric Sciences example is the daily weather report, which in a simplistic sense, can take the state of either sunny, rainy, or windy. Usually, if it is rainy today, then there is a relatively high chance it is rainy tomorrow, i.e.\ \smash{$P(\text{rainy}^{[k+1]}|\text{rainy}^{[k]})$} is high. \par
For $N$ finite, distinct states/events in a well-defined, \textit{closed} system so that it takes one and only one of the events as its state, we have the following observation.
\begin{proper}
\label{proper:condprobsumto1}
The sum of conditional probabilities for a changing system with $N$ \textit{mutually exclusive} (only one state at a time) and \textit{exhaustive} (the states cover all possibilities) events $A_1, A_2, \cdots, A_N$ is
\begin{align}
\sum_{i=1}^N P(A_i^{[k+1]}|A_j^{[k]}) &= P(A_1^{[k+1]}|A_j^{[k]}) + P(A_2^{[k+1]}|A_j^{[k]}) + \cdots + P(A_N^{[k+1]}|A_j^{[k]}) \nonumber \\
P(\Omega^{[k+1]}|A_j^{[k]}) &= 1
\end{align}
where $\Omega = \bigcup_{i} A_i$ covers all the states and $k$ denotes a particular time step. The time increment is $1$ here but can be replaced by any other positive integer. It means that any given event $A_j$ must consequently lead to one of the possible states (including $A_j$ itself) at the next (few) time step(s).
\end{proper}
\footnotetext{The new $\{x_-\}$ and $\{x_+\}$ will be $\{680, 820, 760, 790, 840\}$ and $\{760, 790, 840, 1030, 1080\}$. We provide the relevant numbers for the readers to check. Their sample (co)variances are $\text{Var}(\{x_-\}) = 3920$, $\text{Var}(\{x_+\}) = 21150$ and $\text{Cov}(\{x_{-}\},\{x_{+}\}) = 5725$. So the lag-$2$ auto-correlation will be $r_2 = \smash{\frac{5725}{\sqrt{(3920)(21150)}}} = 0.6287$.}
This is essentially a rephrasing of the \index{Law of Total Probability}\textit{Law of Total Probability} from elementary Statistics. As a result, we can express all the conditional probabilities \smash{$P(A_i^{[k+1]}|A_j^{[k]})$} for a particular previous state $A_j$ in a system using a column vector with components that sum up to $1$. Using the daily weather as an example, assume there are only three states (sunny/windy/rainy), if a sunny day has a chance of $0.8$ to be followed by another sunny day, and $0.15$/$0.05$ for another windy/rainy day. Then we can write
\begin{center}
\begin{tabular}{|c|c|}
\hline
$k+1 \; \backslash \; k$ & Sunny \\
\hline
Sunny & $0.8$ \\
\hline
Windy & $0.15$ \\
\hline 
Rainy & $0.05$ \\
\hline
\end{tabular}
\end{center}
as a column vector
\begin{align*}
\begin{bmatrix}
0.8 \\
0.15 \\
0.05
\end{bmatrix}
\end{align*}
We can do the same for the other two states. If a windy day has a probability of $0.2$/$0.5$/$0.3$ leading to a sunny/windy/rainy day, and a rainy day has a probability of $0.1$/$0.2$/$0.7$ leading to a sunny/windy/rainy day, then
\begin{center}
\begin{tabular}{|c|c|c|c|}
\hline
$k+1 \; \backslash \; k$ & Sunny & Windy & Rainy \\
\hline
Sunny & $0.8$ & $0.2$ & $0.1$\\
\hline
Windy & $0.15$ & $0.5$ & $0.2$ \\
\hline 
Rainy & $0.05$ & $0.3$ & $0.7$ \\
\hline
\end{tabular}
\end{center}
This can be summarized by the so-called \index{Stochastic Matrix}\keywordhl{stochastic (transition) matrix} as
\begin{align*}
P = 
\begin{bmatrix}
0.8 & 0.2 & 0.1\\
0.15 & 0.5 & 0.2 \\
0.05 & 0.3 & 0.7
\end{bmatrix}
\end{align*}
\begin{defn}[Stochastic Matrix]
\label{defn:stocmat}
The stochastic matrix for a closed system with mutually exclusive and exhaustive events $A_1, A_2, \ldots, A_N$, is
\begin{align}
P =
\begin{bmatrix}
P(A_1^{[k+1]}|A_1^{[k]}) & P(A_1^{[k+1]}|A_2^{[k]}) & \cdots & P(A_1^{[k+1]}|A_N^{[k]})\\[3pt]
P(A_2^{[k+1]}|A_1^{[k]}) & P(A_2^{[k+1]}|A_2^{[k]}) & & P(A_2^{[k+1]}|A_N^{[k]}) \\[3pt]
\vdots & & \ddots & \vdots \\[3pt]
P(A_N^{[k+1]}|A_1^{[k]}) & P(A_N^{[k+1]}|A_2^{[k]}) & \cdots & P(A_N^{[k+1]}|A_N^{[k]})
\end{bmatrix}
\label{eqn:stocP}
\end{align}
where $P_{ij} = P(A_i^{[k+1]}|A_j^{[k]})$ is the conditional probability of the system moving to state $i$ at the next time step from the present state $j$.
\end{defn}
Again, notice that the entries along any column add up to $1$. The $j$-th column holds the conditional probabilities of state $j$ leading to different states at the next time step.

\section{Construction of and Prediction by Markov Chains}

\subsection{State Vector, Steady State}

Processes that can be represented by such stochastic matrices proposed above are known as \index{Markov Chain}\keywordhl{Markov Chains}. It is assumed that the conditional probabilities outlined in the stochastic matrices $P$ do not change in time (\index{Stationary}\textit{stationary}). Given a probability vector (or a \index{State Vector}\keywordhl{state vector}) $\vec{x}^{[k]}$ consisting of the probabilities of having different states at a certain time step $k$, we can calculate the probability vector $\vec{x}^{[k+1]}$ at the next time step $k+1$ as $P\vec{x}^{[k]}$.

\begin{proper}
Given a Markov Chain, with a stochastic matrix $P$ described by (\ref{eqn:stocP}), then the state vector $\vec{x}^{[k+1]}$ at time step $k+1$ is decided by
\begin{align}
\vec{x}^{[k+1]} &= P\vec{x}^{[k]}   
\end{align}
\end{proper}
\begin{proof}
If we look at the $i$-th entry on both sides, we have
\begin{align*}
\vec{x_i}^{[k+1]} &= P_{i1} \vec{x_1}^{[k]} + P_{i2} \vec{x_2}^{[k]} + P_{i3} \vec{x_3}^{[k]} \cdots 
\end{align*}
When explicitly written in terms of (conditional) probabilities, it is
\begin{align*}
P(A_i^{[k+1]}) &= P(A_i^{[k+1]}|A_1^{[k]}) P(A_1^{[k]}) + P(A_i^{[k+1]}|A_2^{[k]}) P(A_2^{[k]}) + \cdots \\
&= \sum_{j=1}^N P(A_i^{[k+1]}|A_j^{[k]}) P(A_j^{[k]})
\end{align*}
which is exactly the manifestation of the \textit{Law of Total Probability} again, given the events are mutually exclusive and exhaustive.
\end{proof} 
Similarly, at the $k+2$-th time step, the probability vector is $\vec{x}^{[k+2]} = P^2\vec{x}^{[k]}$. In general, we have $\vec{x}^{[1]} = P\vec{x}^{[0]}, \vec{x}^{[2]} = P\vec{x}^{[1]} = P(P\vec{x}^{[0]}) = P^2\vec{x}^{[0]}$ and 
\begin{align}
\vec{x}^{[n]} &= P^n\vec{x}^{[0]} & & (\vec{x}^{[k+n]} = P^n\vec{x}^{[k]})  \label{eqn:markovpredict}
\end{align}

\begin{exmp}
\label{exmp:weathermarkov}
Using the previous example of daily weather, the stochastic matrix is
\begin{align*}
P = 
\begin{bmatrix}
0.8 & 0.2 & 0.1\\
0.15 & 0.5 & 0.2 \\
0.05 & 0.3 & 0.7
\end{bmatrix}   
\end{align*}
Find the probabilities of each type of weather occurring on Day 2 and Day 3, if we know that the chances of being sunny/windy/rainy on Day 1 are $0.3$/$0.4$/$0.3$. 
\end{exmp}
\begin{solution}
By Formula (\ref{eqn:markovpredict}) we have just derived, the required state vector is
\begin{align*}
\vec{x}^{[2]} &= P\vec{x}^{[1]} \\
&=
\begin{bmatrix}
0.8 & 0.2 & 0.1\\
0.15 & 0.5 & 0.2 \\
0.05 & 0.3 & 0.7
\end{bmatrix}   
\begin{bmatrix}
0.3 \\
0.4 \\
0.3
\end{bmatrix} \\
&= 0.3
\begin{bmatrix}
0.8 \\
0.15 \\
0.05
\end{bmatrix}
+ 0.4
\begin{bmatrix}
0.2 \\
0.5 \\
0.3
\end{bmatrix}
+ 0.3
\begin{bmatrix}
0.1 \\
0.2 \\
0.7
\end{bmatrix} \\
&=
\begin{bmatrix}
0.35 \\
0.305 \\
0.345
\end{bmatrix}
\end{align*}
So on Day 2, the chances of sunny/windy/rainy are $0.35$/$0.305$/$0.345$. It is emphasized that in the end, the required state vector on the next day is just the linear combination of the column vectors that contain the conditional probabilities, with the weightings specified by the current state vector. Again, it is the Law of Total Probability working. Similarly, for Day 3, the state vector will be
\begin{align*}
\vec{x}^{[3]} &= P\vec{x}^{[2]} = P^2\vec{x}^{[1]} \\
&=
\begin{bmatrix}
0.8 & 0.2 & 0.1\\
0.15 & 0.5 & 0.2 \\
0.05 & 0.3 & 0.7
\end{bmatrix}   
\begin{bmatrix}
0.35 \\
0.305 \\
0.345
\end{bmatrix} \\
&= 0.35
\begin{bmatrix}
0.8 \\
0.15 \\
0.05
\end{bmatrix}
+ 0.305
\begin{bmatrix}
0.2 \\
0.5 \\
0.3
\end{bmatrix}
+ 0.345
\begin{bmatrix}
0.1 \\
0.2 \\
0.7
\end{bmatrix} \\
&=
\begin{bmatrix}
0.3755\\ 
0.274\\
0.3505
\end{bmatrix}
\end{align*}
\end{solution}
$\blacktriangleright$ Short Exercise: Find the state vector on the next day if today is windy.\footnotemark

For a Markov Chain which has been ongoing for a long period of time, and the initial state becomes effectively forgotten (\index{Memoryless}\textit{memoryless}), it is desirable to obtain the \index{Steady-state Vector}\keywordhl{steady-state vector} which represents the "average" probability of every state given no prior knowledge of the initial state. The steady-state vector $\vec{q}$ remains the same for any time step and the probabilities are stationary. Hence, we have 
\begin{align}
\label{eqn:qeqPq}
\vec{q} = P\vec{q} = \cdots = P^n\vec{q}    
\end{align}
Rearrangement of the relation $\vec{q} = P\vec{q}$ gives 
\begin{align}
(I-P)\vec{q} = \textbf{0}    
\end{align}
which can be recognized as an eigenvalue-eigenvector problem (see Section \ref{section:eigensection}): The steady-state vector $\vec{q}$ corresponds to the eigenvector of $P$ with an eigenvalue of $\lambda = 1$. This eigenvector $\vec{q}$ is then found by following the usual procedure to solve the linear system $(I-P)\vec{q} = \textbf{0}$. Some may suspect if this steady-state vector always exists. To address this question, we will first show that one of the eigenvalues in any Markov Chain must be $1$.\footnotetext{It is simply $P\vec{x}$ where $\vec{x} = (0,1,0)^T$ and thus
\begin{align*}
\begin{bmatrix}
0.8 & 0.2 & 0.1\\
0.15 & 0.5 & 0.2 \\
0.05 & 0.3 & 0.7
\end{bmatrix}  
\begin{bmatrix}
0 \\
1 \\
0
\end{bmatrix}
=
\begin{bmatrix}
0.2 \\
0.5 \\
0.3
\end{bmatrix}
\end{align*} so the probabilities are simply given by the components in the second column of the stochastic matrix.}
\begin{proper}
\label{proper:markoveigen1}
The stochastic matrix $P$ of a Markov chain always has an eigenvalue of $\lambda = 1$. The steady state of a Markov Chain is then simply the eigenvector of $P$ that corresponds to the eigenvalue of $\lambda = 1$ as in $(P - \lambda I)\vec{q} = \textbf{0}$, and normalized by the sum of entries so that they add up to $1$.
\end{proper}
\begin{proof}
We will show that the transpose $P^T$ has an eigenvalue of $\lambda = 1$. Then by Properties \ref{proper:eigentransinv}, $P$ must also share this same eigenvalue and we are done. Particularly, we can observe that $\vec{z} = (1,1,1,\ldots,1)^T$ is an eigenvector of $P^T$ that corresponds to the desired eigenvalue of $1$, because
\begin{align*}
P^T \vec{z} &= 
\begin{bmatrix}
P_{11} & P_{21} & \cdots & P_{n1} \\
P_{12} & P_{22} & \cdots & P_{n2} \\
\vdots & & \ddots & \vdots \\
P_{1n} & P_{2n} & \cdots & P_{nn}
\end{bmatrix}
\begin{bmatrix}
1 \\
1 \\
\vdots \\
1
\end{bmatrix}
=
\begin{bmatrix}
P_{11} + P_{21} + \cdots + P_{n1} \\
P_{12} + P_{22} + \cdots + P_{n2} \\
\vdots \\
P_{1n} + P_{2n} + \cdots + P_{nn}
\end{bmatrix}
=
\begin{bmatrix}
1 \\
1 \\
\vdots \\
1
\end{bmatrix}
= \vec{z}
\end{align*}
where by Properties \ref{proper:condprobsumto1} the entries in each column of $P$ sum to one: $P_{1j} + P_{2j} + \cdots + P_{nj} = 1$.
\end{proof}

\begin{exmp}
\label{exmp:dailyweathersteady}
Find the steady-state vector in the last example about daily weather where
\begin{align*}
P = 
\begin{bmatrix}
0.8 & 0.2 & 0.1\\
0.15 & 0.5 & 0.2 \\
0.05 & 0.3 & 0.7
\end{bmatrix}   
\end{align*}
\end{exmp}
\begin{solution}
We seek to solve the linear system
\begin{align*}
I - P = 
\begin{bmatrix}
0.2 & -0.2 & -0.1\\
-0.15 & 0.5 & -0.2 \\
-0.05 & -0.3 & 0.3
\end{bmatrix}   
\end{align*}
and a simple calculation by Gaussian Elimination reveals that the desired eigenvector of $\lambda=1$ is $\vec{q} = (\frac{9}{7}, \frac{11}{14}, 1)^T$ (You may check by substituting this into $(I-P)\vec{q} = \textbf{0}$). Since the probabilities have to add up to $1$, we divide each component by their sum and this yields the steady-state vector $\vec{q} = (\frac{18}{43}, \frac{11}{43}, \frac{14}{43})^T \approx (0.419, 0.256, 0.326)^T$, meaning that there is $41.9\% / 25.6\% / 32.6\%$ chance of having a sunny/windy/rainy day on average.
\end{solution}
Nevertheless, there is a subtlety. We have only shown that there is always an eigenvalue of $1$ (or in general it is the modulus $\abs{\lambda} = 1$ equals to one, as we will see soon) for a Markov Chain, but we don't know if there is only one unique corresponding eigenvector. Consider an extreme example where there exist three states, A, B, and C. State A always leads to state B, which in turn always leads to state C, and state C will always move back to state A. The stochastic matrix is then simply
\begin{align*}
P = \begin{bmatrix}
0 & 0 & 1\\
1 & 0 & 0 \\
0 & 1 & 0
\end{bmatrix}
\end{align*}
It can be found that there are three eigenvalues $\lambda = 1, -\frac{1}{2} + \frac{\sqrt{3}}{2}i, -\frac{1}{2} - \frac{\sqrt{3}}{2}i$ where the two complex eigenvalues have a modulus of $1$ as well. This system can be easily seen to be unstable. Any state vector other than $(\frac{1}{3}, \frac{1}{3}, \frac{1}{3})^T$ (the eigenvector for $\lambda = 1$) will keep oscillating where the components are "cycled". Therefore, the system will not converge to the steady state despite the existence of a "steady-state" vector of $(\frac{1}{3}, \frac{1}{3}, \frac{1}{3})^T$. The imaginary part of the two complex eigenvalues represents a rotation of the two complex eigenvectors and the modulus of $\abs{\lambda} = 1$ means that the magnitude of this rotation will not decay.\footnote{For the complex eigenvalues $\lambda_{\text{c}}$ and complex eigenvectors $\vec{x}_{\text{c}}$, $\norm{P\vec{x}_{\text{c}}}^2 = \norm{\lambda\vec{x}_{\text{c}}}^2 = \abs{\lambda}^2 \norm{\vec{x}_{\text{c}}}^2 = \norm{\vec{x}_{\text{c}}}^2$ as $\abs{\lambda} = 1$. However, the entries of a state vector are always real-valued and it will be a linear combination of these conjugate complex eigenvectors where the imaginary part is exactly canceled out underneath.} So there should be some restriction if the system has to converge to a unique steady state. It turns out that a sufficient condition is whether the Markov Chain is \textit{regular}. To proceed, we need to introduce some terminologies.

\begin{defn}[Regular Markov Chain]
\label{defn:regularstoc}
A stochastic matrix $P$, or the corresponding Markov Chain, is said to be \index{Regular Markov Chain}\keywordhl{regular} if some power of the stochastic matrix $P^k$, where $k$ is any positive integer, has all positive entries, i.e.\ $P^k$ is a \index{Positive Stochastic Matrix}\keywordhl{positive stochastic matrix} (or simply positive).
\end{defn}

\begin{proper}
\label{proper:positivestoceig}
For a positive stochastic matrix $A$, the only eigenvalue with a modulus $\abs{\lambda} = 1$ is $\lambda = 1$. Moreover, the geometric multiplicity of this eigenvalue $\lambda = 1$ belonging to the matrix $A$ is strictly $1$, i.e.\ there is only one eigenvector corresponding to $\abs{\lambda} = 1$.
\end{proper}
The proof is provided in Appendix \ref{section:Markovappend}. For a regular stochastic matrix $P$, $A = P^k$ is positive for some $k$ by Definition \ref{defn:regularstoc} and hence the only eigenvalue of $P^k$ that has a modulus of $1$ is $\lambda = 1$ and its geometric multiplicity is $1$ by Properties \ref{proper:positivestoceig} above. It is not hard to show that, therefore, $P$ will also have $\lambda = 1$ as the only eigenvalue that has a modulus of one, $\abs{\lambda} = 1$, and the geometric multiplicity is also $1$.\footnotemark{} Hence by combining the two results, we have
\begin{proper}
\label{proper:regularmarkovsteady}
A regular Markov chain always has a unique steady-state vector with an eigenvalue of $1$.  
\end{proper}
The stochastic matrix in Example \ref{exmp:dailyweathersteady} is obviously regular, hence we have been able to derive the steady state for it. However, note that Properties \ref{proper:regularmarkovsteady} provides a sufficient condition only, and a non-regular Markov chain may still converge to a unique steady state. Take\footnotetext{Assume the contrary that there are two different eigenvectors of $P$, $\vec{q}^{(1)}$ and $\vec{q}^{(2)}$, the eigenvalues of which have a modulus $\abs{\lambda_1} = \abs{\lambda_2} = 1$ of one. Then $\smash{P^k\vec{q}^{(1)}} = \smash{P^{k-1}(P\vec{q}^{(1)})} = \lambda_1 \smash{P^{k-1}\vec{q}^{(1)}} = \lambda_1 \smash{P^{k-2} (P\vec{q}^{(1)})} = \smash{\lambda_1^2 P^{k-2}\vec{q}^{(1)}} = \cdots = \smash{\lambda_1^k} \vec{q}^{(1)}$, and similarly we have $P^k\vec{q}^{(2)} = \smash{\lambda_2^k} \vec{q}^{(2)}$. These show that $\vec{q}^{(1)}$ and $\vec{q}^{(2)}$ will be also two eigenvectors to $A = P^k$ with an eigenvalue of $\lambda_1^k$ and $\lambda_2^k$ where both of their modulus $\smash{\abs{\lambda_j^k}} = \abs{\lambda_j}^k = 1$, $j = 1,2$, equal to one, contradicting Properties \ref{proper:positivestoceig}.}
\begin{align*}
P = 
\begin{bmatrix}
1 & 0.5 \\
0 & 0.5
\end{bmatrix}
\end{align*}
as an example. It is not hard to see that the bottom-left entry in $P^k$ where $k$ is any positive integer, will always be $0$.\footnote{In general, for any two $2 \times 2$ matrices both in the form of
\begin{align*}
\begin{bmatrix}
* & * \\
0 & *    
\end{bmatrix}
\end{align*} their product
\begin{align*}
\begin{bmatrix}
* & * \\
0 & *    
\end{bmatrix}
\begin{bmatrix}
* & * \\
0 & *    
\end{bmatrix}
=
\begin{bmatrix}
* & * \\
(0)(*) + (*)(0) & *    
\end{bmatrix}
=
\begin{bmatrix}
* & * \\
0 & *    
\end{bmatrix}
\end{align*} will take such a form too.} However, it is also easy to see that given any starting probability vector, it will approach the steady-state vector $(1,0)^T$ as it goes on. Another problem is that in Definition \ref{defn:regularstoc} we haven't specified how large the $k$ we have to keep checking and when we can stop. For the small example above and by the accompanying footnote, knowing $P^2$ is not positive is adequate to decide that it is not regular. For completeness, we note that in general, given an $n \times n$ stochastic matrix $P$, the required $k$ to test is up to $n^2 - 2n + 2$ (so for $n = 2$, we will check $k = 1,2$; and $n = 3$, $k = 1,2,3,4,5$).\footnote{This "magic number" is due to a result by Wielandt but the proof is a bit too advanced to be included here.}

\subsection{Expected Value Problem}
For the special case in which one of the states in Markov Chain is \textit{absorbing}, i.e.\ the probability of staying in this node/state is $1$ and the probabilities of moving to other states are all $0$, all pathways will eventually lead to accumulation in this particular \textit{sink} no matter what the initial state is. This is referred to as an \index{Absorbing Markov Chain}\keywordhl{absorbing Markov Chain} and it is possible to predict the expected time needed for any state to arrive at the absorbing state. Note that the previous example of
\begin{align*}
P = 
\begin{bmatrix}
1 & 0.5 \\
0 & 0.5
\end{bmatrix}    
\end{align*}
is exactly an absorbing Markov Chain where the sink is in the first node.
\par

We can attack this problem by finding the total expected time spent in other non-absorbing nodes which is equivalent to the expected time to arrive at the sink. At the zeroth time step, or $t=0$, the time spent in the starting node and other nodes is one and zero unit time respectively. At $t=k$, the average times spent during the $k$-th time step in different nodes are exactly the probabilities of landing on these nodes after $k$ time steps. \par
Hence, the idea is to add up the times staying in the non-absorbing nodes over every time step. Therefore, we consider the variant of the stochastic matrix in which the row and column of the absorbing state are deleted. Assume that there are three states in the absorbing Markov Chain, and its stochastic matrix is
\begin{align*}
P &= 
\begin{bmatrix}
0.2 & 0.8 & 0 \\
0.7 & 0 & 0 \\
0.1 & 0.2 & 1
\end{bmatrix}
\end{align*}
Along the third column, the third component is equal to $1$ and the others are $0$, indicating that the third node is absorbing. After deleting the row and column related to the absorbing state, and only looking at those non-absorbing states, we have
\begin{align*}
P_0 &= 
\begin{bmatrix}
0.2 & 0.8\\
0.7 & 0 \\
\end{bmatrix}   
\end{align*}
Assume we start at the first node, i.e.\ the probability vector representing the initial state is $(1, 0)^T$ in this case, the formula to calculate the total staying time in the non-absorbing nodes over all time steps would be
\begin{align*}
I
\begin{bmatrix}
1 \\
0
\end{bmatrix}
+
P
\begin{bmatrix}
1 \\
0
\end{bmatrix}
+
P^2
\begin{bmatrix}
1 \\
0
\end{bmatrix}
+ \cdots 
=
(I + P_0 + P_0^2 + \cdots)
\begin{bmatrix}
1 \\
0
\end{bmatrix}
=
\begin{bmatrix}
t_1 \\
t_2
\end{bmatrix}
\end{align*}
where the first term on L.H.S. is the time staying in other nodes at the zeroth time step, the second term is that at the first time step, and so on. $t_1$ and $t_2$ will be the total expected time spent at nodes $1$ and $2$ throughout the process. For the case where the starting condition is probabilistic, e.g.\ 50\% chance in the first node
and 50\% chance in the second node, we simply change $(1,0)^T$ to $(0.5,0.5)^T$. We can utilize the formula of \index{Geometric Sum (Matrix)}\textit{geometric sum for a matrix}, and generalize it to accommodate more states.
\begin{proper}
The expected time for an initial state in a Markov Chain to be absorbed into the sink (assumed that there is only one sink) can be inferred from the relation
\begin{subequations}
\begin{align}
\begin{bmatrix}
t_1 \\
t_2 \\
t_3 \\
\vdots 
\end{bmatrix}
&= (I + P_0 + P_0^2 + \cdots)\vec{x}^{[0]} \\
&= (I - P_0)^{-1}
\vec{x}^{[0]}
\end{align}    
\end{subequations}
where $P$ is the associated stochastic matrix and $P_0$ is $P$ with the row and column of the absorbing state removed. $\vec{x}^{[0]}$ is the initial state vector, also with the entry corresponding to the absorbing node removed. The required absorption time to the sink is the sum of $t_i$ over all valid $i$. 
\end{proper}
The only caveat is that the formula of geometric sum for a matrix may not hold in the second equality, just like the original formula for a number will fail if the common ratio has a magnitude larger than or equal to $1$. However, it is guaranteed that, for a stochastic matrix without other sinks or closed loops which make traveling from any state to the sink impossible
\begin{align}
(I - P_0)^{-1} = I + P_0 + P_0^2 + \cdots    
\end{align}
indeed holds.\footnote{An equivalent condition is that the original stochastic matrix $P$ only has one eigenvalue with the value of $\lambda = 1$ that represents the sink and there are no other eigenvalues that also have a modulus of $\abs{\lambda} = 1$.}
\begin{exmp}
For the absorbing Markov Chain that we have been discussing, where
\begin{align*}
& P = 
\begin{bmatrix}
0.2 & 0.8 & 0 \\
0.7 & 0 & 0 \\
0.1 & 0.2 & 1
\end{bmatrix}
&
& P_0 = 
\begin{bmatrix}
0.2 & 0.8\\
0.7 & 0 \\
\end{bmatrix}   
\end{align*}
Find the expected time required to reach the third node if we start from the first node.
\end{exmp}
\begin{solution}
We have
\begin{align*}
I - P_0 &= 
\begin{bmatrix}
0.8 & -0.8\\
-0.7 & 1 
\end{bmatrix} 
\end{align*}
It is not hard to obtain (for example, just do it like in Example \ref{exmp:2x2})
\begin{align*}
(I - P_0)^{-1} &= 
\begin{bmatrix}
4.166 & 3.333 \\
2.917 & 3.333
\end{bmatrix} \\
(I - P_0)^{-1}\vec{x}^{[0]} &= 
\begin{bmatrix}
4.166 & 3.333 \\
2.917 & 3.333
\end{bmatrix}
\begin{bmatrix}
1 \\
0
\end{bmatrix} 
=
\begin{bmatrix}
4.166 \\
2.917
\end{bmatrix}
\end{align*}
Hence the expected time required to reach the third node starting from the first node will be just the sum of the first column in $P_0$, $4.166 + 2.917 = 7.083$.
\end{solution}
$\blacktriangleright$ Short Exercise: Find the expected time required to reach the third node if it has $50\%/50\%$ chances to start at the first/second node instead.\footnotemark

\subsection{Lagged Auto-correlation Predicted by Markov Chains}

The last topic we will touch on Markov Chains is their inherent auto-correlation (see the beginning of this chapter). Consider the simplest case, where a Markov Chain only has two possible states, with a $2 \times 2$ stochastic matrix, in the form of
\begin{align}
P = 
\begin{bmatrix}
P_{11} & P_{12} \\
P_{21} & P_{22}
\end{bmatrix}
=
\begin{bmatrix}
P_{11} & P_{12} \\
1 - P_{11} & 1 - P_{12} \\
\end{bmatrix}
\end{align}
Assume the first/second state represents a value of $1$/$0$ without the loss of generality. The theoretical lag-$1$ auto-correlation between the binary states, predicted by the Markov Chain, can be inferred from (\ref{eqn:autocorrvar}) in Definition \ref{defn:autocorr}, that is
\begin{align*}
r_1 &= \frac{\text{Cov}(\{x_{-}\},\{x_{+}\})}{\sqrt{\text{Var}(\{x_{-}\}) \text{Var}(\{x_{+}\})}} \\
&= \frac{E[X_{-}X_{+}] - E[X_{-}]E[X_{+}]}{\sqrt{(E[X_{-}^2] - (E[X_{-}])^2)(E[X_{+}^2] - (E[X_{+}])^2)}}
\end{align*}
where we have used the short-cut formulae for variance and covariance listed in Section \ref{section:variancesec}. If the Markov Chain has operated for a long enough time, then the means of the two time series, despite lagged by a day, will be equal due to the loss of memory, implying that
\begin{align}
E[X_{-}] = E[X_{+}] &= q_1
\end{align}
where $\vec{q}$ is the steady-state vector, with $q_1$ being the average probability of being in state $1$. Similarly, we have
\begin{align}
E[X_{-}^2] = E[X_{+}^2] &= q_1 
\end{align}
since squaring a time series composed purely of $1$ and $0$ will return itself. The remaining job is to determine $E[X_{-}X_{+}]$, which is\footnotetext{It is simply a matter of computing
\begin{align*}
(I - P_0)^{-1}\vec{x}^{[0]} &= 
\begin{bmatrix}
4.166 & 3.333 \\
2.917 & 3.333
\end{bmatrix}
\begin{bmatrix}
0.5 \\
0.5
\end{bmatrix}=
\begin{bmatrix}
3.75 \\
3.125
\end{bmatrix}
\end{align*}
where $\vec{x}^{[0]}$ is now $(0.5, 0.5)^T$ and the required time is then $3.75 + 3.125 = 6.875$.}
\begin{align}
E[X_{-}X_{+}] &= P(X_{+}=1 \text{ and } X_{-}=1) \nonumber \\
&= P(X_{+}=1|X_{-}=1) P(X_{-}=1) \nonumber \\
&= P_{11} q_1
\end{align}
as per the definition of conditional probability. So the lag-$1$ auto-correlation is
\begin{align}
r_1 &= \frac{(P_{11}q_1-q_1^2)}{\sqrt{(q_1-q_1^2)(q_1-q_1^2)}} \nonumber \\
&= \frac{q_1(P_{11}-q_1)}{(q_1-q_1^2)} \nonumber \\
&= \frac{P_{11}-q_1}{1-q_1} \label{eqn:binaryr1}
\end{align}
But, by the definition of the stationary eigenvector $P\vec{q} = \vec{q}$ in (\ref{eqn:qeqPq}), considering its first component we have
\begin{align}
P_{11}q_1 + P_{12}q_2 &= q_1  \nonumber \\
P_{11}q_1 + P_{12}(1-q_1) &= q_1 \nonumber  \\
(1 - P_{11} + P_{12}) q_1 &= P_{12}
\end{align}
Substituting this into the formula of lag-$1$ auto-correlation (\ref{eqn:binaryr1}) just derived, we arrive at
\begin{align}
r_1 = \frac{P_{11}-q_1}{1-q_1} &= \frac{(1 - P_{11} + P_{12})(P_{11}-q_1)}{(1 - P_{11} + P_{12})(1-q_1)}  \nonumber \\
&= \frac{P_{11}(1 - P_{11} + P_{12}) - (1 - P_{11} + P_{12})q_1}{1 - P_{11} + P_{12} - (1 - P_{11} + P_{12})q_1}  \nonumber \\
&= \frac{P_{11}(1 - P_{11} + P_{12}) - P_{12}}{1 - P_{11} + P_{12} - P_{12}} \nonumber  \\
&= \frac{(1-P_{11})(P_{11}-P_{12})}{1 - P_{11}} \nonumber  \\
&= P_{11} - P_{12}
\end{align}
Therefore, the magnitude of lag-$1$ auto-correlation predicted by a binary Markov Chain is simply the difference between the two conditional probabilities on the same row in the stochastic matrix. By the same essence, the lag-$k$ auto-correlation can be inferred from $r_k = Q_{11} - Q_{12}$, where $Q = P^k$. Fortunately, due to the structure of stochastic matrices, we can derive a straightforward formula for lag-$k$ auto-correlation, which is just the $k$-th power of the lag-$1$ auto-correlation, $r_k = r_1^k$.

\begin{proper}
The lag-$k$ auto-correlation of a binary Markov Chain, associated with a stochastic matrix $P$, is
\begin{align}
r_k = (P^k)_{11} - (P^k)_{12} = (P_{11} - P_{12})^k = r_1^k
\end{align}
where $r_1 = P_{11} - P_{12}$.
\end{proper}
\begin{proof}
The formula holds for $k=1$ as we just see, and now we will prove the equality $(P^k)_{11} - (P^k)_{12} = (P_{11} - P_{12})^k$ for any general $k$. First, it is instructive to find the eigenvalues and eigenvectors of $P$ and diagonalize the matrix. We set out to solve the characteristic equation
\begin{align}
\det(P - \lambda I) =
\begin{vmatrix}
P_{11} - \lambda & P_{12} \\
1 - P_{11} & 1 - P_{12} - \lambda
\end{vmatrix} &= 0 \nonumber \\
(P_{11} - \lambda)(1 - P_{12} - \lambda) - (1 - P_{11})P_{12} &= 0 \nonumber \\
P_{11} - P_{11}P_{12} - \lambda P_{11} - \lambda + \lambda P_{12} + \lambda^2 - P_{12} + P_{11}P_{12} &= 0 \nonumber \\ 
(P_{11} - P_{12}) - \lambda (1 + (P_{11} - P_{12})) + \lambda^2 &= 0 \nonumber\\ 
(1-\lambda)((P_{11} - P_{12})-\lambda)&= 0
\end{align}
So the eigenvalues are $\lambda_1 = 1$ (as expected) and $\lambda_2 = P_{11} - P_{12}$. It is not hard to find the steady-state vector (not yet normalized) $\vec{q}$ that corresponds to $\lambda = 1$, which is
\begin{align}
\vec{q} = 
\begin{bmatrix}
P_{12} \\
1 - P_{11} 
\end{bmatrix}
\end{align}
and we leave it to the readers to check. For $\lambda_2 = P_{11} - P_{12}$, the eigenvector can also be easily found from 
\begin{align*}
& \left[\begin{array}{@{}cc|c@{\,}}
P_{11} - (P_{11} - P_{12}) & P_{12} & 0 \\
1 - P_{11} & 1 - P_{12} - (P_{11} - P_{12}) & 0 \\
\end{array}\right]  \\
=& 
\left[\begin{array}{@{}wc{32pt}wc{30pt}|c@{\,}}
P_{12} & P_{12} & 0 \\
1 - P_{11} & 1 - P_{11} & 0 \\
\end{array}\right]  \\
\to& \left[\begin{array}{@{}wc{32pt}wc{30pt}|c@{\,}}
1 & 1 & 0 \\
1 - P_{11} & 1 - P_{11} & 0 \\
\end{array}\right] & \frac{1}{P_{12}}R_1 \to R_1 \\
\to& \left[\begin{array}{@{}wc{32pt}wc{30pt}|c@{\,}}
1 & 1 & 0 \\
0 & 0 & 0 \\
\end{array}\right] & R_2 - (1 - P_{11})R_1 \to R_2
\end{align*}
hence it has a very simple form of $(-1, 1)^T$. Using the same technique as in Example \ref{exmp:powerdiag}, we diagonalize $P$ according to $S^{-1}PS = D$ where now we use $S$ to denote the matrix formed by the eigenvectors of $P$, and then
\begin{align}
P^k &= (SDS^{-1})^k \nonumber \\
&= (SDS^{-1})(SDS^{-1}) \cdots (SDS^{-1})(SDS^{-1}) \nonumber \\
&= SD (S^{-1}S)D \cdots D (S^{-1}S) DS^{-1} = SD^kS^{-1} \nonumber \\
&= 
\begin{bmatrix}
P_{12} & -1 \\
1 - P_{11} & 1
\end{bmatrix}
\begin{bmatrix}
1 & 0 \\
0 & (P_{11} - P_{12})^k
\end{bmatrix}
\begin{bmatrix}
P_{12} & -1 \\
1 - P_{11} & 1
\end{bmatrix}^{-1} \nonumber \\
&= 
\begin{bmatrix}
P_{12} & -(P_{11} - P_{12})^k \\
1 - P_{11} & (P_{11} - P_{12})^k
\end{bmatrix}
\left(\frac{1}{P_{12} + (1 - P_{11})}
\begin{bmatrix}
1 & 1 \\
-(1 - P_{11}) & P_{12} \\
\end{bmatrix}
\right) \nonumber \\
&= 
\frac{1}{P_{12} + (1 - P_{11})}
\scriptsize
\begin{bmatrix}
P_{12} + (1-P_{11})(P_{11}-P_{12})^k & P_{12} - P_{12}(P_{11}-P_{12})^k \\
(1-P_{11}) - (1-P_{11})(P_{11}-P_{12})^k & (1-P_{11}) + P_{12}(P_{11}-P_{12})^k
\end{bmatrix}
\end{align}
So
\begin{align*}
&\quad (P^k)_{11} - (P^k)_{12} \\
&= \frac{(P_{12} + (1-P_{11})(P_{11}-P_{12})^k) - (P_{12} - P_{12}(P_{11}-P_{12})^k)}{P_{12} + (1 - P_{11})}\\
&= \frac{((1-P_{11}) + P_{12})(P_{11}-P_{12})^k}{P_{12} + (1 - P_{11})} = (P_{11} - P_{12})^k    
\end{align*}
the desired formula is established.
\end{proof}

For instance, given a two-state Markov Chain with the stochastic matrix
\begin{align*}
P = 
\begin{bmatrix}
0.35 & 0.75 \\
0.65 & 0.25
\end{bmatrix}
\end{align*}
Its lag-$1$ auto-correlation will be $0.35 - 0.75 = 0.25 - 0.65 = -0.4$, and the lag-$4$ auto-correlation will be $(-0.4)^4 = 0.0256$. As just shown, the lag-$k$ auto-correlation of any (first-order) binary Markov Chain always decays exponentially.

\section{Python Programming and Earth System Applications}

Let's use the 2024 Hualien Earthquake sequence, the mainshock of which occurred on April 3rd as an illustration. First, download the .csv records from \href{https://scweb.cwa.gov.tw/en-us/earthquake/data}{https://scweb.cwa.gov.tw/en-us/earthquake/data} for April and May. We will first read the data in April and extract the largest earthquake magnitude each day.
\begin{lstlisting}
import numpy as np
import pandas as pd

equake_Apr = pd.read_csv("Seismic activity_April.csv", usecols=np.arange(7), parse_dates=["Orgin date"])
equake_Apr_daily_mag_max = equake_Apr.groupby(equake_Apr["Orgin date"].dt.day)["Magnitude"].max()
\end{lstlisting}
Write a custom function to separate the cases into three levels of magnitude: $6.0$--$6.9$ $\to 0$, $5.0$--$5.9$ $\to 1$, $4.0$--$4.9$ $\to 2$.
\begin{lstlisting}
# Three magnitude classes: 4.0-4.9, 5.0-5.9, 6.0-6.9
def classify_mag(mag):
    if mag >= 6.0:
        return(0)
    elif (5.0 <= mag <= 5.9):
        return(1)
    elif (mag <= 4.9):
        return(2)
\end{lstlisting}
Now comes the main part. Shift the categorized time series by one day and count the incidence of transitions between different states using \verb|np.unique| and the function above, to construct the stochastic matrix.
\begin{lstlisting}
equake_Apr_prev_states = (equake_Apr_daily_mag_max.loc[3:29].apply(classify_mag)).values
equake_Apr_next_states = (equake_Apr_daily_mag_max.loc[4:30].apply(classify_mag)).values

transition_matrix = np.zeros([3,3])
trans_states, Apr_counts = np.unique(np.c_[equake_Apr_next_states, equake_Apr_prev_states], return_counts=True, axis=0)

transition_matrix[trans_states[:,0], trans_states[:,1]] = Apr_counts
transition_matrix = transition_matrix/np.sum(transition_matrix, axis=0)
\end{lstlisting}
We have normalized the columns by their respective sums. Finally, we compute the steady-state eigenvector for this Markov Chain. (Note that the eigenvalue of $1$ will be the one with the largest modulus.)
\begin{lstlisting}
import scipy.linalg as linalg

eigvals, eigvecs = linalg.eig(transition_matrix)
q = eigvecs[:, np.argmax(np.abs(eigvals))] / np.sum(eigvecs[:, np.argmax(np.abs(eigvals))])
print(q)
\end{lstlisting}
This gives \verb|[0.0767 0.4063 0.5170]|, which means that on average, during the active period on any day, the magnitude of the most intense earthquake will have $7.7\%$/$40.6\%$/$51.7\%$ to be $\leq 4.9$, within $5.0$--$5.9$, and $\geq 6.0$ respectively. This is obtained over the training dataset. Now let's read the file for May and see if the actual probabilities match those expected by our Markov Chain formulated from data in April.
\begin{lstlisting}
equake_May = pd.read_csv("Seismic activity_May.csv", usecols=np.arange(7), parse_dates=["Orgin date"])
equake_May_daily_mag_max = equake_May.groupby(equake_May["Orgin date"].dt.day)["Magnitude"].max()
equake_May_states = (equake_May_daily_mag_max.apply(classify_mag)).values
May_counts = np.bincount(equake_May_states)
May_expected_counts_from_Markov = q*31
print(May_counts, May_expected_counts_from_Markov)
\end{lstlisting}
which returns \verb|[2 5 24]| and \verb|[2.4 12.6 16.0]| correspondingly. There are differences clearly in the last two bins and we can confirm it using the \textit{Chi-square Test}:
\begin{lstlisting}
from scipy.stats import chisquare
print(chisquare(f_obs=May_counts, f_exp=May_expected_counts_from_Markov))  
\end{lstlisting}
that yields a $p$-value of \verb|0.0135| which exceeds the $5\%$ significance level. Therefore, we can conclude that the actual distribution of Earthquake events over the testing set deviates from that predicted by the Markov Chain with $95\%$ confidence. This makes both logical and physical sense. Earthquakes are dynamic processes, which means that the transition probabilities will not be stationary, violating the most fundamental assumption of a Markov Chain. Moreover, we implicitly use a first-order Markov Chain in which the current state only depends on the previous day. However, the more earlier earthquakes will also consume the accumulated stress in the tectonic plates, influencing and retarding much later earthquake events, so the Markov process should be a higher-order one. Particularly, the seismic activity should decay and its distribution will gradually shift to the dormant side.

\section{Exercise}

\begin{Exercise}
For the island Gu-Nei-Ng-Dou, if it is a sunny day, then the next day has $75\%$ chance of being sunny, $10\%$ chance of being windy, and $15\%$ of being rainy. Meanwhile, a windy day has a $60\%$/$25\%$/$15\%$ chance of leading to a sunny/windy/rainy day, and a rainy day has a $30\%$/$20\%$/$50\%$ chance of leading to a sunny/windy/rainy day. Find:
\begin{enumerate}[label=(\alph*)]
\item the probability of getting a rainy day three days after a rainy day,
\item the steady-state vector for all three types of weather,
\item the probability of getting at least one sunny day today and tomorrow if it was sunny yesterday.
\end{enumerate}
\end{Exercise}
\begin{Answer}
$P=
\begin{bmatrix}
0.75 & 0.6 & 0.3\\
0.1 & 0.25 & 0.2\\
0.15 & 0.15 & 0.5
\end{bmatrix}$
\begin{enumerate}[label=(\alph*)]
\item 
\begin{align*}
\vec{q} &= P^3
\begin{bmatrix}
0\\
0\\
1
\end{bmatrix} \\
&=
\begin{bmatrix}
0.639375&0.636&0.57675\\ 
0.13975&0.143125&0.1595\\ 
0.220875&0.220875&0.26375
\end{bmatrix}
\begin{bmatrix}
0\\
0\\
1
\end{bmatrix} 
=
\begin{bmatrix}
0.57675\\
0.1595\\
0.26375
\end{bmatrix}  
\end{align*}
So the required probability is $26.375\%$.
\item $I-P=
\begin{bmatrix}
0.25 & -0.6 & -0.3\\
-0.1 & 0.75 & -0.2\\
-0.15 & -0.15 & 0.5    
\end{bmatrix}$,\\
and the eigenvector (normalized to the steady-state vector) for $\lambda = 1$ can be found to be $(0.6244, 0.1448, 0.2308)^T$.
\item
\begin{align*}
&\quad \text{The required probability} \\
&= P(\text{today is sunny}) + P(\text{today is not sunny and tomorrow is sunny})\\
&= 0.75 + (0.1 \times 0.6 + 0.15 \times 0.3) \\
&= 0.855
\end{align*}
\end{enumerate}
\end{Answer}

\begin{Exercise}
Go to \href{https://ggweather.com/enso/oni.htm}{https://ggweather.com/enso/oni.htm} for the historical yearly ENSO status and construct a Markov Chain for the El Niño/La Niña/Neutral conditions.
\end{Exercise}

\begin{Exercise}
\label{ex:15.3}
For the carbon cycle shown below (Figure \ref{fig:ex15.3}), calculate the rate constant $k = \frac{\text{flux}}{\text{stock mass}}$ for each transport process, and verify that on average it takes about 213000 years for carbon in the ocean reservoir to be buried in the sediment by (a) absorbing Markov Chain, (b) considering the biosphere-atmosphere-ocean as a single reservoir and calculating the lifetime $\tau = \frac{1}{k} = \frac{\text{total stock mass}}{\text{flux}}$. Also, if the pathway to the sediment is neglected, find the steady-state vector of the biosphere-atmosphere-ocean system and confirm they are already in a steady state as one would expect.
\begin{figure}[ht!]
    \centering
    \includegraphics[scale = 1.1]{graphics/carboncycle.png}
    \caption{\textit{Transport between reservoirs in Exercise \ref{ex:15.3}.}}
    \label{fig:ex15.3}
\end{figure}
\end{Exercise}
\begin{Answer}
Node 1/2/3/4: Biosphere/Atmosphere/Ocean/Sediment. $k_{21} = 0.03$, $k_{12} = 0.1$,
$k_{32} = 0.1$, $k_{23} = 0.0015$, $k_{43} = 0.000005$.
\begin{enumerate}[label=(\alph*)]
\item
\begin{align*}
P &= 
\begin{bmatrix}
0.97 & 0.1 & 0 & 0\\
0.03 & 0.8 & 0.0015 & 0\\
0 & 0.1 & 0.998495 & 0\\
0 & 0 & 0.000005 & 1
\end{bmatrix} \\
P_0 &= 
\begin{bmatrix}
0.97 & 0.1 & 0\\
0.03 & 0.8 & 0.0015\\
0 & 0.1 & 0.998495
\end{bmatrix} \\
I-P_0 &=
\begin{bmatrix}
0.03 & -0.1 & 0\\
-0.03 & 0.2 & -0.0015\\
0 & -0.1 & 0.001505
\end{bmatrix} \\
(I-P_0)^{-1} &=
\begin{bmatrix}
10066.66 & 10033.33 & 10000\\
3010 & 3010 & 3000\\
200000 & 200000 & 200000
\end{bmatrix} 
\end{align*}
The sum of the third column is $213000$ yrs.
\item $\tau = \frac{40000+600+2000}{0.2} = 213000$ yrs.
\end{enumerate}
\end{Answer}

\begin{Exercise}
\label{ex:15.4}
Daniel is drunk in a bar. He wants to go to the train station so that he can go home. However, since he is drunk, he loses his sense of direction, and will move randomly along the green edge (shown in the map as Figure \ref{fig:ex15.4} below) once at a time. Assume that at any location, the chances of moving to all other neighboring locations are equal. Nevertheless, once he arrives at the train station, he will stop wandering and take the train. How many moves does it take on average for Daniel to reach the train station?
\begin{figure}
    \centering
    \includegraphics[scale = 0.3]{graphics/wander.png}
    \caption{\textit{The map for Exercise \ref{ex:15.4}.}}
    \label{fig:ex15.4}
\end{figure}
\end{Exercise}
\begin{Answer}
Index 1/2/3/4/5/6/7/8: Start/A/B/C/D/E/F/End. 
\begin{align*}
P_0 &=
\begin{bmatrix}
0 & \frac{1}{2} & \frac{1}{2} & 0 & 0 & 0 & 0\\
\frac{1}{2} & 0 & 0 & \frac{1}{3} & 0 & 0 & 0\\
\frac{1}{2} & 0 & 0 & 0 & \frac{1}{3} & 0 & 0\\
0 & \frac{1}{2} & 0 & 0 & \frac{1}{3} & \frac{1}{3} & 0\\
0 & 0 & \frac{1}{2} & \frac{1}{3} & 0 & 0 & \frac{1}{3}\\
0 & 0 & 0 & \frac{1}{3} & 0 & 0 & \frac{1}{3}\\
0 & 0 & 0 & 0 & \frac{1}{3} & \frac{1}{3} & 0
\end{bmatrix} \\
I - P_0 &= 
\begin{bmatrix}
1&-\frac{1}{2}&-\frac{1}{2}&0&0&0&0\\ 
-\frac{1}{2}&1&0&-\frac{1}{3}&0&0&0\\ 
-\frac{1}{2}&0&1&0&-\frac{1}{3}&0&0\\ 
0&-\frac{1}{2}&0&1&-\frac{1}{3}&-\frac{1}{3}&0\\ 
0&0&-\frac{1}{2}&-\frac{1}{3}&1&0&-\frac{1}{3}\\ 
0&0&0&-\frac{1}{3}&0&1&-\frac{1}{3}\\ 
0&0&0&0&-\frac{1}{3}&-\frac{1}{3}&1
\end{bmatrix} 
\end{align*}
The first column of the inverse $(I - P_0)^{-1}$ can be checked as $(4,3,3,3,3,\frac{3}{2},\frac{3}{2})^T$ whose sum is $19$, which is the required number of moves.
\end{Answer}

\begin{Exercise}
Attempt \href{https://projecteuler.net/problem=84}{Project Euler Problem 84}, preferably with programming.
\end{Exercise}
\chapter{Matrix Factorization Methods}

In this chapter, we are going to discuss some matrix factorization methods. We have introduced one of them (QR Decomposition) in \autoref{chap:6}. In the past, matrix factorization is not a hot topic in Earth Science. However, as Machine Learning gains popularity in different Scientific fields, many Earth Science research starts to involve matrix factorization, which has been a main instrument in Machine Learning. Apart from this, a less visible usage of matrix factorization is embedded in the implementation of linear algebra package in programming languages (e.g. LAPACK, for Fortran). Those matrix factorization enables a much faster and stable computation of linear algebra problems, such as finding inverses or solving linear systems. Other potential applications include image processing, data compression, and more.

\section{Square Matrix Factorization}
\subsection{Cholesky Factorization}

\textit{Cholesky Decomposition} is for a special class of matrices which are symmetric and positive-definite. We have talked about how a matrix is positive-definite in section \ref{Conic}. For a symmetric and positive-definite matrix $A$, Cholesky Decomposition factorizes it into $U^TU$, where $U$ is an upper triangular matrix, $U^T$ is hence a lower triangular matrix. (Upper/lower triangular implies non-zero entries only present along or above/below the main diagonal) An example would be
\begin{align*}
A = 
\begin{bmatrix}
1 & 0 & 1 \\
0 & 1 & 1 \\
1 & 1 & 6
\end{bmatrix}
&=
\begin{bmatrix}
1 & 0 & 0 \\
0 & 1 & 0 \\
1 & 1 & 2
\end{bmatrix}
\begin{bmatrix}
1 & 0 & 1 \\
0 & 1 & 1 \\
0 & 0 & 2
\end{bmatrix} \\
&= 
\begin{bmatrix}
1 & 0 & 1 \\
0 & 1 & 1 \\
0 & 0 & 2
\end{bmatrix}^T
\begin{bmatrix}
1 & 0 & 1 \\
0 & 1 & 1 \\
0 & 0 & 2
\end{bmatrix} \\
&= U^TU 
\end{align*}

We can compute Cholesky Decomposition step by step, first rewriting the $n \times n$ matrix $A$ into the proposed factorized form of
\begin{align*}
A = U^T U = 
\begin{bmatrix}
u_{11} & \vec{r_1}^T \\
\vec{0} & U_b 
\end{bmatrix}^T
\begin{bmatrix}
u_{11} & \vec{r_1}^T \\
\vec{0} & U_b 
\end{bmatrix} = 
\begin{bmatrix}
u_{11} & \vec{0}^T \\
\vec{r_1} & U_b^T
\end{bmatrix}
\begin{bmatrix}
u_{11} & \vec{r_1}^T \\
\vec{0} & U_b 
\end{bmatrix}
\end{align*}
where $u_{11}$ is the first diagonal element of $U$, $\vec{r_1}$ is a column vector of length $n-1$ and $U_b$ is a block matrix with size $(n-1, n-1)$. The matrix dot product at R.H.S. gives
\begin{align*}
A = 
\begin{bmatrix}
\alpha_{11} & \vec{a_1}^T \\
\vec{a_1} & A_b 
\end{bmatrix}
=
\begin{bmatrix}
u_{11}^2 & u_{11}\vec{r_1}^T \\
u_{11}\vec{r_1} & \vec{r_1}\vec{r_1}^T + U_b^T U_b
\end{bmatrix}
= U^TU
\end{align*}
where $\alpha_{11}$ is the first diagonal element of $A$, $\vec{a_1}$ and $A_b$ is also a column vector just like $\vec{r_1}$ and $U_b$. The dot product involving block matrices is carried out just like usual matrix dot product. Comparing the both sides, we have
\begin{align*}
\alpha_{11} &= u_{11}^2 \\
\vec{a_1} &= u_{11}\vec{r_1} \\
A_b &= \vec{r_1}\vec{r_1}^T + U_b^T U_b
\end{align*}
\begin{align*}
u_{11} &= \sqrt{\alpha_{11}} \\
\vec{r_1} &= \frac{\vec{a_1}}{\sqrt{\alpha_{11}}} \\
U_b^T U_b &= A_b - \vec{r_1}\vec{r_1}^T
\end{align*}
By the relations above, we determine the first row and column of $U$. Subsequently, the remaining block $U_b$ is constructed by applying the same procedure on $A_2 = U_b^T U_b$ which has been obtained from the last relation, and then keep repeating the method to reduce the resulted block matrix on until the last entry is processed.
\begin{defn}
The Cholesky Factorization $U^TU$ of a symmetric, positive-definite matrix $A$, is constructed by the recursive relations
\begin{align*}
u_{mm} &= \sqrt{\alpha_{mm}} \\
\vec{r_m} &= \frac{\vec{a_m}}{\sqrt{\alpha_{mm}}} \\
U_b^T U_b &= A_b - \vec{r_m}\vec{r_m}^T
\end{align*}
where the subscript $m$ implies the $m$-th step. The formula are applied iteratively on $A_m = U_b^T U_b$ acquired at every step.
\end{defn}

\begin{exmp}
Perform Cholesky Factorization on the symmetric, positive-definite matrox
\begin{align*}
A &=
\begin{bmatrix}
4 & 2 & 0 \\
2 & 2 & 2 \\
0 & 2 & 5 
\end{bmatrix}
\end{align*}
The first step results in
\begin{align*}
u_{11} &= \sqrt{4} = \textcolor{red}{2} \\
\vec{r_1} &= \frac{1}{\sqrt{4}}
\begin{bmatrix}
2 \\
0
\end{bmatrix}
= 
\begin{bmatrix}
\textcolor{blue}{1} \\
\textcolor{blue}{0}
\end{bmatrix} \\
A_2 = U_b^T U_b &= 
\begin{bmatrix}
2 & 2 \\
2 & 5
\end{bmatrix}
-
\begin{bmatrix}
1 \\
0
\end{bmatrix}
\begin{bmatrix}
1 & 0
\end{bmatrix} \\
&= \begin{bmatrix}
1 & 2 \\
2 & 5 
\end{bmatrix}
\end{align*}
So we know that
\begin{align*}
U &=
\begin{bmatrix}
\textcolor{red}{2} & \textcolor{blue}{1} & \textcolor{blue}{0} \\
0 & ? & ? \\
0 & ? & ? \\
\end{bmatrix}
\end{align*}
The next iteration $A_2$ on gives
\begin{align*}
u_{22} &= \sqrt{1} = \textcolor{red}{1} \\  
\vec{r_2} &= \frac{1}{\sqrt{1}}
\begin{bmatrix}
2
\end{bmatrix}
=
\begin{bmatrix}
\textcolor{blue}{2}
\end{bmatrix} \\
U_b^T U_b &=
5 - 
\begin{bmatrix}
2
\end{bmatrix}
\begin{bmatrix}
2
\end{bmatrix} \\
&= 
\begin{bmatrix}
1
\end{bmatrix}
\end{align*}
We still need to deal with the $1 \times 1$ matrix that is left. At the third step, we simply take
\begin{align*}
u_{33} = \sqrt{1} = \textcolor{Green}{1}
\end{align*}
So the final expression of $U$ is given by
\begin{align*}
U = 
\begin{bmatrix}
2 & 1 & 0 \\
0 & \textcolor{red}{1} & \textcolor{blue}{2} \\
0 & 0 & \textcolor{Green}{1}
\end{bmatrix}
\end{align*}
Short Exercise: Check if $A = U^T U$.
\end{exmp}

\subsection{LU/LDU Factorization}

\textit{LU Factorization} is similar to Cholesky Factorization. It decomposes any matrix into one upper and one lower triangular matrix, but the original matrix needs not to be symmetric or positive-definite. The key to LU factorization is the method of Gaussian Elimination discussed in sections \ref{Echelon} and \ref{subsection:invGauss}. By Gaussian Elimination we can always reduce a given matrix to an upper triangular matrix (row echelon form), through a sequence of elementary row operations, or equivalently taking the dot product with a sequence of elementary matrices to the left. \\
\\
Denote such operations with $E_1', E_2', \cdots, E_n'$, in the spirit similar to theorem \ref{GaussElimPrinciple}, we have $E_n'\cdots E_2'E_1'A = U$, where $U$ is an upper triangular matrix produced from the forward phase of Gaussian Elimination. Rearrangement gives
\begin{align*}
A = (E_1'^{-1}E_2'^{-1}\cdots E_n'^{-1})U    
\end{align*}
If all $E_i'$ involves no row interchanges, then they would be all lower triangular matrices as required by the forward stage of Gaussian Elimination (if there is row interchange, then $E_i'$ will not be a lower triangular matrix.). As a result, $(E_1'^{-1}E_2'^{-1}\cdots E_n'^{-1})$ will be a lower triangular matrix $L$, and the LU Factorization $A = LU$ is constructed.
\begin{thm}
LU Factorization of a matrix $A$ is possible if it can be reduced to a row echelon form $U$, which will be an upper triangular matrix, by forward Gaussian Elimination. The steps and elementary matrices used to produce $U$ can be grouped together, and inverted to give a lower triangular matrix $L$. The LU Factorization will then be $A = LU$.
\end{thm}

\begin{exmp}
Compute a LU Factorization for
\begin{align*}
A = 
\begin{bmatrix}
2 & 0 & 0 \\
0 & 1 & 1 \\
2 & 0 & 4
\end{bmatrix}
\end{align*}
By Gaussian Elimination, we have
\begin{align*}
\begin{bmatrix}
2 & 0 & 0 \\
0 & 1 & 1 \\
2 & 0 & 4
\end{bmatrix}
&\rightarrow 
\begin{bmatrix}
1 & 0 & 0 \\
0 & 1 & 1 \\
2 & 0 & 4 
\end{bmatrix}
& E_1' =
\begin{bmatrix}
\frac{1}{2} & 0 & 0 \\
0 & 1 & 0 \\
0 & 0 & 1 \\
\end{bmatrix}
\\
&\rightarrow 
\begin{bmatrix}
1 & 0 & 0 \\
0 & 1 & 1 \\
0 & 0 & 4 
\end{bmatrix}
&
E_2' =
\begin{bmatrix}
1 & 0 & 0 \\
0 & 1 & 0 \\
-2 & 0 & 1 
\end{bmatrix}
\\
&\rightarrow 
\begin{bmatrix}
1 & 0 & 0 \\
0 & 1 & 1 \\
0 & 0 & 1
\end{bmatrix}
&
E_3' = 
\begin{bmatrix}
1 & 0 & 0 \\
0 & 1 & 0 \\
0 & 0 & \frac{1}{4}
\end{bmatrix}
\end{align*}
Therefore, we can take
\begin{align*}
U &= 
\begin{bmatrix}
1 & 0 & 0 \\
0 & 1 & 1 \\
0 & 0 & 1 
\end{bmatrix}
\\
L &= E_3'^{-1}E_2'^{-1}E_1'^{-1} \\
&=
\begin{bmatrix}
\frac{1}{2} & 0 & 0 \\
0 & 1 & 0 \\
0 & 0 & 1 
\end{bmatrix}^{-1}
\begin{bmatrix}
1 & 0 & 0 \\
0 & 1 & 0 \\
-2 & 0 & 1 
\end{bmatrix}^{-1}
\begin{bmatrix}
1 & 0 & 0 \\
0 & 1 & 0 \\
0 & 0 & \frac{1}{4} 
\end{bmatrix}
^{-1} \\
&= \begin{bmatrix}
2 & 0 & 0 \\
0 & 1 & 0 \\
0 & 0 & 1 
\end{bmatrix}
\begin{bmatrix}
1 & 0 & 0 \\
0 & 1 & 0 \\
2 & 0 & 1 
\end{bmatrix}
\begin{bmatrix}
1 & 0 & 0 \\
0 & 1 & 0 \\
0 & 0 & 4 
\end{bmatrix}
\end{align*}
This sequence amounts to a series of elementary row operations. Doing them starting from the right to the left, it is easy to arrive at
\begin{align*}
L &=
\begin{bmatrix}
2 & 0 & 0 \\
0 & 1 & 0 \\
2 & 0 & 4
\end{bmatrix}
\end{align*}
\end{exmp}

Note that LU Factorization is not unique. However, one can derive \textit{LDU Factorization} that is unique given it has one, where $D$ is a diagonal matrix, while $L$ and $U$ have all the diagonal entries being one. Using the above example, we observe that
\begin{align*}
L &=
\begin{bmatrix}
2 & 0 & 0 \\
0 & 1 & 0 \\
2 & 0 & 4
\end{bmatrix} = 
\begin{bmatrix}
1 & 0 & 0 \\
0 & 1 & 0 \\
1 & 0 & 1
\end{bmatrix}
\begin{bmatrix}
2 & 0 & 0 \\
0 & 1 & 0 \\
0 & 0 & 4
\end{bmatrix}
\end{align*}
where we factor out a diagonal matrix from $L$ to force all the entries along the main diagonal to be $1$. So the corresponding LDU Factorization is
\begin{align*}
A =
\begin{bmatrix}
1 & 0 & 0 \\
0 & 1 & 0 \\
1 & 0 & 1
\end{bmatrix}
\begin{bmatrix}
2 & 0 & 0 \\
0 & 1 & 0 \\
0 & 0 & 4
\end{bmatrix}
\begin{bmatrix}
1 & 0 & 0 \\
0 & 1 & 1 \\
0 & 0 & 1 
\end{bmatrix}
\end{align*}

There is another variant of LU Factorization which is called \textit{PLU Factorization}, applicable to any matrix $A$. The idea is to first interchange the rows in $A$ so that LU Factorization becomes possible. After obtaining the LU Factorization, we collect all the elementary row matrices involved in the initial row interchanges as a matrix $P$ to be added at the left.

\subsection{Solving Linear Systems with LU Decomposition}
As mentioned in the start of the chapter, matrix factorization helps a lot when it comes to electronic calculation. For instance, $LU$ factorization is widely applied in solving linear systems in computer, preferred over direct Gaussian Elimination or using inverse, due to the reasons we are going to discuss later.\\
\\
Basically, it involves a two step procedure. For a linear system $A\vec{x} = \vec{h}$, if we have the LU Decomposition, then it can be rewritten as $LU\vec{x} = \vec{h}$. It then becomes a two step process to solve the system, where we let $U\vec{x} = \vec{y}$, and solve $L\vec{y} = \vec{h}$ first, and then come back to solve $U\vec{x} = \vec{y}$. Due to the nature of $L$ and $U$, we can do the forward and backward substitution to solve the two systems with relative ease.

\begin{exmp}
Given a linear system $A\vec{x} = \vec{h}$, where
\begin{align*}
A &= 
\begin{bmatrix}
1 & 1 & 0 \\
0 & 2 & 2 \\
1 & 2 & 4 
\end{bmatrix}
& \vec{h} = 
\begin{bmatrix}
3 \\
2 \\
4
\end{bmatrix}
\end{align*}
and we are given the LU factorization of A as
\begin{align*}
A &= LU = 
\begin{bmatrix}
1 & 0 & 0 \\
0 & 2 & 0 \\
1 & 1 & 3 
\end{bmatrix}
\begin{bmatrix}
1 & 1 & 0 \\
0 & 1 & 1 \\
0 & 0 & 1 
\end{bmatrix}
\end{align*}
We can solve the equivalent problem $LU\vec{x} = \vec{h}$, where $U\vec{x} = \vec{y}$ and thus $L\vec{y} = \vec{h}$. The latter one leads to
\begin{align*}
\begin{bmatrix}
1 & 0 & 0 \\
0 & 2 & 0 \\
1 & 1 & 3 
\end{bmatrix}
\begin{bmatrix}
y_1 \\
y_2 \\
y_3
\end{bmatrix}
&=
\begin{bmatrix}
3 \\
2 \\
4
\end{bmatrix} \\
\vec{y} = 
\begin{bmatrix}
y_1 \\
y_2 \\
y_3
\end{bmatrix}
&=
\begin{bmatrix}
3 \\
1 \\
0
\end{bmatrix}
\end{align*}
We leave the last step of solving $U\vec{x} = \vec{y}$ to the readers, the solution of which is $\vec{x} = (2,1,0)^T$.
\end{exmp}

\paragraph{Remark} To solve a family of linear systems $A\vec{x} = \vec{h_i}$, where $\vec{h_i}$ are many different column vectors, can be very time-consuming if we use Gaussian Elimination every time. LU Decomposition extracts and encapsulates the information from Gaussian Elimination, and saves the effort needed to compute Gaussian Elimination for different $\vec{h_i}$ every time. While it is also possible to achieve similar effects by utilizing the inverse $A^{-1}$, calculating the inverse for large-scale systems can be unstable and expensive.

\section{Non-Square Matrix Factorization}

\subsection{Singular Value Decomposition}

LU Factorization can also be applied on a non-square matrix. However, there is another factorization method called \textit{Singular Value Decomposition} that is very useful in data compression. This method is not simple, and the readers may need some time to digest the instructions below. First, let's define what are singular values.
\begin{defn}
For a $m \times n$ matrix $A$, its singular values $\sigma_i$ are the square roots of the eigenvalues $\lambda_i$ of $A^TA$ (an $n \times n$ symmetric matrix, guaranteed to have $n$ eigenvalue-eigenvector pairs by theorem \ref{symdiag}). So
\begin{align*}
\sigma_i = \sqrt{\lambda_i}
\end{align*}
for $i = 1, 2, \cdots, n$.
\end{defn}
Attentive readers may notice a potential pitfall if $\lambda_i$ is negative. However, it cannot be the case. Consider the norm/length of the vector $A\hat{v_i}$ where $\hat{v_i}$ is the $i$-th unit eigenvector of $A^TA$. Then
\begin{align*}
\norm{A\hat{v_i}}^2 &= A\hat{v_i} \cdot A\hat{v_i}\\
&= \hat{v_i} \cdot (A^TA\hat{v_i}) \\
&= \hat{v_i} \cdot (\lambda_i\hat{v_i}) \\
&= \lambda_i (\hat{v_i} \cdot \hat{v_i}) = \lambda_i \norm{\hat{v_i}}^2 = \lambda_i
\end{align*}
where properties \ref{dotproper} and the definition \ref{eigen} of eigenvector are used. Since $\norm{A\vec{v_i}}^2$ must be non-negative, $\lambda_i$ will be non-negative as well. For example, the matrix
\begin{align*}
A &= 
\begin{bmatrix}
1 & 0 & 1\\
2 & 1 & 0
\end{bmatrix}
\end{align*}
leads to
\begin{align*}
A^TA &= 
\begin{bmatrix}
1 & 2\\
0 & 1 \\
1 & 0
\end{bmatrix} 
\begin{bmatrix}
1 & 0 & 1\\
2 & 1 & 0
\end{bmatrix} \\
&=
\begin{bmatrix}
5 & 2 & 1 \\
2 & 1 & 0 \\
1 & 0 & 1
\end{bmatrix}
\end{align*}
as a symmetric matrix that has eigenvalues of $\lambda = 6, 1, 0$, and hence $A$ has a sequence of singular values of $\sigma = \sqrt{6}, 1, 0$. The readers are invited to confirm the computation. We also note a key observation, that we are not going to prove. (It requires some fundamental concepts that we have chosen not to be included in the book)
\begin{thm}
For a $m \times n$ matrix $A$, where $m < n$, so that $A$ has more columns than rows, then there must be at least $n-m$ zero eigenvalues for $A^TA$, and hence the same number of zero singular values for $A$.
\end{thm}
If $m > n$, it is obvious that there are only at most $n$ eigenvalues for $A^TA$. Collectively, it implies that for a $m \times n$ matrix $A$, it will have at most $\min(m,n)$ (the smaller of $m$ and $n$) non-zero singular values. This is consistent with the example we have seen above. Now we are prepared to see how Singular Value Decomposition is done.
\begin{defn}
The Singular Value Decomposition for a $m \times n$ matrix $A$ is
\begin{align*}
A = U\Sigma V^T
\end{align*}
where $U$, $\Sigma$, $V$ is a $m \times m$, $m \times n$ and $n \times n$ matrix. $V$ is constructed by
\begin{align*}
V = 
\begin{bmatrix}
\hat{v_1} | \hat{v_2} | \cdots | \hat{v_n}
\end{bmatrix}
\end{align*}
where the unit column vectors $\hat{v_1}, \hat{v_2}, \cdots, \hat{v_n}$ are the orthonormal eigenvectors of the symmetric matrix $A^TA$ (refers to the notes for theorem \ref{symdiag}), ordered by decreasing eigenvalues. $\Sigma$ consists of the non-zero singular values $\sigma_j$ (also in decreasing order) of the corresponding column vectors $\hat{v_j}$ in $V$ along the main diagonal, and has a value of zero elsewhere. $U$ is made up of the unit column vectors (that are also orthogonal to each other) that satisfies
\begin{align*}
\hat{u_j} = \frac{1}{\sigma_j} A\hat{v_j}
\end{align*}
for all $j$ that represents a non-zero singular value $\sigma_j \neq 0$. In the case where there are not enough column vectors to construct $U$ by the relation above, use Gram-Schmidt Orthogonalization (or other feasible methods) to create orthonormal vectors that extend the basis of $\hat{u_j}$ up to $j = m$.
\end{defn}

\begin{exmp}
Again, for the $2 \times 3$ matrix
\begin{align*}
A &= 
\begin{bmatrix}
1 & 0 & 1\\
2 & 1 & 0
\end{bmatrix}
\end{align*}    
We have show that the eigenvalues of $A^TA$ are $\lambda = 6,1,0$, and the singular values of $A$ are $\sigma = \sqrt{6}, 1, 0$. The corresponding unit eigenvectors for $A^TA$ are $(\frac{\sqrt{5}}{\sqrt{6}}, \frac{\sqrt{2}}{\sqrt{15}}, \frac{1}{\sqrt{30}})^T$, $(0, -\frac{1}{\sqrt{5}}, \frac{2}{\sqrt{5}})^T$, and $(-\frac{1}{\sqrt{6}}, \frac{2}{\sqrt{6}}, \frac{1}{\sqrt{6}})^T$. So we have
\begin{align*}
&V =
\begin{bmatrix}
\frac{\sqrt{5}}{\sqrt{6}} & 0 & -\frac{1}{\sqrt{6}}\\
\frac{\sqrt{2}}{\sqrt{15}} & -\frac{1}{\sqrt{5}} & \frac{2}{\sqrt{6}}\\
\frac{1}{\sqrt{30}} & \frac{2}{\sqrt{5}} & \frac{1}{\sqrt{6}}
\end{bmatrix}
&\Sigma =
\begin{bmatrix}
\sqrt{6} & 0 & 0 \\
0 & 1 & 0
\end{bmatrix}
\end{align*}
The two column vectors for $U$ are found by
\begin{align*}
\hat{u_1} &= \frac{1}{\sigma_1} A\hat{v_1} \\
&= \frac{1}{\sqrt{6}}
\begin{bmatrix}
1 & 0 & 1\\
2 & 1 & 0
\end{bmatrix}
\begin{bmatrix}
\frac{\sqrt{5}}{\sqrt{6}} \\
\frac{\sqrt{2}}{\sqrt{15}} \\
\frac{1}{\sqrt{30}}
\end{bmatrix} \\
&=
\begin{bmatrix}
\frac{1}{\sqrt{5}} \\
\frac{2}{\sqrt{5}}
\end{bmatrix}
\end{align*}
and
\begin{align*}
\hat{u_2} &= \frac{1}{1}
\begin{bmatrix}
1 & 0 & 1\\
2 & 1 & 0
\end{bmatrix}
\begin{bmatrix}
0 \\
-\frac{1}{\sqrt{5}} \\
\frac{2}{\sqrt{5}}
\end{bmatrix} \\
&= 
\begin{bmatrix}
\frac{2}{\sqrt{5}} \\
-\frac{1}{\sqrt{5}}
\end{bmatrix}
\end{align*}
\end{exmp}
Therefore, we conclude that
\begin{align*}
U &=
\begin{bmatrix}
\frac{1}{\sqrt{5}} & \frac{2}{\sqrt{5}} \\
\frac{2}{\sqrt{5}} & -\frac{1}{\sqrt{5}}
\end{bmatrix} \\
A &= U\Sigma V^T \\
&= 
\begin{bmatrix}
\frac{1}{\sqrt{5}} & \frac{2}{\sqrt{5}} \\
\frac{2}{\sqrt{5}} & -\frac{1}{\sqrt{5}}
\end{bmatrix}
\begin{bmatrix}
\sqrt{6} & 0 & 0 \\
0 & 1 & 0
\end{bmatrix}
\begin{bmatrix}
\frac{\sqrt{5}}{\sqrt{6}} & \frac{\sqrt{2}}{\sqrt{15}} & \frac{1}{\sqrt{30}} \\
0 & -\frac{1}{\sqrt{5}} & \frac{2}{\sqrt{5}} \\
-\frac{1}{\sqrt{6}} & \frac{2}{\sqrt{6}} & \frac{1}{\sqrt{6}}
\end{bmatrix}
\end{align*}

\begin{exmp}
Carry out Singular Value Decomposition for the $3 \times 2$ matrix
\begin{align*}
A =
\begin{bmatrix}
1 & 1 \\
0 & 1 \\
1 & 0
\end{bmatrix}
\end{align*}
We leave to the readers to check that the orthonormal eigenvectors of 
\begin{align*}
A^TA = 
\begin{bmatrix}
2 & 1 \\
1 & 2
\end{bmatrix}
\end{align*}
are 
\begin{align*}
& \hat{v_1} =
\begin{bmatrix}
\frac{1}{\sqrt{2}} \\
\frac{1}{\sqrt{2}}
\end{bmatrix}
& \hat{v_2} =
\begin{bmatrix}
-\frac{1}{\sqrt{2}} \\
\frac{1}{\sqrt{2}}
\end{bmatrix}
\end{align*}
for $\lambda_1 = 3$, $\lambda_2 = 1$, so $\sigma_1 = \sqrt{3}$, $\sigma_2 = 1$, and
\begin{align*}
\Sigma =
\begin{bmatrix}
\sqrt{3} & 0 \\
0 & 1 \\
0 & 0
\end{bmatrix}   
\end{align*}
$\hat{u_1}$, $\hat{u_2}$ are then given by
\begin{align*}
\hat{u_1} &= \frac{1}{\sqrt{3}}A
\begin{bmatrix}
\frac{1}{\sqrt{2}} \\
\frac{1}{\sqrt{2}}
\end{bmatrix}
=
\begin{bmatrix}
\frac{2}{\sqrt{6}} \\
\frac{1}{\sqrt{6}} \\
\frac{1}{\sqrt{6}} 
\end{bmatrix} \\
\hat{u_1} &= \frac{1}{1}A
\begin{bmatrix}
-\frac{1}{\sqrt{2}} \\
\frac{1}{\sqrt{2}}
\end{bmatrix}
=
\begin{bmatrix}
0 \\
\frac{1}{\sqrt{2}} \\
-\frac{1}{\sqrt{2}} 
\end{bmatrix}
\end{align*}
We need another unit vector $\hat{u_3}$ that is orthogonal to both $\hat{u_1}$, $\hat{u_2}$. A quick way for the special case of three-dimensional vector is to find the cross product $\hat{u_1} \times \hat{u_2}$, that is
\begin{align*}
\hat{u_3} &= \hat{u_1} \times \hat{u_2} \\
&=
\begin{vmatrix}
\hat{i} & \hat{j} & \hat{k} \\
\frac{2}{\sqrt{6}} & \frac{1}{\sqrt{6}} & \frac{1}{\sqrt{6}}  \\
0 & \frac{1}{\sqrt{2}} & -\frac{1}{\sqrt{2}}
\end{vmatrix}
= -\frac{1}{\sqrt{3}} \hat{i} + \frac{1}{\sqrt{3}} \hat{j} + \frac{1}{\sqrt{3}} \hat{k}
\end{align*}
\end{exmp}
As a result,
\begin{align*}
U &=
\begin{bmatrix}
\frac{2}{\sqrt{6}} & 0 & -\frac{1}{\sqrt{3}} \\
\frac{1}{\sqrt{6}} & \frac{1}{\sqrt{2}} & \frac{1}{\sqrt{3}}\\
\frac{1}{\sqrt{6}} & -\frac{1}{\sqrt{2}} & \frac{1}{\sqrt{3}}
\end{bmatrix} \\
A &= U\Sigma V^T \\
&= 
\begin{bmatrix}
\frac{2}{\sqrt{6}} & 0 & -\frac{1}{\sqrt{3}} \\
\frac{1}{\sqrt{6}} & \frac{1}{\sqrt{2}} & \frac{1}{\sqrt{3}}\\
\frac{1}{\sqrt{6}} & -\frac{1}{\sqrt{2}} & \frac{1}{\sqrt{3}}
\end{bmatrix}
\begin{bmatrix}
\sqrt{3} & 0 \\
0 & 1 \\
0 & 0
\end{bmatrix} 
\begin{bmatrix}
\frac{1}{\sqrt{2}} & \frac{1}{\sqrt{2}}\\
-\frac{1}{\sqrt{2}} & \frac{1}{\sqrt{2}}
\end{bmatrix}
\end{align*}

\subsection{Truncated Singular Value Decomposition}
The Singular Value Decomposition of a matrix $A$
\begin{align*}
A = U\Sigma V^T = \begin{bmatrix}
\hat{u_1}|\hat{u_2}|\cdots|\hat{u_m}
\end{bmatrix}
\begin{bmatrix}
\sigma_1 & 0 & \cdots \\
0 & \sigma_2 & \\
\vdots & & \ddots
\end{bmatrix}
\left[
\begin{array}{c}
\hat{v_1}^T \\
\hline
\hat{v_2}^T \\
\hline
\cdots \\
\hline
\hat{v_n}^T
\end{array}
\right]
\end{align*}
can be reduced to the so-called \textit{Compact SVD}, by only keeping entries corresponds up to the last ($k$-th) non-zero singular value, where $U_k$ and $V_k$ only keeps the first $k$ columns, with $\Sigma_k$ being a diagonal $k \times k$ matrix that preserves the first $k$ non-zero singular values along the main diagonal. This results in
\begin{align*}
A = U\Sigma V^T = \begin{bmatrix}
\hat{u_1}|\hat{u_2}|\cdots|\hat{u_k}
\end{bmatrix}
\begin{bmatrix}
\sigma_1 & 0 & \cdots & 0 \\
0 & \sigma_2 & & \\
\vdots & & \ddots & \\
0 & & & \sigma_k
\end{bmatrix}
\left[
\begin{array}{c}
\hat{v_1}^T \\
\hline
\hat{v_2}^T \\
\hline
\cdots \\
\hline
\hat{v_k}^T
\end{array}
\right]
\end{align*}
It can also be expanded in a manner similar to Spectral Decomposition suggested at the end of section \ref{orthogonaldiagreal}, which gives
\begin{align*}

\end{align*}


\subsection{Relations to Principal Component Analysis}
\chapter{Index Notation and Introduction to Tensor}
\chapter*{Answers to Exercises}
\addcontentsline{toc}{chapter}{Answers to Exercises}
\rohead{Answer to Exercises}
\lehead{Answer to Exercises}
\shipoutAnswer
\printindex
\rohead{Index}
\lehead{Index}
%\addcontentsline{toc}{chapter}{Index}

\printbibliography[heading=bibintoc]
\rohead{Bibliography}
\lehead{Bibliography}
%\addcontentsline{toc}{chapter}{Bibliography}

\end{document}