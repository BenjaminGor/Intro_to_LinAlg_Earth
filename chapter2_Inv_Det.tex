\chapter{Inverses and Determinants}

In this chapter, we are going to discuss two important concepts about matrices, which are their \textit{inverses} and \textit{determinants}. They will appear from time to time in the remaining parts of this book. To derive them, we need to introduce some prerequisite ideas first, including the \textit{identity matrix}, \textit{transpose}, and the methods of \textit{Gaussian Elimination} and \textit{cofactor expansion}.

\section{Identity Matrices and Transpose}

\subsection{Identity Matrices}
One important class of matrices is the \index{Identity Matrix}\keywordhl{identity matrices}. They are $n \times n$ square matrices, where $n$ can be any positive integer, with entries along the \textit{main diagonal} (where index of row $=$ column) being $1$ and other off-diagonal elements being $0$. Usually, they are denoted by $I_n$, or simply $I$.
\begin{align*}
I_2 &= 
\begin{bmatrix}
\textcolor{red}{1} & 0 \\
0 & \textcolor{red}{1}
\end{bmatrix}
& I_3 &= 
\begin{bmatrix}
\textcolor{red}{1} & 0 & 0 \\
0 & \textcolor{red}{1} & 0 \\
0 & 0 & \textcolor{red}{1}
\end{bmatrix}
\end{align*}
Identity matrices of size $2 \times 2$ and $3 \times 3$ with the main diagonal $1$s highlighted.
\begin{defn}[Identity Matrix]
\label{defn:identity}
An identity matrix of the square shape $n \times n$ $I_n$ is defined as $[I_{n}]_{ij} = 1$, for $i = j$, and $[I_{n}]_{ij} = 0$, for $i \neq j$, where $1 \leq i,j \leq n$.
\end{defn}
Short Exercise: Explicitly write down $I_5$.\footnote{$I_5=
\begin{bmatrix}
1 & 0 & 0 & 0 & 0 \\
0 & 1 & 0 & 0 & 0 \\
0 & 0 & 1 & 0 & 0 \\
0 & 0 & 0 & 1 & 0 \\
0 & 0 & 0 & 0 & 1
\end{bmatrix}$\\}\\
\\
One important property of identity matrices is
\begin{proper}
\label{proper:identity}
Matrix product between any matrix $A$ with an identity matrix $I$ always returns $A$ whenever the matrix product is defined. If $A$ is of the shape $m \times n$, then $AI_n = I_mA = A$. If A is now a square matrix such that $m=n$ (and $I_m = I_n = I$), then we have $AI = IA = A$.
\end{proper}
In other words, the identity $I$ can be regarded to be the "$1$" in the world of matrices. This is one of the cases that $AB = BA$ commutes (if both of them are square and either one of them is the identity matrix). Using the matrix
\begin{align*}
A =
\begin{bmatrix}
a & b & c \\
d & e & f
\end{bmatrix}
\end{align*}
as an example, the readers can try to computed $AI_3$ and $I_2A$ to see if the results are $A$ itself.\footnote{\vspace{-15pt}
\begin{align*}
AI_3 &=
\begin{bmatrix}
a & b & c \\
d & e & f
\end{bmatrix}
\begin{bmatrix}
1 & 0 & 0\\
0 & 1 & 0 \\
0 & 0 & 1
\end{bmatrix} \\
&=
\begin{bmatrix}
(a)(1) + (b)(0) + (c)(0) & (a)(0) + (b)(1) + (c)(0) &  (a)(0) + (b)(0) + (c)(1) \\
(d)(1) + (e)(0) + (f)(0) & (d)(0) + (e)(1) + (f)(0) &  (d)(0) + (e)(0) + (f)(1)
\end{bmatrix} \\
&=
\begin{bmatrix}
a & b & c \\
d & e & f
\end{bmatrix}
= A    
\end{align*} The calculation of $I_2A = A$ is similar. }

\subsection{Transpose}
\index{Transpose}\keywordhl{Transpose} of a matrix, denoted by adding the superscript $^T$, is formed by interchanging its rows and columns, that is, flipping the elements about the main diagonal.
\begin{defn}[Transpose]
The transpose of an $m \times n$ matrix $A$, denoted as $A^T$, is formed according to the relation $[A^T]_{pq} = A_{qp}$, $1 \leq p \leq n$, $1 \leq q \leq m$, i.e.\ swapping the row and column indices. Now $A^T$ is an $n \times m$ matrix.
\end{defn}
Two examples are given below to show the effect of applying transpose on matrices.
\begin{align*}
A &= 
\begin{bmatrix}
1 & 2 & 3 \\
4 & 5 & 6
\end{bmatrix}
& A^T &= 
\begin{bmatrix}
1 & 4 \\
2 & 5 \\
3 & 6
\end{bmatrix} \\
B &= 
\begin{tikzpicture}[baseline=-\the\dimexpr\fontdimen22\textfont2\relax]
\matrix(mymatrix)[matrix of math nodes, left delimiter={[}, 
right delimiter={]}, row sep=2pt, column sep=-1pt, outer sep=-6pt, nodes={text width=14pt, align=center}, ampersand replacement=\&]
{1 \& -4 \& 3 \\
2 \& -2 \& 0 \\
-3 \& 1 \& 4 \\};
\draw (mymatrix-1-1.center) edge[dashed, thick, Red] (mymatrix-3-3.center);
\draw (mymatrix-1-2.center) edge[dashed, Green, <->] (mymatrix-2-1.center);
\draw (mymatrix-1-3.center) edge[dashed, Green, <->] (mymatrix-3-1.center);
\draw (mymatrix-2-3.center) edge[dashed, Green, <->] (mymatrix-3-2.center);
\end{tikzpicture}
& B^T &= 
\begin{tikzpicture}[baseline=-\the\dimexpr\fontdimen22\textfont2\relax]
\matrix(mymatrix)[matrix of math nodes, left delimiter={[}, 
right delimiter={]}, row sep=2pt, column sep=-1pt, outer sep=-6pt, nodes={text width=14pt, align=center}, ampersand replacement=\&]
{1 \& 2 \& -3 \\
-4 \& -2 \& 1 \\
3 \& 0 \& 4 \\};
\draw (mymatrix-1-1.center) edge[dashed, thick, Red] (mymatrix-3-3.center);
\draw (mymatrix-1-2.center) edge[dashed, Green, <->] (mymatrix-2-1.center);
\draw (mymatrix-1-3.center) edge[dashed, Green, <->] (mymatrix-3-1.center);
\draw (mymatrix-2-3.center) edge[dashed, Green, <->] (mymatrix-3-2.center);
\end{tikzpicture}
\end{align*}
Particularly, in the second example, we have outlined the main diagonal of $B$ (as well as $B^T$) and how the elements flip about it when transpose is carried out. Some useful properties about transpose are listed as follows.
\begin{proper}
\label{proper:transp}
For two matrices $A$ and $B$, we have
\begin{enumerate}
\item $(cA)^T = cA^T$, where $c$ is any constant;
\item $(A^T)^T = A$, i.e.\ transposing twice returns the original matrix (which is obvious);
\item $(A \pm B)^T = A^T \pm B^T$, if $A$ and $B$ have the same shape;
\item $(AB)^T = B^TA^T$, if $A$ and $B$ are conformable;
\item $A_{kk} = A^T_{kk}$ for any $k$ that $A_{kk}$ is defined, i.e.\ the main diagonal is unaffected by transpose.
\end{enumerate}
\end{proper}
Short Exercise: Show that $(ABC)^T = C^TB^TA^T$ if the matrices have compatible shapes for the matrix multiplcation.\footnote{By (4), $(ABC)^T = ((AB)(C))^T = C^T(AB)^T = C^TB^TA^T$}

\subsection{Symmetric Matrices}
A \index{Symmetric Matrix}\keywordhl{symmetric matrix} has its elements mirrored about the main diagonal. Taking transpose of such a matrix will leave it unchanged. Implicitly, it is required to be a square matrix.
\begin{defn}[Symmetric Matrix]
If an $n \times n$ square matrix $A$ and its transpose $A^T$ are equal, i.e.\ $A_{pq} = [A^T]_{pq} = A_{qp}$ for all $1 \leq p, q \leq n$, or simply $A = A^T$, then $A$, and also $A^T$, are symmetric.
\end{defn}
As an example,
\begin{flalign*}
&\begin{bmatrix}
1 & 2 & -2 \\
2 & 0 & 4 \\
-2 & 4 & 3
\end{bmatrix}&
\end{flalign*}
is a $3 \times 3$ symmetric matrix.\\
\\
Short Exercise: Show that $Y = XX^T$ and $Z = X^TX$ are symmetric for any matrix $X$.\footnote{By Properties \ref{proper:transp}, $Y^T = (XX^T)^T = (X^T)^T(X)^T = XX^T = Y$, similar goes for $Z = X^TX$.}\par
In contrast, we also have \index{Skew-symmetric Matrix}\keywordhl{skew-symmetric matrices} such that $A^T = -A$. This automatically requires elements along the main diagonal to be all zeros.
\begin{flalign*}
&\begin{bmatrix}
0 & 2 & 1 \\
-2 & 0 & -3 \\
-1 & 3 & 0
\end{bmatrix}&
\end{flalign*}
A $3 \times 3$ skew-symmetric matrix.

\section{Inverses}
\label{section:inv}
\subsection{Definition and Properties of Inverses}
\label{subsection:invsub}
\index{Inverse}\keywordhl{Inverse} of a square matrix, denoted by appending the superscript $^{-1}$, is another square matrix such that the matrix product between these two matrices (in either order) yields an identity matrix.
\begin{defn}[Inverse]
\label{defn:inverse}
An $n \times n$ square matrix $B$ is said to be the inverse of another $n \times n$ square matrix $A$ if $AB = BA = I_n$. This inverse matrix is denoted as $B = A^{-1}$, and the relation becomes $AA^{-1} = A^{-1}A = I$. The opposite direction also holds, i.e.\ $A$ is the inverse of $A^{-1}$. Hence, we say that $A$ and $A^{-1}$ are the inverse of each other.
\end{defn}
If there exists an inverse $A^{-1}$ for the square matrix $A$, then both $A$ and $A^{-1}$ are called \index{Invertible}\keywordhl{invertible}. Otherwise, $A$ is said to be \index{Singular}\keywordhl{singular}. This is another situation in which a matrix product $AB = BA$ (if $B=A^{-1}$) can commute.\footnote{$AA^{-1} = I$ implies $A^{-1}A = I$ and vice versa. However, while looking innocent, showing this is actually not trivial and prone to circular logic. A heuristic way to "prove" it is to note that
\begin{align*}
AA^{-1} &= I \\
AA^{-1}A &= IA \\
A(A^{-1}A) &= A
\end{align*}
which implies that multiplying $A$ by $A^{-1}A$ returns $A$ itself, so it should be reasonable to assume $A^{-1}A = I$.}\par %We can show why $AA^{-1} = I$ implies $A^{-1}A = I$, or vice versa.
%\begin{proof}
%Assume only $AA^{-1} = I$ is true, then multiplying $A$ to the right on both sides of the equation leads to
%\begin{align*}
%AA^{-1}A &= IA \\
%A(A^{-1}A) &= A & & \text{(Properties \ref{proper:matmul} and Properties \ref{proper:identity})}\\
%AP &= A
%\end{align*}
%where we write $P = A^{-1}A$. The above implies that multiplying $A$ by $P$ returns $A$ itself. We are tempted to claim that $P = I$ by observing Properties \ref{proper:identity}. However, it is not trivial to show that $P$ cannot be any matrix other than $I$, and to do so we have to wait until later chapters. Nevertheless, it is indeed true as long as $A$ is invertible. Therefore, $P = A^{-1}A = I$. The opposite direction is proved similarly.    
%\end{proof}
In the last chapter, we only define addition, subtraction, and multiplication for matrices, omitting division like it is an elephant in the room. The inverse serves as a remedy for this by acting as the reciprocal in the world of matrices. This allows us to "divide" on both sides of a matrix equation provided the relevant inverse exists. Remember, in the last chapter, we mentioned that cancellation may not work for something like $AB = AC$. But if $A^{-1}$ exists, then by multiplying it to the left of both sides of the equation, we can effectively "divide by $A$" 
\begin{align*}
AB &= AC \\
A^{-1}AB &= A^{-1}AC \\
(A^{-1}A)B &= (A^{-1}A)C & \text{(Properties \ref{proper:matmul})} \\
IB &= IC & \text{(Definition \ref{defn:inverse})} \\
B &= C & \text{(Properties \ref{proper:identity})}
\end{align*}
so that cancellation holds in this situation. Take the matrix equation $AG = H$ as another example, if $A$ has an inverse $A^{-1}$, then we may do a matrix "division" as follows:
\begin{align*}
AG &= H \\
A^{-1}AG &= A^{-1}H \\
\textcolor{gray}{((A^{-1}A)G = IG =)} \; G &= A^{-1}H & 
\text{(Properties \ref{proper:matmul} and \ref{proper:identity}, Definition \ref{defn:inverse})}
\end{align*}
In addition, the inverse of a matrix, if exists, must be unique.
\begin{proper}[Uniqueness of Inverse]
\label{proper:uniqueinverse}
If $A$ has an inverse $A^{-1}$, it is unique.
\end{proper}
\begin{proof}
This property can be proved easily by first assuming that the invertible matrix $A$ has two different inverses, $B$ and $C$. Subsequently, by Definition \ref{defn:inverse}, we have $BA = I$ (and also $AC = I$). Multiplying by $C$ to the right on both sides gives
\begin{align*}
BAC &= IC \\
B(AC) &= C & \text{(Properties \ref{proper:matmul} and \ref{proper:identity})}\\
B(I) &= C & \text{($AC = I$ from assumption)} \\
B &= C & \text{(Properties \ref{proper:identity})}
\end{align*}
So, $B$ and $C$ are actually the same matrix, implying that the inverse of $A$ is unique.    
\end{proof}
\begin{exmp}
Let 
\begin{align*}
& A =
\begin{bmatrix}
4 & 6 \\
3 & 5
\end{bmatrix}
& B =
\begin{bmatrix}
\frac{5}{2} & -3 \\
-\frac{3}{2} & 2
\end{bmatrix}
\end{align*}
Show that $A$ and $B$ are inverse to each other.
\end{exmp}
\begin{solution}
\begin{align*}
AB &= 
\begin{bmatrix}
4 & 6 \\
3 & 5
\end{bmatrix}
\begin{bmatrix}
\frac{5}{2} & -3 \\
-\frac{3}{2} & 2
\end{bmatrix} \\
&= 
\begin{bmatrix}
(4)(\frac{5}{2})+(6)(-\frac{3}{2}) & (4)(-3)+(6)(2) \\
(3)(\frac{5}{2})+(5)(-\frac{3}{2}) & (3)(-3)+(5)(2)
\end{bmatrix} \\
&= 
\begin{bmatrix}
1 & 0 \\
0 & 1
\end{bmatrix} = I_2
\end{align*}
We leave it to the readers for showing $BA = I$ too as an exercise. Hence, $AB = BA = I$, $A$ and $B$ are indeed the inverse of each other.
\end{solution}

The followings are some properties of inverses.
\begin{proper}
\label{proper:inverse}
If a square matrix $A$ is invertible and has an inverse $A^{-1}$, then
\begin{enumerate}
\item $(cA)^{-1} = \frac{1}{c}A^{-1}$, for any constant $c \neq 0$;
\item $(A^{-1})^{-1} = A$, i.e.\ the inverse of an inverse returns the original matrix;
\item $(A^n)^{-1} = (A^{-1})^n$, for any positive integer $n$;
\item $(AB)^{-1} = B^{-1}A^{-1}$, provided that $B$ is invertible too (and they are square matrices of the same size);
\item $(A^T)^{-1} = (A^{-1})^T$.
\end{enumerate}
\end{proper}
However, $(A\pm B)^{-1}$ may not be equal to $A^{-1} \pm B^{-1}$, or even may be singular. We shall briefly prove (4) here.
\begin{proof}
It is given that $A$ and $B$ is invertible, and by Definition \ref{defn:inverse}, we have $AA^{-1} = I$, as well as 
\begin{align*}
BB^{-1} = I    
\end{align*}
Multiplying by $A$ and $A^{-1}$ to the left and right on both sides of above respectively yields
\begin{align*}
ABB^{-1}A^{-1} &= AIA^{-1} \\
AB(B^{-1}A^{-1}) &= (AI)A^{-1} = AA^{-1} & \text{(Properties \ref{proper:matmul} and \ref{proper:identity})} \\
&= I & \text{(Definition \ref{defn:inverse})}
\end{align*}
This shows that multiplying $AB$ by $B^{-1}A^{-1}$ produces an identity matrix, and therefore $(AB)^{-1} = B^{-1}A^{-1}$ is the unique inverse of $AB$ by Definition \ref{defn:inverse} and Properties \ref{proper:uniqueinverse}.
\end{proof}
Short Exercise: Show that $(ABC)^{-1} = C^{-1}B^{-1}A^{-1}$ if $A$, $B$ and $C$ are invertible and conformable.\footnote{By (4), $(ABC)^{-1} = ((AB)(C))^{-1} = C^{-1}(AB)^{-1} = C^{-1}B^{-1}A^{-1}$}\par
(4) of Properties \ref{proper:inverse} explicitly shows that the product $AB$ is invertible if $A$ and $B$ are themselves invertible. The converse is actually true as well.\footnote{Let's assume $AB$ is invertible and has an inverse $C = (AB)^{-1}$, hence we have $(AB)C = I$ by Definition \ref{defn:inverse}, (notice that $A$, $B$, and $C$ are all square matrices of the same extent) and by Properties \ref{proper:matmul}, $A(BC) = I$. Using Definition \ref{defn:inverse} (as well as Properties \ref{proper:uniqueinverse}) again, we immediately identify $BC$ as the inverse of $A$ and $A$ is invertible. The case for $B$ is similarly proved.} Hence
\begin{proper}
\label{proper:ABinv}
For two square matrices $A$ and $B$, $AB$ is invertible if and only if $A$ and $B$ are invertible.
\end{proper}

\subsection{(Reduced) Row Echelon Form}
\label{section:echelon}
Naturally, the next question is how to compute the inverse of any square matrix. For this, we have to understand a specific form of matrices called the \index{Row Echelon Form}\index{Row Echelon Form!Reduced Row Echelon Form}\keywordhl{(reduced) row echelon form} first. A matrix is in reduced row echelon form (\textit{rref}) when it satisfies the following requirements.
\begin{defn}[(Reduced) Row Echelon Form]
\label{defn:rref}
A matrix is in row echelon form if
\begin{enumerate}
\item The first non-zero number in every row is $1$, which is known as the \textit{"Leading 1"} (sometimes referred to as a \textit{pivot}),
\item \textit{"Leading 1"} of a lower row must appear farther to the right than that of any higher row,
\item Any row consisted of all zeros is placed at the bottom;
\item If additionally, any column containing a leading $1$ (sometimes called a \textit{pivotal column}) have zeros elsewhere in that column, then it is in \textit{reduced} row echelon form.
\end{enumerate}
\end{defn}
It is apparent that all identity matrices are in (reduced) row echelon form. Examples of row echelon form (but not \textit{reduced}), with the leading $1$s highlighted are
\begin{align*}
A &=
\begin{bmatrix}
\textcolor{red}{1} & 2 & 0 \\
0 & \textcolor{red}{1} & 1 \\
0 & 0 & \textcolor{red}{1}
\end{bmatrix}
& B &=
\begin{bmatrix}
\textcolor{red}{1} & 3 & 1 & 2 \\
0 & 0 & \textcolor{red}{1} & 5 \\
0 & 0 & 0 & \textcolor{red}{1}
\end{bmatrix} \\
C &=
\begin{bmatrix}
\textcolor{red}{1} & 4 \\
0 & \textcolor{red}{1} \\
0 & 0 
\end{bmatrix}
& D &=
\begin{bmatrix}
0 & \textcolor{red}{1} & 0 & 2 \\
0 & 0 & 0 & \textcolor{red}{1} \\
0 & 0 & 0 & 0
\end{bmatrix}
\end{align*}
Meanwhile, examples of \textit{reduced} row echelon form are
\begin{align*}
& G =
\begin{bmatrix}
\textcolor{red}{1} & 0 & 0 \\
0 & 0 & \textcolor{red}{1}\\
0 & 0 & 0 
\end{bmatrix}
& H =
\begin{bmatrix}
\textcolor{red}{1} & 0 & 2 & 0 \\
0 & \textcolor{red}{1} & 1 & 0 \\
0 & 0 & 0 & \textcolor{red}{1}
\end{bmatrix}
\end{align*}
The following matrices are \textit{not} in row echelon form. (why?)\footnote{$P$ violates (2) and $Q$ does not satisfy (1) and (3) of Definition \ref{defn:rref}.}
\begin{align*}
& P =
\begin{bmatrix}
1 & 0 & 0 \\
0 & 0 & 1 \\
0 & 1 & 0 \\
\end{bmatrix}
& Q =
\begin{bmatrix}
1 & 0 & 0 \\
0 & 0 & 0 \\
0 & 3 & 1
\end{bmatrix}
\end{align*}
Short Exercise: Decide if the following matrices are in (reduced) row echelon form or not.\footnote{Yes, Yes (reduced), No.}
\begin{align*}
X &= 
\begin{bmatrix}
1 & 0 & 0 & 1 \\
0 & 1 & 2 & 0 \\
0 & 0 & 0 & 1
\end{bmatrix}
& Y&=
\begin{bmatrix}
1 & 0 & 0 & 0 & 0 \\
0 & 0 & 0 & 1 & 0 \\
0 & 0 & 0 & 0 & 1 \\
0 & 0 & 0 & 0 & 0
\end{bmatrix}
& Z&=
\begin{bmatrix}
1 & 0 & 0 \\
0 & 0 & 1 \\
0 & 0 & 1 
\end{bmatrix}
\end{align*}
We have studied about elementary row operations in the last chapter, which now can be used to transform matrices into their reduced row echelon form. The procedure is comprised of two major parts, the \textit{forward phase}, converting the matrix to row echelon form first, and the \textit{backward phase}, eventually transforming it into reduced row echelon form. The first phase is also named \index{Gaussian Elimination}\keywordhl{Gaussian Elimination}, and together they are called \index{Gaussian Elimination!Gauss-Jordan Elimination}\keywordhl{Gauss-Jordan Elimination}\footnote{Often we just write Gaussian Elimination in place of Gauss-Jordan Elimination.}. We demonstrate the entire procedure using an example.
\begin{exmp}
Carry out Gauss-Jordan Elimination on the following matrix to make it become reduced row echelon form.
\begin{align*}
A =
\begin{bmatrix}
2 & 0 & 4 & 6 \\
3 & 3 & 1 & 0 \\
1 & 2 & 3 & 4
\end{bmatrix}    
\end{align*}
\end{exmp}
\begin{solution}
At each step of the forward phase, the strategy is to look at on the leftmost column that has at least one non-zero entries (any column consisting of full zeros is ignored). Along that column, we either find an existing leading $1$, or create a leading $1$ via multiplying some row having a starting entry $a$ that is as large as possible in magnitude, by the constant $1/a$. (The leading entry selected by this algorithm is commonly called the \index{Pivot}\keywordhl{pivot}, and the process is called \index{Pivot!Pivoting}\keywordhl{pivoting}.) The row holding the leading $1$ is subsequently put at the top, by an interchanging of rows if needed. In this example, such rows will be highlighted in red.
\begin{align*}
\left[\begin{array}{@{\,}wc{12pt}wc{12pt}wc{12pt}wc{12pt}@{\,}}
2 & 1 & 4 & 6 \\
3 & 3 & 1 & 0 \\
1 & 2 & 3 & 4
\end{array}\right]
&\to
\begin{bmatrix}
\textcolor{red}{1} & \textcolor{red}{\frac{1}{2}} & \textcolor{red}{2} & \textcolor{red}{3} \\
3 & 3 & 1 & 0 \\
1 & 2 & 3 & 4
\end{bmatrix}
& \frac{1}{2}R_1 \to R_1
\end{align*}
We have picked the first row $R_1$ for the leading 1 through multiplying it by a factor of $\frac{1}{2}$ here, but a leading 1 can be obtained from the other two rows as well. Subsequently, we make all the elements below the leading $1$ along that \textit{pivotal column} become zero, by adding the top row (which holds the leading 1), times $-a_i$ (where $a_i$ is the corresponding leading entry of row $i$) to the other rows. Those zeros produced in this way will be highlighted in blue.
\begin{align*}
\begin{bmatrix}
\textcolor{red}{1} & \textcolor{red}{\frac{1}{2}} & \textcolor{red}{2} & \textcolor{red}{3} \\
3 & 3 & 1 & 0 \\
1 & 2 & 3 & 4
\end{bmatrix}
&\to
\begin{bmatrix}
\textcolor{red}{1} & \textcolor{red}{\frac{1}{2}} & \textcolor{red}{2} & \textcolor{red}{3} \\
\textcolor{blue}{0} & \frac{3}{2} & -5 & -9 \\
1 & 2 & 3 & 4
\end{bmatrix}
& R_2-3R_1 \to R_2 \\
&\to
\begin{bmatrix}
\textcolor{red}{1} & \textcolor{red}{\frac{1}{2}} & \textcolor{red}{2} & \textcolor{red}{3} \\
\textcolor{blue}{0} & \frac{3}{2} & -5 & -9 \\
\textcolor{blue}{0} & \frac{3}{2} & 1 & 1
\end{bmatrix}
& R_3-R_1 \to R_3
\end{align*}
The first iteration is finished. We now repeat the same process over the remaining submatrix made up of elements that are not yet highlighted in colour, from left to right recursively.
\begin{align*}
\begin{bmatrix}
\textcolor{red}{1} & \textcolor{red}{\frac{1}{2}} & \textcolor{red}{2} & \textcolor{red}{3} \\
\textcolor{blue}{0} & \frac{3}{2} & -5 & -9 \\
\textcolor{blue}{0} & \frac{3}{2} & 1 & 1
\end{bmatrix}
&\to
\begin{bmatrix}
\textcolor{red}{1} & \textcolor{red}{\frac{1}{2}} & \textcolor{red}{2} & \textcolor{red}{3} \\
\textcolor{blue}{0} & \textcolor{red}{1} & \textcolor{red}{-\frac{10}{3}} & \textcolor{red}{-6} \\
\textcolor{blue}{0} & \frac{3}{2} & 1 & 1
\end{bmatrix}
& \frac{2}{3}R_2 \to R_2 \\
&\to
\begin{bmatrix}
\textcolor{red}{1} & \textcolor{red}{\frac{1}{2}} & \textcolor{red}{2} & \textcolor{red}{3} \\
\textcolor{blue}{0} & \textcolor{red}{1} & \textcolor{red}{-\frac{10}{3}} & \textcolor{red}{-6} \\
\textcolor{blue}{0} & \textcolor{blue}{0} & 6 & 10
\end{bmatrix}
& R_3-\frac{3}{2}R_2 \to R_3 \\
&\to
\begin{bmatrix}
\textcolor{red}{1} & \textcolor{red}{\frac{1}{2}} & \textcolor{red}{2} & \textcolor{red}{3} \\
\textcolor{blue}{0} & \textcolor{red}{1} & \textcolor{red}{-\frac{10}{3}} & \textcolor{red}{-6} \\
\textcolor{blue}{0} & \textcolor{blue}{0} & \textcolor{red}{1} & \textcolor{red}{\frac{5}{3}}
\end{bmatrix}
& \frac{1}{6}R_3 \to R_3 
\end{align*}
Now, all entries below every leading $1$ are zeros, and the forward phase is completed. We have obtained the row echelon form as an intermediate product. The backward phase is done similarly but in a bottom-up fashion, from right to left. By adding appropriate multiples of lower rows to higher rows, we turn all the non-zero elements above the leading $1$ along every pivotal column into zeros. Non-pivotal columns (the last column here) are ignored.
\begin{align*}
\begin{bmatrix}
1 & \frac{1}{2} & 2 & 3 \\
0 & 1 & -\frac{10}{3} & -6 \\
0 & 0 & 1 & \frac{5}{3}
\end{bmatrix} 
&\to
\begin{bmatrix}
1 & \frac{1}{2} & 2 & 3 \\
0 & 1 & 0 & -\frac{4}{9} \\
0 & 0 & 1 & \frac{5}{3}
\end{bmatrix} 
& R_2 + \frac{10}{3}R_3 \to R_2 \\
&\to
\begin{bmatrix}
1 & \frac{1}{2} & 0 & -\frac{1}{3} \\
0 & 1 & 0 & -\frac{4}{9} \\
0 & 0 & 1 & \frac{5}{3}
\end{bmatrix} 
& R_2 - 2R_3 \to R_1 \\
&\to
\begin{bmatrix}
1 & 0 & 0 & -\frac{1}{9} \\
0 & 1 & 0 & -\frac{4}{9} \\
0 & 0 & 1 & \frac{5}{3}
\end{bmatrix} 
& R_1 - \frac{1}{2}R_2 \to R_1
\end{align*}
The matrix is now in reduced row echelon form as required. The amount of leading 1s in the rref of the matrix is known as its \textit{rank}, which equals to $3$ here.
\end{solution}
Short Exercise: Repeat the example above but start by interchanging $R_1$ and $R_3$.\footnote{For checking, after the first iteration, it will be
(WIP)
and the end result will be the same.}\\
\\
From the short exercise above, we can see that even if we apply different elementary row operations (particularly for the creation of leading 1s) during Gauss-Jordan Elimination, we will acquire the same reduced echelon form in the end. In fact,
\begin{thm}[Uniqueness of Reduced Row Echelon Form]
\label{thm:uniquerref}
Reduced row echelon form of a matrix is unique.
\end{thm}
We shall omit the proof here. The following properties further reveal how elementary row operations are associated with reduced row echelon form.
\begin{proper}
\label{proper:rowequiv}
If a matrix can be transformed into another matrix by elementary row operations, they are said to be \index{Row Equivalent}\keywordhl{row equivalent}.
\end{proper}
Since for any pair of row equivalent matrices, either of them can be transformed into the other one by elementary row operations, and hence can be further transformed into the reduced row echelon form of the other matrix, by Theorem \ref{thm:uniquerref}, the uniqueness of rref implies that
\begin{proper}
\label{proper:rowequivreduce}
Row equivalent matrices have the same reduced row echelon form. Particularly, they are row equivalent to this rref. If two matrices have different reduced row echelon forms, then they are not row equivalent, and vice versa.
\end{proper}
Let's go through one more simple example about Gauss-Jordan Elimination.
\begin{exmp}
\label{exmp:rref2}
Transform the following matrix into reduced row echelon form.
\begin{align*}
A =
\begin{bmatrix}
2 & 2 & 1 \\
6 & 4 & 1 \\
2 & 3 & 2 \\
2 & 1 & 0
\end{bmatrix}    
\end{align*}
\end{exmp}
\begin{solution}
One possible way to do the forward elimination is
\begin{align*}
\left[\begin{array}{@{\,}wc{10pt}wc{10pt}wc{10pt}@{\,}}
2 & 2 & 1 \\
6 & 4 & 1 \\
2 & 3 & 2 \\
2 & 1 & 0
\end{array}\right]
&\to
\left[\begin{array}{@{\,}wc{10pt}wc{10pt}wc{10pt}@{\,}}
6 & 4 & 1 \\
2 & 2 & 1 \\
2 & 3 & 2 \\
2 & 1 & 0
\end{array}\right]
& R_1 \leftrightarrow R_2 \\
&\to
\left[\begin{array}{@{\,}wc{10pt}wc{10pt}wc{10pt}@{\,}}
1 & \frac{2}{3} & \frac{1}{6} \\
2 & 2 & 1 \\
2 & 3 & 2 \\
2 & 1 & 0
\end{array}\right]
& \frac{1}{6}R_1 \to R_1 \\
&\to
\left[\begin{array}{@{\,}wc{10pt}wc{10pt}wc{10pt}@{\,}}
1 & \frac{2}{3} & \frac{1}{6} \\
0 & \frac{2}{3} & \frac{2}{3} \\
0 & \frac{5}{3} & \frac{5}{3} \\
0 & -\frac{1}{3} & -\frac{1}{3}
\end{array}\right]
& 
\begin{aligned}
R_2 - 2R_1 &\to R_2 \\
R_3 - 2R_1 &\to R_3 \\
R_4 - 2R_1 &\to R_4 
\end{aligned}\\
&\to
\left[\begin{array}{@{\,}wc{10pt}wc{10pt}wc{10pt}@{\,}}
1 & \frac{2}{3} & \frac{1}{6} \\
0 & 1 & 1 \\
0 & \frac{5}{3} & \frac{5}{3} \\
0 & -\frac{1}{3} & -\frac{1}{3}
\end{array}\right]
& \frac{3}{2}R_2 \to R_2 \\
&\to
\left[\begin{array}{@{\,}wc{10pt}wc{10pt}wc{10pt}@{\,}}
1 & \frac{2}{3} & \frac{1}{6} \\
0 & 1 & 1 \\
0 & 0 & 0 \\
0 & 0 & 0
\end{array}\right]
&
\begin{aligned}
R_3 - \frac{5}{3}R_2 &\to R_3\\
R_4 + \frac{1}{3}R_2 &\to R_4     
\end{aligned}
\end{align*}
The backward elimination is straight-forward.
\begin{align*}
\left[\begin{array}{@{\,}wc{10pt}wc{10pt}wc{10pt}@{\,}}
1 & \frac{2}{3} & \frac{1}{6} \\
0 & 1 & 1 \\
0 & 0 & 0 \\
0 & 0 & 0
\end{array}\right] 
&\to
\left[\begin{array}{@{\,}wc{10pt}wc{10pt}wc{10pt}@{\,}}
1 & 0 & -\frac{1}{2} \\
0 & 1 & 1 \\
0 & 0 & 0 \\
0 & 0 & 0
\end{array}\right]
& R_1 - \frac{2}{3}R_2 \to R_1
\end{align*}
The rank of the matrix can be readily seen to be $2$.
\end{solution}

\subsection{Finding Inverses by Gaussian Elimination}
\label{subsection:invGauss}
With Gaussian Elimination, obtaining the inverse $A^{-1}$ of any invertible matrix $A$ is now possible. We start by writing out an identity matrix $I$ of the same shape and concatenate this identity matrix to the right of $A$, leading to an augmented form of $[A|I]$. Then we carry out elementary row operations simultaneously on both sides of $[A|I]$ such that the matrix to the left, originally as $A$, is reduced to the identity matrix $I$ by Gaussian Elimination. The identity matrix to the right will then be transformed into the desired inverse by the same set of elementary operations, such that the concatenated matrix will appear as $[I|A^{-1}]$, 
\begin{exmp}
Find the inverse of
\begin{align*}
A =
\begin{bmatrix}
1 & 4 & 5 \\
0 & 2 & 3 \\
0 & 1 & 1
\end{bmatrix}
\end{align*}
by Gaussian Elimination
\end{exmp}
\begin{solution}
Appending an $3 \times 3$ identity matrix to the right, we have
\begin{align*}
\left[\begin{array}{@{}wc{10pt}wc{10pt}wc{10pt}|wc{10pt}wc{10pt}wc{10pt}@{\,}}
1 & 4 & 5 & 1 & 0 & 0 \\
0 & 2 & 3 & 0 & 1 & 0 \\
0 & 1 & 1 & 0 & 0 & 1
\end{array}\right] 
& \to
\left[\begin{array}{@{}wc{10pt}wc{10pt}wc{10pt}|wc{10pt}wc{10pt}wc{10pt}@{\,}}
1 & 4 & 5 & 1 & 0 & 0 \\
0 & 0 & 1 & 0 & 1 & -2 \\
0 & 1 & 1 & 0 & 0 & 1
\end{array}\right] & R_2 - 2R_3 \to R_2 \\
& \to
\left[\begin{array}{@{}wc{10pt}wc{10pt}wc{10pt}|wc{10pt}wc{10pt}wc{10pt}@{\,}}
1 & 4 & 5 & 1 & 0 & 0 \\
0 & 1 & 1 & 0 & 0 & 1 \\
0 & 0 & 1 & 0 & 1 & -2 
\end{array}\right] & R_2 \leftrightarrow R_3 \\
& \to
\left[\begin{array}{@{}wc{10pt}wc{10pt}wc{10pt}|wc{10pt}wc{10pt}wc{10pt}@{\,}}
1 & 4 & 5 & 1 & 0 & 0 \\
0 & 1 & 0 & 0 & -1 & 3 \\
0 & 0 & 1 & 0 & 1 & -2 
\end{array}\right] & R_2 - R_3 \to R_2 \\
& \to
\left[\begin{array}{@{}wc{10pt}wc{10pt}wc{10pt}|wc{10pt}wc{10pt}wc{10pt}@{\,}}
1 & 4 & 0 & 1 & -5 & 10 \\
0 & 1 & 0 & 0 & -1 & 3 \\
0 & 0 & 1 & 0 & 1 & -2 
\end{array}\right] & R_1 - 5R_3 \to R_1 \\
& \to
\left[\begin{array}{@{}wc{10pt}wc{10pt}wc{10pt}|wc{10pt}wc{10pt}wc{10pt}@{\,}}
1 & 0 & 0 & 1 & -1 & -2 \\
0 & 1 & 0 & 0 & -1 & 3 \\
0 & 0 & 1 & 0 & 1 & -2 
\end{array}\right] & R_1 - 4R_2 \to R_1 
\end{align*}
Hence the required inverse is
\begin{align*}
A^{-1} =
\begin{bmatrix}
1 & -1 & -2 \\
0 & -1 & 3 \\
0 & 1 & -2 
\end{bmatrix}    
\end{align*}
\end{solution}
Short Exercise: Verify the inverse of $A^{-1}$ above is just $A$ by the same method. \footnote{You should be able to retrieve the matrix $A$ back. The first column of $A^{-1}$ already holds a leading 1 and elements below which are zeros. A possible next step is to multiply $R_2$ by $-1$ and then subtract $R_3$ by $R_2$.}\par
The underlying reason why the above procedure can produce the inverse matrix is the equivalence between elementary row operations and multiplication by appropriate \index{Elementary Matrix}\keywordhl{elementary matrices}.
\begin{proper}[Elementary Matrices]
\label{proper:elementarymat}
Any elementary row operation on an $m \times n$ matrix can be represented by multiplying it to the left with a suitable \textit{elementary matrix}. Such a matrix is essentially the one appeared after applying that particular elementary row operation on an identity matrix. For the three types of elementary row operations described in Definition \ref{defn:elerowop}:
\begin{enumerate}
\item $cR_{p} \to R_{p}$, $c \neq 0$,
\item $R_{p} + cR_{q} \to R_{p}$,
\item $R_{p} \leftrightarrow R_{q}$
\end{enumerate}
their corresponding elementary matrices $E$ are square ($m \times m$), and \textit{invertible} (see the following remark) in which
\begin{enumerate}
\item $E_{kk} = 1$ for any $k$, except $E_{pp} = c$;
\item $E_{kk} = 1$ for all $k$, with $E_{pq} = c$;
\item $E_{kk} = 1$ for any $k$, except $E_{pp} = 0$ and $E_{qq} = 0$, with $E_{pq} = E_{qp} = 1$. 
\end{enumerate}
Entries not mentioned are all zeros.
\end{proper}
Since it is quite abstract, it is useful to have some actual examples.
\begin{align*}
&
\begin{bmatrix}
1 & 0 & 0 \\
0 & 2 & 0 \\
0 & 0 & 1
\end{bmatrix} & \text{Multiplying $R_2$ by a factor of $2$: } 2R_2 \to R_2 \\
&
\begin{bmatrix}
1 & 3 & 0 \\
0 & 1 & 0 \\
0 & 0 & 1
\end{bmatrix} & \text{Adding 3 times $R_2$ to $R_1$: } R_1 + 3R_2 \to R_1 \\
&
\begin{bmatrix}
0 & 0 & 1 \\
0 & 1 & 0 \\
1 & 0 & 0
\end{bmatrix} & \text{Swapping $R_1$ and $R_3$: } R_1 \leftrightarrow R_3 
\end{align*}
Any elementary row operation can be apparently undone by an inverse elementary row operation (addition vs subtraction, multiplication vs division ($c \neq 0$), swapping twice). Accordingly, any elementary matrix has another corresponding elementary matrix as its inverse, and the readers are invited to think about their forms in the exercise below. \\
\\
Short Exercise: Write down the inverses of the three example elementary matrices above.\footnote{$
\begin{bmatrix}
1 & 0 & 0 \\
0 & \frac{1}{2} & 0 \\
0 & 0 & 1
\end{bmatrix},
\begin{bmatrix}
1 & -3 & 0 \\
0 & 1 & 0 \\
0 & 0 & 1
\end{bmatrix},
\begin{bmatrix}
0 & 0 & 1 \\
0 & 1 & 0 \\
1 & 0 & 0
\end{bmatrix}
$\\}\par
For instance, consider a matrix
\begin{align*}
\begin{bmatrix}
1 & 4 & 3 \\
2 & 5 & 1 \\
-1 & 0 & 2
\end{bmatrix}     
\end{align*}
then the action of subtracting $R_2$ from $R_3$, $R_3 - R_2 \to R_3$. can be expressed as
\begin{align*}
\begin{bmatrix}
1 & 0 & 0 \\
0 & 1 & 0 \\
0 & -1 & 1
\end{bmatrix} 
\begin{bmatrix}
1 & 4 & 3 \\
2 & 5 & 1 \\
-1 & 0 & 2
\end{bmatrix} 
&= 
\begin{bmatrix}
1 & 4 & 3 \\
2 & 5 & 1 \\
-3 & -5 & 1
\end{bmatrix} 
\end{align*}
Short Exercise: Find out the $3 \times 3$ elementary matrix for subtracting $2$ times the third row from the first row. What happens when we apply this elementary matrix to the left of the matrix above? \footnote{$
\begin{bmatrix}
1 & 0 & -2 \\
0 & 1 & 0 \\
0 & 0 & 1
\end{bmatrix}$:  
$\begin{bmatrix}
1 & 0 & -2 \\
0 & 1 & 0 \\
0 & 0 & 1
\end{bmatrix}
\begin{bmatrix}
1 & 4 & 3 \\
2 & 5 & 1 \\
-3 & -5 & 1
\end{bmatrix}
=
\begin{bmatrix}
7 & 14 & 1 \\
2 & 5 & 1 \\
-3 & -5 & 1
\end{bmatrix}
$}\par
Now we are ready to see why finding inverses by Gaussian Elimination works.
\begin{thm}
\label{thm:Gausselimprincip}
If a matrix $A$ can be converted to an identity matrix $I$ as its reduced row echelon form by Gaussian Elimination, then it is invertible since the same steps can in turn be applied on $I$, producing its inverse $A^{-1}$. 
\end{thm}
Using the language of Properties \ref{proper:rowequivreduce}, the matrix $A$ has to be row equivalent to $I$ for $A^{-1}$ to exist. This also means if Gaussian Elimination fails to reduce $A$ to $I$ (i.e.\ the reduced row echelon form of $A$ is some matrix other than the identity), then $A^{-1}$ does not exist.
\begin{proof}
Assume $A$ is invertible and hence $AA^{-1} = I$ (Definition \ref{defn:inverse}). From Properties \ref{proper:elementarymat}, When doing Gaussian Elimination over $A$, the $i$-th elementary row operation executed can be represented by an elementary matrix, denoted as $E_{i}$, for $i = 1,2,\ldots,n$ where $n$ is the total number of steps. If we multiply these $E_{i}$ successively to the left on both sides of the equation $AA^{-1} = I$, we have
\begin{align*}
E_n \cdots E_{3}E_{2}E_{1} AA^{-1} &= E_n \cdots E_{3}E_{2}E_{1}I \\
(E_n \cdots E_{3}E_{2}E_{1}A) A^{-1} &= E_n \cdots E_{3}E_{2}E_{1}I & \text{(Properties \ref{proper:matmul})} \\
(I)A^{-1} &= E_n \cdots E_{3}E_{2}E_{1}I \\
A^{-1} &= E_n \cdots E_{3}E_{2}E_{1}I & \text{(Properties \ref{proper:identity})}
\end{align*}
from the second line to the third line, we have $E_n \cdots E_{3}E_{2}E_{1}A = I$ because the elementary row operations during Gaussian Elimination, represented by $E_i$, $i = 1,2,\ldots,n$, reduce $A$ to $I$ as we demand in the assumption. With $A^{-1} = E_n \cdots E_{3}E_{2}E_{1}I$, we immediately see that the same set of elementary matrices and hence elementary row operations can also transform $I$ into $A^{-1}$, explicitly showing that $A$ is invertible.    
\end{proof}
As a corollary, because we have $E_n \cdots E_{3}E_{2}E_{1}A = I$ from above, and all $E_i$ are invertible by Properties \ref{proper:elementarymat}, we can multiply their inverses $E'_i = E_i^{-1}$ (which are also elementary matrices), to the left on both sides successively, where $i$ runs backwards from $n$ to $1$. This leads to
\begin{align*}
E_{1}^{-1}E_{2}^{-1}E_{3}^{-1}\cdots E_n^{-1}E_n \cdots E_{3}E_{2}E_{1}A &= E_{1}^{-1}E_{2}^{-1}E_{3}^{-1}\cdots E_n^{-1}I \\
A &= E'_{1}E'_{2}E'_{3}\cdots E'_n
\end{align*}
as each of the pairs $E_n^{-1}E_n$, $E_{n-1}^{-1}E_{n-1}$, $\ldots$, $E_2^{-1}E_2$, $E_1^{-1}E_1$ cancels out to produce $I$, and hence
\begin{proper}
\label{proper:invseqelement}
All invertible matrices can be written as a product of some sequence of elementary matrices. 
\end{proper}

\section{Determinants}
\label{section:det}
\subsection{Computing Determinants}
The \index{Determinant}\keywordhl{determinant} of a \textit{square} matrix $A$, denoted by det($A$) or $|A|$, is a number associated with certain intrinsic properties of the matrix which can help us to find its inverse (Determinant of non-square matrices is undefined). Determinant of a $1 \times 1$ matrix is equal to the matrix's only entry. Determinants of $2 \times 2$ and $3 \times 3$ matrices can be calculated by a trick called \index{Sarrus' Rule}\keywordhl{Sarrus' Rule}.
\subsubsection{Sarrus' Rule}
\begin{proper}[Sarrus' Rule]
\label{proper:sarrus}
Determinants of size $2 \times 2$ and $3 \times 3$ matrices can be found by the Sarrus' Rule. For a $2 \times 2$ matrix
\begin{align*}
A =
\begin{bmatrix}
a_1 & b_1 \\
a_2 & b_2
\end{bmatrix} 
\end{align*}
Its determinant is computed by
\begin{center}
\begin{tikzpicture}
% https://tex.stackexchange.com/questions/32978/typesetting-a-matrix-with-crossing-arrows-on-it
\matrix[matrix of math nodes, inner sep=5pt, outer sep=-5pt] (sarrus) 
{a_{11} & a_{12} \\
a_{21} & a_{22} \\ };
\path
($(sarrus-1-1.north west)-(0.5em,0)$) edge[thick] ($(sarrus-2-1.south west)-(0.5em,0)$)
($(sarrus-1-2.north east)+(0.5em,0)$) edge[thick] ($(sarrus-2-2.south east)+(0.5em,0)$)
(sarrus-1-1.center)                          edge[red, line width=1, ->]           (sarrus-2-2.south east)
(sarrus-2-1.center)                          edge[blue, line width=1, ->, dashed]  (sarrus-1-2.north east);
\end{tikzpicture}
\end{center}
\vspace{-25pt}
\begin{align*}
|A| &= \textcolor{red}{a_{11}a_{22}} - \textcolor{blue}{a_{21}a_{12}}
\end{align*}
which is the product of elements crossed by the red arrow, minus the blue one. Similarly, for a $3 \times 3$ matrix
\begin{align*}
A =
\begin{bmatrix}
a_{11} & a_{12} & a_{13} \\
a_{21} & a_{22} & a_{23} \\
a_{31} & a_{32} & a_{33}
\end{bmatrix} 
\end{align*}
Its determinant can be found by
\begin{center}
\begin{tikzpicture}
\matrix[matrix of math nodes, inner sep=5pt, outer sep=-5pt] (sarrus) 
{a_{11} & a_{12} & a_{13} & \textcolor{gray}{a_{11}} & \textcolor{gray}{a_{12}} \\
a_{21} & a_{22} & a_{23} & \textcolor{gray}{a_{21}} & \textcolor{gray}{a_{22}} \\
a_{31} & a_{32} & a_{33} & \textcolor{gray}{a_{31}} & \textcolor{gray}{a_{32}} \\ };
\path
($(sarrus-1-1.north west)-(0.5em,0)$) edge[thick] ($(sarrus-3-1.south west)-(0.5em,0)$)
($(sarrus-1-3.north east)+(0.5em,0)$) edge[thick] ($(sarrus-3-3.south east)+(0.5em,0)$)
(sarrus-1-1.center)                          edge[red, line width=1, ->]           (sarrus-3-3.south east)
(sarrus-1-2.center)                          edge[red, line width=1, ->]           (sarrus-3-4.south east)
(sarrus-1-3.center)                          edge[red, line width=1, ->]           (sarrus-3-5.south east)
(sarrus-3-1.center)                          edge[blue, line width=1, ->, dashed]  (sarrus-1-3.north east)
(sarrus-3-2.center)                          edge[blue, line width=1, ->, dashed]  (sarrus-1-4.north east)
(sarrus-3-3.center)                          edge[blue, line width=1, ->, dashed]  (sarrus-1-5.north east);
\end{tikzpicture}
\end{center}
\vspace{-25pt}
\begin{align*}
|A| ={}& (\textcolor{red}{a_{11}a_{22}a_{33}} + \textcolor{red}{a_{12}a_{23}a_{31}} + \textcolor{red}{a_{13}a_{21}a_{32}}) \\
&- (\textcolor{blue}{a_{31}a_{22}a_{13}} + \textcolor{blue}{a_{32}a_{23}a_{11}} + \textcolor{blue}{a_{33}a_{21}a_{12}})
\end{align*}
\end{proper}

\begin{exmp}
Find the determinant of the following matrix.
\begin{align*}
A &=
\begin{bmatrix}
1 & 2 & 4 \\
-5 & 0 & -3 \\
4 & 3 & 1
\end{bmatrix} 
\end{align*}
\end{exmp}
\begin{solution}
By Sarrus's Rule (Properties \ref{proper:sarrus}), we have
\begin{align*}
|A| &=
\begin{vmatrix}
1 & 2 & 4 \\
-5 & 0 & -3 \\
4 & 3 & 1
\end{vmatrix} \\
&= ((1)(0)(1) + (2)(-3)(4) + (4)(-5)(3)) \\
&\quad- ((4)(0)(4) + (3)(-3)(1) + (1)(-5)(2)) \\
&= (0 - 24 - 60) - (0 - 9 - 10) \\
&= -65
\end{align*}
\end{solution}

\subsubsection{Cofactor Expansion}
Another commonly used method to calculate determinants is \textit{Cofactor Expansion}, also known as \textit{Laplace Expansion}. Before discussing cofactor expansion, it is necessary to know what \textit{cofactors} are.
\begin{defn}[Cofactor and Minor]
\label{defn:cofactor}
The \index{Cofactor}\keywordhl{cofactor} $C_{ij}$ at the $(i, j)$ position of a matrix $A$ is simply the determinant of the submatrix formed by deleting the $i$-th row and $j$-th column of $A$: $M_{ij}$ (called the \index{Minor}\keywordhl{minor} at $(i, j)$), times the factor of $(-1)^{i+j}$, that is, $C_{ij} = (-1)^{i+j} M_{ij}$.
\end{defn}
The $(-1)^{i+j}$ factor can be visualized as a checkerboard pattern like
\begin{align*}
\begin{bmatrix}
+ & - & + & \cdots \\
- & + & - &  \\
+ & - & + &  \\
\vdots & &  & \ddots
\end{bmatrix}
\end{align*}
So, for a matrix like
\begin{center}
\begin{tikzpicture}
\matrix[matrix of math nodes, inner sep=3.5pt, outer sep=-4pt, left delimiter={[}, right delimiter={]}] (cofactor) 
{1 & \textcolor{blue}{3} & \textcolor{blue}{5} \\
2 & 4 & 6 \\
3 & \textcolor{blue}{5} & \textcolor{blue}{7} \\ };
\path
(cofactor-1-1.north) edge[red, thick] (cofactor-3-1.south)
(cofactor-2-1.west) edge[red, thick] (cofactor-2-3.east);
\end{tikzpicture}
\end{center}
Its cofactor at $(2, 1)$ is
\begin{align*}
C_{21} &= (-1)^{(2+1)}
\begin{vmatrix}
\textcolor{blue}{3} & \textcolor{blue}{5} \\
\textcolor{blue}{5} & \textcolor{blue}{7}
\end{vmatrix} & \text{(Definition \ref{defn:cofactor})} \\
&= (-1)((3)(7) - (5)(5)) & \text{(Properties \ref{proper:sarrus})} \\
&= 4
\end{align*}
Short Exercise: Find $C_{13}$ and $C_{32}$ for the matrix above.\footnote{$C_{13} = (-1)^{1+3}\begin{vmatrix}
2 & 4 \\
3 & 5
\end{vmatrix} = (1)((2)(5)-(3)(4)) = -2$, similarly $C_{32} = 4$.}\\
\\
With \index{Cofactor Expansion}\index{Laplace Expansion}\keywordhl{Cofactor (Laplace) Expansion}, the determinant of a matrix is computed as the sum of products between each entry and the corresponding cofactor along a picked row/column of it.
\begin{proper}[Cofactor/Laplace Expansion]
\label{proper:cofactorex}
The determinant of a $n \times n$ square matrix $A$, $|A|$, can be found by selecting either a fixed row $i$, or column $j$, and adding up the products of every element-cofactor pair along that row/column. For the former case (selected the $i$-th row), the determinant is computed as
\begin{align*}
|A| &= A_{i1}C_{i1} + A_{i2}C_{i2} + \cdots + A_{in}C_{in} \\
&= \sum_{k=1}^{n} A_{ik}C_{ik}
\end{align*}
For the latter case (fixed the $j$-th column), the determinant is similarly found by
\begin{align*}
|A| &= A_{1j}C_{1j} + A_{2j}C_{2j} + \cdots + A_{nj}C_{nj} \\
&= \sum_{k=1}^{n} A_{kj}C_{kj}
\end{align*}
where each of the cofactors $C_{ij}$ is defined as in Definition \ref{defn:cofactor}. \textcolor{red}{Important: regardless of which row or column is chosen, the result is always the same.}
\end{proper}
\begin{exmp}
Again, for the matrix
\begin{align*}
A =
\begin{bmatrix}
\textcolor{red}{1} & \textcolor{red}{3} & \textcolor{red}{5} \\
2 & 4 & 6 \\
3 & 5 & 7 
\end{bmatrix}   
\end{align*}
Find its determinant via cofactor expansion.
\end{exmp}
\begin{solution}
According to Properties \ref{proper:cofactorex}, if we choose the first row to be expanded, its determinant is
\begin{align*}
|A| &= A_{11}C_{11} + A_{12}C_{12} + A_{13}C_{13} \\
&= 
(\textcolor{red}{1})((-1)^{1+1}
\begin{vmatrix}
4 & 6 \\
5 & 7
\end{vmatrix})
+
(\textcolor{red}{3})((-1)^{1+2}
\begin{vmatrix}
2 & 6 \\
3 & 7
\end{vmatrix}) \\
&\quad+ 
(\textcolor{red}{5})((-1)^{1+3}
\begin{vmatrix}
2 & 4 \\
3 & 5
\end{vmatrix}) & \text{(Definition \ref{defn:cofactor})}\\
&= (1)(-2) + (3)(4) + (5)(-2) = 0 & \text{(Properties \ref{proper:sarrus})} 
\end{align*}
\end{solution}
Short Exercise: Confirm the answer by carrying out cofactor expansion on another row or column.\footnote{You should able to get $|A| = 0$, no matter which row/column is selected.}
\begin{exmp}
\label{exmp:4x4det}
Find the determinant of
\begin{align*}
A = 
\begin{bmatrix}
1 & 4 & 4 & 4 \\
2 & 0 & 4 & 6 \\
2 & 1 & 1 & 0 \\
6 & 2 & 3 & 1
\end{bmatrix}
\end{align*}\
\end{exmp}
\begin{solution}
It is a $4 \times 4$ matrix and we have to apply cofactor expansion. We can choose row or column that contains some zero(s) to reduce the computation. Here we pick the second column and by Properties \ref{proper:cofactorex}, we have
\begin{align*}
|A| &= (4)(-1)^{1+2}
\begin{vmatrix}
2 & 4 & 6 \\
2 & 1 & 0 \\
6 & 3 & 1 \\
\end{vmatrix}
+ (0)(-1)^{2+2}
\begin{vmatrix}
1 & 4 & 4 \\
2 & 1 & 0 \\
6 & 3 & 1 \\
\end{vmatrix} \\
&\quad + (1)(-1)^{3+2}
\begin{vmatrix}
1 & 4 & 4 \\
2 & 4 & 6 \\
6 & 3 & 1 \\
\end{vmatrix} 
+ (2)(-1)^{4+2}
\begin{vmatrix}
1 & 4 & 4 \\
2 & 4 & 6 \\
2 & 1 & 0
\end{vmatrix} 
\end{align*}
By Sarrus' Rule (Properties \ref{proper:sarrus}), we can calculate each of the four $3 \times 3$ determinants (the detailed calculations are omitted, notice that we don't need to actually compute the second determinant) and obtain
\begin{align*}
|A| = (-4)(-6) + 0 + (-1)(50) + (2)(18) = 10
\end{align*}
\end{solution}
Finally, we can derive two simple results about determinants from the perspective of cofactor expansion.
\begin{proper}
\label{proper:zerodet}
If a matrix have a row/column with full zeros, or two identical/proportional rows/columns, then it has a determinant of zero.
\end{proper}
The first case is trivial (just do the expansion along the row/column with full zeros) and we will show the second case alongside the introduction of the properties of determinants in the upcoming subsection (??).

\subsection{Properties of Determinants} There are some notable properties about determinants. First of all, it is very easy to see that determinants for any $n \times n$ identity matrix $I_n$ is just $1$. Second, there is a close relation between elementary row operations/elementary matrices and (their effects on) determinants, noted as follows.
\begin{proper}
\label{proper:elementaryopdet}
The three types of elementary row operations in Definition \ref{defn:elerowop}, when applied on some square matrix $A$,
\begin{enumerate}
\item $cR_{p} \to R_{p}$, $c \neq 0$,
\item $R_{p} + cR_{q} \to R_{p}$,
\item $R_{p} \leftrightarrow R_{q}$,
\end{enumerate}
change the determinant of $A$ by a factor of $c$, $1$ (unchanged), and $-1$ (switching the sign), respectively.
\end{proper}
\begin{proper}
\label{proper:elementarymatdet}
The three types of elementary matrices $E$ in Properties \ref{proper:elementarymat} that correspond to the elementary row operations in Definition \ref{defn:elerowop},
\begin{enumerate}
\item $E_{kk} = 1$ for any $k$, except $E_{pp} = c$ ($cR_{p} \to R_{p}$, $c \neq 0$),
\item $E_{kk} = 1$ for all $k$, with $E_{pq} = c$ ($R_{p} + cR_{q} \to R_{p}$),
\item $E_{kk} = 1$ for any $k$, except $E_{pp} = 0$ and $E_{qq} = 0$, with $E_{pq} = E_{qp} = 1$ ($R_{p} \leftrightarrow R_{q}$),
\end{enumerate}
have a determinant of $c$, $1$, and $-1$, respectively.
\end{proper}
Since the determinants of elementary matrices, by Properties \ref{proper:elementarymatdet}, coincide exactly with the factors by how the determinant of a square matrix $A$ changes when the corresponding elementary row operations are applied on $A$ (represented by multiplication to the left of $A$ by these elementary matrices) as shown in Properties \ref{proper:elementaryopdet}, we conclude that
\begin{thm}
\label{thm:elementarytimesdet}
For any elementary matrix $E$ and a square matrix $A$, we have
\begin{align*}
\det(EA) = \det(E)\det(A)
\end{align*}
\end{thm}
This theorem will be of use when we later prove other properties of determinant. However, before doing so, we will demonstrate how to utilize Properties \ref{proper:elementaryopdet} (or equivalently \ref{proper:elementarymatdet}) to ease the calculation of determinants.
\begin{exmp}
Re-do Example \ref{exmp:4x4det} utilizing Properties \ref{proper:elementaryopdet}.
\end{exmp}
\begin{solution}
We can factor out the $2$ in second row and subtract $3$ times the third row from the fourth row. By Properties \ref{proper:elementaryopdet}, we have
\begin{align*}
|A| = 
\begin{vmatrix}
1 & 4 & 4 & 4 \\
2 & 0 & 4 & 6 \\
2 & 1 & 1 & 0 \\
6 & 2 & 3 & 1
\end{vmatrix}
&=
2
\begin{vmatrix}
1 & 4 & 4 & 4 \\
1 & 0 & 2 & 3 \\
2 & 1 & 1 & 0 \\
6 & 2 & 3 & 1
\end{vmatrix} \\
&= 
2
\begin{vmatrix}
1 & 4 & 4 & 4 \\
1 & 0 & 2 & 3 \\
2 & 1 & 1 & 0 \\
0 & -1 & 0 & 1
\end{vmatrix} 
\end{align*}
The determinant in the last line can be computed by doing cofactor expansion along the fourth row which now contains two zeros. With Properties \ref{proper:cofactorex} and \ref{proper:sarrus}, it is
\begin{align*}
\begin{vmatrix}
1 & 4 & 4 & 4 \\
1 & 0 & 2 & 3 \\
2 & 1 & 1 & 0 \\
0 & -1 & 0 & 1
\end{vmatrix}
&= 0 + (-1)^{4+2}(-1)
\begin{vmatrix}
1 & 4 & 4 \\
1 & 2 & 3 \\
2 & 1 & 0 
\end{vmatrix} 
+ 0 + (-1)^{4+4}(1)
\begin{vmatrix}
1 & 4 & 4 \\
1 & 0 & 2 \\
2 & 1 & 1 
\end{vmatrix} \\
&= 0 + (-1)(9) + 0 + (1)(14) = 5
\end{align*}
and hence $|A| = 2(5) = 10$.
\end{solution}

With Theorem \ref{thm:elementarytimesdet}, we can unearth the relation between invertibility of a square matrix and its determinant.
\begin{proper}
\label{proper:invnonzerodet}
An invertible matrix has a non-zero determinant. Otherwise, a singular matrix has a determinant of zero.
\end{proper}
\begin{proof}
Let's denote the matrix in question as $A$. For the case in which $A$ is invertible, by Properties \ref{proper:invseqelement} it can be written as the product of some elementary matrices $E_1$, $E_2$, $\ldots$, $E_{n-1}$, $E_n$, i.e.\
\begin{align*}
A = E_{1}E_{2} \cdots E_{n-1}E_n
\end{align*}
Taking the determinant of both sides, we have
\begin{align*}
\det(A) &= \det(E_{1}E_{2} \cdots E_{n-1}E_n)
\end{align*}
By repetitively using Theorem \ref{thm:elementarytimesdet}, we have
\begin{align*}
\det(A) &= \det(E_{1}(E_{2} \cdots E_{n-1}E_n)) \\
&= \det(E_1) \det(E_{2} \cdots E_{n-1}E_n) \\
&= \det(E_1) \det(E_{2}) \det(\cdots E_{n-1}E_n) \\
&= \det(E_1) \det(E_{2}) \cdots \det(E_{n-1})\det(E_n)
\end{align*}
Since by Properties \ref{proper:elementarymatdet}, all elementary matrices have a non-zero determinant (particularly we have required $c \neq 0$ when multiplying a row), i.e. $\det(E_i) \neq 0$ for all $i$, we have $\det(A) \neq 0$. We will not go through the details for singular matrices, which are put in the footnote below only for reference.\footnote{By Theorem \ref{thm:Gausselimprincip}, singular matrices have reduced row echelon forms that are not the identity. Observe that all other square rrefs that are not the identity must have at least one row of full zeros, and by Properties \ref{proper:zerodet} has a determinant of zero.} 
\end{proof}
Other properties of determinants include:
\begin{proper}
\label{proper:properdet}
For any $n \times n$ square matrices $A$ and $B$, we have
\begin{enumerate}
\item $\det(A^T) = \det(A)$,
\item $\det(kA) = k^n \det(A)$, for any constant $k$,
\item $\det(AB) = \det(A)\det(B)$, and
\item $\det(A^{-1}) = \dfrac{1}{\det(A)}$, if $A$ is invertible.
\end{enumerate}
By extension, $\det(A_1A_2\cdots A_n) = \det(A_1)\det(A_2)\cdots\det(A_n)$.
\end{proper}
For instance, if
\begin{align*}
&A = 
\begin{bmatrix}
2 & 3 \\
5 & 9 \\
\end{bmatrix}
&B = 
\begin{bmatrix}
4 & 5 \\
1 & 0 \\
\end{bmatrix}
\end{align*}
then
\begin{align*}
&|A| = (2)(9) - (3)(5) = 3 
&|B| = (4)(0) - (5)(1) = -5
\end{align*}
\begin{align*}
AB &= 
\begin{bmatrix}
2 & 3 \\
5 & 9 \\
\end{bmatrix}
\begin{bmatrix}
4 & 5 \\
1 & 0 \\
\end{bmatrix} \\
&= 
\begin{bmatrix}
(2)(4)+(3)(1) & (2)(5)+(3)(0) \\
(5)(4)+(9)(1) & (5)(5)+(9)(0)
\end{bmatrix} \\
&= 
\begin{bmatrix}
11 & 10 \\
29 & 25
\end{bmatrix} \\
|AB| &= (11)(25) - (10)(29) \\
&= -15 = (3)(-5) = |A||B|
\end{align*}
So we can see in this case, $\det(AB) = \det(A)\det(B)$ indeed. We put the formal proof for (3) of Properties \ref{proper:properdet} in the footnote for reference.\footnote{There are two cases to consider, $A$ being invertible or singular. If $A$ is singular, then by Properties \ref{proper:ABinv}, $AB$ is also singular. And by Properties \ref{proper:invnonzerodet}, both $\det(A)$ and $\det(AB)$ will be zero, and the equality holds trivially. Otherwise, if $A$ is invertible, then we can follow the idea in the proof of Properties \ref{proper:invnonzerodet}, and let $A = E_{1}E_{2} \cdots E_{n-1}E_n$ as a sequence of elementary matrices. By using Theorem \ref{thm:elementarytimesdet} back and forth, we have
\begin{align*}
\det(AB) &= \det(E_{1}E_{2} \cdots E_{n-1}E_nB) \\
&= \det(E_1) \det(E_{2}) \cdots \det(E_{n-1})\det(E_n)\det(B) & \text{(Theorem \ref{thm:elementarytimesdet})} \\
&= (\det(E_1) \det(E_{2}) \cdots \det(E_{n-1})\det(E_n))\det(B) \\
&= \det(E_{1}E_{2} \cdots E_{n-1}E_n)\det(B) & \text{(Theorem \ref{thm:elementarytimesdet})} \\
&= \det(A)\det(B)
\end{align*}
So the equality is true in both cases.}

Short Exericse: Prove (4) of Properties \ref{proper:properdet}.\footnote{Consider $A^{-1}A=I$, and take determinant on both sides. By (3), we have
\begin{align*}
\det(A^{-1}A) &= \det(I) \\
\det(A^{-1})\det(A) &= 1 & \text{(The identity always has a determinant of $1$)} \\
\det(A^{-1}) &= \frac{1}{\det(A)}
\end{align*}}

\subsection{Finding Inverses by Adjugate}
An alternative method to compute the inverse of a matrix is by using its \index{Adjugate}\keywordhl{adjugate}, which is the transpose of its cofactor matrix associated to it.
\begin{defn}[Adjugate]
For a matrix $A$, its adjugate is defined as
\begin{align*}
[\text{adj}(A)]_{pq} = (C_{pq})^T = C_{qp}
\end{align*}
where $C_{pq}$ is the cofactor of $A$ at $(p, q)$, formulated as in Definition \ref{defn:cofactor}.
\end{defn}
\begin{proper}
\label{proper:invadj}
The inverse of a matrix $A$ can be computed from its adjugate by
\begin{align*}
A^{-1} = \frac{1}{\det(A)}\text{adj}(A)
\end{align*}
\end{proper}
From this formula, it is obvious that singular matrices, having a determinant of zero, does not has an inverse.
\begin{exmp}
\label{exmp:2x2}
For a $2 \times 2$ matrix
\begin{align*}
\begin{bmatrix}
a & b \\
c & d
\end{bmatrix}    
\end{align*}
It is not difficult to see that the determinant is $ad - bc$, and the adjugate matrix is
\begin{align*}
\begin{bmatrix}
d & -c \\
-b & a 
\end{bmatrix}^T = 
\begin{bmatrix}
d & -b \\
-c & a 
\end{bmatrix}    
\end{align*}
So the inverse, if $ad - bc \neq 0$, is
\begin{align*}
\frac{1}{ad-bc}
\begin{bmatrix}
d & -b \\
-c & a 
\end{bmatrix}
\end{align*}
\end{exmp}
\begin{exmp}
Find the inverse of the following matrix by evaluating its adjugate.
\begin{align*}
A &= 
\begin{bmatrix}
1 & 2 & 3 \\
1 & 3 & 5 \\
1 & 4 & 11
\end{bmatrix}
\end{align*}
\end{exmp}
\begin{solution}
First of all, by Sarrus' Rule (Properties \ref{proper:sarrus})
\begin{align*}
|A| &= ((1)(3)(11) + (2)(5)(1) + (3)(1)(4))\\
&\quad- ((3)(3)(1) + (1)(5)(4) + (2)(1)(11)) \\
&= (33 + 10 + 12) - (9 + 20 + 22) \\
&= 4
\end{align*}
The adjugate matrix is
\begin{align*}
\text{adj}(A) &=
\begin{bmatrix}
\begin{vmatrix}
3 & 5 \\
4 & 11
\end{vmatrix}
&
-\begin{vmatrix}
1 & 5 \\
1 & 11
\end{vmatrix}
&
\begin{vmatrix}
1 & 3 \\
1 & 4
\end{vmatrix}\\
-\begin{vmatrix}
2 & 3 \\
4 & 11
\end{vmatrix}
&
\begin{vmatrix}
1 & 3 \\
1 & 11
\end{vmatrix}
&
-\begin{vmatrix}
1 & 2 \\
1 & 4
\end{vmatrix}\\
\begin{vmatrix}
2 & 3 \\
3 & 5
\end{vmatrix}
&
-\begin{vmatrix}
1 & 3 \\
1 & 5
\end{vmatrix}
&
\begin{vmatrix}
1 & 2 \\
1 & 3 
\end{vmatrix} 
\end{bmatrix}^{\color{red}{T}} \\
&= 
\begin{bmatrix}
13 & \textcolor{red}{-6} & 1 \\
\textcolor{red}{-10} & 8 & -2 \\
1 & -2 & 1
\end{bmatrix}^{\color{red}{T}} = 
\begin{bmatrix}
13 & \textcolor{red}{-10} & 1 \\
\textcolor{red}{-6} & 8 & -2 \\
1 & -2 & 1
\end{bmatrix}
\end{align*}
(be careful of not forgetting the transpose!) Putting the pieces together according to the formula in Properties \ref{proper:invadj}, we have
\begin{align*}
A^{-1} &= \frac{1}{\det(A)}\text{adj}(A) \\
&= \frac{1}{4}
\begin{bmatrix}
13 & -10 & 1 \\
-6 & 8 & -2 \\
1 & -2 & 1
\end{bmatrix} \\
&= 
\begin{bmatrix}
\frac{13}{4} & -\frac{5}{2} & \frac{1}{4} \\
-\frac{3}{2} & 2 & -\frac{1}{2} \\
\frac{1}{4} & -\frac{1}{2} & \frac{1}{4}
\end{bmatrix}
\end{align*}
\end{solution}
A summarizing point to be emphasized is that
\begin{thm}[Equivalence Statements]
\label{thm:equiv1}
For a square matrix $A$, the followings are equivalent:
\begin{enumerate}[label=(\alph*)]
\item $A$ is invertible, i.e.\ $A^{-1}$ exists,
\item $\det(A) \neq 0$,
\item The reduced row echelon form of $A$ is $I$.
\end{enumerate}
\end{thm}
which is just a rephrasing of Properties \ref{proper:invnonzerodet} and Theorem \ref{thm:Gausselimprincip}. 
Particularly, invertibility is equivalent to a non-zero determinant. We will see the expansion of these equivalence statements in later chapters.

\section{Python Programming}
\label{section:ch2python}
To create an identity matrix of size $n$, we use \verb|np.identity(n)|. For example,
\begin{lstlisting}
import numpy as np
I4 = np.identity(4)
print(I4)
\end{lstlisting}
returns
\begin{lstlisting}
[[1. 0. 0. 0.]
 [0. 1. 0. 0.]
 [0. 0. 1. 0.]
 [0. 0. 0. 1.]]
\end{lstlisting}
Applying transpose on a matrix is simple where we just add \verb|.T| after the array variable, like
\begin{lstlisting}
myMatrix1 = np.array([[1., 0., 3.],
                      [1., 4., 1.],
                      [-1., 2., 4.]])
print(myMatrix1)
print(myMatrix1.T)
\end{lstlisting}
yields
\begin{lstlisting}
[[ 1.  0.  3.]
 [ 1.  4.  1.]
 [-1.  2.  4.]]
[[ 1.  1. -1.]
 [ 0.  4.  2.]
 [ 3.  1.  4.]]
\end{lstlisting}
Finding the inverse of a matrix requires the \verb|scipy.linalg| library and call the \verb|inv| function.
\begin{lstlisting}
from scipy import linalg
myMatrix2 = linalg.inv(myMatrix1)
print(myMatrix2)
print(myMatrix1@myMatrix2) # Check: should give the identity
\end{lstlisting}
gives the expected results of
\begin{lstlisting}
[[ 0.4375   0.1875  -0.375  ]
 [-0.15625  0.21875  0.0625 ]
 [ 0.1875  -0.0625   0.125  ]]
[[1. 0. 0.]
 [0. 1. 0.]
 [0. 0. 1.]]
\end{lstlisting}
Meanwhile, we can use the \verb|det| function to calculate the determinant of a matrix as follows. First,
\begin{lstlisting}
print(linalg.det(myMatrix1))
\end{lstlisting}
gives the expected output of \verb|32.0|. As another example,
\begin{lstlisting}
myMatrix3 = np.array([[3., 1., 3., 2.],
                      [0., -1., -3., 1.],
                      [1., -1., -2., 0.],
                      [2., 0., 1., 0.]])
print(linalg.det(myMatrix3))  
\end{lstlisting}
produces an extremely small value of \verb|1.1102230246251562e-16|. In fact, the matrix
\begin{align*}
\begin{bmatrix}
3 & 1 & 3 & 2 \\
0 & -1 & -3 & 1 \\
1 & -1 & -2 & 0 \\
2 & 0 & 1 & 0    
\end{bmatrix}
\end{align*}
has a determinant of exactly zero. It is an artifact of numerical error when using floating point numbers. If we keep going ahead and computes its inverse by \verb|linalg.inv(myMatrix3)|, we will obtain an absurd output of
\begin{lstlisting}
[[ 1.200959e+15 -2.401919e+15  3.602879e+15 -3.602879e+15]
 [ 6.004799e+15 -1.200959e+16  1.801439e+16 -1.801439e+16]
 [-2.401919e+15  4.803839e+15 -7.205759e+15  7.205759e+15]
 [-1.200959e+15  2.401919e+15 -3.602879e+15  3.602879e+15]]
\end{lstlisting}
that have entries of extremely large magnitude. This phenomenon is due to the extremely small "determinant", through Properties \ref{proper:invadj}, magnifies the adjugate by being in the denominator. (The actual computation does not use Properties \ref{proper:invadj} directly but this is a heuristic perspective to view the problem.) To prevent this, we can add a \verb|if| condition to look for singularity, defining a function like
\begin{lstlisting}
def safe_inv(matrix):
    if np.abs(linalg.det(matrix)) < np.finfo(float).eps:
        print("Warning: The matrix is highly singular!")
        return(np.nan)
    else:
        return(linalg.inv(matrix))
\end{lstlisting}
where \verb|np.finfo(float).eps| gives the so-called \textit{machine epsilon} $\epsilon$ (the order of relative round-off error) of \verb|float| and we want the absolute value of the determinant be larger than that. Subsequently, calling \verb|safe_inv(myMatrix3)| will print a warning. Finally, we note that we can use \verb|sympy| to acquire the reduced row echelon form of a matrix. Let's use the matrix in Example \ref{exmp:rref2} for demonstration.
\begin{lstlisting}
import sympy

myMatrix4 = np.array([[2., 2., 1.],
                      [6., 4., 1.],
                      [2., 3., 2.],
                      [2., 1., 0.]])
myMatrix4_sympy = sympy.Matrix(myMatrix4) # Convert the numpy array to a sympy matrix
print(myMatrix4_sympy.rref())
\end{lstlisting}
then returns two objects
\begin{lstlisting}
(Matrix([
[1, 0, -0.5],
[0, 1,  1.0],
[0, 0,    0],
[0, 0,    0]]), (0, 1))    
\end{lstlisting}
The first one is the reduced row echelon form we want, and the second is a tuple which keeps the column indices of the pivots. \verb|sympy| also does \textit{zero testing} such that
\begin{lstlisting}
myMatrix3_sympy = sympy.Matrix(myMatrix3)
print(myMatrix3_sympy**(-1))    
\end{lstlisting}
raises properly the error of
\begin{lstlisting}
NonInvertibleMatrixError("Matrix det == 0; not invertible.") sympy.matrices.common.NonInvertibleMatrixError: Matrix det == 0; not invertible. 
\end{lstlisting}

\section{Exercises}

\begin{Exercise}
Find the determinant of the matrix below by inspection.
\begin{align*}
\begin{bmatrix}
1 & 2 & 3 & 4 & 5 \\
0 & 6 & 7 & 8 & 9 \\
0 & 0 & 10 & 11 & 12 \\
0 & 0 & 0 & 13 & 14 \\
0 & 0 & 0 & 0 & 15
\end{bmatrix}    
\end{align*}
By the same logic, derive a general formula for the determinant of any upper(lower)-triangular matrix.\footnote{An upper(lower)-triangular matrix is a matrix who elements below (above) the main diagonal are all zeros.}
\end{Exercise}
\begin{Answer}
(Applying cofactor expansion along the leftmost column recursively) The determinant is just the product of the diagonal elements $= (1)(6)(10)(13)(15) = 11700$.    
\end{Answer}

\begin{Exercise}
Let
\begin{align*}
&A =
\begin{bmatrix}
2 & 3\\
5 & 7
\end{bmatrix}
&B =
\begin{bmatrix}
4 & 6\\
0 & 1
\end{bmatrix}  
\end{align*}
Verify:
\begin{enumerate}[label=(\alph*)]
\item $(AB)^T = B^TA^T$,
\item $(AB)^{-1} = B^{-1}A^{-1}$, and
\item $\det(AB) = \det(A)\det(B)$.
\end{enumerate} for this particular case.
\end{Exercise}
\begin{Answer}
\begin{enumerate}[label=(\alph*)]
\item 
$\begin{bmatrix}
8 & 20\\
15 & 37
\end{bmatrix}
=
\begin{bmatrix}
4 & 0 \\
6 & 1
\end{bmatrix}
\begin{bmatrix}
2 & 5 \\
3 & 7
\end{bmatrix}
$
\item $\begin{bmatrix}
-\frac{37}{4} & \frac{15}{4}\\
5 & -2    
\end{bmatrix} = 
\begin{bmatrix}
\frac{1}{4} & -\frac{3}{2}\\
0 & 1 
\end{bmatrix}
\begin{bmatrix}
-7 & 3\\
5 & -2  
\end{bmatrix}$
\item $\begin{vmatrix}
8 & 15\\
20 & 37
\end{vmatrix}
= -4 = (-1)(4) =
\begin{vmatrix}
2 & 3\\
5 & 7
\end{vmatrix}
\begin{vmatrix}
4 & 6\\
0 & 1
\end{vmatrix}$
\end{enumerate}
\end{Answer}

\begin{Exercise}
If
\begin{align*}
A =
\begin{bmatrix}
3 & 2 & 9\\
1 & 2 & 3\\
4 & 0 & 4
\end{bmatrix}  
\end{align*}
Find its inverse by 
\begin{enumerate}[label=(\alph*)]
\item Gaussian Elimination, and
\item Determinant and adjugate.
\end{enumerate}
\end{Exercise}
\begin{Answer}
\begin{enumerate}[label=(\alph*)]
\item \begin{align*}
& \left[\begin{array}{@{}ccc|ccc@{}}
3 & 2 & 9 & 1 & 0 & 0 \\
1 & 2 & 3 & 0 & 1 & 0 \\
4 & 0 & 4 & 0 & 0 & 1
\end{array}\right] \\
\to & 
\left[\begin{array}{@{}ccc|ccc@{}}
1 & 2 & 3 & 0 & 1 & 0 \\
3 & 2 & 9 & 1 & 0 & 0 \\
4 & 0 & 4 & 0 & 0 & 1
\end{array}\right] & R_1 \leftrightarrow R_2 \\    
\to & 
\left[\begin{array}{@{}ccc|ccc@{}}
1 & 2 & 3 & 0 & 1 & 0 \\
0 & -4 & 0 & 1 & -3 & 0 \\
0 & -8 & -8 & 0 & -4 & 1
\end{array}\right] & 
R_2 - 3R_1 \to R_2, R_3 - 4R_1 \to R_3 \\
\to & 
\left[\begin{array}{@{}ccc|ccc@{}}
1 & 2 & 3 & 0 & 1 & 0 \\
0 & 1 & 0 & -\frac{1}{4} & \frac{3}{4} & 0 \\
0 & 1 & 1 & 0 & \frac{1}{2} & -\frac{1}{8}
\end{array}\right] & 
-\frac{1}{4}R_2 \to R_2, -\frac{1}{8}R_3 \to R_3 \\
\to & 
\left[\begin{array}{@{}ccc|ccc@{}}
1 & 2 & 3 & 0 & 1 & 0 \\
0 & 1 & 0 & -\frac{1}{4} & \frac{3}{4} & 0 \\
0 & 0 & 1 & \frac{1}{4} & -\frac{1}{4} & -\frac{1}{8}
\end{array}\right] & 
R_3 - R_2 \to R_3 \\
\to &
\left[\begin{array}{@{}ccc|ccc@{}}
1 & 0 & 0 & -\frac{1}{4} & \frac{1}{4} & \frac{3}{8} \\
0 & 1 & 0 & -\frac{1}{4} & \frac{3}{4} & 0 \\
0 & 0 & 1 & \frac{1}{4} & -\frac{1}{4} & -\frac{1}{8}
\end{array}\right] & 
R_1 - 3R_3 - 2R_2 \to R_1 
\end{align*}
\item $\det(A) = -32$ and
\begin{align*}
\text{adj}(A) &=
\begin{bmatrix}
\begin{vmatrix}
2 & 3 \\
0 & 4
\end{vmatrix} &
-\begin{vmatrix}
1 & 3 \\
4 & 4
\end{vmatrix} &
\begin{vmatrix}
1 & 2 \\
4 & 0
\end{vmatrix} 
\\
-\begin{vmatrix}
2 & 9 \\
0 & 4
\end{vmatrix} &
\begin{vmatrix}
3 & 9 \\
4 & 4
\end{vmatrix} &
-\begin{vmatrix}
3 & 2 \\
4 & 0
\end{vmatrix}
\\
\begin{vmatrix}
2 & 9 \\
2 & 3
\end{vmatrix} &
-\begin{vmatrix}
3 & 9 \\
1 & 3
\end{vmatrix} &
\begin{vmatrix}
3 & 2 \\
1 & 2
\end{vmatrix} 
\end{bmatrix}^T \\
&=
\begin{bmatrix}
8 & 8 & -8 \\
-8 & -24 & 8 \\
-12 & 0 & 4
\end{bmatrix}^T \\
&=
\begin{bmatrix}
8 & -8 & -12 \\
8 & -24 & 0 \\
-8 & 8 & 4
\end{bmatrix}   
\end{align*}
Hence 
\begin{align*}
A^{-1} &= \frac{1}{\det(A)} \text{adj}(A) \\
&= -\frac{1}{32}
\begin{bmatrix}
8 & -8 & -12 \\
8 & -24 & 0 \\
-8 & 8 & 4
\end{bmatrix} \\
&= 
\begin{bmatrix}
-\frac{1}{4} & \frac{1}{4} & \frac{3}{8} \\
-\frac{1}{4} & \frac{3}{4} & 0 \\
\frac{1}{4} & -\frac{1}{4} & -\frac{1}{8}
\end{bmatrix}
\end{align*}
\end{enumerate}
\end{Answer}

\begin{Exercise}
Let
\begin{align*}
&A =
\begin{bmatrix}
0 & 2 & 5\\
0 & 4 & 9\\
1 & 2 & 1
\end{bmatrix}
&B =
\begin{bmatrix}
2 & 3 & 4\\
2 & 4 & 6\\
3 & 5 & 8
\end{bmatrix}  
\end{align*}
Verify:
\begin{enumerate}[label=(\alph*)]
\item $(AB)^T = B^TA^T$,
\item $(AB)^{-1} = B^{-1}A^{-1}$, and
\item $\det(AB) = \det(A)\det(B)$.
\end{enumerate} for this particular case. 
\end{Exercise}
\begin{Answer}
\begin{enumerate}[label=(\alph*)]
\item $\begin{bmatrix}
19 & 35 & 9 \\
33 & 61 & 16 \\
52 & 96 & 24
\end{bmatrix}
=
\begin{bmatrix}
2 & 2 & 3\\
3 & 4 & 5\\
4 & 6 & 8
\end{bmatrix} 
\begin{bmatrix}
0 & 0 & 1\\
2 & 4 & 2\\
5 & 9 & 1
\end{bmatrix}$
\item $
\begin{bmatrix}
18 & -10 & 1 \\
-6 & 3 & 1 \\
-\frac{11}{4} & \frac{7}{4} & - 1
\end{bmatrix}
=
\begin{bmatrix}
1 & -2 & 1 \\
1 & 2 & -2 \\
-1 & -\frac{1}{2} & 1 \\
\end{bmatrix}
\begin{bmatrix}
7 & -4 & 1 \\
-\frac{9}{2} & \frac{5}{2} & 0 \\
2 & -1 & 0 \\
\end{bmatrix}$
\item
$ \begin{vmatrix}
19 & 33 & 52 \\
35 & 61 & 96 \\
9 & 16 & 24
\end{vmatrix} = -4 = (-2)(2) =
\begin{vmatrix}
0 & 2 & 5\\
0 & 4 & 9\\
1 & 2 & 1
\end{vmatrix}
\begin{vmatrix}
2 & 3 & 4\\
2 & 4 & 6\\
3 & 5 & 8
\end{vmatrix}$
\end{enumerate}
\end{Answer}

\begin{Exercise}
Show that
\begin{align*}
A = 
\begin{bmatrix}
1 & 2 & 3 \\
3 & 0 & -1 \\
2 & 1 & 1 
\end{bmatrix}
\end{align*}
is singular.
\end{Exercise}
\begin{Answer}
Either by evaluating the determinant to show that $|A| = 0$, or find its reduced row echelon form which is
\begin{align*}
\begin{bmatrix}
1 & 0 & -\frac{1}{3} \\
0 & 1 & \frac{5}{3} \\
0 & 0 & 0
\end{bmatrix}
\end{align*}
and not equal to the identity.
\end{Answer}


\begin{Exercise}
Given
\begin{align*}
A =
\begin{bmatrix}
1 & 9 & 1 & 4\\
0 & 6 & 2 & 8\\
1 & 9 & 3 & 9\\
0 & 9 & 0 & 1
\end{bmatrix}  
\end{align*}
Find its determinant, inverse, and determinant of the inverse. 
\end{Exercise}
\begin{Answer}
\begin{align*}
\det(A) &= -42\\
\det(A^{-1}) &= -\frac{1}{42}\\
A^{-1} &= 
\def\arraystretch{1.25}
\begin{bmatrix}
\frac{9}{7} & -\frac{3}{14} & -\frac{2}{7} & -\frac{6}{7} \\
-\frac{1}{21} & -\frac{1}{21} & \frac{1}{21} & \frac{1}{7} \\
-\frac{11}{7} & -\frac{15}{14} & \frac{11}{7} & \frac{5}{7} \\
\frac{3}{7} & \frac{3}{7} & -\frac{3}{7} & -\frac{2}{7}
\end{bmatrix}
\end{align*}
\end{Answer}

\begin{Exercise}
For the following matrix,
\begin{align*}
A = 
\begin{bmatrix}
p & 1 & 2\\
0 & 2 & p\\
4 & -2 & 0
\end{bmatrix} 
\end{align*}
Find the values of $p$ such that $A$ is invertible.
\end{Exercise}
\begin{Answer}
By cofactor expansion along the first column, we can obtain the determinant of $A$ as
\begin{align*}
|A| = 2p^2 + 4p - 16
\end{align*}
which has two roots, $p = -4$ and $p = 2$ such that $|A| = 0$ and $A$ is not invertible. All values of $p$ other than $p = -4$ and $p = 2$ make $A$ invertible.
\end{Answer}

\begin{Exercise}
\label{ex:symskew}
Show that for any square matrix $A$, $A + A^T$ is symmetric and $A - A^T$ is skew-symmetric. Hence show with an explicit formula that any square matrix $A$ can be written as the sum of a symmetric matrix and a skew-symmetric matrix.
\end{Exercise}
\begin{Answer}
$(A + A^T)^T = A^T + (A^T)^T = A^T+A = A+A^T$,\\
and $(A - A^T)^T = A^T - (A^T)^T = A^T-A = -(A - A^T)$. We can split $A$ into
\begin{align*}
A &= A + \frac{1}{2}(A^T - A^T) \\
&= \frac{1}{2}A + \frac{1}{2}A + \frac{1}{2}A^T - \frac{1}{2}A^T \\
&= \frac{1}{2}A + \frac{1}{2}A^T + \frac{1}{2}A - \frac{1}{2}A^T \\
&= \frac{1}{2}(A + A^T) + \frac{1}{2}(A - A^T)
\end{align*}
where the first term is symmetric and the second term is skew-symmetric.
\end{Answer}

\begin{Exercise}
Prove that if $A$ is an invertible $n \times n$ matrix, $|A| \neq 0$, then we have
\begin{align*}
\det(\text{adj}(A))=(\det(A))^{n-1}    
\end{align*}
using Properties \ref{proper:properdet} and \ref{proper:invadj}.
\end{Exercise}
\begin{Answer}
\begin{align*}
A^{-1} &= \frac{1}{\det(A)}\text{adj}(A) \\
\det(A^{-1}) &= \det(\frac{1}{\det(A)}\text{adj}(A)) & \text{(Notice that $\frac{1}{\det(A)}$ is now a scalar)}\\
\frac{1}{\det(A)} &= (\frac{1}{\det(A)})^n\det(\text{adj}(A)) \\
\det(\text{adj}(A)) &= (\det(A))^{n-1}  
\end{align*}
\end{Answer}