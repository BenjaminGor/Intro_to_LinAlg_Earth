\chapter{Discrete Fourier Transform}
We now discuss a powerful mathematical tool that can be approached from a least-square approximation view point. In last sections, we see how to fit a polynomial curve to some data. It is natural to ask, if there are any other types of curves that can be fitted to our interests. In the area of Earth Science, many phenomena can be described by the notion of waves, which are generally sinusoidal, e.g. atmospheric wave, seismic wave, electromagnetic wave. Hence, we may want to investigate if we can interpolate some data by using sines and cosines, and this is the central idea of what is known as Fourier Transform.

\section{Background and Types of Discrete Fourier Transform}
Fourier Transform and other related results are supported by the Fourier Theorem, which suggests that a function can be approximated by the sum of some sine-cosine series under some conditions. This implies that we can decompose a function into different sinusoidal waves, each having a certain amplitude, or called the Fourier coefficient. The resultant series has the name Fourier Series.
\begin{thm}
A periodic, smooth function $f(x)$ in the interval $[0,2\pi]$ can be approximated by the Fourier Series
\begin{align*}
S(f(x)) &= C_0 + A_1 \cos(x) + A_2 \cos(2x) + \cdots + B_1 \sin(x) + B_2 \sin(2x) + \cdots \\
&= C_0 + \sum_{k=1}^{\infty} (A_k \cos(kx) + B_k \sin(kx))
\end{align*}
\end{thm}
For a function that has an interval $[a,b]$ not exactly the same as $[0,2\pi]$, we can always scale it by replacing $x$ by the substitution $x' = 2\pi(\frac{x-a}{b-a})$. For non-periodic function over a fixed interval, we can extend it so that it behaves like a periodic one, by copying and pasting the same function over and over periodically.\\
\\
Here we provide a well-known example, the Fourier Series of $f(x) = x$ over $[0, 2\pi]$. We omit the calculation, and the answer is
\begin{align*}
S(x) &= \pi - 2 \sum_{k=1}^{\infty} \frac{\sin(kx)}{k} \\
&= \pi - 2 (\sin(x) + \frac{1}{2} \sin(2x) + \frac{1}{3} \sin(3x) + \cdots)
\end{align*}

\subsection{Discrete Fourier Transform, Real}
In Earth Science, and other fields like Engineering, often we are not given a function that has an analytical form to work with. Instead, we collect data from measurements at a fixed sampling rate, and what we obtain is a discrete time series. However, we can still apply the ideas of Fourier, and try to approximate and interpolate the finite time series with sinusoidal waves. Assume that, we have $n$ data points collected for the time series $f(t)$, evenly spaced at time $t = 0, 1, 2, \cdots, n-1$. We can scale the time range to $[0,2\pi]$ as what is suggested above, by making a change of variable $t' = 2\pi \frac{t}{n}$.\\
\\
Then, we can search a way to express the time series with sines and cosines of different frequencies like $\sin(t') = \sin((\frac{2\pi}{n})t), \sin(2t') = \sin((\frac{2\pi}{n})2t), \cdots$, and $\cos(t') = \cos((\frac{2\pi}{n})t), \cos(2t') = \cos((\frac{2\pi}{n})2t), \cdots$. The corresponding frequencies are $\frac{2\pi}{n}, 2(\frac{2\pi}{n}), \cdots$. In general, we propose an approximated form up to order $m$ as
\begin{align*}
S(f(t)) &= C_0 + A_1\cos((\frac{2\pi}{n})t) + A_2\cos((\frac{2\pi}{n})2t) + \cdots + A_m\cos((\frac{2\pi}{n})mt)\\
&\quad + B_1\sin((\frac{2\pi}{n})t) + B_2\sin((\frac{2\pi}{n})2t) + \cdots + B_m\sin((\frac{2\pi}{n})mt)\\
&= C_0 + \sum_{k=1}^{m} \left[A_k \cos((\frac{2\pi kt}{n})) + B_k \sin((\frac{2\pi kt}{n}))\right]
\end{align*}
Assume that we only use a few of sines and cosines for the approximation, such that the order $m$ much is less than the number of data points $n$, specifically, $2m+1 \leq n$. The least-square method is then set to find the best-fit parameters $C_0, A_1, B_1, \cdots, A_m, B_m$ for the system $G\vec{\beta} = \vec{d}$, where
\begin{align*}
G =
\begin{bmatrix}
1 & 1 & 0 & \cdots & 1 & 0 \\
1 & \cos(\frac{2\pi}{n}) & \sin(\frac{2\pi}{n}) & \cdots & \cos(\frac{2\pi m}{n}) & \sin(\frac{2\pi m}{n}) \\
1 & \cos(\frac{2\pi(2)}{n}) & \sin(\frac{2\pi(2)}{n}) & \cdots & \cos(\frac{2\pi (2m)}{n}) & \sin(\frac{2\pi (2m)}{n}) \\
\vdots & \vdots & \vdots & & \vdots & \vdots \\
1 & \cos(\frac{2\pi(n-1)}{n}) & \sin(\frac{2\pi(n-1)}{n}) & \cdots & \cos(\frac{2\pi (n-1)m}{n}) & \sin(\frac{2\pi (n-1)m}{n}) \\
\end{bmatrix}
\end{align*}
is a $n \times (2m+1)$ matrix and
\begin{align*}
&\vec{\beta} = 
\begin{bmatrix}
C_0 \\
A_1 \\
B_1 \\
\vdots \\
A_m \\
B_m
\end{bmatrix}
&\vec{d} = 
\begin{bmatrix}
f(0) \\
f(1) \\
f(2) \\
\vdots \\
f(n-1)
\end{bmatrix}
\end{align*}
are vectors with $2m+1$ and $n$ entries respectively. From theorem \ref{bestfit}, we know that the best parameters are found by
\begin{align*}
\vec{\beta_f} = (G^TG)^{-1}G^T\vec{d}
\end{align*}
However, we can greatly simplify the expression, by noticing that every column vector of sine/cosine is orthogonal to each other. If we write each sine-cosine term as a complex exponential using Euler's formula in definition \ref{Euler}, then each column vector of a sine-cosine pair in $G$ with a frequency of $2\pi k/n$, where both $k$ and $n$ are integers, can be expressed by the real and imaginary parts of
\begin{align*}
\left[\exp(\imath \frac{2\pi k}{n} t)\right]_{t = 0,1,2,\cdots,n-1}
&= \left[\cos(\frac{2\pi k}{n} t) + \imath\sin(\frac{2\pi k}{n} t) \right]_{t = 0,1,2,\cdots,n-1}
\end{align*}
or in the other way around,
\begin{align*}
\left[\exp(-\imath \frac{2\pi k}{n} t)\right]_{t = 0,1,2,\cdots,n-1}
&= \left[\cos(\frac{2\pi k}{n} t) - \imath\sin(\frac{2\pi k}{n} t) \right]_{t = 0,1,2,\cdots,n-1}
\end{align*}
We now first prove that the two column vectors of a sine-cosine pair that has the same frequency are orthogonal. We take the sum of squares of the first expression, which gives
\begin{align*}
\sum_{t=0}^{n-1} (\exp(\imath \frac{2\pi k}{n} t))^2 &= \sum_{t=0}^{n-1} (\cos(\frac{2\pi k}{n} t) + \imath\sin(\frac{2\pi k}{n} t))^2 \\
\sum_{t=0}^{n-1} \exp(\imath \frac{4\pi k}{n} t) &= \sum_{t=0}^{n-1} (\cos^2(\frac{2\pi k}{n} t) + 2\imath\sin(\frac{2\pi k}{n} t)\cos(\frac{2\pi k}{n} t) - \sin^2(\frac{2\pi k}{n} t))
\end{align*}
Notice that the left hand side is a geometric sequence with a common ratio $r = \exp(\imath 4\pi k/n)$, whose sum is seen to be
\begin{align*}
\frac{1-r^n}{1-r} &= \frac{1 - \exp(\imath 4\pi k)}{1 - \exp(\imath 4\pi k/n)} = 0
\end{align*}
as $\exp(\imath 4\pi k)$ is just $1$. By comparing the real and imaginary parts, we know that
\begin{align*}
\sum_{t=0}^{n-1} \cos^2(\frac{2\pi k}{n} t) &= \sum_{t=0}^{n-1} \sin^2(\frac{2\pi k}{n} t) \\
\sum_{t=0}^{n-1} \sin(\frac{2\pi k}{n} t)\cos(\frac{2\pi k}{n} t) &= 0
\end{align*}
The second equation shows that the two column vectors representing the sine and cosine wave of the same frequency have a dot product of zero and hence are orthogonal.\\
\\
Utilizing the complex formulations, we can also prove that column vectors of sine/cosine waves with different frequencies are orthogonal as well. Here we prove one of the cases, where the first wave is a sine a frequency of $2\pi k/n$, and the second wave is also a sine, of a frequency of $2\pi l/n$, where $k \neq l$ are both integers. We start by considering the sum of products between
\begin{align*}
\left[\exp(\imath \frac{2\pi k}{n} t)\right]_{t = 0,1,2,\cdots,n-1} = \left[\cos(\frac{2\pi k}{n} t) + \imath\sin(\frac{2\pi k}{n} t) \right]_{t = 0,1,2,\cdots,n-1}   
\end{align*}
and
\begin{align*}
\left[\exp(\imath \frac{2\pi l}{n} t)\right]_{t = 0,1,2,\cdots,n-1} = \left[\cos(\frac{2\pi l}{n} t) + \imath\sin(\frac{2\pi l}{n} t) \right]_{t = 0,1,2,\cdots,n-1}   
\end{align*}
The results are similar to the above analysis. Particularly,
\begin{align*}
\sum_{t=0}^{n-1} \cos(\frac{2\pi k}{n} t)\cos(\frac{2\pi l}{n} t) - \sum_{t=0}^{n-1} \sin(\frac{2\pi k}{n} t)\sin(\frac{2\pi l}{n} t) &= 0 
\end{align*}
We can also consider another dot product between $\left[\exp(\imath \frac{2\pi k}{n} t)\right]_{t = 0,1,2,\cdots,n-1}$ and
\begin{align*}
\left[\exp(-\imath \frac{2\pi l}{n} t)\right]_{t = 0,1,2,\cdots,n-1} = \left[\cos(\frac{2\pi l}{n} t) - \imath\sin(\frac{2\pi l}{n} t) \right]_{t = 0,1,2,\cdots,n-1}       
\end{align*}
This yields another relation as
\begin{align*}
\sum_{t=0}^{n-1} \cos(\frac{2\pi k}{n} t)\cos(\frac{2\pi l}{n} t) + \sum_{t=0}^{n-1} \sin(\frac{2\pi k}{n} t)\sin(\frac{2\pi l}{n} t) &= 0 
\end{align*}
From the two derived equations, we can conclude that
\begin{align*}
\sum_{t=0}^{n-1} \sin(\frac{2\pi k}{n} t)\sin(\frac{2\pi l}{n} t) &= 0    
\end{align*}
The orthogonality relations can be proven for sine vs cosine/cosine vs cosine of different frequencies as well. We will now establish the last result, the dot product of any cosine/sine column vector with a specific frequency $2\pi k/n$ with itself. We can consider the sum of products between
$\left[\exp(\imath \frac{2\pi k}{n} t)\right]_{t = 0,1,2,\cdots,n-1}$ and $\left[\exp(-\imath \frac{2\pi k}{n} t)\right]_{t = 0,1,2,\cdots,n-1}$. This time, the left hand side is not a geometric series, but rather $n$ terms of $1$. The relation is
\begin{align*}
\sum_{t=0}^{n-1} \cos^2(\frac{2\pi k}{n} t) + \sum_{t=0}^{n-1} \sin^2(\frac{2\pi k}{n} t) &= \sum_{t=0}^{n-1} \exp(\imath \frac{2\pi k}{n} t)\exp(-\imath \frac{2\pi k}{n} t) \\
&= \sum_{t=0}^{n-1} (1) \\
&= n
\end{align*}
Actually it roots from the fact that the sum of a sine-cosine square pair is $1$. Not long before, we have arrived at the equation
\begin{align*}
\sum_{t=0}^{n-1} \cos^2(\frac{2\pi k}{n} t) &= \sum_{t=0}^{n-1} \sin^2(\frac{2\pi k}{n} t)     
\end{align*}
Solving the two equations yields
\begin{align*}
\sum_{t=0}^{n-1} \cos^2(\frac{2\pi k}{n} t) &= \sum_{t=0}^{n-1} \sin^2(\frac{2\pi k}{n} t) = \frac{n}{2}    
\end{align*}
Hence the product $G^TG$, where each entry will be the dot products between vectors of sines and cosines, will be
\begin{align*}
G^TG =
\begin{bmatrix}
n & 0 & 0 & \cdots \\
0 & \frac{n}{2} & 0 & \\
0 & 0 & \frac{n}{2} & \\
\vdots & & & \ddots
\end{bmatrix}
\end{align*}
and
\begin{align*}
(G^TG)^{-1} = \frac{1}{n}
\begin{bmatrix}
1 & 0 & 0 & \cdots \\
0 & 2 & 0 & \\
0 & 0 & 2 & \\
\vdots & & & \ddots
\end{bmatrix}    
\end{align*}
So the best-fit parameters are
\begin{align*}
\vec{\beta_f} &= (G^TG)^{-1}G^T\vec{d} \\
\begin{bmatrix}
C_0 \\
A_1 \\
B_1 \\
\vdots \\
\end{bmatrix}
&= \frac{1}{n} 
\begin{bmatrix}
1 & 0 & 0 & \cdots \\
0 & 2 & 0 & \\
0 & 0 & 2 & \\
\vdots & & & \ddots
\end{bmatrix} 
\begin{bmatrix}
1 & 1 & 1 & \cdots \\
1 & \cos(\frac{2\pi}{n}) & \cos(\frac{2\pi(2)}{n}) & \\
0 & \sin(\frac{2\pi}{n}) & \sin(\frac{2\pi(2)}{n}) &  \\
\vdots & & & \ddots
\end{bmatrix}
\begin{bmatrix}
f(0)\\
f(1)\\
f(2)\\
\vdots
\end{bmatrix} \\
\begin{bmatrix}
C_0 \\
A_1 \\
B_1 \\
\vdots \\
\end{bmatrix}
&= 
\frac{1}{n}
\begin{bmatrix}
1 & 1 & 1 & \cdots \\
2 & 2\cos(\frac{2\pi}{n}) & 2\cos(\frac{2\pi(2)}{n}) & \\
0 & 2\sin(\frac{2\pi}{n}) & 2\sin(\frac{2\pi(2)}{n}) &  \\
\vdots & & & \ddots
\end{bmatrix}
\begin{bmatrix}
f(0)\\
f(1)\\
f(2)\\
\vdots
\end{bmatrix} 
\end{align*}
Detailed expressions are
\begin{align*}
C_0 &= \frac{1}{n}(f(0) + f(1) + f(2) + \cdots + f(n-1)) \\
&= \frac{1}{n}\sum_{t=0}^{n-1} f(t) \\
A_m &= \frac{2}{n}(f(0) + f(1)\cos(\frac{2\pi m}{n}) + \cdots + f(n-1)\cos(\frac{2\pi (n-1)m}{n})) \\
&= \frac{2}{n}\sum_{t=0}^{n-1} f(t)\cos(\frac{2\pi mt}{n}) \\
B_m &= \frac{2}{n}(f(1)\sin(\frac{2\pi m}{n}) + \cdots + f(n-1)\sin(\frac{2\pi (n-1)m}{n})) \\
&= \frac{2}{n}\sum_{t=0}^{n-1} f(t)\sin(\frac{2\pi mt}{n})
\end{align*}

\begin{exmp}
\label{ex11.2.1}
Fit the following time-series with sinusoidal waves up to order $m=1$. $f(0) = 1, f(1) = 4, f(2) = 6, f(3) = 3, f(4) = 5$.\\
\\
The order $m=1$ means that there are only three constituents, which are the constant term, and a sine-cosine pair with a frequency of $\frac{2\pi k}{n}$ where $k=1$, $n=5$. The best-fit parameters will be
\begin{align*}
\begin{bmatrix}
C_0 \\
A_1 \\
B_1
\end{bmatrix}
&= 
\frac{1}{5}
\begin{bmatrix}
1 & 1 & 1 & 1 & 1 \\
2 & 2\cos(\frac{2\pi}{5}) & 2\cos(\frac{2\pi(2)}{5}) & 2\cos(\frac{2\pi(3)}{5}) & 2\cos(\frac{2\pi(4)}{5}) \\
0 & 2\sin(\frac{2\pi}{5}) & 2\sin(\frac{2\pi(2)}{5}) & 2\sin(\frac{2\pi(3)}{5}) & 2\sin(\frac{2\pi(4)}{5})
\end{bmatrix}
\begin{bmatrix}
f(0)\\
f(1)\\
f(2)\\
f(3)\\
f(4)
\end{bmatrix} 
\end{align*}
\begin{align*}
C_0 &= \frac{1}{5} (1+4+6+3+5) = \frac{19}{5} \\
A_1 &= \frac{2}{5} (1 + 4\cos(\frac{2\pi}{5}) + 6\cos(\frac{2\pi(2)}{5}) + 3\cos(\frac{2\pi(3)}{5}) + 5\cos(\frac{2\pi(4)}{5})) \\
&= -\frac{7}{5} \\
B_1 &= \frac{2}{5} (4\sin(\frac{2\pi}{5}) + 6\sin(\frac{2\pi(2)}{5}) + 3\sin(\frac{2\pi(3)}{5}) + 5\sin(\frac{2\pi(4)}{5})) \\
&\approx 0.325
\end{align*}
So the best sinusoidal fit of order $m=1$ for the time-series concerned is
\begin{align*}
f(t) = \frac{19}{5} - \frac{7}{5} \cos(\frac{2\pi}{5}t) + 0.325 \sin(\frac{2\pi}{5}t)
\end{align*}
Short Exercise: Find an improved approximation with order $m = 2$.
\end{exmp}

However, there are two problems with the formulation. First, since we may want to make the approximation as good as possible, we have to know how high the order $m$ needs to be. Let's continue with the assumption that the time step is $1$, so that the sampling points are at $t = 0, 1, 2, \cdots, n-1$. Then the Nyquist Sampling Theorem tells us that there is a cap for $m$.
\begin{thm}
For a evenly spaced time-series that has a time step of $1$, any wave with a linear frequency $f > f_{ny} = 1/2$ or angular freqency $\omega > \omega_{ny} = \pi$ cannot be detected. More generally, for a time step of $\Delta t$, the Nyquist frequnecy is $\omega_{ny} = \frac{\pi}{\Delta t}$.
\end{thm}
So it means that we only need the sines/cosines up to the order $m_c$, where $m_c$ is inferred by looking at the expression of angular frequency inside the expression of sines/cosines, that is
\begin{align*}
\frac{2\pi m}{n} \leq \frac{2\pi m_c}{n} &= \pi \\
m \leq m_c &= \frac{n}{2}
\end{align*}
\paragraph{Proof} We illustrate a portion of the theorem by proving
\begin{align*}
\cos(\frac{2\pi (m_c + (m_c - m))}{n}) &= \cos(\frac{2\pi (m_c - (m_c - m))}{n}) \\
&= \cos(\frac{2\pi m}{n})
\end{align*}
so that mirroring $m$ about $m_c$ has no effect on the cosine wave. But the left hand side is simply
\begin{align*}
\cos(\frac{2\pi (m_c + (m_c - m))}{n}) &= \cos(\frac{4\pi mc}{n} - \frac{2\pi m}{n}) \\
&= \cos(\frac{4\pi (n/2)}{n} - \frac{2\pi m}{n}) \\
&= \cos(2\pi - \frac{2\pi m}{n}) \\
&= \cos(\frac{2\pi m}{n})
\end{align*}

For $n$ is even, then we have frequencies from zero to $m_c = n/2$. Notice that the zero frequency constant term contributes a single parameter, every other frequencies contribute two parameters as sines/cosines, and the maximum frequency $m_c = n/2$ only contributes one parameter of cosine (the corresponding diagonal entry for $(G^TG)^{-1}$ will be $1/n$ instead of $2/n$), as the sine function will be always zero at that frequency. This means that there are $n$ parameters as total.\\
\\
For $n$ is odd, the situation is the same except at the maximum frequency $m_c = (n-1)/2$ which contributes two parameters as a sine-cosine pair. Counting also reveals that the amount of parameters is also $n$. In both cases, there can be at most $n$ sinusoidal curves for fitting $n$ data points. Since the amount of parameters is the same as the number of data, it becomes an interpolation that passes through all the given data points, and this forms the foundation of Discrete Fourier Transform. But the difference between odd/even $n$ has to be remembered when we are tackling the maximum frequency signal. \\
\\
Another drawback of the approach is that it uses sine-cosine pairs for computation. In \autoref{section:complexno}, we have learnt the power of complex exponentials to simultaneously represent sines and cosines. Therefore, one may explore the possibility of using complex exponentials to do Discrete Fourier Transform. This has the benefits of simplicity and also convenience when programming.

\subsection{Discrete Fourier Transform, Complex}

Now extending the ideas from the last section, we propose an interpolation scheme which uses $\exp(\imath\frac{2\pi k}{n}t)$, for a time-series with $n$ data, evenly spaced by a time step of $1$. The range of $k$ will be from $-\frac{n}{2}, -(\frac{n}{2}-1), \cdots, 0, \frac{n}{2}-1$ for even $n$ and $-\frac{n-1}{2}, -(\frac{n-1}{2}-1), \cdots, 0, \frac{n-1}{2}$ for odd $n$. Both ranges ensure that the total number of sinusoidal waves used is exactly $n$.\\
\\
Each pair of $k$ and $-k$, or $\exp(\imath\frac{2\pi k}{n}t)$, and $\exp(-\imath\frac{2\pi k}{n}t)$ in combination, gives rise to $\cos(\frac{2\pi k}{n}t)$ and $\sin(\frac{2\pi k}{n}t)$ by properties \ref{sincoscomplex}, and so we see the correspondence between the real and complex version of Discrete Fourier Transform.\\
\\
Now, the matrix $G$ has the form of
\begin{align*}
G =
\begin{bmatrix}
1 & 1 & \cdots & 1 & 1 & \cdots\\
1 & \exp(\imath\frac{2\pi}{n}) & \cdots & \exp(\imath\frac{2\pi(n/2-1)}{n}) & \exp(\imath\frac{2\pi(-n/2)}{n}) & \cdots \\
1 & \exp(\imath\frac{2\pi(2)}{n}) & \cdots & \exp(\imath\frac{2\pi(n/2-1)(2)}{n}) & \exp(\imath\frac{2\pi(-n/2)(2)}{n}) & \cdots\\
\vdots & \vdots & & \vdots & \vdots & \\
1 & \exp(\imath\frac{2\pi(n-1)}{n}) & \cdots & \exp(\imath\frac{2\pi(n/2-1)(n-1)}{n}) & \exp(\imath\frac{2\pi(-n/2)(n-1)}{n}) & \cdots
\end{bmatrix}
\end{align*}
for even $n$, where each column representing frequencies that have $k = 0, 1, \cdots, \frac{n}{2}-1$, $-\frac{n}{2}, \cdots, -1$. This is a common convention where we starts from $k = 0$, and increases to the largest positive $k$, then flips the sign and resumes from the largest negative $k$, goes all the way back to $k = -1$. Odd $n$ has a similar form of $G$, but with all the $k$ replaced by the appropriate integer range.\\
\\
Notice that as we are now working with complex matrices, the normal equation is modified and the formula of the best-fit parameters are now $\vec{\beta}_f = (G^*G)^{-1}G^*\vec{d}$, contrary to theorem \ref{bestfit} as the transpose is replaced by conjugate transpose/Hermitian. The entries of $G^*G$ are the complex dot products between the column vectors of $G$.\\
\\
The orthogonality relation between different column vectors of complex exponentials is very easy to find. The procedure is similar to what we have done when proving the orthogonality for the real case, but even less tedious. The key point is to write the complex dot product along the two columns as a geometric sequence, that when the values of $k$ are different, will be evaluated to zero. The readers are invited to verify this result, as well as the fact that the complex dot product between any column vector of complex exponentials and itself is $n$. Thus, the expression $G^*G$ is just $nI$, and $(G^*G)^{-1} = \frac{1}{n} I$.\\
\\
Now we denote the best-fit parameters, or coefficients by $C_k$. Subsequently,
\begin{align*}
\vec{\beta}_f &= (G^*G)^{-1}G^*\vec{d} \\
&= \frac{1}{n}G^*\vec{d} \\
\begin{bmatrix}
C_0 \\
C_1 \\
\vdots \\
C_{n/2-1} \\
C_{-n/2} \\
\vdots
\end{bmatrix}
&= 
\frac{1}{n}
\begin{bmatrix}
1 & 1 & 1 & \cdots \\
1 & \exp(-\imath\frac{2\pi}{n}) & \exp(-\imath\frac{2\pi(2)}{n}) & \cdots \\
\vdots & \vdots & \vdots & \\
1 & \exp(-\imath\frac{2\pi(n/2-1)}{n}) & \exp(-\imath\frac{2\pi(n/2-1)(2)}{n}) & \cdots \\
1 & \exp(-\imath\frac{2\pi(-n/2)}{n}) & \exp(-\imath\frac{2\pi(-n/2)(2)}{n}) & \cdots 
\end{bmatrix}
\begin{bmatrix}
f(0) \\
f(1) \\
f(2) \\
\vdots
\end{bmatrix}
\end{align*}
Again this is for even $n$, we ought to replace the coefficients and complex exponentials appropriately for odd $n$. We now conclude the method of Complex Discrete Fourier Transform in a compact way as follows.
\begin{defn}
The coefficients, or amplitudes, of the Discrete Fourier Transform in complex form are
\begin{align*}
C_k = \sum_{t=0}^{n-1} f(t)\exp(-\imath\frac{2\pi k}{n}t)
\end{align*}
where $n$ is the number of data. $k$ are all integers ranging from $[-\frac{n}{2}, \frac{n}{2}-1]$ for even $n$, and $[-\frac{n-1}{2}, \frac{n-1}{2}]$ for odd $n$. The negative sign in the complex exponentials comes from the conjugate transpose in the formulation. In the most common convention the factor $1/n$ is removed. For a time step $\Delta t$ different from $1$, the expression is
\begin{align*}
C_k = \sum_{t=0}^{n-1} f(t)\exp(-\imath\frac{2\pi k}{n\Delta t}t)
\end{align*}
We will denote the series of $C_k$ by $F(k)$ or $\hat{f}(k)$.
\end{defn}
The relation between $C_k$ and the parameters $A_k$, $B_k$ in the real counterpart are inferred from properties \ref{sincoscomplex} and comparing the expression for the real case, that is
\begin{align*}
A_k &= \frac{C_k + C_{-k}}{n} \\
B_k &= -\frac{(C_k - C_{-k})}{n\imath}
\end{align*}
for $k \neq 0$. A closer inspection reveals that, since sine is an odd function, and cosine is an even function, $\Re(C_k) = \Re(C_{-k})$, and $\Im(C_k) = -\Im(C_{-k})$, or in other words, $C_k$ and $C_{-k}$ are a pair of complex conjugate. And so the relation between the real and complex Discrete Fourier Transform is
\begin{proper}
\label{FTrealcomplex}
The amplitudes $A_k$, $B_k$ in real Discrete Fourier Transform, and $C_k$ in complex Discrete Fourier Transform satisfy
\begin{align*}
A_k &= 2\Re(C_k)/n \\
B_k &= -2\Im(C_k)/n
\end{align*}
\end{proper}

\begin{exmp}
Find the Complex Discrete Fourier Transform for the time-series in example \ref{ex11.2.1}.\\
\\
Using the formula above, we have
\begin{align*}
C_0 &= 1+4+6+3+5 = 19 \\
C_1 &= 1 + 4\exp(-\imath\frac{2\pi}{5}) + 6\exp(-\imath\frac{2\pi(2)}{5}) + 3\exp(-\imath\frac{2\pi(3)}{5}) + 5\exp(-\imath\frac{2\pi(4)}{5}) \\
&= -3.5 - 0.812\imath \\
C_2 &= 1 + 4\exp(-\imath\frac{2\pi(2)}{5}) + 6\exp(-\imath\frac{2\pi(2)(2)}{5}) \\
&\quad+ 3\exp(-\imath\frac{2\pi(2)(3)}{5}) + 5\exp(-\imath\frac{2\pi(2)(4)}{5}) = -3.5 + 3.441 \imath
\end{align*}
$C_{-1}$ and $C_{-2}$ are complex conjugates of $C_1$ and $C_2$ respectively. \\
Short Exercise: Check if properties \ref{FTrealcomplex} is true in this example.
\end{exmp}

We can also undo the Discrete Fourier Transform and recover the initial time-series by its inverse, defined as follows.
\begin{defn}
The inverse Discrete Fourier Transform is done by
\begin{align*}
f(t) &= \frac{1}{n}\sum_{k=0}^{n-1} F(k)\exp(\imath\frac{2\pi k}{n}t)
\end{align*}
Sometimes we use the symbol $F^{-1}$ to denote the inverse operation.
\end{defn}

\section{Convolution}
Convolution is often used in the world of Fourier Transform due to a beautiful theorem that will be introduced later. In the area of Earth Science, as well as Physics and Statistics, convolution also has a place, for describing phenomena, or their solutions. Now we will introduce how convolution in a discrete sense is defined first, and a daily example will be provided as an illustration.\\
\\
Discrete Convolutions always involved two time-series as below.
\begin{defn}
The convolution $h(t)$ between two discrete time-series $f(t)$ and $g(t)$ is written $f(t) * g(t)$, defined as
\begin{align*}
h(t) = f(t) * g(t) = \sum_{\tau=-\infty}^{\infty} f(\tau) g(t-\tau)
\end{align*}
if $\tau$ takes a value so that the index in either $f$ or $g$ is out of the range, then the term becomes zero.
\end{defn}
For instance, if $f(t)$ is defined from $t = 0$ and $t = 4$ and $g(t)$ is defined from $t = 0$ and $t = 6$, then
\begin{align*}
h(0) &= f(0)g(0) \\
h(1) &= f(0)g(1-0) + f(1)g(1-1) = f(0)g(1) + f(1)g(0) \\
h(4) &= f(0)g(4-0) + f(1)g(4-1) + f(2)g(4-2) + f(3)g(4-3) + f(4)g(4-4) \\
&= f(0)g(4) + f(1)g(3) + f(2)g(2) + f(3)g(1) + f(4)g(0)
\end{align*}
and
\begin{align*}
h(6) &= f(0)g(6) + f(1)g(5) + f(2)g(4) + f(3)g(3) + f(4)g(2) \\
h(9) &= f(3)g(6) + f(4)g(5) \\
h(10) &= f(4)g(6)
\end{align*}
It can be deduced that the resulted convolution will has a length of $m + n - 1$ if the two input time-series have a length of $m$ and $n$.

\begin{exmp}
The probability of Mary getting married within some years can be described by the time-series $q(t)$, where
\begin{align*}
q(t) = 
\begin{cases}
0.2 & 0 \leq t \leq 2 \\
0.1 & 3 \leq t \leq 6 \\
0 & t \geq 7
\end{cases}
\end{align*}
The probability $r(t)$ that Mary gives birth to a baby some years $t$ after marriage is
\begin{align*}
r(t) = 
\begin{cases}
0 & t = 0 \\
0.15 & 1 \leq t \leq 4 \\
0.08 & 5 \leq t \leq 9 \\
0 & t \geq 10
\end{cases}    
\end{align*}
Find the net probability of Mary getting a baby at some years $t$ from now on, assuming she will only get pregnant after married. \\
\\
We identify the required probability $p(t) = q(t) * r(t)$ is a convolution between $q(t)$ and $r(t)$, which effectively have a length of $7$ and $9$. Particularly, taking $t = 5$ as an example, then we have
\begin{align*}
p(t = 5) &= P(\text{Birth 5 years later}|\text{Married now})P(\text{Married now}) \\
&\quad+ P(\text{Birth 4 years later}|\text{Married 1 year later})P(\text{Married 1 year later})\\
&\quad+ \cdots \\
&\quad+ P(\text{Birth 1 year later}|\text{Married 4 years later})P(\text{Married 4 years later}) \\
&= q(0)r(5) + q(1)r(4) + q(2)r(3) + q(3)r(2) + q(4)r(1) \\
&= (0.2)(0.08) + (0.2)(0.15) + (0.2)(0.15) + (0.1)(0.15) + (0.1)(0.15) \\
&= 0.106
\end{align*}
where $P(A|B)$ is the probability of getting $A$ when $B$ happens.\\
Short Exercise: Find the chance of Mary getting a baby at $8$ years later by convolution.
\end{exmp}

If, the two discrete time-series are infinitely long, then we have the following weaker version of Convolution Theorem for discrete Fourier Transform.
\begin{thm}
For two infinitely long discrete time-series $f(t)$ and $g(t)$, we have
\begin{align*}
F[f(t) * g(t)](k) &= \hat{f}(k)\hat{g}(k)
\end{align*}
The Discrete Fourier Transform of the convolution between the two time-series is equal to the product of the two Discrete Fourier Transforms for their own.
\end{thm}
It is worth emphasized again that this is a much weaker statement compared to the full form, which is known as the Circular Convolution Theorem, that do the convolution for two finite time-series in a cyclic manner, and works in two directions.

\section{Exercise}
\begin{Exercise}
Carry out Discrete Fourier Transform for the following data.
\begin{center}
\begin{tabular}{|c|c|c|c|c|c|c|}
\hline
unit time & 0 & 1 & 2 & 3 & 4 \\
\hline
f(t) & 4 & 6 & 7 & 1 & 3  \\
\hline
unit time & 5 & 6 & 7 & 8 & 9 \\
\hline
f(t) & 2 & 3 & 5 & 2 & 6\\
\hline
\end{tabular}
\end{center}
Find the amplitude and phase of the sinusoidal wave corresponding to an angular frequency of $2\pi\frac{3}{10}$. 
\end{Exercise}

\begin{Exercise}
Download an \href{https://cds.climate.copernicus.eu/cdsapp#!/dataset/reanalysis-era5-single-levels?tab=overview}{ERA5 Temperature Dataset} over any time period of $5$ years. Select a location as you like, and extract the temperature time-series there. Apply Discrete Fourier Transform on the time series, and identify any dominant frequency or time period whose amplitude (the magnitude of complex coefficient) is large. Explain the peaks with Earth Science knowledge.
\end{Exercise}

\begin{Exercise}
Perform Fourier Transform on the time-series of the first Principal Direction in exercise (10.5).
\end{Exercise}